\documentclass[12pt]{article}
\newcommand{\alert}[1]{{\bf \color{red} [Alert:] #1}}
\newcommand{\todo}[1]{{\bf \color{orange} [TODO:] #1}}
\newcommand{\real}[1][]{\mathbb{R}^{#1}}
\newcommand{\myeqn}[1]{(\ref{#1})}
\newcommand{\myex}[1]{Example \ref{#1}}
\newcommand{\defeq}{\stackrel{\mathrm{def}}{=}}
\newcommand{\parder}[2]{\frac{\partial #1}{\partial #2}}
\newcommand{\Lie}[3][]{\mathsf{L}_{#3}^{#1} #2}
\newcommand{\LieA}[1]{\mathsf{Lie}(#1)}
\newcommand{\lieder}[2]{\mathcal{L}_{#2} #1}
\renewcommand{\t}{^{\mbox{\tiny\sf T}}}
\newcommand{\trans}{^{\mbox{\tiny\sf T}}}
\newcommand{\markup}[1]{\{\textbf{#1}\}}
\newcommand{\msub}[1]{_\mathrm{#1}}
\newcommand{\msup}[1]{^\mathrm{#1}}
\newcommand{\inv}[1]{#1^{-1}}
\newcommand{\pinv}[1]{{#1}^{+}}
\newcommand{\myfracA}[2]{\displaystyle{\frac{#1}{#2}}}
\newcommand{\myfracB}[2]{{#1}/{#2}}
\newcommand{\mydiffA}[1]{\dot{#1}}
\newcommand{\mydiffB}[2]{\myfracA{\mathrm{d}{#1}}{\mathrm{d}{#2}}}
\newcommand{\ball}[2]{\mathcal{B}_{#1}\left(#2\right)}
\newcommand{\acos}[1]{\cos^{-1}\left(#1\right)}
\newcommand{\asin}[1]{\sin^{-1}\left(#1\right)}
\newcommand{\mani}{\mathcal{M}}
\newcommand{\tang}[2]{\mathsf{T}_{#1} #2}
\newcommand{\LieB}[2]{[ #1, #2 ]}
\newcommand{\LieBad}[3][]{\mathsf{ad}_{#2}^{#1} #3}
\newcommand{\ReachVT}{\mathcal{R}^V_T}
\newcommand{\ReachVt}{\mathcal{R}^V_t}
\newcommand{\ReachVTe}{\mathcal{R}^V_{\le T}}
\newcommand{\ReachT}{\mathcal{R}_T}
\newcommand{\Reacht}{\mathcal{R}_t}
\newcommand{\ReachTe}{\mathcal{R}_{\le T}}
\newcommand{\accLA}[1]{\mathsf{Lie}(#1)}
\newcommand{\accD}{\Delta_{\mathcal{F}}}
\newcommand{\accSA}{\mathsf{Lie}(\mathcal{G},f)}
\newcommand{\accDS}{\Delta_{\mathcal{G}}}
\newcommand{\eval}[3]{\mathsf{Ev}^{#2}_{#1}\left( #3 \right)}
\newcommand{\stlc}{\textsc{stlc}}
\newcommand{\clf}{\textsc{clf}}
\newcommand{\jqlf}{\textsc{jqlf}}
\newcommand{\dlas}{\textsc{dlas}}
\newcommand{\Ad}[2]{\mathsf{Ad}_{#1} #2}
\newcommand{\xe}{\ensuremath{x_e}}
\newcommand{\lebg}[1]{\mathcal{L}_{#1}}
\newcommand{\lebgx}[1]{\mathcal{L}_{#1 \mathrm{e}}}
\newcommand{\dom}{D}
\newcommand{\domT}{[t_0,\infty) \times D}
\newcommand{\rarrow}{\rightarrow}
\renewcommand{\d}{\mathrm{d}}
\renewcommand{\Re}{\mathbb{R}}
\newcommand{\C}{\mathrm{C}}

\newcommand{\QED}{{\unskip\nobreak\hfil\penalty50\hskip2em\vadjust{}
		\nobreak\hfil$\Box$\parfillskip=0pt\finalhyphendemerits=0\par}\vspace{0.1cm}}
\newcommand{\eoEx}{{\unskip\nobreak\hfil\penalty50\hskip0em\vadjust{}
		\nobreak\hfil$\Large\Diamond$\parfillskip=0pt\finalhyphendemerits=0\par}\vspace{0.1cm}}

\newcommand{\sgn}{\ensuremath{\operatorname{sgn}}}
\newcommand{\sat}{\ensuremath{\operatorname{sat}}}

\newcommand{\half}{\frac{1}{2}}
\newcommand{\shalf}{\mbox{$\frac{1}{2}$}}
\newcommand{\marcom}[1]{\marginpar{\footnotesize #1}}
\newcommand{\der}{\mathrm{D}}
\newcommand{\e}{\mathrm{e}}
\newcommand{\dt}{\mathrm{d}t}

\newcommand{\cA}{\ensuremath{\mathcal{A}}}
\newcommand{\cB}{\ensuremath{\mathcal{B}}}
\newcommand{\cG}{\ensuremath{\mathcal{G}}}
\newcommand{\cK}{\ensuremath{\mathcal{K}}}
\newcommand{\cW}{\ensuremath{\mathcal{W}}}
\newcommand{\cZ}{\ensuremath{\mathcal{Z}}}
\newcommand{\cS}{\ensuremath{\mathcal{S}}}
\newcommand{\cD}{\ensuremath{\mathcal{D}}}
\newcommand{\cP}{\ensuremath{\mathcal{P}}}
\newcommand{\cV}{\ensuremath{\mathcal{V}}}
\newcommand{\cL}{\ensuremath{\mathcal{L}}}
\newcommand{\cN}{\ensuremath{\mathcal{N}}}
\newcommand{\cI}{\ensuremath{\mathcal{I}}}
\newcommand{\cR}{\ensuremath{\mathcal{R}}}
\newcommand{\cM}{\ensuremath{\mathcal{M}}}
\newcommand{\cC}{\ensuremath{\mathcal{C}}}
\newcommand{\cF}{\ensuremath{\mathcal{F}}}
\newcommand{\cH}{\ensuremath{\mathcal{H}}}
\newcommand{\cO}{\ensuremath{\mathcal{O}}}
\newcommand{\cX}{\ensuremath{\mathcal{X}}}
\newcommand{\cY}{\ensuremath{\mathcal{Y}}}
\newcommand{\Ci}{\ensuremath{\mathcal{C}^\infty}}
\newcommand{\ISS}{\textsc{iss}}
\newcommand{\LISS}{\textsc{liss}}
\newcommand{\GAS}{\textsc{gas}}
\newcommand{\GS}{\textsc{gs}}
\newcommand{\LES}{\textsc{les}}
\newcommand{\GUAS}{\textsc{guas}}
\newcommand{\BIBO}{\textsc{bibo}}
\newcommand{\spec}{\ensuremath{\operatorname{spec}}}
\newcommand{\spn}{\ensuremath{\operatorname{span}}}
\renewcommand{\i}{\mathrm{i\,}}

\renewcommand{\implies}{\Rightarrow}

\renewcommand{\theenumi}{$\roman{enumi})$}
\renewcommand{\labelenumi}{\theenumi}

\font\ptmten=zptmcmrm scaled 1200
\newcommand{\w}{\mbox{{\ptmten w}}}
\newcommand{\z}{\mbox{{\ptmten z}}}
\renewcommand{\Re}{\mathbb{R}}

\newcommand{\cl}{\operatorname{cl}}
\newcommand{\intr}{\operatorname{int}}
\newcommand{\rank}{\operatorname{rank}}
\newcommand{\co}{\operatorname{co}}
\newcommand{\aff}{\operatorname{aff}}

\theoremstyle{plain}
\newtheorem{theorem}{Theorem}[chapter]
\newtheorem{claim}[theorem]{Claim}
\newtheorem{corollary}[theorem]{Corollary}
\newtheorem{prop}[theorem]{Proposition}
\newtheorem{fact}[theorem]{Fact}
\newtheorem{lemma}[theorem]{Lemma}

\newtheorem{remark}{Remark}[chapter]

\theoremstyle{definition}
\newtheorem{assume}[theorem]{Assumption}
\newtheorem{defn}[theorem]{Definition}
\newtheorem{problem}[theorem]{Problem}
\newtheorem{exercise}{Exercise}
\newtheorem{example}[theorem]{Example}


\begin{document}
\centerline {\textsf{\textbf{\LARGE{Homework 5}}}}
\centerline {Jaden Wang}
\vspace{.15in}
\begin{problem}[4.22]
We wish to find a sequence of stopping time tending to $ \infty$ such that it reduces $ N_t$. Define
 \begin{align*}
	T_n(\omega) = \{ \inf t\geq 0: |U(\omega) M_t(\omega)| \geq n \}.  
\end{align*}
We first show that $ T_n(\omega) \uparrow \infty \ \forall \ \omega$.  Since  $ M$ is continuous in  $ t$ and  $ U$ does not depend on time, $ N_t$ is continuous in $ t$. By a fact from class, we obtain the result.

Second, since $ M$ is a CLM, there exists a sequence of stopping time  $ (S_n)$ that reduces it. Since $ T_n \uparrow \infty$, from class we know that $ R_n := T_n \wedge S_n$ reduces $ M$ as well.  we have $M_t^{R_n} \in L^{1}$

It remains to show that $ (R_n)$ reduces $ N_t$.  Since $ |N_t^{R_n}| = |U M_t^{R_n}| \leq n $ by definition of $ R_n$, we have $ N_t^{R_n} \in L^{1}$ so we can consider conditional expectation on it:
 \begin{align*}
	\mathbb{E}\left[ N_t^{R_n}| \Sigma_s \right] &= \mathbb{E}\left[ U M_t^{R_n} | \Sigma_s \right]  \\
	&= U \mathbb{E}\left[ M_t^{R_n} | \Sigma_s \right] && U \in \Sigma_0 \subset \Sigma_s  \\
	&= U M_s^{R_n} \\
	&= N_s^{R_n} .
\end{align*}
\end{problem}
\begin{problem}[4.24]
\begin{enumerate}[label=(\arabic*)]
	\item Since $ M$ is CLM and  $ \mathbb{E}\left[ M_0 \right] =0 < \infty$, from class we know that $ (T_n)$ reduces $ M$.

		$ ( \subseteq ):$ if $ \lim_{ t \to \infty} M_t(\omega) = C < \infty$, we want to show that $ T_n(\omega) = \infty$ for some $ n$. Since  $ M$ is continuous, for any $ \epsilon>0$, there exists a $ T$  s.t.\ $|C - M_T(\omega) | < \epsilon$ and we know that  $ \sup_{t \geq 0 } M_t(\omega)$ is bounded by the max of $ \max_{[0,T]} M_t(\omega)$ and $ |C| + \epsilon$. Therefore, the supremum is bounded and we can take any $ n$ greater than this supremum to get  $ T_n(\omega) = \infty$.
		
		$ ( \supseteq):$ if $ T_n(\omega) = \infty$ for some $ n$, this implies  $ |M_t(\omega)| \leq n$ for all $ t \geq 0$. Since  $ M_t ^{T_n} = M_t$ is a martingale with continuous path, by the Martingale Convergence it has an a.s.\ limit $ M_{\infty}(\omega)$ and is finite since $ M_t(\omega)$ is bounded.

	$ (\subseteq) :$ suppose $ T(\omega) =\infty$, \emph{i.e.}  $ |M_t(\omega)| \leq n$ for all $ t \geq 0$. Since $ M_t^{T_n}$ is a martingale, $ M_0 = 0 \in L^2$, and $ \mathbb{E}\left[ (M_t^{T_n})^2 \right] \leq \sup_{t\geq 0 } (M_t^{T_n})^2 \leq n^2$, we apply the TFAE to obtain $ \mathbb{E}\left[ \left\langle M^{T_n}, M^{T_n} \right\rangle_{\infty} \right] <\infty$. It follows that $\left\langle M^{T_n}, M^{T_n} \right\rangle_\infty < \infty $ a.s. Since $ M_\infty(\omega) = M_\infty^{T_n}(\omega)$, we finally obtain $  \left\langle M,M \right\rangle_\infty(\omega) <\infty$ a.s.
\item 

	$ ( \subseteq ):$ let $ \left\langle M,M \right\rangle_\infty( \omega) < \infty$. Since $ \left\langle M,M \right\rangle_t$ is an increasing process, picking any $ n > \left\langle M,M \right\rangle_\infty(\omega)$ yields $ S_n(\omega) =\infty$.

	$ ( \supseteq ):$ suppose there exists an $ n$  s.t.\ $ \left\langle M,M \right\rangle_t(\omega) \leq n \ \forall \ t$, then by increasing process and Monotone Convergence Theorem,  $ \left\langle M,M \right\rangle_\infty (\omega) \leq n$.

	$ ( \subseteq ):$ under the same assumption, first notice that $ S_n$ is a stopping time because $ \left\langle M,M \right\rangle_t$ is continuous and $ \{n\} $ is a closed set. Thus we have $ \mathbb{E}\left[ \left\langle M^{S_n},M^{S_n} \right\rangle_\infty \right] \leq n<\infty$. Since $ M$ is a CLM, $ M^{S_n}$ is also a CLM. Since $ M_0=0 \in L^2$, we apply the TFAE to obtain that $ M^{S_n}$ is a martingale and bounded in $ L^2$. Since $ M_t^{S_n}(\omega) = M_t(\omega)$, by Martingale Convergence we obtain that $ \lim_{ t \to \infty}  M_t^{Sn}(\omega) = \lim_{ t \to \infty} M_t(\omega)$ converges a.s. and is a.s. finite since it is bounded in $ L^2$.

Putting (1) and (2) together, we obtain the final a.s. set equality.
\end{enumerate}
\end{problem}

\begin{problem}[4.25]
\begin{enumerate}[label=(\arabic*)]
	\item First, $ T_ \epsilon^{n}$ is a stopping time because $ \left\langle M^{n},M^{n} \right\rangle_t$ is continuous and $ [ \epsilon, \infty)$ is a closed set. Second, since $ M^{n}$ is a CLM, $ M^{n, \epsilon}$ is also a CLM. By definition we have  $ \left\langle M^{n, \epsilon}, M^{n, \epsilon} \right\rangle_t \leq \epsilon$, so by increasing process and MCT $ \left\langle M^{n, \epsilon}, M^{n, \epsilon} \right\rangle_\infty \leq \epsilon$ exists. Since $ M_0 = 0 \in L^2$, TFAE yields that $ M_t^{n, \epsilon}$ is a true martingale and is bounded in $ L^2$.
	\item The previous TFAE also yields $ \mathbb{E}\left[ |M_t^{n, \epsilon}|^2 \right]  = \mathbb{E}\left[ \left\langle M^{n, \epsilon}, M^{n, \epsilon} \right\rangle_t \right] \leq \epsilon $. Applying Doob's inequality, we obtain
	\begin{align*}
		\mathbb{E}\left[ \sup_{t \geq 0 } |M_t^{n, \epsilon}|^2 \right] \leq \left( \frac{2}{2-1} \right) ^2 \mathbb{E}\left[ |M_t^{n, \epsilon}|^2 \right] \leq 4 \epsilon . 
	\end{align*}
\item Since when $ T_{ \epsilon}^{n} = \infty$, we have $ M_t^{n} = M_t^{n, \epsilon}$, we can rewrite
	\begin{align*}
		\mathbb{P}\left( \sup_{t\geq 0 } |M_t^{n}| \geq a \right) &= \mathbb{P}\left( \{ \sup_{t\geq 0 } |M_t^{n, \epsilon}| \geq a\}  \cap \{T_ \epsilon^{n} = \infty\}  \right)  + \mathbb{P}\left( \{\sup_{t\geq 0 } |M_t^{n}| \geq a\} \cap \{T_ \epsilon^{n} < \infty\}   \right) \\
									  &\leq \mathbb{P}\left( \{ \sup_{t\geq 0 } |M_t^{n, \epsilon}| \geq a\} \right) + \mathbb{P}\left( T_ \epsilon^{n} < \infty \right) 
	\end{align*}
by monotonicity.
By maximal inequality of martingale, we can bound the first term as
\begin{align*}
	\frac{2}{a} \mathbb{E}\left[ \sup_{ t \geq 0 } |M_t^{n, \epsilon}| \right] \leq \frac{4 \sqrt{ \epsilon} }{ a} 
\end{align*}
Since quadratic variation is increasing, the second term can be bounded by
\begin{align*}
	\mathbb{P}\left( \left\langle M^{n}, M^{n} \right\rangle_{\infty} > \epsilon \right) 
\end{align*}
Let $ \epsilon \to 0$ and $ n \to \infty$, then by the assumption that $ \lim_{ n \to \infty} \left\langle M^{n}, M^{n} \right\rangle_\infty =0$ in probability, we obtain that
\begin{align*}
	\lim_{ n \to \infty} \mathbb{P}\left( \sup_{t\geq 0 } |M_t^{n}| \geq a \right) = 0 .
\end{align*}
Since $ a>0$ is arbitrary, this converges in probability. 

\end{enumerate}
\end{problem}

\begin{problem}[2]
	Since $ M$ is a Gaussian process, and as a martingale$ \mathbb{E}\left[ M_t \right] = \mathbb{E}\left[ M_0 \right] =0$, $ M$ is a centered Gaussian process and $ M_{t+s} - M_t$ is a centered Gaussian. To show that $ M_{t+s} - M_t$ is independent of $ \sigma(M_r: r \in [0,t])$, it suffices to show that $ \Cov\left( M_{t+s} - M_t , M_r \right) =0$. We have
\begin{align*}
	\mathbb{E}\left[ (M_{t+s} - M_t ) M_r \right] &= \mathbb{E}\left[ \mathbb{E}\left[ M_{t+s}M_r | \Sigma_t \right] - M_tM_r \right] \\
						      &= \mathbb{E}\left[ \mathbb{E}\left[ M_{t+s} | \Sigma_t \right] M_r - M_t M_r  \right] && M_r \in \Sigma_r \subset \Sigma_t \\
	&= \mathbb{E}\left[ M_tM_r - M_tM_r \right] =0 .
\end{align*}

Since $ M_t$ is Gaussian,  we know $ M_t \in L^2$. Since $ M$ is also a continuous martingale with  $ M_0=0$, by TFAE we conclude that $ M^2 - l$

We wish to show that $ M_t^2 - \mathbb{E}\left[ M_t^2 \right] $ is a continuous martingale, so by (indist) uniqueness of quadratic variation, we would obtain that $ \left\langle M,M \right\rangle_t = \mathbb{E}\left[ M_t^2 \right] $ which is a deterministic continuous monotone nondecreasing (by definition of quadratic variation) function. Consider
\begin{align*}
	\mathbb{E}\left[ \left( M_t^2 - \mathbb{E}\left[ M_t^2 \right]  \right) | \Sigma_s \right] &= \mathbb{E}\left[ M_t^2 | \Sigma_s \right] - \mathbb{E}\left[ \mathbb{E}\left[ M_t^2 \right] | \Sigma_s \right]  \\
	&= \mathbb{E}\left[ M_t^2 -M_s^2 + M_s^2 | \Sigma_s \right] - \mathbb{E}\left[ \mathbb{E}\left[ M_t^2 - M_s^2 + M_s^2 \right] | \Sigma_s  \right]  \\
	&= \mathbb{E}\left[ M_t^2 - M_s^2 | \Sigma_s \right] + M_s^2 - \underbrace{ \mathbb{E}\left[ \mathbb{E}\left[ M_t^2 - M_s^2 \right] | \Sigma_s \right]  }_{ \textrm{ tower}  }- \mathbb{E}\left[ M_s^2 \right]  \\
	&= M_s^2 - \mathbb{E}\left[ M_s^2 \right] .
\end{align*}
This shows that it is indeed a continuous martingale. 
\end{problem}

\begin{problem}[3]
First, we establish the following equality:
\begin{align*}
	\mathbb{E}\left[ M_{t_i} A_{t_{i-1}} \right] &= \mathbb{E}\left[ \mathbb{E}\left[ M_{t_i} A_{t_{i-1}} | \Sigma_{t_{i-1}} \right]  \right]  \\
						     &= \mathbb{E}\left[ \mathbb{E}\left[ M_{t_i} | \Sigma_{t_{i-1}} \right] A_{t_{i-1}} \right] &&  A_{t_{i-1}} \in \Sigma_{t_{i-1}}\\
&= \mathbb{E}\left[ M_{t_{i-1}} A_{t_{i-1}} \right]  .
\end{align*}
Since both $ M$ and $ A$ are bounded, take the usual mesh, we have
 \begin{align*}
	\left|\sum_{ i= 1}^{ p_n} M_{t_i} (A_{t_i} - A_{t_{i-1}}) \right| \leq C A \leq C' .
\end{align*}
Thus, by bounded convergence theorem, we have
\begin{align*}
	\mathbb{E} \int_{ 0}^{ \infty} M_t \d A_t  &= \mathbb{E}\left[ \lim_{ n \to \infty}  \sum_{ i= 1}^{ p_n} M_{t_i} (A_{t_i} - A_{t_{i-1}}) \right] \\
	&= \lim_{ n \to \infty} \sum_{ i= 1}^{ p_n} \left( \mathbb{E}\left[ M_{t_i}A_{t_i} \right] - \mathbb{E}\left[ M_{t_i} A_{t_{i-1}} \right]  \right)   \\
	&= \lim_{ n \to \infty} \sum_{ i= 1}^{ p_n} \left(  \mathbb{E}\left[ M_{t_i}A_{t_i} \right] - \mathbb{E}\left[ M_{t_{i-1}} A_{t_{i-1}} \right]  \right)  \\
	&= \mathbb{E}\left[ M_t A_t \right] - \mathbb{E}\left[ M_0A_0 \right]  && \textrm{telescope} \\
	&= \mathbb{E}\left[ M_t A_t \right]  .
\end{align*}
Since $ \mathbb{E}\left[ M_t A_t \right] $ is bounded, by bounded convergence theorem we can take $ t \to \infty$ for the equality to hold.
\end{problem}
\begin{problem}[4]
	$ (\implies):$ this follows immediately from the uniqueness of quadratic variation. Suppose $ 0\leq a \leq b$ and $ M_t = M_a$ a.s. for all $ t \in [a,b]$. Then consider $ M_a$ as a martingale with contant sample paths and the stopped CLM $ N_t = M_{t+a}^{b}$ with $ N_0 = M_a$. Since they equal for all $ t \geq 0$, by uniqueness, $ \left\langle M_a, M_a \right\rangle_t = \left\langle N, N \right\rangle_t$ a.s. In particular, we have a.s.
	\begin{align*}
		\left\langle M,M \right\rangle_b &= \left\langle M_{b-a+a}^{b} , M_{b-a+a}^{b} \right\rangle \\
						 &=\left\langle N,N \right\rangle_{b-a} \\
						 &= \left\langle M, M \right\rangle_a .
	\end{align*}
$ (\impliedby):$  First, since the rationals are dense in the reals, and $ M_t$ and  $ \left\langle M,M \right\rangle_t$ are continuous, it suffices to prove for $ a=q,b \in \qq$ and perform the usual countable intersection. Define $ N_t := M_t - M_{t \wedge q}$ (notice that $ N_0 = M_0 - M_0 = 0$) and $ T_q = \inf\{ t > 0: \left\langle N,N \right\rangle_t >0 \}$. Since $ (0,\infty)$ is open, $ T_q$ is a stopping time for  $ ( \Sigma_{t+})$. Under the usual condition of right-continuous filtration, $ T_q$ is indeed a stopping time for  $ ( \Sigma_t)$. Moreover, since $ M$ is a CLM, stopped process $ M_{t \wedge q}$ is also a CLM, thus $ N_t $ and $ N_t^{T_q}$ are CLMs. Suppose $ \left\langle M,M \right\rangle_b = \left\langle M,M \right\rangle_q$ where $ b \geq q$. 

If $ T_q < q$, we have  $ N_t^{T_q} = M_{t \wedge T_q} - M_{t \wedge T_q} = 0$ so $ \left\langle N^{T_q}, N^{T_q} \right\rangle_t \equiv 0$ a.s. but this contradict with the definition of $ T_q$. So we must have $ T_q \geq q$. By the definition of $ T_q$, we have
 \begin{align*}
	 0 \equiv \left\langle N^{T_q}, N^{T_q} \right\rangle_t &= \left\langle M_{t \wedge T_q} - M_{t \wedge q \wedge T_q} , M_{t \wedge T_q} - M_{t \wedge q \wedge T_q}  \right\rangle \\
	&=  \left\langle M_{t \wedge T_q} - M_{t \wedge q} , M_{t \wedge T_q} - M_{t \wedge q}  \right\rangle .
\end{align*}
Since $ N_0 = 0$, we conclude that $M_{t \wedge T_q} = M_{t \wedge q}$ a.s. Thus, $ M_{b \wedge T_q} = M_q$. If we show that $ b \leq T_q$, we are done. This can be shown from the fact that $ \left\langle M,M \right\rangle_t$ is an increasing process so $ \left\langle M,M \right\rangle_t = \left\langle M,M \right\rangle_q$ for all $ t \in [q,b]$. But $ \left\langle N,N \right\rangle_t = \left\langle M_t - M_q, M_t - M_q \right\rangle$ for $ t \in [q,b]$. 

\end{problem}
\end{document}
