\documentclass[12pt]{article}
\newcommand{\alert}[1]{{\bf \color{red} [Alert:] #1}}
\newcommand{\todo}[1]{{\bf \color{orange} [TODO:] #1}}
\newcommand{\real}[1][]{\mathbb{R}^{#1}}
\newcommand{\myeqn}[1]{(\ref{#1})}
\newcommand{\myex}[1]{Example \ref{#1}}
\newcommand{\defeq}{\stackrel{\mathrm{def}}{=}}
\newcommand{\parder}[2]{\frac{\partial #1}{\partial #2}}
\newcommand{\Lie}[3][]{\mathsf{L}_{#3}^{#1} #2}
\newcommand{\LieA}[1]{\mathsf{Lie}(#1)}
\newcommand{\lieder}[2]{\mathcal{L}_{#2} #1}
\renewcommand{\t}{^{\mbox{\tiny\sf T}}}
\newcommand{\trans}{^{\mbox{\tiny\sf T}}}
\newcommand{\markup}[1]{\{\textbf{#1}\}}
\newcommand{\msub}[1]{_\mathrm{#1}}
\newcommand{\msup}[1]{^\mathrm{#1}}
\newcommand{\inv}[1]{#1^{-1}}
\newcommand{\pinv}[1]{{#1}^{+}}
\newcommand{\myfracA}[2]{\displaystyle{\frac{#1}{#2}}}
\newcommand{\myfracB}[2]{{#1}/{#2}}
\newcommand{\mydiffA}[1]{\dot{#1}}
\newcommand{\mydiffB}[2]{\myfracA{\mathrm{d}{#1}}{\mathrm{d}{#2}}}
\newcommand{\ball}[2]{\mathcal{B}_{#1}\left(#2\right)}
\newcommand{\acos}[1]{\cos^{-1}\left(#1\right)}
\newcommand{\asin}[1]{\sin^{-1}\left(#1\right)}
\newcommand{\mani}{\mathcal{M}}
\newcommand{\tang}[2]{\mathsf{T}_{#1} #2}
\newcommand{\LieB}[2]{[ #1, #2 ]}
\newcommand{\LieBad}[3][]{\mathsf{ad}_{#2}^{#1} #3}
\newcommand{\ReachVT}{\mathcal{R}^V_T}
\newcommand{\ReachVt}{\mathcal{R}^V_t}
\newcommand{\ReachVTe}{\mathcal{R}^V_{\le T}}
\newcommand{\ReachT}{\mathcal{R}_T}
\newcommand{\Reacht}{\mathcal{R}_t}
\newcommand{\ReachTe}{\mathcal{R}_{\le T}}
\newcommand{\accLA}[1]{\mathsf{Lie}(#1)}
\newcommand{\accD}{\Delta_{\mathcal{F}}}
\newcommand{\accSA}{\mathsf{Lie}(\mathcal{G},f)}
\newcommand{\accDS}{\Delta_{\mathcal{G}}}
\newcommand{\eval}[3]{\mathsf{Ev}^{#2}_{#1}\left( #3 \right)}
\newcommand{\stlc}{\textsc{stlc}}
\newcommand{\clf}{\textsc{clf}}
\newcommand{\jqlf}{\textsc{jqlf}}
\newcommand{\dlas}{\textsc{dlas}}
\newcommand{\Ad}[2]{\mathsf{Ad}_{#1} #2}
\newcommand{\xe}{\ensuremath{x_e}}
\newcommand{\lebg}[1]{\mathcal{L}_{#1}}
\newcommand{\lebgx}[1]{\mathcal{L}_{#1 \mathrm{e}}}
\newcommand{\dom}{D}
\newcommand{\domT}{[t_0,\infty) \times D}
\newcommand{\rarrow}{\rightarrow}
\renewcommand{\d}{\mathrm{d}}
\renewcommand{\Re}{\mathbb{R}}
\newcommand{\C}{\mathrm{C}}

\newcommand{\QED}{{\unskip\nobreak\hfil\penalty50\hskip2em\vadjust{}
		\nobreak\hfil$\Box$\parfillskip=0pt\finalhyphendemerits=0\par}\vspace{0.1cm}}
\newcommand{\eoEx}{{\unskip\nobreak\hfil\penalty50\hskip0em\vadjust{}
		\nobreak\hfil$\Large\Diamond$\parfillskip=0pt\finalhyphendemerits=0\par}\vspace{0.1cm}}

\newcommand{\sgn}{\ensuremath{\operatorname{sgn}}}
\newcommand{\sat}{\ensuremath{\operatorname{sat}}}

\newcommand{\half}{\frac{1}{2}}
\newcommand{\shalf}{\mbox{$\frac{1}{2}$}}
\newcommand{\marcom}[1]{\marginpar{\footnotesize #1}}
\newcommand{\der}{\mathrm{D}}
\newcommand{\e}{\mathrm{e}}
\newcommand{\dt}{\mathrm{d}t}

\newcommand{\cA}{\ensuremath{\mathcal{A}}}
\newcommand{\cB}{\ensuremath{\mathcal{B}}}
\newcommand{\cG}{\ensuremath{\mathcal{G}}}
\newcommand{\cK}{\ensuremath{\mathcal{K}}}
\newcommand{\cW}{\ensuremath{\mathcal{W}}}
\newcommand{\cZ}{\ensuremath{\mathcal{Z}}}
\newcommand{\cS}{\ensuremath{\mathcal{S}}}
\newcommand{\cD}{\ensuremath{\mathcal{D}}}
\newcommand{\cP}{\ensuremath{\mathcal{P}}}
\newcommand{\cV}{\ensuremath{\mathcal{V}}}
\newcommand{\cL}{\ensuremath{\mathcal{L}}}
\newcommand{\cN}{\ensuremath{\mathcal{N}}}
\newcommand{\cI}{\ensuremath{\mathcal{I}}}
\newcommand{\cR}{\ensuremath{\mathcal{R}}}
\newcommand{\cM}{\ensuremath{\mathcal{M}}}
\newcommand{\cC}{\ensuremath{\mathcal{C}}}
\newcommand{\cF}{\ensuremath{\mathcal{F}}}
\newcommand{\cH}{\ensuremath{\mathcal{H}}}
\newcommand{\cO}{\ensuremath{\mathcal{O}}}
\newcommand{\cX}{\ensuremath{\mathcal{X}}}
\newcommand{\cY}{\ensuremath{\mathcal{Y}}}
\newcommand{\Ci}{\ensuremath{\mathcal{C}^\infty}}
\newcommand{\ISS}{\textsc{iss}}
\newcommand{\LISS}{\textsc{liss}}
\newcommand{\GAS}{\textsc{gas}}
\newcommand{\GS}{\textsc{gs}}
\newcommand{\LES}{\textsc{les}}
\newcommand{\GUAS}{\textsc{guas}}
\newcommand{\BIBO}{\textsc{bibo}}
\newcommand{\spec}{\ensuremath{\operatorname{spec}}}
\newcommand{\spn}{\ensuremath{\operatorname{span}}}
\renewcommand{\i}{\mathrm{i\,}}

\renewcommand{\implies}{\Rightarrow}

\renewcommand{\theenumi}{$\roman{enumi})$}
\renewcommand{\labelenumi}{\theenumi}

\font\ptmten=zptmcmrm scaled 1200
\newcommand{\w}{\mbox{{\ptmten w}}}
\newcommand{\z}{\mbox{{\ptmten z}}}
\renewcommand{\Re}{\mathbb{R}}

\newcommand{\cl}{\operatorname{cl}}
\newcommand{\intr}{\operatorname{int}}
\newcommand{\rank}{\operatorname{rank}}
\newcommand{\co}{\operatorname{co}}
\newcommand{\aff}{\operatorname{aff}}

\theoremstyle{plain}
\newtheorem{theorem}{Theorem}[chapter]
\newtheorem{claim}[theorem]{Claim}
\newtheorem{corollary}[theorem]{Corollary}
\newtheorem{prop}[theorem]{Proposition}
\newtheorem{fact}[theorem]{Fact}
\newtheorem{lemma}[theorem]{Lemma}

\newtheorem{remark}{Remark}[chapter]

\theoremstyle{definition}
\newtheorem{assume}[theorem]{Assumption}
\newtheorem{defn}[theorem]{Definition}
\newtheorem{problem}[theorem]{Problem}
\newtheorem{exercise}{Exercise}
\newtheorem{example}[theorem]{Example}


\begin{document}
\centerline {\textsf{\textbf{\LARGE{Homework 4}}}}
\centerline {Jaden Wang}
\vspace{.15in}
\begin{problem}[1]
\begin{enumerate}[label=(\alph*)]
	\item No. Here, jointly measurable means that $ (\omega,t) \mapsto Z_t(\omega)$ is $ \mathcal{ B}(\rr) \otimes \mathcal{ B}([0, \infty)) $ measurable. That implies that any slice is measurable, \emph{i.e.} if we fix an $ \omega$, then $ Z_t(\omega) = \omega(t)$ should be $ \mathcal{ B}([0,\infty))$ measurable. That is, $ \omega$ should be a $ \mathcal{ B}([0,\infty))$-measurable function. However, setting $ \omega$ to be 0 on a Vitali subset of $ [0,\infty)$ and 1 on its complement makes it not measurable, a contradiction. 
	\item Versions only care about same distribution whereas modifications care about a.s. pointwise equality, \emph{i.e.} $ \mathbb{P}\left( X_t = \widetilde{ X}_t \right) =1$ for all $ t$. Consider $ (X_t)$ to be an iid standard Gaussian process and $ \widetilde{ X}_t := - X_t$. Since standard Gaussian is symmetric about 0, they clearly have the same distribution. However, they only equal each other at exactly one point ($ 0$), so they are not modifications of each other.
\end{enumerate}
\end{problem}
\begin{problem}[2]
	Since $ f$ is smooth, it has bounded  $ f'$ in  $ [0,1]$. Similarly, $ \sqrt{t} $ is also bounded. Therefore, we have
\begin{align*}
	\frac{1}{\sqrt{2 \pi} } \int_{- \infty}^{ \infty} \int_{ 0}^{ 1} |f'(t)| \sqrt{t} x e^{-\frac{x^2}{ 2}} \d x \d t < \infty.
\end{align*}
This allows us to apply Fubini:
\begin{align*}
	\mathbb{E}\left[ \int_{ 0}^{ 1} f'(t) B_t \d t   \right] &= \mathbb{E}\left[ \int_{ 0}^{ 1} f'(t) \sqrt{t} B_1 \d t    \right]  \\
	&= \frac{1}{\sqrt{2 \pi} } \int_{- \infty}^{ \infty} \int_{ 0}^{ 1} f'(t) \sqrt{t} x e^{-\frac{x^2}{ 2}} \d x \d t  \\
	&= \frac{1}{\sqrt{2 \pi} } \int_{ 0}^{ 1} f'(t) \sqrt{t} \left( \int_{ -\infty}^{ \infty} xe^{-\frac{x^2}{2}} \d x \right) \d t    \\
	&= 0 .
\end{align*}
So the mean is 0.
\begin{align*}
	\mathbb{E}\left[ \left( \int_{ 0}^{ 1} f'(t) B_t \d t   \right)  \right] &= \mathbb{E}\left[ \int_{ 0}^{ 1} f'(t)\int_{ 0}^{ 1}  f'(s) B_t B_s \d s \d t    \right]  \\
	&= \int_{ 0}^{ 1} f'(t)\int_{ 0}^{ 1}  f'(s) \mathbb{E}\left[ B_t B_s \right] \d s \d t    \\
	&= \int_{ 0}^{ 1} f'(t)\int_{ 0}^{ 1}  f'(s) s \wedge t \d s \d t    \\
	&= \int_{ 0}^{ 1} f'(t) \left( \int_{ 0}^{ t} f'(s) s \d s + t \int_{ t}^{ 1} f'(s) \d s   \right)   \\
	&= \int_{ 0}^{ 1} f'(t) \left(( f(s) s) \big|_0^{t} - \int_{ 0}^{ t} f(s) \d s + t f(s)\big|_t^{1} \right)  \d t  \\
	&= \int_{ 0}^{ 1} f'(t) \left( f(t)t - \int_{ 0}^{ t} f(s) \d s - f(t)t  \right)  \d t  \\
	&= - \int_{ 0}^{ 1} f'(t) \int_{ 0}^{ t} f(s) \d s \d t    \\
	&= \int_{ 0}^{ 1} f(s) \int_{ 1}^{ s} f'(t) \d t \d s  && \textrm{Fubini}  \\
	&= \int_{ 0}^{ 1} f(s) f(t)\big|_1^{s} \d s  \\
	&= \int_{ 0}^{ 1} f(s)^2 \d s  .
\end{align*}
\end{problem}

\begin{problem}[3]
	Notice that since $ T_{k+1} \geq T_k$, we can rewrite
\begin{align*}
	T_{k+1} - T_k &= \inf\{ t \geq T_k: B_t = 1 + B_{T_k} \} - T_k  \\
	&= \inf\{ t - T_k \geq 0: B_t - B_{T_k} = 1 \}  \\
	&= \inf \{s \geq 0: B_{s+T_k} - B_{T_k}=1\}  \\
	&= \inf \{s \geq 0: B_s ^{(T_k)} =1\}  .
\end{align*}
By the strong Markov property, we know that $ B_t^{(T_k)} \sim B_t $ and is independent of $ \Sigma_{T_k} \supset \Sigma_{T_{k-1}} \supset \cdots \supset \Sigma_{T_1}$. Thus it is iid with $ T_1, T_2 - T_1, \ldots, T_{k-1} - T_{k-2}$ by simple induction.
\end{problem}

\begin{problem}[4]
Take any limit point $ t$ of $ E$ with sequence  $ (t_k) \subset E$ such that $ t_k \to t$, by continuity $ B_t = \lim_{ k \to \infty} B_{t_k} = \lim_{ k \to \infty} 1 = 1 $, ie any limit point  of $ E$ is in $ E$. Hence $ E$ is closed. Suppose  $ E$ is finite, then let $s:= \max E < \infty $. By the Markov property, $ B_t^{(s)}= B_{t+s} - B_s$ is a BM. Therefore it crosses $ B_s = 1$ a.s. at least once after time  $ s$ by the fact of BM that $ \sup B_t^{(s)} >0$ and  $ \inf B_t^{(s)}< 0$ for any time after $ s$, a contradiction that  $ s$ is the maximum. Thus, $ E$ must be infinite. 

Next, notice that for a fixed $ t$, we have $ \mathbb{P}\left( B_t = 1 \right) =0 $. Since $(t,\omega) \mapsto  B_t(\omega)$ is jointly measurable, we can apply Fubini on the product Lebesgue measure:
\begin{align*}
	\mu(E) = \int_{\Omega \times [0,\infty)} \mathbbm{1}_{E } \d \mu = \int_{ 0}^{ \infty} \int_{\Omega}  \mathbbm{1}_{B_t = 1 } \d \lambda \d t = \int_0^\infty \mathbb{P}\left( B_t = 1 \right)  \d t = \int_0^\infty 0 \d t = 0 .    
\end{align*}
Thus, $ E$ has Lebesgue measure zero. 

Finally, for any $ 0\leq a < b$ such that  $ (a,b) \subset E$, by the Markov property we have BM $ B_t^{(a)} \equiv 0 \ \forall \ t \in(0,b-a)$. But as a BM, it must satisfies a.s. $ \sup_{t \in (0,b-a) } >0$ and $ \inf_{t \in (0,b-a)} <0 $, so it must be that $ \mathbb{P}\left( (a,b) \subset E: 0\leq a <b \right)=0 $.
\end{problem}

\begin{problem}[5]
For any $ t_0 \in \rr$, if $ B_s = t_0$, then $ B_s ^{(t_0)}$ again has the supremum $ >0$ infimum $ <0$ property for any neighborhood, so $ t_0$ is not a local maximum. To show a.s. the set $ S$ of local maximum of BM is countably infinite, we need to establish a bijection between that and a countable set, $ \qq_+$.

First, we establish that $ \qq_+$ injects into  $S$. WLOG take any $ p<r \in \qq_+$, pick any $q,s \in \qq_+$ such that $p<q<r<s$. By continuous path and EVT, there exist local maximum$ a = \sup_{[p,q] } B_t $ and $ b = \sup_{[r,s] } B_t$. Then by HW2, we know a.s. $ a \neq b$, which yields a.s. injectivity. 

Given $ a \in S$, since it is a local maximum, there exists a neighborhood with rational boundary such that $ a$ is a maximum there. Taking its rational lower boundary yields surjectivity. Thus, there exists a.s. a bijection between $ \qq_+$ and $ S$ by taking the rational lower boundary of the local maximum.

Finally, density again follows from continuity of paths and EVT. For any $ 0\leq s<t$, there exists an $ r \in (s,t)$ such that $ B_r$ is a local maximum of  $ [s+ \epsilon, t- \epsilon]$ for sufficiently small $ \epsilon$.

\end{problem}
\end{document}
