\documentclass[12pt]{article}
\newcommand{\alert}[1]{{\bf \color{red} [Alert:] #1}}
\newcommand{\todo}[1]{{\bf \color{orange} [TODO:] #1}}
\newcommand{\real}[1][]{\mathbb{R}^{#1}}
\newcommand{\myeqn}[1]{(\ref{#1})}
\newcommand{\myex}[1]{Example \ref{#1}}
\newcommand{\defeq}{\stackrel{\mathrm{def}}{=}}
\newcommand{\parder}[2]{\frac{\partial #1}{\partial #2}}
\newcommand{\Lie}[3][]{\mathsf{L}_{#3}^{#1} #2}
\newcommand{\LieA}[1]{\mathsf{Lie}(#1)}
\newcommand{\lieder}[2]{\mathcal{L}_{#2} #1}
\renewcommand{\t}{^{\mbox{\tiny\sf T}}}
\newcommand{\trans}{^{\mbox{\tiny\sf T}}}
\newcommand{\markup}[1]{\{\textbf{#1}\}}
\newcommand{\msub}[1]{_\mathrm{#1}}
\newcommand{\msup}[1]{^\mathrm{#1}}
\newcommand{\inv}[1]{#1^{-1}}
\newcommand{\pinv}[1]{{#1}^{+}}
\newcommand{\myfracA}[2]{\displaystyle{\frac{#1}{#2}}}
\newcommand{\myfracB}[2]{{#1}/{#2}}
\newcommand{\mydiffA}[1]{\dot{#1}}
\newcommand{\mydiffB}[2]{\myfracA{\mathrm{d}{#1}}{\mathrm{d}{#2}}}
\newcommand{\ball}[2]{\mathcal{B}_{#1}\left(#2\right)}
\newcommand{\acos}[1]{\cos^{-1}\left(#1\right)}
\newcommand{\asin}[1]{\sin^{-1}\left(#1\right)}
\newcommand{\mani}{\mathcal{M}}
\newcommand{\tang}[2]{\mathsf{T}_{#1} #2}
\newcommand{\LieB}[2]{[ #1, #2 ]}
\newcommand{\LieBad}[3][]{\mathsf{ad}_{#2}^{#1} #3}
\newcommand{\ReachVT}{\mathcal{R}^V_T}
\newcommand{\ReachVt}{\mathcal{R}^V_t}
\newcommand{\ReachVTe}{\mathcal{R}^V_{\le T}}
\newcommand{\ReachT}{\mathcal{R}_T}
\newcommand{\Reacht}{\mathcal{R}_t}
\newcommand{\ReachTe}{\mathcal{R}_{\le T}}
\newcommand{\accLA}[1]{\mathsf{Lie}(#1)}
\newcommand{\accD}{\Delta_{\mathcal{F}}}
\newcommand{\accSA}{\mathsf{Lie}(\mathcal{G},f)}
\newcommand{\accDS}{\Delta_{\mathcal{G}}}
\newcommand{\eval}[3]{\mathsf{Ev}^{#2}_{#1}\left( #3 \right)}
\newcommand{\stlc}{\textsc{stlc}}
\newcommand{\clf}{\textsc{clf}}
\newcommand{\jqlf}{\textsc{jqlf}}
\newcommand{\dlas}{\textsc{dlas}}
\newcommand{\Ad}[2]{\mathsf{Ad}_{#1} #2}
\newcommand{\xe}{\ensuremath{x_e}}
\newcommand{\lebg}[1]{\mathcal{L}_{#1}}
\newcommand{\lebgx}[1]{\mathcal{L}_{#1 \mathrm{e}}}
\newcommand{\dom}{D}
\newcommand{\domT}{[t_0,\infty) \times D}
\newcommand{\rarrow}{\rightarrow}
\renewcommand{\d}{\mathrm{d}}
\renewcommand{\Re}{\mathbb{R}}
\newcommand{\C}{\mathrm{C}}

\newcommand{\QED}{{\unskip\nobreak\hfil\penalty50\hskip2em\vadjust{}
		\nobreak\hfil$\Box$\parfillskip=0pt\finalhyphendemerits=0\par}\vspace{0.1cm}}
\newcommand{\eoEx}{{\unskip\nobreak\hfil\penalty50\hskip0em\vadjust{}
		\nobreak\hfil$\Large\Diamond$\parfillskip=0pt\finalhyphendemerits=0\par}\vspace{0.1cm}}

\newcommand{\sgn}{\ensuremath{\operatorname{sgn}}}
\newcommand{\sat}{\ensuremath{\operatorname{sat}}}

\newcommand{\half}{\frac{1}{2}}
\newcommand{\shalf}{\mbox{$\frac{1}{2}$}}
\newcommand{\marcom}[1]{\marginpar{\footnotesize #1}}
\newcommand{\der}{\mathrm{D}}
\newcommand{\e}{\mathrm{e}}
\newcommand{\dt}{\mathrm{d}t}

\newcommand{\cA}{\ensuremath{\mathcal{A}}}
\newcommand{\cB}{\ensuremath{\mathcal{B}}}
\newcommand{\cG}{\ensuremath{\mathcal{G}}}
\newcommand{\cK}{\ensuremath{\mathcal{K}}}
\newcommand{\cW}{\ensuremath{\mathcal{W}}}
\newcommand{\cZ}{\ensuremath{\mathcal{Z}}}
\newcommand{\cS}{\ensuremath{\mathcal{S}}}
\newcommand{\cD}{\ensuremath{\mathcal{D}}}
\newcommand{\cP}{\ensuremath{\mathcal{P}}}
\newcommand{\cV}{\ensuremath{\mathcal{V}}}
\newcommand{\cL}{\ensuremath{\mathcal{L}}}
\newcommand{\cN}{\ensuremath{\mathcal{N}}}
\newcommand{\cI}{\ensuremath{\mathcal{I}}}
\newcommand{\cR}{\ensuremath{\mathcal{R}}}
\newcommand{\cM}{\ensuremath{\mathcal{M}}}
\newcommand{\cC}{\ensuremath{\mathcal{C}}}
\newcommand{\cF}{\ensuremath{\mathcal{F}}}
\newcommand{\cH}{\ensuremath{\mathcal{H}}}
\newcommand{\cO}{\ensuremath{\mathcal{O}}}
\newcommand{\cX}{\ensuremath{\mathcal{X}}}
\newcommand{\cY}{\ensuremath{\mathcal{Y}}}
\newcommand{\Ci}{\ensuremath{\mathcal{C}^\infty}}
\newcommand{\ISS}{\textsc{iss}}
\newcommand{\LISS}{\textsc{liss}}
\newcommand{\GAS}{\textsc{gas}}
\newcommand{\GS}{\textsc{gs}}
\newcommand{\LES}{\textsc{les}}
\newcommand{\GUAS}{\textsc{guas}}
\newcommand{\BIBO}{\textsc{bibo}}
\newcommand{\spec}{\ensuremath{\operatorname{spec}}}
\newcommand{\spn}{\ensuremath{\operatorname{span}}}
\renewcommand{\i}{\mathrm{i\,}}

\renewcommand{\implies}{\Rightarrow}

\renewcommand{\theenumi}{$\roman{enumi})$}
\renewcommand{\labelenumi}{\theenumi}

\font\ptmten=zptmcmrm scaled 1200
\newcommand{\w}{\mbox{{\ptmten w}}}
\newcommand{\z}{\mbox{{\ptmten z}}}
\renewcommand{\Re}{\mathbb{R}}

\newcommand{\cl}{\operatorname{cl}}
\newcommand{\intr}{\operatorname{int}}
\newcommand{\rank}{\operatorname{rank}}
\newcommand{\co}{\operatorname{co}}
\newcommand{\aff}{\operatorname{aff}}

\theoremstyle{plain}
\newtheorem{theorem}{Theorem}[chapter]
\newtheorem{claim}[theorem]{Claim}
\newtheorem{corollary}[theorem]{Corollary}
\newtheorem{prop}[theorem]{Proposition}
\newtheorem{fact}[theorem]{Fact}
\newtheorem{lemma}[theorem]{Lemma}

\newtheorem{remark}{Remark}[chapter]

\theoremstyle{definition}
\newtheorem{assume}[theorem]{Assumption}
\newtheorem{defn}[theorem]{Definition}
\newtheorem{problem}[theorem]{Problem}
\newtheorem{exercise}{Exercise}
\newtheorem{example}[theorem]{Example}


\begin{document}
\centerline {\textsf{\textbf{\LARGE{Homework 2}}}}
\centerline {Jaden Wang}
\vspace{.15in}
\begin{problem}[2.25]
\begin{enumerate}[label=(\arabic*)]
	\item First, we show that $ (W_t)$ is a pre-BM. We shall use the second definition and show that it is a centered Gaussian process with  $ K(s,t) = s \wedge t$.

		Uniform scaling doesn't affect the mean so $ W_t$ is still zero-mean.  Since $ (B_t)$ is a centered Gaussian process, then clearly  $ \frac{1}{t} B_{\frac{1}{t}}$ are scalar multiple of centered Gaussians from the same centered Gaussian space and therefore a centered Gaussian process. Let $0<s<t $, then we have $ 0< \frac{1}{t}<\frac{1}{s}$ and
		\begin{align*}
			K(s,t) = \mathbb{E}\left[ W_t W_s \right] = \mathbb{E}\left[ t B_{\frac{1}{t}} s B_{\frac{1}{s}} \right] = ts \left( \frac{1}{t}\wedge \frac{1}{s} \right) = ts \frac{1}{t} = s = s \wedge t.
		\end{align*}
		With $ W_0 :=0$ with $ K(0,0) =0 \wedge 0 = 0$, this proves that $ (W_t)$ is a pre-BM on $ [0,\infty)$. 

Now we wish to show that $ (W_t)$ has continuous sample path on  $ [0,\infty)$. For $ (0, \infty)$, we know inversion $ i: (0,\infty) \to (0, \infty), t \mapsto \frac{1}{t}$ is continuous, $ t \mapsto  B_t( \omega)$ is continuous for a fixed $ \omega$, and multiplication by $ t$ is continuous, so the composition  $ t B_{\frac{1}{t}} (\omega)$ is continuous on $ (0,\infty)$. It remains to show that $ W_t$ is right continuous at  $ t=0$. 

Recall that in the proof of Komogorov lemma, we applied the analytic lemma to show that for a pre-BM $ (W_t)$ (since it satisfies the assumption of Komogorov), we have
 \begin{align*}
	\mathbb{P}\left( W_s( \omega), W_t(\omega) \right) \leq \frac{2k( \omega)}{ 1- 2^{- \alpha}} |t-s|^{ \alpha},
\end{align*}
for $ t,s \in D$ (dyadic rationals that partitions the interval $ [0,1]$) some $ k( \omega) < \infty$ a.s. and $ \alpha \in \left( 0, \frac{1}{2} - \frac{1}{q} \right) $ with $ q >2$. This implies that as $ t \to 0$, over $ D$ we have a.s. $ W_t \to W_0 :=0$. Since $ W_t$ is continuous on  $ (0,\infty)$ and $ D$ is dense in  $ [0,1]$, we can extend this result to the entire  $ [0,1]$ a.s. by the typical density argument. That is,  $ W_t(\omega)$ is continuous a.s. on $ [0,\infty)$. Hence, $ (W_t)$ is indistinguishable of a BM.
\item Since time inversion is an involution, we have $ B_t = t W_{\frac{1}{t}}$ a.s., so we have a.s.
	\begin{align*}
		\lim_{ t \to \infty} \frac{B_t}{t} = \lim_{ t \to \infty} W_{\frac{1}{t}} = W_0 = 0.
	\end{align*}
\end{enumerate}
\end{problem}
\begin{problem}[2.28]
We wish to use a fact from Theorem 2.21: for every $ t >0$,  $ \sup_{s\leq t } B_s$ has the same distribution as $ |B_t|$. Since  $ (B_t)$ is a Gaussian process,  $ |B_t|$ has continuous distribution (wrt $ \Sigma_t$) as well. It follows that $ \sup_{s\leq t } B_s $ has continuous distribution for BM $ (B_t)$ wrt to $ \Sigma_t$.

We want to get to the above sup form. Under sup we can WLOG subtract $ B_r$ from both LHS and RHS, which yields the LHS
\begin{align*}
	\sup_{p\leq t\leq q} B_t - B_r &= \sup_{p\leq t\leq q } (B_t - B_p) + (B_p - B_r) \\
	&= \sup_{t \leq q-p } B_t^{(p)} + (B_p-B_r) .
\end{align*}
and the RHS
\begin{align*}
	\sup_{r\leq t\leq s } B_t - B_r = \sup_{ t \leq s-r } B_t^{(r)}   . 
\end{align*}
Since $ (B_t^{(p)}) $ and $ (B_t^{(r)})$ are also BMs with continuous distributions (wrt $ \Sigma_r$ and $\Sigma_s$), by the fact the sup terms have continuous distributions, and $ B_p - B_r$ also have continuous distributions wrt to $ \Sigma_r$. Since  $ p<q<r<s$, by Markov property the RHS is independent of  $ \Sigma_r$, which is the sigma algebra that makes LHS distributions continuous. Thus, the probability that two independent r.v.s with continuous distributions equal is 0 (which follows from separating the double integral where the inner integral is over a single point and thus 0). That is, $ \mathbb{P}\left( \sup_{p \leq t \leq q } B_t \neq \sup_{r\leq t \leq s } B_t   \right) =1$.
\end{problem}

\begin{problem}[2.29]
The first event is equivalent to $ A =  \bigcap_{ n =1}^{\infty} \sup_{0<t\leq \frac{1}{n} } \left\{\frac{B_t}{ \sqrt{t} } = +\infty \right\}  $. We wish to show that $ A \in \Sigma_{0+} $ and $ \mathbb{P}\left( A \right) >0$ to conclude that $ \mathbb{P}\left( A \right) =1$ by Blumenthal's 0/1 law. The infimum case follows from symmetry.

Since $ A$ is a tail event, the first finitely many events don't matter in the intersection, so for all $ r >0$, $ n \in \nn$, we have
 \begin{align*}
	A = A_r := \bigcap_{ n \geq \frac{1}{r}} \sup_{0<t < \frac{1}{n} } \left\{  \frac{B_t}{\sqrt{t} } = +\infty \right\}=: \bigcap_{ n \geq \frac{1}{r}} A_n. 
\end{align*}
Since $ \frac{1}{n} < r$, by the Markov property, $ A_n \in \Sigma_r $ so the countable intersection $ A_r$ is in $ \Sigma_r$. Intersecting over all $ r>0$ yields $ A \in \Sigma_{0+} $.

For $ M > 0$ and $ N \in \nn$, we have
\begin{align*}
	\mathbb{P}\left( A \right) &\geq \limsup_{ n \to \infty} \mathbb{P}\left( A_n = + \infty \right)  && \text{Fatou}  \\
	&\geq \limsup_{ n \to \infty} \mathbb{P}\left( A_n > M \right)   \\
	&\geq \mathbb{P}\left( A_N > M \right)   \\
	&= \mathbb{P}\left( \frac{B_{\frac{1}{N}}}{ \sqrt{\frac{1}{N}} } > M \right)  \\
	&= 1 - \mathbb{P}\left(  \frac{B_{\frac{1}{N}}}{ \sqrt{\frac{1}{N}} } \leq  M \right)  \\
	& > 0 ,
\end{align*}
where the last inequality comes from fact of the CDF of Gaussian distribution. It follows that $ \mathbb{P}\left( A \right) =1$.
\end{problem}
To show that the right derivative doesn't exist, it suffices to show that the following limsup and liminf do not agree. For $ s \in [0, \infty)$, we have
\begin{align*}
	\limsup_{ t \to 0} \frac{B_{s+t} - B_s}{ t} &= \limsup_{ t \to 0} \frac{1}{\sqrt{t} } \frac{B_t^{(s)}}{ \sqrt{t} } \\
	&\geq \limsup_{ t \to 0} \frac{B_t^{(s)}}{ \sqrt{t} }  = + \infty .
\end{align*}
However,
\begin{align*}
	\liminf_{ t \to 0} \frac{1}{\sqrt{t} } \frac{B_t^{(s)}}{ \sqrt{t} } < 0,
\end{align*}
since the first term is positive and the second term tends to  $ -\infty$. Therefore, they disagree.

Alternatively, I believe we can use the same time rescaling proof technique we used for the non-montonoe property to prove the statement directly (use rescaling $ t \mapsto t \lambda^{-1}$, where $ \lambda = \frac{\delta}{ M}$). But since it doesn't directly use 0/1 law and instead use its consequence involving the sup, I omit it. 
\end{document}
