\documentclass[12pt]{article}
\newcommand{\alert}[1]{{\bf \color{red} [Alert:] #1}}
\newcommand{\todo}[1]{{\bf \color{orange} [TODO:] #1}}
\newcommand{\real}[1][]{\mathbb{R}^{#1}}
\newcommand{\myeqn}[1]{(\ref{#1})}
\newcommand{\myex}[1]{Example \ref{#1}}
\newcommand{\defeq}{\stackrel{\mathrm{def}}{=}}
\newcommand{\parder}[2]{\frac{\partial #1}{\partial #2}}
\newcommand{\Lie}[3][]{\mathsf{L}_{#3}^{#1} #2}
\newcommand{\LieA}[1]{\mathsf{Lie}(#1)}
\newcommand{\lieder}[2]{\mathcal{L}_{#2} #1}
\renewcommand{\t}{^{\mbox{\tiny\sf T}}}
\newcommand{\trans}{^{\mbox{\tiny\sf T}}}
\newcommand{\markup}[1]{\{\textbf{#1}\}}
\newcommand{\msub}[1]{_\mathrm{#1}}
\newcommand{\msup}[1]{^\mathrm{#1}}
\newcommand{\inv}[1]{#1^{-1}}
\newcommand{\pinv}[1]{{#1}^{+}}
\newcommand{\myfracA}[2]{\displaystyle{\frac{#1}{#2}}}
\newcommand{\myfracB}[2]{{#1}/{#2}}
\newcommand{\mydiffA}[1]{\dot{#1}}
\newcommand{\mydiffB}[2]{\myfracA{\mathrm{d}{#1}}{\mathrm{d}{#2}}}
\newcommand{\ball}[2]{\mathcal{B}_{#1}\left(#2\right)}
\newcommand{\acos}[1]{\cos^{-1}\left(#1\right)}
\newcommand{\asin}[1]{\sin^{-1}\left(#1\right)}
\newcommand{\mani}{\mathcal{M}}
\newcommand{\tang}[2]{\mathsf{T}_{#1} #2}
\newcommand{\LieB}[2]{[ #1, #2 ]}
\newcommand{\LieBad}[3][]{\mathsf{ad}_{#2}^{#1} #3}
\newcommand{\ReachVT}{\mathcal{R}^V_T}
\newcommand{\ReachVt}{\mathcal{R}^V_t}
\newcommand{\ReachVTe}{\mathcal{R}^V_{\le T}}
\newcommand{\ReachT}{\mathcal{R}_T}
\newcommand{\Reacht}{\mathcal{R}_t}
\newcommand{\ReachTe}{\mathcal{R}_{\le T}}
\newcommand{\accLA}[1]{\mathsf{Lie}(#1)}
\newcommand{\accD}{\Delta_{\mathcal{F}}}
\newcommand{\accSA}{\mathsf{Lie}(\mathcal{G},f)}
\newcommand{\accDS}{\Delta_{\mathcal{G}}}
\newcommand{\eval}[3]{\mathsf{Ev}^{#2}_{#1}\left( #3 \right)}
\newcommand{\stlc}{\textsc{stlc}}
\newcommand{\clf}{\textsc{clf}}
\newcommand{\jqlf}{\textsc{jqlf}}
\newcommand{\dlas}{\textsc{dlas}}
\newcommand{\Ad}[2]{\mathsf{Ad}_{#1} #2}
\newcommand{\xe}{\ensuremath{x_e}}
\newcommand{\lebg}[1]{\mathcal{L}_{#1}}
\newcommand{\lebgx}[1]{\mathcal{L}_{#1 \mathrm{e}}}
\newcommand{\dom}{D}
\newcommand{\domT}{[t_0,\infty) \times D}
\newcommand{\rarrow}{\rightarrow}
\renewcommand{\d}{\mathrm{d}}
\renewcommand{\Re}{\mathbb{R}}
\newcommand{\C}{\mathrm{C}}

\newcommand{\QED}{{\unskip\nobreak\hfil\penalty50\hskip2em\vadjust{}
		\nobreak\hfil$\Box$\parfillskip=0pt\finalhyphendemerits=0\par}\vspace{0.1cm}}
\newcommand{\eoEx}{{\unskip\nobreak\hfil\penalty50\hskip0em\vadjust{}
		\nobreak\hfil$\Large\Diamond$\parfillskip=0pt\finalhyphendemerits=0\par}\vspace{0.1cm}}

\newcommand{\sgn}{\ensuremath{\operatorname{sgn}}}
\newcommand{\sat}{\ensuremath{\operatorname{sat}}}

\newcommand{\half}{\frac{1}{2}}
\newcommand{\shalf}{\mbox{$\frac{1}{2}$}}
\newcommand{\marcom}[1]{\marginpar{\footnotesize #1}}
\newcommand{\der}{\mathrm{D}}
\newcommand{\e}{\mathrm{e}}
\newcommand{\dt}{\mathrm{d}t}

\newcommand{\cA}{\ensuremath{\mathcal{A}}}
\newcommand{\cB}{\ensuremath{\mathcal{B}}}
\newcommand{\cG}{\ensuremath{\mathcal{G}}}
\newcommand{\cK}{\ensuremath{\mathcal{K}}}
\newcommand{\cW}{\ensuremath{\mathcal{W}}}
\newcommand{\cZ}{\ensuremath{\mathcal{Z}}}
\newcommand{\cS}{\ensuremath{\mathcal{S}}}
\newcommand{\cD}{\ensuremath{\mathcal{D}}}
\newcommand{\cP}{\ensuremath{\mathcal{P}}}
\newcommand{\cV}{\ensuremath{\mathcal{V}}}
\newcommand{\cL}{\ensuremath{\mathcal{L}}}
\newcommand{\cN}{\ensuremath{\mathcal{N}}}
\newcommand{\cI}{\ensuremath{\mathcal{I}}}
\newcommand{\cR}{\ensuremath{\mathcal{R}}}
\newcommand{\cM}{\ensuremath{\mathcal{M}}}
\newcommand{\cC}{\ensuremath{\mathcal{C}}}
\newcommand{\cF}{\ensuremath{\mathcal{F}}}
\newcommand{\cH}{\ensuremath{\mathcal{H}}}
\newcommand{\cO}{\ensuremath{\mathcal{O}}}
\newcommand{\cX}{\ensuremath{\mathcal{X}}}
\newcommand{\cY}{\ensuremath{\mathcal{Y}}}
\newcommand{\Ci}{\ensuremath{\mathcal{C}^\infty}}
\newcommand{\ISS}{\textsc{iss}}
\newcommand{\LISS}{\textsc{liss}}
\newcommand{\GAS}{\textsc{gas}}
\newcommand{\GS}{\textsc{gs}}
\newcommand{\LES}{\textsc{les}}
\newcommand{\GUAS}{\textsc{guas}}
\newcommand{\BIBO}{\textsc{bibo}}
\newcommand{\spec}{\ensuremath{\operatorname{spec}}}
\newcommand{\spn}{\ensuremath{\operatorname{span}}}
\renewcommand{\i}{\mathrm{i\,}}

\renewcommand{\implies}{\Rightarrow}

\renewcommand{\theenumi}{$\roman{enumi})$}
\renewcommand{\labelenumi}{\theenumi}

\font\ptmten=zptmcmrm scaled 1200
\newcommand{\w}{\mbox{{\ptmten w}}}
\newcommand{\z}{\mbox{{\ptmten z}}}
\renewcommand{\Re}{\mathbb{R}}

\newcommand{\cl}{\operatorname{cl}}
\newcommand{\intr}{\operatorname{int}}
\newcommand{\rank}{\operatorname{rank}}
\newcommand{\co}{\operatorname{co}}
\newcommand{\aff}{\operatorname{aff}}

\theoremstyle{plain}
\newtheorem{theorem}{Theorem}[chapter]
\newtheorem{claim}[theorem]{Claim}
\newtheorem{corollary}[theorem]{Corollary}
\newtheorem{prop}[theorem]{Proposition}
\newtheorem{fact}[theorem]{Fact}
\newtheorem{lemma}[theorem]{Lemma}

\newtheorem{remark}{Remark}[chapter]

\theoremstyle{definition}
\newtheorem{assume}[theorem]{Assumption}
\newtheorem{defn}[theorem]{Definition}
\newtheorem{problem}[theorem]{Problem}
\newtheorem{exercise}{Exercise}
\newtheorem{example}[theorem]{Example}


\begin{document}
\centerline {\textsf{\textbf{\LARGE{Homework 3}}}}
\centerline {Jaden Wang}
\vspace{.15in}
\begin{problem}[2]
\begin{enumerate}[label=(\alph*)]
	\item Recall from HW2 that $ \lim_{ t \to \infty} \frac{B_t}{ t} = 0$ a.s. so $ \lim_{ t \to \infty} \frac{B_t}{ t} -\frac{1}{2} = -\frac{1}{2}$ a.s. Therefore, we have a.s.
	\begin{align*}
		\lim_{ t \to \infty} X_t &= \lim_{ t \to \infty} \exp \left( t \left( \frac{B_t}{t} - \frac{1}{2} \right)  \right)  \\
		&= \exp (- \infty) = 0.
	\end{align*}
\item No. Suppose that $ X_t \to X$ in  $ L^{1}$ for some $ X$, then after rational discretization there exists a subsequence such that  $ X_t \to X$ a.s., but since $ X_t \to 0$ a.s., this forces $ X = 0$ a.s. However, we have
	 \begin{align*}
		\mathbb{E}\left[ X_t \right] = \mathbb{E}\left[ e^{B_t} e^{-\frac{t}{2}} \right]  = \mathbb{E}\left[ e^{\sqrt{t} B_1} \right] e^{-\frac{t}{2}} = e^{\frac{t}{2}} e^{-\frac{t}{2}} = 1 \neq 0 = \mathbb{E}\left[ X \right] ,
	\end{align*}
	a contradiction.
\end{enumerate}
\end{problem}
\begin{problem}[3]
\begin{enumerate}[label=(\alph*)]
	\item Since each $ T_n$ is a stopping time, $ \{T_n \leq t\} \in \mathcal{ F}_t \ \forall \ n,t$. Then
		\begin{align*}
			\{ \sup_{n } T_n \leq t\} &= \bigcap_{ n} \{T_n \leq t\} \in \mathcal{ F}_t .  
		\end{align*}
		Thus $ \sup_{n } T_n$ is a stopping time.
	\item The case for infimum is different because for infimum to be $ \leq t$,  it is possible that all $T_n > t$ but the infimum converges to $ t$. Since $ \mathcal{ F}_t$ is right continuous, we have
	\begin{align*}
		\{ \inf_{n } T_n \leq t \} &= \bigcup_{ n} \underbrace{ \{T_n \leq t\}}_{ \mathcal{ F}_t} \cup \underbrace{ \bigcap_{ s >t}  \{T_n \leq s\}}_{ \mathcal{ F}_{t+}  = \mathcal{ F}_t}   \in \mathcal{ F}_t.
	\end{align*}
	Thus $ \inf_{n } T_n$ is a stopping time.

	Define $ S_n := \sup_{m\geq n } T_m$ and $ I_n := \inf_{m\geq n } T_m $. It follows from above that $ S_n$ and $ I_n$ are stopping times. Then
	\begin{align*}
		\{\limsup_n T_n \leq t \} &= \{\inf_{n } \sup_{m\geq n } T_m \leq t\} = \{\inf_{n } S_n \leq t\} \in \mathcal{ F}_t \\
		\{\liminf_n T_n \leq t\} &= \{\sup_{n } \inf_{m\geq n } T_m \leq t \} = \{\sup_{n } I_n \leq t \} \in \mathcal{ F}_t  . 
	\end{align*}
\end{enumerate}
\end{problem}
\begin{problem}[4]
We first show that $ (X_n)$ has uncorrelated increments. Let $ m\leq n \in \nn$, we repeatedly apply the Tower property:
\begin{align*}
	\mathbb{E}\left[ (X_n-X_m)^2 \right] &= \mathbb{E}\left[ \mathbb{E}\left[ (X_n-X_m)^2 | \mathcal{ F}_m \right]  \right]  \\
	&= \mathbb{E}\left[ \mathbb{E}\left[ X_n^2 - 2 X_nX_m + X_m^2 | \mathcal{F}_{ m}  \right]  \right]  \\
	&= \mathbb{E}\left[ \mathbb{E}\left[ X_n^2 | \mathcal{F}_{ m}  \right] - 2 \mathbb{E}\left[ X_n|\mathcal{F}_{ m}  \right]X_m + X_m^2 \right]  \\
	&= \mathbb{E}\left[ \mathbb{E}\left[ X_n^2 | \mathcal{F}_{ m}  \right] - 2 X_m^2 + X_m^2 \right]  \\
	&= \mathbb{E}\left[ \mathbb{E}\left[ X_n^2| \mathcal{F}_{ m}  \right] - X_m^2 \right]  \\
	&= \mathbb{E}\left[ X_n^2-X_m^2 \right]  .
\end{align*}
Let $S := \sup_{n } \mathbb{E}\left[ X_n^2 \right] < \infty$. Then $ \mathbb{E}\left[ X_n^2 - X_m^2 \right] \leq 2S $. Consider
\begin{align*}
	\mathbb{E}\left[ (X_n - X_0)^2 \right] &=\mathbb{E}\left[ X_n^2 - X_0^2 \right] \\
	&= \mathbb{E}\left[ \sum_{ i=0}^{ n} (X_{i+1}^2 - X_i^2) \right] && \textrm{ telescope}  \\
	&=  \mathbb{E}\left[ \sum_{ i=0}^{ n} (X_{i+1} - X_i)^2 \right]  \\
	&= \sum_{ i=0}^{ n} \mathbb{E}\left[ (X_{i+1}-X_i)^2 \right]  \leq 2S.
\end{align*}
Since each term is nonnegative, the partial sum is an monotone increasing sequence, and is bounded above by $ 2S$, the series converges by MCT, and the tail sum tends to 0. That is, given any $ \epsilon>0$, by choosing  $ m,n$ large enough,  $ \mathbb{E}\left[ (X_n-X_m)^2 \right] < \epsilon$. Thus $ (X_n)$ is Cauchy and therefore converges in $ L^2$.

\end{problem}
\begin{problem}[5]
\begin{enumerate}[label=(\alph*)]
	We first prove the hints.
	\item $ (\implies)$ is a straightforward computation using LOTUS.

		$ (\impliedby):$ If the MGF $ \mathbb{E}\left[ e^{ \lambda X} \right] = e^{ \lambda^2 /2}$ for every real $ \lambda$, then since both sides are analytic functions, by analytic continuation we can extend the equality to the complex plane and obtain the characteristic function $ \mathbb{E}\left[ e^{ i\lambda X} \right] = e^{ -\lambda^2 /2}$. Since characteristic function uniquely determines the distribution, $ X$ must be standard Gaussian.
	\item If the conditional MGF equals unconditional MGF, then by the characteristic equation argument above, the conditional and unconditional probability distributions of $ X$ must be the same. That is, for any $ x \in \rr$ and $ B \in \mathcal{ B}$ such that $ \mathbb{P}\left( B \right) >0 $, we have
\begin{align*}
	\mathbb{P}\left( X\leq x | B \right) &= \mathbb{P}\left( X \leq x \right) \\
	\frac{\mathbb{P}\left( X\leq x \cap B \right) }{ \mathbb{P}\left( B \right) } &= \mathbb{P}\left( X\leq x \right)  \\
	\mathbb{P}\left( X \leq x \cap B \right) &= \mathbb{P}\left( X\leq x \right) \mathbb{P}\left( B \right)  .
\end{align*}
When $ B$ is a null-set,  \emph{i.e.} $ \mathbb{P}\left(B \right) =0$, the equality trivially holds. This proves independence since $ X\leq x$ and  $ B$ are generators of  $ \sigma(X)$ and $ \mathcal{ B}$.
\end{enumerate}
Since $ X_t$ already has continuous paths, it remains to show that it is a pre-BM. We shall use definition 3. We already have  $ X_0 = 0$ a.s. Notice that since $ M_t^{(1)}$ is $ \mathcal{F}_{ t} $-measurable, $ X_t = \log M_t^{(1)} + \frac{t}{2}$ is also $ \mathcal{F}_{ t} $-measurable. 

%Moreover, we have
%\begin{align*}
%	\mathbb{E}\left[ M_t^{( \alpha)} \right] &= \mathbb{E}\left[ M_0^{( \alpha)} \right] = M_0^{( \alpha)} = 1 \\
%	\mathbb{E}\left[ e^{ \alpha X_t} \right]  &= e^{ \frac{ \alpha^2}{ 2}t} .
%\end{align*}
%By hint (a), we have $ \frac{1}{\sqrt{t} } X_t \sim N(0,1)$ and therefore $ X_t \sim N(0,t)$.

Let $ 0 \leq s \leq t$, so $\mathcal{F}_{ s} \subset \mathcal{F}_{ t}$. Next we show for  $ X_t - X_s \sim N(0, t-s)$. Consider
\begin{align*}
	\mathbb{E}\left[ e^{X_t- X_s} \right] &= \mathbb{E}\left[ \mathbb{E}\left[ e^{X_t-X_s} | \mathcal{F}_{ s}  \right]  \right] && \text{tower rule}  \\
					      &= \mathbb{E}\left[ \mathbb{E}\left[ e^{X_t} |\mathcal{F}_{ s} \right] e^{-X_s}   \right]  && X_s \in \mathcal{F}_{ s} \\
					      &= \mathbb{E}\left[ \mathbb{E}\left[ e^{X_t - \frac{1}{2}t} | \mathcal{F}_{ s}  \right] e^{-X_s + \frac{1}{2} s} \right] e^{\frac{1}{2}(t-s)}  \\
					      &= \mathbb{E}\left[ \mathbb{E}\left[ M_t^{(1)} | \mathcal{F}_{ s}  \right] \frac{1}{M_s^{(1)} }  \right] e^{\frac{1}{2}(t-s)} \\
					      &= \mathbb{E}\left[ M_s^{(1)} \frac{1}{M_s ^{(1)}} \right] e^{\frac{1}{2}(t-s)} && \text{Martingale} \\
					      &= e^{\frac{1}{2}(t-s)} .
\end{align*}
Then the result follows from hint (a).

Finally, we show independent increment. Since $ \mathcal{F}_{ r} \subset \mathcal{F}_{ s}$ for all $ 0\leq r \leq s$, it suffices to show that $ X_t - X_s$ is independent of $ \mathcal{F}_{ s} $. Using similar computation from above, we obtain 
\begin{align*}
	\mathbb{E}\left[ e^{ X_t- X_s} | \mathcal{F}_{ s}  \right] &= e^{\frac{1}{2}(t-s)}\\
	&= \mathbb{E}\left[ e^{X_t-X_s} \right]  .
\end{align*}
The result follows from hint (b). Therefore, $ X_t$ is a BM.
\end{problem}
\end{document}
