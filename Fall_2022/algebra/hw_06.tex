\documentclass[12pt]{article}
\newcommand{\alert}[1]{{\bf \color{red} [Alert:] #1}}
\newcommand{\todo}[1]{{\bf \color{orange} [TODO:] #1}}
\newcommand{\real}[1][]{\mathbb{R}^{#1}}
\newcommand{\myeqn}[1]{(\ref{#1})}
\newcommand{\myex}[1]{Example \ref{#1}}
\newcommand{\defeq}{\stackrel{\mathrm{def}}{=}}
\newcommand{\parder}[2]{\frac{\partial #1}{\partial #2}}
\newcommand{\Lie}[3][]{\mathsf{L}_{#3}^{#1} #2}
\newcommand{\LieA}[1]{\mathsf{Lie}(#1)}
\newcommand{\lieder}[2]{\mathcal{L}_{#2} #1}
\renewcommand{\t}{^{\mbox{\tiny\sf T}}}
\newcommand{\trans}{^{\mbox{\tiny\sf T}}}
\newcommand{\markup}[1]{\{\textbf{#1}\}}
\newcommand{\msub}[1]{_\mathrm{#1}}
\newcommand{\msup}[1]{^\mathrm{#1}}
\newcommand{\inv}[1]{#1^{-1}}
\newcommand{\pinv}[1]{{#1}^{+}}
\newcommand{\myfracA}[2]{\displaystyle{\frac{#1}{#2}}}
\newcommand{\myfracB}[2]{{#1}/{#2}}
\newcommand{\mydiffA}[1]{\dot{#1}}
\newcommand{\mydiffB}[2]{\myfracA{\mathrm{d}{#1}}{\mathrm{d}{#2}}}
\newcommand{\ball}[2]{\mathcal{B}_{#1}\left(#2\right)}
\newcommand{\acos}[1]{\cos^{-1}\left(#1\right)}
\newcommand{\asin}[1]{\sin^{-1}\left(#1\right)}
\newcommand{\mani}{\mathcal{M}}
\newcommand{\tang}[2]{\mathsf{T}_{#1} #2}
\newcommand{\LieB}[2]{[ #1, #2 ]}
\newcommand{\LieBad}[3][]{\mathsf{ad}_{#2}^{#1} #3}
\newcommand{\ReachVT}{\mathcal{R}^V_T}
\newcommand{\ReachVt}{\mathcal{R}^V_t}
\newcommand{\ReachVTe}{\mathcal{R}^V_{\le T}}
\newcommand{\ReachT}{\mathcal{R}_T}
\newcommand{\Reacht}{\mathcal{R}_t}
\newcommand{\ReachTe}{\mathcal{R}_{\le T}}
\newcommand{\accLA}[1]{\mathsf{Lie}(#1)}
\newcommand{\accD}{\Delta_{\mathcal{F}}}
\newcommand{\accSA}{\mathsf{Lie}(\mathcal{G},f)}
\newcommand{\accDS}{\Delta_{\mathcal{G}}}
\newcommand{\eval}[3]{\mathsf{Ev}^{#2}_{#1}\left( #3 \right)}
\newcommand{\stlc}{\textsc{stlc}}
\newcommand{\clf}{\textsc{clf}}
\newcommand{\jqlf}{\textsc{jqlf}}
\newcommand{\dlas}{\textsc{dlas}}
\newcommand{\Ad}[2]{\mathsf{Ad}_{#1} #2}
\newcommand{\xe}{\ensuremath{x_e}}
\newcommand{\lebg}[1]{\mathcal{L}_{#1}}
\newcommand{\lebgx}[1]{\mathcal{L}_{#1 \mathrm{e}}}
\newcommand{\dom}{D}
\newcommand{\domT}{[t_0,\infty) \times D}
\newcommand{\rarrow}{\rightarrow}
\renewcommand{\d}{\mathrm{d}}
\renewcommand{\Re}{\mathbb{R}}
\newcommand{\C}{\mathrm{C}}

\newcommand{\QED}{{\unskip\nobreak\hfil\penalty50\hskip2em\vadjust{}
		\nobreak\hfil$\Box$\parfillskip=0pt\finalhyphendemerits=0\par}\vspace{0.1cm}}
\newcommand{\eoEx}{{\unskip\nobreak\hfil\penalty50\hskip0em\vadjust{}
		\nobreak\hfil$\Large\Diamond$\parfillskip=0pt\finalhyphendemerits=0\par}\vspace{0.1cm}}

\newcommand{\sgn}{\ensuremath{\operatorname{sgn}}}
\newcommand{\sat}{\ensuremath{\operatorname{sat}}}

\newcommand{\half}{\frac{1}{2}}
\newcommand{\shalf}{\mbox{$\frac{1}{2}$}}
\newcommand{\marcom}[1]{\marginpar{\footnotesize #1}}
\newcommand{\der}{\mathrm{D}}
\newcommand{\e}{\mathrm{e}}
\newcommand{\dt}{\mathrm{d}t}

\newcommand{\cA}{\ensuremath{\mathcal{A}}}
\newcommand{\cB}{\ensuremath{\mathcal{B}}}
\newcommand{\cG}{\ensuremath{\mathcal{G}}}
\newcommand{\cK}{\ensuremath{\mathcal{K}}}
\newcommand{\cW}{\ensuremath{\mathcal{W}}}
\newcommand{\cZ}{\ensuremath{\mathcal{Z}}}
\newcommand{\cS}{\ensuremath{\mathcal{S}}}
\newcommand{\cD}{\ensuremath{\mathcal{D}}}
\newcommand{\cP}{\ensuremath{\mathcal{P}}}
\newcommand{\cV}{\ensuremath{\mathcal{V}}}
\newcommand{\cL}{\ensuremath{\mathcal{L}}}
\newcommand{\cN}{\ensuremath{\mathcal{N}}}
\newcommand{\cI}{\ensuremath{\mathcal{I}}}
\newcommand{\cR}{\ensuremath{\mathcal{R}}}
\newcommand{\cM}{\ensuremath{\mathcal{M}}}
\newcommand{\cC}{\ensuremath{\mathcal{C}}}
\newcommand{\cF}{\ensuremath{\mathcal{F}}}
\newcommand{\cH}{\ensuremath{\mathcal{H}}}
\newcommand{\cO}{\ensuremath{\mathcal{O}}}
\newcommand{\cX}{\ensuremath{\mathcal{X}}}
\newcommand{\cY}{\ensuremath{\mathcal{Y}}}
\newcommand{\Ci}{\ensuremath{\mathcal{C}^\infty}}
\newcommand{\ISS}{\textsc{iss}}
\newcommand{\LISS}{\textsc{liss}}
\newcommand{\GAS}{\textsc{gas}}
\newcommand{\GS}{\textsc{gs}}
\newcommand{\LES}{\textsc{les}}
\newcommand{\GUAS}{\textsc{guas}}
\newcommand{\BIBO}{\textsc{bibo}}
\newcommand{\spec}{\ensuremath{\operatorname{spec}}}
\newcommand{\spn}{\ensuremath{\operatorname{span}}}
\renewcommand{\i}{\mathrm{i\,}}

\renewcommand{\implies}{\Rightarrow}

\renewcommand{\theenumi}{$\roman{enumi})$}
\renewcommand{\labelenumi}{\theenumi}

\font\ptmten=zptmcmrm scaled 1200
\newcommand{\w}{\mbox{{\ptmten w}}}
\newcommand{\z}{\mbox{{\ptmten z}}}
\renewcommand{\Re}{\mathbb{R}}

\newcommand{\cl}{\operatorname{cl}}
\newcommand{\intr}{\operatorname{int}}
\newcommand{\rank}{\operatorname{rank}}
\newcommand{\co}{\operatorname{co}}
\newcommand{\aff}{\operatorname{aff}}

\theoremstyle{plain}
\newtheorem{theorem}{Theorem}[chapter]
\newtheorem{claim}[theorem]{Claim}
\newtheorem{corollary}[theorem]{Corollary}
\newtheorem{prop}[theorem]{Proposition}
\newtheorem{fact}[theorem]{Fact}
\newtheorem{lemma}[theorem]{Lemma}

\newtheorem{remark}{Remark}[chapter]

\theoremstyle{definition}
\newtheorem{assume}[theorem]{Assumption}
\newtheorem{defn}[theorem]{Definition}
\newtheorem{problem}[theorem]{Problem}
\newtheorem{exercise}{Exercise}
\newtheorem{example}[theorem]{Example}


\begin{document}
\centerline {\textsf{\textbf{\LARGE{Homework 6}}}}
\centerline {Jaden Wang}
\vspace{.15in}

\begin{problem}[1]
~\begin{enumerate}[label=(\arabic*)]
	\item Suppose $ R$ is a local ring with the unique maximal ideal  $ M$. Given  $ r \in R - M$, we see that $ \langle r \rangle$ cannot be a proper ideal, otherwise $ \langle r \rangle \leq M$ since all proper ideals must be contained in this unique maximal ideal, contradicting that $ r \not\in M$. This forces $ \langle r \rangle = \langle 1 \rangle =R$. Hence $ r$ is a unit.
	\item Let  $ M$ be the set of non-units of  $ R$ where $ M$ is an ideal. Suppose  $ M \leq J \leq R$. If there exists an  $ r \in J -M$, that means that $ r$ is a unit. Then  $R= \langle r \rangle \leq J$ so $ J=R$. Hence $ M$ is maximal. Uniqueness follows from the fact that any ideal not contained in $ M$ must have an element not in $ M$ so it cannot be proper (by argument above).
	\item We wish to show that $ \langle 2 \rangle$ is precisely the set of non-units $ M$ in $ R$. Given $ 2 \frac{p}{q} \in \langle 2 \rangle$ where $ \frac{p}{q}$ is a reduced rational number with $ q$ odd, then  $ \frac{2p}{q} r = 1$ yields that  $ r= \frac{q}{2p}$ which clearly has even denominator. So $ 2 \frac{p}{q}$ cannot be a unit so $ \langle 2 \rangle \leq M$. Given any non-unit (reduced) $ \frac{m}{n} \in M$, it must be that $ m = 2k$ for some  $ k \in \zz$, otherwise if $ m$ is odd then $ \frac{n}{m} \in R$ is the inverse. So $ \frac{m}{n} \in \langle 2 \rangle$. Thus $ \langle 2 \rangle = M$. Then by part b, we obtain that $ R$ is a local ring with $ M$  as its maximal ideal.
\end{enumerate}
\end{problem}

\begin{problem}[2]
~\begin{enumerate}[label=(\alph*)]
	\item Suppose $ x^{m} = 0$ for some $ m \in \zz^{+}$ and that $ yr =1$ for some $ r \in R$. Then it suffices to show that $ -(-y)^{m} \in \langle x+y \rangle$ since $ -(-y)^{m}$ is a unit (with inverse $ -(-r)^{m})$. Recall that $ x^{m} - (-y)^{m} = (x-(-y))A$ for some polynomial $ A$ in $ x,y$. Then $ A \in R$ so $ x^{m} - (-y)^{m} = -(-y)^{m} \in \langle 1 \rangle$.
\item First $ N(R)$ is not empty since $0 $ is nilpotent. Given $ a,b \in N(R)$, where $ a^{m} = b^{n} = 0$. Then $ (a+b)^{mn+m+n} = 0$ by binomial theorem so it's closed under addition. It is clearly closed under negation. Given $ r \in R$, we see that $ (ar)^{m} = a^{m} r^{m} = 0 \cdot r^{m}=0$ so $ N(R)$ is an ideal.

	Consider $ R = M_2(\rr)$. Clearly $ \begin{pmatrix} 0&1\\0&0 \end{pmatrix} $ and $ \begin{pmatrix} 0&0\\1&0 \end{pmatrix} $ are nilpotent but their sum is $ \begin{pmatrix} 0&1\\1&0 \end{pmatrix} $ which is full rank and invertible. So $ N(R)$ is not closed under addition so not an ideal.
\item Suppose there exists $ rN(R) \in R /N(R)$ s.t.\ it is nilpotent,  \emph{i.e.} $ r^{m} \in N(R)$ for some positive integer $ m$. Then $ r^{m} = a$ for some nilpotent $ a$  s.t.\ $ a^{n}=0$. Hence $ r^{mn} = a^{n} = 0$, so $ r$ is nilpotent and  $ rN(R) = N(R)$. Hence the only nilpotent element of  $ R /N(R)$ is the identity  $ N(R)$.
\item (collab with Ari, Griffin, Will) Suppose $ a \in R$ is not nilpotent and let $ \Sigma$ be the set of all ideals not containing $ a$-positive powers. Given any chain in this set $ I_1 \subseteq I_2 \subseteq \cdots$. Define $ I = \bigcup_{ i} I_i$. I claim that $ I$ is an ideal not containing  $ a$-positive powers. The fact that $ I$ is an ideal is routine. Since none of the  $ I_i$ contains $ a$-positive power, the union doesn't contain  $ a$-positive power either. Thus  $ I$ is clearly an upper bound of the chain. Then by Zorn's lemma, we have a maximal element $ P$ of $ \Sigma$. Note that $ P$ is an ideal not containing $ a$-positive power. Then suppose  $ x,y \not\in P$, since $ P$ is maximal,  $ \langle P,x \rangle$ and $ \langle P,y \rangle$ must contain $ a$-positive powers. That is, there exists  $ a,b,c,d \in R$ s.t.\ 
	\begin{align*}
		ap+bx &= a^{m}\\
		cp'+dy &= a^{n} \\
		\underbrace{ acpp'+bxcp+dyap}_{ \in P} + bdxy &= \underbrace{ a^{m+n} }_{ \not\in P} 
	\end{align*}
	This implies that $ (bd)xy \not\in P$. Since $ P$ is an ideal,  $xy \not\in P $ either. This proves that $ P$ is prime. Since $ P$ doesn't contain  $ a$-positive power it doesn't contain  $ a$. Thus we show that if $ a$ is not nilpotent, then there exists a prime ideal of $ R$ not containing $ a$. Thus the contrapositive is true: if all prime ideals of $ R$ contains some element $ x$, then  $ x$ is nilpotent. That is,  $ \bigcap_{ i} P_i \subseteq N(R) $.

	Given $ x \in N(R)$, then $ x^{m}= x x^{m-1} = 0 \in \bigcap_{ i} P_i$. Since $ P_i$ are prime, either  $ x \in P_i$ or $ x^{m-1} \in P_i \ \forall \ i$. If $ x^{m-1} \in P_i$, we can rewrite it as $ x x^{m-2} \in P_i$ and repeat this process, which terminates because $ m$ is finite. Eventually we must have  $ x \in P_i \ \forall \ i$ so $ N(R) \subseteq \bigcap_{ i} P_i $.

\end{enumerate}
\end{problem}
\begin{problem}[3]
$ (i) \implies (ii):$ Suppose $ R$ has a unique prime ideal. 
By 2(d), $ N(R)$ must be the unique prime ideal $ P$. Moreover, any maximal ideal is a prime ideal, so $ R$ has a unique maximal ideal. By 1(a), we see that every element in $ R - N(R) $ is a unit. Thus any element in $ R$ is either nilpotent or a unit.

$ (ii) \implies (iii):$ Notice the set of non-units in $ R$ is simply $ N(R)$ which is an ideal by 2(b). Then by 1(b), $ N(R)$ is the unique maximal ideal of  $ R$ so $ R /N(R)$ is a field.

 $ (iii) \implies (i):$ Since $ R /N(R)$ is a field,  $ N(R)$ is maximal. Since  $ N(R)$ is in the intersection of all prime ideals, given any prime ideal $ P$, we have $ N(R) \subseteq P$. Since $ N(R)$ is maximal, and prime ideals are proper, $ P = N(R)$. Hence $ N(R)$ is the unique prime ideal of  $ R$.
\end{problem}
\begin{problem}[4]
	(collab with Ari, Will, Griffin): Suppose to the contrary that $ I \subseteq \bigcup_{ i} P_i $ but $ I \not \subseteq P_i \ \forall \ i$. We wish to prove by induction. If $ k=1$, then  $ I \subseteq P_1$ trivially holds. Suppose if $I \subseteq \bigcup_{ i= 1}^{k-1} P_i$, then $ I \subseteq P_i$ for some $ i$ when there are $ k-1$ prime ideals. To prove the inductive step, suppose that $ I \subseteq \bigcup_{ j= 1}^{k} P_j $ but $ I \not \subseteq  \bigcup_{ j \neq i} P_j $ and $ I \not \subseteq P_i$ for all $ 1\leq i \leq k$. This implies that for every $1\leq i \leq k$, there exists an  $ a_i \in I$ s.t.\ $ a_i \not\in \bigcup_{ j\neq i} P_j $ which forces $ a_i \in P_i$. Denote $ \wh{ a}_i = \prod_{j \neq i} a_j$. Notice that $ \wh{ a}_i \not\in P_i$ by construction. Now consider
\begin{align*}
	x &:=\sum_{ i= 1}^{ k} \prod_{ j\neq i} a_j.
\end{align*}
Since $ x \in I$, we have $ x \in P_n$ for some $1\leq n \leq k$. Since $ P_n$ is an ideal, $ r a_n \in P_n \ \forall \ r \in R$, so
\begin{align*}
	x &= \underbrace{ \left(\sum_{i \neq n} \wh{ a}_{i,n} \right) a_{n} }_{ \in P_{n} }  + \underbrace{ \wh{ a}_n }_{ \not\in P_{n}}\\
	&\not\in P_n.
\end{align*}
This is a contradiction. So it must be that $ I \subseteq \bigcup_{ j \neq i} P_j $ or $ I \subseteq P_j$ for some $ j$. If it's the former we are done by inductive hypothesis. If it is the latter we are done immediately. Therefore by induction, for any $ k$, if $I \subseteq \bigcup_{ i= 1}^{k} P_i $, then $ I \subseteq P_i$ for some $ i$.
\end{problem}
\end{document}
