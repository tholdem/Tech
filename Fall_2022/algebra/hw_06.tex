\documentclass[12pt]{article}
%Fall 2022
% Some basic packages
\usepackage{standalone}[subpreambles=true]
\usepackage[utf8]{inputenc}
\usepackage[T1]{fontenc}
\usepackage{textcomp}
\usepackage[english]{babel}
\usepackage{url}
\usepackage{graphicx}
%\usepackage{quiver}
\usepackage{float}
\usepackage{enumitem}
\usepackage{lmodern}
\usepackage{comment}
\usepackage{hyperref}
\usepackage[usenames,svgnames,dvipsnames]{xcolor}
\usepackage[margin=1in]{geometry}
\usepackage{pdfpages}

\pdfminorversion=7

% Don't indent paragraphs, leave some space between them
\usepackage{parskip}

% Hide page number when page is empty
\usepackage{emptypage}
\usepackage{subcaption}
\usepackage{multicol}
\usepackage[b]{esvect}

% Math stuff
\usepackage{amsmath, amsfonts, mathtools, amsthm, amssymb}
\usepackage{bbm}
\usepackage{stmaryrd}
\allowdisplaybreaks

% Fancy script capitals
\usepackage{mathrsfs}
\usepackage{cancel}
% Bold math
\usepackage{bm}
% Some shortcuts
\newcommand{\rr}{\ensuremath{\mathbb{R}}}
\newcommand{\zz}{\ensuremath{\mathbb{Z}}}
\newcommand{\qq}{\ensuremath{\mathbb{Q}}}
\newcommand{\nn}{\ensuremath{\mathbb{N}}}
\newcommand{\ff}{\ensuremath{\mathbb{F}}}
\newcommand{\cc}{\ensuremath{\mathbb{C}}}
\newcommand{\ee}{\ensuremath{\mathbb{E}}}
\newcommand{\hh}{\ensuremath{\mathbb{H}}}
\renewcommand\O{\ensuremath{\emptyset}}
\newcommand{\norm}[1]{{\left\lVert{#1}\right\rVert}}
\newcommand{\dbracket}[1]{{\left\llbracket{#1}\right\rrbracket}}
\newcommand{\ve}[1]{{\bm{#1}}}
\newcommand\allbold[1]{{\boldmath\textbf{#1}}}
\DeclareMathOperator{\lcm}{lcm}
\DeclareMathOperator{\im}{im}
\DeclareMathOperator{\coim}{coim}
\DeclareMathOperator{\dom}{dom}
\DeclareMathOperator{\tr}{tr}
\DeclareMathOperator{\rank}{rank}
\DeclareMathOperator*{\var}{Var}
\DeclareMathOperator*{\ev}{E}
\DeclareMathOperator{\dg}{deg}
\DeclareMathOperator{\aff}{aff}
\DeclareMathOperator{\conv}{conv}
\DeclareMathOperator{\inte}{int}
\DeclareMathOperator*{\argmin}{argmin}
\DeclareMathOperator*{\argmax}{argmax}
\DeclareMathOperator{\graph}{graph}
\DeclareMathOperator{\sgn}{sgn}
\DeclareMathOperator*{\Rep}{Rep}
\DeclareMathOperator{\Proj}{Proj}
\DeclareMathOperator{\mat}{mat}
\DeclareMathOperator{\diag}{diag}
\DeclareMathOperator{\aut}{Aut}
\DeclareMathOperator{\gal}{Gal}
\DeclareMathOperator{\inn}{Inn}
\DeclareMathOperator{\edm}{End}
\DeclareMathOperator{\Hom}{Hom}
\DeclareMathOperator{\ext}{Ext}
\DeclareMathOperator{\tor}{Tor}
\DeclareMathOperator{\Span}{Span}
\DeclareMathOperator{\Stab}{Stab}
\DeclareMathOperator{\cont}{cont}
\DeclareMathOperator{\Ann}{Ann}
\DeclareMathOperator{\Div}{div}
\DeclareMathOperator{\curl}{curl}
\DeclareMathOperator{\nat}{Nat}
\DeclareMathOperator{\gr}{Gr}
\DeclareMathOperator{\vect}{Vect}
\DeclareMathOperator{\id}{id}
\DeclareMathOperator{\Mod}{Mod}
\DeclareMathOperator{\sign}{sign}
\DeclareMathOperator{\Surf}{Surf}
\DeclareMathOperator{\fcone}{fcone}
\DeclareMathOperator{\Rot}{Rot}
\DeclareMathOperator{\grad}{grad}
\DeclareMathOperator{\atan2}{atan2}
\DeclareMathOperator{\Ric}{Ric}
\let\vec\relax
\DeclareMathOperator{\vec}{vec}
\let\Re\relax
\DeclareMathOperator{\Re}{Re}
\let\Im\relax
\DeclareMathOperator{\Im}{Im}
% Put x \to \infty below \lim
\let\svlim\lim\def\lim{\svlim\limits}

%wide hat
\usepackage{scalerel,stackengine}
\stackMath
\newcommand*\wh[1]{%
\savestack{\tmpbox}{\stretchto{%
  \scaleto{%
    \scalerel*[\widthof{\ensuremath{#1}}]{\kern-.6pt\bigwedge\kern-.6pt}%
    {\rule[-\textheight/2]{1ex}{\textheight}}%WIDTH-LIMITED BIG WEDGE
  }{\textheight}% 
}{0.5ex}}%
\stackon[1pt]{#1}{\tmpbox}%
}
\parskip 1ex

%Make implies and impliedby shorter
\let\implies\Rightarrow
\let\impliedby\Leftarrow
\let\iff\Leftrightarrow
\let\epsilon\varepsilon

% Add \contra symbol to denote contradiction
\usepackage{stmaryrd} % for \lightning
\newcommand\contra{\scalebox{1.5}{$\lightning$}}

% \let\phi\varphi

% Command for short corrections
% Usage: 1+1=\correct{3}{2}

\definecolor{correct}{HTML}{009900}
\newcommand\correct[2]{\ensuremath{\:}{\color{red}{#1}}\ensuremath{\to }{\color{correct}{#2}}\ensuremath{\:}}
\newcommand\green[1]{{\color{correct}{#1}}}

% horizontal rule
\newcommand\hr{
    \noindent\rule[0.5ex]{\linewidth}{0.5pt}
}

% hide parts
\newcommand\hide[1]{}

% si unitx
\usepackage{siunitx}
\sisetup{locale = FR}

%allows pmatrix to stretch
\makeatletter
\renewcommand*\env@matrix[1][\arraystretch]{%
  \edef\arraystretch{#1}%
  \hskip -\arraycolsep
  \let\@ifnextchar\new@ifnextchar
  \array{*\c@MaxMatrixCols c}}
\makeatother

\renewcommand{\arraystretch}{0.8}

\renewcommand{\baselinestretch}{1.5}

\usepackage{graphics}
\usepackage{epstopdf}

\RequirePackage{hyperref}
%%
%% Add support for color in order to color the hyperlinks.
%% 
\hypersetup{
  colorlinks = true,
  urlcolor = blue,
  citecolor = blue
}
%%fakesection Links
\hypersetup{
    colorlinks,
    linkcolor={red!50!black},
    citecolor={green!50!black},
    urlcolor={blue!80!black}
}
%customization of cleveref
\RequirePackage[capitalize,nameinlink]{cleveref}[0.19]

% Per SIAM Style Manual, "section" should be lowercase
\crefname{section}{section}{sections}
\crefname{subsection}{subsection}{subsections}
\Crefname{section}{Section}{Sections}
\Crefname{subsection}{Subsection}{Subsections}

% Per SIAM Style Manual, "Figure" should be spelled out in references
\Crefname{figure}{Figure}{Figures}

% Per SIAM Style Manual, don't say equation in front on an equation.
\crefformat{equation}{\textup{#2(#1)#3}}
\crefrangeformat{equation}{\textup{#3(#1)#4--#5(#2)#6}}
\crefmultiformat{equation}{\textup{#2(#1)#3}}{ and \textup{#2(#1)#3}}
{, \textup{#2(#1)#3}}{, and \textup{#2(#1)#3}}
\crefrangemultiformat{equation}{\textup{#3(#1)#4--#5(#2)#6}}%
{ and \textup{#3(#1)#4--#5(#2)#6}}{, \textup{#3(#1)#4--#5(#2)#6}}{, and \textup{#3(#1)#4--#5(#2)#6}}

% But spell it out at the beginning of a sentence.
\Crefformat{equation}{#2Equation~\textup{(#1)}#3}
\Crefrangeformat{equation}{Equations~\textup{#3(#1)#4--#5(#2)#6}}
\Crefmultiformat{equation}{Equations~\textup{#2(#1)#3}}{ and \textup{#2(#1)#3}}
{, \textup{#2(#1)#3}}{, and \textup{#2(#1)#3}}
\Crefrangemultiformat{equation}{Equations~\textup{#3(#1)#4--#5(#2)#6}}%
{ and \textup{#3(#1)#4--#5(#2)#6}}{, \textup{#3(#1)#4--#5(#2)#6}}{, and \textup{#3(#1)#4--#5(#2)#6}}

% Make number non-italic in any environment.
\crefdefaultlabelformat{#2\textup{#1}#3}

% Environments
\makeatother
% For box around Definition, Theorem, \ldots
%%fakesection Theorems
\usepackage{thmtools}
\usepackage[framemethod=TikZ]{mdframed}

\theoremstyle{definition}
\mdfdefinestyle{mdbluebox}{%
	roundcorner = 10pt,
	linewidth=1pt,
	skipabove=12pt,
	innerbottommargin=9pt,
	skipbelow=2pt,
	nobreak=true,
	linecolor=blue,
	backgroundcolor=TealBlue!5,
}
\declaretheoremstyle[
	headfont=\sffamily\bfseries\color{MidnightBlue},
	mdframed={style=mdbluebox},
	headpunct={\\[3pt]},
	postheadspace={0pt}
]{thmbluebox}

\mdfdefinestyle{mdredbox}{%
	linewidth=0.5pt,
	skipabove=12pt,
	frametitleaboveskip=5pt,
	frametitlebelowskip=0pt,
	skipbelow=2pt,
	frametitlefont=\bfseries,
	innertopmargin=4pt,
	innerbottommargin=8pt,
	nobreak=false,
	linecolor=RawSienna,
	backgroundcolor=Salmon!5,
}
\declaretheoremstyle[
	headfont=\bfseries\color{RawSienna},
	mdframed={style=mdredbox},
	headpunct={\\[3pt]},
	postheadspace={0pt},
]{thmredbox}

\declaretheorem[%
style=thmbluebox,name=Theorem,numberwithin=section]{thm}
\declaretheorem[style=thmbluebox,name=Lemma,sibling=thm]{lem}
\declaretheorem[style=thmbluebox,name=Proposition,sibling=thm]{prop}
\declaretheorem[style=thmbluebox,name=Corollary,sibling=thm]{coro}
\declaretheorem[style=thmredbox,name=Example,sibling=thm]{eg}

\mdfdefinestyle{mdgreenbox}{%
	roundcorner = 10pt,
	linewidth=1pt,
	skipabove=12pt,
	innerbottommargin=9pt,
	skipbelow=2pt,
	nobreak=true,
	linecolor=ForestGreen,
	backgroundcolor=ForestGreen!5,
}

\declaretheoremstyle[
	headfont=\bfseries\sffamily\color{ForestGreen!70!black},
	bodyfont=\normalfont,
	spaceabove=2pt,
	spacebelow=1pt,
	mdframed={style=mdgreenbox},
	headpunct={ --- },
]{thmgreenbox}

\declaretheorem[style=thmgreenbox,name=Definition,sibling=thm]{defn}

\mdfdefinestyle{mdgreenboxsq}{%
	linewidth=1pt,
	skipabove=12pt,
	innerbottommargin=9pt,
	skipbelow=2pt,
	nobreak=true,
	linecolor=ForestGreen,
	backgroundcolor=ForestGreen!5,
}
\declaretheoremstyle[
	headfont=\bfseries\sffamily\color{ForestGreen!70!black},
	bodyfont=\normalfont,
	spaceabove=2pt,
	spacebelow=1pt,
	mdframed={style=mdgreenboxsq},
	headpunct={},
]{thmgreenboxsq}
\declaretheoremstyle[
	headfont=\bfseries\sffamily\color{ForestGreen!70!black},
	bodyfont=\normalfont,
	spaceabove=2pt,
	spacebelow=1pt,
	mdframed={style=mdgreenboxsq},
	headpunct={},
]{thmgreenboxsq*}

\mdfdefinestyle{mdblackbox}{%
	skipabove=8pt,
	linewidth=3pt,
	rightline=false,
	leftline=true,
	topline=false,
	bottomline=false,
	linecolor=black,
	backgroundcolor=RedViolet!5!gray!5,
}
\declaretheoremstyle[
	headfont=\bfseries,
	bodyfont=\normalfont\small,
	spaceabove=0pt,
	spacebelow=0pt,
	mdframed={style=mdblackbox}
]{thmblackbox}

\theoremstyle{plain}
\declaretheorem[name=Question,sibling=thm,style=thmblackbox]{ques}
\declaretheorem[name=Remark,sibling=thm,style=thmgreenboxsq]{remark}
\declaretheorem[name=Remark,sibling=thm,style=thmgreenboxsq*]{remark*}
\newtheorem{ass}[thm]{Assumptions}

\theoremstyle{definition}
\newtheorem*{problem}{Problem}
\newtheorem{claim}[thm]{Claim}
\theoremstyle{remark}
\newtheorem*{case}{Case}
\newtheorem*{notation}{Notation}
\newtheorem*{note}{Note}
\newtheorem*{motivation}{Motivation}
\newtheorem*{intuition}{Intuition}
\newtheorem*{conjecture}{Conjecture}

% Make section starts with 1 for report type
%\renewcommand\thesection{\arabic{section}}

% End example and intermezzo environments with a small diamond (just like proof
% environments end with a small square)
\usepackage{etoolbox}
\AtEndEnvironment{vb}{\null\hfill$\diamond$}%
\AtEndEnvironment{intermezzo}{\null\hfill$\diamond$}%
% \AtEndEnvironment{opmerking}{\null\hfill$\diamond$}%

% Fix some spacing
% http://tex.stackexchange.com/questions/22119/how-can-i-change-the-spacing-before-theorems-with-amsthm
\makeatletter
\def\thm@space@setup{%
  \thm@preskip=\parskip \thm@postskip=0pt
}

% Fix some stuff
% %http://tex.stackexchange.com/questions/76273/multiple-pdfs-with-page-group-included-in-a-single-page-warning
\pdfsuppresswarningpagegroup=1


% My name
\author{Jaden Wang}



\begin{document}
\centerline {\textsf{\textbf{\LARGE{Homework 6}}}}
\centerline {Jaden Wang}
\vspace{.15in}

\begin{problem}[1]
~\begin{enumerate}[label=(\arabic*)]
	\item Suppose $ R$ is a local ring with the unique maximal ideal  $ M$. Given  $ r \in R - M$, we see that $ \langle r \rangle$ cannot be a proper ideal, otherwise $ \langle r \rangle \leq M$ since all proper ideals must be contained in this unique maximal ideal, contradicting that $ r \not\in M$. This forces $ \langle r \rangle = \langle 1 \rangle =R$. Hence $ r$ is a unit.
	\item Let  $ M$ be the set of non-units of  $ R$ where $ M$ is an ideal. Suppose  $ M \leq J \leq R$. If there exists an  $ r \in J -M$, that means that $ r$ is a unit. Then  $R= \langle r \rangle \leq J$ so $ J=R$. Hence $ M$ is maximal. Uniqueness follows from the fact that any ideal not contained in $ M$ must have an element not in $ M$ so it cannot be proper (by argument above).
	\item We wish to show that $ \langle 2 \rangle$ is precisely the set of non-units $ M$ in $ R$. Given $ 2 \frac{p}{q} \in \langle 2 \rangle$ where $ \frac{p}{q}$ is a reduced rational number with $ q$ odd, then  $ \frac{2p}{q} r = 1$ yields that  $ r= \frac{q}{2p}$ which clearly has even denominator. So $ 2 \frac{p}{q}$ cannot be a unit so $ \langle 2 \rangle \leq M$. Given any non-unit (reduced) $ \frac{m}{n} \in M$, it must be that $ m = 2k$ for some  $ k \in \zz$, otherwise if $ m$ is odd then $ \frac{n}{m} \in R$ is the inverse. So $ \frac{m}{n} \in \langle 2 \rangle$. Thus $ \langle 2 \rangle = M$. Then by part b, we obtain that $ R$ is a local ring with $ M$  as its maximal ideal.
\end{enumerate}
\end{problem}

\begin{problem}[2]
~\begin{enumerate}[label=(\alph*)]
	\item Suppose $ x^{m} = 0$ for some $ m \in \zz^{+}$ and that $ yr =1$ for some $ r \in R$. Then it suffices to show that $ -(-y)^{m} \in \langle x+y \rangle$ since $ -(-y)^{m}$ is a unit (with inverse $ -(-r)^{m})$. Recall that $ x^{m} - (-y)^{m} = (x-(-y))A$ for some polynomial $ A$ in $ x,y$. Then $ A \in R$ so $ x^{m} - (-y)^{m} = -(-y)^{m} \in \langle 1 \rangle$.
\item First $ N(R)$ is not empty since $0 $ is nilpotent. Given $ a,b \in N(R)$, where $ a^{m} = b^{n} = 0$. Then $ (a+b)^{mn+m+n} = 0$ by binomial theorem so it's closed under addition. It is clearly closed under negation. Given $ r \in R$, we see that $ (ar)^{m} = a^{m} r^{m} = 0 \cdot r^{m}=0$ so $ N(R)$ is an ideal.

	Consider $ R = M_2(\rr)$. Clearly $ \begin{pmatrix} 0&1\\0&0 \end{pmatrix} $ and $ \begin{pmatrix} 0&0\\1&0 \end{pmatrix} $ are nilpotent but their sum is $ \begin{pmatrix} 0&1\\1&0 \end{pmatrix} $ which is full rank and invertible. So $ N(R)$ is not closed under addition so not an ideal.
\item Suppose there exists $ rN(R) \in R /N(R)$ s.t.\ it is nilpotent,  \emph{i.e.} $ r^{m} \in N(R)$ for some positive integer $ m$. Then $ r^{m} = a$ for some nilpotent $ a$  s.t.\ $ a^{n}=0$. Hence $ r^{mn} = a^{n} = 0$, so $ r$ is nilpotent and  $ rN(R) = N(R)$. Hence the only nilpotent element of  $ R /N(R)$ is the identity  $ N(R)$.
\item (collab with Ari, Griffin, Will) Suppose $ a \in R$ is not nilpotent and let $ \Sigma$ be the set of all ideals not containing $ a$-positive powers. Given any chain in this set $ I_1 \subseteq I_2 \subseteq \cdots$. Define $ I = \bigcup_{ i} I_i$. I claim that $ I$ is an ideal not containing  $ a$-positive powers. The fact that $ I$ is an ideal is routine. Since none of the  $ I_i$ contains $ a$-positive power, the union doesn't contain  $ a$-positive power either. Thus  $ I$ is clearly an upper bound of the chain. Then by Zorn's lemma, we have a maximal element $ P$ of $ \Sigma$. Note that $ P$ is an ideal not containing $ a$-positive power. Then suppose  $ x,y \not\in P$, since $ P$ is maximal,  $ \langle P,x \rangle$ and $ \langle P,y \rangle$ must contain $ a$-positive powers. That is, there exists  $ a,b,c,d \in R$ s.t.\ 
	\begin{align*}
		ap+bx &= a^{m}\\
		cp'+dy &= a^{n} \\
		\underbrace{ acpp'+bxcp+dyap}_{ \in P} + bdxy &= \underbrace{ a^{m+n} }_{ \not\in P} 
	\end{align*}
	This implies that $ (bd)xy \not\in P$. Since $ P$ is an ideal,  $xy \not\in P $ either. This proves that $ P$ is prime. Since $ P$ doesn't contain  $ a$-positive power it doesn't contain  $ a$. Thus we show that if $ a$ is not nilpotent, then there exists a prime ideal of $ R$ not containing $ a$. Thus the contrapositive is true: if all prime ideals of $ R$ contains some element $ x$, then  $ x$ is nilpotent. That is,  $ \bigcap_{ i} P_i \subseteq N(R) $.

	Given $ x \in N(R)$, then $ x^{m}= x x^{m-1} = 0 \in \bigcap_{ i} P_i$. Since $ P_i$ are prime, either  $ x \in P_i$ or $ x^{m-1} \in P_i \ \forall \ i$. If $ x^{m-1} \in P_i$, we can rewrite it as $ x x^{m-2} \in P_i$ and repeat this process, which terminates because $ m$ is finite. Eventually we must have  $ x \in P_i \ \forall \ i$ so $ N(R) \subseteq \bigcap_{ i} P_i $.

\end{enumerate}
\end{problem}
\begin{problem}[3]
$ (i) \implies (ii):$ Suppose $ R$ has a unique prime ideal. 
By 2(d), $ N(R)$ must be the unique prime ideal $ P$. Moreover, any maximal ideal is a prime ideal, so $ R$ has a unique maximal ideal. By 1(a), we see that every element in $ R - N(R) $ is a unit. Thus any element in $ R$ is either nilpotent or a unit.

$ (ii) \implies (iii):$ Notice the set of non-units in $ R$ is simply $ N(R)$ which is an ideal by 2(b). Then by 1(b), $ N(R)$ is the unique maximal ideal of  $ R$ so $ R /N(R)$ is a field.

 $ (iii) \implies (i):$ Since $ R /N(R)$ is a field,  $ N(R)$ is maximal. Since  $ N(R)$ is in the intersection of all prime ideals, given any prime ideal $ P$, we have $ N(R) \subseteq P$. Since $ N(R)$ is maximal, and prime ideals are proper, $ P = N(R)$. Hence $ N(R)$ is the unique prime ideal of  $ R$.
\end{problem}
\begin{problem}[4]
	(collab with Ari, Will, Griffin): Suppose to the contrary that $ I \subseteq \bigcup_{ i} P_i $ but $ I \not \subseteq P_i \ \forall \ i$. We wish to prove by induction. If $ k=1$, then  $ I \subseteq P_1$ trivially holds. Suppose if $I \subseteq \bigcup_{ i= 1}^{k-1} P_i$, then $ I \subseteq P_i$ for some $ i$ when there are $ k-1$ prime ideals. To prove the inductive step, suppose that $ I \subseteq \bigcup_{ j= 1}^{k} P_j $ but $ I \not \subseteq  \bigcup_{ j \neq i} P_j $ and $ I \not \subseteq P_i$ for all $ 1\leq i \leq k$. This implies that for every $1\leq i \leq k$, there exists an  $ a_i \in I$ s.t.\ $ a_i \not\in \bigcup_{ j\neq i} P_j $ which forces $ a_i \in P_i$. Denote $ \wh{ a}_i = \prod_{j \neq i} a_j$. Notice that $ \wh{ a}_i \not\in P_i$ by construction. Now consider
\begin{align*}
	x &:=\sum_{ i= 1}^{ k} \prod_{ j\neq i} a_j.
\end{align*}
Since $ x \in I$, we have $ x \in P_n$ for some $1\leq n \leq k$. Since $ P_n$ is an ideal, $ r a_n \in P_n \ \forall \ r \in R$, so
\begin{align*}
	x &= \underbrace{ \left(\sum_{i \neq n} \wh{ a}_{i,n} \right) a_{n} }_{ \in P_{n} }  + \underbrace{ \wh{ a}_n }_{ \not\in P_{n}}\\
	&\not\in P_n.
\end{align*}
This is a contradiction. So it must be that $ I \subseteq \bigcup_{ j \neq i} P_j $ or $ I \subseteq P_j$ for some $ j$. If it's the former we are done by inductive hypothesis. If it is the latter we are done immediately. Therefore by induction, for any $ k$, if $I \subseteq \bigcup_{ i= 1}^{k} P_i $, then $ I \subseteq P_i$ for some $ i$.
\end{problem}
\end{document}
