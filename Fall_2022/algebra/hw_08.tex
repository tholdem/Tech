\documentclass[12pt]{article}
\newcommand{\alert}[1]{{\bf \color{red} [Alert:] #1}}
\newcommand{\todo}[1]{{\bf \color{orange} [TODO:] #1}}
\newcommand{\real}[1][]{\mathbb{R}^{#1}}
\newcommand{\myeqn}[1]{(\ref{#1})}
\newcommand{\myex}[1]{Example \ref{#1}}
\newcommand{\defeq}{\stackrel{\mathrm{def}}{=}}
\newcommand{\parder}[2]{\frac{\partial #1}{\partial #2}}
\newcommand{\Lie}[3][]{\mathsf{L}_{#3}^{#1} #2}
\newcommand{\LieA}[1]{\mathsf{Lie}(#1)}
\newcommand{\lieder}[2]{\mathcal{L}_{#2} #1}
\renewcommand{\t}{^{\mbox{\tiny\sf T}}}
\newcommand{\trans}{^{\mbox{\tiny\sf T}}}
\newcommand{\markup}[1]{\{\textbf{#1}\}}
\newcommand{\msub}[1]{_\mathrm{#1}}
\newcommand{\msup}[1]{^\mathrm{#1}}
\newcommand{\inv}[1]{#1^{-1}}
\newcommand{\pinv}[1]{{#1}^{+}}
\newcommand{\myfracA}[2]{\displaystyle{\frac{#1}{#2}}}
\newcommand{\myfracB}[2]{{#1}/{#2}}
\newcommand{\mydiffA}[1]{\dot{#1}}
\newcommand{\mydiffB}[2]{\myfracA{\mathrm{d}{#1}}{\mathrm{d}{#2}}}
\newcommand{\ball}[2]{\mathcal{B}_{#1}\left(#2\right)}
\newcommand{\acos}[1]{\cos^{-1}\left(#1\right)}
\newcommand{\asin}[1]{\sin^{-1}\left(#1\right)}
\newcommand{\mani}{\mathcal{M}}
\newcommand{\tang}[2]{\mathsf{T}_{#1} #2}
\newcommand{\LieB}[2]{[ #1, #2 ]}
\newcommand{\LieBad}[3][]{\mathsf{ad}_{#2}^{#1} #3}
\newcommand{\ReachVT}{\mathcal{R}^V_T}
\newcommand{\ReachVt}{\mathcal{R}^V_t}
\newcommand{\ReachVTe}{\mathcal{R}^V_{\le T}}
\newcommand{\ReachT}{\mathcal{R}_T}
\newcommand{\Reacht}{\mathcal{R}_t}
\newcommand{\ReachTe}{\mathcal{R}_{\le T}}
\newcommand{\accLA}[1]{\mathsf{Lie}(#1)}
\newcommand{\accD}{\Delta_{\mathcal{F}}}
\newcommand{\accSA}{\mathsf{Lie}(\mathcal{G},f)}
\newcommand{\accDS}{\Delta_{\mathcal{G}}}
\newcommand{\eval}[3]{\mathsf{Ev}^{#2}_{#1}\left( #3 \right)}
\newcommand{\stlc}{\textsc{stlc}}
\newcommand{\clf}{\textsc{clf}}
\newcommand{\jqlf}{\textsc{jqlf}}
\newcommand{\dlas}{\textsc{dlas}}
\newcommand{\Ad}[2]{\mathsf{Ad}_{#1} #2}
\newcommand{\xe}{\ensuremath{x_e}}
\newcommand{\lebg}[1]{\mathcal{L}_{#1}}
\newcommand{\lebgx}[1]{\mathcal{L}_{#1 \mathrm{e}}}
\newcommand{\dom}{D}
\newcommand{\domT}{[t_0,\infty) \times D}
\newcommand{\rarrow}{\rightarrow}
\renewcommand{\d}{\mathrm{d}}
\renewcommand{\Re}{\mathbb{R}}
\newcommand{\C}{\mathrm{C}}

\newcommand{\QED}{{\unskip\nobreak\hfil\penalty50\hskip2em\vadjust{}
		\nobreak\hfil$\Box$\parfillskip=0pt\finalhyphendemerits=0\par}\vspace{0.1cm}}
\newcommand{\eoEx}{{\unskip\nobreak\hfil\penalty50\hskip0em\vadjust{}
		\nobreak\hfil$\Large\Diamond$\parfillskip=0pt\finalhyphendemerits=0\par}\vspace{0.1cm}}

\newcommand{\sgn}{\ensuremath{\operatorname{sgn}}}
\newcommand{\sat}{\ensuremath{\operatorname{sat}}}

\newcommand{\half}{\frac{1}{2}}
\newcommand{\shalf}{\mbox{$\frac{1}{2}$}}
\newcommand{\marcom}[1]{\marginpar{\footnotesize #1}}
\newcommand{\der}{\mathrm{D}}
\newcommand{\e}{\mathrm{e}}
\newcommand{\dt}{\mathrm{d}t}

\newcommand{\cA}{\ensuremath{\mathcal{A}}}
\newcommand{\cB}{\ensuremath{\mathcal{B}}}
\newcommand{\cG}{\ensuremath{\mathcal{G}}}
\newcommand{\cK}{\ensuremath{\mathcal{K}}}
\newcommand{\cW}{\ensuremath{\mathcal{W}}}
\newcommand{\cZ}{\ensuremath{\mathcal{Z}}}
\newcommand{\cS}{\ensuremath{\mathcal{S}}}
\newcommand{\cD}{\ensuremath{\mathcal{D}}}
\newcommand{\cP}{\ensuremath{\mathcal{P}}}
\newcommand{\cV}{\ensuremath{\mathcal{V}}}
\newcommand{\cL}{\ensuremath{\mathcal{L}}}
\newcommand{\cN}{\ensuremath{\mathcal{N}}}
\newcommand{\cI}{\ensuremath{\mathcal{I}}}
\newcommand{\cR}{\ensuremath{\mathcal{R}}}
\newcommand{\cM}{\ensuremath{\mathcal{M}}}
\newcommand{\cC}{\ensuremath{\mathcal{C}}}
\newcommand{\cF}{\ensuremath{\mathcal{F}}}
\newcommand{\cH}{\ensuremath{\mathcal{H}}}
\newcommand{\cO}{\ensuremath{\mathcal{O}}}
\newcommand{\cX}{\ensuremath{\mathcal{X}}}
\newcommand{\cY}{\ensuremath{\mathcal{Y}}}
\newcommand{\Ci}{\ensuremath{\mathcal{C}^\infty}}
\newcommand{\ISS}{\textsc{iss}}
\newcommand{\LISS}{\textsc{liss}}
\newcommand{\GAS}{\textsc{gas}}
\newcommand{\GS}{\textsc{gs}}
\newcommand{\LES}{\textsc{les}}
\newcommand{\GUAS}{\textsc{guas}}
\newcommand{\BIBO}{\textsc{bibo}}
\newcommand{\spec}{\ensuremath{\operatorname{spec}}}
\newcommand{\spn}{\ensuremath{\operatorname{span}}}
\renewcommand{\i}{\mathrm{i\,}}

\renewcommand{\implies}{\Rightarrow}

\renewcommand{\theenumi}{$\roman{enumi})$}
\renewcommand{\labelenumi}{\theenumi}

\font\ptmten=zptmcmrm scaled 1200
\newcommand{\w}{\mbox{{\ptmten w}}}
\newcommand{\z}{\mbox{{\ptmten z}}}
\renewcommand{\Re}{\mathbb{R}}

\newcommand{\cl}{\operatorname{cl}}
\newcommand{\intr}{\operatorname{int}}
\newcommand{\rank}{\operatorname{rank}}
\newcommand{\co}{\operatorname{co}}
\newcommand{\aff}{\operatorname{aff}}

\theoremstyle{plain}
\newtheorem{theorem}{Theorem}[chapter]
\newtheorem{claim}[theorem]{Claim}
\newtheorem{corollary}[theorem]{Corollary}
\newtheorem{prop}[theorem]{Proposition}
\newtheorem{fact}[theorem]{Fact}
\newtheorem{lemma}[theorem]{Lemma}

\newtheorem{remark}{Remark}[chapter]

\theoremstyle{definition}
\newtheorem{assume}[theorem]{Assumption}
\newtheorem{defn}[theorem]{Definition}
\newtheorem{problem}[theorem]{Problem}
\newtheorem{exercise}{Exercise}
\newtheorem{example}[theorem]{Example}


\begin{document}
\centerline {\textsf{\textbf{\LARGE{Homework 8}}}}
\centerline {Jaden Wang}
\vspace{.15in}

\begin{problem}[1]
~\begin{enumerate}[label=(\alph*)]
	\item First it's clear that $ \langle a \rangle+ \langle b \rangle = \langle a,b \rangle$. So $ \langle 2, x^3+1 \rangle = \langle 2 \rangle + \langle x^3+1 \rangle$. Then by the third isomorphism theorem,  FIX: Use Proposition 9.2.
\begin{align*}
	\zz [x] / \langle 2, x^3+1 \rangle = \zz[x] / (\langle 2 \rangle+ \langle x^3+1 \rangle) \cong \frac{\zz[x] / \langle 2 \rangle}{ (\langle 2 \rangle+ \langle x^3+1 \rangle) / \langle 2 \rangle} \cong \zz_2[x] / \langle x^3+1 \rangle.
\end{align*}
By the correspondence theorem, there is a bijection between ideals of $ \zz_2[x] / \langle x^3+1 \rangle$ and ideals of $\zz_2[x] $ containing $ \langle x^3+1 \rangle$. Since $ \zz_2$ is a field, $ \zz_2[x]$ is a PID. Suppose $ \langle x^3+1 \rangle \subseteq I$ in $ \zz_2[x]$, then $ I = \langle p(x) \rangle$ and $ x^3+1 = a(x)p(x)$ for some $ a(x) \in \zz_2[x]$. We see that $ x^3+1 = (x+1)(x^2-x+1)$. Since $ 0,1$ is not a root of  $ x^2-x+1$, $ x^2-x+1$ is irreducible in $ \zz_2[x]$. Since $ \zz_2[x]$ is a UFD, this is the unique factorization into irreducibles and there is no more factors. Together with the factors 1 and $ x^3+1$, we obtain that $ p(x) =1, x+1, x^2-x+1,$ or $ x^3+1$. Therefore, $\langle 1 \rangle=\zz_2[x], \langle x+1 \rangle, \langle x^2-x+1 \rangle$, and $ \langle x^3+1 \rangle$ are the only ideals containing $ \langle x^3+1 \rangle$ in $ \zz_2[x]$, and the corresponding ideals modulo $ \langle x^3+1 \rangle$ are the only ideals of $ \zz_2[x] /\langle x^3+1 \rangle \cong \zz[x] / \langle 2, x^3+1 \rangle$.
\item Let $ f(x) = x^3+2x+2$. First let's consider the case when $ n=1$. It is easy to see that  $ \langle 1,f(x) \rangle = \langle 1 \rangle = \zz[x]$ so the quotient is zero which is not a field. Next, I claim that $ n \neq 1$ must be a prime for  $ I$ to be maximal. If  $ n$ is not prime, then  $ \zz_n$ contains zero divisors. Let $ a,b \in \zz_n$ be zero divisors  s.t.\ $ ab=0$. Then $ \overline{a} ,\overline{b}$ are zero divisors of $ \zz_n[x] / \langle f(x) \rangle$ so it cannot be a field (so $ I$ cannot be maximal). Thus by 2a, we only need to check whether $ f(x)$ is irreducible for $ n=2,3,5,7$. Note in $ \zz[x]$, evaluation yields $ f(0) = 2, f(1)=5, f(2)=14$. When  $ n=2$,  $ x^3+2x+2 = x^3$ is clearly reducible. If $ n=3$, 0,1,2 are not roots of $ f(x)$, so $ f(x)$ is irreducible. If  $ n=5$, 1 is a root so  $ f(x)$ is reducible. If  $ n=7$,  2 is a root so $ f(x)$ is reducible.

	In summary,  $ I$ is maximal iff the quotient is a field only when  $ n=3$ for $ 1\leq n \leq 7$.
\end{enumerate}
\end{problem}
\begin{problem}[2]
~\begin{enumerate}[label=(\alph*)]
	\item $ (\implies):$ If $ K[x] / \langle f(x) \rangle$ is a field, then $ \langle f(x) \rangle$ is a maximal ideal. It follows that if $ \langle f(x) \rangle \leq \langle p(x) \rangle \leq \langle 1 \rangle = F[x]$, then $ \langle p(x) \rangle = \langle f(x) \rangle$ or $ \langle p(x) \rangle = \langle 1 \rangle$. Either way, if $ f(x)=a(x)p(x)$ then $ a(x)$ or  $ p(x)$ is a unit, showing that  $ f(x)$ is irreducible.

		$ (\impliedby):$ if $ f(x)$ is irreducible, for any  $ p(x)$  s.t.\ $ \langle f(x) \rangle \leq \langle p(x) \rangle \leq \langle 1 \rangle$, \emph{i.e.} $ f(x) = a(x)p(x)$, either  $ a(x) =u$ or  $ p(x) = u$ where  $ u$ is a unit, then  $ \langle p(x) \rangle = \langle f(x) \rangle$ or $ \langle p(x) \rangle = \langle 1 \rangle$ respectively. Thus $ \langle f(x) \rangle$ is maximal and $ F[x] / \langle f(x) \rangle$ is a field.
	\item By 2a we know $ K[x] / \langle f(x) \rangle$ is a field. Since $ K$ is a field,  $ K[x]$ is a Euclidean domain, thus by division algorithm the elements in the quotient all have degree less than $ n$ and has the form $ a_{n-1} x^{n-1} + \cdots +a_0$ where $ a_i \in K$. I claim that all values of $ a_i$ are achieved since we can just multiply the reduced polynomial with $ f(x)$ to get a polynomial in  $ K[x]$ that reduces to this polynomial in the quotient. There are $ n$ number of coefficients, and each coefficient has  $|K|= p$ possible values so there are  $ p^{n}$ possible combinations of coefficients and thus $ p^{n}$ distinct elements in the quotient.
\end{enumerate}
\end{problem}
\begin{problem}[3]
	Since $ \qq$ is a field, $ \qq[x]$ is clearly an integral domain so $ R \subseteq \qq[x]$ is also an integral domain. Since $ x$ is not a unit in  $ \qq[x]$, it is also not a unit in the subset. Suppose  $ x = p_1^{k_1}(x) \cdots p_n^{k_n}(x)$. By degree consideration, exactly one $ p_i^{k_i}(x)$ has degree 1 and the other factors must all be constants. This forces $ p_i^{k_i} = ax, a \in \qq \setminus \{0\} $. However, since we can always factor $ ax = \frac{a}{b}x \cdot b$ for some $ b \in \zz\setminus \{0\} $, $ ax$ is not an irreducible. This implies that $ x$ cannot be written as a product of irreducibles. Thus $ R$ is not a UFD.
\end{problem}

\begin{problem}[4]
	(collab with Daniel): Let $f(x) = \frac{a_n}{ b_n}x^{n}+ \cdots + \frac{a_0}{ b_0},g(x)=\frac{c_m}{ d_m} x^{m} + \cdots + \frac{c_0}{ d_0} \in \qq[x] $ s.t.\ $ f(x)g(x) \in \zz[x]$. Recall that a content $ \cont( f) $ of $ f(x)$ is a gcd of numerators of coefficients dividing a lcm of denominators of coefficients. Since $ fg \in \zz[x]$, $ \cont( fg) \in \zz$ so we can WLOG assume $ fg$ is primitive,  \emph{i.e.} $ \langle \cont( fg)  \rangle = \langle 1 \rangle$ (if the statement is true for primitive $ fg$, it is clearly true for general $ fg$ since we just multiply by integers). Since $ \zz$ is a UFD, by Gauss's lemma,
	\begin{align*}
		\langle 1 \rangle = \langle \cont( fg)  \rangle &= \langle \cont( f)  \rangle \langle \cont( g)  \rangle \\
		&=\left\langle \frac{\gcd ( a_0,\ldots,a_n) \cdot \gcd ( c_0,\ldots,c_m) }{ \lcm \left( b_0,\ldots,b_n \right) \cdot \lcm \left( d_0,\ldots,d_m \right) } \right\rangle =: \left \langle  \frac{p}{ q} \right\rangle 
	\end{align*}
	This forces $ \frac{p}{q}$ to be a unit in $ \zz[x]$, \emph{i.e.} $ \frac{p}{q} = \pm 1$. This implies that $ q|p$. Given any product $ \frac{a_i c_j}{ b_i d_j}$ of coefficients of $ f$ with that of  $ g$, by the definition of gcd and lcm, $ p$ divides the numerator whereas $ b_i d_j$ divides $ q$. Since $ q|p$, we have  $ b_i d_j | q|p| a_i c_j$, and therefore $ \frac{a_i c_j}{ b_i d_j} \in \zz$.
\end{problem}
\begin{problem}[5]
	Since $ \zz[i]$ is a Euclidean domain, it is also a UFD so the irreducibles are also primes. By Proposition 8.18,
\begin{case}[1]
	If $ p = 3 \bmod 4$, then primes $ p \in \zz$ are also primes in $ \zz[i]$. Thus by Eisenstein, $ p|p$ but  $ p^2 \not | p$ so $ x^{n}-p$ is irreducible over $ \zz[i]$.
\end{case}
\begin{case}[2]
	If $ p=1 \bmod 4 = a^2+b^2 = (a+bi)(a-bi)$, then $ (a+bi)$ is irreducible and thus prime in  $ \zz[i]$. By Eisenstein, $ (a+bi) |p$ but  $ (a+bi)^2 \not | p$ so $ x^{n}-p$ is irreducible over $ \zz[i]$.
\end{case}

FIX: remove this case.
\begin{case}[3]
	If $ p=2=(1+i)(1-i)$ (the only even prime), then we see that $(1+i)|2 $ but $ (1+i)^2 \not | 2$ so by Eisenstein $ x^{n}-2$ is irreducible over $ \zz[i]$. 
\end{case}
\end{problem}

\begin{problem}[6]
	Recall that $ \cc[x,y]$ is the same as $ (\cc[x])[y]$. Notice $ x^{m}+1 = 0$ has exactly $ m$ unique roots in the form  $ \zeta^{k}$ where $ \zeta:= e^{i pi /m}$ and $ 0<k<2m$ odd. The irreducibles in $ \cc[x]$ are degree 1 polynomials (as $ \cc$ is algebraically closed so we can always split higher degree polynomials into linear factors) so $ x -\zeta$ is irreducible in $ \cc[x]$ and therefore prime. By Eisenstein, we see that $ (x-\zeta)|x^{m}+1$ and $ (x-\zeta)^2 \not | x^{m}+1$ by uniqueness, thus $ x^{m}+y^{m}+1$ is irreducible over $ \cc[x]$ and therefore irreducible in $ \cc[x,y]$.
\end{problem}

\begin{problem}[7]
~\begin{enumerate}[label=(\alph*)]
	\item Consider the module homomorphism $ \phi_n: R \to N, r \mapsto rn$. Then $ \ker \phi_n = \{r \in R: rn=0\} $ is a submodule of $ R$. Notice that $\Ann_{ R}( N) = \bigcap_{ n \in N} \ker \phi_n$ and we know arbitrary intersection of submodules is a submodule as long as it is nonempty, which is true since $ 0 \in \Ann_{ R}( N) $. Since  submodules of $ R$ correspond to ideals of  $ R$,  $ \Ann_{ R}( N) $ is an ideal of $ R$.
	\item Consider the module homomorphism $ \phi_a: M \to M, m \mapsto am$. Then $ \ker \phi_a = \{m \in M: am=0 \} $. Again $ \Ann_{ M}( I) = \bigcap_{ a \in I} \ker \phi_a $ is a submodule of $ M$. It is nonempty since $ 0 \in \Ann_{ M}( I) $.
	\item Given $ n \in N$, let $ I:= \Ann_{ R}( N) = \{r \in R: rn=0 \ \forall \ n \in N\}  $. Then $ \Ann_{M}( I) = \{m \in M: am=0 \ \forall \ a \in I\}$. Since $ an=0 \ \forall \ a \in I$, $  n \in \Ann_{ M}( I)$.

		Let $ N := \langle x \rangle \leq \zz_6[x]=:M$ and $ R:= \zz$, \emph{i.e.} we treat $ \zz_6[x]$ as an abelian group. Then it suffices to annilate the generator $ x$, and it's easy to see that $ \Ann_{ R}( N) = \langle 6 \rangle =:I$. But $ \Ann_{ M}( I) = M \neq N$ since $ 6$ annilates any element of  $ M$.
	\item Given $ a \in I$, let $ N:= \Ann_{ M}( I) = \{m \in M: am=0 \ \forall \ a \in I\}  $. Then $ \Ann_{ R}( N) = \{r \in R: rn=0 \ \forall \ n \in N \}$. Since $ an = 0 \ \forall \ n \in N$, $ a \in \Ann_{ R}( N) $.

		Let $ I := \langle x \rangle \leq \zz_6[x]=:R=M$. Then it is easy to see that $ p(x) \cdot x = 0 \iff p(x)=0$ so $ \Ann_{ M}( I) = 0 =:N$. But $ \Ann_{ R}( N) = \zz_6[x] \neq I$.
\end{enumerate}
\end{problem}

\begin{problem}[8]

	$ (\implies):$ Suppose $ M$ is simple. Given $ m \in M \setminus \{0\} $, we must have $ \langle m \rangle = M$, \emph{i.e.} every element $ m' \in M$ can be expressed as $ rm$ for some  $ r \in R$. Then let $ I:= \ker \phi_m = \{r \in R: rm=0\} $. Define $ \phi: M \to R /I, rm\mapsto r+I$. This is a module homomorphism:
\begin{align*}
	\phi(s(rm) + (r'm)) &= \phi((s r+ r')m) \\
	&= s r+ r' + I \\
	&= (s r+I) + (r'+I) \\
	&= s (r+I) + (r'+I) \\
	&= s \phi(rm) + \phi(r'm) 
\end{align*}
It is clearly surjective. Suppose $ \phi(rm) = r+I= I$, then $ r \in I$. Thus $ rm = 0$ by definition of annilator. It follows that  $ \ker \phi = \{0\} $ and $ \phi$ is injective. Therefore, $ M \cong R /I$. Since $ M$ is simple, by the isomorphism $ R /I$ also has no proper nontrivial submodules and thus has no proper nontrivial ideals. Hence  $ R /I$ is a field (every nonzero element generates $ R /I = \langle 1 \rangle$ and therefore is a unit) so $ I$ is maximal.

$ (\impliedby):$ Suppose $ I$ is maximal and  $ M \cong R /I$ as $ R$ modules. Since  $ R /I$ is a field, it has no proper nontrivial ideals so it has no proper nontrivial submodules. By the isomorphism so is  $ M$. Thus  $ M$ is simple.
\end{problem}
\end{document}
