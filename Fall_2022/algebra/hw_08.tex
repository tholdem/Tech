\documentclass[12pt]{article}
%Fall 2022
% Some basic packages
\usepackage{standalone}[subpreambles=true]
\usepackage[utf8]{inputenc}
\usepackage[T1]{fontenc}
\usepackage{textcomp}
\usepackage[english]{babel}
\usepackage{url}
\usepackage{graphicx}
%\usepackage{quiver}
\usepackage{float}
\usepackage{enumitem}
\usepackage{lmodern}
\usepackage{comment}
\usepackage{hyperref}
\usepackage[usenames,svgnames,dvipsnames]{xcolor}
\usepackage[margin=1in]{geometry}
\usepackage{pdfpages}

\pdfminorversion=7

% Don't indent paragraphs, leave some space between them
\usepackage{parskip}

% Hide page number when page is empty
\usepackage{emptypage}
\usepackage{subcaption}
\usepackage{multicol}
\usepackage[b]{esvect}

% Math stuff
\usepackage{amsmath, amsfonts, mathtools, amsthm, amssymb}
\usepackage{bbm}
\usepackage{stmaryrd}
\allowdisplaybreaks

% Fancy script capitals
\usepackage{mathrsfs}
\usepackage{cancel}
% Bold math
\usepackage{bm}
% Some shortcuts
\newcommand{\rr}{\ensuremath{\mathbb{R}}}
\newcommand{\zz}{\ensuremath{\mathbb{Z}}}
\newcommand{\qq}{\ensuremath{\mathbb{Q}}}
\newcommand{\nn}{\ensuremath{\mathbb{N}}}
\newcommand{\ff}{\ensuremath{\mathbb{F}}}
\newcommand{\cc}{\ensuremath{\mathbb{C}}}
\newcommand{\ee}{\ensuremath{\mathbb{E}}}
\newcommand{\hh}{\ensuremath{\mathbb{H}}}
\renewcommand\O{\ensuremath{\emptyset}}
\newcommand{\norm}[1]{{\left\lVert{#1}\right\rVert}}
\newcommand{\dbracket}[1]{{\left\llbracket{#1}\right\rrbracket}}
\newcommand{\ve}[1]{{\bm{#1}}}
\newcommand\allbold[1]{{\boldmath\textbf{#1}}}
\DeclareMathOperator{\lcm}{lcm}
\DeclareMathOperator{\im}{im}
\DeclareMathOperator{\coim}{coim}
\DeclareMathOperator{\dom}{dom}
\DeclareMathOperator{\tr}{tr}
\DeclareMathOperator{\rank}{rank}
\DeclareMathOperator*{\var}{Var}
\DeclareMathOperator*{\ev}{E}
\DeclareMathOperator{\dg}{deg}
\DeclareMathOperator{\aff}{aff}
\DeclareMathOperator{\conv}{conv}
\DeclareMathOperator{\inte}{int}
\DeclareMathOperator*{\argmin}{argmin}
\DeclareMathOperator*{\argmax}{argmax}
\DeclareMathOperator{\graph}{graph}
\DeclareMathOperator{\sgn}{sgn}
\DeclareMathOperator*{\Rep}{Rep}
\DeclareMathOperator{\Proj}{Proj}
\DeclareMathOperator{\mat}{mat}
\DeclareMathOperator{\diag}{diag}
\DeclareMathOperator{\aut}{Aut}
\DeclareMathOperator{\gal}{Gal}
\DeclareMathOperator{\inn}{Inn}
\DeclareMathOperator{\edm}{End}
\DeclareMathOperator{\Hom}{Hom}
\DeclareMathOperator{\ext}{Ext}
\DeclareMathOperator{\tor}{Tor}
\DeclareMathOperator{\Span}{Span}
\DeclareMathOperator{\Stab}{Stab}
\DeclareMathOperator{\cont}{cont}
\DeclareMathOperator{\Ann}{Ann}
\DeclareMathOperator{\Div}{div}
\DeclareMathOperator{\curl}{curl}
\DeclareMathOperator{\nat}{Nat}
\DeclareMathOperator{\gr}{Gr}
\DeclareMathOperator{\vect}{Vect}
\DeclareMathOperator{\id}{id}
\DeclareMathOperator{\Mod}{Mod}
\DeclareMathOperator{\sign}{sign}
\DeclareMathOperator{\Surf}{Surf}
\DeclareMathOperator{\fcone}{fcone}
\DeclareMathOperator{\Rot}{Rot}
\DeclareMathOperator{\grad}{grad}
\DeclareMathOperator{\atan2}{atan2}
\DeclareMathOperator{\Ric}{Ric}
\let\vec\relax
\DeclareMathOperator{\vec}{vec}
\let\Re\relax
\DeclareMathOperator{\Re}{Re}
\let\Im\relax
\DeclareMathOperator{\Im}{Im}
% Put x \to \infty below \lim
\let\svlim\lim\def\lim{\svlim\limits}

%wide hat
\usepackage{scalerel,stackengine}
\stackMath
\newcommand*\wh[1]{%
\savestack{\tmpbox}{\stretchto{%
  \scaleto{%
    \scalerel*[\widthof{\ensuremath{#1}}]{\kern-.6pt\bigwedge\kern-.6pt}%
    {\rule[-\textheight/2]{1ex}{\textheight}}%WIDTH-LIMITED BIG WEDGE
  }{\textheight}% 
}{0.5ex}}%
\stackon[1pt]{#1}{\tmpbox}%
}
\parskip 1ex

%Make implies and impliedby shorter
\let\implies\Rightarrow
\let\impliedby\Leftarrow
\let\iff\Leftrightarrow
\let\epsilon\varepsilon

% Add \contra symbol to denote contradiction
\usepackage{stmaryrd} % for \lightning
\newcommand\contra{\scalebox{1.5}{$\lightning$}}

% \let\phi\varphi

% Command for short corrections
% Usage: 1+1=\correct{3}{2}

\definecolor{correct}{HTML}{009900}
\newcommand\correct[2]{\ensuremath{\:}{\color{red}{#1}}\ensuremath{\to }{\color{correct}{#2}}\ensuremath{\:}}
\newcommand\green[1]{{\color{correct}{#1}}}

% horizontal rule
\newcommand\hr{
    \noindent\rule[0.5ex]{\linewidth}{0.5pt}
}

% hide parts
\newcommand\hide[1]{}

% si unitx
\usepackage{siunitx}
\sisetup{locale = FR}

%allows pmatrix to stretch
\makeatletter
\renewcommand*\env@matrix[1][\arraystretch]{%
  \edef\arraystretch{#1}%
  \hskip -\arraycolsep
  \let\@ifnextchar\new@ifnextchar
  \array{*\c@MaxMatrixCols c}}
\makeatother

\renewcommand{\arraystretch}{0.8}

\renewcommand{\baselinestretch}{1.5}

\usepackage{graphics}
\usepackage{epstopdf}

\RequirePackage{hyperref}
%%
%% Add support for color in order to color the hyperlinks.
%% 
\hypersetup{
  colorlinks = true,
  urlcolor = blue,
  citecolor = blue
}
%%fakesection Links
\hypersetup{
    colorlinks,
    linkcolor={red!50!black},
    citecolor={green!50!black},
    urlcolor={blue!80!black}
}
%customization of cleveref
\RequirePackage[capitalize,nameinlink]{cleveref}[0.19]

% Per SIAM Style Manual, "section" should be lowercase
\crefname{section}{section}{sections}
\crefname{subsection}{subsection}{subsections}
\Crefname{section}{Section}{Sections}
\Crefname{subsection}{Subsection}{Subsections}

% Per SIAM Style Manual, "Figure" should be spelled out in references
\Crefname{figure}{Figure}{Figures}

% Per SIAM Style Manual, don't say equation in front on an equation.
\crefformat{equation}{\textup{#2(#1)#3}}
\crefrangeformat{equation}{\textup{#3(#1)#4--#5(#2)#6}}
\crefmultiformat{equation}{\textup{#2(#1)#3}}{ and \textup{#2(#1)#3}}
{, \textup{#2(#1)#3}}{, and \textup{#2(#1)#3}}
\crefrangemultiformat{equation}{\textup{#3(#1)#4--#5(#2)#6}}%
{ and \textup{#3(#1)#4--#5(#2)#6}}{, \textup{#3(#1)#4--#5(#2)#6}}{, and \textup{#3(#1)#4--#5(#2)#6}}

% But spell it out at the beginning of a sentence.
\Crefformat{equation}{#2Equation~\textup{(#1)}#3}
\Crefrangeformat{equation}{Equations~\textup{#3(#1)#4--#5(#2)#6}}
\Crefmultiformat{equation}{Equations~\textup{#2(#1)#3}}{ and \textup{#2(#1)#3}}
{, \textup{#2(#1)#3}}{, and \textup{#2(#1)#3}}
\Crefrangemultiformat{equation}{Equations~\textup{#3(#1)#4--#5(#2)#6}}%
{ and \textup{#3(#1)#4--#5(#2)#6}}{, \textup{#3(#1)#4--#5(#2)#6}}{, and \textup{#3(#1)#4--#5(#2)#6}}

% Make number non-italic in any environment.
\crefdefaultlabelformat{#2\textup{#1}#3}

% Environments
\makeatother
% For box around Definition, Theorem, \ldots
%%fakesection Theorems
\usepackage{thmtools}
\usepackage[framemethod=TikZ]{mdframed}

\theoremstyle{definition}
\mdfdefinestyle{mdbluebox}{%
	roundcorner = 10pt,
	linewidth=1pt,
	skipabove=12pt,
	innerbottommargin=9pt,
	skipbelow=2pt,
	nobreak=true,
	linecolor=blue,
	backgroundcolor=TealBlue!5,
}
\declaretheoremstyle[
	headfont=\sffamily\bfseries\color{MidnightBlue},
	mdframed={style=mdbluebox},
	headpunct={\\[3pt]},
	postheadspace={0pt}
]{thmbluebox}

\mdfdefinestyle{mdredbox}{%
	linewidth=0.5pt,
	skipabove=12pt,
	frametitleaboveskip=5pt,
	frametitlebelowskip=0pt,
	skipbelow=2pt,
	frametitlefont=\bfseries,
	innertopmargin=4pt,
	innerbottommargin=8pt,
	nobreak=false,
	linecolor=RawSienna,
	backgroundcolor=Salmon!5,
}
\declaretheoremstyle[
	headfont=\bfseries\color{RawSienna},
	mdframed={style=mdredbox},
	headpunct={\\[3pt]},
	postheadspace={0pt},
]{thmredbox}

\declaretheorem[%
style=thmbluebox,name=Theorem,numberwithin=section]{thm}
\declaretheorem[style=thmbluebox,name=Lemma,sibling=thm]{lem}
\declaretheorem[style=thmbluebox,name=Proposition,sibling=thm]{prop}
\declaretheorem[style=thmbluebox,name=Corollary,sibling=thm]{coro}
\declaretheorem[style=thmredbox,name=Example,sibling=thm]{eg}

\mdfdefinestyle{mdgreenbox}{%
	roundcorner = 10pt,
	linewidth=1pt,
	skipabove=12pt,
	innerbottommargin=9pt,
	skipbelow=2pt,
	nobreak=true,
	linecolor=ForestGreen,
	backgroundcolor=ForestGreen!5,
}

\declaretheoremstyle[
	headfont=\bfseries\sffamily\color{ForestGreen!70!black},
	bodyfont=\normalfont,
	spaceabove=2pt,
	spacebelow=1pt,
	mdframed={style=mdgreenbox},
	headpunct={ --- },
]{thmgreenbox}

\declaretheorem[style=thmgreenbox,name=Definition,sibling=thm]{defn}

\mdfdefinestyle{mdgreenboxsq}{%
	linewidth=1pt,
	skipabove=12pt,
	innerbottommargin=9pt,
	skipbelow=2pt,
	nobreak=true,
	linecolor=ForestGreen,
	backgroundcolor=ForestGreen!5,
}
\declaretheoremstyle[
	headfont=\bfseries\sffamily\color{ForestGreen!70!black},
	bodyfont=\normalfont,
	spaceabove=2pt,
	spacebelow=1pt,
	mdframed={style=mdgreenboxsq},
	headpunct={},
]{thmgreenboxsq}
\declaretheoremstyle[
	headfont=\bfseries\sffamily\color{ForestGreen!70!black},
	bodyfont=\normalfont,
	spaceabove=2pt,
	spacebelow=1pt,
	mdframed={style=mdgreenboxsq},
	headpunct={},
]{thmgreenboxsq*}

\mdfdefinestyle{mdblackbox}{%
	skipabove=8pt,
	linewidth=3pt,
	rightline=false,
	leftline=true,
	topline=false,
	bottomline=false,
	linecolor=black,
	backgroundcolor=RedViolet!5!gray!5,
}
\declaretheoremstyle[
	headfont=\bfseries,
	bodyfont=\normalfont\small,
	spaceabove=0pt,
	spacebelow=0pt,
	mdframed={style=mdblackbox}
]{thmblackbox}

\theoremstyle{plain}
\declaretheorem[name=Question,sibling=thm,style=thmblackbox]{ques}
\declaretheorem[name=Remark,sibling=thm,style=thmgreenboxsq]{remark}
\declaretheorem[name=Remark,sibling=thm,style=thmgreenboxsq*]{remark*}
\newtheorem{ass}[thm]{Assumptions}

\theoremstyle{definition}
\newtheorem*{problem}{Problem}
\newtheorem{claim}[thm]{Claim}
\theoremstyle{remark}
\newtheorem*{case}{Case}
\newtheorem*{notation}{Notation}
\newtheorem*{note}{Note}
\newtheorem*{motivation}{Motivation}
\newtheorem*{intuition}{Intuition}
\newtheorem*{conjecture}{Conjecture}

% Make section starts with 1 for report type
%\renewcommand\thesection{\arabic{section}}

% End example and intermezzo environments with a small diamond (just like proof
% environments end with a small square)
\usepackage{etoolbox}
\AtEndEnvironment{vb}{\null\hfill$\diamond$}%
\AtEndEnvironment{intermezzo}{\null\hfill$\diamond$}%
% \AtEndEnvironment{opmerking}{\null\hfill$\diamond$}%

% Fix some spacing
% http://tex.stackexchange.com/questions/22119/how-can-i-change-the-spacing-before-theorems-with-amsthm
\makeatletter
\def\thm@space@setup{%
  \thm@preskip=\parskip \thm@postskip=0pt
}

% Fix some stuff
% %http://tex.stackexchange.com/questions/76273/multiple-pdfs-with-page-group-included-in-a-single-page-warning
\pdfsuppresswarningpagegroup=1


% My name
\author{Jaden Wang}



\begin{document}
\centerline {\textsf{\textbf{\LARGE{Homework 8}}}}
\centerline {Jaden Wang}
\vspace{.15in}

\begin{problem}[1]
~\begin{enumerate}[label=(\alph*)]
	\item First it's clear that $ \langle a \rangle+ \langle b \rangle = \langle a,b \rangle$. So $ \langle 2, x^3+1 \rangle = \langle 2 \rangle + \langle x^3+1 \rangle$. Then by the third isomorphism theorem,  FIX: Use Proposition 9.2.
\begin{align*}
	\zz [x] / \langle 2, x^3+1 \rangle = \zz[x] / (\langle 2 \rangle+ \langle x^3+1 \rangle) \cong \frac{\zz[x] / \langle 2 \rangle}{ (\langle 2 \rangle+ \langle x^3+1 \rangle) / \langle 2 \rangle} \cong \zz_2[x] / \langle x^3+1 \rangle.
\end{align*}
By the correspondence theorem, there is a bijection between ideals of $ \zz_2[x] / \langle x^3+1 \rangle$ and ideals of $\zz_2[x] $ containing $ \langle x^3+1 \rangle$. Since $ \zz_2$ is a field, $ \zz_2[x]$ is a PID. Suppose $ \langle x^3+1 \rangle \subseteq I$ in $ \zz_2[x]$, then $ I = \langle p(x) \rangle$ and $ x^3+1 = a(x)p(x)$ for some $ a(x) \in \zz_2[x]$. We see that $ x^3+1 = (x+1)(x^2-x+1)$. Since $ 0,1$ is not a root of  $ x^2-x+1$, $ x^2-x+1$ is irreducible in $ \zz_2[x]$. Since $ \zz_2[x]$ is a UFD, this is the unique factorization into irreducibles and there is no more factors. Together with the factors 1 and $ x^3+1$, we obtain that $ p(x) =1, x+1, x^2-x+1,$ or $ x^3+1$. Therefore, $\langle 1 \rangle=\zz_2[x], \langle x+1 \rangle, \langle x^2-x+1 \rangle$, and $ \langle x^3+1 \rangle$ are the only ideals containing $ \langle x^3+1 \rangle$ in $ \zz_2[x]$, and the corresponding ideals modulo $ \langle x^3+1 \rangle$ are the only ideals of $ \zz_2[x] /\langle x^3+1 \rangle \cong \zz[x] / \langle 2, x^3+1 \rangle$.
\item Let $ f(x) = x^3+2x+2$. First let's consider the case when $ n=1$. It is easy to see that  $ \langle 1,f(x) \rangle = \langle 1 \rangle = \zz[x]$ so the quotient is zero which is not a field. Next, I claim that $ n \neq 1$ must be a prime for  $ I$ to be maximal. If  $ n$ is not prime, then  $ \zz_n$ contains zero divisors. Let $ a,b \in \zz_n$ be zero divisors  s.t.\ $ ab=0$. Then $ \overline{a} ,\overline{b}$ are zero divisors of $ \zz_n[x] / \langle f(x) \rangle$ so it cannot be a field (so $ I$ cannot be maximal). Thus by 2a, we only need to check whether $ f(x)$ is irreducible for $ n=2,3,5,7$. Note in $ \zz[x]$, evaluation yields $ f(0) = 2, f(1)=5, f(2)=14$. When  $ n=2$,  $ x^3+2x+2 = x^3$ is clearly reducible. If $ n=3$, 0,1,2 are not roots of $ f(x)$, so $ f(x)$ is irreducible. If  $ n=5$, 1 is a root so  $ f(x)$ is reducible. If  $ n=7$,  2 is a root so $ f(x)$ is reducible.

	In summary,  $ I$ is maximal iff the quotient is a field only when  $ n=3$ for $ 1\leq n \leq 7$.
\end{enumerate}
\end{problem}
\begin{problem}[2]
~\begin{enumerate}[label=(\alph*)]
	\item $ (\implies):$ If $ K[x] / \langle f(x) \rangle$ is a field, then $ \langle f(x) \rangle$ is a maximal ideal. It follows that if $ \langle f(x) \rangle \leq \langle p(x) \rangle \leq \langle 1 \rangle = F[x]$, then $ \langle p(x) \rangle = \langle f(x) \rangle$ or $ \langle p(x) \rangle = \langle 1 \rangle$. Either way, if $ f(x)=a(x)p(x)$ then $ a(x)$ or  $ p(x)$ is a unit, showing that  $ f(x)$ is irreducible.

		$ (\impliedby):$ if $ f(x)$ is irreducible, for any  $ p(x)$  s.t.\ $ \langle f(x) \rangle \leq \langle p(x) \rangle \leq \langle 1 \rangle$, \emph{i.e.} $ f(x) = a(x)p(x)$, either  $ a(x) =u$ or  $ p(x) = u$ where  $ u$ is a unit, then  $ \langle p(x) \rangle = \langle f(x) \rangle$ or $ \langle p(x) \rangle = \langle 1 \rangle$ respectively. Thus $ \langle f(x) \rangle$ is maximal and $ F[x] / \langle f(x) \rangle$ is a field.
	\item By 2a we know $ K[x] / \langle f(x) \rangle$ is a field. Since $ K$ is a field,  $ K[x]$ is a Euclidean domain, thus by division algorithm the elements in the quotient all have degree less than $ n$ and has the form $ a_{n-1} x^{n-1} + \cdots +a_0$ where $ a_i \in K$. I claim that all values of $ a_i$ are achieved since we can just multiply the reduced polynomial with $ f(x)$ to get a polynomial in  $ K[x]$ that reduces to this polynomial in the quotient. There are $ n$ number of coefficients, and each coefficient has  $|K|= p$ possible values so there are  $ p^{n}$ possible combinations of coefficients and thus $ p^{n}$ distinct elements in the quotient.
\end{enumerate}
\end{problem}
\begin{problem}[3]
	Since $ \qq$ is a field, $ \qq[x]$ is clearly an integral domain so $ R \subseteq \qq[x]$ is also an integral domain. Since $ x$ is not a unit in  $ \qq[x]$, it is also not a unit in the subset. Suppose  $ x = p_1^{k_1}(x) \cdots p_n^{k_n}(x)$. By degree consideration, exactly one $ p_i^{k_i}(x)$ has degree 1 and the other factors must all be constants. This forces $ p_i^{k_i} = ax, a \in \qq \setminus \{0\} $. However, since we can always factor $ ax = \frac{a}{b}x \cdot b$ for some $ b \in \zz\setminus \{0\} $, $ ax$ is not an irreducible. This implies that $ x$ cannot be written as a product of irreducibles. Thus $ R$ is not a UFD.
\end{problem}

\begin{problem}[4]
	(collab with Daniel): Let $f(x) = \frac{a_n}{ b_n}x^{n}+ \cdots + \frac{a_0}{ b_0},g(x)=\frac{c_m}{ d_m} x^{m} + \cdots + \frac{c_0}{ d_0} \in \qq[x] $ s.t.\ $ f(x)g(x) \in \zz[x]$. Recall that a content $ \cont( f) $ of $ f(x)$ is a gcd of numerators of coefficients dividing a lcm of denominators of coefficients. Since $ fg \in \zz[x]$, $ \cont( fg) \in \zz$ so we can WLOG assume $ fg$ is primitive,  \emph{i.e.} $ \langle \cont( fg)  \rangle = \langle 1 \rangle$ (if the statement is true for primitive $ fg$, it is clearly true for general $ fg$ since we just multiply by integers). Since $ \zz$ is a UFD, by Gauss's lemma,
	\begin{align*}
		\langle 1 \rangle = \langle \cont( fg)  \rangle &= \langle \cont( f)  \rangle \langle \cont( g)  \rangle \\
		&=\left\langle \frac{\gcd ( a_0,\ldots,a_n) \cdot \gcd ( c_0,\ldots,c_m) }{ \lcm \left( b_0,\ldots,b_n \right) \cdot \lcm \left( d_0,\ldots,d_m \right) } \right\rangle =: \left \langle  \frac{p}{ q} \right\rangle 
	\end{align*}
	This forces $ \frac{p}{q}$ to be a unit in $ \zz[x]$, \emph{i.e.} $ \frac{p}{q} = \pm 1$. This implies that $ q|p$. Given any product $ \frac{a_i c_j}{ b_i d_j}$ of coefficients of $ f$ with that of  $ g$, by the definition of gcd and lcm, $ p$ divides the numerator whereas $ b_i d_j$ divides $ q$. Since $ q|p$, we have  $ b_i d_j | q|p| a_i c_j$, and therefore $ \frac{a_i c_j}{ b_i d_j} \in \zz$.
\end{problem}
\begin{problem}[5]
	Since $ \zz[i]$ is a Euclidean domain, it is also a UFD so the irreducibles are also primes. By Proposition 8.18,
\begin{case}[1]
	If $ p = 3 \bmod 4$, then primes $ p \in \zz$ are also primes in $ \zz[i]$. Thus by Eisenstein, $ p|p$ but  $ p^2 \not | p$ so $ x^{n}-p$ is irreducible over $ \zz[i]$.
\end{case}
\begin{case}[2]
	If $ p=1 \bmod 4 = a^2+b^2 = (a+bi)(a-bi)$, then $ (a+bi)$ is irreducible and thus prime in  $ \zz[i]$. By Eisenstein, $ (a+bi) |p$ but  $ (a+bi)^2 \not | p$ so $ x^{n}-p$ is irreducible over $ \zz[i]$.
\end{case}

FIX: remove this case.
\begin{case}[3]
	If $ p=2=(1+i)(1-i)$ (the only even prime), then we see that $(1+i)|2 $ but $ (1+i)^2 \not | 2$ so by Eisenstein $ x^{n}-2$ is irreducible over $ \zz[i]$. 
\end{case}
\end{problem}

\begin{problem}[6]
	Recall that $ \cc[x,y]$ is the same as $ (\cc[x])[y]$. Notice $ x^{m}+1 = 0$ has exactly $ m$ unique roots in the form  $ \zeta^{k}$ where $ \zeta:= e^{i pi /m}$ and $ 0<k<2m$ odd. The irreducibles in $ \cc[x]$ are degree 1 polynomials (as $ \cc$ is algebraically closed so we can always split higher degree polynomials into linear factors) so $ x -\zeta$ is irreducible in $ \cc[x]$ and therefore prime. By Eisenstein, we see that $ (x-\zeta)|x^{m}+1$ and $ (x-\zeta)^2 \not | x^{m}+1$ by uniqueness, thus $ x^{m}+y^{m}+1$ is irreducible over $ \cc[x]$ and therefore irreducible in $ \cc[x,y]$.
\end{problem}

\begin{problem}[7]
~\begin{enumerate}[label=(\alph*)]
	\item Consider the module homomorphism $ \phi_n: R \to N, r \mapsto rn$. Then $ \ker \phi_n = \{r \in R: rn=0\} $ is a submodule of $ R$. Notice that $\Ann_{ R}( N) = \bigcap_{ n \in N} \ker \phi_n$ and we know arbitrary intersection of submodules is a submodule as long as it is nonempty, which is true since $ 0 \in \Ann_{ R}( N) $. Since  submodules of $ R$ correspond to ideals of  $ R$,  $ \Ann_{ R}( N) $ is an ideal of $ R$.
	\item Consider the module homomorphism $ \phi_a: M \to M, m \mapsto am$. Then $ \ker \phi_a = \{m \in M: am=0 \} $. Again $ \Ann_{ M}( I) = \bigcap_{ a \in I} \ker \phi_a $ is a submodule of $ M$. It is nonempty since $ 0 \in \Ann_{ M}( I) $.
	\item Given $ n \in N$, let $ I:= \Ann_{ R}( N) = \{r \in R: rn=0 \ \forall \ n \in N\}  $. Then $ \Ann_{M}( I) = \{m \in M: am=0 \ \forall \ a \in I\}$. Since $ an=0 \ \forall \ a \in I$, $  n \in \Ann_{ M}( I)$.

		Let $ N := \langle x \rangle \leq \zz_6[x]=:M$ and $ R:= \zz$, \emph{i.e.} we treat $ \zz_6[x]$ as an abelian group. Then it suffices to annilate the generator $ x$, and it's easy to see that $ \Ann_{ R}( N) = \langle 6 \rangle =:I$. But $ \Ann_{ M}( I) = M \neq N$ since $ 6$ annilates any element of  $ M$.
	\item Given $ a \in I$, let $ N:= \Ann_{ M}( I) = \{m \in M: am=0 \ \forall \ a \in I\}  $. Then $ \Ann_{ R}( N) = \{r \in R: rn=0 \ \forall \ n \in N \}$. Since $ an = 0 \ \forall \ n \in N$, $ a \in \Ann_{ R}( N) $.

		Let $ I := \langle x \rangle \leq \zz_6[x]=:R=M$. Then it is easy to see that $ p(x) \cdot x = 0 \iff p(x)=0$ so $ \Ann_{ M}( I) = 0 =:N$. But $ \Ann_{ R}( N) = \zz_6[x] \neq I$.
\end{enumerate}
\end{problem}

\begin{problem}[8]

	$ (\implies):$ Suppose $ M$ is simple. Given $ m \in M \setminus \{0\} $, we must have $ \langle m \rangle = M$, \emph{i.e.} every element $ m' \in M$ can be expressed as $ rm$ for some  $ r \in R$. Then let $ I:= \ker \phi_m = \{r \in R: rm=0\} $. Define $ \phi: M \to R /I, rm\mapsto r+I$. This is a module homomorphism:
\begin{align*}
	\phi(s(rm) + (r'm)) &= \phi((s r+ r')m) \\
	&= s r+ r' + I \\
	&= (s r+I) + (r'+I) \\
	&= s (r+I) + (r'+I) \\
	&= s \phi(rm) + \phi(r'm) 
\end{align*}
It is clearly surjective. Suppose $ \phi(rm) = r+I= I$, then $ r \in I$. Thus $ rm = 0$ by definition of annilator. It follows that  $ \ker \phi = \{0\} $ and $ \phi$ is injective. Therefore, $ M \cong R /I$. Since $ M$ is simple, by the isomorphism $ R /I$ also has no proper nontrivial submodules and thus has no proper nontrivial ideals. Hence  $ R /I$ is a field (every nonzero element generates $ R /I = \langle 1 \rangle$ and therefore is a unit) so $ I$ is maximal.

$ (\impliedby):$ Suppose $ I$ is maximal and  $ M \cong R /I$ as $ R$ modules. Since  $ R /I$ is a field, it has no proper nontrivial ideals so it has no proper nontrivial submodules. By the isomorphism so is  $ M$. Thus  $ M$ is simple.
\end{problem}
\end{document}
