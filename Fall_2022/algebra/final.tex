\documentclass[12pt]{article}
\newcommand{\alert}[1]{{\bf \color{red} [Alert:] #1}}
\newcommand{\todo}[1]{{\bf \color{orange} [TODO:] #1}}
\newcommand{\real}[1][]{\mathbb{R}^{#1}}
\newcommand{\myeqn}[1]{(\ref{#1})}
\newcommand{\myex}[1]{Example \ref{#1}}
\newcommand{\defeq}{\stackrel{\mathrm{def}}{=}}
\newcommand{\parder}[2]{\frac{\partial #1}{\partial #2}}
\newcommand{\Lie}[3][]{\mathsf{L}_{#3}^{#1} #2}
\newcommand{\LieA}[1]{\mathsf{Lie}(#1)}
\newcommand{\lieder}[2]{\mathcal{L}_{#2} #1}
\renewcommand{\t}{^{\mbox{\tiny\sf T}}}
\newcommand{\trans}{^{\mbox{\tiny\sf T}}}
\newcommand{\markup}[1]{\{\textbf{#1}\}}
\newcommand{\msub}[1]{_\mathrm{#1}}
\newcommand{\msup}[1]{^\mathrm{#1}}
\newcommand{\inv}[1]{#1^{-1}}
\newcommand{\pinv}[1]{{#1}^{+}}
\newcommand{\myfracA}[2]{\displaystyle{\frac{#1}{#2}}}
\newcommand{\myfracB}[2]{{#1}/{#2}}
\newcommand{\mydiffA}[1]{\dot{#1}}
\newcommand{\mydiffB}[2]{\myfracA{\mathrm{d}{#1}}{\mathrm{d}{#2}}}
\newcommand{\ball}[2]{\mathcal{B}_{#1}\left(#2\right)}
\newcommand{\acos}[1]{\cos^{-1}\left(#1\right)}
\newcommand{\asin}[1]{\sin^{-1}\left(#1\right)}
\newcommand{\mani}{\mathcal{M}}
\newcommand{\tang}[2]{\mathsf{T}_{#1} #2}
\newcommand{\LieB}[2]{[ #1, #2 ]}
\newcommand{\LieBad}[3][]{\mathsf{ad}_{#2}^{#1} #3}
\newcommand{\ReachVT}{\mathcal{R}^V_T}
\newcommand{\ReachVt}{\mathcal{R}^V_t}
\newcommand{\ReachVTe}{\mathcal{R}^V_{\le T}}
\newcommand{\ReachT}{\mathcal{R}_T}
\newcommand{\Reacht}{\mathcal{R}_t}
\newcommand{\ReachTe}{\mathcal{R}_{\le T}}
\newcommand{\accLA}[1]{\mathsf{Lie}(#1)}
\newcommand{\accD}{\Delta_{\mathcal{F}}}
\newcommand{\accSA}{\mathsf{Lie}(\mathcal{G},f)}
\newcommand{\accDS}{\Delta_{\mathcal{G}}}
\newcommand{\eval}[3]{\mathsf{Ev}^{#2}_{#1}\left( #3 \right)}
\newcommand{\stlc}{\textsc{stlc}}
\newcommand{\clf}{\textsc{clf}}
\newcommand{\jqlf}{\textsc{jqlf}}
\newcommand{\dlas}{\textsc{dlas}}
\newcommand{\Ad}[2]{\mathsf{Ad}_{#1} #2}
\newcommand{\xe}{\ensuremath{x_e}}
\newcommand{\lebg}[1]{\mathcal{L}_{#1}}
\newcommand{\lebgx}[1]{\mathcal{L}_{#1 \mathrm{e}}}
\newcommand{\dom}{D}
\newcommand{\domT}{[t_0,\infty) \times D}
\newcommand{\rarrow}{\rightarrow}
\renewcommand{\d}{\mathrm{d}}
\renewcommand{\Re}{\mathbb{R}}
\newcommand{\C}{\mathrm{C}}

\newcommand{\QED}{{\unskip\nobreak\hfil\penalty50\hskip2em\vadjust{}
		\nobreak\hfil$\Box$\parfillskip=0pt\finalhyphendemerits=0\par}\vspace{0.1cm}}
\newcommand{\eoEx}{{\unskip\nobreak\hfil\penalty50\hskip0em\vadjust{}
		\nobreak\hfil$\Large\Diamond$\parfillskip=0pt\finalhyphendemerits=0\par}\vspace{0.1cm}}

\newcommand{\sgn}{\ensuremath{\operatorname{sgn}}}
\newcommand{\sat}{\ensuremath{\operatorname{sat}}}

\newcommand{\half}{\frac{1}{2}}
\newcommand{\shalf}{\mbox{$\frac{1}{2}$}}
\newcommand{\marcom}[1]{\marginpar{\footnotesize #1}}
\newcommand{\der}{\mathrm{D}}
\newcommand{\e}{\mathrm{e}}
\newcommand{\dt}{\mathrm{d}t}

\newcommand{\cA}{\ensuremath{\mathcal{A}}}
\newcommand{\cB}{\ensuremath{\mathcal{B}}}
\newcommand{\cG}{\ensuremath{\mathcal{G}}}
\newcommand{\cK}{\ensuremath{\mathcal{K}}}
\newcommand{\cW}{\ensuremath{\mathcal{W}}}
\newcommand{\cZ}{\ensuremath{\mathcal{Z}}}
\newcommand{\cS}{\ensuremath{\mathcal{S}}}
\newcommand{\cD}{\ensuremath{\mathcal{D}}}
\newcommand{\cP}{\ensuremath{\mathcal{P}}}
\newcommand{\cV}{\ensuremath{\mathcal{V}}}
\newcommand{\cL}{\ensuremath{\mathcal{L}}}
\newcommand{\cN}{\ensuremath{\mathcal{N}}}
\newcommand{\cI}{\ensuremath{\mathcal{I}}}
\newcommand{\cR}{\ensuremath{\mathcal{R}}}
\newcommand{\cM}{\ensuremath{\mathcal{M}}}
\newcommand{\cC}{\ensuremath{\mathcal{C}}}
\newcommand{\cF}{\ensuremath{\mathcal{F}}}
\newcommand{\cH}{\ensuremath{\mathcal{H}}}
\newcommand{\cO}{\ensuremath{\mathcal{O}}}
\newcommand{\cX}{\ensuremath{\mathcal{X}}}
\newcommand{\cY}{\ensuremath{\mathcal{Y}}}
\newcommand{\Ci}{\ensuremath{\mathcal{C}^\infty}}
\newcommand{\ISS}{\textsc{iss}}
\newcommand{\LISS}{\textsc{liss}}
\newcommand{\GAS}{\textsc{gas}}
\newcommand{\GS}{\textsc{gs}}
\newcommand{\LES}{\textsc{les}}
\newcommand{\GUAS}{\textsc{guas}}
\newcommand{\BIBO}{\textsc{bibo}}
\newcommand{\spec}{\ensuremath{\operatorname{spec}}}
\newcommand{\spn}{\ensuremath{\operatorname{span}}}
\renewcommand{\i}{\mathrm{i\,}}

\renewcommand{\implies}{\Rightarrow}

\renewcommand{\theenumi}{$\roman{enumi})$}
\renewcommand{\labelenumi}{\theenumi}

\font\ptmten=zptmcmrm scaled 1200
\newcommand{\w}{\mbox{{\ptmten w}}}
\newcommand{\z}{\mbox{{\ptmten z}}}
\renewcommand{\Re}{\mathbb{R}}

\newcommand{\cl}{\operatorname{cl}}
\newcommand{\intr}{\operatorname{int}}
\newcommand{\rank}{\operatorname{rank}}
\newcommand{\co}{\operatorname{co}}
\newcommand{\aff}{\operatorname{aff}}

\theoremstyle{plain}
\newtheorem{theorem}{Theorem}[chapter]
\newtheorem{claim}[theorem]{Claim}
\newtheorem{corollary}[theorem]{Corollary}
\newtheorem{prop}[theorem]{Proposition}
\newtheorem{fact}[theorem]{Fact}
\newtheorem{lemma}[theorem]{Lemma}

\newtheorem{remark}{Remark}[chapter]

\theoremstyle{definition}
\newtheorem{assume}[theorem]{Assumption}
\newtheorem{defn}[theorem]{Definition}
\newtheorem{problem}[theorem]{Problem}
\newtheorem{exercise}{Exercise}
\newtheorem{example}[theorem]{Example}


\begin{document}
\centerline {\textsf{\textbf{\LARGE{Algebra 1 Final}}}}
\centerline {Jaden Wang}
\vspace{.15in}
Please do NOT grade Problem 4.

\begin{problem}[1]
Let $ |G:H| = n$.  
\begin{case}[1]
If $ n$ is a prime, then it must be the smallest prime dividing  $ |G|$ by assumption. Then $ |H|$ has index the smallest prime and therefore  $ H \trianglelefteq G$ by Corollary 4.5.
\end{case}
\begin{case}[2]
Supoose $ n$ is not a prime. Then the prime factors of $ |H|$ must be strictly  larger than $ n$. Consider the action representation $ \phi: G \to S_n$ of $ G$ acting on cosets $ G /H$ by left multiplication. Per Homework 2 Problem 2 we know that $ K:=\ker \phi$ is the largest normal subgroup contained in $ H$. Therefore,  $|G:H| =n$ divides $ |G:K| $ divides $ |G|=|H|n$. That is, $ |G:K| = dn$ where  $ d| |H|$. But since we know that every prime dividing  $ |H|$ is strictly larger than $ n$, each prime factor of $ d$ is also greater than $ n$ as well. Moreover,  $ |\im \phi| $ divides $ |S_n|=n!$, so by the definition of factorial, prime factors of $ |\im \phi|$ must all be no greater than  $ n$. By the first isomorphism theorem, we know that  $ |G:K| = | \im \phi|$. However, if $ d > 1$, then  $ |G:K|$ would contain prime factors strictly larger than any prime factor in $ | \im \phi|$ so they wouldn't equal, a contradiction. It follows that $ d=1$ and therefore  $ |G:K| = |G:H|$. This implies that  $ H= K$ and therefore  $ H \trianglelefteq G$ as $ K$ does.
\end{case}
\end{problem}
\begin{problem}[2]
By the class equation, 
\begin{align*}
	pq-p-q=|X| = |Z| + \sum_{ a \in A} [G:G_a]
\end{align*}
where $ Z$ is the set of elements with trivial orbits ( \emph{i.e.} fixed points) and $ A$ is the set consisting of one representative from each nontrivial orbit. Since  $ |G|=pq$,  $[G:G_a]$ must divide  $ |G|$ so  $ [G:G_a] =1, p,q$ or  $ pq$. If $ [G:G_a]=1$, then  $ G$ fixes  $ a$ and we are done. If  $ [G:G_a] =pq$, this violates that the sum is at most $ pq-p-q<pq$. We are left of the cases where  $ [G:G_a] = p$ or  $ q$.  Then the sum can be written as
\begin{align*}
	\sum_{a \in A} [G:G_a] = mp+nq
\end{align*}
for some $ m,n \in \nn$ (note when $ A = \O$, $ m=n=0$). It follows that
\begin{align*}
	pq-p-q &= |Z| + mp+nq \\
	|Z| &= pq-(m+1)p - (n+1)q
\end{align*}
Suppose to the contrary that $ |Z| = 0$, then  $ pq = (m+1)p+(n+1)q$. This implies that  $ p|(m+1)p+(n+1)q$ and  $ q|(m+1)p+(n+1)q$ which by the definition of primes implies  $ p|n+1$ and  $ q|m+1$. But since $ m+1,n+1>0$, we must have  $ m+1 \geq q$ and  $ n+1 \geq p$, then  $ (m+1)p+(n+1)q \geq qp+pq =2pq>pq$, a contradiction.  This forces $ |Z|>0$. That is, there is at least one fixed point.
\end{problem}

\begin{problem}[3]
Given $ P \in Syl_{ p}( G) $ and $ Q \in Syl_{ p}( H)$, since $ |Q|=p ^{k}$ for some $ k$ and  $ Q \leq H \leq G$, $ Q$ is a  $ p$-subgroup of  $ G$. Thus by Sylow Theorem, there exists a $ g \in G$ s.t.\ $ Q \leq gPg^{-1}$. This establishes that $ Q \leq gPg ^{-1} \cap H$. 

Let $ |gPg^{-1}| = p ^{n}$ for some $ n \geq k$, and $ |H|=p ^{k}m$ where $ p$ doesn't divide  $ m$. We see that $ gPg^{-1} \cap H$ must have order dividing both $ p ^{n}$ and $ p ^{k}m$ and therefore must divide $ p ^{k} = |Q|$, \emph{i.e.} $ |gPg^{-1} \cap H|\leq |Q|$. It follows that $ Q = gPg^{-1} \cap H$.
 
\end{problem}

\begin{problem}[4] (DO NOT GRADE). 
For any $ a \in \zz$, denote $ a_m:= a \bmod m$ and $ a_n := a \bmod n$. Since $ \zz_n = \langle 1_n \rangle$ and $ \zz_m = \langle 1_m \rangle$ as abelian groups, we can define an abelian group (or $ \zz$-module) homomorphism $ \phi$ by mapping generator to generator: $ \phi: \zz_n \to \zz_m, 1_n \mapsto 1_m$. Note $ \phi(a_n) = a_m$. This is well-defined since $ m|n$ and $ 1_m \cdot n = 1_m \cdot m \cdot k = 0 \cdot k = 0$ satisfies the relation on the generator $ 1_n$. Surjectivity is clear from that the generator $ 1_m$  is hit. It remains to check that $ \phi$ respects multiplication: given $ a_n,b_n \in \zz_n$, we have
\begin{align*}
	\phi(a_n b_n) &= \phi((ab)_n) \\
	&= (ab)_m\\
	&= a_m b_m \\
	&= \phi(a_n) \phi(b_n) 
\end{align*}
Hence $ \phi$ is a ring homomorphism.

Given $ a_n \in \zz_n^{ \times }$, we know that $ \gcd ( a,n) =1$. Since $ m|n$, we have  $ \gcd ( a,m)=1 $ so $ a_m = \phi(a_n) \in \zz_m^{ \times }$. Thus the restriction $ \overline{\phi}: \zz_n^{ \times } \to \zz_m^{ \times }$ of $ \phi$ is well-defined and clearly remains a homomorphism. Let $ n=km$ for some $ k \in \zz_{+}$. Consider $ \ker \overline{\phi} = \{a_n \in \zz_n^{ \times }: \overline{\phi}(a_n) = a_m =1_m\} = \{a_n \in \zz_n^{ \times }: a = 1+\ell m, \ell \in \zz\} = \{a_n: a = 1+ \ell m , 0\leq \ell \leq k-1, \gcd ( 1+\ell m,km)=1 \} $. Since it is clear that $ \gcd ( 1+\ell m, km) =1 =$, we finally simplify to
\begin{align*}
	\ker \overline{\phi} = \{a_n: a=1+\ell m, 0\leq \ell \leq k-1, \gcd (a, k) =1\}.
\end{align*}
\end{problem}

\begin{problem}[5]
	It doesn't seem like commutativity is required?

	First I show an elementary proof. Suppose we have a multiplication $ \times $ structure with identity on $ \qq / \zz$. Let the identity be $ [p /q] \neq [0]$, \emph{i.e.} $ q \neq 0$ and $p /q \not\in \zz$, as $ \qq /\zz$ is not the trivial ring. Then by the axiom of identity, $ [p /q] \times [1 /2] = [1 /2] \neq [0]$. However,
\begin{align*}
	[p /q] \times [1/2] &= [p /2q+ p /2q] \times [1 /2]\\
			    &= ([p /2q] + [p / 2q])\times [1 /2] \\
			    &= [p /2q] \times [1 /2] + [p /2q] \times [1 /2] && \text{ distributivity} \\
			    &= [p /2q] \times ([1 /2] + [1 /2]) && \text{ distributivity} \\
			    &= [p /2q] \times [1] \\
			    &= [p /2q] \times [0]\\
			    &= [0],
\end{align*}
a contradiction. Hence no such ring structure exists.

Here I also show a similar proof using the language of tensor products as a bonus.

	It suffices to show that no compatible multiplication can be defined on $ \qq / \zz$. Suppose that we have an associative and distributive binary operation (demanded by a ring multiplication) $ m: \qq / \zz \times \qq /\zz \to \qq / \zz$ defined on the $ \zz$-module $ \qq / \zz$, with a multiplicative identity $ \mathbbm{1}$. Then by distributivity laws, $ m$ is a $ \zz$-bilinear map. By the universal property of tensor products of modules over a commutative ring ($ \zz$) (Theorem 10.10 and Corollary 10.12), we have the commutative diagram:
% https://q.uiver.app/?q=WzAsMyxbMCwwLCJcXG1hdGhiYntRfS9cXG1hdGhiYntafSBcXHRpbWVzIFxcbWF0aGJie1F9L1xcbWF0aGJie1p9Il0sWzEsMCwiXFxtYXRoYmJ7UX0vXFxtYXRoYmJ7Wn0gXFxvdGltZXNfXFxtYXRoYmJ7Wn0gXFxtYXRoYmJ7UX0vXFxtYXRoYmJ7Wn0iXSxbMSwxLCJcXG1hdGhiYntRfS9cXG1hdGhiYntafSJdLFswLDEsIlxcaW90YSJdLFsxLDIsIlxccGhpIl0sWzAsMiwibSIsMl1d
\[\begin{tikzcd}
	{\mathbb{Q}/\mathbb{Z} \times \mathbb{Q}/\mathbb{Z}} & {\mathbb{Q}/\mathbb{Z} \otimes_\mathbb{Z} \mathbb{Q}/\mathbb{Z}} \\
	& {\mathbb{Q}/\mathbb{Z}}
	\arrow["\iota", from=1-1, to=1-2]
	\arrow["\phi", from=1-2, to=2-2]
	\arrow["m"', from=1-1, to=2-2]
\end{tikzcd}\]
However, since $ \qq / \zz \otimes _\zz \qq / \zz = 0$ (Example 4 under Corollary 10.12), this forces $ \iota = 0$ so $ m$ is the zero $ \zz$-module homomorphism as well. But this violates the axiom for the identity $ m(\mathbbm{1},[1 /2]) = [1/2] \neq [0]$ (since $ 1 / 2 \not\in  \zz$), a contradiction. Hence no such ring structure exists for the abelian group $ \qq / \zz$ so there is no such isomorphic ring either.
\end{problem}
\begin{problem}[8]
Since $ G$ is finitely generated, we can apply the structure theorem for $ \zz$-modules. In particular, we wish to diagonalize the presentation represented by the relation matrix via the Smith Normal Form.
\begin{align*}
	\begin{pmatrix} 1&-1&3\\2&1&0 \end{pmatrix} & \xrightarrow{ R_2 - 2R_1} \begin{pmatrix} 1&-1&3\\0&3&-6 \end{pmatrix} \\
			 & \xrightarrow{ C_2+ C_1, C_3-3C_1} \begin{pmatrix} 1&0&0\\0&3&-6 \end{pmatrix}  \\
			 & \xrightarrow{ C_3+2C_2} \begin{pmatrix} 1&0&0\\0&3&0 \end{pmatrix}  
\end{align*}
Thus we obtain a new set of generators $ x',y',z'$ with the relations $ x' = 0, 3y'=0,0z'=0$. Therefore, $ \Ann_{ \zz}( \langle x' \rangle) = \zz $, $ \Ann_{ \zz}( \langle y' \rangle) = \langle 3 \rangle$, and $ \Ann_{ \zz}( \langle z' \rangle) =0$ since $ z'$ has no relation. This implies that
\begin{align*}
	G &\cong \zz/ \Ann_{ \zz}( \langle x' \rangle ) \oplus \zz / \Ann_{ \zz}( \langle y' \rangle) \oplus \zz / \Ann_{ \zz}( \langle z' \rangle) \\
	&= \zz / \zz \oplus \zz / \langle 3 \rangle \oplus \zz /0\\
	&\cong \zz \oplus \zz_3
\end{align*}
\end{problem}
\begin{problem}[9]
The number of possible JCF (up to permutation of Jordan blocks) over $ \cc$ is the same as the number of possible RCF over $ \cc$. First we see that the minimal polynomial $ m(x) $ must be divisible by  $ x(x^2-1)(x^2+1) = x(x+1)(x-1)(x+i)(x-i)$. Then Cayley-Hamilton yields the following possible invariant factors over $ \cc$:
\begin{enumerate}[label=(\arabic*)]
	\item $ x(x^2+1)| x(x^2-1)(x^2+1)=m(x)$.
	\item $ x^2+1|x^2(x^2-1)(x^2+1) = m(x)$.
	\item $ x|x(x^2-1)(x^2+1)^2 = m(x)$.
	\item $ x^2(x^2-1)(x^2+1)^2=m(x)$.
\end{enumerate}
Therefore the number is 4.
\end{problem}

\begin{problem}[11]
By Eisenstein $ p=5$, we see that  $ 5|20$ but  $ 25 $ does not divide $ 20$ so  $ x^{15}+20$ is irreducible.
\begin{align*}
	x^{15}+20 &= 0 \\
	x^{15} &= -20 \\
	x &= -\sqrt[15]{20} e^{(2k\pi i) /15}
\end{align*}
Let $ \gamma:= \sqrt[15]{20} $ and $ \zeta = e^{2\pi i/15}$. Clearly $ \zeta, \gamma$ generate all the roots.

Since $ -\gamma$ is a root of $ x^{15}+20$ which is irreducible, $ [ \qq( \gamma): \qq] = 15$. Since $ \zeta$ is a primitive 15th roots of unit, by Corollary 13.42, $ [ \qq(\zeta): \qq] = \phi(15)=\phi(3)\phi(5)=2 \cdot 4= 8$. Since $ \gcd ( 15,8)=1 $, by Lagrange $ [\qq( \gamma, \zeta) : \qq] = 15 \cdot 8 = 120$. Since any field that $ x^{15}+20$ splits must contain $\gamma, \zeta$, and $ \qq( \gamma, \zeta)$ is the smallest field containing $ \gamma,\zeta$ by definition, we conclude that it is the splitting field of $ x^{15}+20$ with degree 120.
\end{problem}
\end{document}
