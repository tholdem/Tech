\documentclass[12pt]{article}
%Fall 2022
% Some basic packages
\usepackage{standalone}[subpreambles=true]
\usepackage[utf8]{inputenc}
\usepackage[T1]{fontenc}
\usepackage{textcomp}
\usepackage[english]{babel}
\usepackage{url}
\usepackage{graphicx}
%\usepackage{quiver}
\usepackage{float}
\usepackage{enumitem}
\usepackage{lmodern}
\usepackage{comment}
\usepackage{hyperref}
\usepackage[usenames,svgnames,dvipsnames]{xcolor}
\usepackage[margin=1in]{geometry}
\usepackage{pdfpages}

\pdfminorversion=7

% Don't indent paragraphs, leave some space between them
\usepackage{parskip}

% Hide page number when page is empty
\usepackage{emptypage}
\usepackage{subcaption}
\usepackage{multicol}
\usepackage[b]{esvect}

% Math stuff
\usepackage{amsmath, amsfonts, mathtools, amsthm, amssymb}
\usepackage{bbm}
\usepackage{stmaryrd}
\allowdisplaybreaks

% Fancy script capitals
\usepackage{mathrsfs}
\usepackage{cancel}
% Bold math
\usepackage{bm}
% Some shortcuts
\newcommand{\rr}{\ensuremath{\mathbb{R}}}
\newcommand{\zz}{\ensuremath{\mathbb{Z}}}
\newcommand{\qq}{\ensuremath{\mathbb{Q}}}
\newcommand{\nn}{\ensuremath{\mathbb{N}}}
\newcommand{\ff}{\ensuremath{\mathbb{F}}}
\newcommand{\cc}{\ensuremath{\mathbb{C}}}
\newcommand{\ee}{\ensuremath{\mathbb{E}}}
\newcommand{\hh}{\ensuremath{\mathbb{H}}}
\renewcommand\O{\ensuremath{\emptyset}}
\newcommand{\norm}[1]{{\left\lVert{#1}\right\rVert}}
\newcommand{\dbracket}[1]{{\left\llbracket{#1}\right\rrbracket}}
\newcommand{\ve}[1]{{\bm{#1}}}
\newcommand\allbold[1]{{\boldmath\textbf{#1}}}
\DeclareMathOperator{\lcm}{lcm}
\DeclareMathOperator{\im}{im}
\DeclareMathOperator{\coim}{coim}
\DeclareMathOperator{\dom}{dom}
\DeclareMathOperator{\tr}{tr}
\DeclareMathOperator{\rank}{rank}
\DeclareMathOperator*{\var}{Var}
\DeclareMathOperator*{\ev}{E}
\DeclareMathOperator{\dg}{deg}
\DeclareMathOperator{\aff}{aff}
\DeclareMathOperator{\conv}{conv}
\DeclareMathOperator{\inte}{int}
\DeclareMathOperator*{\argmin}{argmin}
\DeclareMathOperator*{\argmax}{argmax}
\DeclareMathOperator{\graph}{graph}
\DeclareMathOperator{\sgn}{sgn}
\DeclareMathOperator*{\Rep}{Rep}
\DeclareMathOperator{\Proj}{Proj}
\DeclareMathOperator{\mat}{mat}
\DeclareMathOperator{\diag}{diag}
\DeclareMathOperator{\aut}{Aut}
\DeclareMathOperator{\gal}{Gal}
\DeclareMathOperator{\inn}{Inn}
\DeclareMathOperator{\edm}{End}
\DeclareMathOperator{\Hom}{Hom}
\DeclareMathOperator{\ext}{Ext}
\DeclareMathOperator{\tor}{Tor}
\DeclareMathOperator{\Span}{Span}
\DeclareMathOperator{\Stab}{Stab}
\DeclareMathOperator{\cont}{cont}
\DeclareMathOperator{\Ann}{Ann}
\DeclareMathOperator{\Div}{div}
\DeclareMathOperator{\curl}{curl}
\DeclareMathOperator{\nat}{Nat}
\DeclareMathOperator{\gr}{Gr}
\DeclareMathOperator{\vect}{Vect}
\DeclareMathOperator{\id}{id}
\DeclareMathOperator{\Mod}{Mod}
\DeclareMathOperator{\sign}{sign}
\DeclareMathOperator{\Surf}{Surf}
\DeclareMathOperator{\fcone}{fcone}
\DeclareMathOperator{\Rot}{Rot}
\DeclareMathOperator{\grad}{grad}
\DeclareMathOperator{\atan2}{atan2}
\DeclareMathOperator{\Ric}{Ric}
\let\vec\relax
\DeclareMathOperator{\vec}{vec}
\let\Re\relax
\DeclareMathOperator{\Re}{Re}
\let\Im\relax
\DeclareMathOperator{\Im}{Im}
% Put x \to \infty below \lim
\let\svlim\lim\def\lim{\svlim\limits}

%wide hat
\usepackage{scalerel,stackengine}
\stackMath
\newcommand*\wh[1]{%
\savestack{\tmpbox}{\stretchto{%
  \scaleto{%
    \scalerel*[\widthof{\ensuremath{#1}}]{\kern-.6pt\bigwedge\kern-.6pt}%
    {\rule[-\textheight/2]{1ex}{\textheight}}%WIDTH-LIMITED BIG WEDGE
  }{\textheight}% 
}{0.5ex}}%
\stackon[1pt]{#1}{\tmpbox}%
}
\parskip 1ex

%Make implies and impliedby shorter
\let\implies\Rightarrow
\let\impliedby\Leftarrow
\let\iff\Leftrightarrow
\let\epsilon\varepsilon

% Add \contra symbol to denote contradiction
\usepackage{stmaryrd} % for \lightning
\newcommand\contra{\scalebox{1.5}{$\lightning$}}

% \let\phi\varphi

% Command for short corrections
% Usage: 1+1=\correct{3}{2}

\definecolor{correct}{HTML}{009900}
\newcommand\correct[2]{\ensuremath{\:}{\color{red}{#1}}\ensuremath{\to }{\color{correct}{#2}}\ensuremath{\:}}
\newcommand\green[1]{{\color{correct}{#1}}}

% horizontal rule
\newcommand\hr{
    \noindent\rule[0.5ex]{\linewidth}{0.5pt}
}

% hide parts
\newcommand\hide[1]{}

% si unitx
\usepackage{siunitx}
\sisetup{locale = FR}

%allows pmatrix to stretch
\makeatletter
\renewcommand*\env@matrix[1][\arraystretch]{%
  \edef\arraystretch{#1}%
  \hskip -\arraycolsep
  \let\@ifnextchar\new@ifnextchar
  \array{*\c@MaxMatrixCols c}}
\makeatother

\renewcommand{\arraystretch}{0.8}

\renewcommand{\baselinestretch}{1.5}

\usepackage{graphics}
\usepackage{epstopdf}

\RequirePackage{hyperref}
%%
%% Add support for color in order to color the hyperlinks.
%% 
\hypersetup{
  colorlinks = true,
  urlcolor = blue,
  citecolor = blue
}
%%fakesection Links
\hypersetup{
    colorlinks,
    linkcolor={red!50!black},
    citecolor={green!50!black},
    urlcolor={blue!80!black}
}
%customization of cleveref
\RequirePackage[capitalize,nameinlink]{cleveref}[0.19]

% Per SIAM Style Manual, "section" should be lowercase
\crefname{section}{section}{sections}
\crefname{subsection}{subsection}{subsections}
\Crefname{section}{Section}{Sections}
\Crefname{subsection}{Subsection}{Subsections}

% Per SIAM Style Manual, "Figure" should be spelled out in references
\Crefname{figure}{Figure}{Figures}

% Per SIAM Style Manual, don't say equation in front on an equation.
\crefformat{equation}{\textup{#2(#1)#3}}
\crefrangeformat{equation}{\textup{#3(#1)#4--#5(#2)#6}}
\crefmultiformat{equation}{\textup{#2(#1)#3}}{ and \textup{#2(#1)#3}}
{, \textup{#2(#1)#3}}{, and \textup{#2(#1)#3}}
\crefrangemultiformat{equation}{\textup{#3(#1)#4--#5(#2)#6}}%
{ and \textup{#3(#1)#4--#5(#2)#6}}{, \textup{#3(#1)#4--#5(#2)#6}}{, and \textup{#3(#1)#4--#5(#2)#6}}

% But spell it out at the beginning of a sentence.
\Crefformat{equation}{#2Equation~\textup{(#1)}#3}
\Crefrangeformat{equation}{Equations~\textup{#3(#1)#4--#5(#2)#6}}
\Crefmultiformat{equation}{Equations~\textup{#2(#1)#3}}{ and \textup{#2(#1)#3}}
{, \textup{#2(#1)#3}}{, and \textup{#2(#1)#3}}
\Crefrangemultiformat{equation}{Equations~\textup{#3(#1)#4--#5(#2)#6}}%
{ and \textup{#3(#1)#4--#5(#2)#6}}{, \textup{#3(#1)#4--#5(#2)#6}}{, and \textup{#3(#1)#4--#5(#2)#6}}

% Make number non-italic in any environment.
\crefdefaultlabelformat{#2\textup{#1}#3}

% Environments
\makeatother
% For box around Definition, Theorem, \ldots
%%fakesection Theorems
\usepackage{thmtools}
\usepackage[framemethod=TikZ]{mdframed}

\theoremstyle{definition}
\mdfdefinestyle{mdbluebox}{%
	roundcorner = 10pt,
	linewidth=1pt,
	skipabove=12pt,
	innerbottommargin=9pt,
	skipbelow=2pt,
	nobreak=true,
	linecolor=blue,
	backgroundcolor=TealBlue!5,
}
\declaretheoremstyle[
	headfont=\sffamily\bfseries\color{MidnightBlue},
	mdframed={style=mdbluebox},
	headpunct={\\[3pt]},
	postheadspace={0pt}
]{thmbluebox}

\mdfdefinestyle{mdredbox}{%
	linewidth=0.5pt,
	skipabove=12pt,
	frametitleaboveskip=5pt,
	frametitlebelowskip=0pt,
	skipbelow=2pt,
	frametitlefont=\bfseries,
	innertopmargin=4pt,
	innerbottommargin=8pt,
	nobreak=false,
	linecolor=RawSienna,
	backgroundcolor=Salmon!5,
}
\declaretheoremstyle[
	headfont=\bfseries\color{RawSienna},
	mdframed={style=mdredbox},
	headpunct={\\[3pt]},
	postheadspace={0pt},
]{thmredbox}

\declaretheorem[%
style=thmbluebox,name=Theorem,numberwithin=section]{thm}
\declaretheorem[style=thmbluebox,name=Lemma,sibling=thm]{lem}
\declaretheorem[style=thmbluebox,name=Proposition,sibling=thm]{prop}
\declaretheorem[style=thmbluebox,name=Corollary,sibling=thm]{coro}
\declaretheorem[style=thmredbox,name=Example,sibling=thm]{eg}

\mdfdefinestyle{mdgreenbox}{%
	roundcorner = 10pt,
	linewidth=1pt,
	skipabove=12pt,
	innerbottommargin=9pt,
	skipbelow=2pt,
	nobreak=true,
	linecolor=ForestGreen,
	backgroundcolor=ForestGreen!5,
}

\declaretheoremstyle[
	headfont=\bfseries\sffamily\color{ForestGreen!70!black},
	bodyfont=\normalfont,
	spaceabove=2pt,
	spacebelow=1pt,
	mdframed={style=mdgreenbox},
	headpunct={ --- },
]{thmgreenbox}

\declaretheorem[style=thmgreenbox,name=Definition,sibling=thm]{defn}

\mdfdefinestyle{mdgreenboxsq}{%
	linewidth=1pt,
	skipabove=12pt,
	innerbottommargin=9pt,
	skipbelow=2pt,
	nobreak=true,
	linecolor=ForestGreen,
	backgroundcolor=ForestGreen!5,
}
\declaretheoremstyle[
	headfont=\bfseries\sffamily\color{ForestGreen!70!black},
	bodyfont=\normalfont,
	spaceabove=2pt,
	spacebelow=1pt,
	mdframed={style=mdgreenboxsq},
	headpunct={},
]{thmgreenboxsq}
\declaretheoremstyle[
	headfont=\bfseries\sffamily\color{ForestGreen!70!black},
	bodyfont=\normalfont,
	spaceabove=2pt,
	spacebelow=1pt,
	mdframed={style=mdgreenboxsq},
	headpunct={},
]{thmgreenboxsq*}

\mdfdefinestyle{mdblackbox}{%
	skipabove=8pt,
	linewidth=3pt,
	rightline=false,
	leftline=true,
	topline=false,
	bottomline=false,
	linecolor=black,
	backgroundcolor=RedViolet!5!gray!5,
}
\declaretheoremstyle[
	headfont=\bfseries,
	bodyfont=\normalfont\small,
	spaceabove=0pt,
	spacebelow=0pt,
	mdframed={style=mdblackbox}
]{thmblackbox}

\theoremstyle{plain}
\declaretheorem[name=Question,sibling=thm,style=thmblackbox]{ques}
\declaretheorem[name=Remark,sibling=thm,style=thmgreenboxsq]{remark}
\declaretheorem[name=Remark,sibling=thm,style=thmgreenboxsq*]{remark*}
\newtheorem{ass}[thm]{Assumptions}

\theoremstyle{definition}
\newtheorem*{problem}{Problem}
\newtheorem{claim}[thm]{Claim}
\theoremstyle{remark}
\newtheorem*{case}{Case}
\newtheorem*{notation}{Notation}
\newtheorem*{note}{Note}
\newtheorem*{motivation}{Motivation}
\newtheorem*{intuition}{Intuition}
\newtheorem*{conjecture}{Conjecture}

% Make section starts with 1 for report type
%\renewcommand\thesection{\arabic{section}}

% End example and intermezzo environments with a small diamond (just like proof
% environments end with a small square)
\usepackage{etoolbox}
\AtEndEnvironment{vb}{\null\hfill$\diamond$}%
\AtEndEnvironment{intermezzo}{\null\hfill$\diamond$}%
% \AtEndEnvironment{opmerking}{\null\hfill$\diamond$}%

% Fix some spacing
% http://tex.stackexchange.com/questions/22119/how-can-i-change-the-spacing-before-theorems-with-amsthm
\makeatletter
\def\thm@space@setup{%
  \thm@preskip=\parskip \thm@postskip=0pt
}

% Fix some stuff
% %http://tex.stackexchange.com/questions/76273/multiple-pdfs-with-page-group-included-in-a-single-page-warning
\pdfsuppresswarningpagegroup=1


% My name
\author{Jaden Wang}



\begin{document}
\centerline {\textsf{\textbf{\LARGE{Algebra 1 Final}}}}
\centerline {Jaden Wang}
\vspace{.15in}
Please do NOT grade Problem 4.

\begin{problem}[1]
Let $ |G:H| = n$.  
\begin{case}[1]
If $ n$ is a prime, then it must be the smallest prime dividing  $ |G|$ by assumption. Then $ |H|$ has index the smallest prime and therefore  $ H \trianglelefteq G$ by Corollary 4.5.
\end{case}
\begin{case}[2]
Supoose $ n$ is not a prime. Then the prime factors of $ |H|$ must be strictly  larger than $ n$. Consider the action representation $ \phi: G \to S_n$ of $ G$ acting on cosets $ G /H$ by left multiplication. Per Homework 2 Problem 2 we know that $ K:=\ker \phi$ is the largest normal subgroup contained in $ H$. Therefore,  $|G:H| =n$ divides $ |G:K| $ divides $ |G|=|H|n$. That is, $ |G:K| = dn$ where  $ d| |H|$. But since we know that every prime dividing  $ |H|$ is strictly larger than $ n$, each prime factor of $ d$ is also greater than $ n$ as well. Moreover,  $ |\im \phi| $ divides $ |S_n|=n!$, so by the definition of factorial, prime factors of $ |\im \phi|$ must all be no greater than  $ n$. By the first isomorphism theorem, we know that  $ |G:K| = | \im \phi|$. However, if $ d > 1$, then  $ |G:K|$ would contain prime factors strictly larger than any prime factor in $ | \im \phi|$ so they wouldn't equal, a contradiction. It follows that $ d=1$ and therefore  $ |G:K| = |G:H|$. This implies that  $ H= K$ and therefore  $ H \trianglelefteq G$ as $ K$ does.
\end{case}
\end{problem}
\begin{problem}[2]
By the class equation, 
\begin{align*}
	pq-p-q=|X| = |Z| + \sum_{ a \in A} [G:G_a]
\end{align*}
where $ Z$ is the set of elements with trivial orbits ( \emph{i.e.} fixed points) and $ A$ is the set consisting of one representative from each nontrivial orbit. Since  $ |G|=pq$,  $[G:G_a]$ must divide  $ |G|$ so  $ [G:G_a] =1, p,q$ or  $ pq$. If $ [G:G_a]=1$, then  $ G$ fixes  $ a$ and we are done. If  $ [G:G_a] =pq$, this violates that the sum is at most $ pq-p-q<pq$. We are left of the cases where  $ [G:G_a] = p$ or  $ q$.  Then the sum can be written as
\begin{align*}
	\sum_{a \in A} [G:G_a] = mp+nq
\end{align*}
for some $ m,n \in \nn$ (note when $ A = \O$, $ m=n=0$). It follows that
\begin{align*}
	pq-p-q &= |Z| + mp+nq \\
	|Z| &= pq-(m+1)p - (n+1)q
\end{align*}
Suppose to the contrary that $ |Z| = 0$, then  $ pq = (m+1)p+(n+1)q$. This implies that  $ p|(m+1)p+(n+1)q$ and  $ q|(m+1)p+(n+1)q$ which by the definition of primes implies  $ p|n+1$ and  $ q|m+1$. But since $ m+1,n+1>0$, we must have  $ m+1 \geq q$ and  $ n+1 \geq p$, then  $ (m+1)p+(n+1)q \geq qp+pq =2pq>pq$, a contradiction.  This forces $ |Z|>0$. That is, there is at least one fixed point.
\end{problem}

\begin{problem}[3]
Given $ P \in Syl_{ p}( G) $ and $ Q \in Syl_{ p}( H)$, since $ |Q|=p ^{k}$ for some $ k$ and  $ Q \leq H \leq G$, $ Q$ is a  $ p$-subgroup of  $ G$. Thus by Sylow Theorem, there exists a $ g \in G$ s.t.\ $ Q \leq gPg^{-1}$. This establishes that $ Q \leq gPg ^{-1} \cap H$. 

Let $ |gPg^{-1}| = p ^{n}$ for some $ n \geq k$, and $ |H|=p ^{k}m$ where $ p$ doesn't divide  $ m$. We see that $ gPg^{-1} \cap H$ must have order dividing both $ p ^{n}$ and $ p ^{k}m$ and therefore must divide $ p ^{k} = |Q|$, \emph{i.e.} $ |gPg^{-1} \cap H|\leq |Q|$. It follows that $ Q = gPg^{-1} \cap H$.
 
\end{problem}

\begin{problem}[4] (DO NOT GRADE). 
For any $ a \in \zz$, denote $ a_m:= a \bmod m$ and $ a_n := a \bmod n$. Since $ \zz_n = \langle 1_n \rangle$ and $ \zz_m = \langle 1_m \rangle$ as abelian groups, we can define an abelian group (or $ \zz$-module) homomorphism $ \phi$ by mapping generator to generator: $ \phi: \zz_n \to \zz_m, 1_n \mapsto 1_m$. Note $ \phi(a_n) = a_m$. This is well-defined since $ m|n$ and $ 1_m \cdot n = 1_m \cdot m \cdot k = 0 \cdot k = 0$ satisfies the relation on the generator $ 1_n$. Surjectivity is clear from that the generator $ 1_m$  is hit. It remains to check that $ \phi$ respects multiplication: given $ a_n,b_n \in \zz_n$, we have
\begin{align*}
	\phi(a_n b_n) &= \phi((ab)_n) \\
	&= (ab)_m\\
	&= a_m b_m \\
	&= \phi(a_n) \phi(b_n) 
\end{align*}
Hence $ \phi$ is a ring homomorphism.

Given $ a_n \in \zz_n^{ \times }$, we know that $ \gcd ( a,n) =1$. Since $ m|n$, we have  $ \gcd ( a,m)=1 $ so $ a_m = \phi(a_n) \in \zz_m^{ \times }$. Thus the restriction $ \overline{\phi}: \zz_n^{ \times } \to \zz_m^{ \times }$ of $ \phi$ is well-defined and clearly remains a homomorphism. Let $ n=km$ for some $ k \in \zz_{+}$. Consider $ \ker \overline{\phi} = \{a_n \in \zz_n^{ \times }: \overline{\phi}(a_n) = a_m =1_m\} = \{a_n \in \zz_n^{ \times }: a = 1+\ell m, \ell \in \zz\} = \{a_n: a = 1+ \ell m , 0\leq \ell \leq k-1, \gcd ( 1+\ell m,km)=1 \} $. Since it is clear that $ \gcd ( 1+\ell m, km) =1 =$, we finally simplify to
\begin{align*}
	\ker \overline{\phi} = \{a_n: a=1+\ell m, 0\leq \ell \leq k-1, \gcd (a, k) =1\}.
\end{align*}
\end{problem}

\begin{problem}[5]
	It doesn't seem like commutativity is required?

	First I show an elementary proof. Suppose we have a multiplication $ \times $ structure with identity on $ \qq / \zz$. Let the identity be $ [p /q] \neq [0]$, \emph{i.e.} $ q \neq 0$ and $p /q \not\in \zz$, as $ \qq /\zz$ is not the trivial ring. Then by the axiom of identity, $ [p /q] \times [1 /2] = [1 /2] \neq [0]$. However,
\begin{align*}
	[p /q] \times [1/2] &= [p /2q+ p /2q] \times [1 /2]\\
			    &= ([p /2q] + [p / 2q])\times [1 /2] \\
			    &= [p /2q] \times [1 /2] + [p /2q] \times [1 /2] && \text{ distributivity} \\
			    &= [p /2q] \times ([1 /2] + [1 /2]) && \text{ distributivity} \\
			    &= [p /2q] \times [1] \\
			    &= [p /2q] \times [0]\\
			    &= [0],
\end{align*}
a contradiction. Hence no such ring structure exists.

Here I also show a similar proof using the language of tensor products as a bonus.

	It suffices to show that no compatible multiplication can be defined on $ \qq / \zz$. Suppose that we have an associative and distributive binary operation (demanded by a ring multiplication) $ m: \qq / \zz \times \qq /\zz \to \qq / \zz$ defined on the $ \zz$-module $ \qq / \zz$, with a multiplicative identity $ \mathbbm{1}$. Then by distributivity laws, $ m$ is a $ \zz$-bilinear map. By the universal property of tensor products of modules over a commutative ring ($ \zz$) (Theorem 10.10 and Corollary 10.12), we have the commutative diagram:
% https://q.uiver.app/?q=WzAsMyxbMCwwLCJcXG1hdGhiYntRfS9cXG1hdGhiYntafSBcXHRpbWVzIFxcbWF0aGJie1F9L1xcbWF0aGJie1p9Il0sWzEsMCwiXFxtYXRoYmJ7UX0vXFxtYXRoYmJ7Wn0gXFxvdGltZXNfXFxtYXRoYmJ7Wn0gXFxtYXRoYmJ7UX0vXFxtYXRoYmJ7Wn0iXSxbMSwxLCJcXG1hdGhiYntRfS9cXG1hdGhiYntafSJdLFswLDEsIlxcaW90YSJdLFsxLDIsIlxccGhpIl0sWzAsMiwibSIsMl1d
\[\begin{tikzcd}
	{\mathbb{Q}/\mathbb{Z} \times \mathbb{Q}/\mathbb{Z}} & {\mathbb{Q}/\mathbb{Z} \otimes_\mathbb{Z} \mathbb{Q}/\mathbb{Z}} \\
	& {\mathbb{Q}/\mathbb{Z}}
	\arrow["\iota", from=1-1, to=1-2]
	\arrow["\phi", from=1-2, to=2-2]
	\arrow["m"', from=1-1, to=2-2]
\end{tikzcd}\]
However, since $ \qq / \zz \otimes _\zz \qq / \zz = 0$ (Example 4 under Corollary 10.12), this forces $ \iota = 0$ so $ m$ is the zero $ \zz$-module homomorphism as well. But this violates the axiom for the identity $ m(\mathbbm{1},[1 /2]) = [1/2] \neq [0]$ (since $ 1 / 2 \not\in  \zz$), a contradiction. Hence no such ring structure exists for the abelian group $ \qq / \zz$ so there is no such isomorphic ring either.
\end{problem}
\begin{problem}[8]
Since $ G$ is finitely generated, we can apply the structure theorem for $ \zz$-modules. In particular, we wish to diagonalize the presentation represented by the relation matrix via the Smith Normal Form.
\begin{align*}
	\begin{pmatrix} 1&-1&3\\2&1&0 \end{pmatrix} & \xrightarrow{ R_2 - 2R_1} \begin{pmatrix} 1&-1&3\\0&3&-6 \end{pmatrix} \\
			 & \xrightarrow{ C_2+ C_1, C_3-3C_1} \begin{pmatrix} 1&0&0\\0&3&-6 \end{pmatrix}  \\
			 & \xrightarrow{ C_3+2C_2} \begin{pmatrix} 1&0&0\\0&3&0 \end{pmatrix}  
\end{align*}
Thus we obtain a new set of generators $ x',y',z'$ with the relations $ x' = 0, 3y'=0,0z'=0$. Therefore, $ \Ann_{ \zz}( \langle x' \rangle) = \zz $, $ \Ann_{ \zz}( \langle y' \rangle) = \langle 3 \rangle$, and $ \Ann_{ \zz}( \langle z' \rangle) =0$ since $ z'$ has no relation. This implies that
\begin{align*}
	G &\cong \zz/ \Ann_{ \zz}( \langle x' \rangle ) \oplus \zz / \Ann_{ \zz}( \langle y' \rangle) \oplus \zz / \Ann_{ \zz}( \langle z' \rangle) \\
	&= \zz / \zz \oplus \zz / \langle 3 \rangle \oplus \zz /0\\
	&\cong \zz \oplus \zz_3
\end{align*}
\end{problem}
\begin{problem}[9]
The number of possible JCF (up to permutation of Jordan blocks) over $ \cc$ is the same as the number of possible RCF over $ \cc$. First we see that the minimal polynomial $ m(x) $ must be divisible by  $ x(x^2-1)(x^2+1) = x(x+1)(x-1)(x+i)(x-i)$. Then Cayley-Hamilton yields the following possible invariant factors over $ \cc$:
\begin{enumerate}[label=(\arabic*)]
	\item $ x(x^2+1)| x(x^2-1)(x^2+1)=m(x)$.
	\item $ x^2+1|x^2(x^2-1)(x^2+1) = m(x)$.
	\item $ x|x(x^2-1)(x^2+1)^2 = m(x)$.
	\item $ x^2(x^2-1)(x^2+1)^2=m(x)$.
\end{enumerate}
Therefore the number is 4.
\end{problem}

\begin{problem}[11]
By Eisenstein $ p=5$, we see that  $ 5|20$ but  $ 25 $ does not divide $ 20$ so  $ x^{15}+20$ is irreducible.
\begin{align*}
	x^{15}+20 &= 0 \\
	x^{15} &= -20 \\
	x &= -\sqrt[15]{20} e^{(2k\pi i) /15}
\end{align*}
Let $ \gamma:= \sqrt[15]{20} $ and $ \zeta = e^{2\pi i/15}$. Clearly $ \zeta, \gamma$ generate all the roots.

Since $ -\gamma$ is a root of $ x^{15}+20$ which is irreducible, $ [ \qq( \gamma): \qq] = 15$. Since $ \zeta$ is a primitive 15th roots of unit, by Corollary 13.42, $ [ \qq(\zeta): \qq] = \phi(15)=\phi(3)\phi(5)=2 \cdot 4= 8$. Since $ \gcd ( 15,8)=1 $, by Lagrange $ [\qq( \gamma, \zeta) : \qq] = 15 \cdot 8 = 120$. Since any field that $ x^{15}+20$ splits must contain $\gamma, \zeta$, and $ \qq( \gamma, \zeta)$ is the smallest field containing $ \gamma,\zeta$ by definition, we conclude that it is the splitting field of $ x^{15}+20$ with degree 120.
\end{problem}
\end{document}
