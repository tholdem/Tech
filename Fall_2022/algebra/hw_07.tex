\documentclass[12pt,class=article,crop=false]{standalone} 
\newcommand{\alert}[1]{{\bf \color{red} [Alert:] #1}}
\newcommand{\todo}[1]{{\bf \color{orange} [TODO:] #1}}
\newcommand{\real}[1][]{\mathbb{R}^{#1}}
\newcommand{\myeqn}[1]{(\ref{#1})}
\newcommand{\myex}[1]{Example \ref{#1}}
\newcommand{\defeq}{\stackrel{\mathrm{def}}{=}}
\newcommand{\parder}[2]{\frac{\partial #1}{\partial #2}}
\newcommand{\Lie}[3][]{\mathsf{L}_{#3}^{#1} #2}
\newcommand{\LieA}[1]{\mathsf{Lie}(#1)}
\newcommand{\lieder}[2]{\mathcal{L}_{#2} #1}
\renewcommand{\t}{^{\mbox{\tiny\sf T}}}
\newcommand{\trans}{^{\mbox{\tiny\sf T}}}
\newcommand{\markup}[1]{\{\textbf{#1}\}}
\newcommand{\msub}[1]{_\mathrm{#1}}
\newcommand{\msup}[1]{^\mathrm{#1}}
\newcommand{\inv}[1]{#1^{-1}}
\newcommand{\pinv}[1]{{#1}^{+}}
\newcommand{\myfracA}[2]{\displaystyle{\frac{#1}{#2}}}
\newcommand{\myfracB}[2]{{#1}/{#2}}
\newcommand{\mydiffA}[1]{\dot{#1}}
\newcommand{\mydiffB}[2]{\myfracA{\mathrm{d}{#1}}{\mathrm{d}{#2}}}
\newcommand{\ball}[2]{\mathcal{B}_{#1}\left(#2\right)}
\newcommand{\acos}[1]{\cos^{-1}\left(#1\right)}
\newcommand{\asin}[1]{\sin^{-1}\left(#1\right)}
\newcommand{\mani}{\mathcal{M}}
\newcommand{\tang}[2]{\mathsf{T}_{#1} #2}
\newcommand{\LieB}[2]{[ #1, #2 ]}
\newcommand{\LieBad}[3][]{\mathsf{ad}_{#2}^{#1} #3}
\newcommand{\ReachVT}{\mathcal{R}^V_T}
\newcommand{\ReachVt}{\mathcal{R}^V_t}
\newcommand{\ReachVTe}{\mathcal{R}^V_{\le T}}
\newcommand{\ReachT}{\mathcal{R}_T}
\newcommand{\Reacht}{\mathcal{R}_t}
\newcommand{\ReachTe}{\mathcal{R}_{\le T}}
\newcommand{\accLA}[1]{\mathsf{Lie}(#1)}
\newcommand{\accD}{\Delta_{\mathcal{F}}}
\newcommand{\accSA}{\mathsf{Lie}(\mathcal{G},f)}
\newcommand{\accDS}{\Delta_{\mathcal{G}}}
\newcommand{\eval}[3]{\mathsf{Ev}^{#2}_{#1}\left( #3 \right)}
\newcommand{\stlc}{\textsc{stlc}}
\newcommand{\clf}{\textsc{clf}}
\newcommand{\jqlf}{\textsc{jqlf}}
\newcommand{\dlas}{\textsc{dlas}}
\newcommand{\Ad}[2]{\mathsf{Ad}_{#1} #2}
\newcommand{\xe}{\ensuremath{x_e}}
\newcommand{\lebg}[1]{\mathcal{L}_{#1}}
\newcommand{\lebgx}[1]{\mathcal{L}_{#1 \mathrm{e}}}
\newcommand{\dom}{D}
\newcommand{\domT}{[t_0,\infty) \times D}
\newcommand{\rarrow}{\rightarrow}
\renewcommand{\d}{\mathrm{d}}
\renewcommand{\Re}{\mathbb{R}}
\newcommand{\C}{\mathrm{C}}

\newcommand{\QED}{{\unskip\nobreak\hfil\penalty50\hskip2em\vadjust{}
		\nobreak\hfil$\Box$\parfillskip=0pt\finalhyphendemerits=0\par}\vspace{0.1cm}}
\newcommand{\eoEx}{{\unskip\nobreak\hfil\penalty50\hskip0em\vadjust{}
		\nobreak\hfil$\Large\Diamond$\parfillskip=0pt\finalhyphendemerits=0\par}\vspace{0.1cm}}

\newcommand{\sgn}{\ensuremath{\operatorname{sgn}}}
\newcommand{\sat}{\ensuremath{\operatorname{sat}}}

\newcommand{\half}{\frac{1}{2}}
\newcommand{\shalf}{\mbox{$\frac{1}{2}$}}
\newcommand{\marcom}[1]{\marginpar{\footnotesize #1}}
\newcommand{\der}{\mathrm{D}}
\newcommand{\e}{\mathrm{e}}
\newcommand{\dt}{\mathrm{d}t}

\newcommand{\cA}{\ensuremath{\mathcal{A}}}
\newcommand{\cB}{\ensuremath{\mathcal{B}}}
\newcommand{\cG}{\ensuremath{\mathcal{G}}}
\newcommand{\cK}{\ensuremath{\mathcal{K}}}
\newcommand{\cW}{\ensuremath{\mathcal{W}}}
\newcommand{\cZ}{\ensuremath{\mathcal{Z}}}
\newcommand{\cS}{\ensuremath{\mathcal{S}}}
\newcommand{\cD}{\ensuremath{\mathcal{D}}}
\newcommand{\cP}{\ensuremath{\mathcal{P}}}
\newcommand{\cV}{\ensuremath{\mathcal{V}}}
\newcommand{\cL}{\ensuremath{\mathcal{L}}}
\newcommand{\cN}{\ensuremath{\mathcal{N}}}
\newcommand{\cI}{\ensuremath{\mathcal{I}}}
\newcommand{\cR}{\ensuremath{\mathcal{R}}}
\newcommand{\cM}{\ensuremath{\mathcal{M}}}
\newcommand{\cC}{\ensuremath{\mathcal{C}}}
\newcommand{\cF}{\ensuremath{\mathcal{F}}}
\newcommand{\cH}{\ensuremath{\mathcal{H}}}
\newcommand{\cO}{\ensuremath{\mathcal{O}}}
\newcommand{\cX}{\ensuremath{\mathcal{X}}}
\newcommand{\cY}{\ensuremath{\mathcal{Y}}}
\newcommand{\Ci}{\ensuremath{\mathcal{C}^\infty}}
\newcommand{\ISS}{\textsc{iss}}
\newcommand{\LISS}{\textsc{liss}}
\newcommand{\GAS}{\textsc{gas}}
\newcommand{\GS}{\textsc{gs}}
\newcommand{\LES}{\textsc{les}}
\newcommand{\GUAS}{\textsc{guas}}
\newcommand{\BIBO}{\textsc{bibo}}
\newcommand{\spec}{\ensuremath{\operatorname{spec}}}
\newcommand{\spn}{\ensuremath{\operatorname{span}}}
\renewcommand{\i}{\mathrm{i\,}}

\renewcommand{\implies}{\Rightarrow}

\renewcommand{\theenumi}{$\roman{enumi})$}
\renewcommand{\labelenumi}{\theenumi}

\font\ptmten=zptmcmrm scaled 1200
\newcommand{\w}{\mbox{{\ptmten w}}}
\newcommand{\z}{\mbox{{\ptmten z}}}
\renewcommand{\Re}{\mathbb{R}}

\newcommand{\cl}{\operatorname{cl}}
\newcommand{\intr}{\operatorname{int}}
\newcommand{\rank}{\operatorname{rank}}
\newcommand{\co}{\operatorname{co}}
\newcommand{\aff}{\operatorname{aff}}

\theoremstyle{plain}
\newtheorem{theorem}{Theorem}[chapter]
\newtheorem{claim}[theorem]{Claim}
\newtheorem{corollary}[theorem]{Corollary}
\newtheorem{prop}[theorem]{Proposition}
\newtheorem{fact}[theorem]{Fact}
\newtheorem{lemma}[theorem]{Lemma}

\newtheorem{remark}{Remark}[chapter]

\theoremstyle{definition}
\newtheorem{assume}[theorem]{Assumption}
\newtheorem{defn}[theorem]{Definition}
\newtheorem{problem}[theorem]{Problem}
\newtheorem{exercise}{Exercise}
\newtheorem{example}[theorem]{Example}


\begin{document}
\begin{problem}[1]
Since every $ F$ has only two ideals, namely $ 0 = \langle 0 \rangle$ and  $ F = \langle 1 \rangle$, clearly every ideal of $ F$ is finitely generated so it is Noetherian. Therefore by repeated application of Hilbert basis theorem, $ F[x_1,\ldots,x_n]$ is Noetherian. Hence $ I = \langle a_1,\ldots,a_n \rangle$ is finitely generated.

FIX:
Since $ I = \langle S \rangle$, $ a_i \in \langle S \rangle$ implies that $ a_i$ is a finite sum of elements of $ S$. Hence  $ I$ is generated by finitely elements of  $ S$.
\end{problem}
\begin{problem}[2]
~\begin{enumerate}[label=(\alph*)]
	\item 
		\begin{enumerate}[label=(\roman*)]
			\item reflexive: given $ (r,s) \in R \times S$, clearly $ 1(rs-rs) = 0$ so $ (r,s) \sim (r,s)$.
			\item symmetric: suppose $ (r_1,s_1) \sim (r_2,s_2) \in R \times S$, \emph{i.e.} there exists $ t \in S$ s.t.\  $t(r_1s_2-r_2s_1) =0$. Then
				\begin{align*}
					t(r_1s_2-r_2s_1) &=-t(r_2s_1 - r_1s_2) & \\
							 &=t(r_1s_2 - r_2 s_1)\\ 
							 &= 0 
				\end{align*}
				Thus $ (r_2,r_2) \sim (r_1,s_1)$.
			\item transitive: suppose additionally that $ (r_2,s_2) \sim (r_3,s_3)$, \emph{i.e.} there exists $ t' \in S$ s.t.\ $ t'(r_2s_3-r_3s_2)$. Then since $ S$ is closed under multiplication, $ tt's_2 \in S$, so
				\begin{align*}
					tt's_2(r_1s_3 - r_3s_1) &= tt'(r_1s_2s_3-r_3s_1s_2) \\
					&= tt'(r_1s_2s_3 - r_2s_1s_3 + r_2 s_1 s_3 -r_3s_1s_2)  \\
					&= tt'(s_3(r_1s_2 - r_2s_1) + s_1(r_2s_3-r_3s_2)) \\
					&= s_3t'(t(r_1s_2-r_2s_1)) + s_1t(t'(r_2s_3-r_3s_2))\\
					&= 0+0 = 0
				\end{align*}
		\end{enumerate}
		Take $ \frac{r_1}{s_1}, \frac{r_2}{s_2}$, we want to check that the obvious addition and multiplication are well-defined. WLOG we just check independence of representatives for one term: let $ \frac{r_1}{s_1} \sim \frac{r_3}{s_3}$ with $ t$, then
		\begin{align*}
			\frac{r_1}{s_1}+ \frac{r_2}{s_2} &= \frac{r_1s_2+r_2s_1}{s_1s_2 }\\
			\frac{r_3}{ s_3} + \frac{r_2}{ s_2}&= \frac{r_3s_2+r_2s_3}{s_2s_3 } 
		\end{align*}
Notice $ ts_2^2 \in S$, so
		\begin{align*}
			t((r_1s_2+r_2s_1)(s_2s_3) - (r_3s_2+r_2s_3)(s_1s_2)) &= t(r_1s_2^2s_3-r_3s_2^2s_1) \\
			&=s_2^2 t(r_1s_3-r_3s_1) \\
			&= 0 
		\end{align*}
		So addition is well-defined and clearly associative and commutative. Similarly,
		\begin{align*}
			t(r_1r_2s_3s_2-r_3r_2s_1s_2) &=r_2s_2 t(r_1 s_3 - r_3s_1) \\
			&= 0 
		\end{align*}
		so multiplication is well-defined and clearly associative and commutative. The identity is obviously $ \frac{1}{1}$ as it satisfies the axiom.
	\item $ (\implies):$ We prove the contrapositive. Suppose that there is a zero-divisor $ t \in S$ of some $ r \in R$, \emph{i.e.} $ tr=0$ but  $ t,r \neq 0$. Then $ t(r-0)=0$ so  $ \frac{r}{1} = \frac{0}{1}$, \emph{i.e.}  $r \in \ker j$. Since the kernel is nontrivial, $ j$ is not injective.
		
		$ (\impliedby):$ suppose $ S$ contains no zero-divisors of  $ R$, then whenever  $ \frac{r}{1} = \frac{0}{1}$, \emph{i.e.} $ tr=0$ for some  $ t \in S$, since $ t$ is not a zero-divisor, by definition of zero-divisor $ tr =0 \iff r=0$, which proves injectivity.
\end{enumerate}
\end{problem}
\begin{problem}[3]
Since $ R$ is an integral domain, it doesn't contain any zero divisors. We also see that for any prime ideal $P $ of  $ R$ and  $ S_P:= R -P$, $ 1 \in S_P$, and if $ a,b \in S_P$, $ab \not\in P$ so $ ab \in S_P$. Thus $ S_P$ is closed under multiplication. Therefore by Problem 2, $ j: R \to R_P, r \mapsto \frac{r}{1}$ is injective. That is, there is a canonical identification of $ R$ in each $ R_P$, \emph{i.e.} $ R \subseteq R_P$. Hence $ R \subseteq \bigcap_{ P}R_P $. 

For the other direction, given $ \frac{r}{s} \in \bigcap_{ P} R_P = \{\frac{r}{s}: r \in R, s \in R-P \ \forall \ P\} $. I claim that $ s$ is a unit of $ R$. Suppose not, then  $ \langle s \rangle$ is a proper ideal of $ R$ so it must be contained in some maximal ideal  $ P$ which is also a prime ideal. But then $ s \not\in R-P$, a contradiction. Thus we see that $ (r -(rs^{-1})s) =0$ so $ (r,s) \sim (rs ^{-1},1)$ so $ \frac{r}{s} \in R$. Hence $ R = \bigcap_{ P} R_P $.
\end{problem}

\begin{problem}[4]
~\begin{enumerate}[label=(\alph*)]
	\item Let $ a+b\sqrt{-n} $ be a factor of 2 which implies that at least one of $ a,b$ is not $ 0$.
		\begin{align*}
			\frac{2}{a+b\sqrt{-n} } &= \frac{2(a-b\sqrt{-n} )}{a^2+b^2n } \\
			&= \frac{2a}{a^2+b^2n } - \frac{2b}{a^2+b^2n} \sqrt{-n} 
		\end{align*}
If $ b \neq 0$, then  $ \frac{2b}{ b^2n} = \frac{2}{bn}$ can never be an integer since $ n>3$ and $ |b|\geq 1$. Thus $ \frac{2b}{a^2+b^2n }$ can also never be an integer since $ a^2\geq 0$.

If $ b=0$, then  $ a \neq 0$, so  $\frac{2a}{a^2} =\frac{2}{a}$ is an integer only if $ a=\pm 1$ or $ \pm 2$. Hence we found factorizations $ 2 = \pm 1 \cdot \pm 2$ so 2 is irreducible.

Alternatively, $ 4= N(2) = N(a)N(b)$ so  $ N(a)$ and  $ N(b)$ can only be 1,2, or 4. But the norm is greater than 4 unless the real part is 1 or 2 and imaginary part is 0 (since  $ n>3$). So  $ 2=1 \cdot 2$ is the only possible factorization, hence $ 2$ is irreducible.

Similarly, we see that
\begin{align*}
	\frac{\sqrt{-n} }{ a+b\sqrt{-n} } &= \frac{a\sqrt{-n} +bn}{ a^2+b^2n} \\
	&= \frac{bn}{ a^2+b^2n} + \frac{a}{a^2+b^2n}\sqrt{-n} 
\end{align*}
If $ b \neq 0$, then  $ \frac{bn}{b^2n} = \frac{1}{b}$ is integer only if $ b = \pm 1$. In that case  $ \frac{a}{a^2+n}$ can never be integer unless $ a=0$. So we found factorizations $\sqrt{-n} = \pm 1 \cdot \pm \sqrt{-n}$.

If $ b=0$, then  $ a\neq 0$, so  $ \frac{a}{a^2+0} = \frac{1}{a}$ is an integer only if $ a = \pm$ which yields the case above. Hence $ \sqrt{-n} $ is an irreducible.

\begin{align*}
	\frac{1+\sqrt{-n} }{ a+b\sqrt{-n} } &= \frac{a+bn}{ a^2+b^2n} + \frac{a-b}{ a^2+b^2n} \sqrt{-n}  
\end{align*}
Notice
\begin{align*}
	|a-b| \leq |a|+|b| \leq a^2+b^2 \leq a^2+b^2n
\end{align*}
Thus $ \frac{a-b}{ a^2+b^2n}$ can be an integer only if $ a-b=0$ or if equality is achieved throughout the inequality chain. 

In the latter case, For the last inequality to be equality, it forces $ b=0$. The middle equality forces $ a = \pm 1$. Thus we have the factorization $ 1+\sqrt{-n} = \pm 1 \cdot \pm ( 1+\sqrt{-n} ) $.

If $ a-b=0$  \emph{i.e.} $ a=b \neq 0$, then  $ \frac{a+bn}{ a^2+b^2n} = \frac{a(n+1)}{a^2(n+1} = \frac{1}{a}$ is an integer only if $ a =b = \pm 1$ so we recover the factorization above. Hence  $ 1+\sqrt{-n} $ is irreducible.
\item If $ n$ is odd,  then $ 1+n$ is even  so $ 1+n = (1+\sqrt{-n} )(1-\sqrt{-n}) = 2 \cdot \frac{n+1}{ 2}$. We already know $ 1+\sqrt{-n} $ doesn't divide 2 by irreducibility. Moreover,
	\begin{align*}
		\frac{(n+1) /2}{1+\sqrt{-n} } = \frac{(n+1) /2}{ n+1} - \frac{(n+1) /2}{n+1 } \sqrt{-n} = \frac{1}{2} - \frac{1}{2} \sqrt{-n}  
	\end{align*}
	which are not integer coefficients so $ 1+\sqrt{-n}$ is not prime and $ \zz[\sqrt{-n} ]$ is not a UFD when $ n$ is odd.

	If $ n$ is even, then  $ -n = \sqrt{-n}^2 = 2 \cdot \frac{-n}{ 2}$. We know $ \sqrt{-n} $ doesn't divide 2. Moreover,
	\begin{align*}
		\frac{-n /2 }{ \sqrt{-n} } = \frac{1}{2} \sqrt{-n} 
	\end{align*}
	which are not integer coefficients so $ \sqrt{-n} $ is not prime and $ \zz[\sqrt{-n} ]$ is not a UFD when $ n$ is even. That's all the cases.
\item Consider $ \langle 2, \sqrt{-n}  \rangle$. Suppose to the contrary that $ \langle 2, \sqrt{-n}  \rangle = \langle a+b\sqrt{-n}  \rangle$. That means $ 2 = (a+b\sqrt{-n} )(c+d\sqrt{-n} )$ which by irreducibility forces $ a=\pm 2,b=0$ or  $ a=\pm 1,b=0$. But then $ \sqrt{-n} = (a+b\sqrt{-n} ) (x+y\sqrt{-n} )$ doesn't have factors with such values of $ a,b$ as we showed above. This is a contradiction so  $ \langle 2, \sqrt{-n}  \rangle$ is not a principal ideal.
\end{enumerate}
\end{problem}

\begin{problem}[5]
	~\begin{enumerate}[label=(\alph*)]
		\item Let $ N(a+b\sqrt{-2} ) = a^2+2b^2$ be the norm on $ \zz[\sqrt{-2} ]$. Then for $ x,y \in \zz[\sqrt{-2} ]$, $ y \neq 0$, it suffices to find a $ q,r \in \zz[\sqrt{-2} ]$ s.t.\  $ x=qy+r$ with $ N(r)<N(y)$. Notice
	\begin{align*}
		q+\frac{r}{y} &= \frac{x}{y}\\
			      &=\frac{a+b\sqrt{-2} }{c+d\sqrt{-2}  }\\ 
			      &= \frac{(a+b\sqrt{-2} )(c-d\sqrt{-2})}{ c^2+2d^2}\\
		&= \frac{ac+2bd+(ad+bc)\sqrt{-2} }{ c^2+2d^2} \\
		&= \frac{ac+2bd}{c^2+2d^2}+ \frac{(ad+bc)}{ c^2+2d^2} \sqrt{-2}  \\
		&=: \alpha + \beta \sqrt{-2}  
	\end{align*}
	Choose $ q$ to be $ m+n\sqrt{-2}$, where $ m,n$ are the closest integers to  $ \alpha, \beta$ respectively. That is $|\Re \frac{r}{y}|= |m- \alpha|\leq \frac{1}{2}$ and $|\Im \frac{r}{y}| = |n- \beta| \leq \frac{1}{2}$. Thus $ N\left( \frac{r}{y} \right) = \frac{1}{4} + 2 \cdot \frac{1}{4} = \frac{3}{4} <1$ so $ N( r) < N(y)$ as desired. 
\item Rewrite $ x^3 - y^2 = 2$ as $ y^2+2 = x^3$. Factoring it in $ \zz[\sqrt{-2} ]$ yields
	\begin{align*}
		(y-\sqrt{-2})(y+\sqrt{-2})  = x^3 
	\end{align*}
	I claim that $ y-\sqrt{-2} $ and $ y+\sqrt{-2} $ are relatively prime. First I claim that $ \sqrt{-2}$ is irreducible. To see this, suppose $ \sqrt{-2}=ab$, then $ 2 = N(\sqrt{-2}) = N(a)N(b)$, so one factor must have norm 1. It is easy to see that the only elements with norm 1 in $ \zz[\sqrt{-2} ]$ are $ \pm 1$, which are units. Hence $ \sqrt{-2} $ is an irreducible. Now suppose $ y-\sqrt{-2} $ and $ y+\sqrt{-2} $ have a common irreducible factor $ p$. Then $ p$ must also divide the sum and difference of the two,  \emph{i.e.}  $ p| 2y$ and $ p|2\sqrt{-2}$. Notice that $ 2\sqrt{-2} = -\sqrt{-2}^3$ is a product of irreducibles. Since we are in a Euclidean domain, it is also a UFD. So this factorization is unique. Therefore, the only irreducibles dividing $ 2\sqrt{-2} $ is $ \sqrt{-2}$, which also divides $ 2y$. However, since  $ x^3$ is a cube, we must have $ \sqrt{-2}^3$ dividing $ x^3$. This forces that at least one more $ \sqrt{-2} $ has to divide $ y-\sqrt{-2} / \sqrt{2} $. But then we wouldn't have integer coefficients. So $ \sqrt{-2} $ cannot be the common factor. Hence this forces the two to be relatively prime.

	Since they share no common factor, and their product is a perfect cube, in a UFD it must be that each of them is a perfect cube. Hence
	\begin{align*}
		y+\sqrt{-2} &= (a+b\sqrt{-2} )^3 \\ 
		&= a^3 -6ab^2+(3a^2b- 2b^3)\sqrt{-2} \\
		a(a^2-6b^2)&=y, \qquad b(3a^2-2b^2) = 1
	\end{align*}
	This says that $b$ must a unit of  $ \zz$, \emph{i.e.} $ b = \pm 1$. If $ b=1$, then  $ 3a^2-2 = 1$ and $ a=\pm 1$. If  $ b=-1$. then  $ 3a^2-2=-1$ so $ a$ has no integer solution. Thus $ y=a(a^2-6b^2) = \pm 5$ and $ x^3 = 25 +2 $ so $ x = 3$. Hence  $ (3,5)$ and  $ (3,-5)$ are the only solutions for $ x^3-y^2=2$.
	\end{enumerate}
	\end{problem}
\end{document}
