\documentclass[12pt,class=article,crop=false]{standalone} 
%Fall 2022
% Some basic packages
\usepackage{standalone}[subpreambles=true]
\usepackage[utf8]{inputenc}
\usepackage[T1]{fontenc}
\usepackage{textcomp}
\usepackage[english]{babel}
\usepackage{url}
\usepackage{graphicx}
%\usepackage{quiver}
\usepackage{float}
\usepackage{enumitem}
\usepackage{lmodern}
\usepackage{comment}
\usepackage{hyperref}
\usepackage[usenames,svgnames,dvipsnames]{xcolor}
\usepackage[margin=1in]{geometry}
\usepackage{pdfpages}

\pdfminorversion=7

% Don't indent paragraphs, leave some space between them
\usepackage{parskip}

% Hide page number when page is empty
\usepackage{emptypage}
\usepackage{subcaption}
\usepackage{multicol}
\usepackage[b]{esvect}

% Math stuff
\usepackage{amsmath, amsfonts, mathtools, amsthm, amssymb}
\usepackage{bbm}
\usepackage{stmaryrd}
\allowdisplaybreaks

% Fancy script capitals
\usepackage{mathrsfs}
\usepackage{cancel}
% Bold math
\usepackage{bm}
% Some shortcuts
\newcommand{\rr}{\ensuremath{\mathbb{R}}}
\newcommand{\zz}{\ensuremath{\mathbb{Z}}}
\newcommand{\qq}{\ensuremath{\mathbb{Q}}}
\newcommand{\nn}{\ensuremath{\mathbb{N}}}
\newcommand{\ff}{\ensuremath{\mathbb{F}}}
\newcommand{\cc}{\ensuremath{\mathbb{C}}}
\newcommand{\ee}{\ensuremath{\mathbb{E}}}
\newcommand{\hh}{\ensuremath{\mathbb{H}}}
\renewcommand\O{\ensuremath{\emptyset}}
\newcommand{\norm}[1]{{\left\lVert{#1}\right\rVert}}
\newcommand{\dbracket}[1]{{\left\llbracket{#1}\right\rrbracket}}
\newcommand{\ve}[1]{{\bm{#1}}}
\newcommand\allbold[1]{{\boldmath\textbf{#1}}}
\DeclareMathOperator{\lcm}{lcm}
\DeclareMathOperator{\im}{im}
\DeclareMathOperator{\coim}{coim}
\DeclareMathOperator{\dom}{dom}
\DeclareMathOperator{\tr}{tr}
\DeclareMathOperator{\rank}{rank}
\DeclareMathOperator*{\var}{Var}
\DeclareMathOperator*{\ev}{E}
\DeclareMathOperator{\dg}{deg}
\DeclareMathOperator{\aff}{aff}
\DeclareMathOperator{\conv}{conv}
\DeclareMathOperator{\inte}{int}
\DeclareMathOperator*{\argmin}{argmin}
\DeclareMathOperator*{\argmax}{argmax}
\DeclareMathOperator{\graph}{graph}
\DeclareMathOperator{\sgn}{sgn}
\DeclareMathOperator*{\Rep}{Rep}
\DeclareMathOperator{\Proj}{Proj}
\DeclareMathOperator{\mat}{mat}
\DeclareMathOperator{\diag}{diag}
\DeclareMathOperator{\aut}{Aut}
\DeclareMathOperator{\gal}{Gal}
\DeclareMathOperator{\inn}{Inn}
\DeclareMathOperator{\edm}{End}
\DeclareMathOperator{\Hom}{Hom}
\DeclareMathOperator{\ext}{Ext}
\DeclareMathOperator{\tor}{Tor}
\DeclareMathOperator{\Span}{Span}
\DeclareMathOperator{\Stab}{Stab}
\DeclareMathOperator{\cont}{cont}
\DeclareMathOperator{\Ann}{Ann}
\DeclareMathOperator{\Div}{div}
\DeclareMathOperator{\curl}{curl}
\DeclareMathOperator{\nat}{Nat}
\DeclareMathOperator{\gr}{Gr}
\DeclareMathOperator{\vect}{Vect}
\DeclareMathOperator{\id}{id}
\DeclareMathOperator{\Mod}{Mod}
\DeclareMathOperator{\sign}{sign}
\DeclareMathOperator{\Surf}{Surf}
\DeclareMathOperator{\fcone}{fcone}
\DeclareMathOperator{\Rot}{Rot}
\DeclareMathOperator{\grad}{grad}
\DeclareMathOperator{\atan2}{atan2}
\DeclareMathOperator{\Ric}{Ric}
\let\vec\relax
\DeclareMathOperator{\vec}{vec}
\let\Re\relax
\DeclareMathOperator{\Re}{Re}
\let\Im\relax
\DeclareMathOperator{\Im}{Im}
% Put x \to \infty below \lim
\let\svlim\lim\def\lim{\svlim\limits}

%wide hat
\usepackage{scalerel,stackengine}
\stackMath
\newcommand*\wh[1]{%
\savestack{\tmpbox}{\stretchto{%
  \scaleto{%
    \scalerel*[\widthof{\ensuremath{#1}}]{\kern-.6pt\bigwedge\kern-.6pt}%
    {\rule[-\textheight/2]{1ex}{\textheight}}%WIDTH-LIMITED BIG WEDGE
  }{\textheight}% 
}{0.5ex}}%
\stackon[1pt]{#1}{\tmpbox}%
}
\parskip 1ex

%Make implies and impliedby shorter
\let\implies\Rightarrow
\let\impliedby\Leftarrow
\let\iff\Leftrightarrow
\let\epsilon\varepsilon

% Add \contra symbol to denote contradiction
\usepackage{stmaryrd} % for \lightning
\newcommand\contra{\scalebox{1.5}{$\lightning$}}

% \let\phi\varphi

% Command for short corrections
% Usage: 1+1=\correct{3}{2}

\definecolor{correct}{HTML}{009900}
\newcommand\correct[2]{\ensuremath{\:}{\color{red}{#1}}\ensuremath{\to }{\color{correct}{#2}}\ensuremath{\:}}
\newcommand\green[1]{{\color{correct}{#1}}}

% horizontal rule
\newcommand\hr{
    \noindent\rule[0.5ex]{\linewidth}{0.5pt}
}

% hide parts
\newcommand\hide[1]{}

% si unitx
\usepackage{siunitx}
\sisetup{locale = FR}

%allows pmatrix to stretch
\makeatletter
\renewcommand*\env@matrix[1][\arraystretch]{%
  \edef\arraystretch{#1}%
  \hskip -\arraycolsep
  \let\@ifnextchar\new@ifnextchar
  \array{*\c@MaxMatrixCols c}}
\makeatother

\renewcommand{\arraystretch}{0.8}

\renewcommand{\baselinestretch}{1.5}

\usepackage{graphics}
\usepackage{epstopdf}

\RequirePackage{hyperref}
%%
%% Add support for color in order to color the hyperlinks.
%% 
\hypersetup{
  colorlinks = true,
  urlcolor = blue,
  citecolor = blue
}
%%fakesection Links
\hypersetup{
    colorlinks,
    linkcolor={red!50!black},
    citecolor={green!50!black},
    urlcolor={blue!80!black}
}
%customization of cleveref
\RequirePackage[capitalize,nameinlink]{cleveref}[0.19]

% Per SIAM Style Manual, "section" should be lowercase
\crefname{section}{section}{sections}
\crefname{subsection}{subsection}{subsections}
\Crefname{section}{Section}{Sections}
\Crefname{subsection}{Subsection}{Subsections}

% Per SIAM Style Manual, "Figure" should be spelled out in references
\Crefname{figure}{Figure}{Figures}

% Per SIAM Style Manual, don't say equation in front on an equation.
\crefformat{equation}{\textup{#2(#1)#3}}
\crefrangeformat{equation}{\textup{#3(#1)#4--#5(#2)#6}}
\crefmultiformat{equation}{\textup{#2(#1)#3}}{ and \textup{#2(#1)#3}}
{, \textup{#2(#1)#3}}{, and \textup{#2(#1)#3}}
\crefrangemultiformat{equation}{\textup{#3(#1)#4--#5(#2)#6}}%
{ and \textup{#3(#1)#4--#5(#2)#6}}{, \textup{#3(#1)#4--#5(#2)#6}}{, and \textup{#3(#1)#4--#5(#2)#6}}

% But spell it out at the beginning of a sentence.
\Crefformat{equation}{#2Equation~\textup{(#1)}#3}
\Crefrangeformat{equation}{Equations~\textup{#3(#1)#4--#5(#2)#6}}
\Crefmultiformat{equation}{Equations~\textup{#2(#1)#3}}{ and \textup{#2(#1)#3}}
{, \textup{#2(#1)#3}}{, and \textup{#2(#1)#3}}
\Crefrangemultiformat{equation}{Equations~\textup{#3(#1)#4--#5(#2)#6}}%
{ and \textup{#3(#1)#4--#5(#2)#6}}{, \textup{#3(#1)#4--#5(#2)#6}}{, and \textup{#3(#1)#4--#5(#2)#6}}

% Make number non-italic in any environment.
\crefdefaultlabelformat{#2\textup{#1}#3}

% Environments
\makeatother
% For box around Definition, Theorem, \ldots
%%fakesection Theorems
\usepackage{thmtools}
\usepackage[framemethod=TikZ]{mdframed}

\theoremstyle{definition}
\mdfdefinestyle{mdbluebox}{%
	roundcorner = 10pt,
	linewidth=1pt,
	skipabove=12pt,
	innerbottommargin=9pt,
	skipbelow=2pt,
	nobreak=true,
	linecolor=blue,
	backgroundcolor=TealBlue!5,
}
\declaretheoremstyle[
	headfont=\sffamily\bfseries\color{MidnightBlue},
	mdframed={style=mdbluebox},
	headpunct={\\[3pt]},
	postheadspace={0pt}
]{thmbluebox}

\mdfdefinestyle{mdredbox}{%
	linewidth=0.5pt,
	skipabove=12pt,
	frametitleaboveskip=5pt,
	frametitlebelowskip=0pt,
	skipbelow=2pt,
	frametitlefont=\bfseries,
	innertopmargin=4pt,
	innerbottommargin=8pt,
	nobreak=false,
	linecolor=RawSienna,
	backgroundcolor=Salmon!5,
}
\declaretheoremstyle[
	headfont=\bfseries\color{RawSienna},
	mdframed={style=mdredbox},
	headpunct={\\[3pt]},
	postheadspace={0pt},
]{thmredbox}

\declaretheorem[%
style=thmbluebox,name=Theorem,numberwithin=section]{thm}
\declaretheorem[style=thmbluebox,name=Lemma,sibling=thm]{lem}
\declaretheorem[style=thmbluebox,name=Proposition,sibling=thm]{prop}
\declaretheorem[style=thmbluebox,name=Corollary,sibling=thm]{coro}
\declaretheorem[style=thmredbox,name=Example,sibling=thm]{eg}

\mdfdefinestyle{mdgreenbox}{%
	roundcorner = 10pt,
	linewidth=1pt,
	skipabove=12pt,
	innerbottommargin=9pt,
	skipbelow=2pt,
	nobreak=true,
	linecolor=ForestGreen,
	backgroundcolor=ForestGreen!5,
}

\declaretheoremstyle[
	headfont=\bfseries\sffamily\color{ForestGreen!70!black},
	bodyfont=\normalfont,
	spaceabove=2pt,
	spacebelow=1pt,
	mdframed={style=mdgreenbox},
	headpunct={ --- },
]{thmgreenbox}

\declaretheorem[style=thmgreenbox,name=Definition,sibling=thm]{defn}

\mdfdefinestyle{mdgreenboxsq}{%
	linewidth=1pt,
	skipabove=12pt,
	innerbottommargin=9pt,
	skipbelow=2pt,
	nobreak=true,
	linecolor=ForestGreen,
	backgroundcolor=ForestGreen!5,
}
\declaretheoremstyle[
	headfont=\bfseries\sffamily\color{ForestGreen!70!black},
	bodyfont=\normalfont,
	spaceabove=2pt,
	spacebelow=1pt,
	mdframed={style=mdgreenboxsq},
	headpunct={},
]{thmgreenboxsq}
\declaretheoremstyle[
	headfont=\bfseries\sffamily\color{ForestGreen!70!black},
	bodyfont=\normalfont,
	spaceabove=2pt,
	spacebelow=1pt,
	mdframed={style=mdgreenboxsq},
	headpunct={},
]{thmgreenboxsq*}

\mdfdefinestyle{mdblackbox}{%
	skipabove=8pt,
	linewidth=3pt,
	rightline=false,
	leftline=true,
	topline=false,
	bottomline=false,
	linecolor=black,
	backgroundcolor=RedViolet!5!gray!5,
}
\declaretheoremstyle[
	headfont=\bfseries,
	bodyfont=\normalfont\small,
	spaceabove=0pt,
	spacebelow=0pt,
	mdframed={style=mdblackbox}
]{thmblackbox}

\theoremstyle{plain}
\declaretheorem[name=Question,sibling=thm,style=thmblackbox]{ques}
\declaretheorem[name=Remark,sibling=thm,style=thmgreenboxsq]{remark}
\declaretheorem[name=Remark,sibling=thm,style=thmgreenboxsq*]{remark*}
\newtheorem{ass}[thm]{Assumptions}

\theoremstyle{definition}
\newtheorem*{problem}{Problem}
\newtheorem{claim}[thm]{Claim}
\theoremstyle{remark}
\newtheorem*{case}{Case}
\newtheorem*{notation}{Notation}
\newtheorem*{note}{Note}
\newtheorem*{motivation}{Motivation}
\newtheorem*{intuition}{Intuition}
\newtheorem*{conjecture}{Conjecture}

% Make section starts with 1 for report type
%\renewcommand\thesection{\arabic{section}}

% End example and intermezzo environments with a small diamond (just like proof
% environments end with a small square)
\usepackage{etoolbox}
\AtEndEnvironment{vb}{\null\hfill$\diamond$}%
\AtEndEnvironment{intermezzo}{\null\hfill$\diamond$}%
% \AtEndEnvironment{opmerking}{\null\hfill$\diamond$}%

% Fix some spacing
% http://tex.stackexchange.com/questions/22119/how-can-i-change-the-spacing-before-theorems-with-amsthm
\makeatletter
\def\thm@space@setup{%
  \thm@preskip=\parskip \thm@postskip=0pt
}

% Fix some stuff
% %http://tex.stackexchange.com/questions/76273/multiple-pdfs-with-page-group-included-in-a-single-page-warning
\pdfsuppresswarningpagegroup=1


% My name
\author{Jaden Wang}



\begin{document}
\begin{problem}[1]
Since every $ F$ has only two ideals, namely $ 0 = \langle 0 \rangle$ and  $ F = \langle 1 \rangle$, clearly every ideal of $ F$ is finitely generated so it is Noetherian. Therefore by repeated application of Hilbert basis theorem, $ F[x_1,\ldots,x_n]$ is Noetherian. Hence $ I = \langle a_1,\ldots,a_n \rangle$ is finitely generated.

FIX:
Since $ I = \langle S \rangle$, $ a_i \in \langle S \rangle$ implies that $ a_i$ is a finite sum of elements of $ S$. Hence  $ I$ is generated by finitely elements of  $ S$.
\end{problem}
\begin{problem}[2]
~\begin{enumerate}[label=(\alph*)]
	\item 
		\begin{enumerate}[label=(\roman*)]
			\item reflexive: given $ (r,s) \in R \times S$, clearly $ 1(rs-rs) = 0$ so $ (r,s) \sim (r,s)$.
			\item symmetric: suppose $ (r_1,s_1) \sim (r_2,s_2) \in R \times S$, \emph{i.e.} there exists $ t \in S$ s.t.\  $t(r_1s_2-r_2s_1) =0$. Then
				\begin{align*}
					t(r_1s_2-r_2s_1) &=-t(r_2s_1 - r_1s_2) & \\
							 &=t(r_1s_2 - r_2 s_1)\\ 
							 &= 0 
				\end{align*}
				Thus $ (r_2,r_2) \sim (r_1,s_1)$.
			\item transitive: suppose additionally that $ (r_2,s_2) \sim (r_3,s_3)$, \emph{i.e.} there exists $ t' \in S$ s.t.\ $ t'(r_2s_3-r_3s_2)$. Then since $ S$ is closed under multiplication, $ tt's_2 \in S$, so
				\begin{align*}
					tt's_2(r_1s_3 - r_3s_1) &= tt'(r_1s_2s_3-r_3s_1s_2) \\
					&= tt'(r_1s_2s_3 - r_2s_1s_3 + r_2 s_1 s_3 -r_3s_1s_2)  \\
					&= tt'(s_3(r_1s_2 - r_2s_1) + s_1(r_2s_3-r_3s_2)) \\
					&= s_3t'(t(r_1s_2-r_2s_1)) + s_1t(t'(r_2s_3-r_3s_2))\\
					&= 0+0 = 0
				\end{align*}
		\end{enumerate}
		Take $ \frac{r_1}{s_1}, \frac{r_2}{s_2}$, we want to check that the obvious addition and multiplication are well-defined. WLOG we just check independence of representatives for one term: let $ \frac{r_1}{s_1} \sim \frac{r_3}{s_3}$ with $ t$, then
		\begin{align*}
			\frac{r_1}{s_1}+ \frac{r_2}{s_2} &= \frac{r_1s_2+r_2s_1}{s_1s_2 }\\
			\frac{r_3}{ s_3} + \frac{r_2}{ s_2}&= \frac{r_3s_2+r_2s_3}{s_2s_3 } 
		\end{align*}
Notice $ ts_2^2 \in S$, so
		\begin{align*}
			t((r_1s_2+r_2s_1)(s_2s_3) - (r_3s_2+r_2s_3)(s_1s_2)) &= t(r_1s_2^2s_3-r_3s_2^2s_1) \\
			&=s_2^2 t(r_1s_3-r_3s_1) \\
			&= 0 
		\end{align*}
		So addition is well-defined and clearly associative and commutative. Similarly,
		\begin{align*}
			t(r_1r_2s_3s_2-r_3r_2s_1s_2) &=r_2s_2 t(r_1 s_3 - r_3s_1) \\
			&= 0 
		\end{align*}
		so multiplication is well-defined and clearly associative and commutative. The identity is obviously $ \frac{1}{1}$ as it satisfies the axiom.
	\item $ (\implies):$ We prove the contrapositive. Suppose that there is a zero-divisor $ t \in S$ of some $ r \in R$, \emph{i.e.} $ tr=0$ but  $ t,r \neq 0$. Then $ t(r-0)=0$ so  $ \frac{r}{1} = \frac{0}{1}$, \emph{i.e.}  $r \in \ker j$. Since the kernel is nontrivial, $ j$ is not injective.
		
		$ (\impliedby):$ suppose $ S$ contains no zero-divisors of  $ R$, then whenever  $ \frac{r}{1} = \frac{0}{1}$, \emph{i.e.} $ tr=0$ for some  $ t \in S$, since $ t$ is not a zero-divisor, by definition of zero-divisor $ tr =0 \iff r=0$, which proves injectivity.
\end{enumerate}
\end{problem}
\begin{problem}[3]
Since $ R$ is an integral domain, it doesn't contain any zero divisors. We also see that for any prime ideal $P $ of  $ R$ and  $ S_P:= R -P$, $ 1 \in S_P$, and if $ a,b \in S_P$, $ab \not\in P$ so $ ab \in S_P$. Thus $ S_P$ is closed under multiplication. Therefore by Problem 2, $ j: R \to R_P, r \mapsto \frac{r}{1}$ is injective. That is, there is a canonical identification of $ R$ in each $ R_P$, \emph{i.e.} $ R \subseteq R_P$. Hence $ R \subseteq \bigcap_{ P}R_P $. 

For the other direction, given $ \frac{r}{s} \in \bigcap_{ P} R_P = \{\frac{r}{s}: r \in R, s \in R-P \ \forall \ P\} $. I claim that $ s$ is a unit of $ R$. Suppose not, then  $ \langle s \rangle$ is a proper ideal of $ R$ so it must be contained in some maximal ideal  $ P$ which is also a prime ideal. But then $ s \not\in R-P$, a contradiction. Thus we see that $ (r -(rs^{-1})s) =0$ so $ (r,s) \sim (rs ^{-1},1)$ so $ \frac{r}{s} \in R$. Hence $ R = \bigcap_{ P} R_P $.
\end{problem}

\begin{problem}[4]
~\begin{enumerate}[label=(\alph*)]
	\item Let $ a+b\sqrt{-n} $ be a factor of 2 which implies that at least one of $ a,b$ is not $ 0$.
		\begin{align*}
			\frac{2}{a+b\sqrt{-n} } &= \frac{2(a-b\sqrt{-n} )}{a^2+b^2n } \\
			&= \frac{2a}{a^2+b^2n } - \frac{2b}{a^2+b^2n} \sqrt{-n} 
		\end{align*}
If $ b \neq 0$, then  $ \frac{2b}{ b^2n} = \frac{2}{bn}$ can never be an integer since $ n>3$ and $ |b|\geq 1$. Thus $ \frac{2b}{a^2+b^2n }$ can also never be an integer since $ a^2\geq 0$.

If $ b=0$, then  $ a \neq 0$, so  $\frac{2a}{a^2} =\frac{2}{a}$ is an integer only if $ a=\pm 1$ or $ \pm 2$. Hence we found factorizations $ 2 = \pm 1 \cdot \pm 2$ so 2 is irreducible.

Alternatively, $ 4= N(2) = N(a)N(b)$ so  $ N(a)$ and  $ N(b)$ can only be 1,2, or 4. But the norm is greater than 4 unless the real part is 1 or 2 and imaginary part is 0 (since  $ n>3$). So  $ 2=1 \cdot 2$ is the only possible factorization, hence $ 2$ is irreducible.

Similarly, we see that
\begin{align*}
	\frac{\sqrt{-n} }{ a+b\sqrt{-n} } &= \frac{a\sqrt{-n} +bn}{ a^2+b^2n} \\
	&= \frac{bn}{ a^2+b^2n} + \frac{a}{a^2+b^2n}\sqrt{-n} 
\end{align*}
If $ b \neq 0$, then  $ \frac{bn}{b^2n} = \frac{1}{b}$ is integer only if $ b = \pm 1$. In that case  $ \frac{a}{a^2+n}$ can never be integer unless $ a=0$. So we found factorizations $\sqrt{-n} = \pm 1 \cdot \pm \sqrt{-n}$.

If $ b=0$, then  $ a\neq 0$, so  $ \frac{a}{a^2+0} = \frac{1}{a}$ is an integer only if $ a = \pm$ which yields the case above. Hence $ \sqrt{-n} $ is an irreducible.

\begin{align*}
	\frac{1+\sqrt{-n} }{ a+b\sqrt{-n} } &= \frac{a+bn}{ a^2+b^2n} + \frac{a-b}{ a^2+b^2n} \sqrt{-n}  
\end{align*}
Notice
\begin{align*}
	|a-b| \leq |a|+|b| \leq a^2+b^2 \leq a^2+b^2n
\end{align*}
Thus $ \frac{a-b}{ a^2+b^2n}$ can be an integer only if $ a-b=0$ or if equality is achieved throughout the inequality chain. 

In the latter case, For the last inequality to be equality, it forces $ b=0$. The middle equality forces $ a = \pm 1$. Thus we have the factorization $ 1+\sqrt{-n} = \pm 1 \cdot \pm ( 1+\sqrt{-n} ) $.

If $ a-b=0$  \emph{i.e.} $ a=b \neq 0$, then  $ \frac{a+bn}{ a^2+b^2n} = \frac{a(n+1)}{a^2(n+1} = \frac{1}{a}$ is an integer only if $ a =b = \pm 1$ so we recover the factorization above. Hence  $ 1+\sqrt{-n} $ is irreducible.
\item If $ n$ is odd,  then $ 1+n$ is even  so $ 1+n = (1+\sqrt{-n} )(1-\sqrt{-n}) = 2 \cdot \frac{n+1}{ 2}$. We already know $ 1+\sqrt{-n} $ doesn't divide 2 by irreducibility. Moreover,
	\begin{align*}
		\frac{(n+1) /2}{1+\sqrt{-n} } = \frac{(n+1) /2}{ n+1} - \frac{(n+1) /2}{n+1 } \sqrt{-n} = \frac{1}{2} - \frac{1}{2} \sqrt{-n}  
	\end{align*}
	which are not integer coefficients so $ 1+\sqrt{-n}$ is not prime and $ \zz[\sqrt{-n} ]$ is not a UFD when $ n$ is odd.

	If $ n$ is even, then  $ -n = \sqrt{-n}^2 = 2 \cdot \frac{-n}{ 2}$. We know $ \sqrt{-n} $ doesn't divide 2. Moreover,
	\begin{align*}
		\frac{-n /2 }{ \sqrt{-n} } = \frac{1}{2} \sqrt{-n} 
	\end{align*}
	which are not integer coefficients so $ \sqrt{-n} $ is not prime and $ \zz[\sqrt{-n} ]$ is not a UFD when $ n$ is even. That's all the cases.
\item Consider $ \langle 2, \sqrt{-n}  \rangle$. Suppose to the contrary that $ \langle 2, \sqrt{-n}  \rangle = \langle a+b\sqrt{-n}  \rangle$. That means $ 2 = (a+b\sqrt{-n} )(c+d\sqrt{-n} )$ which by irreducibility forces $ a=\pm 2,b=0$ or  $ a=\pm 1,b=0$. But then $ \sqrt{-n} = (a+b\sqrt{-n} ) (x+y\sqrt{-n} )$ doesn't have factors with such values of $ a,b$ as we showed above. This is a contradiction so  $ \langle 2, \sqrt{-n}  \rangle$ is not a principal ideal.
\end{enumerate}
\end{problem}

\begin{problem}[5]
	~\begin{enumerate}[label=(\alph*)]
		\item Let $ N(a+b\sqrt{-2} ) = a^2+2b^2$ be the norm on $ \zz[\sqrt{-2} ]$. Then for $ x,y \in \zz[\sqrt{-2} ]$, $ y \neq 0$, it suffices to find a $ q,r \in \zz[\sqrt{-2} ]$ s.t.\  $ x=qy+r$ with $ N(r)<N(y)$. Notice
	\begin{align*}
		q+\frac{r}{y} &= \frac{x}{y}\\
			      &=\frac{a+b\sqrt{-2} }{c+d\sqrt{-2}  }\\ 
			      &= \frac{(a+b\sqrt{-2} )(c-d\sqrt{-2})}{ c^2+2d^2}\\
		&= \frac{ac+2bd+(ad+bc)\sqrt{-2} }{ c^2+2d^2} \\
		&= \frac{ac+2bd}{c^2+2d^2}+ \frac{(ad+bc)}{ c^2+2d^2} \sqrt{-2}  \\
		&=: \alpha + \beta \sqrt{-2}  
	\end{align*}
	Choose $ q$ to be $ m+n\sqrt{-2}$, where $ m,n$ are the closest integers to  $ \alpha, \beta$ respectively. That is $|\Re \frac{r}{y}|= |m- \alpha|\leq \frac{1}{2}$ and $|\Im \frac{r}{y}| = |n- \beta| \leq \frac{1}{2}$. Thus $ N\left( \frac{r}{y} \right) = \frac{1}{4} + 2 \cdot \frac{1}{4} = \frac{3}{4} <1$ so $ N( r) < N(y)$ as desired. 
\item Rewrite $ x^3 - y^2 = 2$ as $ y^2+2 = x^3$. Factoring it in $ \zz[\sqrt{-2} ]$ yields
	\begin{align*}
		(y-\sqrt{-2})(y+\sqrt{-2})  = x^3 
	\end{align*}
	I claim that $ y-\sqrt{-2} $ and $ y+\sqrt{-2} $ are relatively prime. First I claim that $ \sqrt{-2}$ is irreducible. To see this, suppose $ \sqrt{-2}=ab$, then $ 2 = N(\sqrt{-2}) = N(a)N(b)$, so one factor must have norm 1. It is easy to see that the only elements with norm 1 in $ \zz[\sqrt{-2} ]$ are $ \pm 1$, which are units. Hence $ \sqrt{-2} $ is an irreducible. Now suppose $ y-\sqrt{-2} $ and $ y+\sqrt{-2} $ have a common irreducible factor $ p$. Then $ p$ must also divide the sum and difference of the two,  \emph{i.e.}  $ p| 2y$ and $ p|2\sqrt{-2}$. Notice that $ 2\sqrt{-2} = -\sqrt{-2}^3$ is a product of irreducibles. Since we are in a Euclidean domain, it is also a UFD. So this factorization is unique. Therefore, the only irreducibles dividing $ 2\sqrt{-2} $ is $ \sqrt{-2}$, which also divides $ 2y$. However, since  $ x^3$ is a cube, we must have $ \sqrt{-2}^3$ dividing $ x^3$. This forces that at least one more $ \sqrt{-2} $ has to divide $ y-\sqrt{-2} / \sqrt{2} $. But then we wouldn't have integer coefficients. So $ \sqrt{-2} $ cannot be the common factor. Hence this forces the two to be relatively prime.

	Since they share no common factor, and their product is a perfect cube, in a UFD it must be that each of them is a perfect cube. Hence
	\begin{align*}
		y+\sqrt{-2} &= (a+b\sqrt{-2} )^3 \\ 
		&= a^3 -6ab^2+(3a^2b- 2b^3)\sqrt{-2} \\
		a(a^2-6b^2)&=y, \qquad b(3a^2-2b^2) = 1
	\end{align*}
	This says that $b$ must a unit of  $ \zz$, \emph{i.e.} $ b = \pm 1$. If $ b=1$, then  $ 3a^2-2 = 1$ and $ a=\pm 1$. If  $ b=-1$. then  $ 3a^2-2=-1$ so $ a$ has no integer solution. Thus $ y=a(a^2-6b^2) = \pm 5$ and $ x^3 = 25 +2 $ so $ x = 3$. Hence  $ (3,5)$ and  $ (3,-5)$ are the only solutions for $ x^3-y^2=2$.
	\end{enumerate}
	\end{problem}
\end{document}
