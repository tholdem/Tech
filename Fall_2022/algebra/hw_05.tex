\documentclass[12pt]{article}
%Fall 2022
% Some basic packages
\usepackage{standalone}[subpreambles=true]
\usepackage[utf8]{inputenc}
\usepackage[T1]{fontenc}
\usepackage{textcomp}
\usepackage[english]{babel}
\usepackage{url}
\usepackage{graphicx}
%\usepackage{quiver}
\usepackage{float}
\usepackage{enumitem}
\usepackage{lmodern}
\usepackage{comment}
\usepackage{hyperref}
\usepackage[usenames,svgnames,dvipsnames]{xcolor}
\usepackage[margin=1in]{geometry}
\usepackage{pdfpages}

\pdfminorversion=7

% Don't indent paragraphs, leave some space between them
\usepackage{parskip}

% Hide page number when page is empty
\usepackage{emptypage}
\usepackage{subcaption}
\usepackage{multicol}
\usepackage[b]{esvect}

% Math stuff
\usepackage{amsmath, amsfonts, mathtools, amsthm, amssymb}
\usepackage{bbm}
\usepackage{stmaryrd}
\allowdisplaybreaks

% Fancy script capitals
\usepackage{mathrsfs}
\usepackage{cancel}
% Bold math
\usepackage{bm}
% Some shortcuts
\newcommand{\rr}{\ensuremath{\mathbb{R}}}
\newcommand{\zz}{\ensuremath{\mathbb{Z}}}
\newcommand{\qq}{\ensuremath{\mathbb{Q}}}
\newcommand{\nn}{\ensuremath{\mathbb{N}}}
\newcommand{\ff}{\ensuremath{\mathbb{F}}}
\newcommand{\cc}{\ensuremath{\mathbb{C}}}
\newcommand{\ee}{\ensuremath{\mathbb{E}}}
\newcommand{\hh}{\ensuremath{\mathbb{H}}}
\renewcommand\O{\ensuremath{\emptyset}}
\newcommand{\norm}[1]{{\left\lVert{#1}\right\rVert}}
\newcommand{\dbracket}[1]{{\left\llbracket{#1}\right\rrbracket}}
\newcommand{\ve}[1]{{\bm{#1}}}
\newcommand\allbold[1]{{\boldmath\textbf{#1}}}
\DeclareMathOperator{\lcm}{lcm}
\DeclareMathOperator{\im}{im}
\DeclareMathOperator{\coim}{coim}
\DeclareMathOperator{\dom}{dom}
\DeclareMathOperator{\tr}{tr}
\DeclareMathOperator{\rank}{rank}
\DeclareMathOperator*{\var}{Var}
\DeclareMathOperator*{\ev}{E}
\DeclareMathOperator{\dg}{deg}
\DeclareMathOperator{\aff}{aff}
\DeclareMathOperator{\conv}{conv}
\DeclareMathOperator{\inte}{int}
\DeclareMathOperator*{\argmin}{argmin}
\DeclareMathOperator*{\argmax}{argmax}
\DeclareMathOperator{\graph}{graph}
\DeclareMathOperator{\sgn}{sgn}
\DeclareMathOperator*{\Rep}{Rep}
\DeclareMathOperator{\Proj}{Proj}
\DeclareMathOperator{\mat}{mat}
\DeclareMathOperator{\diag}{diag}
\DeclareMathOperator{\aut}{Aut}
\DeclareMathOperator{\gal}{Gal}
\DeclareMathOperator{\inn}{Inn}
\DeclareMathOperator{\edm}{End}
\DeclareMathOperator{\Hom}{Hom}
\DeclareMathOperator{\ext}{Ext}
\DeclareMathOperator{\tor}{Tor}
\DeclareMathOperator{\Span}{Span}
\DeclareMathOperator{\Stab}{Stab}
\DeclareMathOperator{\cont}{cont}
\DeclareMathOperator{\Ann}{Ann}
\DeclareMathOperator{\Div}{div}
\DeclareMathOperator{\curl}{curl}
\DeclareMathOperator{\nat}{Nat}
\DeclareMathOperator{\gr}{Gr}
\DeclareMathOperator{\vect}{Vect}
\DeclareMathOperator{\id}{id}
\DeclareMathOperator{\Mod}{Mod}
\DeclareMathOperator{\sign}{sign}
\DeclareMathOperator{\Surf}{Surf}
\DeclareMathOperator{\fcone}{fcone}
\DeclareMathOperator{\Rot}{Rot}
\DeclareMathOperator{\grad}{grad}
\DeclareMathOperator{\atan2}{atan2}
\DeclareMathOperator{\Ric}{Ric}
\let\vec\relax
\DeclareMathOperator{\vec}{vec}
\let\Re\relax
\DeclareMathOperator{\Re}{Re}
\let\Im\relax
\DeclareMathOperator{\Im}{Im}
% Put x \to \infty below \lim
\let\svlim\lim\def\lim{\svlim\limits}

%wide hat
\usepackage{scalerel,stackengine}
\stackMath
\newcommand*\wh[1]{%
\savestack{\tmpbox}{\stretchto{%
  \scaleto{%
    \scalerel*[\widthof{\ensuremath{#1}}]{\kern-.6pt\bigwedge\kern-.6pt}%
    {\rule[-\textheight/2]{1ex}{\textheight}}%WIDTH-LIMITED BIG WEDGE
  }{\textheight}% 
}{0.5ex}}%
\stackon[1pt]{#1}{\tmpbox}%
}
\parskip 1ex

%Make implies and impliedby shorter
\let\implies\Rightarrow
\let\impliedby\Leftarrow
\let\iff\Leftrightarrow
\let\epsilon\varepsilon

% Add \contra symbol to denote contradiction
\usepackage{stmaryrd} % for \lightning
\newcommand\contra{\scalebox{1.5}{$\lightning$}}

% \let\phi\varphi

% Command for short corrections
% Usage: 1+1=\correct{3}{2}

\definecolor{correct}{HTML}{009900}
\newcommand\correct[2]{\ensuremath{\:}{\color{red}{#1}}\ensuremath{\to }{\color{correct}{#2}}\ensuremath{\:}}
\newcommand\green[1]{{\color{correct}{#1}}}

% horizontal rule
\newcommand\hr{
    \noindent\rule[0.5ex]{\linewidth}{0.5pt}
}

% hide parts
\newcommand\hide[1]{}

% si unitx
\usepackage{siunitx}
\sisetup{locale = FR}

%allows pmatrix to stretch
\makeatletter
\renewcommand*\env@matrix[1][\arraystretch]{%
  \edef\arraystretch{#1}%
  \hskip -\arraycolsep
  \let\@ifnextchar\new@ifnextchar
  \array{*\c@MaxMatrixCols c}}
\makeatother

\renewcommand{\arraystretch}{0.8}

\renewcommand{\baselinestretch}{1.5}

\usepackage{graphics}
\usepackage{epstopdf}

\RequirePackage{hyperref}
%%
%% Add support for color in order to color the hyperlinks.
%% 
\hypersetup{
  colorlinks = true,
  urlcolor = blue,
  citecolor = blue
}
%%fakesection Links
\hypersetup{
    colorlinks,
    linkcolor={red!50!black},
    citecolor={green!50!black},
    urlcolor={blue!80!black}
}
%customization of cleveref
\RequirePackage[capitalize,nameinlink]{cleveref}[0.19]

% Per SIAM Style Manual, "section" should be lowercase
\crefname{section}{section}{sections}
\crefname{subsection}{subsection}{subsections}
\Crefname{section}{Section}{Sections}
\Crefname{subsection}{Subsection}{Subsections}

% Per SIAM Style Manual, "Figure" should be spelled out in references
\Crefname{figure}{Figure}{Figures}

% Per SIAM Style Manual, don't say equation in front on an equation.
\crefformat{equation}{\textup{#2(#1)#3}}
\crefrangeformat{equation}{\textup{#3(#1)#4--#5(#2)#6}}
\crefmultiformat{equation}{\textup{#2(#1)#3}}{ and \textup{#2(#1)#3}}
{, \textup{#2(#1)#3}}{, and \textup{#2(#1)#3}}
\crefrangemultiformat{equation}{\textup{#3(#1)#4--#5(#2)#6}}%
{ and \textup{#3(#1)#4--#5(#2)#6}}{, \textup{#3(#1)#4--#5(#2)#6}}{, and \textup{#3(#1)#4--#5(#2)#6}}

% But spell it out at the beginning of a sentence.
\Crefformat{equation}{#2Equation~\textup{(#1)}#3}
\Crefrangeformat{equation}{Equations~\textup{#3(#1)#4--#5(#2)#6}}
\Crefmultiformat{equation}{Equations~\textup{#2(#1)#3}}{ and \textup{#2(#1)#3}}
{, \textup{#2(#1)#3}}{, and \textup{#2(#1)#3}}
\Crefrangemultiformat{equation}{Equations~\textup{#3(#1)#4--#5(#2)#6}}%
{ and \textup{#3(#1)#4--#5(#2)#6}}{, \textup{#3(#1)#4--#5(#2)#6}}{, and \textup{#3(#1)#4--#5(#2)#6}}

% Make number non-italic in any environment.
\crefdefaultlabelformat{#2\textup{#1}#3}

% Environments
\makeatother
% For box around Definition, Theorem, \ldots
%%fakesection Theorems
\usepackage{thmtools}
\usepackage[framemethod=TikZ]{mdframed}

\theoremstyle{definition}
\mdfdefinestyle{mdbluebox}{%
	roundcorner = 10pt,
	linewidth=1pt,
	skipabove=12pt,
	innerbottommargin=9pt,
	skipbelow=2pt,
	nobreak=true,
	linecolor=blue,
	backgroundcolor=TealBlue!5,
}
\declaretheoremstyle[
	headfont=\sffamily\bfseries\color{MidnightBlue},
	mdframed={style=mdbluebox},
	headpunct={\\[3pt]},
	postheadspace={0pt}
]{thmbluebox}

\mdfdefinestyle{mdredbox}{%
	linewidth=0.5pt,
	skipabove=12pt,
	frametitleaboveskip=5pt,
	frametitlebelowskip=0pt,
	skipbelow=2pt,
	frametitlefont=\bfseries,
	innertopmargin=4pt,
	innerbottommargin=8pt,
	nobreak=false,
	linecolor=RawSienna,
	backgroundcolor=Salmon!5,
}
\declaretheoremstyle[
	headfont=\bfseries\color{RawSienna},
	mdframed={style=mdredbox},
	headpunct={\\[3pt]},
	postheadspace={0pt},
]{thmredbox}

\declaretheorem[%
style=thmbluebox,name=Theorem,numberwithin=section]{thm}
\declaretheorem[style=thmbluebox,name=Lemma,sibling=thm]{lem}
\declaretheorem[style=thmbluebox,name=Proposition,sibling=thm]{prop}
\declaretheorem[style=thmbluebox,name=Corollary,sibling=thm]{coro}
\declaretheorem[style=thmredbox,name=Example,sibling=thm]{eg}

\mdfdefinestyle{mdgreenbox}{%
	roundcorner = 10pt,
	linewidth=1pt,
	skipabove=12pt,
	innerbottommargin=9pt,
	skipbelow=2pt,
	nobreak=true,
	linecolor=ForestGreen,
	backgroundcolor=ForestGreen!5,
}

\declaretheoremstyle[
	headfont=\bfseries\sffamily\color{ForestGreen!70!black},
	bodyfont=\normalfont,
	spaceabove=2pt,
	spacebelow=1pt,
	mdframed={style=mdgreenbox},
	headpunct={ --- },
]{thmgreenbox}

\declaretheorem[style=thmgreenbox,name=Definition,sibling=thm]{defn}

\mdfdefinestyle{mdgreenboxsq}{%
	linewidth=1pt,
	skipabove=12pt,
	innerbottommargin=9pt,
	skipbelow=2pt,
	nobreak=true,
	linecolor=ForestGreen,
	backgroundcolor=ForestGreen!5,
}
\declaretheoremstyle[
	headfont=\bfseries\sffamily\color{ForestGreen!70!black},
	bodyfont=\normalfont,
	spaceabove=2pt,
	spacebelow=1pt,
	mdframed={style=mdgreenboxsq},
	headpunct={},
]{thmgreenboxsq}
\declaretheoremstyle[
	headfont=\bfseries\sffamily\color{ForestGreen!70!black},
	bodyfont=\normalfont,
	spaceabove=2pt,
	spacebelow=1pt,
	mdframed={style=mdgreenboxsq},
	headpunct={},
]{thmgreenboxsq*}

\mdfdefinestyle{mdblackbox}{%
	skipabove=8pt,
	linewidth=3pt,
	rightline=false,
	leftline=true,
	topline=false,
	bottomline=false,
	linecolor=black,
	backgroundcolor=RedViolet!5!gray!5,
}
\declaretheoremstyle[
	headfont=\bfseries,
	bodyfont=\normalfont\small,
	spaceabove=0pt,
	spacebelow=0pt,
	mdframed={style=mdblackbox}
]{thmblackbox}

\theoremstyle{plain}
\declaretheorem[name=Question,sibling=thm,style=thmblackbox]{ques}
\declaretheorem[name=Remark,sibling=thm,style=thmgreenboxsq]{remark}
\declaretheorem[name=Remark,sibling=thm,style=thmgreenboxsq*]{remark*}
\newtheorem{ass}[thm]{Assumptions}

\theoremstyle{definition}
\newtheorem*{problem}{Problem}
\newtheorem{claim}[thm]{Claim}
\theoremstyle{remark}
\newtheorem*{case}{Case}
\newtheorem*{notation}{Notation}
\newtheorem*{note}{Note}
\newtheorem*{motivation}{Motivation}
\newtheorem*{intuition}{Intuition}
\newtheorem*{conjecture}{Conjecture}

% Make section starts with 1 for report type
%\renewcommand\thesection{\arabic{section}}

% End example and intermezzo environments with a small diamond (just like proof
% environments end with a small square)
\usepackage{etoolbox}
\AtEndEnvironment{vb}{\null\hfill$\diamond$}%
\AtEndEnvironment{intermezzo}{\null\hfill$\diamond$}%
% \AtEndEnvironment{opmerking}{\null\hfill$\diamond$}%

% Fix some spacing
% http://tex.stackexchange.com/questions/22119/how-can-i-change-the-spacing-before-theorems-with-amsthm
\makeatletter
\def\thm@space@setup{%
  \thm@preskip=\parskip \thm@postskip=0pt
}

% Fix some stuff
% %http://tex.stackexchange.com/questions/76273/multiple-pdfs-with-page-group-included-in-a-single-page-warning
\pdfsuppresswarningpagegroup=1


% My name
\author{Jaden Wang}



\begin{document}
\centerline {\textsf{\textbf{\LARGE{Homework 5}}}}
\centerline {Jaden Wang}
\vspace{.15in}

\begin{problem}[1]
	(collab with Ari and Will): Let $ G$ be a finitely generated group with generators $ \{g_1,\ldots,g_m\} $ and suppose $ H \leq G$ with $ [G:H] = n$ for some $ n \in \zz^{+}$. Let the cosets of $ H$ in  $ G$ be $ \{eH,a_2H,\ldots,a_n H\} $ (note we choose $ a_1 = e$). Since $ g_i a_j$ must be in one of the cosets $ a_k H$, we see that $ g_i a_j = a_k^{ij} h_{ij}$ for some $ h_{ij} \in H$. Moreover, for any given $ a_k$ and $ g_i$, let $ a_j$ be the representative of the coset that $ g_i ^{-1} a_k$ is in, then $ a_k H = g_i g_i ^{-1} a_k H = g_i a_j H$. Hence for every $ g_i$ we have $ g_i a_j = a_k^{ij} h_{ij}$ for some $ a_j$. That is, $ a_j$ (and therefore $ h_{ij}$ ) is determined solely by the choice of $ a_k$ and $ g_i$.

	Now I claim that $ \{h_{ij}:1\leq i \leq m, 1 \leq j\leq n\} $ generates $ H$. Since elements of $ H$ are words of generators of  $ G$, any $ h \in H$ is a finite string of $ g_i$. We wish to use $ h_{ij}$ to recover $ h$. So we start from the left: if the first letter in $ h$ is $ g_i$, then by using $ e =a_1 \in H$, we determine an $ a_j$ and $ h_{ij}$. This yields
	\begin{align*}
		h_{ij} = e_{ij} h_{ij} = g_i a_j 
	\end{align*}
So we recover the first letter $ g_i$ with an additional $ a_j$ on the right. Now suppose the second letter is $ g_\ell$, then $ a_j$ and $ g_{\ell}$ determine an $ a_k$ and $ h_{\ell k}$. Thus
\begin{align*}
	 h_{ij} h_{\ell k} &= g_i a_j h_{\ell k} \\
	&= g_i g_\ell a_k 
\end{align*}
So we recover the second letter, with an $ a_k$ on the right. Repeating this process until we recover the entire string of $ h$ with an $ a_p$ on the right. That is,
\begin{align*}
	h_{ij} h_{\ell k} \ldots h_{qp} = g_i g_\ell \ldots g_q a_p  = h a_p
\end{align*}
But since the LHS is in $ H$ and $ h \in H$, we have that $ a_p$ is also in  $ H$. This forces $ a_p = a_1 = e$ (otherwise it would be a representative of a coset not equal to $ H$). Hence $ h = h_{ij} h_{\ell k} \ldots h_{qp}$. That is, it is a product of generators of the form $ h_{ij}$ as desired.
\end{problem}

\begin{problem}[2]
$ 270 = 2 \cdot 3^3 \cdot 5$. Thus we only need to consider the partition of 3 which yields three cases: $ 3,(2,1),(1,1,1)$.
\begin{table}[H]
	\centering
	\begin{tabular}{c|c}
	\hline
	invariant factor & elementary divisor\\
	\hline
	$ Z_{270}$ & $Z_2 \times  Z_{3^3} \times Z_5$ \\
	$ Z_{90} \times \zz_3$ & $ Z_2 \times Z_{3^2} \times Z_5$ \\
	$ Z_{30} \times \zz_3 \times \zz_3$ & $ Z_2 \times Z_3 \times Z_3 \times Z_3 \times Z_5$. 
	\end{tabular}
\end{table}
\end{problem}
\begin{problem}[3]
There is no element of order $ 2$ in cyclic groups of odd order by Lagrange and there is a unique element of order 2 in each cyclic group with even order. Since for any cyclic group with even order $ Z_n = \langle g \rangle$, $ o(g) = n$, so $ o(g^{ \alpha}) = 2$ implies that $ g^{ \alpha} = g^{ - \alpha} \implies g^{ 2\alpha} = g^{n} =e$ (since $ \alpha \neq 0$ as that would be order 1). So $ \alpha = n /2$ which is unique. Hence we have an element of order 2 from each even cyclic group, yielding 3 in total. Their lcm order is also 2 so we have $ 2^3 -1 = 7$ ways to construct elements of order 2 as we exclude the identity.
\end{problem}
\begin{problem}[4]
Since $ G$ is finite, we can resort to dual group. Due to the isomorphisms, it suffices to find an injective map from $ \Hom( G, U)$ to $ \Hom( G, U)$ which would induce an injection from $ G /H \to G$ by isomorphisms. There are two ways to do this.
\begin{enumerate}[label=(\arabic*)]
	\item Let $ U$ denote the group of all roots of unity (or replace it with $ \cc^* $). Consider the map $ i^* : \Hom( G /H, U) \to \Hom( G, U), f \mapsto f \circ \pi$, where $ \pi: G \to G /H$ is the canonical projection map. Let $ \phi: G \to U$ be the trivial homomorphism that maps everything to 1. By the universal property of quotient groups, this induces a homomorphism $ \Phi: G /H \to U$ s.t.\ $ \phi = \Phi \circ \pi$. If $ H= G$ the problem is trivial so WLOG assume $ H$ is a proper subgroup. Then  $ \pi$  is not the trivial map. This forces $ \Phi$ to be the trivial map in $ \Hom( G /H, U)$, which shows that $ \ker i^* $ is trivial so $ i^*$ is injective. By the isomorphisms we get an injective map $ i: G /H \to G$ so $G /H \cong \im i \leq G$.
% https://q.uiver.app/?q=WzAsNSxbMCwwLCJHL0giXSxbMiwwLCJHIl0sWzAsMiwiXFx0ZXh0e0hvbX0oRy9ILFUpIl0sWzIsMiwiXFx0ZXh0e0hvbX0oRyxVKSJdLFsxLDEsIlxcY2lyYyJdLFswLDEsImkiLDAseyJzdHlsZSI6eyJ0YWlsIjp7Im5hbWUiOiJob29rIiwic2lkZSI6InRvcCJ9fX1dLFswLDIsIiIsMix7InN0eWxlIjp7InRhaWwiOnsibmFtZSI6Im1vbm8ifSwiaGVhZCI6eyJuYW1lIjoiZXBpIn19fV0sWzIsMywiaV4qIiwyLHsic3R5bGUiOnsidGFpbCI6eyJuYW1lIjoiaG9vayIsInNpZGUiOiJ0b3AifX19XSxbMSwzLCIiLDAseyJzdHlsZSI6eyJ0YWlsIjp7Im5hbWUiOiJtb25vIn0sImhlYWQiOnsibmFtZSI6ImVwaSJ9fX1dXQ==
\[\begin{tikzcd}
	{G/H} && G \\
	& \circ \\
	{\text{Hom}(G/H,U)} && {\text{Hom}(G,U)}
	\arrow["i", hook, from=1-1, to=1-3]
	\arrow[tail, two heads, from=1-1, to=3-1]
	\arrow["{i^*}"', hook, from=3-1, to=3-3]
	\arrow[tail, two heads, from=1-3, to=3-3]
\end{tikzcd}\]
	\item By viewing $ G$ as a  $ \zz$-module, recall that $ \Hom( -, U)$ is a right-adjoint functor between the cateogry of  \textsf{R-mod} so it preserves colimits including cokernel. Consider the short exact sequence:
\begin{align*}
	0 \to H \xrightarrow{i} G \xrightarrow{ \pi} G /H \to 0 .
\end{align*}
Applying $ \Hom( -, U)$ to the sequence yields an left-exact sequence
		\begin{align*}
			0 \to \Hom( G /H, U) \xrightarrow{i^* } \Hom( G, U) \to \Hom( H,U )
		\end{align*}
		By exactness, $ \ker i^*  = \im 0 = 0$ so $ i^* $ is injective. This yields the $ i: G /H \to G$ we seek. 
\end{enumerate}

If we omit finite (so we cannot use dual group isomorphisms), then consider $ G = \zz$, all subgroups of $ G$ are of infinite order, but for $H = 2\zz $, $ G /H \cong Z_2$ which has finite order so it cannot be isomorphic to a subgroup of $ G$.

If we omit abelian (so we cannot use either quotient group universal property or left-exactness of $ \Hom( -, U)$), then consider  $ G= S_3$ and $ H = \{e,(1,2)\} $. $ G /H$ is not a group as  $ H$ is not a normal subgroup, so  $ G /H$ clearly cannot be isomorphic to a subgroup.
\end{problem}
\begin{problem}[5]
~\begin{enumerate}[label=(\alph*)]
	\item First we show that $ \wh{ G}$ is a group. It contains the identity $ 1_{\wh{ G}}: g \mapsto 1$ and is clearly associative. Given $ f,g \in \wh{ G}$, since $ \cc^* $ is abelian, $ (f \cdot g)(xy) = f(xy)g(xy) = f(x)f(y)g(x)g(y) = f(x)g(x)f(y)g(x) = (f \cdot g)(x) (f \cdot g)(y)$ so it is closed under pointwise multiplication. The inverse of $ f$ is just  $ f^{-1} : x \mapsto \frac{1}{f(x)}$ which is well-defined as $ 0 \not\in \cc^* $. Commutativity is obvious as $ \cc^* $ is abelian.
	\item Since $ G$ is finite abelian, by FToFGAB, $ G \cong \langle g_1 \rangle \times \cdots \times \langle g_n \rangle $. We denote the order of each $ g_i$ by $ n_i$. We wish to show that $G \cong \Hom(G, \cc^* )$. Define $ x_i: G \to \cc^*, g_i \mapsto e^{i2\pi / n_i}, g_j \mapsto 1,j\neq i$. Notice that $ x_i^{n_i}: g_i \mapsto e^{i2\pi n_i /n_i} = 1, g_j \mapsto $ so $ n_i$ is the smallest power that makes $ x_i$ the trivial homomorphism so $ o(x_i) = n_i$. I claim that $ \{x_i\} $ generates $ \wh{ G}$. Given $ f \in \Hom(G, \cc^* )$, it suffices to specify where $ f$ maps each generator  $ g_i$. Since $ f$ is a homomorphism, the order of  $ f(g_i)$ must divide $ n_i$. Thus $ f$ must map  $ g_i$ to a root of unity, \emph{i.e.}  $ f: g_i \mapsto ( e^{i2 \pi /n_i})^{d_i} = (x_i(g_i))^{d_i}$ where $ d_i$ is some divisor of  $ n_i$. It follows that
		\begin{align*}
			f = \prod_{ i= 1}^{ n}  x_i^{d_i} 
		\end{align*}
where the product denotes pointwise multiplication. Thus $ \{x_i\} $ generates $ \wh{ G}$. Then map
\begin{align*}
	\phi: G \to \wh{ G}, g_i \mapsto x_i
\end{align*}
is thus a well-defined homomorphism as we map generators to generators of the same orders. Surjectivity follows from hitting all $ x_i$. Suppose $ \phi(g) = 1_{\wh{ G}}$ where $ 1_{\wh{ G}}: G \to U, g \mapsto 1$ is the trivial homomorphism, since $ \phi$ maps all generators $ g_i$ to a non-trivial homomorphism, the kernel must be trivial. Thus $ \phi$ is an isomorphism.
\end{enumerate}
\end{problem}

\begin{problem}[6]
Given $ x,y \in R$, $ x^2 = x$ and $ y^2 = y$, then notice $-x = (-x)^2 = x^2 = x$. Moreover,
\begin{align*}
	x+y=(x+y)^2 &= x^2 + xy+yx+y^2 \\
	&= x+xy+yx+y \\
	0 &= xy+yx \\
	0 &= -xy + yx \\
	xy &= yx 
\end{align*}
So $ R$ is commutative.
\end{problem}

\begin{problem}[7]
~\begin{enumerate}[label=(\alph*)]
	\item Clearly $ IJ$ is non-empty. It is closed under addition because sum of finite sums is still finite. It is closed under negation because  $ I,J$ are. Given  $ x_1 y_1+ \cdots + x_ky_k \in IJ$ and $ r \in R$, since $ I$ is an ideal, any  $ rx_i \in I$, so
		\begin{align*}
			r(x_1y_1 + \cdots +x_k y_k) &= (rx_1)y_1+ \cdots + (rx_k)y_k \in IJ
		\end{align*}
	So this proves closure under multiplication as well and shows that $ IJ$ is an ideal. Since  $ x_i y_i$ is both in $ I$ and $ J$, viewed as a left-ideal and right ideal respectively, the sum is also in both. So  $ IJ \subseteq I \cap J$. 

	Let $R= \zz, I = \langle 2 \rangle, J = \langle 4 \rangle$. Then $ I \cap J = J$ but $ IJ = \{2r_1 4r_1' + 2r_2 4 r_2' + \cdots + 2r_k 4 r_k' : r_i, r_i' \in R\} = \langle 8 \rangle \neq \langle 4 \rangle = I \cap J$.
\item Suppose $ I+J = R$. Then given  $ s \in I \cap J$, given $ r \in R$, we can write it as $ r= x+y$,  $ x \in I,y \in J$. Then
	\begin{align*}
		sr = s(x+y) = sx+ sy=xs + sy \in IJ
	\end{align*}
	Since $ IJ$ is an ideal, $ (s r) r^{-1} = s \in IJ$ and $ I \cap J \subseteq IJ$ which yields equality.
\end{enumerate}
\end{problem}
\end{document}
