\documentclass[12pt]{article}
%Fall 2022
% Some basic packages
\usepackage{standalone}[subpreambles=true]
\usepackage[utf8]{inputenc}
\usepackage[T1]{fontenc}
\usepackage{textcomp}
\usepackage[english]{babel}
\usepackage{url}
\usepackage{graphicx}
%\usepackage{quiver}
\usepackage{float}
\usepackage{enumitem}
\usepackage{lmodern}
\usepackage{comment}
\usepackage{hyperref}
\usepackage[usenames,svgnames,dvipsnames]{xcolor}
\usepackage[margin=1in]{geometry}
\usepackage{pdfpages}

\pdfminorversion=7

% Don't indent paragraphs, leave some space between them
\usepackage{parskip}

% Hide page number when page is empty
\usepackage{emptypage}
\usepackage{subcaption}
\usepackage{multicol}
\usepackage[b]{esvect}

% Math stuff
\usepackage{amsmath, amsfonts, mathtools, amsthm, amssymb}
\usepackage{bbm}
\usepackage{stmaryrd}
\allowdisplaybreaks

% Fancy script capitals
\usepackage{mathrsfs}
\usepackage{cancel}
% Bold math
\usepackage{bm}
% Some shortcuts
\newcommand{\rr}{\ensuremath{\mathbb{R}}}
\newcommand{\zz}{\ensuremath{\mathbb{Z}}}
\newcommand{\qq}{\ensuremath{\mathbb{Q}}}
\newcommand{\nn}{\ensuremath{\mathbb{N}}}
\newcommand{\ff}{\ensuremath{\mathbb{F}}}
\newcommand{\cc}{\ensuremath{\mathbb{C}}}
\newcommand{\ee}{\ensuremath{\mathbb{E}}}
\newcommand{\hh}{\ensuremath{\mathbb{H}}}
\renewcommand\O{\ensuremath{\emptyset}}
\newcommand{\norm}[1]{{\left\lVert{#1}\right\rVert}}
\newcommand{\dbracket}[1]{{\left\llbracket{#1}\right\rrbracket}}
\newcommand{\ve}[1]{{\bm{#1}}}
\newcommand\allbold[1]{{\boldmath\textbf{#1}}}
\DeclareMathOperator{\lcm}{lcm}
\DeclareMathOperator{\im}{im}
\DeclareMathOperator{\coim}{coim}
\DeclareMathOperator{\dom}{dom}
\DeclareMathOperator{\tr}{tr}
\DeclareMathOperator{\rank}{rank}
\DeclareMathOperator*{\var}{Var}
\DeclareMathOperator*{\ev}{E}
\DeclareMathOperator{\dg}{deg}
\DeclareMathOperator{\aff}{aff}
\DeclareMathOperator{\conv}{conv}
\DeclareMathOperator{\inte}{int}
\DeclareMathOperator*{\argmin}{argmin}
\DeclareMathOperator*{\argmax}{argmax}
\DeclareMathOperator{\graph}{graph}
\DeclareMathOperator{\sgn}{sgn}
\DeclareMathOperator*{\Rep}{Rep}
\DeclareMathOperator{\Proj}{Proj}
\DeclareMathOperator{\mat}{mat}
\DeclareMathOperator{\diag}{diag}
\DeclareMathOperator{\aut}{Aut}
\DeclareMathOperator{\gal}{Gal}
\DeclareMathOperator{\inn}{Inn}
\DeclareMathOperator{\edm}{End}
\DeclareMathOperator{\Hom}{Hom}
\DeclareMathOperator{\ext}{Ext}
\DeclareMathOperator{\tor}{Tor}
\DeclareMathOperator{\Span}{Span}
\DeclareMathOperator{\Stab}{Stab}
\DeclareMathOperator{\cont}{cont}
\DeclareMathOperator{\Ann}{Ann}
\DeclareMathOperator{\Div}{div}
\DeclareMathOperator{\curl}{curl}
\DeclareMathOperator{\nat}{Nat}
\DeclareMathOperator{\gr}{Gr}
\DeclareMathOperator{\vect}{Vect}
\DeclareMathOperator{\id}{id}
\DeclareMathOperator{\Mod}{Mod}
\DeclareMathOperator{\sign}{sign}
\DeclareMathOperator{\Surf}{Surf}
\DeclareMathOperator{\fcone}{fcone}
\DeclareMathOperator{\Rot}{Rot}
\DeclareMathOperator{\grad}{grad}
\DeclareMathOperator{\atan2}{atan2}
\DeclareMathOperator{\Ric}{Ric}
\let\vec\relax
\DeclareMathOperator{\vec}{vec}
\let\Re\relax
\DeclareMathOperator{\Re}{Re}
\let\Im\relax
\DeclareMathOperator{\Im}{Im}
% Put x \to \infty below \lim
\let\svlim\lim\def\lim{\svlim\limits}

%wide hat
\usepackage{scalerel,stackengine}
\stackMath
\newcommand*\wh[1]{%
\savestack{\tmpbox}{\stretchto{%
  \scaleto{%
    \scalerel*[\widthof{\ensuremath{#1}}]{\kern-.6pt\bigwedge\kern-.6pt}%
    {\rule[-\textheight/2]{1ex}{\textheight}}%WIDTH-LIMITED BIG WEDGE
  }{\textheight}% 
}{0.5ex}}%
\stackon[1pt]{#1}{\tmpbox}%
}
\parskip 1ex

%Make implies and impliedby shorter
\let\implies\Rightarrow
\let\impliedby\Leftarrow
\let\iff\Leftrightarrow
\let\epsilon\varepsilon

% Add \contra symbol to denote contradiction
\usepackage{stmaryrd} % for \lightning
\newcommand\contra{\scalebox{1.5}{$\lightning$}}

% \let\phi\varphi

% Command for short corrections
% Usage: 1+1=\correct{3}{2}

\definecolor{correct}{HTML}{009900}
\newcommand\correct[2]{\ensuremath{\:}{\color{red}{#1}}\ensuremath{\to }{\color{correct}{#2}}\ensuremath{\:}}
\newcommand\green[1]{{\color{correct}{#1}}}

% horizontal rule
\newcommand\hr{
    \noindent\rule[0.5ex]{\linewidth}{0.5pt}
}

% hide parts
\newcommand\hide[1]{}

% si unitx
\usepackage{siunitx}
\sisetup{locale = FR}

%allows pmatrix to stretch
\makeatletter
\renewcommand*\env@matrix[1][\arraystretch]{%
  \edef\arraystretch{#1}%
  \hskip -\arraycolsep
  \let\@ifnextchar\new@ifnextchar
  \array{*\c@MaxMatrixCols c}}
\makeatother

\renewcommand{\arraystretch}{0.8}

\renewcommand{\baselinestretch}{1.5}

\usepackage{graphics}
\usepackage{epstopdf}

\RequirePackage{hyperref}
%%
%% Add support for color in order to color the hyperlinks.
%% 
\hypersetup{
  colorlinks = true,
  urlcolor = blue,
  citecolor = blue
}
%%fakesection Links
\hypersetup{
    colorlinks,
    linkcolor={red!50!black},
    citecolor={green!50!black},
    urlcolor={blue!80!black}
}
%customization of cleveref
\RequirePackage[capitalize,nameinlink]{cleveref}[0.19]

% Per SIAM Style Manual, "section" should be lowercase
\crefname{section}{section}{sections}
\crefname{subsection}{subsection}{subsections}
\Crefname{section}{Section}{Sections}
\Crefname{subsection}{Subsection}{Subsections}

% Per SIAM Style Manual, "Figure" should be spelled out in references
\Crefname{figure}{Figure}{Figures}

% Per SIAM Style Manual, don't say equation in front on an equation.
\crefformat{equation}{\textup{#2(#1)#3}}
\crefrangeformat{equation}{\textup{#3(#1)#4--#5(#2)#6}}
\crefmultiformat{equation}{\textup{#2(#1)#3}}{ and \textup{#2(#1)#3}}
{, \textup{#2(#1)#3}}{, and \textup{#2(#1)#3}}
\crefrangemultiformat{equation}{\textup{#3(#1)#4--#5(#2)#6}}%
{ and \textup{#3(#1)#4--#5(#2)#6}}{, \textup{#3(#1)#4--#5(#2)#6}}{, and \textup{#3(#1)#4--#5(#2)#6}}

% But spell it out at the beginning of a sentence.
\Crefformat{equation}{#2Equation~\textup{(#1)}#3}
\Crefrangeformat{equation}{Equations~\textup{#3(#1)#4--#5(#2)#6}}
\Crefmultiformat{equation}{Equations~\textup{#2(#1)#3}}{ and \textup{#2(#1)#3}}
{, \textup{#2(#1)#3}}{, and \textup{#2(#1)#3}}
\Crefrangemultiformat{equation}{Equations~\textup{#3(#1)#4--#5(#2)#6}}%
{ and \textup{#3(#1)#4--#5(#2)#6}}{, \textup{#3(#1)#4--#5(#2)#6}}{, and \textup{#3(#1)#4--#5(#2)#6}}

% Make number non-italic in any environment.
\crefdefaultlabelformat{#2\textup{#1}#3}

% Environments
\makeatother
% For box around Definition, Theorem, \ldots
%%fakesection Theorems
\usepackage{thmtools}
\usepackage[framemethod=TikZ]{mdframed}

\theoremstyle{definition}
\mdfdefinestyle{mdbluebox}{%
	roundcorner = 10pt,
	linewidth=1pt,
	skipabove=12pt,
	innerbottommargin=9pt,
	skipbelow=2pt,
	nobreak=true,
	linecolor=blue,
	backgroundcolor=TealBlue!5,
}
\declaretheoremstyle[
	headfont=\sffamily\bfseries\color{MidnightBlue},
	mdframed={style=mdbluebox},
	headpunct={\\[3pt]},
	postheadspace={0pt}
]{thmbluebox}

\mdfdefinestyle{mdredbox}{%
	linewidth=0.5pt,
	skipabove=12pt,
	frametitleaboveskip=5pt,
	frametitlebelowskip=0pt,
	skipbelow=2pt,
	frametitlefont=\bfseries,
	innertopmargin=4pt,
	innerbottommargin=8pt,
	nobreak=false,
	linecolor=RawSienna,
	backgroundcolor=Salmon!5,
}
\declaretheoremstyle[
	headfont=\bfseries\color{RawSienna},
	mdframed={style=mdredbox},
	headpunct={\\[3pt]},
	postheadspace={0pt},
]{thmredbox}

\declaretheorem[%
style=thmbluebox,name=Theorem,numberwithin=section]{thm}
\declaretheorem[style=thmbluebox,name=Lemma,sibling=thm]{lem}
\declaretheorem[style=thmbluebox,name=Proposition,sibling=thm]{prop}
\declaretheorem[style=thmbluebox,name=Corollary,sibling=thm]{coro}
\declaretheorem[style=thmredbox,name=Example,sibling=thm]{eg}

\mdfdefinestyle{mdgreenbox}{%
	roundcorner = 10pt,
	linewidth=1pt,
	skipabove=12pt,
	innerbottommargin=9pt,
	skipbelow=2pt,
	nobreak=true,
	linecolor=ForestGreen,
	backgroundcolor=ForestGreen!5,
}

\declaretheoremstyle[
	headfont=\bfseries\sffamily\color{ForestGreen!70!black},
	bodyfont=\normalfont,
	spaceabove=2pt,
	spacebelow=1pt,
	mdframed={style=mdgreenbox},
	headpunct={ --- },
]{thmgreenbox}

\declaretheorem[style=thmgreenbox,name=Definition,sibling=thm]{defn}

\mdfdefinestyle{mdgreenboxsq}{%
	linewidth=1pt,
	skipabove=12pt,
	innerbottommargin=9pt,
	skipbelow=2pt,
	nobreak=true,
	linecolor=ForestGreen,
	backgroundcolor=ForestGreen!5,
}
\declaretheoremstyle[
	headfont=\bfseries\sffamily\color{ForestGreen!70!black},
	bodyfont=\normalfont,
	spaceabove=2pt,
	spacebelow=1pt,
	mdframed={style=mdgreenboxsq},
	headpunct={},
]{thmgreenboxsq}
\declaretheoremstyle[
	headfont=\bfseries\sffamily\color{ForestGreen!70!black},
	bodyfont=\normalfont,
	spaceabove=2pt,
	spacebelow=1pt,
	mdframed={style=mdgreenboxsq},
	headpunct={},
]{thmgreenboxsq*}

\mdfdefinestyle{mdblackbox}{%
	skipabove=8pt,
	linewidth=3pt,
	rightline=false,
	leftline=true,
	topline=false,
	bottomline=false,
	linecolor=black,
	backgroundcolor=RedViolet!5!gray!5,
}
\declaretheoremstyle[
	headfont=\bfseries,
	bodyfont=\normalfont\small,
	spaceabove=0pt,
	spacebelow=0pt,
	mdframed={style=mdblackbox}
]{thmblackbox}

\theoremstyle{plain}
\declaretheorem[name=Question,sibling=thm,style=thmblackbox]{ques}
\declaretheorem[name=Remark,sibling=thm,style=thmgreenboxsq]{remark}
\declaretheorem[name=Remark,sibling=thm,style=thmgreenboxsq*]{remark*}
\newtheorem{ass}[thm]{Assumptions}

\theoremstyle{definition}
\newtheorem*{problem}{Problem}
\newtheorem{claim}[thm]{Claim}
\theoremstyle{remark}
\newtheorem*{case}{Case}
\newtheorem*{notation}{Notation}
\newtheorem*{note}{Note}
\newtheorem*{motivation}{Motivation}
\newtheorem*{intuition}{Intuition}
\newtheorem*{conjecture}{Conjecture}

% Make section starts with 1 for report type
%\renewcommand\thesection{\arabic{section}}

% End example and intermezzo environments with a small diamond (just like proof
% environments end with a small square)
\usepackage{etoolbox}
\AtEndEnvironment{vb}{\null\hfill$\diamond$}%
\AtEndEnvironment{intermezzo}{\null\hfill$\diamond$}%
% \AtEndEnvironment{opmerking}{\null\hfill$\diamond$}%

% Fix some spacing
% http://tex.stackexchange.com/questions/22119/how-can-i-change-the-spacing-before-theorems-with-amsthm
\makeatletter
\def\thm@space@setup{%
  \thm@preskip=\parskip \thm@postskip=0pt
}

% Fix some stuff
% %http://tex.stackexchange.com/questions/76273/multiple-pdfs-with-page-group-included-in-a-single-page-warning
\pdfsuppresswarningpagegroup=1


% My name
\author{Jaden Wang}



\begin{document}
\centerline {\textsf{\textbf{\LARGE{Homework 4}}}}
\centerline {Jaden Wang}
\vspace{.15in}

\begin{problem}[1]
Suppose $|G|= 24= 2^3 \cdot 3$. If $ G$ is abelian, then it is a product of its Sylow subgroups which are normal in $ G$ so  $ G$ cannot be simple. Suppose $ G$ is non-abelian. The only divisor $ n$ where  $ \gcd ( n,24 /n) =1$ are $ n=1,3,8,24$. Since Sylow $ 2$-subgroups have order 8, Sylow 3-subgroups has order 3, and  $ \{e\} $ and $ G$ exist, by Theorem on P105,  $ G$ is solvable (or use Burnside's Theorem). Then by definition there exists a chain for $ G$:
 \begin{align*}
	\{e\} \trianglelefteq G_1 \trianglelefteq \cdots \trianglelefteq   G_{s-1} \trianglelefteq G_s = G 
\end{align*}
where $ G_{i+1} / G_i$ is abelian. Since $ G$ is not abelian but  $ G / G_{s-1}$ is abelian, $ G_{s-1} \neq \{e\} $ so we have found a normal subgroup $ G_{s-1}$ of $ G$.

\allbold{Alternatively} : By Sylow we see that $ n_2 = 1,3$ and $ n_3 = 1,4$. Suppose $ n_2 = 3$ and $ n_3 =4$. $ G$ acts on  $ P := Syl_{ 2}( G) $ by conjugation. This yields a homomorphism $ \phi: G \to S_3$. Notice that $ \phi$ is not the trivial action as the action permutes three Sylow 2-subgroups. Hence  $ |\im \phi| > 1 $ and $ | \ker \phi| = |G| / | \im \phi| \geq 24 / 6 = 4$ but $ | \ker \phi| = |G| / \im \phi < |G|$. Therefore, $ \ker \phi$ is a non-trivial proper normal subgroup of $ G$ so $ G$ is not simple.
\end{problem}

\begin{problem}[2]
~\begin{enumerate}[label=(\alph*)]
	\item The $ (\impliedby)$ direction is immediate as cyclic groups are abelian. $ (\implies):$ suppose $ G$ is finite and solvable. Then by the FToFGAB, any $ G_{i+1} / G_i \cong Z_{p_1}^{ \alpha_1} \times \cdots \times Z_{p_k}^{ \alpha_k}$. We know $ Z_{p_1}^{ \alpha_1-1} \times \cdots \times Z_{p_k}^{ \alpha_k} \trianglelefteq G_{i+1} / G_i$, so by the 4th isomorphism theorem, we can pull back this normal subgroup to a normal subgroup $ G_i' \trianglelefteq G_{i+1}$ with index $ p_1$. Therefore $ G_{i+1} / G_i' \cong Z_{p_1}$. Continue this process inductively on $ G_i' / G_i$ until the quotient is cyclic and do this for all $ i$. This yields a composition series with all cyclic quotients.
	\item First we prove a lemma.
\begin{lem}
A group $ G$ is solvable iff its derived series terminates.
\end{lem}
\begin{proof}
Denote the commutator subgroup of $ G$ as  $ G^{(1)}$, whose commutator subgroup is denoted as $ G^{(2)}$, and so on.

$ (\implies):$ Suppose $ G$ is solvable, with the series
 \begin{align*}
	\{e\} \trianglelefteq N_{s-1} \trianglelefteq \cdots \trianglelefteq N_1 \trianglelefteq N_0 = G.
\end{align*}
Since $ G / N_1$ is abelian, and $ G^{(1)}$ is the smallest normal subgroup that yields abelian quotient, $ G^{(1)} \leq N_1$ so we establish the base case. Assume $ G^{(i-1)} \leq N_{i-1}$. Notice $ G^{(i)} = [G^{(i-1)},G^{(i-1)}] \leq [N_{i-1},N_{i-1}]$. Since $ N_{i-1} / N_i$ is abelian, $[N_{i-1},N_{i-1}] \leq N_i$ so $ G^{(i)} \leq N_i$ and we complete the induction.

$ (\impliedby):$ Suppose the derived series of $ G$ terminates. Since $ G^{(i)} \trianglelefteq G^{(i-1)}$ and $ G^{(i-1)} / G^{(i)}$ is abelian by properties of commutator subgroups, the derived series satisfy the definition of solvable group.
\end{proof}

With this equivalent but more concrete definition of solvable groups, we can give elegant proofs to this part.

Suppose $ G$ is solvable and let $ H \leq G$. We wish to show that the derived series of $ H$ terminates. First notice that intersecting the derived series of $ G$ with $ H$ factorwise yields a terminating series
\begin{align*}
	\{e\} \trianglelefteq H \cap G^{(s-1)} \trianglelefteq \cdots \trianglelefteq H \cap  G^{(1)} \trianglelefteq H 
\end{align*}
Moreover, we see that for every $ h \in H^{(i)}$, clearly $ h \in H$ and $ h \in G^{(i)}$, so $ H^{(i)} \leq H \cap G^{(i)}$. Therefore, we establish factorwise containment of the derived series of $ H$ with the terminating series above. This shows that the derived series of  $ H$ also terminates so  $ H$ is solvable.

\begin{prop}
Let $ G$ be a solvable group and $ \phi:G \to H$ be a group homomorphism, then $\ker \phi$ and $ \im \phi$ are solvable.
\end{prop}
\begin{proof}
	Denote $ N:= \ker \phi$ and $K:= \im \phi \cong G /N$. Since $N \trianglelefteq G $, it is solvable by argument above. Since $ \phi$ is a homomorphism, it maps generators $ [g_1,g_2]$ of $ G^{(1)}$ to $[\phi(g_1),\phi(g_2)]$ which are generators of $ K^{(1)}$. Since $ \phi$ is surjective onto its image $ K$, all generators of $ K^{(1)}$ can be hit this way so we have the base case $K^{(1)} \leq \phi(G^{(1)})$. Assume $ K^{(i-1)} \leq \phi (G^{(i-1)})$. Then
	\begin{align*}
		K^{(i)} = [K^{(i-1)},K^{(i-1)}] \leq [\phi(G^{(i-1)}),\phi(G^{(i-1)})] = \phi([G^{(i-1)},G^{(i-1)}]) = \phi (G^{(i)}).
	\end{align*}
	Hence we complete the induction. Since the factorwise image of derived series of $ G$ terminates, so does the derived series of $ K$. Hence  $ K$ is solvable.
\end{proof}
It follows immediately that for any $ N \trianglelefteq G$, $ G /N$ is solvable.
\end{enumerate}
\end{problem}
\begin{problem}[3]
	(collab with Will and Ari) Suppose  $ |G| = 200 = 2^3 \cdot 5^2$. By Sylow, $ n_5 | 8$ and $ n_5 = 1 \bmod 5$ so $ n_5 = 1$. That is, Sylow 5-subgroup $ P_5$ is normal. Moreover, $ P_5$ has order 25 so it is a  $ p$-group and therefore solvable. Consider $ G / P_5$ which has order $ 8$. It is also a $ p$-group so it is solvable.  By Theorem, $ G$ is therefore solvable.
\end{problem}
\begin{problem}[4]
$ (\implies):$ Suppose every group of odd order is solvable. Then any simple group of odd order $ G$ must be abelian since the only possible solvable series is $ \{e\} \trianglelefteq G $ by simplicity so $ G / \{e\} $ is abelian. Since $ G$ is finite, by the FToFGAB,  $ G \cong Z_{p_1}^{ \alpha_1} \times \cdots \times Z_{p_k}^{ \alpha_k}$. Since $ G$ is simple, it cannot have more than one product and it cannot be trivial so  $ G \cong Z_{p_j}$, where $ p_j$ is a prime. Then $ |G| = p_j$ which forces $ p_j$ to be odd.
BETTER: use 2(a).

$ (\impliedby):$ Suppose that every simple group of odd order is $ Z_p$ for some odd prime  $ p$. Then given a group  $ G$ of odd order and its decomposition series, the quotients  $ G_{i+1} / G_i$ is simple and must have odd order by Lagrange, so $ G_{i+1} / G_i \cong Z_{p_i}$ by assumption, which is abelian. Hence $ G$ is solvable.
\end{problem}
\begin{problem}[5]
~\begin{enumerate}[label=(\alph*)]
	\item First, any $ hk \in HK$ can be expressed as $ hkh^{-1}h = (hkh^{-1})h \in KH$ since $ K \trianglelefteq G$. Likewise any $ kh = hh^{-1}kh \in HK$. So $ KH=HK$. $ HK$ is clearly non-empty. Given  $ h_1k_1,h_2k_2 \in HK$, we see that $ h_1 k_1 h_2 k_2 = (h_1 h_2) (h_2^{-1} k_1 h_2 k_2) \in HK$ since $ h_2 ^{-1} k_1 h_2 \in K$ by normality so $ HK$ is closed under operation. Clearly $ (hk)^{-1} = k^{-1}h^{-1} \in KH = HK$ so it is closed under inverses. Hence $ KH=HK \leq G$.

		This part claims that $ H$ acts on $ K$ by left conjugation. Given $ h_1 , h_2 \in H$, we see that
		\begin{align*}
			 \gamma(h_1 h_2) (k) &= \gamma_{h_1 h_2} (k) \\
			 &= h_2^{-1}h_1^{-1} k h_1 h_2 \\
			 &= h_2 ^{-1} \gamma_{h_1}(k) h_2\\
			 &= \gamma_{h_2} \circ \gamma_{h_1}(k) \\
			 &= \phi(h_1) \circ \phi(h_2) (k) 
		\end{align*}
		So $ \gamma$ is a homomorphism.
	\item Consider the map $ \phi: HK \to H \times K,hk \mapsto (h,hkh^{-1})$. Then
		\begin{align*}
			\phi(h_1 k_1 h_2 k_2) &= \phi(h_1 h_2 h_2^{-1} k_1 h_2 k_2) \\
			&= \phi((h_1h_2)(h_2^{-1}k_1h_2k_2) \\
			&= (h_1h_2,h_2^{-1} k_1 h_2 k_2) && h_2^{-1}k_1 h_2 \in K\\
			&= (h_1, k_1) * (h_2, k_2) \\
			&= \phi(h_1k_1) \phi(h_2k_2)
		\end{align*}
		shows that $ \phi$ is a homomorphism. Surjectivity is clear. Clearly if $ (h,k)=0$, then  $ h,k=0$ so  $ hk=0$. Hence  $ \phi$ is injective and therefore an isomorphism. That is, $HK= G \cong H \times K$.
	\item Closure under operation is clear from definition of the operation. 
		\begin{align*}
			((x_1,y_1) * (x_2,y_2))*(x_3,y_3) &= (x_1x_2,\phi_{x_2^{-1}}(y_1)y_2)*(x_3,y_3) \\
			&= (x_1x_2x_3,\phi_{x_3^{-1}}(\phi_{x_2^{-1}}(y_1)y_2)y_3) \\
			&= (x_1x_2x_3, (\phi_{x_3^{-1}} \circ \phi_{x_2^{-1}}) (y_1) \phi_{x_3^{-1}}(y_2)y_3) \\
			&= (x_1x_2x_3, (\phi_{(x_2x_3)^{-1} }) (y_1) \phi_{x_3^{-1}}(y_2)y_3) \\
			&= (x_1,y_1)*(x_2x_3,\phi_{x_3^{-1}}(y_2)y_3) \\
			&=(x_1,y_1) * ((x_2,y_2)*(x_3,y_3))
		\end{align*}

		I claim that $ (e_H,e_K)$ is the identity. Not surprisingly, $ (h,k)*(e_H,e_K) = (he_H,\phi_{e_H^{-1}} (k) e_K  ) = (h, \phi_{e_H} (k) ) = (h,k)$. The other direction is similar. Finally, $ (h^{-1},\phi_h(k^{-1}))$ is the inverse of $ (h,k)$:
		\begin{align*}
			(h,k)(h^{-1},\phi_{h}(k^{-1})) &= (hh^{-1},\phi_{h}(k) \phi_h(k^{-1}) \\
			&= (e_H,\phi_h(k k ^{-1})) \\
			&= (e_H,\phi_h(e_K) )\\
			&= (e_H,e_K) 
		\end{align*}
		And the other direction is similar. Hence $ G$ is a group.
	\item Given $ (h,k) \in G,(e,k') \in K$, notice
		\begin{align*}
			(h,k)*(e,k')*(h,k)^{-1} &= (h,kk') * (h^{-1},\phi_h(k^{-1})) \\
			&= (e,\phi_h(kk'k^{-1}) \in K
		\end{align*}
		So $ K \trianglelefteq G$. Clearly $ H \cap K = (e,e)$. Then by counting, $ |HK| = |H||K| /|H \cap K| = |H||K| = |H \times K|$ so $ G = HK$.
\end{enumerate}
\end{problem}
\begin{problem}[6]
Clearly $ G \cong G \times \{e\} \leq H$. Also $ G \times Z_2 \cong Z_2 \times G$. Denote $ e_i$ to be $ 1$ in the  $ i$th entry and zero elsewhere. Then clearly $ e_i$ for all $ i \in \nn_+$ are the generators of $ G$ and  $ H$. Define $ \phi:Z_2 \times G \to G, (1,0,0,\ldots)) \mapsto (2,0,0,\ldots)$ and then $ e_i \mapsto e_{i+1}$ for $ i >1$. Then $ \phi$ is clearly a homomorphism. It is also injective as none of the generators is mapped to 0 so the kernel is trivial. Hence  $ H \cong \phi(H) \leq G$. However, $ H$ and  $ G$ are not isomorphic because  $ G$ has no generators of order 2, but  $ H$ does. This is a structural difference.
\end{problem}
\end{document}
