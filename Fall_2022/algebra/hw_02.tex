\documentclass[12pt]{article}
%Fall 2022
% Some basic packages
\usepackage{standalone}[subpreambles=true]
\usepackage[utf8]{inputenc}
\usepackage[T1]{fontenc}
\usepackage{textcomp}
\usepackage[english]{babel}
\usepackage{url}
\usepackage{graphicx}
%\usepackage{quiver}
\usepackage{float}
\usepackage{enumitem}
\usepackage{lmodern}
\usepackage{comment}
\usepackage{hyperref}
\usepackage[usenames,svgnames,dvipsnames]{xcolor}
\usepackage[margin=1in]{geometry}
\usepackage{pdfpages}

\pdfminorversion=7

% Don't indent paragraphs, leave some space between them
\usepackage{parskip}

% Hide page number when page is empty
\usepackage{emptypage}
\usepackage{subcaption}
\usepackage{multicol}
\usepackage[b]{esvect}

% Math stuff
\usepackage{amsmath, amsfonts, mathtools, amsthm, amssymb}
\usepackage{bbm}
\usepackage{stmaryrd}
\allowdisplaybreaks

% Fancy script capitals
\usepackage{mathrsfs}
\usepackage{cancel}
% Bold math
\usepackage{bm}
% Some shortcuts
\newcommand{\rr}{\ensuremath{\mathbb{R}}}
\newcommand{\zz}{\ensuremath{\mathbb{Z}}}
\newcommand{\qq}{\ensuremath{\mathbb{Q}}}
\newcommand{\nn}{\ensuremath{\mathbb{N}}}
\newcommand{\ff}{\ensuremath{\mathbb{F}}}
\newcommand{\cc}{\ensuremath{\mathbb{C}}}
\newcommand{\ee}{\ensuremath{\mathbb{E}}}
\newcommand{\hh}{\ensuremath{\mathbb{H}}}
\renewcommand\O{\ensuremath{\emptyset}}
\newcommand{\norm}[1]{{\left\lVert{#1}\right\rVert}}
\newcommand{\dbracket}[1]{{\left\llbracket{#1}\right\rrbracket}}
\newcommand{\ve}[1]{{\bm{#1}}}
\newcommand\allbold[1]{{\boldmath\textbf{#1}}}
\DeclareMathOperator{\lcm}{lcm}
\DeclareMathOperator{\im}{im}
\DeclareMathOperator{\coim}{coim}
\DeclareMathOperator{\dom}{dom}
\DeclareMathOperator{\tr}{tr}
\DeclareMathOperator{\rank}{rank}
\DeclareMathOperator*{\var}{Var}
\DeclareMathOperator*{\ev}{E}
\DeclareMathOperator{\dg}{deg}
\DeclareMathOperator{\aff}{aff}
\DeclareMathOperator{\conv}{conv}
\DeclareMathOperator{\inte}{int}
\DeclareMathOperator*{\argmin}{argmin}
\DeclareMathOperator*{\argmax}{argmax}
\DeclareMathOperator{\graph}{graph}
\DeclareMathOperator{\sgn}{sgn}
\DeclareMathOperator*{\Rep}{Rep}
\DeclareMathOperator{\Proj}{Proj}
\DeclareMathOperator{\mat}{mat}
\DeclareMathOperator{\diag}{diag}
\DeclareMathOperator{\aut}{Aut}
\DeclareMathOperator{\gal}{Gal}
\DeclareMathOperator{\inn}{Inn}
\DeclareMathOperator{\edm}{End}
\DeclareMathOperator{\Hom}{Hom}
\DeclareMathOperator{\ext}{Ext}
\DeclareMathOperator{\tor}{Tor}
\DeclareMathOperator{\Span}{Span}
\DeclareMathOperator{\Stab}{Stab}
\DeclareMathOperator{\cont}{cont}
\DeclareMathOperator{\Ann}{Ann}
\DeclareMathOperator{\Div}{div}
\DeclareMathOperator{\curl}{curl}
\DeclareMathOperator{\nat}{Nat}
\DeclareMathOperator{\gr}{Gr}
\DeclareMathOperator{\vect}{Vect}
\DeclareMathOperator{\id}{id}
\DeclareMathOperator{\Mod}{Mod}
\DeclareMathOperator{\sign}{sign}
\DeclareMathOperator{\Surf}{Surf}
\DeclareMathOperator{\fcone}{fcone}
\DeclareMathOperator{\Rot}{Rot}
\DeclareMathOperator{\grad}{grad}
\DeclareMathOperator{\atan2}{atan2}
\DeclareMathOperator{\Ric}{Ric}
\let\vec\relax
\DeclareMathOperator{\vec}{vec}
\let\Re\relax
\DeclareMathOperator{\Re}{Re}
\let\Im\relax
\DeclareMathOperator{\Im}{Im}
% Put x \to \infty below \lim
\let\svlim\lim\def\lim{\svlim\limits}

%wide hat
\usepackage{scalerel,stackengine}
\stackMath
\newcommand*\wh[1]{%
\savestack{\tmpbox}{\stretchto{%
  \scaleto{%
    \scalerel*[\widthof{\ensuremath{#1}}]{\kern-.6pt\bigwedge\kern-.6pt}%
    {\rule[-\textheight/2]{1ex}{\textheight}}%WIDTH-LIMITED BIG WEDGE
  }{\textheight}% 
}{0.5ex}}%
\stackon[1pt]{#1}{\tmpbox}%
}
\parskip 1ex

%Make implies and impliedby shorter
\let\implies\Rightarrow
\let\impliedby\Leftarrow
\let\iff\Leftrightarrow
\let\epsilon\varepsilon

% Add \contra symbol to denote contradiction
\usepackage{stmaryrd} % for \lightning
\newcommand\contra{\scalebox{1.5}{$\lightning$}}

% \let\phi\varphi

% Command for short corrections
% Usage: 1+1=\correct{3}{2}

\definecolor{correct}{HTML}{009900}
\newcommand\correct[2]{\ensuremath{\:}{\color{red}{#1}}\ensuremath{\to }{\color{correct}{#2}}\ensuremath{\:}}
\newcommand\green[1]{{\color{correct}{#1}}}

% horizontal rule
\newcommand\hr{
    \noindent\rule[0.5ex]{\linewidth}{0.5pt}
}

% hide parts
\newcommand\hide[1]{}

% si unitx
\usepackage{siunitx}
\sisetup{locale = FR}

%allows pmatrix to stretch
\makeatletter
\renewcommand*\env@matrix[1][\arraystretch]{%
  \edef\arraystretch{#1}%
  \hskip -\arraycolsep
  \let\@ifnextchar\new@ifnextchar
  \array{*\c@MaxMatrixCols c}}
\makeatother

\renewcommand{\arraystretch}{0.8}

\renewcommand{\baselinestretch}{1.5}

\usepackage{graphics}
\usepackage{epstopdf}

\RequirePackage{hyperref}
%%
%% Add support for color in order to color the hyperlinks.
%% 
\hypersetup{
  colorlinks = true,
  urlcolor = blue,
  citecolor = blue
}
%%fakesection Links
\hypersetup{
    colorlinks,
    linkcolor={red!50!black},
    citecolor={green!50!black},
    urlcolor={blue!80!black}
}
%customization of cleveref
\RequirePackage[capitalize,nameinlink]{cleveref}[0.19]

% Per SIAM Style Manual, "section" should be lowercase
\crefname{section}{section}{sections}
\crefname{subsection}{subsection}{subsections}
\Crefname{section}{Section}{Sections}
\Crefname{subsection}{Subsection}{Subsections}

% Per SIAM Style Manual, "Figure" should be spelled out in references
\Crefname{figure}{Figure}{Figures}

% Per SIAM Style Manual, don't say equation in front on an equation.
\crefformat{equation}{\textup{#2(#1)#3}}
\crefrangeformat{equation}{\textup{#3(#1)#4--#5(#2)#6}}
\crefmultiformat{equation}{\textup{#2(#1)#3}}{ and \textup{#2(#1)#3}}
{, \textup{#2(#1)#3}}{, and \textup{#2(#1)#3}}
\crefrangemultiformat{equation}{\textup{#3(#1)#4--#5(#2)#6}}%
{ and \textup{#3(#1)#4--#5(#2)#6}}{, \textup{#3(#1)#4--#5(#2)#6}}{, and \textup{#3(#1)#4--#5(#2)#6}}

% But spell it out at the beginning of a sentence.
\Crefformat{equation}{#2Equation~\textup{(#1)}#3}
\Crefrangeformat{equation}{Equations~\textup{#3(#1)#4--#5(#2)#6}}
\Crefmultiformat{equation}{Equations~\textup{#2(#1)#3}}{ and \textup{#2(#1)#3}}
{, \textup{#2(#1)#3}}{, and \textup{#2(#1)#3}}
\Crefrangemultiformat{equation}{Equations~\textup{#3(#1)#4--#5(#2)#6}}%
{ and \textup{#3(#1)#4--#5(#2)#6}}{, \textup{#3(#1)#4--#5(#2)#6}}{, and \textup{#3(#1)#4--#5(#2)#6}}

% Make number non-italic in any environment.
\crefdefaultlabelformat{#2\textup{#1}#3}

% Environments
\makeatother
% For box around Definition, Theorem, \ldots
%%fakesection Theorems
\usepackage{thmtools}
\usepackage[framemethod=TikZ]{mdframed}

\theoremstyle{definition}
\mdfdefinestyle{mdbluebox}{%
	roundcorner = 10pt,
	linewidth=1pt,
	skipabove=12pt,
	innerbottommargin=9pt,
	skipbelow=2pt,
	nobreak=true,
	linecolor=blue,
	backgroundcolor=TealBlue!5,
}
\declaretheoremstyle[
	headfont=\sffamily\bfseries\color{MidnightBlue},
	mdframed={style=mdbluebox},
	headpunct={\\[3pt]},
	postheadspace={0pt}
]{thmbluebox}

\mdfdefinestyle{mdredbox}{%
	linewidth=0.5pt,
	skipabove=12pt,
	frametitleaboveskip=5pt,
	frametitlebelowskip=0pt,
	skipbelow=2pt,
	frametitlefont=\bfseries,
	innertopmargin=4pt,
	innerbottommargin=8pt,
	nobreak=false,
	linecolor=RawSienna,
	backgroundcolor=Salmon!5,
}
\declaretheoremstyle[
	headfont=\bfseries\color{RawSienna},
	mdframed={style=mdredbox},
	headpunct={\\[3pt]},
	postheadspace={0pt},
]{thmredbox}

\declaretheorem[%
style=thmbluebox,name=Theorem,numberwithin=section]{thm}
\declaretheorem[style=thmbluebox,name=Lemma,sibling=thm]{lem}
\declaretheorem[style=thmbluebox,name=Proposition,sibling=thm]{prop}
\declaretheorem[style=thmbluebox,name=Corollary,sibling=thm]{coro}
\declaretheorem[style=thmredbox,name=Example,sibling=thm]{eg}

\mdfdefinestyle{mdgreenbox}{%
	roundcorner = 10pt,
	linewidth=1pt,
	skipabove=12pt,
	innerbottommargin=9pt,
	skipbelow=2pt,
	nobreak=true,
	linecolor=ForestGreen,
	backgroundcolor=ForestGreen!5,
}

\declaretheoremstyle[
	headfont=\bfseries\sffamily\color{ForestGreen!70!black},
	bodyfont=\normalfont,
	spaceabove=2pt,
	spacebelow=1pt,
	mdframed={style=mdgreenbox},
	headpunct={ --- },
]{thmgreenbox}

\declaretheorem[style=thmgreenbox,name=Definition,sibling=thm]{defn}

\mdfdefinestyle{mdgreenboxsq}{%
	linewidth=1pt,
	skipabove=12pt,
	innerbottommargin=9pt,
	skipbelow=2pt,
	nobreak=true,
	linecolor=ForestGreen,
	backgroundcolor=ForestGreen!5,
}
\declaretheoremstyle[
	headfont=\bfseries\sffamily\color{ForestGreen!70!black},
	bodyfont=\normalfont,
	spaceabove=2pt,
	spacebelow=1pt,
	mdframed={style=mdgreenboxsq},
	headpunct={},
]{thmgreenboxsq}
\declaretheoremstyle[
	headfont=\bfseries\sffamily\color{ForestGreen!70!black},
	bodyfont=\normalfont,
	spaceabove=2pt,
	spacebelow=1pt,
	mdframed={style=mdgreenboxsq},
	headpunct={},
]{thmgreenboxsq*}

\mdfdefinestyle{mdblackbox}{%
	skipabove=8pt,
	linewidth=3pt,
	rightline=false,
	leftline=true,
	topline=false,
	bottomline=false,
	linecolor=black,
	backgroundcolor=RedViolet!5!gray!5,
}
\declaretheoremstyle[
	headfont=\bfseries,
	bodyfont=\normalfont\small,
	spaceabove=0pt,
	spacebelow=0pt,
	mdframed={style=mdblackbox}
]{thmblackbox}

\theoremstyle{plain}
\declaretheorem[name=Question,sibling=thm,style=thmblackbox]{ques}
\declaretheorem[name=Remark,sibling=thm,style=thmgreenboxsq]{remark}
\declaretheorem[name=Remark,sibling=thm,style=thmgreenboxsq*]{remark*}
\newtheorem{ass}[thm]{Assumptions}

\theoremstyle{definition}
\newtheorem*{problem}{Problem}
\newtheorem{claim}[thm]{Claim}
\theoremstyle{remark}
\newtheorem*{case}{Case}
\newtheorem*{notation}{Notation}
\newtheorem*{note}{Note}
\newtheorem*{motivation}{Motivation}
\newtheorem*{intuition}{Intuition}
\newtheorem*{conjecture}{Conjecture}

% Make section starts with 1 for report type
%\renewcommand\thesection{\arabic{section}}

% End example and intermezzo environments with a small diamond (just like proof
% environments end with a small square)
\usepackage{etoolbox}
\AtEndEnvironment{vb}{\null\hfill$\diamond$}%
\AtEndEnvironment{intermezzo}{\null\hfill$\diamond$}%
% \AtEndEnvironment{opmerking}{\null\hfill$\diamond$}%

% Fix some spacing
% http://tex.stackexchange.com/questions/22119/how-can-i-change-the-spacing-before-theorems-with-amsthm
\makeatletter
\def\thm@space@setup{%
  \thm@preskip=\parskip \thm@postskip=0pt
}

% Fix some stuff
% %http://tex.stackexchange.com/questions/76273/multiple-pdfs-with-page-group-included-in-a-single-page-warning
\pdfsuppresswarningpagegroup=1


% My name
\author{Jaden Wang}



\begin{document}
\centerline {\textsf{\textbf{\LARGE{Homework 2}}}}
\centerline {Jaden Wang}
\vspace{.15in}

\begin{problem}[1]
Since $ G /Z(G)$ is cyclic, let $a Z(G)$ be its generator. Given $ g,h \in G$, we know that $ g = a^{i} x$ and $ h = z^{j} y$ for some $ i,j \in \nn$ and $ x,y \in Z(G)$. Hence,
\begin{align*}
	gh = a^{i}xa^{j}y = a^{i}a^{j}xy= a^{i+j} yx= a^{j} y a^{i}x = hg.
\end{align*}
\end{problem}

\begin{problem}[2]
~\begin{enumerate}[label=(\alph*)]
	\item Define the set map $ \phi_g: G /H \to G /H, g'H \mapsto g g'H $. We wish to show that $ \Phi: G \to \aut_{|G /H|} \cong S_{|G /H|}, g \mapsto \phi_g$ is a group homomorphism.
		\begin{align*}
			\Phi(g_1g_2)(g'H) &= \phi_{g_1 g_2}(g'H) \\
			&= g_1 g_2 g' H \\
			&= \phi_{g_1}(g_2 g' H) \\
			&= \phi_{g_1} \phi_{g_2}(g'H)\\
			&= \Phi(g_1) \Phi(g_2) (g'H) 
		\end{align*}
	Hence $ \Phi(g_1 g_2) = \Phi(g_1) \Phi(g_2)$.
\item Given $ g_1 H$ and $ g_2 H$, we see that $g_2 H = g_2 g_1 ^{-1} g_1H = \Phi(g_2 g_1 ^{-1})(g_1 H)$ so the action is transitive. Moreover, $ \Stab_{ G}( H) = \{g \in G: gH = H\} = \{g \in H\} =H$.
\item Clearly $ K= \bigcap_{ g' \in G} \Stab_{ G}( g'H) \subseteq \Stab_{ G}( H) = H$. Since $ K = \ker \Phi$, it is a normal subgroup of $ G$. 
\item  Consider the cosets $ G /H$ which has order  $ 4$. Then the kernel $ K$ of the action  $ \Phi: G \to S_4$ is the largest normal subgroup contained in $ H$ by part c. Since $ K \trianglelefteq H$, by order consideration $ |K| = 1,5,7$ or  $ 35$ so $ |G /K| = 140,28,20$ or $4$. Since $ \im \Phi \leq S_4$, $ |\im \Phi| | 24$ so it is $ 1,2,3,4,6,12$ or  $ 24$. By the first isomorphism theorem, $ |G /K| = \im \Phi$ so they must equal 4. But this implies that $ |K| = |H|$ so  $ K = H$. It follows that  $ H \trianglelefteq G$.
\end{enumerate}
\end{problem}
\begin{problem}[3]
~\begin{enumerate}[label=(\alph*)]
	\item Given $ a,b \in A$, since the action is transitive, there exists a $ g \in G$ s.t.\ $ b=g.a$. I claim that $H_b = gH_ag^{-1} $. Given $ h \in H_b$, then
		\begin{align*}
			h.b&= b \\
			h.(g.a)  &= g.a \\
			(g^{-1}hg) .a &= a
		\end{align*}
		Since $ H \trianglelefteq G$, $ g^{-1}hg \in H$. So $ g^{-1} h g \in H_a$ and $ h = g g^{-1} h g g^{-1}$. Given $ g h'g^{-1} \in gH_a g^{-1}$, we know since $ H$ is normal, $ gh'g^{-1} \in H$.
		\begin{align*}
			h'.a &= a \\ 
			(g^{-1} g h'g^{-1} g).a &= a \\
			(gh'g^{-1}).(g.a) &= g.a \\
			(gh'g^{-1}).b &= b
		\end{align*}
		So $ gh'g^{-1} \in H_b$ and the equality follows. Since action of $ g$ by conjugation on $ G$ is an automorphism of $ G$, the action restricted to  $ H_a$ is still a bijection so  $ |H_b| = |gH_ag^{-1}|$. Since $ a,b$ are arbitrary, we show that all stablizers of element in $ A$ have the same cardinality. By the Orbit-Stablizer Theorem, the orbits also have the same cardinality $| \mathcal{ O} |= |H:H_a|$.
	\item It is easy to see that $ H \cap G_a \subseteq H_a$. Given $ h \in H_a$, clearly $ h \in H$ and since $ h.a=a$, $ h \in G_a$. So $ h \in H \cap G_a$ and therefore $ H_a = H \cap G_a$. It follows that $ | \mathcal{ O}| = |H:H_a| = |H:H \cap G_a|$.
\item Suppose the number of orbits of $ H$ on  $ A$ is $ n$. Since $ |A|$ is finite, and the orbits partition  $ A$, $ n$ is finite and  $ |H:H \cap G_a|$ is finite. Thus $ |HG_a: G_a| = |H: H \cap G_a|$ is also finite. But since $ |HG_a| \leq |H||G_a|$ we have
		\begin{align*}
			n &= \frac{|A|}{| \mathcal{ O}| } \\
			&= \frac{|A|}{|H: H \cap G_a| } \\
			&= \frac{|G.a|}{|H: H \cap G_a| } \\
			&= \frac{|G:G_a|}{|H:H \cap G_a| } \\
			&= \frac{|G: G_a|}{|HG_a : G_a| } \\
			&= |G:HG_a| && \text{ 4th iso for cosets} 
		\end{align*}
ALTER (using HW1.8 lemma):
\begin{align*}
	\frac{|G:G_a|}{|H:H \cap G_a| } &= \frac{|G:HG_a||HG_a:G_a|}{|H : H \cap G_a| } \\
					&= |G:HG_a| && \text{ 2nd iso} 
\end{align*}
\end{enumerate}
\end{problem}

\begin{problem}[4]
~\begin{enumerate}[label=(\alph*)]
	\item We know that $ \gcd (|N|,|G:N|) = 1$. Now suppose there exists a $ H \leq G$  s.t.\ $ |H| = |N|$. Since  $ N \trianglelefteq G$, $ HN \leq G$. 
		 \begin{align*}
			|N| &= \frac{|G|}{|G:N|} \\
			&= \frac{|G:HN||HN|}{ |G:N|} \\
			&= \frac{|G:HN||H||N|}{ |G:N||H \cap N|} \\
			&= \frac{|G:HN|}{ |G:N|} \frac{|N|^2}{ |H \cap N|} 
		\end{align*}
		Since $ |N|$ is an integer, all denominators in the expression must vanish. Since  $ |G:N|$ and  $ |N|$ are coprime, this implies that  $ |G:N| | |G:HN|$. But we also have  $ |G:N|=|G:HN||HN:N| $ so $ |G:HN|||G:N|$. Since these are positive integers,  $ |G:N| = |G:HN|$ which forces $ HN = N$. It follows that $ H \leq N$ and therefore  $ H=N$.

		Alternatively (collab with Ari), we know that since $ NH \leq G$, $ |NH|||G|$. Hence
		\begin{align*}
			\frac{|NH|}{|N|} &\bigg| \frac{ |G|}{ |N|} \\
			\frac{|H|}{|H \cap N|}&\bigg| |G:N| \\
			\frac{|N|}{|H \cap N|}&\bigg| |G:N| 
		\end{align*}
		Since $ \frac{|N|}{|H \cap N|} = |N:H \cap N|$ also divides $ |N|$ by Lagrange, $ \frac{|N|}{|H \cap N|}$ divides $ \gcd ( |N|, |G:N|) =1$. It must be that $ \frac{|N|}{|H \cap N|} = 1$ so $ |N| = |H \cap N|$ which implies $ N = H \cap N$. Hence $ H \leq N$ and thus  $ H = N$.
	\item We have $ \gcd ( |H|,|G:H|) =1$ and $ N \trianglelefteq G$. By the second isomorphism theorem, $ |N: H \cap N| = |NH: H|$. Since $ |G:H| = |G:NH||NH:H|$, and  $ |H \cap N|$ divides $ |H|$ by Lagrange, $ \gcd ( |H \cap N|, |N: H \cap N|) $ must divide $ \gcd ( |H|,|G:H|)=1 $ which forces it to be 1 as well. Hence $ H \cap N$ is a Hall subgroup of $ N$.

		By the third isomorphism theorem, $ |G /N : NH /N| = |G:NH|$ which divides $ |G:H|$. Also $ |NH /N| = |NH:N|$. And again by the third isomorphism theorem, $ |NH:N| = |H : H \cap N|$ which divides $ |H|$. So again we have $ \gcd ( |NH/N|, |G:NH|)$ dividing $ \gcd ( |H|,|G:H|) =1$ which yields that $ NH /N$ is a Hall subgroup of  $ G /N$.
\end{enumerate}
\end{problem}

\begin{problem}[5]
~\begin{enumerate}[label=(\alph*)]
	\item We know that $ S_n$ is generated by successively increasing transpositions (Exercise 3.5.3). So it suffices to show that $ (1,2)$ and  $ (1,2,3,\ldots,n)$ generate all such transpositions. We know that conjugating $ (1,2)$ by  $ (1,2,\ldots,n)$ just becomes $ (2,3)$. Then conjugating  $ (2,3)$ by  $ (1,2,\ldots,n)$ yields $ (3,4)$. Repeat until we get  $ (n-1,n)$ and that's all the generators we need for  $ S_n$.
	\item It suffices to show that we can obtain $ (1,2)$ from any transposition $ (i,j)$ with $ 1 \leq i <j \leq p$ and $ (1,2,\ldots,p)$. Since $ p$ is prime, powers ($ <p$) of the  $ p$-cycle remains a  $ p$-cycle. Moreover, each power permutes the last letter in the cycle so we can always obtain some power (in fact, $ (1-i) \bmod p$) that looks like $ (1,\ldots,i)$. Conjugating $ (i,j)$ by this yields  $ (1,j)$. Moreover,  $ (1,2,\ldots,p)^{(2-j) \bmod p}$ should put $ 2$ immediately after  $ j$. Conjugating $ (1,j)$ by this yields  $ (1,2)$.
	\item No. We see that in  $ S_4$, $ (1,4)$  cannot be generated by  $ (2,4)$ and  $ (1,2,3,4)$.
\end{enumerate}
\end{problem}

\begin{problem}[6]
Given $ k \in K$, we know that $ |K| = |S_n : \Stab_{ S_n}( \{k\} ) |$ by Orbit-Stablizer Theorem. We also know from 3b that $ \Stab_{ A_n}( \{k\} ) = H \cap \Stab_{ S_n}( \{k\} ) $. Let the conjugacy class of $ k$ acted by  $ A_n$ be $ K'$. $ |K'| = |A_n: \Stab_{ A_n}( \{k\} )| = |A_n: A_n \cap \Stab_{ S_n}( \{k\})|$. Recall that $ A_n \trianglelefteq S_n$ so $A_n \Stab_{ S_n}(\{k\}  ) $ is a subgroup of $ S_n$ containing $ A_n$. Since $ A_n$ is maximal, $ A_n \Stab_{ S_n}( \{k\} ) $ either equals $ A_n$ or $ S_n$. If it is $ A_n$, then by the second isomorphism theorem, $ |K'| = |A_n : A_n \cap \Stab_{ S_n}( \{k\} ) | = |A_n \Stab_{ S_n}( \{k\} ): \Stab_{ S_n}( \{k\} ) | = |S_n: \Stab_{ S_n}( \{k\} ) | /|S_n: A_n| = |K|/2$. Since $ k$ is arbitrary, picking another element not in  $ K'$ yields another orbit of size $ |K|/2$ and that is all of  $ K$. So we have two orbits of equal size in this case. If  $ A_n \Stab_{ S_n}( \{k\} ) = S_n$, then following the computation above, we obtain that $ |K'| = |K|$ so there is only one orbit.
\end{problem}

\begin{problem}[7]
Every group has the identity conjugacy class $ \{e\} $. Let $ g$ be the representative of the other conjugacy class. By the class equation, $ |G| = 1+ |G:C_G(g)|$. Since  $ |G:C_G(g)|$ divides $ |G|$, we have that $ n(|G|-1) = |G|$ for some $ n \in \nn$. That is,
\begin{align*}
	n|G|-n &= |G| \\
	(n-1)|G|&= n \\
	|G|&= \frac{n}{n-1}=1+ \frac{1}{n-1} 
\end{align*}
Since $ |G| \in \nn$, it's easy to see that $ n=2$ is the unique solution. Therefore,  $ |G|=2$ so  $ G \cong \zz_2$.
\end{problem}

\begin{problem}[8]
	Let $ S = \{(a_1,\ldots,a_p): a_i \in G, a_1 \cdots a_p = e\}$. Consider the set map $\phi: S \to G^{p-1}, (a_1,\ldots,a_p) \mapsto (a_1,\ldots,a_{p-1})$ by dropping the last entry. Surjectivity is clear. If $ \phi(a_1,\ldots,a_p) = (a_1,\ldots,a_{p-1}) = (b_1,\ldots,b_{p-1}) = \phi(b_1,\ldots,b_{p})$, then $ a_i = b_i$ for all $ 1\leq i <p$, and  $ a_p = b_p = (a_1 \cdots a_{p-1})^{-1}$. So $ \phi$ is injective. Hence $ |S| = |G|^{p-1}$. Let $ Z$ be the set of fixed points of $ S$ from $ \zz / p\zz$ action (by shifting indices). It's easy to see that $ Z = \{(a,\ldots,a): a \in G, a^{p}=e\}$. But since there are exactly $ n$ elements of order  $ p$ in  $ G$, together with  $ e$ we have exactly  $ n+1$ elements in  $ Z$. Since $ \zz/p \zz$ is a $ p$-group, by the lemma from class,
	 \begin{align*}
		|S| \equiv |Z| \bmod p.
	\end{align*}
	But since order of any element divides the order of group,  $ p||G|$ so  $ p| |G|^{p-1} = |S|$. Therefore, $ p$ must also divide $ |Z| = n+1$.
\end{problem}
\end{document}
