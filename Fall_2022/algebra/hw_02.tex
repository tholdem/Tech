\documentclass[12pt]{article}
\newcommand{\alert}[1]{{\bf \color{red} [Alert:] #1}}
\newcommand{\todo}[1]{{\bf \color{orange} [TODO:] #1}}
\newcommand{\real}[1][]{\mathbb{R}^{#1}}
\newcommand{\myeqn}[1]{(\ref{#1})}
\newcommand{\myex}[1]{Example \ref{#1}}
\newcommand{\defeq}{\stackrel{\mathrm{def}}{=}}
\newcommand{\parder}[2]{\frac{\partial #1}{\partial #2}}
\newcommand{\Lie}[3][]{\mathsf{L}_{#3}^{#1} #2}
\newcommand{\LieA}[1]{\mathsf{Lie}(#1)}
\newcommand{\lieder}[2]{\mathcal{L}_{#2} #1}
\renewcommand{\t}{^{\mbox{\tiny\sf T}}}
\newcommand{\trans}{^{\mbox{\tiny\sf T}}}
\newcommand{\markup}[1]{\{\textbf{#1}\}}
\newcommand{\msub}[1]{_\mathrm{#1}}
\newcommand{\msup}[1]{^\mathrm{#1}}
\newcommand{\inv}[1]{#1^{-1}}
\newcommand{\pinv}[1]{{#1}^{+}}
\newcommand{\myfracA}[2]{\displaystyle{\frac{#1}{#2}}}
\newcommand{\myfracB}[2]{{#1}/{#2}}
\newcommand{\mydiffA}[1]{\dot{#1}}
\newcommand{\mydiffB}[2]{\myfracA{\mathrm{d}{#1}}{\mathrm{d}{#2}}}
\newcommand{\ball}[2]{\mathcal{B}_{#1}\left(#2\right)}
\newcommand{\acos}[1]{\cos^{-1}\left(#1\right)}
\newcommand{\asin}[1]{\sin^{-1}\left(#1\right)}
\newcommand{\mani}{\mathcal{M}}
\newcommand{\tang}[2]{\mathsf{T}_{#1} #2}
\newcommand{\LieB}[2]{[ #1, #2 ]}
\newcommand{\LieBad}[3][]{\mathsf{ad}_{#2}^{#1} #3}
\newcommand{\ReachVT}{\mathcal{R}^V_T}
\newcommand{\ReachVt}{\mathcal{R}^V_t}
\newcommand{\ReachVTe}{\mathcal{R}^V_{\le T}}
\newcommand{\ReachT}{\mathcal{R}_T}
\newcommand{\Reacht}{\mathcal{R}_t}
\newcommand{\ReachTe}{\mathcal{R}_{\le T}}
\newcommand{\accLA}[1]{\mathsf{Lie}(#1)}
\newcommand{\accD}{\Delta_{\mathcal{F}}}
\newcommand{\accSA}{\mathsf{Lie}(\mathcal{G},f)}
\newcommand{\accDS}{\Delta_{\mathcal{G}}}
\newcommand{\eval}[3]{\mathsf{Ev}^{#2}_{#1}\left( #3 \right)}
\newcommand{\stlc}{\textsc{stlc}}
\newcommand{\clf}{\textsc{clf}}
\newcommand{\jqlf}{\textsc{jqlf}}
\newcommand{\dlas}{\textsc{dlas}}
\newcommand{\Ad}[2]{\mathsf{Ad}_{#1} #2}
\newcommand{\xe}{\ensuremath{x_e}}
\newcommand{\lebg}[1]{\mathcal{L}_{#1}}
\newcommand{\lebgx}[1]{\mathcal{L}_{#1 \mathrm{e}}}
\newcommand{\dom}{D}
\newcommand{\domT}{[t_0,\infty) \times D}
\newcommand{\rarrow}{\rightarrow}
\renewcommand{\d}{\mathrm{d}}
\renewcommand{\Re}{\mathbb{R}}
\newcommand{\C}{\mathrm{C}}

\newcommand{\QED}{{\unskip\nobreak\hfil\penalty50\hskip2em\vadjust{}
		\nobreak\hfil$\Box$\parfillskip=0pt\finalhyphendemerits=0\par}\vspace{0.1cm}}
\newcommand{\eoEx}{{\unskip\nobreak\hfil\penalty50\hskip0em\vadjust{}
		\nobreak\hfil$\Large\Diamond$\parfillskip=0pt\finalhyphendemerits=0\par}\vspace{0.1cm}}

\newcommand{\sgn}{\ensuremath{\operatorname{sgn}}}
\newcommand{\sat}{\ensuremath{\operatorname{sat}}}

\newcommand{\half}{\frac{1}{2}}
\newcommand{\shalf}{\mbox{$\frac{1}{2}$}}
\newcommand{\marcom}[1]{\marginpar{\footnotesize #1}}
\newcommand{\der}{\mathrm{D}}
\newcommand{\e}{\mathrm{e}}
\newcommand{\dt}{\mathrm{d}t}

\newcommand{\cA}{\ensuremath{\mathcal{A}}}
\newcommand{\cB}{\ensuremath{\mathcal{B}}}
\newcommand{\cG}{\ensuremath{\mathcal{G}}}
\newcommand{\cK}{\ensuremath{\mathcal{K}}}
\newcommand{\cW}{\ensuremath{\mathcal{W}}}
\newcommand{\cZ}{\ensuremath{\mathcal{Z}}}
\newcommand{\cS}{\ensuremath{\mathcal{S}}}
\newcommand{\cD}{\ensuremath{\mathcal{D}}}
\newcommand{\cP}{\ensuremath{\mathcal{P}}}
\newcommand{\cV}{\ensuremath{\mathcal{V}}}
\newcommand{\cL}{\ensuremath{\mathcal{L}}}
\newcommand{\cN}{\ensuremath{\mathcal{N}}}
\newcommand{\cI}{\ensuremath{\mathcal{I}}}
\newcommand{\cR}{\ensuremath{\mathcal{R}}}
\newcommand{\cM}{\ensuremath{\mathcal{M}}}
\newcommand{\cC}{\ensuremath{\mathcal{C}}}
\newcommand{\cF}{\ensuremath{\mathcal{F}}}
\newcommand{\cH}{\ensuremath{\mathcal{H}}}
\newcommand{\cO}{\ensuremath{\mathcal{O}}}
\newcommand{\cX}{\ensuremath{\mathcal{X}}}
\newcommand{\cY}{\ensuremath{\mathcal{Y}}}
\newcommand{\Ci}{\ensuremath{\mathcal{C}^\infty}}
\newcommand{\ISS}{\textsc{iss}}
\newcommand{\LISS}{\textsc{liss}}
\newcommand{\GAS}{\textsc{gas}}
\newcommand{\GS}{\textsc{gs}}
\newcommand{\LES}{\textsc{les}}
\newcommand{\GUAS}{\textsc{guas}}
\newcommand{\BIBO}{\textsc{bibo}}
\newcommand{\spec}{\ensuremath{\operatorname{spec}}}
\newcommand{\spn}{\ensuremath{\operatorname{span}}}
\renewcommand{\i}{\mathrm{i\,}}

\renewcommand{\implies}{\Rightarrow}

\renewcommand{\theenumi}{$\roman{enumi})$}
\renewcommand{\labelenumi}{\theenumi}

\font\ptmten=zptmcmrm scaled 1200
\newcommand{\w}{\mbox{{\ptmten w}}}
\newcommand{\z}{\mbox{{\ptmten z}}}
\renewcommand{\Re}{\mathbb{R}}

\newcommand{\cl}{\operatorname{cl}}
\newcommand{\intr}{\operatorname{int}}
\newcommand{\rank}{\operatorname{rank}}
\newcommand{\co}{\operatorname{co}}
\newcommand{\aff}{\operatorname{aff}}

\theoremstyle{plain}
\newtheorem{theorem}{Theorem}[chapter]
\newtheorem{claim}[theorem]{Claim}
\newtheorem{corollary}[theorem]{Corollary}
\newtheorem{prop}[theorem]{Proposition}
\newtheorem{fact}[theorem]{Fact}
\newtheorem{lemma}[theorem]{Lemma}

\newtheorem{remark}{Remark}[chapter]

\theoremstyle{definition}
\newtheorem{assume}[theorem]{Assumption}
\newtheorem{defn}[theorem]{Definition}
\newtheorem{problem}[theorem]{Problem}
\newtheorem{exercise}{Exercise}
\newtheorem{example}[theorem]{Example}


\begin{document}
\centerline {\textsf{\textbf{\LARGE{Homework 2}}}}
\centerline {Jaden Wang}
\vspace{.15in}

\begin{problem}[1]
Since $ G /Z(G)$ is cyclic, let $a Z(G)$ be its generator. Given $ g,h \in G$, we know that $ g = a^{i} x$ and $ h = z^{j} y$ for some $ i,j \in \nn$ and $ x,y \in Z(G)$. Hence,
\begin{align*}
	gh = a^{i}xa^{j}y = a^{i}a^{j}xy= a^{i+j} yx= a^{j} y a^{i}x = hg.
\end{align*}
\end{problem}

\begin{problem}[2]
~\begin{enumerate}[label=(\alph*)]
	\item Define the set map $ \phi_g: G /H \to G /H, g'H \mapsto g g'H $. We wish to show that $ \Phi: G \to \aut_{|G /H|} \cong S_{|G /H|}, g \mapsto \phi_g$ is a group homomorphism.
		\begin{align*}
			\Phi(g_1g_2)(g'H) &= \phi_{g_1 g_2}(g'H) \\
			&= g_1 g_2 g' H \\
			&= \phi_{g_1}(g_2 g' H) \\
			&= \phi_{g_1} \phi_{g_2}(g'H)\\
			&= \Phi(g_1) \Phi(g_2) (g'H) 
		\end{align*}
	Hence $ \Phi(g_1 g_2) = \Phi(g_1) \Phi(g_2)$.
\item Given $ g_1 H$ and $ g_2 H$, we see that $g_2 H = g_2 g_1 ^{-1} g_1H = \Phi(g_2 g_1 ^{-1})(g_1 H)$ so the action is transitive. Moreover, $ \Stab_{ G}( H) = \{g \in G: gH = H\} = \{g \in H\} =H$.
\item Clearly $ K= \bigcap_{ g' \in G} \Stab_{ G}( g'H) \subseteq \Stab_{ G}( H) = H$. Since $ K = \ker \Phi$, it is a normal subgroup of $ G$. 
\item  Consider the cosets $ G /H$ which has order  $ 4$. Then the kernel $ K$ of the action  $ \Phi: G \to S_4$ is the largest normal subgroup contained in $ H$ by part c. Since $ K \trianglelefteq H$, by order consideration $ |K| = 1,5,7$ or  $ 35$ so $ |G /K| = 140,28,20$ or $4$. Since $ \im \Phi \leq S_4$, $ |\im \Phi| | 24$ so it is $ 1,2,3,4,6,12$ or  $ 24$. By the first isomorphism theorem, $ |G /K| = \im \Phi$ so they must equal 4. But this implies that $ |K| = |H|$ so  $ K = H$. It follows that  $ H \trianglelefteq G$.
\end{enumerate}
\end{problem}
\begin{problem}[3]
~\begin{enumerate}[label=(\alph*)]
	\item Given $ a,b \in A$, since the action is transitive, there exists a $ g \in G$ s.t.\ $ b=g.a$. I claim that $H_b = gH_ag^{-1} $. Given $ h \in H_b$, then
		\begin{align*}
			h.b&= b \\
			h.(g.a)  &= g.a \\
			(g^{-1}hg) .a &= a
		\end{align*}
		Since $ H \trianglelefteq G$, $ g^{-1}hg \in H$. So $ g^{-1} h g \in H_a$ and $ h = g g^{-1} h g g^{-1}$. Given $ g h'g^{-1} \in gH_a g^{-1}$, we know since $ H$ is normal, $ gh'g^{-1} \in H$.
		\begin{align*}
			h'.a &= a \\ 
			(g^{-1} g h'g^{-1} g).a &= a \\
			(gh'g^{-1}).(g.a) &= g.a \\
			(gh'g^{-1}).b &= b
		\end{align*}
		So $ gh'g^{-1} \in H_b$ and the equality follows. Since action of $ g$ by conjugation on $ G$ is an automorphism of $ G$, the action restricted to  $ H_a$ is still a bijection so  $ |H_b| = |gH_ag^{-1}|$. Since $ a,b$ are arbitrary, we show that all stablizers of element in $ A$ have the same cardinality. By the Orbit-Stablizer Theorem, the orbits also have the same cardinality $| \mathcal{ O} |= |H:H_a|$.
	\item It is easy to see that $ H \cap G_a \subseteq H_a$. Given $ h \in H_a$, clearly $ h \in H$ and since $ h.a=a$, $ h \in G_a$. So $ h \in H \cap G_a$ and therefore $ H_a = H \cap G_a$. It follows that $ | \mathcal{ O}| = |H:H_a| = |H:H \cap G_a|$.
\item Suppose the number of orbits of $ H$ on  $ A$ is $ n$. Since $ |A|$ is finite, and the orbits partition  $ A$, $ n$ is finite and  $ |H:H \cap G_a|$ is finite. Thus $ |HG_a: G_a| = |H: H \cap G_a|$ is also finite. But since $ |HG_a| \leq |H||G_a|$ we have
		\begin{align*}
			n &= \frac{|A|}{| \mathcal{ O}| } \\
			&= \frac{|A|}{|H: H \cap G_a| } \\
			&= \frac{|G.a|}{|H: H \cap G_a| } \\
			&= \frac{|G:G_a|}{|H:H \cap G_a| } \\
			&= \frac{|G: G_a|}{|HG_a : G_a| } \\
			&= |G:HG_a| && \text{ 4th iso for cosets} 
		\end{align*}
ALTER (using HW1.8 lemma):
\begin{align*}
	\frac{|G:G_a|}{|H:H \cap G_a| } &= \frac{|G:HG_a||HG_a:G_a|}{|H : H \cap G_a| } \\
					&= |G:HG_a| && \text{ 2nd iso} 
\end{align*}
\end{enumerate}
\end{problem}

\begin{problem}[4]
~\begin{enumerate}[label=(\alph*)]
	\item We know that $ \gcd (|N|,|G:N|) = 1$. Now suppose there exists a $ H \leq G$  s.t.\ $ |H| = |N|$. Since  $ N \trianglelefteq G$, $ HN \leq G$. 
		 \begin{align*}
			|N| &= \frac{|G|}{|G:N|} \\
			&= \frac{|G:HN||HN|}{ |G:N|} \\
			&= \frac{|G:HN||H||N|}{ |G:N||H \cap N|} \\
			&= \frac{|G:HN|}{ |G:N|} \frac{|N|^2}{ |H \cap N|} 
		\end{align*}
		Since $ |N|$ is an integer, all denominators in the expression must vanish. Since  $ |G:N|$ and  $ |N|$ are coprime, this implies that  $ |G:N| | |G:HN|$. But we also have  $ |G:N|=|G:HN||HN:N| $ so $ |G:HN|||G:N|$. Since these are positive integers,  $ |G:N| = |G:HN|$ which forces $ HN = N$. It follows that $ H \leq N$ and therefore  $ H=N$.

		Alternatively (collab with Ari), we know that since $ NH \leq G$, $ |NH|||G|$. Hence
		\begin{align*}
			\frac{|NH|}{|N|} &\bigg| \frac{ |G|}{ |N|} \\
			\frac{|H|}{|H \cap N|}&\bigg| |G:N| \\
			\frac{|N|}{|H \cap N|}&\bigg| |G:N| 
		\end{align*}
		Since $ \frac{|N|}{|H \cap N|} = |N:H \cap N|$ also divides $ |N|$ by Lagrange, $ \frac{|N|}{|H \cap N|}$ divides $ \gcd ( |N|, |G:N|) =1$. It must be that $ \frac{|N|}{|H \cap N|} = 1$ so $ |N| = |H \cap N|$ which implies $ N = H \cap N$. Hence $ H \leq N$ and thus  $ H = N$.
	\item We have $ \gcd ( |H|,|G:H|) =1$ and $ N \trianglelefteq G$. By the second isomorphism theorem, $ |N: H \cap N| = |NH: H|$. Since $ |G:H| = |G:NH||NH:H|$, and  $ |H \cap N|$ divides $ |H|$ by Lagrange, $ \gcd ( |H \cap N|, |N: H \cap N|) $ must divide $ \gcd ( |H|,|G:H|)=1 $ which forces it to be 1 as well. Hence $ H \cap N$ is a Hall subgroup of $ N$.

		By the third isomorphism theorem, $ |G /N : NH /N| = |G:NH|$ which divides $ |G:H|$. Also $ |NH /N| = |NH:N|$. And again by the third isomorphism theorem, $ |NH:N| = |H : H \cap N|$ which divides $ |H|$. So again we have $ \gcd ( |NH/N|, |G:NH|)$ dividing $ \gcd ( |H|,|G:H|) =1$ which yields that $ NH /N$ is a Hall subgroup of  $ G /N$.
\end{enumerate}
\end{problem}

\begin{problem}[5]
~\begin{enumerate}[label=(\alph*)]
	\item We know that $ S_n$ is generated by successively increasing transpositions (Exercise 3.5.3). So it suffices to show that $ (1,2)$ and  $ (1,2,3,\ldots,n)$ generate all such transpositions. We know that conjugating $ (1,2)$ by  $ (1,2,\ldots,n)$ just becomes $ (2,3)$. Then conjugating  $ (2,3)$ by  $ (1,2,\ldots,n)$ yields $ (3,4)$. Repeat until we get  $ (n-1,n)$ and that's all the generators we need for  $ S_n$.
	\item It suffices to show that we can obtain $ (1,2)$ from any transposition $ (i,j)$ with $ 1 \leq i <j \leq p$ and $ (1,2,\ldots,p)$. Since $ p$ is prime, powers ($ <p$) of the  $ p$-cycle remains a  $ p$-cycle. Moreover, each power permutes the last letter in the cycle so we can always obtain some power (in fact, $ (1-i) \bmod p$) that looks like $ (1,\ldots,i)$. Conjugating $ (i,j)$ by this yields  $ (1,j)$. Moreover,  $ (1,2,\ldots,p)^{(2-j) \bmod p}$ should put $ 2$ immediately after  $ j$. Conjugating $ (1,j)$ by this yields  $ (1,2)$.
	\item No. We see that in  $ S_4$, $ (1,4)$  cannot be generated by  $ (2,4)$ and  $ (1,2,3,4)$.
\end{enumerate}
\end{problem}

\begin{problem}[6]
Given $ k \in K$, we know that $ |K| = |S_n : \Stab_{ S_n}( \{k\} ) |$ by Orbit-Stablizer Theorem. We also know from 3b that $ \Stab_{ A_n}( \{k\} ) = H \cap \Stab_{ S_n}( \{k\} ) $. Let the conjugacy class of $ k$ acted by  $ A_n$ be $ K'$. $ |K'| = |A_n: \Stab_{ A_n}( \{k\} )| = |A_n: A_n \cap \Stab_{ S_n}( \{k\})|$. Recall that $ A_n \trianglelefteq S_n$ so $A_n \Stab_{ S_n}(\{k\}  ) $ is a subgroup of $ S_n$ containing $ A_n$. Since $ A_n$ is maximal, $ A_n \Stab_{ S_n}( \{k\} ) $ either equals $ A_n$ or $ S_n$. If it is $ A_n$, then by the second isomorphism theorem, $ |K'| = |A_n : A_n \cap \Stab_{ S_n}( \{k\} ) | = |A_n \Stab_{ S_n}( \{k\} ): \Stab_{ S_n}( \{k\} ) | = |S_n: \Stab_{ S_n}( \{k\} ) | /|S_n: A_n| = |K|/2$. Since $ k$ is arbitrary, picking another element not in  $ K'$ yields another orbit of size $ |K|/2$ and that is all of  $ K$. So we have two orbits of equal size in this case. If  $ A_n \Stab_{ S_n}( \{k\} ) = S_n$, then following the computation above, we obtain that $ |K'| = |K|$ so there is only one orbit.
\end{problem}

\begin{problem}[7]
Every group has the identity conjugacy class $ \{e\} $. Let $ g$ be the representative of the other conjugacy class. By the class equation, $ |G| = 1+ |G:C_G(g)|$. Since  $ |G:C_G(g)|$ divides $ |G|$, we have that $ n(|G|-1) = |G|$ for some $ n \in \nn$. That is,
\begin{align*}
	n|G|-n &= |G| \\
	(n-1)|G|&= n \\
	|G|&= \frac{n}{n-1}=1+ \frac{1}{n-1} 
\end{align*}
Since $ |G| \in \nn$, it's easy to see that $ n=2$ is the unique solution. Therefore,  $ |G|=2$ so  $ G \cong \zz_2$.
\end{problem}

\begin{problem}[8]
	Let $ S = \{(a_1,\ldots,a_p): a_i \in G, a_1 \cdots a_p = e\}$. Consider the set map $\phi: S \to G^{p-1}, (a_1,\ldots,a_p) \mapsto (a_1,\ldots,a_{p-1})$ by dropping the last entry. Surjectivity is clear. If $ \phi(a_1,\ldots,a_p) = (a_1,\ldots,a_{p-1}) = (b_1,\ldots,b_{p-1}) = \phi(b_1,\ldots,b_{p})$, then $ a_i = b_i$ for all $ 1\leq i <p$, and  $ a_p = b_p = (a_1 \cdots a_{p-1})^{-1}$. So $ \phi$ is injective. Hence $ |S| = |G|^{p-1}$. Let $ Z$ be the set of fixed points of $ S$ from $ \zz / p\zz$ action (by shifting indices). It's easy to see that $ Z = \{(a,\ldots,a): a \in G, a^{p}=e\}$. But since there are exactly $ n$ elements of order  $ p$ in  $ G$, together with  $ e$ we have exactly  $ n+1$ elements in  $ Z$. Since $ \zz/p \zz$ is a $ p$-group, by the lemma from class,
	 \begin{align*}
		|S| \equiv |Z| \bmod p.
	\end{align*}
	But since order of any element divides the order of group,  $ p||G|$ so  $ p| |G|^{p-1} = |S|$. Therefore, $ p$ must also divide $ |Z| = n+1$.
\end{problem}
\end{document}
