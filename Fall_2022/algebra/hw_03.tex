\documentclass[12pt]{article}
%Fall 2022
% Some basic packages
\usepackage{standalone}[subpreambles=true]
\usepackage[utf8]{inputenc}
\usepackage[T1]{fontenc}
\usepackage{textcomp}
\usepackage[english]{babel}
\usepackage{url}
\usepackage{graphicx}
%\usepackage{quiver}
\usepackage{float}
\usepackage{enumitem}
\usepackage{lmodern}
\usepackage{comment}
\usepackage{hyperref}
\usepackage[usenames,svgnames,dvipsnames]{xcolor}
\usepackage[margin=1in]{geometry}
\usepackage{pdfpages}

\pdfminorversion=7

% Don't indent paragraphs, leave some space between them
\usepackage{parskip}

% Hide page number when page is empty
\usepackage{emptypage}
\usepackage{subcaption}
\usepackage{multicol}
\usepackage[b]{esvect}

% Math stuff
\usepackage{amsmath, amsfonts, mathtools, amsthm, amssymb}
\usepackage{bbm}
\usepackage{stmaryrd}
\allowdisplaybreaks

% Fancy script capitals
\usepackage{mathrsfs}
\usepackage{cancel}
% Bold math
\usepackage{bm}
% Some shortcuts
\newcommand{\rr}{\ensuremath{\mathbb{R}}}
\newcommand{\zz}{\ensuremath{\mathbb{Z}}}
\newcommand{\qq}{\ensuremath{\mathbb{Q}}}
\newcommand{\nn}{\ensuremath{\mathbb{N}}}
\newcommand{\ff}{\ensuremath{\mathbb{F}}}
\newcommand{\cc}{\ensuremath{\mathbb{C}}}
\newcommand{\ee}{\ensuremath{\mathbb{E}}}
\newcommand{\hh}{\ensuremath{\mathbb{H}}}
\renewcommand\O{\ensuremath{\emptyset}}
\newcommand{\norm}[1]{{\left\lVert{#1}\right\rVert}}
\newcommand{\dbracket}[1]{{\left\llbracket{#1}\right\rrbracket}}
\newcommand{\ve}[1]{{\bm{#1}}}
\newcommand\allbold[1]{{\boldmath\textbf{#1}}}
\DeclareMathOperator{\lcm}{lcm}
\DeclareMathOperator{\im}{im}
\DeclareMathOperator{\coim}{coim}
\DeclareMathOperator{\dom}{dom}
\DeclareMathOperator{\tr}{tr}
\DeclareMathOperator{\rank}{rank}
\DeclareMathOperator*{\var}{Var}
\DeclareMathOperator*{\ev}{E}
\DeclareMathOperator{\dg}{deg}
\DeclareMathOperator{\aff}{aff}
\DeclareMathOperator{\conv}{conv}
\DeclareMathOperator{\inte}{int}
\DeclareMathOperator*{\argmin}{argmin}
\DeclareMathOperator*{\argmax}{argmax}
\DeclareMathOperator{\graph}{graph}
\DeclareMathOperator{\sgn}{sgn}
\DeclareMathOperator*{\Rep}{Rep}
\DeclareMathOperator{\Proj}{Proj}
\DeclareMathOperator{\mat}{mat}
\DeclareMathOperator{\diag}{diag}
\DeclareMathOperator{\aut}{Aut}
\DeclareMathOperator{\gal}{Gal}
\DeclareMathOperator{\inn}{Inn}
\DeclareMathOperator{\edm}{End}
\DeclareMathOperator{\Hom}{Hom}
\DeclareMathOperator{\ext}{Ext}
\DeclareMathOperator{\tor}{Tor}
\DeclareMathOperator{\Span}{Span}
\DeclareMathOperator{\Stab}{Stab}
\DeclareMathOperator{\cont}{cont}
\DeclareMathOperator{\Ann}{Ann}
\DeclareMathOperator{\Div}{div}
\DeclareMathOperator{\curl}{curl}
\DeclareMathOperator{\nat}{Nat}
\DeclareMathOperator{\gr}{Gr}
\DeclareMathOperator{\vect}{Vect}
\DeclareMathOperator{\id}{id}
\DeclareMathOperator{\Mod}{Mod}
\DeclareMathOperator{\sign}{sign}
\DeclareMathOperator{\Surf}{Surf}
\DeclareMathOperator{\fcone}{fcone}
\DeclareMathOperator{\Rot}{Rot}
\DeclareMathOperator{\grad}{grad}
\DeclareMathOperator{\atan2}{atan2}
\DeclareMathOperator{\Ric}{Ric}
\let\vec\relax
\DeclareMathOperator{\vec}{vec}
\let\Re\relax
\DeclareMathOperator{\Re}{Re}
\let\Im\relax
\DeclareMathOperator{\Im}{Im}
% Put x \to \infty below \lim
\let\svlim\lim\def\lim{\svlim\limits}

%wide hat
\usepackage{scalerel,stackengine}
\stackMath
\newcommand*\wh[1]{%
\savestack{\tmpbox}{\stretchto{%
  \scaleto{%
    \scalerel*[\widthof{\ensuremath{#1}}]{\kern-.6pt\bigwedge\kern-.6pt}%
    {\rule[-\textheight/2]{1ex}{\textheight}}%WIDTH-LIMITED BIG WEDGE
  }{\textheight}% 
}{0.5ex}}%
\stackon[1pt]{#1}{\tmpbox}%
}
\parskip 1ex

%Make implies and impliedby shorter
\let\implies\Rightarrow
\let\impliedby\Leftarrow
\let\iff\Leftrightarrow
\let\epsilon\varepsilon

% Add \contra symbol to denote contradiction
\usepackage{stmaryrd} % for \lightning
\newcommand\contra{\scalebox{1.5}{$\lightning$}}

% \let\phi\varphi

% Command for short corrections
% Usage: 1+1=\correct{3}{2}

\definecolor{correct}{HTML}{009900}
\newcommand\correct[2]{\ensuremath{\:}{\color{red}{#1}}\ensuremath{\to }{\color{correct}{#2}}\ensuremath{\:}}
\newcommand\green[1]{{\color{correct}{#1}}}

% horizontal rule
\newcommand\hr{
    \noindent\rule[0.5ex]{\linewidth}{0.5pt}
}

% hide parts
\newcommand\hide[1]{}

% si unitx
\usepackage{siunitx}
\sisetup{locale = FR}

%allows pmatrix to stretch
\makeatletter
\renewcommand*\env@matrix[1][\arraystretch]{%
  \edef\arraystretch{#1}%
  \hskip -\arraycolsep
  \let\@ifnextchar\new@ifnextchar
  \array{*\c@MaxMatrixCols c}}
\makeatother

\renewcommand{\arraystretch}{0.8}

\renewcommand{\baselinestretch}{1.5}

\usepackage{graphics}
\usepackage{epstopdf}

\RequirePackage{hyperref}
%%
%% Add support for color in order to color the hyperlinks.
%% 
\hypersetup{
  colorlinks = true,
  urlcolor = blue,
  citecolor = blue
}
%%fakesection Links
\hypersetup{
    colorlinks,
    linkcolor={red!50!black},
    citecolor={green!50!black},
    urlcolor={blue!80!black}
}
%customization of cleveref
\RequirePackage[capitalize,nameinlink]{cleveref}[0.19]

% Per SIAM Style Manual, "section" should be lowercase
\crefname{section}{section}{sections}
\crefname{subsection}{subsection}{subsections}
\Crefname{section}{Section}{Sections}
\Crefname{subsection}{Subsection}{Subsections}

% Per SIAM Style Manual, "Figure" should be spelled out in references
\Crefname{figure}{Figure}{Figures}

% Per SIAM Style Manual, don't say equation in front on an equation.
\crefformat{equation}{\textup{#2(#1)#3}}
\crefrangeformat{equation}{\textup{#3(#1)#4--#5(#2)#6}}
\crefmultiformat{equation}{\textup{#2(#1)#3}}{ and \textup{#2(#1)#3}}
{, \textup{#2(#1)#3}}{, and \textup{#2(#1)#3}}
\crefrangemultiformat{equation}{\textup{#3(#1)#4--#5(#2)#6}}%
{ and \textup{#3(#1)#4--#5(#2)#6}}{, \textup{#3(#1)#4--#5(#2)#6}}{, and \textup{#3(#1)#4--#5(#2)#6}}

% But spell it out at the beginning of a sentence.
\Crefformat{equation}{#2Equation~\textup{(#1)}#3}
\Crefrangeformat{equation}{Equations~\textup{#3(#1)#4--#5(#2)#6}}
\Crefmultiformat{equation}{Equations~\textup{#2(#1)#3}}{ and \textup{#2(#1)#3}}
{, \textup{#2(#1)#3}}{, and \textup{#2(#1)#3}}
\Crefrangemultiformat{equation}{Equations~\textup{#3(#1)#4--#5(#2)#6}}%
{ and \textup{#3(#1)#4--#5(#2)#6}}{, \textup{#3(#1)#4--#5(#2)#6}}{, and \textup{#3(#1)#4--#5(#2)#6}}

% Make number non-italic in any environment.
\crefdefaultlabelformat{#2\textup{#1}#3}

% Environments
\makeatother
% For box around Definition, Theorem, \ldots
%%fakesection Theorems
\usepackage{thmtools}
\usepackage[framemethod=TikZ]{mdframed}

\theoremstyle{definition}
\mdfdefinestyle{mdbluebox}{%
	roundcorner = 10pt,
	linewidth=1pt,
	skipabove=12pt,
	innerbottommargin=9pt,
	skipbelow=2pt,
	nobreak=true,
	linecolor=blue,
	backgroundcolor=TealBlue!5,
}
\declaretheoremstyle[
	headfont=\sffamily\bfseries\color{MidnightBlue},
	mdframed={style=mdbluebox},
	headpunct={\\[3pt]},
	postheadspace={0pt}
]{thmbluebox}

\mdfdefinestyle{mdredbox}{%
	linewidth=0.5pt,
	skipabove=12pt,
	frametitleaboveskip=5pt,
	frametitlebelowskip=0pt,
	skipbelow=2pt,
	frametitlefont=\bfseries,
	innertopmargin=4pt,
	innerbottommargin=8pt,
	nobreak=false,
	linecolor=RawSienna,
	backgroundcolor=Salmon!5,
}
\declaretheoremstyle[
	headfont=\bfseries\color{RawSienna},
	mdframed={style=mdredbox},
	headpunct={\\[3pt]},
	postheadspace={0pt},
]{thmredbox}

\declaretheorem[%
style=thmbluebox,name=Theorem,numberwithin=section]{thm}
\declaretheorem[style=thmbluebox,name=Lemma,sibling=thm]{lem}
\declaretheorem[style=thmbluebox,name=Proposition,sibling=thm]{prop}
\declaretheorem[style=thmbluebox,name=Corollary,sibling=thm]{coro}
\declaretheorem[style=thmredbox,name=Example,sibling=thm]{eg}

\mdfdefinestyle{mdgreenbox}{%
	roundcorner = 10pt,
	linewidth=1pt,
	skipabove=12pt,
	innerbottommargin=9pt,
	skipbelow=2pt,
	nobreak=true,
	linecolor=ForestGreen,
	backgroundcolor=ForestGreen!5,
}

\declaretheoremstyle[
	headfont=\bfseries\sffamily\color{ForestGreen!70!black},
	bodyfont=\normalfont,
	spaceabove=2pt,
	spacebelow=1pt,
	mdframed={style=mdgreenbox},
	headpunct={ --- },
]{thmgreenbox}

\declaretheorem[style=thmgreenbox,name=Definition,sibling=thm]{defn}

\mdfdefinestyle{mdgreenboxsq}{%
	linewidth=1pt,
	skipabove=12pt,
	innerbottommargin=9pt,
	skipbelow=2pt,
	nobreak=true,
	linecolor=ForestGreen,
	backgroundcolor=ForestGreen!5,
}
\declaretheoremstyle[
	headfont=\bfseries\sffamily\color{ForestGreen!70!black},
	bodyfont=\normalfont,
	spaceabove=2pt,
	spacebelow=1pt,
	mdframed={style=mdgreenboxsq},
	headpunct={},
]{thmgreenboxsq}
\declaretheoremstyle[
	headfont=\bfseries\sffamily\color{ForestGreen!70!black},
	bodyfont=\normalfont,
	spaceabove=2pt,
	spacebelow=1pt,
	mdframed={style=mdgreenboxsq},
	headpunct={},
]{thmgreenboxsq*}

\mdfdefinestyle{mdblackbox}{%
	skipabove=8pt,
	linewidth=3pt,
	rightline=false,
	leftline=true,
	topline=false,
	bottomline=false,
	linecolor=black,
	backgroundcolor=RedViolet!5!gray!5,
}
\declaretheoremstyle[
	headfont=\bfseries,
	bodyfont=\normalfont\small,
	spaceabove=0pt,
	spacebelow=0pt,
	mdframed={style=mdblackbox}
]{thmblackbox}

\theoremstyle{plain}
\declaretheorem[name=Question,sibling=thm,style=thmblackbox]{ques}
\declaretheorem[name=Remark,sibling=thm,style=thmgreenboxsq]{remark}
\declaretheorem[name=Remark,sibling=thm,style=thmgreenboxsq*]{remark*}
\newtheorem{ass}[thm]{Assumptions}

\theoremstyle{definition}
\newtheorem*{problem}{Problem}
\newtheorem{claim}[thm]{Claim}
\theoremstyle{remark}
\newtheorem*{case}{Case}
\newtheorem*{notation}{Notation}
\newtheorem*{note}{Note}
\newtheorem*{motivation}{Motivation}
\newtheorem*{intuition}{Intuition}
\newtheorem*{conjecture}{Conjecture}

% Make section starts with 1 for report type
%\renewcommand\thesection{\arabic{section}}

% End example and intermezzo environments with a small diamond (just like proof
% environments end with a small square)
\usepackage{etoolbox}
\AtEndEnvironment{vb}{\null\hfill$\diamond$}%
\AtEndEnvironment{intermezzo}{\null\hfill$\diamond$}%
% \AtEndEnvironment{opmerking}{\null\hfill$\diamond$}%

% Fix some spacing
% http://tex.stackexchange.com/questions/22119/how-can-i-change-the-spacing-before-theorems-with-amsthm
\makeatletter
\def\thm@space@setup{%
  \thm@preskip=\parskip \thm@postskip=0pt
}

% Fix some stuff
% %http://tex.stackexchange.com/questions/76273/multiple-pdfs-with-page-group-included-in-a-single-page-warning
\pdfsuppresswarningpagegroup=1


% My name
\author{Jaden Wang}



\begin{document}
\centerline {\textsf{\textbf{\LARGE{Homework 3}}}}
\centerline {Jaden Wang}
\vspace{.15in}
\begin{problem}[1]
Since $ G$ acts transitively, the orbit of any  $ a \in A$ is all of $ A$, so  $ |A| = |G:G_a|$. Since  $ |A|>1$, the index of  $ G_a$ is at least 2. Recall that for any $ b=g_b.a$ for some $ g_b \in G$, $ G_b = g_b G_a g_b^{-1}$. Let $S := \bigcup_{ b \in A} g_b G_a g_b ^{-1} \subseteq G$. By exercise 2.1.8, the union of subgroups is a subgroup iff they are contained in one of them. If the union is indeed a subgroup, then $S= G_b$ for some  $ b \in A$. Since the index of $ G_b$ is at least 2, there exists some  $ g \in G \setminus G_b$ which doesn't fix any point by definition of union. If $ S$ is not a subgroup, then since  $ G$ is a group, clearly  $ S \subsetneq G$ so there exists a $ g \in G\setminus S$.
\end{problem}
\begin{problem}[2]
~\begin{enumerate}[label=(\alph*)]
	\item First we consider all types of symmetries of a cube. As we did in class, there are 24 of them.
		\begin{table}[H]
			\centering
			\caption{Cube}
			\begin{tabular}{c|c|c|c}
				type& cycle decomp & \# cycles & elements of this type\\
				\hline
				\hline
				id&(1)(2)(3)(4)(5)(6)(7)(8)&8&1\\
				\hline
				90 upright rot & (1234)(5678)&2& 6\\
				\hline
				180 upright rot & (13)(24)(57)(68)&4&3 \\
				\hline
				120 diag rot & (123)(456)(7)(8)&4&8\\
				\hline
				180 semidiag rot & (17)(28)(34)(56)&4&6
			\end{tabular}
		\end{table}
		So by Burnside, the total distinct ways are
		\begin{align*}
			\frac{1}{24}(k^{8}+6k^2 + 3 k^{4}+8k^{4}+6k^{4}) = \frac{1}{24} (k^{8}+ 17k^{4}+6k^2)
		\end{align*}
	\item 
		\begin{table}[H]
			\centering
			\caption{Tetrahedron}
			\begin{tabular}{c|c|c|c}
				type& cycle decomp & \# cycles & elements of this type\\
				\hline
				\hline
				id&(1)(2)(3)(4)&4&1\\
				\hline
				120 upright rot & (123)(4)&2& 8\\
				\hline
				grand rot & (12)(34)&2& 3\\
			\end{tabular}
		\end{table}
		So total distinct ways are
		\begin{align*}
			\frac{1}{12} (k^{4}+11k^{2})
		\end{align*}
\end{enumerate}
\end{problem}
\begin{problem}[3]
Consider $ S_n$ acting on itself by conjugation. Notice that two elements $ \sigma, \tau$ commute iff $ \sigma \tau \sigma^{-1} = \tau$, that is, $ \tau$ is a fix point of the action of $ \sigma$ by conjugation. Therefore, the number of fixed points of $ \sigma$ action is the same as the number of elements commuting with $ \sigma$. By Burnside lemma,
\begin{align*}
	\# \mathcal{ O} = \frac{1}{|S_n|} \sum_{ \sigma \in S_n} f( \sigma)
\end{align*}
But since conjugacy classes of $ S_n$ are simply distinct cycle types, $ | \mathcal{ O}| = p(n)$. Also notice that the probability $ p_ \sigma$ of fixing an  $ \sigma$ and picking another element commuting with it is precisely $ \frac{ f(\sigma)}{|S_n|}$. Since we assume naive probability for picking elements, the probability for picking two such elements is just the average probability of fixing one and picking another. That is, 
\begin{align*}
p = \frac{1}{|S_n|} \sum_{ \sigma \in S_n} p_ \sigma = \frac{1}{n!} \sum_{ \sigma \in S_n} \frac{f( \sigma)}{|G| } = \frac{p(n)}{ n!}.
\end{align*}
\end{problem}

\begin{problem}[4]
If $ H \trianglelefteq G$, then $ gHg^{-1} = H \ \forall \ g \in G$ so $ n_H =1$. Otherwise,  $ G$ acts on  $ H$ by conjugation and $n_H = | \mathcal{ O}| = |G: G_H|$. Notice that for any $ h \in H$, $ hHh^{-1} = H$, so $ H \leq G_H$. Therefore,  $n_H= |G:G_H|$ divides  $ |G:H| = p$ so it is 1 or  $ p$. But if $ n_H = 1$,  $ gHg^{-1} = H \ \forall \ g \in G$ which shows that $ H$ is normal, a contradiction. So it must be that  $ n_H = p$.
\end{problem}
\begin{problem}[5]
~\begin{enumerate}[label=(\alph*)]
	\item First $ |S_5| =5!= 120 = 2^3 \cdot 3 \cdot 5$. So the Sylow 2-subgroups are conjugates of order 8. By Sylow, $ n_2 $ divides $ 15$. We already know that $ S_4$ has three distinct copies of $ D_8$ and there are $ \binom{5}{4} =5$ distinct copies of $ S_4$ in $S_5$. Thus $n_2$ is at least $ 3\times 5 = 15$. Thus it must be 15. (To form $ D_8$ in $ S_4$, we just need to pick any 4-cycle as the rotation. The reflection comes from one of the transposition in the square of this 4-cycle which is a double transposition. This completely determines $ D_8$. There are $ 3!=6$ distinct 4 cycles in  $ S_4$. Since each $ D_8$ requires two, we have $ 6 /2 = 3$ copies of  $ D_8$ in $ S_4$).
	\item We know that for the $ S_4$ copy on $ \{1,2,3,4\} $, the $ V_4$ of double transpositions $e, (1,3)(2,4)$ $(1,2)(3,4), (1,4)(2,3)$, a 4 cycle and its inverse $ (1,2,3,4),(1,4,3,2)$, and two single transpositions $ (1,3),(2,4)$. Similarly, $ S_4$ on $ \{1,2,3,5\}$ yields another $ D_8$. We simply conjugate one by $ (4,5)$ to obtain the other.
\end{enumerate}
\end{problem}

\begin{problem}[6]
~\begin{enumerate}[label=(\alph*)]
	\item Since $ p$ is an odd prime, any  $ P \in Syl_{ p}( D_{2n}) $ cannot contain any reflection or 2 would divide $ p$ by Lagrange. Therefore, any element in  $ P$ is a power of rotation  $ r$. The smallest such power is clearly a generator of  $ P$ so  $ P$ is cyclic. Since $ r$ commutes with its powers, $ rPr^{-1} = P$. Since $ sr^{i}s = r^{-i}$, $ s$acts on  $ P$ as inversion which is clearly an automorphism so  $ sPs ^{-1} =P$. Since $ r,s$ generates  $ D_{2n}$, we see that $ P \trianglelefteq D_{2n}$.
	\item Let $ n = 2^{k}m$ so given $ P \in Syl_{ 2}( D_{2n}) $, $ |P| = 2^{k+1}$. Since $ o(r) = n = 2^{k}m$, we see that $ o(r^{m})= 2^{k}$ so $ r^{m}$ must be in some Sylow 2-groups and so does its powers. Moreover, all reflections must also be evenly distributed in Sylow 2-groups FIX: since conjugation preserves cycle type). We see that $ \langle s,r^{m} \rangle$ consists of elements of the form $ s^{i}(r^{m})^{j}$ where $ i=0,1$,  $0 \leq j \leq 2^{k} -1$. That is, the order of the group is $ 2^{k}$ rotations and $ 2^{k}$ reflections with a total order of $ 2^{k+1}$. This is precisely the order of a Sylow 2-subgroup. Thus, each Sylow 2-subgroup contains $ 2^{k}$ reflections, so $ n_2 = 2^{k}m / 2^{k} = m$.
\end{enumerate}
\end{problem}

\begin{problem}[7]
~\begin{enumerate}[label=(\alph*)]
	\item $ |G|=pqr,p<q<r$. By Sylow,  $ n_r | pq$ and  $ n_r = 1 \bmod r$. Since $p<q<1+kr $ for  $ k >0$,  it must be that  $ n_r = 1$ or  $ pq$. Suppose  $ n_r = pq$. Then by Lagrange we know these prime order groups all intersect trivially so we have $ pq(r-1)$ distinct nontrivial elements from these groups. Now consider  $ n_q |pr$ and  $ n_q = 1 \bmod q$. Again since $ p<1+kq$ for  $ k>0$, $ n_q = 1,r,pr$. Suppose $ n_q = r$. Then we have  $ (q-1)r \geq pr > pq$ additional nontrivial distinct elements. That is, there are  $ pq(r-1)+(q-1)r > pq(r-1) + pq = pqr$ distinct elements from these two Sylow subgroups alone, a contradiction. Hence at least one of them must unique and thus is normal, which implies that  $ G$ is not simple.
	\item Since at least one of them is normal, if $ R$ is normal we are done. If  $ P \trianglelefteq G$, then $ G /P$ is a quotient group. Let  $ \overline{S} \in Syl_{ r}( G /P)$. Then $ \overline{n}_r = 1$ or $ p$ and  $ \overline{n}_r = 1 \bmod r$. Since $ p \neq 1 \bmod r$ as $ p<r$, we have  $ \overline{n}_r = 1$ so $ \overline{S} \trianglelefteq G/P$. By the 4th isomorphism theorem, there exists a corresponding subgroup $ S \trianglelefteq G$ s.t.\ $ S / P = \overline{S}$. We see that $ |S| = |\overline{S}||P| = pr$. Therefore, $P \leq S $ and there exists an $ R \in Syl_{ r}( G) $ s.t.\  $ R \leq S$. Since $ P \trianglelefteq G$, $ PR$ is a subgroup and $ PR \leq S$. Since $ P \cap R = \{e\} $, by prime and Lagrange $ |PR| = pr / 1 = pr = |S|$ so $ PR = S$ and thus $ PR \trianglelefteq G$. Now, let $ n_r'$ be the number of Sylow $ r$-subgroups in $ PR$. Then $ n_r'= 1$ or $ p$ and $ n_r' = 1 \bmod r$ so $ p<r$ forces $ n_r'=1$. That is, $ R$ is the unique subgroup of order $ r$ of  $ PR$ and therefore characteristic in $ PR$. By ``transitivity'' of characteristic subgroups of a normal subgroup, $ R \trianglelefteq G$ as desired. The case when $ Q \trianglelefteq G$ is similar.
\end{enumerate}
\end{problem}
\end{document}
