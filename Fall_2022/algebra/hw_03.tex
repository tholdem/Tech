\documentclass[12pt]{article}
\newcommand{\alert}[1]{{\bf \color{red} [Alert:] #1}}
\newcommand{\todo}[1]{{\bf \color{orange} [TODO:] #1}}
\newcommand{\real}[1][]{\mathbb{R}^{#1}}
\newcommand{\myeqn}[1]{(\ref{#1})}
\newcommand{\myex}[1]{Example \ref{#1}}
\newcommand{\defeq}{\stackrel{\mathrm{def}}{=}}
\newcommand{\parder}[2]{\frac{\partial #1}{\partial #2}}
\newcommand{\Lie}[3][]{\mathsf{L}_{#3}^{#1} #2}
\newcommand{\LieA}[1]{\mathsf{Lie}(#1)}
\newcommand{\lieder}[2]{\mathcal{L}_{#2} #1}
\renewcommand{\t}{^{\mbox{\tiny\sf T}}}
\newcommand{\trans}{^{\mbox{\tiny\sf T}}}
\newcommand{\markup}[1]{\{\textbf{#1}\}}
\newcommand{\msub}[1]{_\mathrm{#1}}
\newcommand{\msup}[1]{^\mathrm{#1}}
\newcommand{\inv}[1]{#1^{-1}}
\newcommand{\pinv}[1]{{#1}^{+}}
\newcommand{\myfracA}[2]{\displaystyle{\frac{#1}{#2}}}
\newcommand{\myfracB}[2]{{#1}/{#2}}
\newcommand{\mydiffA}[1]{\dot{#1}}
\newcommand{\mydiffB}[2]{\myfracA{\mathrm{d}{#1}}{\mathrm{d}{#2}}}
\newcommand{\ball}[2]{\mathcal{B}_{#1}\left(#2\right)}
\newcommand{\acos}[1]{\cos^{-1}\left(#1\right)}
\newcommand{\asin}[1]{\sin^{-1}\left(#1\right)}
\newcommand{\mani}{\mathcal{M}}
\newcommand{\tang}[2]{\mathsf{T}_{#1} #2}
\newcommand{\LieB}[2]{[ #1, #2 ]}
\newcommand{\LieBad}[3][]{\mathsf{ad}_{#2}^{#1} #3}
\newcommand{\ReachVT}{\mathcal{R}^V_T}
\newcommand{\ReachVt}{\mathcal{R}^V_t}
\newcommand{\ReachVTe}{\mathcal{R}^V_{\le T}}
\newcommand{\ReachT}{\mathcal{R}_T}
\newcommand{\Reacht}{\mathcal{R}_t}
\newcommand{\ReachTe}{\mathcal{R}_{\le T}}
\newcommand{\accLA}[1]{\mathsf{Lie}(#1)}
\newcommand{\accD}{\Delta_{\mathcal{F}}}
\newcommand{\accSA}{\mathsf{Lie}(\mathcal{G},f)}
\newcommand{\accDS}{\Delta_{\mathcal{G}}}
\newcommand{\eval}[3]{\mathsf{Ev}^{#2}_{#1}\left( #3 \right)}
\newcommand{\stlc}{\textsc{stlc}}
\newcommand{\clf}{\textsc{clf}}
\newcommand{\jqlf}{\textsc{jqlf}}
\newcommand{\dlas}{\textsc{dlas}}
\newcommand{\Ad}[2]{\mathsf{Ad}_{#1} #2}
\newcommand{\xe}{\ensuremath{x_e}}
\newcommand{\lebg}[1]{\mathcal{L}_{#1}}
\newcommand{\lebgx}[1]{\mathcal{L}_{#1 \mathrm{e}}}
\newcommand{\dom}{D}
\newcommand{\domT}{[t_0,\infty) \times D}
\newcommand{\rarrow}{\rightarrow}
\renewcommand{\d}{\mathrm{d}}
\renewcommand{\Re}{\mathbb{R}}
\newcommand{\C}{\mathrm{C}}

\newcommand{\QED}{{\unskip\nobreak\hfil\penalty50\hskip2em\vadjust{}
		\nobreak\hfil$\Box$\parfillskip=0pt\finalhyphendemerits=0\par}\vspace{0.1cm}}
\newcommand{\eoEx}{{\unskip\nobreak\hfil\penalty50\hskip0em\vadjust{}
		\nobreak\hfil$\Large\Diamond$\parfillskip=0pt\finalhyphendemerits=0\par}\vspace{0.1cm}}

\newcommand{\sgn}{\ensuremath{\operatorname{sgn}}}
\newcommand{\sat}{\ensuremath{\operatorname{sat}}}

\newcommand{\half}{\frac{1}{2}}
\newcommand{\shalf}{\mbox{$\frac{1}{2}$}}
\newcommand{\marcom}[1]{\marginpar{\footnotesize #1}}
\newcommand{\der}{\mathrm{D}}
\newcommand{\e}{\mathrm{e}}
\newcommand{\dt}{\mathrm{d}t}

\newcommand{\cA}{\ensuremath{\mathcal{A}}}
\newcommand{\cB}{\ensuremath{\mathcal{B}}}
\newcommand{\cG}{\ensuremath{\mathcal{G}}}
\newcommand{\cK}{\ensuremath{\mathcal{K}}}
\newcommand{\cW}{\ensuremath{\mathcal{W}}}
\newcommand{\cZ}{\ensuremath{\mathcal{Z}}}
\newcommand{\cS}{\ensuremath{\mathcal{S}}}
\newcommand{\cD}{\ensuremath{\mathcal{D}}}
\newcommand{\cP}{\ensuremath{\mathcal{P}}}
\newcommand{\cV}{\ensuremath{\mathcal{V}}}
\newcommand{\cL}{\ensuremath{\mathcal{L}}}
\newcommand{\cN}{\ensuremath{\mathcal{N}}}
\newcommand{\cI}{\ensuremath{\mathcal{I}}}
\newcommand{\cR}{\ensuremath{\mathcal{R}}}
\newcommand{\cM}{\ensuremath{\mathcal{M}}}
\newcommand{\cC}{\ensuremath{\mathcal{C}}}
\newcommand{\cF}{\ensuremath{\mathcal{F}}}
\newcommand{\cH}{\ensuremath{\mathcal{H}}}
\newcommand{\cO}{\ensuremath{\mathcal{O}}}
\newcommand{\cX}{\ensuremath{\mathcal{X}}}
\newcommand{\cY}{\ensuremath{\mathcal{Y}}}
\newcommand{\Ci}{\ensuremath{\mathcal{C}^\infty}}
\newcommand{\ISS}{\textsc{iss}}
\newcommand{\LISS}{\textsc{liss}}
\newcommand{\GAS}{\textsc{gas}}
\newcommand{\GS}{\textsc{gs}}
\newcommand{\LES}{\textsc{les}}
\newcommand{\GUAS}{\textsc{guas}}
\newcommand{\BIBO}{\textsc{bibo}}
\newcommand{\spec}{\ensuremath{\operatorname{spec}}}
\newcommand{\spn}{\ensuremath{\operatorname{span}}}
\renewcommand{\i}{\mathrm{i\,}}

\renewcommand{\implies}{\Rightarrow}

\renewcommand{\theenumi}{$\roman{enumi})$}
\renewcommand{\labelenumi}{\theenumi}

\font\ptmten=zptmcmrm scaled 1200
\newcommand{\w}{\mbox{{\ptmten w}}}
\newcommand{\z}{\mbox{{\ptmten z}}}
\renewcommand{\Re}{\mathbb{R}}

\newcommand{\cl}{\operatorname{cl}}
\newcommand{\intr}{\operatorname{int}}
\newcommand{\rank}{\operatorname{rank}}
\newcommand{\co}{\operatorname{co}}
\newcommand{\aff}{\operatorname{aff}}

\theoremstyle{plain}
\newtheorem{theorem}{Theorem}[chapter]
\newtheorem{claim}[theorem]{Claim}
\newtheorem{corollary}[theorem]{Corollary}
\newtheorem{prop}[theorem]{Proposition}
\newtheorem{fact}[theorem]{Fact}
\newtheorem{lemma}[theorem]{Lemma}

\newtheorem{remark}{Remark}[chapter]

\theoremstyle{definition}
\newtheorem{assume}[theorem]{Assumption}
\newtheorem{defn}[theorem]{Definition}
\newtheorem{problem}[theorem]{Problem}
\newtheorem{exercise}{Exercise}
\newtheorem{example}[theorem]{Example}


\begin{document}
\centerline {\textsf{\textbf{\LARGE{Homework 3}}}}
\centerline {Jaden Wang}
\vspace{.15in}
\begin{problem}[1]
Since $ G$ acts transitively, the orbit of any  $ a \in A$ is all of $ A$, so  $ |A| = |G:G_a|$. Since  $ |A|>1$, the index of  $ G_a$ is at least 2. Recall that for any $ b=g_b.a$ for some $ g_b \in G$, $ G_b = g_b G_a g_b^{-1}$. Let $S := \bigcup_{ b \in A} g_b G_a g_b ^{-1} \subseteq G$. By exercise 2.1.8, the union of subgroups is a subgroup iff they are contained in one of them. If the union is indeed a subgroup, then $S= G_b$ for some  $ b \in A$. Since the index of $ G_b$ is at least 2, there exists some  $ g \in G \setminus G_b$ which doesn't fix any point by definition of union. If $ S$ is not a subgroup, then since  $ G$ is a group, clearly  $ S \subsetneq G$ so there exists a $ g \in G\setminus S$.
\end{problem}
\begin{problem}[2]
~\begin{enumerate}[label=(\alph*)]
	\item First we consider all types of symmetries of a cube. As we did in class, there are 24 of them.
		\begin{table}[H]
			\centering
			\caption{Cube}
			\begin{tabular}{c|c|c|c}
				type& cycle decomp & \# cycles & elements of this type\\
				\hline
				\hline
				id&(1)(2)(3)(4)(5)(6)(7)(8)&8&1\\
				\hline
				90 upright rot & (1234)(5678)&2& 6\\
				\hline
				180 upright rot & (13)(24)(57)(68)&4&3 \\
				\hline
				120 diag rot & (123)(456)(7)(8)&4&8\\
				\hline
				180 semidiag rot & (17)(28)(34)(56)&4&6
			\end{tabular}
		\end{table}
		So by Burnside, the total distinct ways are
		\begin{align*}
			\frac{1}{24}(k^{8}+6k^2 + 3 k^{4}+8k^{4}+6k^{4}) = \frac{1}{24} (k^{8}+ 17k^{4}+6k^2)
		\end{align*}
	\item 
		\begin{table}[H]
			\centering
			\caption{Tetrahedron}
			\begin{tabular}{c|c|c|c}
				type& cycle decomp & \# cycles & elements of this type\\
				\hline
				\hline
				id&(1)(2)(3)(4)&4&1\\
				\hline
				120 upright rot & (123)(4)&2& 8\\
				\hline
				grand rot & (12)(34)&2& 3\\
			\end{tabular}
		\end{table}
		So total distinct ways are
		\begin{align*}
			\frac{1}{12} (k^{4}+11k^{2})
		\end{align*}
\end{enumerate}
\end{problem}
\begin{problem}[3]
Consider $ S_n$ acting on itself by conjugation. Notice that two elements $ \sigma, \tau$ commute iff $ \sigma \tau \sigma^{-1} = \tau$, that is, $ \tau$ is a fix point of the action of $ \sigma$ by conjugation. Therefore, the number of fixed points of $ \sigma$ action is the same as the number of elements commuting with $ \sigma$. By Burnside lemma,
\begin{align*}
	\# \mathcal{ O} = \frac{1}{|S_n|} \sum_{ \sigma \in S_n} f( \sigma)
\end{align*}
But since conjugacy classes of $ S_n$ are simply distinct cycle types, $ | \mathcal{ O}| = p(n)$. Also notice that the probability $ p_ \sigma$ of fixing an  $ \sigma$ and picking another element commuting with it is precisely $ \frac{ f(\sigma)}{|S_n|}$. Since we assume naive probability for picking elements, the probability for picking two such elements is just the average probability of fixing one and picking another. That is, 
\begin{align*}
p = \frac{1}{|S_n|} \sum_{ \sigma \in S_n} p_ \sigma = \frac{1}{n!} \sum_{ \sigma \in S_n} \frac{f( \sigma)}{|G| } = \frac{p(n)}{ n!}.
\end{align*}
\end{problem}

\begin{problem}[4]
If $ H \trianglelefteq G$, then $ gHg^{-1} = H \ \forall \ g \in G$ so $ n_H =1$. Otherwise,  $ G$ acts on  $ H$ by conjugation and $n_H = | \mathcal{ O}| = |G: G_H|$. Notice that for any $ h \in H$, $ hHh^{-1} = H$, so $ H \leq G_H$. Therefore,  $n_H= |G:G_H|$ divides  $ |G:H| = p$ so it is 1 or  $ p$. But if $ n_H = 1$,  $ gHg^{-1} = H \ \forall \ g \in G$ which shows that $ H$ is normal, a contradiction. So it must be that  $ n_H = p$.
\end{problem}
\begin{problem}[5]
~\begin{enumerate}[label=(\alph*)]
	\item First $ |S_5| =5!= 120 = 2^3 \cdot 3 \cdot 5$. So the Sylow 2-subgroups are conjugates of order 8. By Sylow, $ n_2 $ divides $ 15$. We already know that $ S_4$ has three distinct copies of $ D_8$ and there are $ \binom{5}{4} =5$ distinct copies of $ S_4$ in $S_5$. Thus $n_2$ is at least $ 3\times 5 = 15$. Thus it must be 15. (To form $ D_8$ in $ S_4$, we just need to pick any 4-cycle as the rotation. The reflection comes from one of the transposition in the square of this 4-cycle which is a double transposition. This completely determines $ D_8$. There are $ 3!=6$ distinct 4 cycles in  $ S_4$. Since each $ D_8$ requires two, we have $ 6 /2 = 3$ copies of  $ D_8$ in $ S_4$).
	\item We know that for the $ S_4$ copy on $ \{1,2,3,4\} $, the $ V_4$ of double transpositions $e, (1,3)(2,4)$ $(1,2)(3,4), (1,4)(2,3)$, a 4 cycle and its inverse $ (1,2,3,4),(1,4,3,2)$, and two single transpositions $ (1,3),(2,4)$. Similarly, $ S_4$ on $ \{1,2,3,5\}$ yields another $ D_8$. We simply conjugate one by $ (4,5)$ to obtain the other.
\end{enumerate}
\end{problem}

\begin{problem}[6]
~\begin{enumerate}[label=(\alph*)]
	\item Since $ p$ is an odd prime, any  $ P \in Syl_{ p}( D_{2n}) $ cannot contain any reflection or 2 would divide $ p$ by Lagrange. Therefore, any element in  $ P$ is a power of rotation  $ r$. The smallest such power is clearly a generator of  $ P$ so  $ P$ is cyclic. Since $ r$ commutes with its powers, $ rPr^{-1} = P$. Since $ sr^{i}s = r^{-i}$, $ s$acts on  $ P$ as inversion which is clearly an automorphism so  $ sPs ^{-1} =P$. Since $ r,s$ generates  $ D_{2n}$, we see that $ P \trianglelefteq D_{2n}$.
	\item Let $ n = 2^{k}m$ so given $ P \in Syl_{ 2}( D_{2n}) $, $ |P| = 2^{k+1}$. Since $ o(r) = n = 2^{k}m$, we see that $ o(r^{m})= 2^{k}$ so $ r^{m}$ must be in some Sylow 2-groups and so does its powers. Moreover, all reflections must also be evenly distributed in Sylow 2-groups FIX: since conjugation preserves cycle type). We see that $ \langle s,r^{m} \rangle$ consists of elements of the form $ s^{i}(r^{m})^{j}$ where $ i=0,1$,  $0 \leq j \leq 2^{k} -1$. That is, the order of the group is $ 2^{k}$ rotations and $ 2^{k}$ reflections with a total order of $ 2^{k+1}$. This is precisely the order of a Sylow 2-subgroup. Thus, each Sylow 2-subgroup contains $ 2^{k}$ reflections, so $ n_2 = 2^{k}m / 2^{k} = m$.
\end{enumerate}
\end{problem}

\begin{problem}[7]
~\begin{enumerate}[label=(\alph*)]
	\item $ |G|=pqr,p<q<r$. By Sylow,  $ n_r | pq$ and  $ n_r = 1 \bmod r$. Since $p<q<1+kr $ for  $ k >0$,  it must be that  $ n_r = 1$ or  $ pq$. Suppose  $ n_r = pq$. Then by Lagrange we know these prime order groups all intersect trivially so we have $ pq(r-1)$ distinct nontrivial elements from these groups. Now consider  $ n_q |pr$ and  $ n_q = 1 \bmod q$. Again since $ p<1+kq$ for  $ k>0$, $ n_q = 1,r,pr$. Suppose $ n_q = r$. Then we have  $ (q-1)r \geq pr > pq$ additional nontrivial distinct elements. That is, there are  $ pq(r-1)+(q-1)r > pq(r-1) + pq = pqr$ distinct elements from these two Sylow subgroups alone, a contradiction. Hence at least one of them must unique and thus is normal, which implies that  $ G$ is not simple.
	\item Since at least one of them is normal, if $ R$ is normal we are done. If  $ P \trianglelefteq G$, then $ G /P$ is a quotient group. Let  $ \overline{S} \in Syl_{ r}( G /P)$. Then $ \overline{n}_r = 1$ or $ p$ and  $ \overline{n}_r = 1 \bmod r$. Since $ p \neq 1 \bmod r$ as $ p<r$, we have  $ \overline{n}_r = 1$ so $ \overline{S} \trianglelefteq G/P$. By the 4th isomorphism theorem, there exists a corresponding subgroup $ S \trianglelefteq G$ s.t.\ $ S / P = \overline{S}$. We see that $ |S| = |\overline{S}||P| = pr$. Therefore, $P \leq S $ and there exists an $ R \in Syl_{ r}( G) $ s.t.\  $ R \leq S$. Since $ P \trianglelefteq G$, $ PR$ is a subgroup and $ PR \leq S$. Since $ P \cap R = \{e\} $, by prime and Lagrange $ |PR| = pr / 1 = pr = |S|$ so $ PR = S$ and thus $ PR \trianglelefteq G$. Now, let $ n_r'$ be the number of Sylow $ r$-subgroups in $ PR$. Then $ n_r'= 1$ or $ p$ and $ n_r' = 1 \bmod r$ so $ p<r$ forces $ n_r'=1$. That is, $ R$ is the unique subgroup of order $ r$ of  $ PR$ and therefore characteristic in $ PR$. By ``transitivity'' of characteristic subgroups of a normal subgroup, $ R \trianglelefteq G$ as desired. The case when $ Q \trianglelefteq G$ is similar.
\end{enumerate}
\end{problem}
\end{document}
