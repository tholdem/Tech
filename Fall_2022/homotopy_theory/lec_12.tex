\documentclass[12pt,class=article,crop=false]{standalone} 
\newcommand{\alert}[1]{{\bf \color{red} [Alert:] #1}}
\newcommand{\todo}[1]{{\bf \color{orange} [TODO:] #1}}
\newcommand{\real}[1][]{\mathbb{R}^{#1}}
\newcommand{\myeqn}[1]{(\ref{#1})}
\newcommand{\myex}[1]{Example \ref{#1}}
\newcommand{\defeq}{\stackrel{\mathrm{def}}{=}}
\newcommand{\parder}[2]{\frac{\partial #1}{\partial #2}}
\newcommand{\Lie}[3][]{\mathsf{L}_{#3}^{#1} #2}
\newcommand{\LieA}[1]{\mathsf{Lie}(#1)}
\newcommand{\lieder}[2]{\mathcal{L}_{#2} #1}
\renewcommand{\t}{^{\mbox{\tiny\sf T}}}
\newcommand{\trans}{^{\mbox{\tiny\sf T}}}
\newcommand{\markup}[1]{\{\textbf{#1}\}}
\newcommand{\msub}[1]{_\mathrm{#1}}
\newcommand{\msup}[1]{^\mathrm{#1}}
\newcommand{\inv}[1]{#1^{-1}}
\newcommand{\pinv}[1]{{#1}^{+}}
\newcommand{\myfracA}[2]{\displaystyle{\frac{#1}{#2}}}
\newcommand{\myfracB}[2]{{#1}/{#2}}
\newcommand{\mydiffA}[1]{\dot{#1}}
\newcommand{\mydiffB}[2]{\myfracA{\mathrm{d}{#1}}{\mathrm{d}{#2}}}
\newcommand{\ball}[2]{\mathcal{B}_{#1}\left(#2\right)}
\newcommand{\acos}[1]{\cos^{-1}\left(#1\right)}
\newcommand{\asin}[1]{\sin^{-1}\left(#1\right)}
\newcommand{\mani}{\mathcal{M}}
\newcommand{\tang}[2]{\mathsf{T}_{#1} #2}
\newcommand{\LieB}[2]{[ #1, #2 ]}
\newcommand{\LieBad}[3][]{\mathsf{ad}_{#2}^{#1} #3}
\newcommand{\ReachVT}{\mathcal{R}^V_T}
\newcommand{\ReachVt}{\mathcal{R}^V_t}
\newcommand{\ReachVTe}{\mathcal{R}^V_{\le T}}
\newcommand{\ReachT}{\mathcal{R}_T}
\newcommand{\Reacht}{\mathcal{R}_t}
\newcommand{\ReachTe}{\mathcal{R}_{\le T}}
\newcommand{\accLA}[1]{\mathsf{Lie}(#1)}
\newcommand{\accD}{\Delta_{\mathcal{F}}}
\newcommand{\accSA}{\mathsf{Lie}(\mathcal{G},f)}
\newcommand{\accDS}{\Delta_{\mathcal{G}}}
\newcommand{\eval}[3]{\mathsf{Ev}^{#2}_{#1}\left( #3 \right)}
\newcommand{\stlc}{\textsc{stlc}}
\newcommand{\clf}{\textsc{clf}}
\newcommand{\jqlf}{\textsc{jqlf}}
\newcommand{\dlas}{\textsc{dlas}}
\newcommand{\Ad}[2]{\mathsf{Ad}_{#1} #2}
\newcommand{\xe}{\ensuremath{x_e}}
\newcommand{\lebg}[1]{\mathcal{L}_{#1}}
\newcommand{\lebgx}[1]{\mathcal{L}_{#1 \mathrm{e}}}
\newcommand{\dom}{D}
\newcommand{\domT}{[t_0,\infty) \times D}
\newcommand{\rarrow}{\rightarrow}
\renewcommand{\d}{\mathrm{d}}
\renewcommand{\Re}{\mathbb{R}}
\newcommand{\C}{\mathrm{C}}

\newcommand{\QED}{{\unskip\nobreak\hfil\penalty50\hskip2em\vadjust{}
		\nobreak\hfil$\Box$\parfillskip=0pt\finalhyphendemerits=0\par}\vspace{0.1cm}}
\newcommand{\eoEx}{{\unskip\nobreak\hfil\penalty50\hskip0em\vadjust{}
		\nobreak\hfil$\Large\Diamond$\parfillskip=0pt\finalhyphendemerits=0\par}\vspace{0.1cm}}

\newcommand{\sgn}{\ensuremath{\operatorname{sgn}}}
\newcommand{\sat}{\ensuremath{\operatorname{sat}}}

\newcommand{\half}{\frac{1}{2}}
\newcommand{\shalf}{\mbox{$\frac{1}{2}$}}
\newcommand{\marcom}[1]{\marginpar{\footnotesize #1}}
\newcommand{\der}{\mathrm{D}}
\newcommand{\e}{\mathrm{e}}
\newcommand{\dt}{\mathrm{d}t}

\newcommand{\cA}{\ensuremath{\mathcal{A}}}
\newcommand{\cB}{\ensuremath{\mathcal{B}}}
\newcommand{\cG}{\ensuremath{\mathcal{G}}}
\newcommand{\cK}{\ensuremath{\mathcal{K}}}
\newcommand{\cW}{\ensuremath{\mathcal{W}}}
\newcommand{\cZ}{\ensuremath{\mathcal{Z}}}
\newcommand{\cS}{\ensuremath{\mathcal{S}}}
\newcommand{\cD}{\ensuremath{\mathcal{D}}}
\newcommand{\cP}{\ensuremath{\mathcal{P}}}
\newcommand{\cV}{\ensuremath{\mathcal{V}}}
\newcommand{\cL}{\ensuremath{\mathcal{L}}}
\newcommand{\cN}{\ensuremath{\mathcal{N}}}
\newcommand{\cI}{\ensuremath{\mathcal{I}}}
\newcommand{\cR}{\ensuremath{\mathcal{R}}}
\newcommand{\cM}{\ensuremath{\mathcal{M}}}
\newcommand{\cC}{\ensuremath{\mathcal{C}}}
\newcommand{\cF}{\ensuremath{\mathcal{F}}}
\newcommand{\cH}{\ensuremath{\mathcal{H}}}
\newcommand{\cO}{\ensuremath{\mathcal{O}}}
\newcommand{\cX}{\ensuremath{\mathcal{X}}}
\newcommand{\cY}{\ensuremath{\mathcal{Y}}}
\newcommand{\Ci}{\ensuremath{\mathcal{C}^\infty}}
\newcommand{\ISS}{\textsc{iss}}
\newcommand{\LISS}{\textsc{liss}}
\newcommand{\GAS}{\textsc{gas}}
\newcommand{\GS}{\textsc{gs}}
\newcommand{\LES}{\textsc{les}}
\newcommand{\GUAS}{\textsc{guas}}
\newcommand{\BIBO}{\textsc{bibo}}
\newcommand{\spec}{\ensuremath{\operatorname{spec}}}
\newcommand{\spn}{\ensuremath{\operatorname{span}}}
\renewcommand{\i}{\mathrm{i\,}}

\renewcommand{\implies}{\Rightarrow}

\renewcommand{\theenumi}{$\roman{enumi})$}
\renewcommand{\labelenumi}{\theenumi}

\font\ptmten=zptmcmrm scaled 1200
\newcommand{\w}{\mbox{{\ptmten w}}}
\newcommand{\z}{\mbox{{\ptmten z}}}
\renewcommand{\Re}{\mathbb{R}}

\newcommand{\cl}{\operatorname{cl}}
\newcommand{\intr}{\operatorname{int}}
\newcommand{\rank}{\operatorname{rank}}
\newcommand{\co}{\operatorname{co}}
\newcommand{\aff}{\operatorname{aff}}

\theoremstyle{plain}
\newtheorem{theorem}{Theorem}[chapter]
\newtheorem{claim}[theorem]{Claim}
\newtheorem{corollary}[theorem]{Corollary}
\newtheorem{prop}[theorem]{Proposition}
\newtheorem{fact}[theorem]{Fact}
\newtheorem{lemma}[theorem]{Lemma}

\newtheorem{remark}{Remark}[chapter]

\theoremstyle{definition}
\newtheorem{assume}[theorem]{Assumption}
\newtheorem{defn}[theorem]{Definition}
\newtheorem{problem}[theorem]{Problem}
\newtheorem{exercise}{Exercise}
\newtheorem{example}[theorem]{Example}


\begin{document}
\section{Classifying Spaces}

Exercise: $ E_n  \xrightarrow{ p} G_n$ is an $ n$-dimensional vector bundle. Hint: IF  $ \ell \in G_n$, let $ \pi_{\ell}: \rr^{\infty} \to \ell$ be orthogonal proj. Let $ U_\ell = \{\ell' \in G_n: \pi_{\ell}(\ell') \text{ has dim }n \} $. Show $ U_\ell$ is open and $ h: p ^{-1}(U_\ell) \to U_{ \ell} \times \ell, (\ell',v)\mapsto (\ell',\pi_\ell(v))$ is a local trivialization.

\begin{thm}
	Let $ X$ be paracompact and $ E_n = \{(\ell,v) \in G_n \times \rr^{\infty}: v \in \ell\} $. Then $ [X,G_n] \to \text{Vect}^{n}(X), f\mapsto f^* E_n $ is a 1 to 1 correspondence.
\end{thm}

\begin{defn}
For a topological group $ G$, there exists a space  $ BG$ and a principal  $ G$-bundle  $ EG$  s.t.\ $ (EG,BG,G,p)$ is a bundle and  $ EG$ is weakly contractible. We call  $ BG$ the  \allbold{classifying space for principal $ G$-bundle} and $ EG$ the  \allbold{universal $ G$-bundle}.  
\end{defn}

\begin{remark}
By the long exact sequence and weakly contractible, $ \pi_k(BG) \cong \pi_{k-1}(G) \ \forall \ k \geq 1$.
\end{remark}

\begin{thm}
	$ [X,BG]$ and principal  $ G$-bundles over  $ X$ is a 1 to 1 correspondence (via $ f\mapsto f^* EG)$.
\end{thm}
\begin{thm}
The homotopy type of $ BG$ is unique.
\end{thm}
\begin{eg}
\begin{enumerate}[label=(\arabic*)]
	\item $ G_n$ is the classifying space of $ \rr^{n}$-bundles. In fact, $ \mathcal{ F}(E_n)$ is an $ \text{GL}_{ n}(\rr) $-bundle and $ G_n$ is the $ \text{GL}_{ n}(\rr) $ classifying space. Exercise: $ \mathcal{ F}(E_n)$ is weakly contractible.
	\item $ \rr \to S^{1}$ is a principal $ \zz$-bundle with $ E \zz = \rr$ and $ B\zz = S^{1}$. Principal $ \zz$-bundles over $ X$ is 1 to 1 correspondence with  $ [X,S^{1}] = [X,K(\zz,1)] \cong H^{1}(X;\zz)$ by Brown representation theorem: $ [X,K(\pi,n)] \cong H^{n}(X; \pi)$.
	\item $ S^{\infty} \to \rr P^{\infty}$ is a principal $ \zz /2$-bundle. Note $ \zz /2 \cong O(1)$. Then $ BO(1) \cong \rr P^{\infty}$ and $ EO(1) \cong S^{\infty}$. Exercise: $ S^{\infty}$ is contractible. So line bundles over $ X$ is 1 to 1 with principal  $ O(1)$-bundles over $ X$ is 1 to 1 with  $ [X,BO(1)] = [X, \rr P^{\infty}] = [X,K(\zz /2,1)] = H^{1}(X; \zz /2)$ by Brown.
	\item $ S^{\infty} \to S^{\infty} / S^{1} \cong \cc P^{\infty}$ is a principal $ S^{1}$-bundle. Note $ S^{1} = U(1)$. So $ BU(1) \cong \cc P^{\infty}$, $ EU(1) \cong S^{\infty}$. Complex line bundles over $ X$ 1 to 1 principal $ U(1)$-bundles over  $ X$ 1 to 1  $ [X,BU(1)]=[X, \cc P^{\infty}] = [X, K(\zz,2)] \cong H^{2}(X;\zz)$.
\end{enumerate}
\end{eg}

\begin{defn}
A $ G$CW-complex is a space $ X$ with a  $ G$-action that is the union of skeleta \ldots
\end{defn}
Exercise:
\begin{enumerate}[label=(\arabic*)]
	\item If $ X$ is a $ G$CW-complex, then  $ X /G$ has a natural  CW structure.
	\item If $ G$ is a compact Lie group, then any principal  $ G$-bundle over a  CW-complex is a $ G$CW-complex.
\end{enumerate}

To construct classifying spaces, we need the definition:
\begin{defn}
Let $ X,Y$ be spaces. Their  \allbold{join} is
\begin{align*}
	X *Y = X \times I \times Y / \ \sim
\end{align*}
where $ (x,0,y_1) \sim (x,0,y_2) \ \forall \ y_1,y_2 \in Y$ and $ (x_1,1,y) \sim (x_2,1,y) \ \forall \ x_1,x_2 \in X$.
\end{defn}
Examples:
\begin{enumerate}[label=(\arabic*)]
	\item $ X* \{*\} \cong \text{Cone}(X) $.
	\item $ X* \{p_1,p_2\} \cong \Sigma X$.
	\item $ \{x_0\} * \cdots * \{x_k\} $ is a k-simplex.
	\item Exercise: $ S^{n}*S^{m} \cong S^{n+m+1}$. Start with $ S^{1} * S^{1} \cong S^3$, $ \rr^2 * \rr^2, \rr^{4}$. DO THIS.
\end{enumerate}
\begin{remark}
The join generalizes the cone.
\end{remark}

There exists inclusions $ X \xrightarrow{ i} X *Y, x\mapsto (x,0,y) $ for any $ y$ and  $ Y \xrightarrow{ j} X*Y, y \mapsto (x,1,y) $ for any $ x$.

\begin{lem}
The inclusion $ i: X \to X*Y$ and $ j: Y \to X*Y$ are nullhomotopic.
\end{lem}

\begin{proof}
For any $ y_0 \in Y$, $ i$ factors through  $ X \to X * \{y_0\} = C(X) $ and hence $ X \to X* \{y_0\} $ is nullhomotopic. The claim follows.
\end{proof}

Given $ G$ a topological group, let  $ G^{*(k+1)} = \underbrace{ G* G * \cdots * G}_{k+1 } $. This has a $ G$-action:
 \begin{align*}
	 (g_0,t_1,g_1,t_2,\ldots,t_k,g_k).g = (g_0 g, t_1, g_1g,t_2,\ldots,t_k,g_k g).
\end{align*}
Exercise:
\begin{enumerate}[label=(\arabic*)]
	\item There exists a natural $ G$-equivariant map $ \Delta^{k} \times G ^{k+1} \to G^{*(k+1)}$ that is a homeomorphism when restricted to interior of $ \Delta ^{k} \times G^{k+1}$. (think simplex example).
	\item Use 1 to show $ G^{*(k+1)}$ has the structure of a $ G$CW-complex.
\end{enumerate}
Let $ \mathcal{ J}(G) = \lim_{ k \to \infty} G^{*(k+1)}$.
\begin{thm}
The quotient map $ p: \mathcal{ J}(G)=EG \to J(G) / G = BG$ is a universal principal $ G$-bundle.
\end{thm}
\begin{proof}
Exercise: show $ p$ is a principal  $ G$-bundle.

We are done if  $ \mathcal{ J}(G)$ is weakly contractible. For any $ \alpha: S^{n} \to \mathcal{ J}(G)$, there exists some $ k$ s.t.\ $ \alpha(S^{n}) \subseteq G^{*(k+1)} \subseteq \mathcal{ J}(G)$ and $ G^{*(k+1)} \to G^{*(k+1)} \subseteq \mathcal{ J}(G)$ is nullhomotopic. So $ \alpha: S^{n} to G^{*(k+1)} \subseteq G^{*(k+2)}$ is nullhomotopic.
\end{proof}

From the construction, given $ f:H to G$ a homo, then we get an induced map  $ Ef: EH = \mathcal{ J}(H) \to EG= \mathcal{ J}(G)$ and $ Bf: BH \to BG$.

Exercise:
\begin{enumerate}[label=(\arabic*)]
	\item $ Bf$ is the classifying map for the bundle  $ BH \times _f G$, \emph{i.e.} $ (Bf)^* EG \cong BH \times _f G$.
	\item If $ H \leq G$ and  $ P \to M$ is a principal $ G$-bundle, then structure group of  $ P$ reduces to  $ H$ iff the classifying map  $ f: M \to BG$ (after homotopy) factors through $ M \to BH$.
\end{enumerate}

A different view of characteristic classes:
\begin{thm}
	$ H^* (BO(n); \zz /2) \cong \zz /2[w_1,\ldots,w_n]$ where $ w_i$ has degree $ i$.

	$ H^* (BU(n); \zz) \cong \zz [c_1,\ldots,c_n]$ where $ c_i$ has degree $ 2i$.
\end{thm}
We can use this theorem to define characteristic classes of an $ \rr^{n}$-bundle $ E \to M$. There exists an associated $ O(n)$-bundle. By theorem 13 there exists a map  $ f: M \to BO(n)$ s.t.\ $ \mathcal{ F}(E) \cong f^* EO(n)$. Define the $ i$th Steifel-Whitney class of  $ E$ to be
 \begin{align*}
	w_i(E) = f^* w_i.
\end{align*}
Simiarly for Chern classes.


\begin{thm}
	$ H^* (BSO(2n+1); \zz) \cong \zz [p_1,\ldots,p_n] \oplus $ Torsion where Torsion is $ \beta(H^{n}(BSO(2n+1); \zz /2))$.

	$ H^* (BSO(2n);\zz) \cong \zz[p_1,\ldots,p_n, e] / \langle e ^2 =p_n  \rangle \oplus $ Torsion.
\end{thm}
\end{document}
