\documentclass[12pt,class=article,crop=false]{standalone} 
%Fall 2022
% Some basic packages
\usepackage{standalone}[subpreambles=true]
\usepackage[utf8]{inputenc}
\usepackage[T1]{fontenc}
\usepackage{textcomp}
\usepackage[english]{babel}
\usepackage{url}
\usepackage{graphicx}
%\usepackage{quiver}
\usepackage{float}
\usepackage{enumitem}
\usepackage{lmodern}
\usepackage{comment}
\usepackage{hyperref}
\usepackage[usenames,svgnames,dvipsnames]{xcolor}
\usepackage[margin=1in]{geometry}
\usepackage{pdfpages}

\pdfminorversion=7

% Don't indent paragraphs, leave some space between them
\usepackage{parskip}

% Hide page number when page is empty
\usepackage{emptypage}
\usepackage{subcaption}
\usepackage{multicol}
\usepackage[b]{esvect}

% Math stuff
\usepackage{amsmath, amsfonts, mathtools, amsthm, amssymb}
\usepackage{bbm}
\usepackage{stmaryrd}
\allowdisplaybreaks

% Fancy script capitals
\usepackage{mathrsfs}
\usepackage{cancel}
% Bold math
\usepackage{bm}
% Some shortcuts
\newcommand{\rr}{\ensuremath{\mathbb{R}}}
\newcommand{\zz}{\ensuremath{\mathbb{Z}}}
\newcommand{\qq}{\ensuremath{\mathbb{Q}}}
\newcommand{\nn}{\ensuremath{\mathbb{N}}}
\newcommand{\ff}{\ensuremath{\mathbb{F}}}
\newcommand{\cc}{\ensuremath{\mathbb{C}}}
\newcommand{\ee}{\ensuremath{\mathbb{E}}}
\newcommand{\hh}{\ensuremath{\mathbb{H}}}
\renewcommand\O{\ensuremath{\emptyset}}
\newcommand{\norm}[1]{{\left\lVert{#1}\right\rVert}}
\newcommand{\dbracket}[1]{{\left\llbracket{#1}\right\rrbracket}}
\newcommand{\ve}[1]{{\bm{#1}}}
\newcommand\allbold[1]{{\boldmath\textbf{#1}}}
\DeclareMathOperator{\lcm}{lcm}
\DeclareMathOperator{\im}{im}
\DeclareMathOperator{\coim}{coim}
\DeclareMathOperator{\dom}{dom}
\DeclareMathOperator{\tr}{tr}
\DeclareMathOperator{\rank}{rank}
\DeclareMathOperator*{\var}{Var}
\DeclareMathOperator*{\ev}{E}
\DeclareMathOperator{\dg}{deg}
\DeclareMathOperator{\aff}{aff}
\DeclareMathOperator{\conv}{conv}
\DeclareMathOperator{\inte}{int}
\DeclareMathOperator*{\argmin}{argmin}
\DeclareMathOperator*{\argmax}{argmax}
\DeclareMathOperator{\graph}{graph}
\DeclareMathOperator{\sgn}{sgn}
\DeclareMathOperator*{\Rep}{Rep}
\DeclareMathOperator{\Proj}{Proj}
\DeclareMathOperator{\mat}{mat}
\DeclareMathOperator{\diag}{diag}
\DeclareMathOperator{\aut}{Aut}
\DeclareMathOperator{\gal}{Gal}
\DeclareMathOperator{\inn}{Inn}
\DeclareMathOperator{\edm}{End}
\DeclareMathOperator{\Hom}{Hom}
\DeclareMathOperator{\ext}{Ext}
\DeclareMathOperator{\tor}{Tor}
\DeclareMathOperator{\Span}{Span}
\DeclareMathOperator{\Stab}{Stab}
\DeclareMathOperator{\cont}{cont}
\DeclareMathOperator{\Ann}{Ann}
\DeclareMathOperator{\Div}{div}
\DeclareMathOperator{\curl}{curl}
\DeclareMathOperator{\nat}{Nat}
\DeclareMathOperator{\gr}{Gr}
\DeclareMathOperator{\vect}{Vect}
\DeclareMathOperator{\id}{id}
\DeclareMathOperator{\Mod}{Mod}
\DeclareMathOperator{\sign}{sign}
\DeclareMathOperator{\Surf}{Surf}
\DeclareMathOperator{\fcone}{fcone}
\DeclareMathOperator{\Rot}{Rot}
\DeclareMathOperator{\grad}{grad}
\DeclareMathOperator{\atan2}{atan2}
\DeclareMathOperator{\Ric}{Ric}
\let\vec\relax
\DeclareMathOperator{\vec}{vec}
\let\Re\relax
\DeclareMathOperator{\Re}{Re}
\let\Im\relax
\DeclareMathOperator{\Im}{Im}
% Put x \to \infty below \lim
\let\svlim\lim\def\lim{\svlim\limits}

%wide hat
\usepackage{scalerel,stackengine}
\stackMath
\newcommand*\wh[1]{%
\savestack{\tmpbox}{\stretchto{%
  \scaleto{%
    \scalerel*[\widthof{\ensuremath{#1}}]{\kern-.6pt\bigwedge\kern-.6pt}%
    {\rule[-\textheight/2]{1ex}{\textheight}}%WIDTH-LIMITED BIG WEDGE
  }{\textheight}% 
}{0.5ex}}%
\stackon[1pt]{#1}{\tmpbox}%
}
\parskip 1ex

%Make implies and impliedby shorter
\let\implies\Rightarrow
\let\impliedby\Leftarrow
\let\iff\Leftrightarrow
\let\epsilon\varepsilon

% Add \contra symbol to denote contradiction
\usepackage{stmaryrd} % for \lightning
\newcommand\contra{\scalebox{1.5}{$\lightning$}}

% \let\phi\varphi

% Command for short corrections
% Usage: 1+1=\correct{3}{2}

\definecolor{correct}{HTML}{009900}
\newcommand\correct[2]{\ensuremath{\:}{\color{red}{#1}}\ensuremath{\to }{\color{correct}{#2}}\ensuremath{\:}}
\newcommand\green[1]{{\color{correct}{#1}}}

% horizontal rule
\newcommand\hr{
    \noindent\rule[0.5ex]{\linewidth}{0.5pt}
}

% hide parts
\newcommand\hide[1]{}

% si unitx
\usepackage{siunitx}
\sisetup{locale = FR}

%allows pmatrix to stretch
\makeatletter
\renewcommand*\env@matrix[1][\arraystretch]{%
  \edef\arraystretch{#1}%
  \hskip -\arraycolsep
  \let\@ifnextchar\new@ifnextchar
  \array{*\c@MaxMatrixCols c}}
\makeatother

\renewcommand{\arraystretch}{0.8}

\renewcommand{\baselinestretch}{1.5}

\usepackage{graphics}
\usepackage{epstopdf}

\RequirePackage{hyperref}
%%
%% Add support for color in order to color the hyperlinks.
%% 
\hypersetup{
  colorlinks = true,
  urlcolor = blue,
  citecolor = blue
}
%%fakesection Links
\hypersetup{
    colorlinks,
    linkcolor={red!50!black},
    citecolor={green!50!black},
    urlcolor={blue!80!black}
}
%customization of cleveref
\RequirePackage[capitalize,nameinlink]{cleveref}[0.19]

% Per SIAM Style Manual, "section" should be lowercase
\crefname{section}{section}{sections}
\crefname{subsection}{subsection}{subsections}
\Crefname{section}{Section}{Sections}
\Crefname{subsection}{Subsection}{Subsections}

% Per SIAM Style Manual, "Figure" should be spelled out in references
\Crefname{figure}{Figure}{Figures}

% Per SIAM Style Manual, don't say equation in front on an equation.
\crefformat{equation}{\textup{#2(#1)#3}}
\crefrangeformat{equation}{\textup{#3(#1)#4--#5(#2)#6}}
\crefmultiformat{equation}{\textup{#2(#1)#3}}{ and \textup{#2(#1)#3}}
{, \textup{#2(#1)#3}}{, and \textup{#2(#1)#3}}
\crefrangemultiformat{equation}{\textup{#3(#1)#4--#5(#2)#6}}%
{ and \textup{#3(#1)#4--#5(#2)#6}}{, \textup{#3(#1)#4--#5(#2)#6}}{, and \textup{#3(#1)#4--#5(#2)#6}}

% But spell it out at the beginning of a sentence.
\Crefformat{equation}{#2Equation~\textup{(#1)}#3}
\Crefrangeformat{equation}{Equations~\textup{#3(#1)#4--#5(#2)#6}}
\Crefmultiformat{equation}{Equations~\textup{#2(#1)#3}}{ and \textup{#2(#1)#3}}
{, \textup{#2(#1)#3}}{, and \textup{#2(#1)#3}}
\Crefrangemultiformat{equation}{Equations~\textup{#3(#1)#4--#5(#2)#6}}%
{ and \textup{#3(#1)#4--#5(#2)#6}}{, \textup{#3(#1)#4--#5(#2)#6}}{, and \textup{#3(#1)#4--#5(#2)#6}}

% Make number non-italic in any environment.
\crefdefaultlabelformat{#2\textup{#1}#3}

% Environments
\makeatother
% For box around Definition, Theorem, \ldots
%%fakesection Theorems
\usepackage{thmtools}
\usepackage[framemethod=TikZ]{mdframed}

\theoremstyle{definition}
\mdfdefinestyle{mdbluebox}{%
	roundcorner = 10pt,
	linewidth=1pt,
	skipabove=12pt,
	innerbottommargin=9pt,
	skipbelow=2pt,
	nobreak=true,
	linecolor=blue,
	backgroundcolor=TealBlue!5,
}
\declaretheoremstyle[
	headfont=\sffamily\bfseries\color{MidnightBlue},
	mdframed={style=mdbluebox},
	headpunct={\\[3pt]},
	postheadspace={0pt}
]{thmbluebox}

\mdfdefinestyle{mdredbox}{%
	linewidth=0.5pt,
	skipabove=12pt,
	frametitleaboveskip=5pt,
	frametitlebelowskip=0pt,
	skipbelow=2pt,
	frametitlefont=\bfseries,
	innertopmargin=4pt,
	innerbottommargin=8pt,
	nobreak=false,
	linecolor=RawSienna,
	backgroundcolor=Salmon!5,
}
\declaretheoremstyle[
	headfont=\bfseries\color{RawSienna},
	mdframed={style=mdredbox},
	headpunct={\\[3pt]},
	postheadspace={0pt},
]{thmredbox}

\declaretheorem[%
style=thmbluebox,name=Theorem,numberwithin=section]{thm}
\declaretheorem[style=thmbluebox,name=Lemma,sibling=thm]{lem}
\declaretheorem[style=thmbluebox,name=Proposition,sibling=thm]{prop}
\declaretheorem[style=thmbluebox,name=Corollary,sibling=thm]{coro}
\declaretheorem[style=thmredbox,name=Example,sibling=thm]{eg}

\mdfdefinestyle{mdgreenbox}{%
	roundcorner = 10pt,
	linewidth=1pt,
	skipabove=12pt,
	innerbottommargin=9pt,
	skipbelow=2pt,
	nobreak=true,
	linecolor=ForestGreen,
	backgroundcolor=ForestGreen!5,
}

\declaretheoremstyle[
	headfont=\bfseries\sffamily\color{ForestGreen!70!black},
	bodyfont=\normalfont,
	spaceabove=2pt,
	spacebelow=1pt,
	mdframed={style=mdgreenbox},
	headpunct={ --- },
]{thmgreenbox}

\declaretheorem[style=thmgreenbox,name=Definition,sibling=thm]{defn}

\mdfdefinestyle{mdgreenboxsq}{%
	linewidth=1pt,
	skipabove=12pt,
	innerbottommargin=9pt,
	skipbelow=2pt,
	nobreak=true,
	linecolor=ForestGreen,
	backgroundcolor=ForestGreen!5,
}
\declaretheoremstyle[
	headfont=\bfseries\sffamily\color{ForestGreen!70!black},
	bodyfont=\normalfont,
	spaceabove=2pt,
	spacebelow=1pt,
	mdframed={style=mdgreenboxsq},
	headpunct={},
]{thmgreenboxsq}
\declaretheoremstyle[
	headfont=\bfseries\sffamily\color{ForestGreen!70!black},
	bodyfont=\normalfont,
	spaceabove=2pt,
	spacebelow=1pt,
	mdframed={style=mdgreenboxsq},
	headpunct={},
]{thmgreenboxsq*}

\mdfdefinestyle{mdblackbox}{%
	skipabove=8pt,
	linewidth=3pt,
	rightline=false,
	leftline=true,
	topline=false,
	bottomline=false,
	linecolor=black,
	backgroundcolor=RedViolet!5!gray!5,
}
\declaretheoremstyle[
	headfont=\bfseries,
	bodyfont=\normalfont\small,
	spaceabove=0pt,
	spacebelow=0pt,
	mdframed={style=mdblackbox}
]{thmblackbox}

\theoremstyle{plain}
\declaretheorem[name=Question,sibling=thm,style=thmblackbox]{ques}
\declaretheorem[name=Remark,sibling=thm,style=thmgreenboxsq]{remark}
\declaretheorem[name=Remark,sibling=thm,style=thmgreenboxsq*]{remark*}
\newtheorem{ass}[thm]{Assumptions}

\theoremstyle{definition}
\newtheorem*{problem}{Problem}
\newtheorem{claim}[thm]{Claim}
\theoremstyle{remark}
\newtheorem*{case}{Case}
\newtheorem*{notation}{Notation}
\newtheorem*{note}{Note}
\newtheorem*{motivation}{Motivation}
\newtheorem*{intuition}{Intuition}
\newtheorem*{conjecture}{Conjecture}

% Make section starts with 1 for report type
%\renewcommand\thesection{\arabic{section}}

% End example and intermezzo environments with a small diamond (just like proof
% environments end with a small square)
\usepackage{etoolbox}
\AtEndEnvironment{vb}{\null\hfill$\diamond$}%
\AtEndEnvironment{intermezzo}{\null\hfill$\diamond$}%
% \AtEndEnvironment{opmerking}{\null\hfill$\diamond$}%

% Fix some spacing
% http://tex.stackexchange.com/questions/22119/how-can-i-change-the-spacing-before-theorems-with-amsthm
\makeatletter
\def\thm@space@setup{%
  \thm@preskip=\parskip \thm@postskip=0pt
}

% Fix some stuff
% %http://tex.stackexchange.com/questions/76273/multiple-pdfs-with-page-group-included-in-a-single-page-warning
\pdfsuppresswarningpagegroup=1


% My name
\author{Jaden Wang}



\begin{document}
\section{Classifying Spaces}

Exercise: $ E_n  \xrightarrow{ p} G_n$ is an $ n$-dimensional vector bundle. Hint: IF  $ \ell \in G_n$, let $ \pi_{\ell}: \rr^{\infty} \to \ell$ be orthogonal proj. Let $ U_\ell = \{\ell' \in G_n: \pi_{\ell}(\ell') \text{ has dim }n \} $. Show $ U_\ell$ is open and $ h: p ^{-1}(U_\ell) \to U_{ \ell} \times \ell, (\ell',v)\mapsto (\ell',\pi_\ell(v))$ is a local trivialization.

\begin{thm}
	Let $ X$ be paracompact and $ E_n = \{(\ell,v) \in G_n \times \rr^{\infty}: v \in \ell\} $. Then $ [X,G_n] \to \text{Vect}^{n}(X), f\mapsto f^* E_n $ is a 1 to 1 correspondence.
\end{thm}

\begin{defn}
For a topological group $ G$, there exists a space  $ BG$ and a principal  $ G$-bundle  $ EG$  s.t.\ $ (EG,BG,G,p)$ is a bundle and  $ EG$ is weakly contractible. We call  $ BG$ the  \allbold{classifying space for principal $ G$-bundle} and $ EG$ the  \allbold{universal $ G$-bundle}.  
\end{defn}

\begin{remark}
By the long exact sequence and weakly contractible, $ \pi_k(BG) \cong \pi_{k-1}(G) \ \forall \ k \geq 1$.
\end{remark}

\begin{thm}
	$ [X,BG]$ and principal  $ G$-bundles over  $ X$ is a 1 to 1 correspondence (via $ f\mapsto f^* EG)$.
\end{thm}
\begin{thm}
The homotopy type of $ BG$ is unique.
\end{thm}
\begin{eg}
\begin{enumerate}[label=(\arabic*)]
	\item $ G_n$ is the classifying space of $ \rr^{n}$-bundles. In fact, $ \mathcal{ F}(E_n)$ is an $ \text{GL}_{ n}(\rr) $-bundle and $ G_n$ is the $ \text{GL}_{ n}(\rr) $ classifying space. Exercise: $ \mathcal{ F}(E_n)$ is weakly contractible.
	\item $ \rr \to S^{1}$ is a principal $ \zz$-bundle with $ E \zz = \rr$ and $ B\zz = S^{1}$. Principal $ \zz$-bundles over $ X$ is 1 to 1 correspondence with  $ [X,S^{1}] = [X,K(\zz,1)] \cong H^{1}(X;\zz)$ by Brown representation theorem: $ [X,K(\pi,n)] \cong H^{n}(X; \pi)$.
	\item $ S^{\infty} \to \rr P^{\infty}$ is a principal $ \zz /2$-bundle. Note $ \zz /2 \cong O(1)$. Then $ BO(1) \cong \rr P^{\infty}$ and $ EO(1) \cong S^{\infty}$. Exercise: $ S^{\infty}$ is contractible. So line bundles over $ X$ is 1 to 1 with principal  $ O(1)$-bundles over $ X$ is 1 to 1 with  $ [X,BO(1)] = [X, \rr P^{\infty}] = [X,K(\zz /2,1)] = H^{1}(X; \zz /2)$ by Brown.
	\item $ S^{\infty} \to S^{\infty} / S^{1} \cong \cc P^{\infty}$ is a principal $ S^{1}$-bundle. Note $ S^{1} = U(1)$. So $ BU(1) \cong \cc P^{\infty}$, $ EU(1) \cong S^{\infty}$. Complex line bundles over $ X$ 1 to 1 principal $ U(1)$-bundles over  $ X$ 1 to 1  $ [X,BU(1)]=[X, \cc P^{\infty}] = [X, K(\zz,2)] \cong H^{2}(X;\zz)$.
\end{enumerate}
\end{eg}

\begin{defn}
A $ G$CW-complex is a space $ X$ with a  $ G$-action that is the union of skeleta \ldots
\end{defn}
Exercise:
\begin{enumerate}[label=(\arabic*)]
	\item If $ X$ is a $ G$CW-complex, then  $ X /G$ has a natural  CW structure.
	\item If $ G$ is a compact Lie group, then any principal  $ G$-bundle over a  CW-complex is a $ G$CW-complex.
\end{enumerate}

To construct classifying spaces, we need the definition:
\begin{defn}
Let $ X,Y$ be spaces. Their  \allbold{join} is
\begin{align*}
	X *Y = X \times I \times Y / \ \sim
\end{align*}
where $ (x,0,y_1) \sim (x,0,y_2) \ \forall \ y_1,y_2 \in Y$ and $ (x_1,1,y) \sim (x_2,1,y) \ \forall \ x_1,x_2 \in X$.
\end{defn}
Examples:
\begin{enumerate}[label=(\arabic*)]
	\item $ X* \{*\} \cong \text{Cone}(X) $.
	\item $ X* \{p_1,p_2\} \cong \Sigma X$.
	\item $ \{x_0\} * \cdots * \{x_k\} $ is a k-simplex.
	\item Exercise: $ S^{n}*S^{m} \cong S^{n+m+1}$. Start with $ S^{1} * S^{1} \cong S^3$, $ \rr^2 * \rr^2, \rr^{4}$. DO THIS.
\end{enumerate}
\begin{remark}
The join generalizes the cone.
\end{remark}

There exists inclusions $ X \xrightarrow{ i} X *Y, x\mapsto (x,0,y) $ for any $ y$ and  $ Y \xrightarrow{ j} X*Y, y \mapsto (x,1,y) $ for any $ x$.

\begin{lem}
The inclusion $ i: X \to X*Y$ and $ j: Y \to X*Y$ are nullhomotopic.
\end{lem}

\begin{proof}
For any $ y_0 \in Y$, $ i$ factors through  $ X \to X * \{y_0\} = C(X) $ and hence $ X \to X* \{y_0\} $ is nullhomotopic. The claim follows.
\end{proof}

Given $ G$ a topological group, let  $ G^{*(k+1)} = \underbrace{ G* G * \cdots * G}_{k+1 } $. This has a $ G$-action:
 \begin{align*}
	 (g_0,t_1,g_1,t_2,\ldots,t_k,g_k).g = (g_0 g, t_1, g_1g,t_2,\ldots,t_k,g_k g).
\end{align*}
Exercise:
\begin{enumerate}[label=(\arabic*)]
	\item There exists a natural $ G$-equivariant map $ \Delta^{k} \times G ^{k+1} \to G^{*(k+1)}$ that is a homeomorphism when restricted to interior of $ \Delta ^{k} \times G^{k+1}$. (think simplex example).
	\item Use 1 to show $ G^{*(k+1)}$ has the structure of a $ G$CW-complex.
\end{enumerate}
Let $ \mathcal{ J}(G) = \lim_{ k \to \infty} G^{*(k+1)}$.
\begin{thm}
The quotient map $ p: \mathcal{ J}(G)=EG \to J(G) / G = BG$ is a universal principal $ G$-bundle.
\end{thm}
\begin{proof}
Exercise: show $ p$ is a principal  $ G$-bundle.

We are done if  $ \mathcal{ J}(G)$ is weakly contractible. For any $ \alpha: S^{n} \to \mathcal{ J}(G)$, there exists some $ k$ s.t.\ $ \alpha(S^{n}) \subseteq G^{*(k+1)} \subseteq \mathcal{ J}(G)$ and $ G^{*(k+1)} \to G^{*(k+1)} \subseteq \mathcal{ J}(G)$ is nullhomotopic. So $ \alpha: S^{n} to G^{*(k+1)} \subseteq G^{*(k+2)}$ is nullhomotopic.
\end{proof}

From the construction, given $ f:H to G$ a homo, then we get an induced map  $ Ef: EH = \mathcal{ J}(H) \to EG= \mathcal{ J}(G)$ and $ Bf: BH \to BG$.

Exercise:
\begin{enumerate}[label=(\arabic*)]
	\item $ Bf$ is the classifying map for the bundle  $ BH \times _f G$, \emph{i.e.} $ (Bf)^* EG \cong BH \times _f G$.
	\item If $ H \leq G$ and  $ P \to M$ is a principal $ G$-bundle, then structure group of  $ P$ reduces to  $ H$ iff the classifying map  $ f: M \to BG$ (after homotopy) factors through $ M \to BH$.
\end{enumerate}

A different view of characteristic classes:
\begin{thm}
	$ H^* (BO(n); \zz /2) \cong \zz /2[w_1,\ldots,w_n]$ where $ w_i$ has degree $ i$.

	$ H^* (BU(n); \zz) \cong \zz [c_1,\ldots,c_n]$ where $ c_i$ has degree $ 2i$.
\end{thm}
We can use this theorem to define characteristic classes of an $ \rr^{n}$-bundle $ E \to M$. There exists an associated $ O(n)$-bundle. By theorem 13 there exists a map  $ f: M \to BO(n)$ s.t.\ $ \mathcal{ F}(E) \cong f^* EO(n)$. Define the $ i$th Steifel-Whitney class of  $ E$ to be
 \begin{align*}
	w_i(E) = f^* w_i.
\end{align*}
Simiarly for Chern classes.


\begin{thm}
	$ H^* (BSO(2n+1); \zz) \cong \zz [p_1,\ldots,p_n] \oplus $ Torsion where Torsion is $ \beta(H^{n}(BSO(2n+1); \zz /2))$.

	$ H^* (BSO(2n);\zz) \cong \zz[p_1,\ldots,p_n, e] / \langle e ^2 =p_n  \rangle \oplus $ Torsion.
\end{thm}
\end{document}
