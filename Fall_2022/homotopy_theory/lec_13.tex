\documentclass[12pt,class=article,crop=false]{standalone} 
\newcommand{\alert}[1]{{\bf \color{red} [Alert:] #1}}
\newcommand{\todo}[1]{{\bf \color{orange} [TODO:] #1}}
\newcommand{\real}[1][]{\mathbb{R}^{#1}}
\newcommand{\myeqn}[1]{(\ref{#1})}
\newcommand{\myex}[1]{Example \ref{#1}}
\newcommand{\defeq}{\stackrel{\mathrm{def}}{=}}
\newcommand{\parder}[2]{\frac{\partial #1}{\partial #2}}
\newcommand{\Lie}[3][]{\mathsf{L}_{#3}^{#1} #2}
\newcommand{\LieA}[1]{\mathsf{Lie}(#1)}
\newcommand{\lieder}[2]{\mathcal{L}_{#2} #1}
\renewcommand{\t}{^{\mbox{\tiny\sf T}}}
\newcommand{\trans}{^{\mbox{\tiny\sf T}}}
\newcommand{\markup}[1]{\{\textbf{#1}\}}
\newcommand{\msub}[1]{_\mathrm{#1}}
\newcommand{\msup}[1]{^\mathrm{#1}}
\newcommand{\inv}[1]{#1^{-1}}
\newcommand{\pinv}[1]{{#1}^{+}}
\newcommand{\myfracA}[2]{\displaystyle{\frac{#1}{#2}}}
\newcommand{\myfracB}[2]{{#1}/{#2}}
\newcommand{\mydiffA}[1]{\dot{#1}}
\newcommand{\mydiffB}[2]{\myfracA{\mathrm{d}{#1}}{\mathrm{d}{#2}}}
\newcommand{\ball}[2]{\mathcal{B}_{#1}\left(#2\right)}
\newcommand{\acos}[1]{\cos^{-1}\left(#1\right)}
\newcommand{\asin}[1]{\sin^{-1}\left(#1\right)}
\newcommand{\mani}{\mathcal{M}}
\newcommand{\tang}[2]{\mathsf{T}_{#1} #2}
\newcommand{\LieB}[2]{[ #1, #2 ]}
\newcommand{\LieBad}[3][]{\mathsf{ad}_{#2}^{#1} #3}
\newcommand{\ReachVT}{\mathcal{R}^V_T}
\newcommand{\ReachVt}{\mathcal{R}^V_t}
\newcommand{\ReachVTe}{\mathcal{R}^V_{\le T}}
\newcommand{\ReachT}{\mathcal{R}_T}
\newcommand{\Reacht}{\mathcal{R}_t}
\newcommand{\ReachTe}{\mathcal{R}_{\le T}}
\newcommand{\accLA}[1]{\mathsf{Lie}(#1)}
\newcommand{\accD}{\Delta_{\mathcal{F}}}
\newcommand{\accSA}{\mathsf{Lie}(\mathcal{G},f)}
\newcommand{\accDS}{\Delta_{\mathcal{G}}}
\newcommand{\eval}[3]{\mathsf{Ev}^{#2}_{#1}\left( #3 \right)}
\newcommand{\stlc}{\textsc{stlc}}
\newcommand{\clf}{\textsc{clf}}
\newcommand{\jqlf}{\textsc{jqlf}}
\newcommand{\dlas}{\textsc{dlas}}
\newcommand{\Ad}[2]{\mathsf{Ad}_{#1} #2}
\newcommand{\xe}{\ensuremath{x_e}}
\newcommand{\lebg}[1]{\mathcal{L}_{#1}}
\newcommand{\lebgx}[1]{\mathcal{L}_{#1 \mathrm{e}}}
\newcommand{\dom}{D}
\newcommand{\domT}{[t_0,\infty) \times D}
\newcommand{\rarrow}{\rightarrow}
\renewcommand{\d}{\mathrm{d}}
\renewcommand{\Re}{\mathbb{R}}
\newcommand{\C}{\mathrm{C}}

\newcommand{\QED}{{\unskip\nobreak\hfil\penalty50\hskip2em\vadjust{}
		\nobreak\hfil$\Box$\parfillskip=0pt\finalhyphendemerits=0\par}\vspace{0.1cm}}
\newcommand{\eoEx}{{\unskip\nobreak\hfil\penalty50\hskip0em\vadjust{}
		\nobreak\hfil$\Large\Diamond$\parfillskip=0pt\finalhyphendemerits=0\par}\vspace{0.1cm}}

\newcommand{\sgn}{\ensuremath{\operatorname{sgn}}}
\newcommand{\sat}{\ensuremath{\operatorname{sat}}}

\newcommand{\half}{\frac{1}{2}}
\newcommand{\shalf}{\mbox{$\frac{1}{2}$}}
\newcommand{\marcom}[1]{\marginpar{\footnotesize #1}}
\newcommand{\der}{\mathrm{D}}
\newcommand{\e}{\mathrm{e}}
\newcommand{\dt}{\mathrm{d}t}

\newcommand{\cA}{\ensuremath{\mathcal{A}}}
\newcommand{\cB}{\ensuremath{\mathcal{B}}}
\newcommand{\cG}{\ensuremath{\mathcal{G}}}
\newcommand{\cK}{\ensuremath{\mathcal{K}}}
\newcommand{\cW}{\ensuremath{\mathcal{W}}}
\newcommand{\cZ}{\ensuremath{\mathcal{Z}}}
\newcommand{\cS}{\ensuremath{\mathcal{S}}}
\newcommand{\cD}{\ensuremath{\mathcal{D}}}
\newcommand{\cP}{\ensuremath{\mathcal{P}}}
\newcommand{\cV}{\ensuremath{\mathcal{V}}}
\newcommand{\cL}{\ensuremath{\mathcal{L}}}
\newcommand{\cN}{\ensuremath{\mathcal{N}}}
\newcommand{\cI}{\ensuremath{\mathcal{I}}}
\newcommand{\cR}{\ensuremath{\mathcal{R}}}
\newcommand{\cM}{\ensuremath{\mathcal{M}}}
\newcommand{\cC}{\ensuremath{\mathcal{C}}}
\newcommand{\cF}{\ensuremath{\mathcal{F}}}
\newcommand{\cH}{\ensuremath{\mathcal{H}}}
\newcommand{\cO}{\ensuremath{\mathcal{O}}}
\newcommand{\cX}{\ensuremath{\mathcal{X}}}
\newcommand{\cY}{\ensuremath{\mathcal{Y}}}
\newcommand{\Ci}{\ensuremath{\mathcal{C}^\infty}}
\newcommand{\ISS}{\textsc{iss}}
\newcommand{\LISS}{\textsc{liss}}
\newcommand{\GAS}{\textsc{gas}}
\newcommand{\GS}{\textsc{gs}}
\newcommand{\LES}{\textsc{les}}
\newcommand{\GUAS}{\textsc{guas}}
\newcommand{\BIBO}{\textsc{bibo}}
\newcommand{\spec}{\ensuremath{\operatorname{spec}}}
\newcommand{\spn}{\ensuremath{\operatorname{span}}}
\renewcommand{\i}{\mathrm{i\,}}

\renewcommand{\implies}{\Rightarrow}

\renewcommand{\theenumi}{$\roman{enumi})$}
\renewcommand{\labelenumi}{\theenumi}

\font\ptmten=zptmcmrm scaled 1200
\newcommand{\w}{\mbox{{\ptmten w}}}
\newcommand{\z}{\mbox{{\ptmten z}}}
\renewcommand{\Re}{\mathbb{R}}

\newcommand{\cl}{\operatorname{cl}}
\newcommand{\intr}{\operatorname{int}}
\newcommand{\rank}{\operatorname{rank}}
\newcommand{\co}{\operatorname{co}}
\newcommand{\aff}{\operatorname{aff}}

\theoremstyle{plain}
\newtheorem{theorem}{Theorem}[chapter]
\newtheorem{claim}[theorem]{Claim}
\newtheorem{corollary}[theorem]{Corollary}
\newtheorem{prop}[theorem]{Proposition}
\newtheorem{fact}[theorem]{Fact}
\newtheorem{lemma}[theorem]{Lemma}

\newtheorem{remark}{Remark}[chapter]

\theoremstyle{definition}
\newtheorem{assume}[theorem]{Assumption}
\newtheorem{defn}[theorem]{Definition}
\newtheorem{problem}[theorem]{Problem}
\newtheorem{exercise}{Exercise}
\newtheorem{example}[theorem]{Example}


\begin{document}
\section{Obstruction Theory Revisited}

Suppose $ A \subseteq X$ and a map $ f: A \to Y$, can we extend $ f$ to a map  $ X \to Y$?

As usual we assume
\begin{enumerate}[label=(\alph*)]
	\item $ (X,A)$ is a relative CW-complex  \emph{i.e.} $ X^{(-1)} = A$.
	\item $ Y$ is  $ n$-simple for all  $ n$,  \emph{i.e.} $ \pi_1(Y)$ acts trivially on $ \pi_n(Y)$ which simplies $ \pi_n(Y) = [S^{n},Y]$ (unbased).
\end{enumerate}

\begin{thm}
Given $ (X,A)$ satisfying  a) and  $ Y$ satisfying  $ b)$ and  $ f: X^{(n)} \to Y$. Then
\begin{enumerate}[label=(\arabic*)]
	\item there exists a cocycle
		\begin{align*}
			\widetilde{ \sigma}(f) \in C^{n+1}(X,A; \pi_n(Y))
		\end{align*}
		which vanishes iff $ f$ extends to  $ X^{(n+1)}$.
	\item $ \sigma(f) = [\widetilde{ \sigma}(f)] \in H^{n+1}(X,A; \pi_n(Y))$ vanishes $ \iff$ $ f|_{x^{(n-1)}}$ extends to $ X^{(n+1)}$.
\end{enumerate}
\end{thm}
\begin{proof}
JUst like in Section A,
\begin{align*}
	\widetilde{ \sigma}(f): C_{n+1}^{CW}(X,A) \to \pi_n(Y)
\end{align*}
is defined as follows:
$ e_i^{n+1}$ is attached to $ X^{(n)}$ by $ \phi_i: S^{n} \to Y$ so
\begin{align*}
	\widetilde{ \sigma}(e_i^{n+1}) = [f \circ \phi_i] \in [S^{n},Y] = \pi_n(Y)
\end{align*}
Exercise:
\begin{enumerate}[label=(\arabic*)]
	\item $ \widetilde{ \sigma}(f) = 0 \iff f$ extends to $ X^{(n+1)}$.
	\item $ \widetilde{ \sigma}(f)$ is unchanged if you homotop $ f$.
	\item  $ \delta \widetilde{ \sigma}(f) = 0$.
	\item Given $ f,g : X^{(n)} to Y$ s.t.\ $ f=g$ on  $ X^{(n-1)}$ then there exists $ \tau(f,g) iin C^{n}(X,A;\pi_n(Y))$ s.t.\ 
		\begin{align*}
			\delta \tau(f,g) = \widetilde{ \sigma}(f) - \widetilde{ \sigma}(g)
		\end{align*}
	\item By varying the homotopy class of $ f$ on  $ X^{(n)}$ relative to $ X^{(n-1)}$, we can change $ \widetilde{ \sigma}(f)$ by an arbitrary coboundary.

The theorem follows.
\end{enumerate}
\end{proof}

\begin{thm}
Let $ f,g: X \to Y$ be given (satisfying a), b)), and $ H: X^{(n)} \times I \to Y$ a homotopy from $ f|_{X^{(n)}} \to g|_{X^{(n)}}$. Then the obstruction to extending $ H$ to  $ X^{(n+1)} \times I \to Y $ lies in $ H^{n}(X,A;\pi_n(Y))$.
\end{thm}

\begin{proof}
Theorem 19 says we get an obstruction in $ H^{n+1}(X \times I,((A \times I) \cup (X \times \{0,1\} ));\pi_n(Y))$.

Let $ U_1 = X \times [0, \frac{3}{4}]$ and $ V_1 = (X \times \{0\} ) \cup  (A \times [0, \frac{3}{4}])$, $ U_2 = X \times [\frac{1}{4},1]$, $ V_2 = (A \times [\frac{1}{4},1] \cup (X \times \{1\} ))$.

By Lemma I.9, since $ (X,A)$ is a NDR-pair, we know  $ V_1$ is a retract of $ U_1$. So $ H^{n}(U_1,V_1) = 0$.
\begin{align*}
	0=H^{n}(U_1,V_1) \oplus  H^{n}(U_2,V_2) \to H^{n}(U_1 \cap U_2, V_1 \cap V_2) \to H^{n+1}(U_1 \cup U_2, V_1 \cup V_2) \to H^{n+1}(U_1,V_1) \oplus H^{n+1}(U_2,V_2) = 0
\end{align*}
So $H^{n}(X,A) \cong H^{n}(X \times [\frac{1}{4},\frac{3}{4}], A \times [\frac{1}{4},\frac{3}{4}]) \cong H^{n+1}(U_1 \cup U_2,V_1 \cup V_2)$.
\end{proof}
\begin{thm}
	Let $ (X,A)$ be a relative CW complex and  $ Y$  $ n$-simple space  $ \ \forall \ n$. If $ \pi_k(Y) \ \forall \ k<n-1$, then for any $ f: A \to Y$, there exists an extension $ \widetilde{ f}: X^{(n)} \to Y$ and the obstruction $ [ \widetilde{ \sigma}(\widetilde{ f})]$  only depends on $ f$ and is denoted  $ \gamma^{n+1}(f)$, the \allbold{primary obstruction}.

	Moreover, if $ g: (X',A') \to (X,A)$, then
	\begin{align*}
		g^* ( \gamma^{n+1}(f)) = \gamma^{n+1}(f \circ g).
	\end{align*}
\end{thm}
\begin{proof}
Same as proof of Theorem 4.
\end{proof}

\begin{thm}[Brown Representation Theorem]
Let $ (X,A)$ be a relative CW-pair, there is a natural bijection
 \begin{align*}
	 [(X,A),K(\pi,n),x_0] \longleftrightarrow H^{n}(X,A; \pi)
\end{align*}
\end{thm}

\begin{proof}
BY Hurewicz, $ H_k(K(\pi,n)) = 0 \ \forall \ k<n$ and $ H_n(K(\pi,n)) = \pi$. By the Universal Coefficients Theorem,
\begin{align*}
	H^{n}(K(\pi,n);\pi) \cong \Hom( H_n(K(\pi,n)), \pi) \oplus \ext(H_{n-1}(K(\pi,n)),\pi) = \Hom( \pi, \pi).
\end{align*}
as $ \ext$ is 0. Let $ \iota \in H^{n}(K(\pi,n);\pi)$ corresponds to $ 1_{\pi}$. Define $ \psi: [(X,A),(K(\pi,n);x_0)] \to H^{n}(X,A;\pi), f\mapsto f^* \iota $.

Note since $ \pi_n(K(\pi,n)) =0$ for $ k<n$. The first obstruction to homotopying a map  $ f:(X,A) \to (K(\pi,n),x_0)$ to be constant lives in $ H^{n}((X,A);\pi)$.
\begin{claim}
This obstruction is $ \psi(f)$.
\end{claim}
\begin{proof}
By naturality, it suffices to check that $ \iota $ is the primary obstruction to homotopying the identity map on $ K(\pi,n)$ to a constant map. 

We know $ (K(\pi,n))^{(n-1)} = \{x_0\} $. So the identity and constant map agree on  $ n-1$ skeleton. The  $ n$-cell  $ e_i^{n}$ corresponding to a generator of $ \pi=\pi_n(K(\pi,n))$.

\begin{claim}
$ \psi$ is onto.
\end{claim}
Let $  \alpha \in H^{n}(X,A;\pi)$ so there exists $ \widetilde{ \alpha} \in C^{n}(X,A; \pi)$ s.t.\ $ \alpha=[ \widetilde{ \alpha}]$, $ \widetilde{ \alpha}: C_n(X,A) \to \pi$, define $ f_{ \alpha}$ to be constant on $ X^{(n-1)}$ and for each $ n$-cell  $ e_i^{n}$ of $ X$. Let  $ f_{ \alpha}: e_i^{n} to K(\pi,n)$. Represent $ [f_{ \alpha}(e_i^{n})] = \widetilde{ \alpha}(e_i^{n}) \in \pi = \pi_n(K(\pi,n))$. This gives $ f_{ \alpha}$ on $ X^{(n)}$. For each $ e_j^{n+1}$,
\begin{align*}
	\widetilde{ \alpha}(\partial e_j^{n+1}) = \delta \widetilde{ \alpha}(e_j^{n+1}) = 0
\end{align*}
So $ f_{ \alpha}(\partial  e_j^{n+1})$ is nullhomotopic. we can extend $ f_{ \alpha}$ over $ e_j^{n+1}$ so $ f_{ \alpha}$ extends to $ X^{(n+1)}$. Since $ \pi_k(K(\pi,n)) = 0 \ \forall \ k>n$. So no obstruction to extending $ f_{ \alpha}$  to all of $ X$. As in the proof of first claim, we see  $ \psi(f_{ \alpha}) = f_{ \alphaa}^*  \iota = \alpha$.

\begin{claim}
$ \psi$ is injective.
\end{claim}

\begin{proof}
$ f,g: (X,A) \to (K(\pi,n),x_0)$ s.t.\ $ \psi(f) = \psi(g)$. The primary and only obstruction to a homotopy from $ f \to g$ lives in $ \theta \in H^{n}(X,A;\pi) \cong \pi_n(K(\pi,n))$. If we evaluate $ \widetilde{ \theta}$ on $ e_i^{n}$, we get

so $ \widetilde{ \theta}(e_i^{n}) = f(e_i^{n}) - g(e_i^{n}) \in \pi_n(K(\pi,n)) = \iota (f_*-g_*)(e_i^{n}) = (f^* \iota -g^* \iota ) e_i^{n = \psi(f) - \psi(g)}e_i^{n} = 0$  since they have different orientations. Thus $ f \simeq g$.
\end{proof}
\end{proof}
\end{proof}
\end{document}
