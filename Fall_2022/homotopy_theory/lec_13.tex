\documentclass[12pt,class=article,crop=false]{standalone} 
%Fall 2022
% Some basic packages
\usepackage{standalone}[subpreambles=true]
\usepackage[utf8]{inputenc}
\usepackage[T1]{fontenc}
\usepackage{textcomp}
\usepackage[english]{babel}
\usepackage{url}
\usepackage{graphicx}
%\usepackage{quiver}
\usepackage{float}
\usepackage{enumitem}
\usepackage{lmodern}
\usepackage{comment}
\usepackage{hyperref}
\usepackage[usenames,svgnames,dvipsnames]{xcolor}
\usepackage[margin=1in]{geometry}
\usepackage{pdfpages}

\pdfminorversion=7

% Don't indent paragraphs, leave some space between them
\usepackage{parskip}

% Hide page number when page is empty
\usepackage{emptypage}
\usepackage{subcaption}
\usepackage{multicol}
\usepackage[b]{esvect}

% Math stuff
\usepackage{amsmath, amsfonts, mathtools, amsthm, amssymb}
\usepackage{bbm}
\usepackage{stmaryrd}
\allowdisplaybreaks

% Fancy script capitals
\usepackage{mathrsfs}
\usepackage{cancel}
% Bold math
\usepackage{bm}
% Some shortcuts
\newcommand{\rr}{\ensuremath{\mathbb{R}}}
\newcommand{\zz}{\ensuremath{\mathbb{Z}}}
\newcommand{\qq}{\ensuremath{\mathbb{Q}}}
\newcommand{\nn}{\ensuremath{\mathbb{N}}}
\newcommand{\ff}{\ensuremath{\mathbb{F}}}
\newcommand{\cc}{\ensuremath{\mathbb{C}}}
\newcommand{\ee}{\ensuremath{\mathbb{E}}}
\newcommand{\hh}{\ensuremath{\mathbb{H}}}
\renewcommand\O{\ensuremath{\emptyset}}
\newcommand{\norm}[1]{{\left\lVert{#1}\right\rVert}}
\newcommand{\dbracket}[1]{{\left\llbracket{#1}\right\rrbracket}}
\newcommand{\ve}[1]{{\bm{#1}}}
\newcommand\allbold[1]{{\boldmath\textbf{#1}}}
\DeclareMathOperator{\lcm}{lcm}
\DeclareMathOperator{\im}{im}
\DeclareMathOperator{\coim}{coim}
\DeclareMathOperator{\dom}{dom}
\DeclareMathOperator{\tr}{tr}
\DeclareMathOperator{\rank}{rank}
\DeclareMathOperator*{\var}{Var}
\DeclareMathOperator*{\ev}{E}
\DeclareMathOperator{\dg}{deg}
\DeclareMathOperator{\aff}{aff}
\DeclareMathOperator{\conv}{conv}
\DeclareMathOperator{\inte}{int}
\DeclareMathOperator*{\argmin}{argmin}
\DeclareMathOperator*{\argmax}{argmax}
\DeclareMathOperator{\graph}{graph}
\DeclareMathOperator{\sgn}{sgn}
\DeclareMathOperator*{\Rep}{Rep}
\DeclareMathOperator{\Proj}{Proj}
\DeclareMathOperator{\mat}{mat}
\DeclareMathOperator{\diag}{diag}
\DeclareMathOperator{\aut}{Aut}
\DeclareMathOperator{\gal}{Gal}
\DeclareMathOperator{\inn}{Inn}
\DeclareMathOperator{\edm}{End}
\DeclareMathOperator{\Hom}{Hom}
\DeclareMathOperator{\ext}{Ext}
\DeclareMathOperator{\tor}{Tor}
\DeclareMathOperator{\Span}{Span}
\DeclareMathOperator{\Stab}{Stab}
\DeclareMathOperator{\cont}{cont}
\DeclareMathOperator{\Ann}{Ann}
\DeclareMathOperator{\Div}{div}
\DeclareMathOperator{\curl}{curl}
\DeclareMathOperator{\nat}{Nat}
\DeclareMathOperator{\gr}{Gr}
\DeclareMathOperator{\vect}{Vect}
\DeclareMathOperator{\id}{id}
\DeclareMathOperator{\Mod}{Mod}
\DeclareMathOperator{\sign}{sign}
\DeclareMathOperator{\Surf}{Surf}
\DeclareMathOperator{\fcone}{fcone}
\DeclareMathOperator{\Rot}{Rot}
\DeclareMathOperator{\grad}{grad}
\DeclareMathOperator{\atan2}{atan2}
\DeclareMathOperator{\Ric}{Ric}
\let\vec\relax
\DeclareMathOperator{\vec}{vec}
\let\Re\relax
\DeclareMathOperator{\Re}{Re}
\let\Im\relax
\DeclareMathOperator{\Im}{Im}
% Put x \to \infty below \lim
\let\svlim\lim\def\lim{\svlim\limits}

%wide hat
\usepackage{scalerel,stackengine}
\stackMath
\newcommand*\wh[1]{%
\savestack{\tmpbox}{\stretchto{%
  \scaleto{%
    \scalerel*[\widthof{\ensuremath{#1}}]{\kern-.6pt\bigwedge\kern-.6pt}%
    {\rule[-\textheight/2]{1ex}{\textheight}}%WIDTH-LIMITED BIG WEDGE
  }{\textheight}% 
}{0.5ex}}%
\stackon[1pt]{#1}{\tmpbox}%
}
\parskip 1ex

%Make implies and impliedby shorter
\let\implies\Rightarrow
\let\impliedby\Leftarrow
\let\iff\Leftrightarrow
\let\epsilon\varepsilon

% Add \contra symbol to denote contradiction
\usepackage{stmaryrd} % for \lightning
\newcommand\contra{\scalebox{1.5}{$\lightning$}}

% \let\phi\varphi

% Command for short corrections
% Usage: 1+1=\correct{3}{2}

\definecolor{correct}{HTML}{009900}
\newcommand\correct[2]{\ensuremath{\:}{\color{red}{#1}}\ensuremath{\to }{\color{correct}{#2}}\ensuremath{\:}}
\newcommand\green[1]{{\color{correct}{#1}}}

% horizontal rule
\newcommand\hr{
    \noindent\rule[0.5ex]{\linewidth}{0.5pt}
}

% hide parts
\newcommand\hide[1]{}

% si unitx
\usepackage{siunitx}
\sisetup{locale = FR}

%allows pmatrix to stretch
\makeatletter
\renewcommand*\env@matrix[1][\arraystretch]{%
  \edef\arraystretch{#1}%
  \hskip -\arraycolsep
  \let\@ifnextchar\new@ifnextchar
  \array{*\c@MaxMatrixCols c}}
\makeatother

\renewcommand{\arraystretch}{0.8}

\renewcommand{\baselinestretch}{1.5}

\usepackage{graphics}
\usepackage{epstopdf}

\RequirePackage{hyperref}
%%
%% Add support for color in order to color the hyperlinks.
%% 
\hypersetup{
  colorlinks = true,
  urlcolor = blue,
  citecolor = blue
}
%%fakesection Links
\hypersetup{
    colorlinks,
    linkcolor={red!50!black},
    citecolor={green!50!black},
    urlcolor={blue!80!black}
}
%customization of cleveref
\RequirePackage[capitalize,nameinlink]{cleveref}[0.19]

% Per SIAM Style Manual, "section" should be lowercase
\crefname{section}{section}{sections}
\crefname{subsection}{subsection}{subsections}
\Crefname{section}{Section}{Sections}
\Crefname{subsection}{Subsection}{Subsections}

% Per SIAM Style Manual, "Figure" should be spelled out in references
\Crefname{figure}{Figure}{Figures}

% Per SIAM Style Manual, don't say equation in front on an equation.
\crefformat{equation}{\textup{#2(#1)#3}}
\crefrangeformat{equation}{\textup{#3(#1)#4--#5(#2)#6}}
\crefmultiformat{equation}{\textup{#2(#1)#3}}{ and \textup{#2(#1)#3}}
{, \textup{#2(#1)#3}}{, and \textup{#2(#1)#3}}
\crefrangemultiformat{equation}{\textup{#3(#1)#4--#5(#2)#6}}%
{ and \textup{#3(#1)#4--#5(#2)#6}}{, \textup{#3(#1)#4--#5(#2)#6}}{, and \textup{#3(#1)#4--#5(#2)#6}}

% But spell it out at the beginning of a sentence.
\Crefformat{equation}{#2Equation~\textup{(#1)}#3}
\Crefrangeformat{equation}{Equations~\textup{#3(#1)#4--#5(#2)#6}}
\Crefmultiformat{equation}{Equations~\textup{#2(#1)#3}}{ and \textup{#2(#1)#3}}
{, \textup{#2(#1)#3}}{, and \textup{#2(#1)#3}}
\Crefrangemultiformat{equation}{Equations~\textup{#3(#1)#4--#5(#2)#6}}%
{ and \textup{#3(#1)#4--#5(#2)#6}}{, \textup{#3(#1)#4--#5(#2)#6}}{, and \textup{#3(#1)#4--#5(#2)#6}}

% Make number non-italic in any environment.
\crefdefaultlabelformat{#2\textup{#1}#3}

% Environments
\makeatother
% For box around Definition, Theorem, \ldots
%%fakesection Theorems
\usepackage{thmtools}
\usepackage[framemethod=TikZ]{mdframed}

\theoremstyle{definition}
\mdfdefinestyle{mdbluebox}{%
	roundcorner = 10pt,
	linewidth=1pt,
	skipabove=12pt,
	innerbottommargin=9pt,
	skipbelow=2pt,
	nobreak=true,
	linecolor=blue,
	backgroundcolor=TealBlue!5,
}
\declaretheoremstyle[
	headfont=\sffamily\bfseries\color{MidnightBlue},
	mdframed={style=mdbluebox},
	headpunct={\\[3pt]},
	postheadspace={0pt}
]{thmbluebox}

\mdfdefinestyle{mdredbox}{%
	linewidth=0.5pt,
	skipabove=12pt,
	frametitleaboveskip=5pt,
	frametitlebelowskip=0pt,
	skipbelow=2pt,
	frametitlefont=\bfseries,
	innertopmargin=4pt,
	innerbottommargin=8pt,
	nobreak=false,
	linecolor=RawSienna,
	backgroundcolor=Salmon!5,
}
\declaretheoremstyle[
	headfont=\bfseries\color{RawSienna},
	mdframed={style=mdredbox},
	headpunct={\\[3pt]},
	postheadspace={0pt},
]{thmredbox}

\declaretheorem[%
style=thmbluebox,name=Theorem,numberwithin=section]{thm}
\declaretheorem[style=thmbluebox,name=Lemma,sibling=thm]{lem}
\declaretheorem[style=thmbluebox,name=Proposition,sibling=thm]{prop}
\declaretheorem[style=thmbluebox,name=Corollary,sibling=thm]{coro}
\declaretheorem[style=thmredbox,name=Example,sibling=thm]{eg}

\mdfdefinestyle{mdgreenbox}{%
	roundcorner = 10pt,
	linewidth=1pt,
	skipabove=12pt,
	innerbottommargin=9pt,
	skipbelow=2pt,
	nobreak=true,
	linecolor=ForestGreen,
	backgroundcolor=ForestGreen!5,
}

\declaretheoremstyle[
	headfont=\bfseries\sffamily\color{ForestGreen!70!black},
	bodyfont=\normalfont,
	spaceabove=2pt,
	spacebelow=1pt,
	mdframed={style=mdgreenbox},
	headpunct={ --- },
]{thmgreenbox}

\declaretheorem[style=thmgreenbox,name=Definition,sibling=thm]{defn}

\mdfdefinestyle{mdgreenboxsq}{%
	linewidth=1pt,
	skipabove=12pt,
	innerbottommargin=9pt,
	skipbelow=2pt,
	nobreak=true,
	linecolor=ForestGreen,
	backgroundcolor=ForestGreen!5,
}
\declaretheoremstyle[
	headfont=\bfseries\sffamily\color{ForestGreen!70!black},
	bodyfont=\normalfont,
	spaceabove=2pt,
	spacebelow=1pt,
	mdframed={style=mdgreenboxsq},
	headpunct={},
]{thmgreenboxsq}
\declaretheoremstyle[
	headfont=\bfseries\sffamily\color{ForestGreen!70!black},
	bodyfont=\normalfont,
	spaceabove=2pt,
	spacebelow=1pt,
	mdframed={style=mdgreenboxsq},
	headpunct={},
]{thmgreenboxsq*}

\mdfdefinestyle{mdblackbox}{%
	skipabove=8pt,
	linewidth=3pt,
	rightline=false,
	leftline=true,
	topline=false,
	bottomline=false,
	linecolor=black,
	backgroundcolor=RedViolet!5!gray!5,
}
\declaretheoremstyle[
	headfont=\bfseries,
	bodyfont=\normalfont\small,
	spaceabove=0pt,
	spacebelow=0pt,
	mdframed={style=mdblackbox}
]{thmblackbox}

\theoremstyle{plain}
\declaretheorem[name=Question,sibling=thm,style=thmblackbox]{ques}
\declaretheorem[name=Remark,sibling=thm,style=thmgreenboxsq]{remark}
\declaretheorem[name=Remark,sibling=thm,style=thmgreenboxsq*]{remark*}
\newtheorem{ass}[thm]{Assumptions}

\theoremstyle{definition}
\newtheorem*{problem}{Problem}
\newtheorem{claim}[thm]{Claim}
\theoremstyle{remark}
\newtheorem*{case}{Case}
\newtheorem*{notation}{Notation}
\newtheorem*{note}{Note}
\newtheorem*{motivation}{Motivation}
\newtheorem*{intuition}{Intuition}
\newtheorem*{conjecture}{Conjecture}

% Make section starts with 1 for report type
%\renewcommand\thesection{\arabic{section}}

% End example and intermezzo environments with a small diamond (just like proof
% environments end with a small square)
\usepackage{etoolbox}
\AtEndEnvironment{vb}{\null\hfill$\diamond$}%
\AtEndEnvironment{intermezzo}{\null\hfill$\diamond$}%
% \AtEndEnvironment{opmerking}{\null\hfill$\diamond$}%

% Fix some spacing
% http://tex.stackexchange.com/questions/22119/how-can-i-change-the-spacing-before-theorems-with-amsthm
\makeatletter
\def\thm@space@setup{%
  \thm@preskip=\parskip \thm@postskip=0pt
}

% Fix some stuff
% %http://tex.stackexchange.com/questions/76273/multiple-pdfs-with-page-group-included-in-a-single-page-warning
\pdfsuppresswarningpagegroup=1


% My name
\author{Jaden Wang}



\begin{document}
\section{Obstruction Theory Revisited}

Suppose $ A \subseteq X$ and a map $ f: A \to Y$, can we extend $ f$ to a map  $ X \to Y$?

As usual we assume
\begin{enumerate}[label=(\alph*)]
	\item $ (X,A)$ is a relative CW-complex  \emph{i.e.} $ X^{(-1)} = A$.
	\item $ Y$ is  $ n$-simple for all  $ n$,  \emph{i.e.} $ \pi_1(Y)$ acts trivially on $ \pi_n(Y)$ which simplies $ \pi_n(Y) = [S^{n},Y]$ (unbased).
\end{enumerate}

\begin{thm}
Given $ (X,A)$ satisfying  a) and  $ Y$ satisfying  $ b)$ and  $ f: X^{(n)} \to Y$. Then
\begin{enumerate}[label=(\arabic*)]
	\item there exists a cocycle
		\begin{align*}
			\widetilde{ \sigma}(f) \in C^{n+1}(X,A; \pi_n(Y))
		\end{align*}
		which vanishes iff $ f$ extends to  $ X^{(n+1)}$.
	\item $ \sigma(f) = [\widetilde{ \sigma}(f)] \in H^{n+1}(X,A; \pi_n(Y))$ vanishes $ \iff$ $ f|_{x^{(n-1)}}$ extends to $ X^{(n+1)}$.
\end{enumerate}
\end{thm}
\begin{proof}
JUst like in Section A,
\begin{align*}
	\widetilde{ \sigma}(f): C_{n+1}^{CW}(X,A) \to \pi_n(Y)
\end{align*}
is defined as follows:
$ e_i^{n+1}$ is attached to $ X^{(n)}$ by $ \phi_i: S^{n} \to Y$ so
\begin{align*}
	\widetilde{ \sigma}(e_i^{n+1}) = [f \circ \phi_i] \in [S^{n},Y] = \pi_n(Y)
\end{align*}
Exercise:
\begin{enumerate}[label=(\arabic*)]
	\item $ \widetilde{ \sigma}(f) = 0 \iff f$ extends to $ X^{(n+1)}$.
	\item $ \widetilde{ \sigma}(f)$ is unchanged if you homotop $ f$.
	\item  $ \delta \widetilde{ \sigma}(f) = 0$.
	\item Given $ f,g : X^{(n)} to Y$ s.t.\ $ f=g$ on  $ X^{(n-1)}$ then there exists $ \tau(f,g) iin C^{n}(X,A;\pi_n(Y))$ s.t.\ 
		\begin{align*}
			\delta \tau(f,g) = \widetilde{ \sigma}(f) - \widetilde{ \sigma}(g)
		\end{align*}
	\item By varying the homotopy class of $ f$ on  $ X^{(n)}$ relative to $ X^{(n-1)}$, we can change $ \widetilde{ \sigma}(f)$ by an arbitrary coboundary.

The theorem follows.
\end{enumerate}
\end{proof}

\begin{thm}
Let $ f,g: X \to Y$ be given (satisfying a), b)), and $ H: X^{(n)} \times I \to Y$ a homotopy from $ f|_{X^{(n)}} \to g|_{X^{(n)}}$. Then the obstruction to extending $ H$ to  $ X^{(n+1)} \times I \to Y $ lies in $ H^{n}(X,A;\pi_n(Y))$.
\end{thm}

\begin{proof}
Theorem 19 says we get an obstruction in $ H^{n+1}(X \times I,((A \times I) \cup (X \times \{0,1\} ));\pi_n(Y))$.

Let $ U_1 = X \times [0, \frac{3}{4}]$ and $ V_1 = (X \times \{0\} ) \cup  (A \times [0, \frac{3}{4}])$, $ U_2 = X \times [\frac{1}{4},1]$, $ V_2 = (A \times [\frac{1}{4},1] \cup (X \times \{1\} ))$.

By Lemma I.9, since $ (X,A)$ is a NDR-pair, we know  $ V_1$ is a retract of $ U_1$. So $ H^{n}(U_1,V_1) = 0$.
\begin{align*}
	0=H^{n}(U_1,V_1) \oplus  H^{n}(U_2,V_2) \to H^{n}(U_1 \cap U_2, V_1 \cap V_2) \to H^{n+1}(U_1 \cup U_2, V_1 \cup V_2) \to H^{n+1}(U_1,V_1) \oplus H^{n+1}(U_2,V_2) = 0
\end{align*}
So $H^{n}(X,A) \cong H^{n}(X \times [\frac{1}{4},\frac{3}{4}], A \times [\frac{1}{4},\frac{3}{4}]) \cong H^{n+1}(U_1 \cup U_2,V_1 \cup V_2)$.
\end{proof}
\begin{thm}
	Let $ (X,A)$ be a relative CW complex and  $ Y$  $ n$-simple space  $ \ \forall \ n$. If $ \pi_k(Y) \ \forall \ k<n-1$, then for any $ f: A \to Y$, there exists an extension $ \widetilde{ f}: X^{(n)} \to Y$ and the obstruction $ [ \widetilde{ \sigma}(\widetilde{ f})]$  only depends on $ f$ and is denoted  $ \gamma^{n+1}(f)$, the \allbold{primary obstruction}.

	Moreover, if $ g: (X',A') \to (X,A)$, then
	\begin{align*}
		g^* ( \gamma^{n+1}(f)) = \gamma^{n+1}(f \circ g).
	\end{align*}
\end{thm}
\begin{proof}
Same as proof of Theorem 4.
\end{proof}

\begin{thm}[Brown Representation Theorem]
Let $ (X,A)$ be a relative CW-pair, there is a natural bijection
 \begin{align*}
	 [(X,A),K(\pi,n),x_0] \longleftrightarrow H^{n}(X,A; \pi)
\end{align*}
\end{thm}

\begin{proof}
BY Hurewicz, $ H_k(K(\pi,n)) = 0 \ \forall \ k<n$ and $ H_n(K(\pi,n)) = \pi$. By the Universal Coefficients Theorem,
\begin{align*}
	H^{n}(K(\pi,n);\pi) \cong \Hom( H_n(K(\pi,n)), \pi) \oplus \ext(H_{n-1}(K(\pi,n)),\pi) = \Hom( \pi, \pi).
\end{align*}
as $ \ext$ is 0. Let $ \iota \in H^{n}(K(\pi,n);\pi)$ corresponds to $ 1_{\pi}$. Define $ \psi: [(X,A),(K(\pi,n);x_0)] \to H^{n}(X,A;\pi), f\mapsto f^* \iota $.

Note since $ \pi_n(K(\pi,n)) =0$ for $ k<n$. The first obstruction to homotopying a map  $ f:(X,A) \to (K(\pi,n),x_0)$ to be constant lives in $ H^{n}((X,A);\pi)$.
\begin{claim}
This obstruction is $ \psi(f)$.
\end{claim}
\begin{proof}
By naturality, it suffices to check that $ \iota $ is the primary obstruction to homotopying the identity map on $ K(\pi,n)$ to a constant map. 

We know $ (K(\pi,n))^{(n-1)} = \{x_0\} $. So the identity and constant map agree on  $ n-1$ skeleton. The  $ n$-cell  $ e_i^{n}$ corresponding to a generator of $ \pi=\pi_n(K(\pi,n))$.

\begin{claim}
$ \psi$ is onto.
\end{claim}
Let $  \alpha \in H^{n}(X,A;\pi)$ so there exists $ \widetilde{ \alpha} \in C^{n}(X,A; \pi)$ s.t.\ $ \alpha=[ \widetilde{ \alpha}]$, $ \widetilde{ \alpha}: C_n(X,A) \to \pi$, define $ f_{ \alpha}$ to be constant on $ X^{(n-1)}$ and for each $ n$-cell  $ e_i^{n}$ of $ X$. Let  $ f_{ \alpha}: e_i^{n} to K(\pi,n)$. Represent $ [f_{ \alpha}(e_i^{n})] = \widetilde{ \alpha}(e_i^{n}) \in \pi = \pi_n(K(\pi,n))$. This gives $ f_{ \alpha}$ on $ X^{(n)}$. For each $ e_j^{n+1}$,
\begin{align*}
	\widetilde{ \alpha}(\partial e_j^{n+1}) = \delta \widetilde{ \alpha}(e_j^{n+1}) = 0
\end{align*}
So $ f_{ \alpha}(\partial  e_j^{n+1})$ is nullhomotopic. we can extend $ f_{ \alpha}$ over $ e_j^{n+1}$ so $ f_{ \alpha}$ extends to $ X^{(n+1)}$. Since $ \pi_k(K(\pi,n)) = 0 \ \forall \ k>n$. So no obstruction to extending $ f_{ \alpha}$  to all of $ X$. As in the proof of first claim, we see  $ \psi(f_{ \alpha}) = f_{ \alphaa}^*  \iota = \alpha$.

\begin{claim}
$ \psi$ is injective.
\end{claim}

\begin{proof}
$ f,g: (X,A) \to (K(\pi,n),x_0)$ s.t.\ $ \psi(f) = \psi(g)$. The primary and only obstruction to a homotopy from $ f \to g$ lives in $ \theta \in H^{n}(X,A;\pi) \cong \pi_n(K(\pi,n))$. If we evaluate $ \widetilde{ \theta}$ on $ e_i^{n}$, we get

so $ \widetilde{ \theta}(e_i^{n}) = f(e_i^{n}) - g(e_i^{n}) \in \pi_n(K(\pi,n)) = \iota (f_*-g_*)(e_i^{n}) = (f^* \iota -g^* \iota ) e_i^{n = \psi(f) - \psi(g)}e_i^{n} = 0$  since they have different orientations. Thus $ f \simeq g$.
\end{proof}
\end{proof}
\end{proof}
\end{document}
