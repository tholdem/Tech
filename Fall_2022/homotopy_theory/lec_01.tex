\documentclass[12pt,class=article,crop=false]{standalone} 
\newcommand{\alert}[1]{{\bf \color{red} [Alert:] #1}}
\newcommand{\todo}[1]{{\bf \color{orange} [TODO:] #1}}
\newcommand{\real}[1][]{\mathbb{R}^{#1}}
\newcommand{\myeqn}[1]{(\ref{#1})}
\newcommand{\myex}[1]{Example \ref{#1}}
\newcommand{\defeq}{\stackrel{\mathrm{def}}{=}}
\newcommand{\parder}[2]{\frac{\partial #1}{\partial #2}}
\newcommand{\Lie}[3][]{\mathsf{L}_{#3}^{#1} #2}
\newcommand{\LieA}[1]{\mathsf{Lie}(#1)}
\newcommand{\lieder}[2]{\mathcal{L}_{#2} #1}
\renewcommand{\t}{^{\mbox{\tiny\sf T}}}
\newcommand{\trans}{^{\mbox{\tiny\sf T}}}
\newcommand{\markup}[1]{\{\textbf{#1}\}}
\newcommand{\msub}[1]{_\mathrm{#1}}
\newcommand{\msup}[1]{^\mathrm{#1}}
\newcommand{\inv}[1]{#1^{-1}}
\newcommand{\pinv}[1]{{#1}^{+}}
\newcommand{\myfracA}[2]{\displaystyle{\frac{#1}{#2}}}
\newcommand{\myfracB}[2]{{#1}/{#2}}
\newcommand{\mydiffA}[1]{\dot{#1}}
\newcommand{\mydiffB}[2]{\myfracA{\mathrm{d}{#1}}{\mathrm{d}{#2}}}
\newcommand{\ball}[2]{\mathcal{B}_{#1}\left(#2\right)}
\newcommand{\acos}[1]{\cos^{-1}\left(#1\right)}
\newcommand{\asin}[1]{\sin^{-1}\left(#1\right)}
\newcommand{\mani}{\mathcal{M}}
\newcommand{\tang}[2]{\mathsf{T}_{#1} #2}
\newcommand{\LieB}[2]{[ #1, #2 ]}
\newcommand{\LieBad}[3][]{\mathsf{ad}_{#2}^{#1} #3}
\newcommand{\ReachVT}{\mathcal{R}^V_T}
\newcommand{\ReachVt}{\mathcal{R}^V_t}
\newcommand{\ReachVTe}{\mathcal{R}^V_{\le T}}
\newcommand{\ReachT}{\mathcal{R}_T}
\newcommand{\Reacht}{\mathcal{R}_t}
\newcommand{\ReachTe}{\mathcal{R}_{\le T}}
\newcommand{\accLA}[1]{\mathsf{Lie}(#1)}
\newcommand{\accD}{\Delta_{\mathcal{F}}}
\newcommand{\accSA}{\mathsf{Lie}(\mathcal{G},f)}
\newcommand{\accDS}{\Delta_{\mathcal{G}}}
\newcommand{\eval}[3]{\mathsf{Ev}^{#2}_{#1}\left( #3 \right)}
\newcommand{\stlc}{\textsc{stlc}}
\newcommand{\clf}{\textsc{clf}}
\newcommand{\jqlf}{\textsc{jqlf}}
\newcommand{\dlas}{\textsc{dlas}}
\newcommand{\Ad}[2]{\mathsf{Ad}_{#1} #2}
\newcommand{\xe}{\ensuremath{x_e}}
\newcommand{\lebg}[1]{\mathcal{L}_{#1}}
\newcommand{\lebgx}[1]{\mathcal{L}_{#1 \mathrm{e}}}
\newcommand{\dom}{D}
\newcommand{\domT}{[t_0,\infty) \times D}
\newcommand{\rarrow}{\rightarrow}
\renewcommand{\d}{\mathrm{d}}
\renewcommand{\Re}{\mathbb{R}}
\newcommand{\C}{\mathrm{C}}

\newcommand{\QED}{{\unskip\nobreak\hfil\penalty50\hskip2em\vadjust{}
		\nobreak\hfil$\Box$\parfillskip=0pt\finalhyphendemerits=0\par}\vspace{0.1cm}}
\newcommand{\eoEx}{{\unskip\nobreak\hfil\penalty50\hskip0em\vadjust{}
		\nobreak\hfil$\Large\Diamond$\parfillskip=0pt\finalhyphendemerits=0\par}\vspace{0.1cm}}

\newcommand{\sgn}{\ensuremath{\operatorname{sgn}}}
\newcommand{\sat}{\ensuremath{\operatorname{sat}}}

\newcommand{\half}{\frac{1}{2}}
\newcommand{\shalf}{\mbox{$\frac{1}{2}$}}
\newcommand{\marcom}[1]{\marginpar{\footnotesize #1}}
\newcommand{\der}{\mathrm{D}}
\newcommand{\e}{\mathrm{e}}
\newcommand{\dt}{\mathrm{d}t}

\newcommand{\cA}{\ensuremath{\mathcal{A}}}
\newcommand{\cB}{\ensuremath{\mathcal{B}}}
\newcommand{\cG}{\ensuremath{\mathcal{G}}}
\newcommand{\cK}{\ensuremath{\mathcal{K}}}
\newcommand{\cW}{\ensuremath{\mathcal{W}}}
\newcommand{\cZ}{\ensuremath{\mathcal{Z}}}
\newcommand{\cS}{\ensuremath{\mathcal{S}}}
\newcommand{\cD}{\ensuremath{\mathcal{D}}}
\newcommand{\cP}{\ensuremath{\mathcal{P}}}
\newcommand{\cV}{\ensuremath{\mathcal{V}}}
\newcommand{\cL}{\ensuremath{\mathcal{L}}}
\newcommand{\cN}{\ensuremath{\mathcal{N}}}
\newcommand{\cI}{\ensuremath{\mathcal{I}}}
\newcommand{\cR}{\ensuremath{\mathcal{R}}}
\newcommand{\cM}{\ensuremath{\mathcal{M}}}
\newcommand{\cC}{\ensuremath{\mathcal{C}}}
\newcommand{\cF}{\ensuremath{\mathcal{F}}}
\newcommand{\cH}{\ensuremath{\mathcal{H}}}
\newcommand{\cO}{\ensuremath{\mathcal{O}}}
\newcommand{\cX}{\ensuremath{\mathcal{X}}}
\newcommand{\cY}{\ensuremath{\mathcal{Y}}}
\newcommand{\Ci}{\ensuremath{\mathcal{C}^\infty}}
\newcommand{\ISS}{\textsc{iss}}
\newcommand{\LISS}{\textsc{liss}}
\newcommand{\GAS}{\textsc{gas}}
\newcommand{\GS}{\textsc{gs}}
\newcommand{\LES}{\textsc{les}}
\newcommand{\GUAS}{\textsc{guas}}
\newcommand{\BIBO}{\textsc{bibo}}
\newcommand{\spec}{\ensuremath{\operatorname{spec}}}
\newcommand{\spn}{\ensuremath{\operatorname{span}}}
\renewcommand{\i}{\mathrm{i\,}}

\renewcommand{\implies}{\Rightarrow}

\renewcommand{\theenumi}{$\roman{enumi})$}
\renewcommand{\labelenumi}{\theenumi}

\font\ptmten=zptmcmrm scaled 1200
\newcommand{\w}{\mbox{{\ptmten w}}}
\newcommand{\z}{\mbox{{\ptmten z}}}
\renewcommand{\Re}{\mathbb{R}}

\newcommand{\cl}{\operatorname{cl}}
\newcommand{\intr}{\operatorname{int}}
\newcommand{\rank}{\operatorname{rank}}
\newcommand{\co}{\operatorname{co}}
\newcommand{\aff}{\operatorname{aff}}

\theoremstyle{plain}
\newtheorem{theorem}{Theorem}[chapter]
\newtheorem{claim}[theorem]{Claim}
\newtheorem{corollary}[theorem]{Corollary}
\newtheorem{prop}[theorem]{Proposition}
\newtheorem{fact}[theorem]{Fact}
\newtheorem{lemma}[theorem]{Lemma}

\newtheorem{remark}{Remark}[chapter]

\theoremstyle{definition}
\newtheorem{assume}[theorem]{Assumption}
\newtheorem{defn}[theorem]{Definition}
\newtheorem{problem}[theorem]{Problem}
\newtheorem{exercise}{Exercise}
\newtheorem{example}[theorem]{Example}


\begin{document}
\section{Homotopy Theory}
\subsection{Homotopy classes of maps}
Recall homotopy, denoted by $ \simeq$.
\begin{eg}
	$ X$ is any space, $ f: X \to I:=[0,1]$ is homotopic to the constant map $ g(x) =0$ as $ I$ is convex. The homotopy is the straight-line homotopy $ \Phi(x,t)=(1-t)f(x)$.
\end{eg}

Let $ C(X,Y) = \{ \text{continuous functions from } X \to Y \}$. Denote $ [X,Y] = C(X,Y) /\simeq $, so homotopic maps are identified.

\begin{eg}
	$ [X,I] = \{g(x)=0\}$.
\end{eg}

In a pointed space, denote $ [X,Y]_0$ to be the homotopy classes of morphisms from pointed spaces  $ (X,x_0)$ to $(Y,y_0) $. If $ f:X \to X'$ a continuous function, then this induces a functor $ f^* : [X',Y]_0 \to [X,Y]_0$ by precomposition. Likewise for postcomposition.

We have the covariant functor $ [X, - ]: \textsf{Top} \to \textsf{Set} $ and the contravariant functor $ [-,X]: \textsf{ Top} \to \textsf{Set}$.

Recall homotopy equivalence, also denoted by $ \simeq$.

\begin{eg}
	$ X = S^{1}$ and $ Y=S^{1} \times [0,1]$. These are homotopy equivalent. We have $ f: X \to Y, \theta \mapsto (\theta,0)$ and $ g:Y \to X, (\theta,t) \mapsto \theta$.
\end{eg}
\begin{eg}
$ X,Y$ are any spaces, morphism $ f: X \to Y$. The \allbold{mapping cylinder} is
\begin{align*}
	C_f= ((X \times I) \cup Y) / \sim
\end{align*}
where $ (x,0) \sim f(x)$.

This is homotopy equivalent to  $ Y$. Exercise. 

Show $ \pi: C_f \to Y, (x,t) \in X \times [0,1] \to f(x), y \in Y \mapsto y$ has homotopy inverse. Note there is an inclusion $ j : X \to C_f, x \mapsto (x,1)$.

Show $ j \cong  i \circ f$.

Moral: any map is an inclusion up to homotopy.
\end{eg}

\begin{defn}
A pointed space $ (Y,y_0)$ is called an \allbold{H-space} if there exists maps $ \mu: Y \times Y \to Y$ and $ \nu: Y \to Y$ s.t.\ 
\begin{enumerate}[label=(\arabic*)]
	\item for $ i_1: y \mapsto (y,y_0)$ and $ i_2: y\mapsto (y_0,y)$, we have
		\[
\mu \circ i_1 \simeq  \text{id}_Y, \mu \circ i_2 \simeq  \text{id}_Y 
.\] 
\item The compositions
\begin{align*}
	Y \times (Y\times Y) \xrightarrow{ \text{id}_Y \times \mu } Y \times Y \xrightarrow{ \mu} Y  
\end{align*}
and
\begin{align*}
	(Y \times Y) \times Y \xrightarrow{ \mu \times \text{id}_Y } Y \times Y \xrightarrow{ \mu}  Y  
\end{align*}
are homotopic.
\item The composition
\begin{align*}
	Y \xrightarrow{ \text{id}_Y  \times \nu} Y \times Y \xrightarrow{ \mu} Y  
\end{align*}
\begin{align*}
	Y \xrightarrow{ \nu \times \text{id}_Y } \times Y \times Y \xrightarrow{ \mu}Y  
\end{align*}
are homotopic to constant maps.
\end{enumerate}
\end{defn}
\begin{remark}
This definition should remind us of group axioms: identity, association, and inverses. We see that $ \mu$ hints at multiplication whereas $ \nu$ hints at inversion.
\end{remark}
\begin{eg}
If $ G$ is a topological group (group with topology  s.t.\ multiplication and inverses are continuous maps).

Exercise: $ (G, e)$ is an H-space.
\end{eg}

\begin{thm}
	The set $ [X,Y]_0$ has a natural group structure for all pointed spaces X  iff $ Y$ is an H-space.
\end{thm}
Natural means if $ f: X \to X'$, then the induced map $ f^* : [X',Y]_0 \to [X,Y]_0$ is a homomorphism. That is, $ [-,Y]_0$ is a functor.

\begin{proof}
	$ (\impliedby)$: suppose $ Y$ is an H-space. Given  $ (X,x_0)$ and notice $ \mu : Y \times Y \to Y$ induces $ \mu^* : [X,Y\times Y]_0 \to [X,Y]_0$. There is also a canonical function
	\begin{align*}
	\phi:	[X,Y]_0 \times [X,Y]_0 \to [X, Y\times Y]_0, ([f],[g]) \mapsto [f \times g].
	\end{align*}
	Exercise: $ \phi$ is well-defined and a bijection. Define multiplication
	\begin{align*}
		m=\mu^*  \circ \phi : [X,Y]_0 \times [X,Y]_0 \to [X,Y]_0.
	\end{align*}
	This is clearly well-defined since $ \phi$ is well-defined and post-composing homotopic functions are still homotopic. Denote $ m([f],[g])$ by  $ [f] \cdot [g]$. Denote $ \nu_x ([f])$ by $ [f]^{-1}$. Let $ e(x)=y_0$ be the constant map. Now we check the group axioms:

	Identity: Tracking the representatives of $ [e] \cdot [g]$ yields
	\begin{align*}
		\mu(y_0,g(x)) &= (\mu \circ i_1) \circ g(x) \\
		& \cong \text{id}_Y \circ g(x) =g(x) 
	\end{align*}
	The other direction follows from using $ i_2$. Thus $ [e] \cdot  [g]= [g] = [g] \cdot  [e]$.

	Associativity: Given $ [f], [g], [h] \in [X, Y]_0$, we see that
	\begin{align*}
		([f] \cdot [g]) \cdot [h] &= \mu^* \circ \phi ([f],[g]) \cdot [h] \\
					  &= \mu^* ([f \times g]) \cdot [h]\\
					  &= [\mu \circ (f\times g)] \cdot [h] \\
					  &= [\mu \circ ( (\mu \circ (f \times g)) \times h)] \\
					  &= [\mu \circ (\mu \times \text{id}_{ Y}) \circ f\times g \times h] \\
					  &= [\mu \circ (\text{id}_{ Y}\times \mu) \circ f \times g \times h] && \text{ condition 2} \\
					  &= [\mu \circ (f \times (\mu \circ (g \times h)))] \\
					  &= [f] \cdot ([g] \cdot  [h]) 
	\end{align*}
	Inverse: Given $ [f] \in [X, Y]_0$, we have
	\begin{align*}
		[f] \cdot [f]^{-1} &= [\mu \circ (f \times (\nu \circ f))] \\
				   &= [\mu \circ (\text{id}_{ Y} \times \nu) \circ f] \\
				   &= [e]
	\end{align*}
	The other direction follows similarly.

	$ (\implies)$: Suppose $ [X,Y]_0$ has a natural group structure for all pointed spaces $ X$.  Take $ X = Y \times Y$, and $ p_1,p_2$ be the projections onto 1st and 2nd factors respectively. This yields $ [p_1], [p_2] \in [Y\times Y,Y]_0$. Let $ \mu$ be a representative of $ [p_1] \cdot [p_2]$. Let $\nu$ be a representative of $ [\text{id}_Y ]^{-1} $.

	Now check condition 1. $ i_1: Y \to Y \times Y, y \mapsto  (y,y_0)$ induces $ i_1^* $ so that
	\begin{align*}
		i_1^* ([p_1]) &= [p_1 \circ i_1] = [ \text{id}_Y ]\\
		i_1^* ([p_2]) &= [e]. 
	\end{align*}
	Therefore,
	\begin{align*}
		i_1^* ([\mu]) &=i^* ([p_1] \cdot [p_2]) \\ 
			      &= [ \text{id}_Y ] \cdot [ e].
	\end{align*}
	Since $ [e]$ is the identity on $ [Y,Y]_0$, $ [\mu \circ i_1] = [\text{ id}_Y ]$.

	2 and 3 are similar so left as exercises.
\end{proof}

\begin{defn}
If $ (Y,y_0)$ is a point space, then the \allbold{loop space} of $ Y$ is
 \begin{align*}
	 \Omega(Y) = C^{0}((I, \{0,1\} ),(Y,y_0)) = C^{0}((S^{1},x_0),(Y,y_0)).
\end{align*}
\end{defn}

\begin{lem}
$ \Omega(Y)$ is an H-space.
\end{lem}
\begin{proof}
Same as the proof for fundamental group. 
\end{proof}

\begin{defn}
Given pointed spaces $ (X,x_0), (Y,y_0)$, the \allbold{wedge product} is $ X \vee Y = (X \times \{y_0\}) \cup (\{x_0\}  \times Y) \subseteq X \times Y$ with base point $ (x_0,y_0)$. 
\end{defn}

\begin{defn}
A pointed space $ (Y,y_0)$ is an \allbold{H'-space} if there are maps $ \mu: Y \to Y \vee Y$ and $ \nu: Y \to Y$ s.t.\ 
\begin{enumerate}[label=(\arabic*)]
	\item $ p_1 \circ \mu \simeq \text{id}_{ Y}$ and $ p_2 \circ \mu \simeq \text{id}_{ Y}$ where $ p_1, p_2 : Y \vee Y \to Y$ are projections onto the 1st and 2nd factors.
	\item The compositions
		\begin{align*}
			Y \xrightarrow{ \mu} Y \vee Y \xrightarrow{ \text{id}_{ Y} \vee  \mu} Y \vee (Y\vee Y)
		\end{align*}
		and
		\begin{align*}
			Y \xrightarrow{ \mu} Y \vee Y \xrightarrow{ \mu \vee \text{id}_{ Y}} (Y \vee Y) \vee Y 
		\end{align*}
		are homotopic.
	\item The compositions
		\begin{align*}
			Y \xrightarrow{ \mu} Y \vee Y \xrightarrow{\nu \vee  \text{id}_{ Y}}  Y
		\end{align*}
		and
		\begin{align*}
			Y \xrightarrow{ \mu} Y \vee Y \xrightarrow{ \text{id}_{ Y} \vee \nu} Y  
		\end{align*}
		are homotopic to the constant map.
\end{enumerate}
\end{defn}

\begin{thm}
	The set $ [Y, X]_0$ has a natural group structure for all $ (X,x_0)$ iff $ Y$ is an H'-space.
\end{thm}
\begin{proof}
Exercise.

$ (\impliedby):$ suppose $ Y$ is a H'-space, then we have $\mu: Y \to Y \vee Y$ and $ \nu: Y \to Y$ that satisfy all the three conditions. 
\end{proof}

\begin{defn}
Given a space $ X$, its  \allbold{suspension} is
\begin{align*}
	\Sigma X = X \times I / \sim,
\end{align*}
where $ X \times \{0\} $ and $ X \times \{1\} $ are collapsed to two distinct points.
\end{defn}
\begin{remark}
	If $ (X,x_0)$ is pointed, then $ \sum X = X \times I / \{X \times \{0\} , X \times \{1\} , \{x_0\} \times I \} $ with base point $ [\{x_0\}] $.
\end{remark}

\begin{eg}
Collapsing the two bases of a cylinder yields
\begin{enumerate}[label=(\arabic*)]
	\item $ S^{n} = \Sigma S^{n-1}$.
	\item $ (S^{n},x_0)= \Sigma (S^{n-1},x_0)$.

		Exercise: if $ M$ is any manifold and  $ C$ is an arc in  $ M$, then prove  $ M /C$ is homeomorphic to  $ M$. (hint: prove it for $ n$-disk.)
\end{enumerate}
\end{eg}

\begin{lem}
For any pointed space $ (Y,y_0)$, its suspension $ \Sigma Y$ is an  H'-space.
\end{lem}
\begin{proof}
Define $ \mu: \Sigma Y \to \Sigma Y \vee \Sigma Y$. See ipad. Exercise.
\end{proof}

\begin{thm}
	If $ X$ is an  H'-space and  $ Y$ is an H-space, then the corresponding binary operations on $ [X, Y]_0$ agree and are commutative.
\end{thm}
\begin{proof}
	Denote the binary operation from H'-space by $ +$ and the other by  $ \cdot $. Let $ f_1, f_2$ be maps representing elements in $ [X, Y]_0$. See ipad for diagram. Let $ \Delta: X \to X \times X$ be the diagonal map, $ \nabla : Y \times Y \to Y, $Note that
	\begin{align*}
		[f_1] \cdot  [f_2] = [\mu_Y \circ (f_1 \times f_2) \circ \Delta],
		[f_1]+[f_2] =[\nabla \circ (f_1 \vee f_2) \circ \mu_X].
	\end{align*}
Condition 1 of H'-space says $ i \circ \mu_X \simeq \Delta$. Condition 1 of H-space says $ \mu_Y \circ j \simeq \nabla $. Now $ (x,x_0) \in X \vee X$, then $ i(x,x_0) = (x,x_0) \in X \times X$. So
\begin{align*}
	(f_1 \times f_2) \circ  i(x,x_0) &= (f_1(x),y_0) \\
	f_1 \vee f_2(x,x_0) &= (f_1(x),y_0) \\
	j \circ  (f_1 \vee f_2) (x,x_0) &= (f_1(x),y_0) 
\end{align*}
Similarly for $ (x_0,x)$ so center square in the diagram commutes.
\begin{align*}
	\nabla \circ  (f_1 \vee f_2) \circ  \mu_X &\simeq \mu_Y \circ j \circ (f_1 \vee f_2) \circ  \mu_X \\
	&= \mu_Y \circ  (f_1 \times f_2) \circ i \circ \mu_X\\
	&\simeq \mu_Y \circ  (f_1 \times f_2) \circ  \Delta
\end{align*}
It remains to show abelian. Fact: if $ \rho : G \times G \to G, (g,h)\mapsto gh$ is a homomorphism then $ G$ is abelian. To see this, notice
 \begin{align*}
	 \rho((g,h) (g^{-1},h^{-1})) &= \rho((gg^{-1},h h^{-1}))\\
	&= \rho(e,e)= e \\
	 \rho(g,h) \rho(g^{-1},h^{-1}) &= gh g^{-1} h^{-1}
\end{align*}

$ \mu_Y: Y \times Y \to Y$ induces a homomorphism $ [X, Y\times Y]_0 \xrightarrow{ \mu_Y} [X, Y]_0$. We also have the bijection
\begin{align*}
	\phi: [X, Y]_0 \times [X, Y]_0 \to [X, Y\times Y]_0, ([f],[g])\mapsto [f \times g].
\end{align*}
\begin{claim}
$ \phi$ is a homomorphism.
\end{claim}
Let $ p_1: Y \times Y \to Y$ be projection to 1st factor, which induces homomorphisms $ p_i: [X, Y\times Y]_0 \to [X, Y]_0$. Clearly $ \phi$ is the inverse of $ (p_1)_* \times (p_2)_*$ so $ \phi$ is a homomorphism. Then
\begin{align*}
	(\mu_Y)_* \circ \phi: [X, Y]_0 \times [X, Y]_0 \to [X, Y]_0, ([f],[g]) \mapsto [f] \cdot [g]
\end{align*}
So the group is abelian by the fact.
\end{proof}
\begin{defn}
A space is \allbold{locally compact} if every point has a neighborhood that is contained in a compact set. 
\end{defn}
\begin{lem}
If $ Y$ is locally compact and Hausdorff, there is a bijection
 \begin{align*}
	C^{0}(X\times Y,Z) \to C^{0}(X,C^{0}(Y,Z)).
\end{align*}
If $ X$ is also Hausdorff, this is a homeomorphism. Note that $ C^{0}$ is simply the $ \Hom$ functor.
\begin{remark}
Any manifold or CW-complexes are locally compact.
\end{remark}
\end{lem}
\begin{remark}
	The lemma implies that $ [X\times Y,Z] = [X,C^{0}(Y,Z)]$.
\end{remark}
\begin{defn}
Given $ C$ a compact set in  $ X$,  $ W$ an open set in  $ Y$, let  $ U(C,W) = \{f \in C^{0}(X,Y) : f(C) \subseteq W)\} $. This forms a subbasis for a topology on $ C^{0}(X,Y)$, called \allbold{compact-open topology}.
\end{defn}
Exercises:
\begin{enumerate}[label=(\arabic*)]
	\item If $ Y$ is a metric space, show that this topology is the topology of compact convergence,  \emph{i.e.} $ f_n \to f$ iff for all compact sets $ C \subseteq X$, $ f_n|_C \to f|_C$ uniformly.
	\item If $ f: X \times Y \to Z$ is continuous, then so is
		\begin{align*}
			F: X \to C^{0}(Y,Z), x\mapsto f_x: Y \to Z, y \mapsto f(x,y)
		\end{align*}
	\item The converse is true if $ Y$ is locally compact.
	\item Prove the theorem.
\end{enumerate}

\begin{proof}
We need a topology on $ C^{0}$: the compact-open topology. 
\end{proof}

\begin{defn}
The \allbold{smash product} of two pointed spaces is
\begin{align*}
	X \wedge Y = X \times Y / X \vee Y = X \times Y / X \times \{y_0\} \cup \{x_0\} \times Y  .
\end{align*}
\end{defn}
Recall that the \allbold{reduced suspension} is
\begin{align*}
	\Sigma X = S^{1} \wedge X = S^{1} \times X / S^{1} \times \{x_0\} \cup \{e_0\} \times X .
\end{align*}
See ipad.

\begin{coro}
If $ Y$ is locally compact, then
 \begin{align*}
	 [X \wedge Y, Z]_0 = [X, C^{0}_{ \text{based} } (Y,Z)]_0.
\end{align*}
\end{coro}
\begin{proof}
	If $ f \in C^{0}_{ \text{based} }(X,C^{0}_{ \text{based} }(Y,Z))$ then it has to send base point to base point: $ f(x_0) =$ constant map $ Y \to Z, y_0 \mapsto z_0$. So $ F: X \times Y \to Z$, $ (x,y) \mapsto f(x)(y)$ sends $ \{x_0\} \times Y \to z_0 $ by the lemma. As $ f(x): Y \to Z$ sends $ y_0 \to z_0$, $ F$ induces a map on  $ X \wedge Y = X \times Y / X \vee Y \xrightarrow{ F} Z$. So $ F \in [X \wedge Y, Z]_0$.

	We can similarly define an inverse: exercise.
\end{proof}

Recall that a loop space is $ \Omega(X) = C^{0}_{ \text{based} }(S^{1},X)$.

\begin{coro}
	$ [\Sigma X, Y]_0 = [X, \Omega(Y)]_0$. That is, suspension is the adjoint of looping.
\end{coro}
\begin{proof}
\begin{align*}
	[\Sigma X, Y]_0 &= [S^{1} \wedge X, Y]_0 \\
		      &= [X, C^{0}_{ \text{based} }(S^{1},Y)]_0 \\
		      &= [X, \Omega(Y)]_0 \\
\end{align*}
\end{proof}
\begin{remark}
They are isomorphic as groups TODO
\end{remark}
\begin{defn}
The \allbold{$ n$th homotopy group} of $ (X,x_0)$ is
\begin{align*}
	\pi_n(X) = [S^{n}, X]_0.
\end{align*}
\end{defn}
Note that
\begin{enumerate}[label=(\arabic*)]
	 \item
		 \begin{align*}
			 \pi_n(X) &= [S^{n}, X]_0 \\
				  &= [\Sigma S^{n-1}, X]_0 \\
				  &= [S^{n-1}, \Omega(X)]_0 \\
				  &= [S^{0}, \Omega^{n}(X)]_0 
		 \end{align*}
		 So $ \pi_n(X) = \pi_0(\Omega^{n}(X))$, the path components of $ \Omega(X)$.
	 \item $ \pi_n(X) = [S^{n-1}, \Omega(X)]_0$. If $ n\geq 2$,  $ \Omega(X)$ is H-space, $ S^{n-1}$ is H'-space, so by Theorem $ \pi_n(X)$ is abelian for $ n \geq 2$.
	 \item If  $ X$ is a Lie group, the it is an H-space so $ \pi_1(X)$ is abelian.
\end{enumerate}
\end{document}
