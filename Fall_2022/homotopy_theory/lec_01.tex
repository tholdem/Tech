\documentclass[12pt,class=article,crop=false]{standalone} 
%Fall 2022
% Some basic packages
\usepackage{standalone}[subpreambles=true]
\usepackage[utf8]{inputenc}
\usepackage[T1]{fontenc}
\usepackage{textcomp}
\usepackage[english]{babel}
\usepackage{url}
\usepackage{graphicx}
%\usepackage{quiver}
\usepackage{float}
\usepackage{enumitem}
\usepackage{lmodern}
\usepackage{comment}
\usepackage{hyperref}
\usepackage[usenames,svgnames,dvipsnames]{xcolor}
\usepackage[margin=1in]{geometry}
\usepackage{pdfpages}

\pdfminorversion=7

% Don't indent paragraphs, leave some space between them
\usepackage{parskip}

% Hide page number when page is empty
\usepackage{emptypage}
\usepackage{subcaption}
\usepackage{multicol}
\usepackage[b]{esvect}

% Math stuff
\usepackage{amsmath, amsfonts, mathtools, amsthm, amssymb}
\usepackage{bbm}
\usepackage{stmaryrd}
\allowdisplaybreaks

% Fancy script capitals
\usepackage{mathrsfs}
\usepackage{cancel}
% Bold math
\usepackage{bm}
% Some shortcuts
\newcommand{\rr}{\ensuremath{\mathbb{R}}}
\newcommand{\zz}{\ensuremath{\mathbb{Z}}}
\newcommand{\qq}{\ensuremath{\mathbb{Q}}}
\newcommand{\nn}{\ensuremath{\mathbb{N}}}
\newcommand{\ff}{\ensuremath{\mathbb{F}}}
\newcommand{\cc}{\ensuremath{\mathbb{C}}}
\newcommand{\ee}{\ensuremath{\mathbb{E}}}
\newcommand{\hh}{\ensuremath{\mathbb{H}}}
\renewcommand\O{\ensuremath{\emptyset}}
\newcommand{\norm}[1]{{\left\lVert{#1}\right\rVert}}
\newcommand{\dbracket}[1]{{\left\llbracket{#1}\right\rrbracket}}
\newcommand{\ve}[1]{{\bm{#1}}}
\newcommand\allbold[1]{{\boldmath\textbf{#1}}}
\DeclareMathOperator{\lcm}{lcm}
\DeclareMathOperator{\im}{im}
\DeclareMathOperator{\coim}{coim}
\DeclareMathOperator{\dom}{dom}
\DeclareMathOperator{\tr}{tr}
\DeclareMathOperator{\rank}{rank}
\DeclareMathOperator*{\var}{Var}
\DeclareMathOperator*{\ev}{E}
\DeclareMathOperator{\dg}{deg}
\DeclareMathOperator{\aff}{aff}
\DeclareMathOperator{\conv}{conv}
\DeclareMathOperator{\inte}{int}
\DeclareMathOperator*{\argmin}{argmin}
\DeclareMathOperator*{\argmax}{argmax}
\DeclareMathOperator{\graph}{graph}
\DeclareMathOperator{\sgn}{sgn}
\DeclareMathOperator*{\Rep}{Rep}
\DeclareMathOperator{\Proj}{Proj}
\DeclareMathOperator{\mat}{mat}
\DeclareMathOperator{\diag}{diag}
\DeclareMathOperator{\aut}{Aut}
\DeclareMathOperator{\gal}{Gal}
\DeclareMathOperator{\inn}{Inn}
\DeclareMathOperator{\edm}{End}
\DeclareMathOperator{\Hom}{Hom}
\DeclareMathOperator{\ext}{Ext}
\DeclareMathOperator{\tor}{Tor}
\DeclareMathOperator{\Span}{Span}
\DeclareMathOperator{\Stab}{Stab}
\DeclareMathOperator{\cont}{cont}
\DeclareMathOperator{\Ann}{Ann}
\DeclareMathOperator{\Div}{div}
\DeclareMathOperator{\curl}{curl}
\DeclareMathOperator{\nat}{Nat}
\DeclareMathOperator{\gr}{Gr}
\DeclareMathOperator{\vect}{Vect}
\DeclareMathOperator{\id}{id}
\DeclareMathOperator{\Mod}{Mod}
\DeclareMathOperator{\sign}{sign}
\DeclareMathOperator{\Surf}{Surf}
\DeclareMathOperator{\fcone}{fcone}
\DeclareMathOperator{\Rot}{Rot}
\DeclareMathOperator{\grad}{grad}
\DeclareMathOperator{\atan2}{atan2}
\DeclareMathOperator{\Ric}{Ric}
\let\vec\relax
\DeclareMathOperator{\vec}{vec}
\let\Re\relax
\DeclareMathOperator{\Re}{Re}
\let\Im\relax
\DeclareMathOperator{\Im}{Im}
% Put x \to \infty below \lim
\let\svlim\lim\def\lim{\svlim\limits}

%wide hat
\usepackage{scalerel,stackengine}
\stackMath
\newcommand*\wh[1]{%
\savestack{\tmpbox}{\stretchto{%
  \scaleto{%
    \scalerel*[\widthof{\ensuremath{#1}}]{\kern-.6pt\bigwedge\kern-.6pt}%
    {\rule[-\textheight/2]{1ex}{\textheight}}%WIDTH-LIMITED BIG WEDGE
  }{\textheight}% 
}{0.5ex}}%
\stackon[1pt]{#1}{\tmpbox}%
}
\parskip 1ex

%Make implies and impliedby shorter
\let\implies\Rightarrow
\let\impliedby\Leftarrow
\let\iff\Leftrightarrow
\let\epsilon\varepsilon

% Add \contra symbol to denote contradiction
\usepackage{stmaryrd} % for \lightning
\newcommand\contra{\scalebox{1.5}{$\lightning$}}

% \let\phi\varphi

% Command for short corrections
% Usage: 1+1=\correct{3}{2}

\definecolor{correct}{HTML}{009900}
\newcommand\correct[2]{\ensuremath{\:}{\color{red}{#1}}\ensuremath{\to }{\color{correct}{#2}}\ensuremath{\:}}
\newcommand\green[1]{{\color{correct}{#1}}}

% horizontal rule
\newcommand\hr{
    \noindent\rule[0.5ex]{\linewidth}{0.5pt}
}

% hide parts
\newcommand\hide[1]{}

% si unitx
\usepackage{siunitx}
\sisetup{locale = FR}

%allows pmatrix to stretch
\makeatletter
\renewcommand*\env@matrix[1][\arraystretch]{%
  \edef\arraystretch{#1}%
  \hskip -\arraycolsep
  \let\@ifnextchar\new@ifnextchar
  \array{*\c@MaxMatrixCols c}}
\makeatother

\renewcommand{\arraystretch}{0.8}

\renewcommand{\baselinestretch}{1.5}

\usepackage{graphics}
\usepackage{epstopdf}

\RequirePackage{hyperref}
%%
%% Add support for color in order to color the hyperlinks.
%% 
\hypersetup{
  colorlinks = true,
  urlcolor = blue,
  citecolor = blue
}
%%fakesection Links
\hypersetup{
    colorlinks,
    linkcolor={red!50!black},
    citecolor={green!50!black},
    urlcolor={blue!80!black}
}
%customization of cleveref
\RequirePackage[capitalize,nameinlink]{cleveref}[0.19]

% Per SIAM Style Manual, "section" should be lowercase
\crefname{section}{section}{sections}
\crefname{subsection}{subsection}{subsections}
\Crefname{section}{Section}{Sections}
\Crefname{subsection}{Subsection}{Subsections}

% Per SIAM Style Manual, "Figure" should be spelled out in references
\Crefname{figure}{Figure}{Figures}

% Per SIAM Style Manual, don't say equation in front on an equation.
\crefformat{equation}{\textup{#2(#1)#3}}
\crefrangeformat{equation}{\textup{#3(#1)#4--#5(#2)#6}}
\crefmultiformat{equation}{\textup{#2(#1)#3}}{ and \textup{#2(#1)#3}}
{, \textup{#2(#1)#3}}{, and \textup{#2(#1)#3}}
\crefrangemultiformat{equation}{\textup{#3(#1)#4--#5(#2)#6}}%
{ and \textup{#3(#1)#4--#5(#2)#6}}{, \textup{#3(#1)#4--#5(#2)#6}}{, and \textup{#3(#1)#4--#5(#2)#6}}

% But spell it out at the beginning of a sentence.
\Crefformat{equation}{#2Equation~\textup{(#1)}#3}
\Crefrangeformat{equation}{Equations~\textup{#3(#1)#4--#5(#2)#6}}
\Crefmultiformat{equation}{Equations~\textup{#2(#1)#3}}{ and \textup{#2(#1)#3}}
{, \textup{#2(#1)#3}}{, and \textup{#2(#1)#3}}
\Crefrangemultiformat{equation}{Equations~\textup{#3(#1)#4--#5(#2)#6}}%
{ and \textup{#3(#1)#4--#5(#2)#6}}{, \textup{#3(#1)#4--#5(#2)#6}}{, and \textup{#3(#1)#4--#5(#2)#6}}

% Make number non-italic in any environment.
\crefdefaultlabelformat{#2\textup{#1}#3}

% Environments
\makeatother
% For box around Definition, Theorem, \ldots
%%fakesection Theorems
\usepackage{thmtools}
\usepackage[framemethod=TikZ]{mdframed}

\theoremstyle{definition}
\mdfdefinestyle{mdbluebox}{%
	roundcorner = 10pt,
	linewidth=1pt,
	skipabove=12pt,
	innerbottommargin=9pt,
	skipbelow=2pt,
	nobreak=true,
	linecolor=blue,
	backgroundcolor=TealBlue!5,
}
\declaretheoremstyle[
	headfont=\sffamily\bfseries\color{MidnightBlue},
	mdframed={style=mdbluebox},
	headpunct={\\[3pt]},
	postheadspace={0pt}
]{thmbluebox}

\mdfdefinestyle{mdredbox}{%
	linewidth=0.5pt,
	skipabove=12pt,
	frametitleaboveskip=5pt,
	frametitlebelowskip=0pt,
	skipbelow=2pt,
	frametitlefont=\bfseries,
	innertopmargin=4pt,
	innerbottommargin=8pt,
	nobreak=false,
	linecolor=RawSienna,
	backgroundcolor=Salmon!5,
}
\declaretheoremstyle[
	headfont=\bfseries\color{RawSienna},
	mdframed={style=mdredbox},
	headpunct={\\[3pt]},
	postheadspace={0pt},
]{thmredbox}

\declaretheorem[%
style=thmbluebox,name=Theorem,numberwithin=section]{thm}
\declaretheorem[style=thmbluebox,name=Lemma,sibling=thm]{lem}
\declaretheorem[style=thmbluebox,name=Proposition,sibling=thm]{prop}
\declaretheorem[style=thmbluebox,name=Corollary,sibling=thm]{coro}
\declaretheorem[style=thmredbox,name=Example,sibling=thm]{eg}

\mdfdefinestyle{mdgreenbox}{%
	roundcorner = 10pt,
	linewidth=1pt,
	skipabove=12pt,
	innerbottommargin=9pt,
	skipbelow=2pt,
	nobreak=true,
	linecolor=ForestGreen,
	backgroundcolor=ForestGreen!5,
}

\declaretheoremstyle[
	headfont=\bfseries\sffamily\color{ForestGreen!70!black},
	bodyfont=\normalfont,
	spaceabove=2pt,
	spacebelow=1pt,
	mdframed={style=mdgreenbox},
	headpunct={ --- },
]{thmgreenbox}

\declaretheorem[style=thmgreenbox,name=Definition,sibling=thm]{defn}

\mdfdefinestyle{mdgreenboxsq}{%
	linewidth=1pt,
	skipabove=12pt,
	innerbottommargin=9pt,
	skipbelow=2pt,
	nobreak=true,
	linecolor=ForestGreen,
	backgroundcolor=ForestGreen!5,
}
\declaretheoremstyle[
	headfont=\bfseries\sffamily\color{ForestGreen!70!black},
	bodyfont=\normalfont,
	spaceabove=2pt,
	spacebelow=1pt,
	mdframed={style=mdgreenboxsq},
	headpunct={},
]{thmgreenboxsq}
\declaretheoremstyle[
	headfont=\bfseries\sffamily\color{ForestGreen!70!black},
	bodyfont=\normalfont,
	spaceabove=2pt,
	spacebelow=1pt,
	mdframed={style=mdgreenboxsq},
	headpunct={},
]{thmgreenboxsq*}

\mdfdefinestyle{mdblackbox}{%
	skipabove=8pt,
	linewidth=3pt,
	rightline=false,
	leftline=true,
	topline=false,
	bottomline=false,
	linecolor=black,
	backgroundcolor=RedViolet!5!gray!5,
}
\declaretheoremstyle[
	headfont=\bfseries,
	bodyfont=\normalfont\small,
	spaceabove=0pt,
	spacebelow=0pt,
	mdframed={style=mdblackbox}
]{thmblackbox}

\theoremstyle{plain}
\declaretheorem[name=Question,sibling=thm,style=thmblackbox]{ques}
\declaretheorem[name=Remark,sibling=thm,style=thmgreenboxsq]{remark}
\declaretheorem[name=Remark,sibling=thm,style=thmgreenboxsq*]{remark*}
\newtheorem{ass}[thm]{Assumptions}

\theoremstyle{definition}
\newtheorem*{problem}{Problem}
\newtheorem{claim}[thm]{Claim}
\theoremstyle{remark}
\newtheorem*{case}{Case}
\newtheorem*{notation}{Notation}
\newtheorem*{note}{Note}
\newtheorem*{motivation}{Motivation}
\newtheorem*{intuition}{Intuition}
\newtheorem*{conjecture}{Conjecture}

% Make section starts with 1 for report type
%\renewcommand\thesection{\arabic{section}}

% End example and intermezzo environments with a small diamond (just like proof
% environments end with a small square)
\usepackage{etoolbox}
\AtEndEnvironment{vb}{\null\hfill$\diamond$}%
\AtEndEnvironment{intermezzo}{\null\hfill$\diamond$}%
% \AtEndEnvironment{opmerking}{\null\hfill$\diamond$}%

% Fix some spacing
% http://tex.stackexchange.com/questions/22119/how-can-i-change-the-spacing-before-theorems-with-amsthm
\makeatletter
\def\thm@space@setup{%
  \thm@preskip=\parskip \thm@postskip=0pt
}

% Fix some stuff
% %http://tex.stackexchange.com/questions/76273/multiple-pdfs-with-page-group-included-in-a-single-page-warning
\pdfsuppresswarningpagegroup=1


% My name
\author{Jaden Wang}



\begin{document}
\section{Homotopy Theory}
\subsection{Homotopy classes of maps}
Recall homotopy, denoted by $ \simeq$.
\begin{eg}
	$ X$ is any space, $ f: X \to I:=[0,1]$ is homotopic to the constant map $ g(x) =0$ as $ I$ is convex. The homotopy is the straight-line homotopy $ \Phi(x,t)=(1-t)f(x)$.
\end{eg}

Let $ C(X,Y) = \{ \text{continuous functions from } X \to Y \}$. Denote $ [X,Y] = C(X,Y) /\simeq $, so homotopic maps are identified.

\begin{eg}
	$ [X,I] = \{g(x)=0\}$.
\end{eg}

In a pointed space, denote $ [X,Y]_0$ to be the homotopy classes of morphisms from pointed spaces  $ (X,x_0)$ to $(Y,y_0) $. If $ f:X \to X'$ a continuous function, then this induces a functor $ f^* : [X',Y]_0 \to [X,Y]_0$ by precomposition. Likewise for postcomposition.

We have the covariant functor $ [X, - ]: \textsf{Top} \to \textsf{Set} $ and the contravariant functor $ [-,X]: \textsf{ Top} \to \textsf{Set}$.

Recall homotopy equivalence, also denoted by $ \simeq$.

\begin{eg}
	$ X = S^{1}$ and $ Y=S^{1} \times [0,1]$. These are homotopy equivalent. We have $ f: X \to Y, \theta \mapsto (\theta,0)$ and $ g:Y \to X, (\theta,t) \mapsto \theta$.
\end{eg}
\begin{eg}
$ X,Y$ are any spaces, morphism $ f: X \to Y$. The \allbold{mapping cylinder} is
\begin{align*}
	C_f= ((X \times I) \cup Y) / \sim
\end{align*}
where $ (x,0) \sim f(x)$.

This is homotopy equivalent to  $ Y$. Exercise. 

Show $ \pi: C_f \to Y, (x,t) \in X \times [0,1] \to f(x), y \in Y \mapsto y$ has homotopy inverse. Note there is an inclusion $ j : X \to C_f, x \mapsto (x,1)$.

Show $ j \cong  i \circ f$.

Moral: any map is an inclusion up to homotopy.
\end{eg}

\begin{defn}
A pointed space $ (Y,y_0)$ is called an \allbold{H-space} if there exists maps $ \mu: Y \times Y \to Y$ and $ \nu: Y \to Y$ s.t.\ 
\begin{enumerate}[label=(\arabic*)]
	\item for $ i_1: y \mapsto (y,y_0)$ and $ i_2: y\mapsto (y_0,y)$, we have
		\[
\mu \circ i_1 \simeq  \text{id}_Y, \mu \circ i_2 \simeq  \text{id}_Y 
.\] 
\item The compositions
\begin{align*}
	Y \times (Y\times Y) \xrightarrow{ \text{id}_Y \times \mu } Y \times Y \xrightarrow{ \mu} Y  
\end{align*}
and
\begin{align*}
	(Y \times Y) \times Y \xrightarrow{ \mu \times \text{id}_Y } Y \times Y \xrightarrow{ \mu}  Y  
\end{align*}
are homotopic.
\item The composition
\begin{align*}
	Y \xrightarrow{ \text{id}_Y  \times \nu} Y \times Y \xrightarrow{ \mu} Y  
\end{align*}
\begin{align*}
	Y \xrightarrow{ \nu \times \text{id}_Y } \times Y \times Y \xrightarrow{ \mu}Y  
\end{align*}
are homotopic to constant maps.
\end{enumerate}
\end{defn}
\begin{remark}
This definition should remind us of group axioms: identity, association, and inverses. We see that $ \mu$ hints at multiplication whereas $ \nu$ hints at inversion.
\end{remark}
\begin{eg}
If $ G$ is a topological group (group with topology  s.t.\ multiplication and inverses are continuous maps).

Exercise: $ (G, e)$ is an H-space.
\end{eg}

\begin{thm}
	The set $ [X,Y]_0$ has a natural group structure for all pointed spaces X  iff $ Y$ is an H-space.
\end{thm}
Natural means if $ f: X \to X'$, then the induced map $ f^* : [X',Y]_0 \to [X,Y]_0$ is a homomorphism. That is, $ [-,Y]_0$ is a functor.

\begin{proof}
	$ (\impliedby)$: suppose $ Y$ is an H-space. Given  $ (X,x_0)$ and notice $ \mu : Y \times Y \to Y$ induces $ \mu^* : [X,Y\times Y]_0 \to [X,Y]_0$. There is also a canonical function
	\begin{align*}
	\phi:	[X,Y]_0 \times [X,Y]_0 \to [X, Y\times Y]_0, ([f],[g]) \mapsto [f \times g].
	\end{align*}
	Exercise: $ \phi$ is well-defined and a bijection. Define multiplication
	\begin{align*}
		m=\mu^*  \circ \phi : [X,Y]_0 \times [X,Y]_0 \to [X,Y]_0.
	\end{align*}
	This is clearly well-defined since $ \phi$ is well-defined and post-composing homotopic functions are still homotopic. Denote $ m([f],[g])$ by  $ [f] \cdot [g]$. Denote $ \nu_x ([f])$ by $ [f]^{-1}$. Let $ e(x)=y_0$ be the constant map. Now we check the group axioms:

	Identity: Tracking the representatives of $ [e] \cdot [g]$ yields
	\begin{align*}
		\mu(y_0,g(x)) &= (\mu \circ i_1) \circ g(x) \\
		& \cong \text{id}_Y \circ g(x) =g(x) 
	\end{align*}
	The other direction follows from using $ i_2$. Thus $ [e] \cdot  [g]= [g] = [g] \cdot  [e]$.

	Associativity: Given $ [f], [g], [h] \in [X, Y]_0$, we see that
	\begin{align*}
		([f] \cdot [g]) \cdot [h] &= \mu^* \circ \phi ([f],[g]) \cdot [h] \\
					  &= \mu^* ([f \times g]) \cdot [h]\\
					  &= [\mu \circ (f\times g)] \cdot [h] \\
					  &= [\mu \circ ( (\mu \circ (f \times g)) \times h)] \\
					  &= [\mu \circ (\mu \times \text{id}_{ Y}) \circ f\times g \times h] \\
					  &= [\mu \circ (\text{id}_{ Y}\times \mu) \circ f \times g \times h] && \text{ condition 2} \\
					  &= [\mu \circ (f \times (\mu \circ (g \times h)))] \\
					  &= [f] \cdot ([g] \cdot  [h]) 
	\end{align*}
	Inverse: Given $ [f] \in [X, Y]_0$, we have
	\begin{align*}
		[f] \cdot [f]^{-1} &= [\mu \circ (f \times (\nu \circ f))] \\
				   &= [\mu \circ (\text{id}_{ Y} \times \nu) \circ f] \\
				   &= [e]
	\end{align*}
	The other direction follows similarly.

	$ (\implies)$: Suppose $ [X,Y]_0$ has a natural group structure for all pointed spaces $ X$.  Take $ X = Y \times Y$, and $ p_1,p_2$ be the projections onto 1st and 2nd factors respectively. This yields $ [p_1], [p_2] \in [Y\times Y,Y]_0$. Let $ \mu$ be a representative of $ [p_1] \cdot [p_2]$. Let $\nu$ be a representative of $ [\text{id}_Y ]^{-1} $.

	Now check condition 1. $ i_1: Y \to Y \times Y, y \mapsto  (y,y_0)$ induces $ i_1^* $ so that
	\begin{align*}
		i_1^* ([p_1]) &= [p_1 \circ i_1] = [ \text{id}_Y ]\\
		i_1^* ([p_2]) &= [e]. 
	\end{align*}
	Therefore,
	\begin{align*}
		i_1^* ([\mu]) &=i^* ([p_1] \cdot [p_2]) \\ 
			      &= [ \text{id}_Y ] \cdot [ e].
	\end{align*}
	Since $ [e]$ is the identity on $ [Y,Y]_0$, $ [\mu \circ i_1] = [\text{ id}_Y ]$.

	2 and 3 are similar so left as exercises.
\end{proof}

\begin{defn}
If $ (Y,y_0)$ is a point space, then the \allbold{loop space} of $ Y$ is
 \begin{align*}
	 \Omega(Y) = C^{0}((I, \{0,1\} ),(Y,y_0)) = C^{0}((S^{1},x_0),(Y,y_0)).
\end{align*}
\end{defn}

\begin{lem}
$ \Omega(Y)$ is an H-space.
\end{lem}
\begin{proof}
Same as the proof for fundamental group. 
\end{proof}

\begin{defn}
Given pointed spaces $ (X,x_0), (Y,y_0)$, the \allbold{wedge product} is $ X \vee Y = (X \times \{y_0\}) \cup (\{x_0\}  \times Y) \subseteq X \times Y$ with base point $ (x_0,y_0)$. 
\end{defn}

\begin{defn}
A pointed space $ (Y,y_0)$ is an \allbold{H'-space} if there are maps $ \mu: Y \to Y \vee Y$ and $ \nu: Y \to Y$ s.t.\ 
\begin{enumerate}[label=(\arabic*)]
	\item $ p_1 \circ \mu \simeq \text{id}_{ Y}$ and $ p_2 \circ \mu \simeq \text{id}_{ Y}$ where $ p_1, p_2 : Y \vee Y \to Y$ are projections onto the 1st and 2nd factors.
	\item The compositions
		\begin{align*}
			Y \xrightarrow{ \mu} Y \vee Y \xrightarrow{ \text{id}_{ Y} \vee  \mu} Y \vee (Y\vee Y)
		\end{align*}
		and
		\begin{align*}
			Y \xrightarrow{ \mu} Y \vee Y \xrightarrow{ \mu \vee \text{id}_{ Y}} (Y \vee Y) \vee Y 
		\end{align*}
		are homotopic.
	\item The compositions
		\begin{align*}
			Y \xrightarrow{ \mu} Y \vee Y \xrightarrow{\nu \vee  \text{id}_{ Y}}  Y
		\end{align*}
		and
		\begin{align*}
			Y \xrightarrow{ \mu} Y \vee Y \xrightarrow{ \text{id}_{ Y} \vee \nu} Y  
		\end{align*}
		are homotopic to the constant map.
\end{enumerate}
\end{defn}

\begin{thm}
	The set $ [Y, X]_0$ has a natural group structure for all $ (X,x_0)$ iff $ Y$ is an H'-space.
\end{thm}
\begin{proof}
Exercise.

$ (\impliedby):$ suppose $ Y$ is a H'-space, then we have $\mu: Y \to Y \vee Y$ and $ \nu: Y \to Y$ that satisfy all the three conditions. 
\end{proof}

\begin{defn}
Given a space $ X$, its  \allbold{suspension} is
\begin{align*}
	\Sigma X = X \times I / \sim,
\end{align*}
where $ X \times \{0\} $ and $ X \times \{1\} $ are collapsed to two distinct points.
\end{defn}
\begin{remark}
	If $ (X,x_0)$ is pointed, then $ \sum X = X \times I / \{X \times \{0\} , X \times \{1\} , \{x_0\} \times I \} $ with base point $ [\{x_0\}] $.
\end{remark}

\begin{eg}
Collapsing the two bases of a cylinder yields
\begin{enumerate}[label=(\arabic*)]
	\item $ S^{n} = \Sigma S^{n-1}$.
	\item $ (S^{n},x_0)= \Sigma (S^{n-1},x_0)$.

		Exercise: if $ M$ is any manifold and  $ C$ is an arc in  $ M$, then prove  $ M /C$ is homeomorphic to  $ M$. (hint: prove it for $ n$-disk.)
\end{enumerate}
\end{eg}

\begin{lem}
For any pointed space $ (Y,y_0)$, its suspension $ \Sigma Y$ is an  H'-space.
\end{lem}
\begin{proof}
Define $ \mu: \Sigma Y \to \Sigma Y \vee \Sigma Y$. See ipad. Exercise.
\end{proof}

\begin{thm}
	If $ X$ is an  H'-space and  $ Y$ is an H-space, then the corresponding binary operations on $ [X, Y]_0$ agree and are commutative.
\end{thm}
\begin{proof}
	Denote the binary operation from H'-space by $ +$ and the other by  $ \cdot $. Let $ f_1, f_2$ be maps representing elements in $ [X, Y]_0$. See ipad for diagram. Let $ \Delta: X \to X \times X$ be the diagonal map, $ \nabla : Y \times Y \to Y, $Note that
	\begin{align*}
		[f_1] \cdot  [f_2] = [\mu_Y \circ (f_1 \times f_2) \circ \Delta],
		[f_1]+[f_2] =[\nabla \circ (f_1 \vee f_2) \circ \mu_X].
	\end{align*}
Condition 1 of H'-space says $ i \circ \mu_X \simeq \Delta$. Condition 1 of H-space says $ \mu_Y \circ j \simeq \nabla $. Now $ (x,x_0) \in X \vee X$, then $ i(x,x_0) = (x,x_0) \in X \times X$. So
\begin{align*}
	(f_1 \times f_2) \circ  i(x,x_0) &= (f_1(x),y_0) \\
	f_1 \vee f_2(x,x_0) &= (f_1(x),y_0) \\
	j \circ  (f_1 \vee f_2) (x,x_0) &= (f_1(x),y_0) 
\end{align*}
Similarly for $ (x_0,x)$ so center square in the diagram commutes.
\begin{align*}
	\nabla \circ  (f_1 \vee f_2) \circ  \mu_X &\simeq \mu_Y \circ j \circ (f_1 \vee f_2) \circ  \mu_X \\
	&= \mu_Y \circ  (f_1 \times f_2) \circ i \circ \mu_X\\
	&\simeq \mu_Y \circ  (f_1 \times f_2) \circ  \Delta
\end{align*}
It remains to show abelian. Fact: if $ \rho : G \times G \to G, (g,h)\mapsto gh$ is a homomorphism then $ G$ is abelian. To see this, notice
 \begin{align*}
	 \rho((g,h) (g^{-1},h^{-1})) &= \rho((gg^{-1},h h^{-1}))\\
	&= \rho(e,e)= e \\
	 \rho(g,h) \rho(g^{-1},h^{-1}) &= gh g^{-1} h^{-1}
\end{align*}

$ \mu_Y: Y \times Y \to Y$ induces a homomorphism $ [X, Y\times Y]_0 \xrightarrow{ \mu_Y} [X, Y]_0$. We also have the bijection
\begin{align*}
	\phi: [X, Y]_0 \times [X, Y]_0 \to [X, Y\times Y]_0, ([f],[g])\mapsto [f \times g].
\end{align*}
\begin{claim}
$ \phi$ is a homomorphism.
\end{claim}
Let $ p_1: Y \times Y \to Y$ be projection to 1st factor, which induces homomorphisms $ p_i: [X, Y\times Y]_0 \to [X, Y]_0$. Clearly $ \phi$ is the inverse of $ (p_1)_* \times (p_2)_*$ so $ \phi$ is a homomorphism. Then
\begin{align*}
	(\mu_Y)_* \circ \phi: [X, Y]_0 \times [X, Y]_0 \to [X, Y]_0, ([f],[g]) \mapsto [f] \cdot [g]
\end{align*}
So the group is abelian by the fact.
\end{proof}
\begin{defn}
A space is \allbold{locally compact} if every point has a neighborhood that is contained in a compact set. 
\end{defn}
\begin{lem}
If $ Y$ is locally compact and Hausdorff, there is a bijection
 \begin{align*}
	C^{0}(X\times Y,Z) \to C^{0}(X,C^{0}(Y,Z)).
\end{align*}
If $ X$ is also Hausdorff, this is a homeomorphism. Note that $ C^{0}$ is simply the $ \Hom$ functor.
\begin{remark}
Any manifold or CW-complexes are locally compact.
\end{remark}
\end{lem}
\begin{remark}
	The lemma implies that $ [X\times Y,Z] = [X,C^{0}(Y,Z)]$.
\end{remark}
\begin{defn}
Given $ C$ a compact set in  $ X$,  $ W$ an open set in  $ Y$, let  $ U(C,W) = \{f \in C^{0}(X,Y) : f(C) \subseteq W)\} $. This forms a subbasis for a topology on $ C^{0}(X,Y)$, called \allbold{compact-open topology}.
\end{defn}
Exercises:
\begin{enumerate}[label=(\arabic*)]
	\item If $ Y$ is a metric space, show that this topology is the topology of compact convergence,  \emph{i.e.} $ f_n \to f$ iff for all compact sets $ C \subseteq X$, $ f_n|_C \to f|_C$ uniformly.
	\item If $ f: X \times Y \to Z$ is continuous, then so is
		\begin{align*}
			F: X \to C^{0}(Y,Z), x\mapsto f_x: Y \to Z, y \mapsto f(x,y)
		\end{align*}
	\item The converse is true if $ Y$ is locally compact.
	\item Prove the theorem.
\end{enumerate}

\begin{proof}
We need a topology on $ C^{0}$: the compact-open topology. 
\end{proof}

\begin{defn}
The \allbold{smash product} of two pointed spaces is
\begin{align*}
	X \wedge Y = X \times Y / X \vee Y = X \times Y / X \times \{y_0\} \cup \{x_0\} \times Y  .
\end{align*}
\end{defn}
Recall that the \allbold{reduced suspension} is
\begin{align*}
	\Sigma X = S^{1} \wedge X = S^{1} \times X / S^{1} \times \{x_0\} \cup \{e_0\} \times X .
\end{align*}
See ipad.

\begin{coro}
If $ Y$ is locally compact, then
 \begin{align*}
	 [X \wedge Y, Z]_0 = [X, C^{0}_{ \text{based} } (Y,Z)]_0.
\end{align*}
\end{coro}
\begin{proof}
	If $ f \in C^{0}_{ \text{based} }(X,C^{0}_{ \text{based} }(Y,Z))$ then it has to send base point to base point: $ f(x_0) =$ constant map $ Y \to Z, y_0 \mapsto z_0$. So $ F: X \times Y \to Z$, $ (x,y) \mapsto f(x)(y)$ sends $ \{x_0\} \times Y \to z_0 $ by the lemma. As $ f(x): Y \to Z$ sends $ y_0 \to z_0$, $ F$ induces a map on  $ X \wedge Y = X \times Y / X \vee Y \xrightarrow{ F} Z$. So $ F \in [X \wedge Y, Z]_0$.

	We can similarly define an inverse: exercise.
\end{proof}

Recall that a loop space is $ \Omega(X) = C^{0}_{ \text{based} }(S^{1},X)$.

\begin{coro}
	$ [\Sigma X, Y]_0 = [X, \Omega(Y)]_0$. That is, suspension is the adjoint of looping.
\end{coro}
\begin{proof}
\begin{align*}
	[\Sigma X, Y]_0 &= [S^{1} \wedge X, Y]_0 \\
		      &= [X, C^{0}_{ \text{based} }(S^{1},Y)]_0 \\
		      &= [X, \Omega(Y)]_0 \\
\end{align*}
\end{proof}
\begin{remark}
They are isomorphic as groups TODO
\end{remark}
\begin{defn}
The \allbold{$ n$th homotopy group} of $ (X,x_0)$ is
\begin{align*}
	\pi_n(X) = [S^{n}, X]_0.
\end{align*}
\end{defn}
Note that
\begin{enumerate}[label=(\arabic*)]
	 \item
		 \begin{align*}
			 \pi_n(X) &= [S^{n}, X]_0 \\
				  &= [\Sigma S^{n-1}, X]_0 \\
				  &= [S^{n-1}, \Omega(X)]_0 \\
				  &= [S^{0}, \Omega^{n}(X)]_0 
		 \end{align*}
		 So $ \pi_n(X) = \pi_0(\Omega^{n}(X))$, the path components of $ \Omega(X)$.
	 \item $ \pi_n(X) = [S^{n-1}, \Omega(X)]_0$. If $ n\geq 2$,  $ \Omega(X)$ is H-space, $ S^{n-1}$ is H'-space, so by Theorem $ \pi_n(X)$ is abelian for $ n \geq 2$.
	 \item If  $ X$ is a Lie group, the it is an H-space so $ \pi_1(X)$ is abelian.
\end{enumerate}
\end{document}
