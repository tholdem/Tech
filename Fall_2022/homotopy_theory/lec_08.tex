\documentclass[12pt,class=article,crop=false]{standalone} 
\newcommand{\alert}[1]{{\bf \color{red} [Alert:] #1}}
\newcommand{\todo}[1]{{\bf \color{orange} [TODO:] #1}}
\newcommand{\real}[1][]{\mathbb{R}^{#1}}
\newcommand{\myeqn}[1]{(\ref{#1})}
\newcommand{\myex}[1]{Example \ref{#1}}
\newcommand{\defeq}{\stackrel{\mathrm{def}}{=}}
\newcommand{\parder}[2]{\frac{\partial #1}{\partial #2}}
\newcommand{\Lie}[3][]{\mathsf{L}_{#3}^{#1} #2}
\newcommand{\LieA}[1]{\mathsf{Lie}(#1)}
\newcommand{\lieder}[2]{\mathcal{L}_{#2} #1}
\renewcommand{\t}{^{\mbox{\tiny\sf T}}}
\newcommand{\trans}{^{\mbox{\tiny\sf T}}}
\newcommand{\markup}[1]{\{\textbf{#1}\}}
\newcommand{\msub}[1]{_\mathrm{#1}}
\newcommand{\msup}[1]{^\mathrm{#1}}
\newcommand{\inv}[1]{#1^{-1}}
\newcommand{\pinv}[1]{{#1}^{+}}
\newcommand{\myfracA}[2]{\displaystyle{\frac{#1}{#2}}}
\newcommand{\myfracB}[2]{{#1}/{#2}}
\newcommand{\mydiffA}[1]{\dot{#1}}
\newcommand{\mydiffB}[2]{\myfracA{\mathrm{d}{#1}}{\mathrm{d}{#2}}}
\newcommand{\ball}[2]{\mathcal{B}_{#1}\left(#2\right)}
\newcommand{\acos}[1]{\cos^{-1}\left(#1\right)}
\newcommand{\asin}[1]{\sin^{-1}\left(#1\right)}
\newcommand{\mani}{\mathcal{M}}
\newcommand{\tang}[2]{\mathsf{T}_{#1} #2}
\newcommand{\LieB}[2]{[ #1, #2 ]}
\newcommand{\LieBad}[3][]{\mathsf{ad}_{#2}^{#1} #3}
\newcommand{\ReachVT}{\mathcal{R}^V_T}
\newcommand{\ReachVt}{\mathcal{R}^V_t}
\newcommand{\ReachVTe}{\mathcal{R}^V_{\le T}}
\newcommand{\ReachT}{\mathcal{R}_T}
\newcommand{\Reacht}{\mathcal{R}_t}
\newcommand{\ReachTe}{\mathcal{R}_{\le T}}
\newcommand{\accLA}[1]{\mathsf{Lie}(#1)}
\newcommand{\accD}{\Delta_{\mathcal{F}}}
\newcommand{\accSA}{\mathsf{Lie}(\mathcal{G},f)}
\newcommand{\accDS}{\Delta_{\mathcal{G}}}
\newcommand{\eval}[3]{\mathsf{Ev}^{#2}_{#1}\left( #3 \right)}
\newcommand{\stlc}{\textsc{stlc}}
\newcommand{\clf}{\textsc{clf}}
\newcommand{\jqlf}{\textsc{jqlf}}
\newcommand{\dlas}{\textsc{dlas}}
\newcommand{\Ad}[2]{\mathsf{Ad}_{#1} #2}
\newcommand{\xe}{\ensuremath{x_e}}
\newcommand{\lebg}[1]{\mathcal{L}_{#1}}
\newcommand{\lebgx}[1]{\mathcal{L}_{#1 \mathrm{e}}}
\newcommand{\dom}{D}
\newcommand{\domT}{[t_0,\infty) \times D}
\newcommand{\rarrow}{\rightarrow}
\renewcommand{\d}{\mathrm{d}}
\renewcommand{\Re}{\mathbb{R}}
\newcommand{\C}{\mathrm{C}}

\newcommand{\QED}{{\unskip\nobreak\hfil\penalty50\hskip2em\vadjust{}
		\nobreak\hfil$\Box$\parfillskip=0pt\finalhyphendemerits=0\par}\vspace{0.1cm}}
\newcommand{\eoEx}{{\unskip\nobreak\hfil\penalty50\hskip0em\vadjust{}
		\nobreak\hfil$\Large\Diamond$\parfillskip=0pt\finalhyphendemerits=0\par}\vspace{0.1cm}}

\newcommand{\sgn}{\ensuremath{\operatorname{sgn}}}
\newcommand{\sat}{\ensuremath{\operatorname{sat}}}

\newcommand{\half}{\frac{1}{2}}
\newcommand{\shalf}{\mbox{$\frac{1}{2}$}}
\newcommand{\marcom}[1]{\marginpar{\footnotesize #1}}
\newcommand{\der}{\mathrm{D}}
\newcommand{\e}{\mathrm{e}}
\newcommand{\dt}{\mathrm{d}t}

\newcommand{\cA}{\ensuremath{\mathcal{A}}}
\newcommand{\cB}{\ensuremath{\mathcal{B}}}
\newcommand{\cG}{\ensuremath{\mathcal{G}}}
\newcommand{\cK}{\ensuremath{\mathcal{K}}}
\newcommand{\cW}{\ensuremath{\mathcal{W}}}
\newcommand{\cZ}{\ensuremath{\mathcal{Z}}}
\newcommand{\cS}{\ensuremath{\mathcal{S}}}
\newcommand{\cD}{\ensuremath{\mathcal{D}}}
\newcommand{\cP}{\ensuremath{\mathcal{P}}}
\newcommand{\cV}{\ensuremath{\mathcal{V}}}
\newcommand{\cL}{\ensuremath{\mathcal{L}}}
\newcommand{\cN}{\ensuremath{\mathcal{N}}}
\newcommand{\cI}{\ensuremath{\mathcal{I}}}
\newcommand{\cR}{\ensuremath{\mathcal{R}}}
\newcommand{\cM}{\ensuremath{\mathcal{M}}}
\newcommand{\cC}{\ensuremath{\mathcal{C}}}
\newcommand{\cF}{\ensuremath{\mathcal{F}}}
\newcommand{\cH}{\ensuremath{\mathcal{H}}}
\newcommand{\cO}{\ensuremath{\mathcal{O}}}
\newcommand{\cX}{\ensuremath{\mathcal{X}}}
\newcommand{\cY}{\ensuremath{\mathcal{Y}}}
\newcommand{\Ci}{\ensuremath{\mathcal{C}^\infty}}
\newcommand{\ISS}{\textsc{iss}}
\newcommand{\LISS}{\textsc{liss}}
\newcommand{\GAS}{\textsc{gas}}
\newcommand{\GS}{\textsc{gs}}
\newcommand{\LES}{\textsc{les}}
\newcommand{\GUAS}{\textsc{guas}}
\newcommand{\BIBO}{\textsc{bibo}}
\newcommand{\spec}{\ensuremath{\operatorname{spec}}}
\newcommand{\spn}{\ensuremath{\operatorname{span}}}
\renewcommand{\i}{\mathrm{i\,}}

\renewcommand{\implies}{\Rightarrow}

\renewcommand{\theenumi}{$\roman{enumi})$}
\renewcommand{\labelenumi}{\theenumi}

\font\ptmten=zptmcmrm scaled 1200
\newcommand{\w}{\mbox{{\ptmten w}}}
\newcommand{\z}{\mbox{{\ptmten z}}}
\renewcommand{\Re}{\mathbb{R}}

\newcommand{\cl}{\operatorname{cl}}
\newcommand{\intr}{\operatorname{int}}
\newcommand{\rank}{\operatorname{rank}}
\newcommand{\co}{\operatorname{co}}
\newcommand{\aff}{\operatorname{aff}}

\theoremstyle{plain}
\newtheorem{theorem}{Theorem}[chapter]
\newtheorem{claim}[theorem]{Claim}
\newtheorem{corollary}[theorem]{Corollary}
\newtheorem{prop}[theorem]{Proposition}
\newtheorem{fact}[theorem]{Fact}
\newtheorem{lemma}[theorem]{Lemma}

\newtheorem{remark}{Remark}[chapter]

\theoremstyle{definition}
\newtheorem{assume}[theorem]{Assumption}
\newtheorem{defn}[theorem]{Definition}
\newtheorem{problem}[theorem]{Problem}
\newtheorem{exercise}{Exercise}
\newtheorem{example}[theorem]{Example}


\begin{document}
\section{Structure gorups of fiber bundles}
Given a local trivial fibration Diagram.
\begin{align*}
	\phi_2 \circ \phi_1 ^{-1} : (U_1 \cap U_2) \times F \to (U_1 \cap U_2) \times F, (x,y) \to (x,\tau_{21}(x)(y))
\end{align*}
where $ \tau_{21}:(U_1 \cap U_2) \to \text{Homeo}(F)$ (compact-open), which is called a \allbold{transition (or clutching) function}.

\begin{remark}
If $ \{(U_{ \alpha}, \phi_{ \alpha})\} $ is a collection of local trivializations s.t.\ $ B = \bigcup_{ \alpha} U_{ \alpha} $. Then the transition maps satisfy $ (*)$:
\begin{align*}
	\tau_{ \alpha \alpha}(x) &= \text{id}_{ F}\\
	(\tau_{ \alpha \beta(x)})^{-1}&= \tau_{ \beta \alpha}(x) \\
	\tau_{ \gamma \beta}(x) \circ \tau_{ \beta \alpha}(x) &= \tau_{ \gamma \alpha}(x)
\end{align*}
Exercise: show that if $ \{U_{ \alpha}\} $ is a cover of $ B$ by open sets and  $ \tau_{ \alpha \beta}:(U_{ \alpha} \cap U_{ \beta}) \to \text{Homeo}(F) $ are maps satisfying $ (*)$, then there exists a bundle  $ E$ over  $ B$ realizing this data as transition maps.

Hint: let $ E = \bigsqcup_{ U_{ \alpha} \times F} / \sim$ where $ (x,y) \in U_{ \alpha} \times F \sim (x',y') \in U_{ \beta}\times F \iff x=x'$ and $ \tau_{ \beta \alpha}(x)(y)=y'$. There is an obvious projection $ p: E \to B$. Prove this is a bundle. 
\end{remark}

Exercise: find the transition maps for Diagram.

 \begin{defn}
Suppose $ G \subseteq \text{Homeo}(F) $ is a topological group. If diagram has a collection of transition functions
\begin{align*}
	\tau_{ \alpha \beta}: U_{ \alpha} \cap U_{ \beta} \to G,
\end{align*}
then we say $ E$ has  \allbold{structure group} $ G$. 

If the transition functions do not map into $ G$ but can be homtoped through functions satisfying  $ (*)$ to one with image in  $ G$, then we say the structure group of  $ E$  \allbold{reduces} to $ G$. 
\end{defn}

\begin{remark}
If $ G$ preserves some structure on $ F$, then the fibers of  $ p:E \to B$ will have this structure.
\end{remark}
\begin{eg}
\begin{enumerate}[label=(\arabic*)]
	\item If $ F = \rr^{n}$ and $ G= \text{GL}_{ n}( R) \subseteq  \text{Homeo}( \rr^{n}) $, then each fiber of a bundle with structure group $ G$ has a linear structure.
	\item If  $ F = \rr^{n}$ and $ G = \text{GL}_{ n}^{+}(\rr) $, then fibers of $ F$ are oriented vector spaces so  $ E$ is an  \allbold{oriented vector bundle}.
	\item If $ F = \rr^{n}$ and $ G = O(n)$. Then  $ E$ is a vector bundle with a metric (inner product). Note  $ O(n) \to \text{GL}_{ n}(\rr) $ (inclusion) is a homotopy equivalence. Hence all vector bundles admit metrics.
	\item If $ F = \rr^{2n} = \cc^{n}$, then $ G = \text{GL}_{ n}( \cc) \iff E$ has a complex structure. $ G = U(n) \iff E$ has a Hermitian structure.
	\item If $ F = \rr^{n}$ and $ G= \text{GL}_{k}( \rr) \times \text{GL}_{n-k}(\rr) \subseteq \text{GL}_{ n}( \rr) ,(A,B) \mapsto \begin{pmatrix} A&0\\0&B \end{pmatrix}  $, then $ E$ has a  $ G$-structure  $ \iff E \cong E_1 \oplus E_2$ where $ E_1$ is an $ \rr^{k}$-bundle, $ E_2$ is an $ \rr^{n-k}$-bundle. Similarly of $ G = \text{GL}_{n-k}(\rr) \subseteq \text{GL}_{ n}(\rr) $, then $ E$ has a  $ G$-structure  $ \iff E \cong E^{1} \oplus \rr^{k}$ where $ E^{1}$ is an $ \rr^{n-k}$-bundle.
\end{enumerate}
\end{eg}

Question: when can we reduce the structure group?

\begin{defn}
If $ G$ is a Lie group (topological group), then a bundle  diagram is a  \allbold{principal $ G$-bundle} if there exists a smooth (or continuous) right $ G$-action  $ P \times G \to P$  s.t.\ 
\begin{enumerate}[label=(\arabic*)]
	\item action preserves fibers, \emph{i.e.} $ y \in p ^{-1}(x) \implies y . g \in p ^{-1}(x) \ \forall \ x,y,g$.
	\item $ G$ acts freely and transitively on  $ p ^{-1}(x) \ \forall \ x$.
\end{enumerate}
\end{defn}
\begin{remark}
This can also be defined as a smooth manifold $ P$ with a smooth right  $ G$-action that is free and proper, \emph{i.e.} for the map $ P \times G \to P \times P, (p,g) \mapsto (p.g,p)$ preimages of compact sets are compact.
\end{remark}

\begin{eg}
\begin{enumerate}[label=(\arabic*)]
	\item If $ (E,B,F,p)$ is a bundle with structure group  $ G$, then there is a cover of  $ B$ by local trivialization  $ \{(U_{ \alpha}, \phi_{ \alpha}\} $ with transition functions $ \tau_{ \alpha \beta} : (U_{ \alpha} \cap U_{ \beta}) \to G$ we can construct a principal $ G$-bundle as follows
		 \begin{align*}
			P_E = \bigsqcup_{ \alpha} U_{ \alpha} \times G / \sim
		\end{align*}
		when $ (x,g) \in U_{ \alpha} \sim (x',g') \in U_{ \beta} \times G \iff x=x'$ and $ \tau_{ \beta \alpha}(x) g= g'$. Exercise: show this is a principal $ G$-bundle.

		If $ E$ is a vector bundle then  $ P_E$ is a principal  $ \text{GL}_{ n}( \rr) $-bundle. It is called the \allbold{frame bundle} because you can think of points in the fibers of $ P_E$ as frames for the fibers of  $ E$. Exercise: think through this. We denote this $ \mathcal{ F}(E)$. Note $O(n) \simeq \text{GL}_{ n}(\rr) $ so we could take $ \mathcal{ F}(E)$ to be a principal $ O(n)$-bundle.
	\item diagram is a principal  $ S^{1}$-bundle.
	\item Regular covering spaces are of a manifold are principal bundles. Exercise: check this and what are fibers? Can an irregular cover be a principal bundle?
\end{enumerate}
\end{eg}

Exercise:
\begin{enumerate}[label=(\arabic*)]
	\item Show a prinicpal $ G$-bundle is trivial iff it admits a section.
	\item If  $ E$ is a vector bundle, the sections of  $ E$ are the same as  $ \text{GL}_{ n}(\rr) $-equivariant maps $ v: \mathcal{ F}(E) \to \rr^{n}$, \emph{i.e.} $ v(y.g) = g^{-1} v(y)$. Hint: given $ s: B \to E$ then for each $ y \in \mathcal{ F}(E)$, let $ v(y) = s(p(y))$ expressed in the frame  $ y$. Then $ p: \mathcal{ F}(E) \to B$. This allows us to turn sections into functions which are easier to work with.
\end{enumerate}

Given $ P \to B$ a principal $ G$-bundle, and  $ \rho: G \to G'$ is a homomorphism, where $ G' \subseteq \text{Homeo}( F) $. Then we can construct an $ F$-bundle with structure group  $ G'$ as follows
 \begin{align*}
	P \times _{\rho} F = P \times F / (p.g, f) \sim (p,\rho(g)f)
\end{align*}
\end{document}
