\documentclass[12pt,class=article,crop=false]{standalone} 
%Fall 2022
% Some basic packages
\usepackage{standalone}[subpreambles=true]
\usepackage[utf8]{inputenc}
\usepackage[T1]{fontenc}
\usepackage{textcomp}
\usepackage[english]{babel}
\usepackage{url}
\usepackage{graphicx}
%\usepackage{quiver}
\usepackage{float}
\usepackage{enumitem}
\usepackage{lmodern}
\usepackage{comment}
\usepackage{hyperref}
\usepackage[usenames,svgnames,dvipsnames]{xcolor}
\usepackage[margin=1in]{geometry}
\usepackage{pdfpages}

\pdfminorversion=7

% Don't indent paragraphs, leave some space between them
\usepackage{parskip}

% Hide page number when page is empty
\usepackage{emptypage}
\usepackage{subcaption}
\usepackage{multicol}
\usepackage[b]{esvect}

% Math stuff
\usepackage{amsmath, amsfonts, mathtools, amsthm, amssymb}
\usepackage{bbm}
\usepackage{stmaryrd}
\allowdisplaybreaks

% Fancy script capitals
\usepackage{mathrsfs}
\usepackage{cancel}
% Bold math
\usepackage{bm}
% Some shortcuts
\newcommand{\rr}{\ensuremath{\mathbb{R}}}
\newcommand{\zz}{\ensuremath{\mathbb{Z}}}
\newcommand{\qq}{\ensuremath{\mathbb{Q}}}
\newcommand{\nn}{\ensuremath{\mathbb{N}}}
\newcommand{\ff}{\ensuremath{\mathbb{F}}}
\newcommand{\cc}{\ensuremath{\mathbb{C}}}
\newcommand{\ee}{\ensuremath{\mathbb{E}}}
\newcommand{\hh}{\ensuremath{\mathbb{H}}}
\renewcommand\O{\ensuremath{\emptyset}}
\newcommand{\norm}[1]{{\left\lVert{#1}\right\rVert}}
\newcommand{\dbracket}[1]{{\left\llbracket{#1}\right\rrbracket}}
\newcommand{\ve}[1]{{\bm{#1}}}
\newcommand\allbold[1]{{\boldmath\textbf{#1}}}
\DeclareMathOperator{\lcm}{lcm}
\DeclareMathOperator{\im}{im}
\DeclareMathOperator{\coim}{coim}
\DeclareMathOperator{\dom}{dom}
\DeclareMathOperator{\tr}{tr}
\DeclareMathOperator{\rank}{rank}
\DeclareMathOperator*{\var}{Var}
\DeclareMathOperator*{\ev}{E}
\DeclareMathOperator{\dg}{deg}
\DeclareMathOperator{\aff}{aff}
\DeclareMathOperator{\conv}{conv}
\DeclareMathOperator{\inte}{int}
\DeclareMathOperator*{\argmin}{argmin}
\DeclareMathOperator*{\argmax}{argmax}
\DeclareMathOperator{\graph}{graph}
\DeclareMathOperator{\sgn}{sgn}
\DeclareMathOperator*{\Rep}{Rep}
\DeclareMathOperator{\Proj}{Proj}
\DeclareMathOperator{\mat}{mat}
\DeclareMathOperator{\diag}{diag}
\DeclareMathOperator{\aut}{Aut}
\DeclareMathOperator{\gal}{Gal}
\DeclareMathOperator{\inn}{Inn}
\DeclareMathOperator{\edm}{End}
\DeclareMathOperator{\Hom}{Hom}
\DeclareMathOperator{\ext}{Ext}
\DeclareMathOperator{\tor}{Tor}
\DeclareMathOperator{\Span}{Span}
\DeclareMathOperator{\Stab}{Stab}
\DeclareMathOperator{\cont}{cont}
\DeclareMathOperator{\Ann}{Ann}
\DeclareMathOperator{\Div}{div}
\DeclareMathOperator{\curl}{curl}
\DeclareMathOperator{\nat}{Nat}
\DeclareMathOperator{\gr}{Gr}
\DeclareMathOperator{\vect}{Vect}
\DeclareMathOperator{\id}{id}
\DeclareMathOperator{\Mod}{Mod}
\DeclareMathOperator{\sign}{sign}
\DeclareMathOperator{\Surf}{Surf}
\DeclareMathOperator{\fcone}{fcone}
\DeclareMathOperator{\Rot}{Rot}
\DeclareMathOperator{\grad}{grad}
\DeclareMathOperator{\atan2}{atan2}
\DeclareMathOperator{\Ric}{Ric}
\let\vec\relax
\DeclareMathOperator{\vec}{vec}
\let\Re\relax
\DeclareMathOperator{\Re}{Re}
\let\Im\relax
\DeclareMathOperator{\Im}{Im}
% Put x \to \infty below \lim
\let\svlim\lim\def\lim{\svlim\limits}

%wide hat
\usepackage{scalerel,stackengine}
\stackMath
\newcommand*\wh[1]{%
\savestack{\tmpbox}{\stretchto{%
  \scaleto{%
    \scalerel*[\widthof{\ensuremath{#1}}]{\kern-.6pt\bigwedge\kern-.6pt}%
    {\rule[-\textheight/2]{1ex}{\textheight}}%WIDTH-LIMITED BIG WEDGE
  }{\textheight}% 
}{0.5ex}}%
\stackon[1pt]{#1}{\tmpbox}%
}
\parskip 1ex

%Make implies and impliedby shorter
\let\implies\Rightarrow
\let\impliedby\Leftarrow
\let\iff\Leftrightarrow
\let\epsilon\varepsilon

% Add \contra symbol to denote contradiction
\usepackage{stmaryrd} % for \lightning
\newcommand\contra{\scalebox{1.5}{$\lightning$}}

% \let\phi\varphi

% Command for short corrections
% Usage: 1+1=\correct{3}{2}

\definecolor{correct}{HTML}{009900}
\newcommand\correct[2]{\ensuremath{\:}{\color{red}{#1}}\ensuremath{\to }{\color{correct}{#2}}\ensuremath{\:}}
\newcommand\green[1]{{\color{correct}{#1}}}

% horizontal rule
\newcommand\hr{
    \noindent\rule[0.5ex]{\linewidth}{0.5pt}
}

% hide parts
\newcommand\hide[1]{}

% si unitx
\usepackage{siunitx}
\sisetup{locale = FR}

%allows pmatrix to stretch
\makeatletter
\renewcommand*\env@matrix[1][\arraystretch]{%
  \edef\arraystretch{#1}%
  \hskip -\arraycolsep
  \let\@ifnextchar\new@ifnextchar
  \array{*\c@MaxMatrixCols c}}
\makeatother

\renewcommand{\arraystretch}{0.8}

\renewcommand{\baselinestretch}{1.5}

\usepackage{graphics}
\usepackage{epstopdf}

\RequirePackage{hyperref}
%%
%% Add support for color in order to color the hyperlinks.
%% 
\hypersetup{
  colorlinks = true,
  urlcolor = blue,
  citecolor = blue
}
%%fakesection Links
\hypersetup{
    colorlinks,
    linkcolor={red!50!black},
    citecolor={green!50!black},
    urlcolor={blue!80!black}
}
%customization of cleveref
\RequirePackage[capitalize,nameinlink]{cleveref}[0.19]

% Per SIAM Style Manual, "section" should be lowercase
\crefname{section}{section}{sections}
\crefname{subsection}{subsection}{subsections}
\Crefname{section}{Section}{Sections}
\Crefname{subsection}{Subsection}{Subsections}

% Per SIAM Style Manual, "Figure" should be spelled out in references
\Crefname{figure}{Figure}{Figures}

% Per SIAM Style Manual, don't say equation in front on an equation.
\crefformat{equation}{\textup{#2(#1)#3}}
\crefrangeformat{equation}{\textup{#3(#1)#4--#5(#2)#6}}
\crefmultiformat{equation}{\textup{#2(#1)#3}}{ and \textup{#2(#1)#3}}
{, \textup{#2(#1)#3}}{, and \textup{#2(#1)#3}}
\crefrangemultiformat{equation}{\textup{#3(#1)#4--#5(#2)#6}}%
{ and \textup{#3(#1)#4--#5(#2)#6}}{, \textup{#3(#1)#4--#5(#2)#6}}{, and \textup{#3(#1)#4--#5(#2)#6}}

% But spell it out at the beginning of a sentence.
\Crefformat{equation}{#2Equation~\textup{(#1)}#3}
\Crefrangeformat{equation}{Equations~\textup{#3(#1)#4--#5(#2)#6}}
\Crefmultiformat{equation}{Equations~\textup{#2(#1)#3}}{ and \textup{#2(#1)#3}}
{, \textup{#2(#1)#3}}{, and \textup{#2(#1)#3}}
\Crefrangemultiformat{equation}{Equations~\textup{#3(#1)#4--#5(#2)#6}}%
{ and \textup{#3(#1)#4--#5(#2)#6}}{, \textup{#3(#1)#4--#5(#2)#6}}{, and \textup{#3(#1)#4--#5(#2)#6}}

% Make number non-italic in any environment.
\crefdefaultlabelformat{#2\textup{#1}#3}

% Environments
\makeatother
% For box around Definition, Theorem, \ldots
%%fakesection Theorems
\usepackage{thmtools}
\usepackage[framemethod=TikZ]{mdframed}

\theoremstyle{definition}
\mdfdefinestyle{mdbluebox}{%
	roundcorner = 10pt,
	linewidth=1pt,
	skipabove=12pt,
	innerbottommargin=9pt,
	skipbelow=2pt,
	nobreak=true,
	linecolor=blue,
	backgroundcolor=TealBlue!5,
}
\declaretheoremstyle[
	headfont=\sffamily\bfseries\color{MidnightBlue},
	mdframed={style=mdbluebox},
	headpunct={\\[3pt]},
	postheadspace={0pt}
]{thmbluebox}

\mdfdefinestyle{mdredbox}{%
	linewidth=0.5pt,
	skipabove=12pt,
	frametitleaboveskip=5pt,
	frametitlebelowskip=0pt,
	skipbelow=2pt,
	frametitlefont=\bfseries,
	innertopmargin=4pt,
	innerbottommargin=8pt,
	nobreak=false,
	linecolor=RawSienna,
	backgroundcolor=Salmon!5,
}
\declaretheoremstyle[
	headfont=\bfseries\color{RawSienna},
	mdframed={style=mdredbox},
	headpunct={\\[3pt]},
	postheadspace={0pt},
]{thmredbox}

\declaretheorem[%
style=thmbluebox,name=Theorem,numberwithin=section]{thm}
\declaretheorem[style=thmbluebox,name=Lemma,sibling=thm]{lem}
\declaretheorem[style=thmbluebox,name=Proposition,sibling=thm]{prop}
\declaretheorem[style=thmbluebox,name=Corollary,sibling=thm]{coro}
\declaretheorem[style=thmredbox,name=Example,sibling=thm]{eg}

\mdfdefinestyle{mdgreenbox}{%
	roundcorner = 10pt,
	linewidth=1pt,
	skipabove=12pt,
	innerbottommargin=9pt,
	skipbelow=2pt,
	nobreak=true,
	linecolor=ForestGreen,
	backgroundcolor=ForestGreen!5,
}

\declaretheoremstyle[
	headfont=\bfseries\sffamily\color{ForestGreen!70!black},
	bodyfont=\normalfont,
	spaceabove=2pt,
	spacebelow=1pt,
	mdframed={style=mdgreenbox},
	headpunct={ --- },
]{thmgreenbox}

\declaretheorem[style=thmgreenbox,name=Definition,sibling=thm]{defn}

\mdfdefinestyle{mdgreenboxsq}{%
	linewidth=1pt,
	skipabove=12pt,
	innerbottommargin=9pt,
	skipbelow=2pt,
	nobreak=true,
	linecolor=ForestGreen,
	backgroundcolor=ForestGreen!5,
}
\declaretheoremstyle[
	headfont=\bfseries\sffamily\color{ForestGreen!70!black},
	bodyfont=\normalfont,
	spaceabove=2pt,
	spacebelow=1pt,
	mdframed={style=mdgreenboxsq},
	headpunct={},
]{thmgreenboxsq}
\declaretheoremstyle[
	headfont=\bfseries\sffamily\color{ForestGreen!70!black},
	bodyfont=\normalfont,
	spaceabove=2pt,
	spacebelow=1pt,
	mdframed={style=mdgreenboxsq},
	headpunct={},
]{thmgreenboxsq*}

\mdfdefinestyle{mdblackbox}{%
	skipabove=8pt,
	linewidth=3pt,
	rightline=false,
	leftline=true,
	topline=false,
	bottomline=false,
	linecolor=black,
	backgroundcolor=RedViolet!5!gray!5,
}
\declaretheoremstyle[
	headfont=\bfseries,
	bodyfont=\normalfont\small,
	spaceabove=0pt,
	spacebelow=0pt,
	mdframed={style=mdblackbox}
]{thmblackbox}

\theoremstyle{plain}
\declaretheorem[name=Question,sibling=thm,style=thmblackbox]{ques}
\declaretheorem[name=Remark,sibling=thm,style=thmgreenboxsq]{remark}
\declaretheorem[name=Remark,sibling=thm,style=thmgreenboxsq*]{remark*}
\newtheorem{ass}[thm]{Assumptions}

\theoremstyle{definition}
\newtheorem*{problem}{Problem}
\newtheorem{claim}[thm]{Claim}
\theoremstyle{remark}
\newtheorem*{case}{Case}
\newtheorem*{notation}{Notation}
\newtheorem*{note}{Note}
\newtheorem*{motivation}{Motivation}
\newtheorem*{intuition}{Intuition}
\newtheorem*{conjecture}{Conjecture}

% Make section starts with 1 for report type
%\renewcommand\thesection{\arabic{section}}

% End example and intermezzo environments with a small diamond (just like proof
% environments end with a small square)
\usepackage{etoolbox}
\AtEndEnvironment{vb}{\null\hfill$\diamond$}%
\AtEndEnvironment{intermezzo}{\null\hfill$\diamond$}%
% \AtEndEnvironment{opmerking}{\null\hfill$\diamond$}%

% Fix some spacing
% http://tex.stackexchange.com/questions/22119/how-can-i-change-the-spacing-before-theorems-with-amsthm
\makeatletter
\def\thm@space@setup{%
  \thm@preskip=\parskip \thm@postskip=0pt
}

% Fix some stuff
% %http://tex.stackexchange.com/questions/76273/multiple-pdfs-with-page-group-included-in-a-single-page-warning
\pdfsuppresswarningpagegroup=1


% My name
\author{Jaden Wang}



\begin{document}
\section{Structure gorups of fiber bundles}
Given a local trivial fibration Diagram.
\begin{align*}
	\phi_2 \circ \phi_1 ^{-1} : (U_1 \cap U_2) \times F \to (U_1 \cap U_2) \times F, (x,y) \to (x,\tau_{21}(x)(y))
\end{align*}
where $ \tau_{21}:(U_1 \cap U_2) \to \text{Homeo}(F)$ (compact-open), which is called a \allbold{transition (or clutching) function}.

\begin{remark}
If $ \{(U_{ \alpha}, \phi_{ \alpha})\} $ is a collection of local trivializations s.t.\ $ B = \bigcup_{ \alpha} U_{ \alpha} $. Then the transition maps satisfy $ (*)$:
\begin{align*}
	\tau_{ \alpha \alpha}(x) &= \text{id}_{ F}\\
	(\tau_{ \alpha \beta(x)})^{-1}&= \tau_{ \beta \alpha}(x) \\
	\tau_{ \gamma \beta}(x) \circ \tau_{ \beta \alpha}(x) &= \tau_{ \gamma \alpha}(x)
\end{align*}
Exercise: show that if $ \{U_{ \alpha}\} $ is a cover of $ B$ by open sets and  $ \tau_{ \alpha \beta}:(U_{ \alpha} \cap U_{ \beta}) \to \text{Homeo}(F) $ are maps satisfying $ (*)$, then there exists a bundle  $ E$ over  $ B$ realizing this data as transition maps.

Hint: let $ E = \bigsqcup_{ U_{ \alpha} \times F} / \sim$ where $ (x,y) \in U_{ \alpha} \times F \sim (x',y') \in U_{ \beta}\times F \iff x=x'$ and $ \tau_{ \beta \alpha}(x)(y)=y'$. There is an obvious projection $ p: E \to B$. Prove this is a bundle. 
\end{remark}

Exercise: find the transition maps for Diagram.

 \begin{defn}
Suppose $ G \subseteq \text{Homeo}(F) $ is a topological group. If diagram has a collection of transition functions
\begin{align*}
	\tau_{ \alpha \beta}: U_{ \alpha} \cap U_{ \beta} \to G,
\end{align*}
then we say $ E$ has  \allbold{structure group} $ G$. 

If the transition functions do not map into $ G$ but can be homtoped through functions satisfying  $ (*)$ to one with image in  $ G$, then we say the structure group of  $ E$  \allbold{reduces} to $ G$. 
\end{defn}

\begin{remark}
If $ G$ preserves some structure on $ F$, then the fibers of  $ p:E \to B$ will have this structure.
\end{remark}
\begin{eg}
\begin{enumerate}[label=(\arabic*)]
	\item If $ F = \rr^{n}$ and $ G= \text{GL}_{ n}( R) \subseteq  \text{Homeo}( \rr^{n}) $, then each fiber of a bundle with structure group $ G$ has a linear structure.
	\item If  $ F = \rr^{n}$ and $ G = \text{GL}_{ n}^{+}(\rr) $, then fibers of $ F$ are oriented vector spaces so  $ E$ is an  \allbold{oriented vector bundle}.
	\item If $ F = \rr^{n}$ and $ G = O(n)$. Then  $ E$ is a vector bundle with a metric (inner product). Note  $ O(n) \to \text{GL}_{ n}(\rr) $ (inclusion) is a homotopy equivalence. Hence all vector bundles admit metrics.
	\item If $ F = \rr^{2n} = \cc^{n}$, then $ G = \text{GL}_{ n}( \cc) \iff E$ has a complex structure. $ G = U(n) \iff E$ has a Hermitian structure.
	\item If $ F = \rr^{n}$ and $ G= \text{GL}_{k}( \rr) \times \text{GL}_{n-k}(\rr) \subseteq \text{GL}_{ n}( \rr) ,(A,B) \mapsto \begin{pmatrix} A&0\\0&B \end{pmatrix}  $, then $ E$ has a  $ G$-structure  $ \iff E \cong E_1 \oplus E_2$ where $ E_1$ is an $ \rr^{k}$-bundle, $ E_2$ is an $ \rr^{n-k}$-bundle. Similarly of $ G = \text{GL}_{n-k}(\rr) \subseteq \text{GL}_{ n}(\rr) $, then $ E$ has a  $ G$-structure  $ \iff E \cong E^{1} \oplus \rr^{k}$ where $ E^{1}$ is an $ \rr^{n-k}$-bundle.
\end{enumerate}
\end{eg}

Question: when can we reduce the structure group?

\begin{defn}
If $ G$ is a Lie group (topological group), then a bundle  diagram is a  \allbold{principal $ G$-bundle} if there exists a smooth (or continuous) right $ G$-action  $ P \times G \to P$  s.t.\ 
\begin{enumerate}[label=(\arabic*)]
	\item action preserves fibers, \emph{i.e.} $ y \in p ^{-1}(x) \implies y . g \in p ^{-1}(x) \ \forall \ x,y,g$.
	\item $ G$ acts freely and transitively on  $ p ^{-1}(x) \ \forall \ x$.
\end{enumerate}
\end{defn}
\begin{remark}
This can also be defined as a smooth manifold $ P$ with a smooth right  $ G$-action that is free and proper, \emph{i.e.} for the map $ P \times G \to P \times P, (p,g) \mapsto (p.g,p)$ preimages of compact sets are compact.
\end{remark}

\begin{eg}
\begin{enumerate}[label=(\arabic*)]
	\item If $ (E,B,F,p)$ is a bundle with structure group  $ G$, then there is a cover of  $ B$ by local trivialization  $ \{(U_{ \alpha}, \phi_{ \alpha}\} $ with transition functions $ \tau_{ \alpha \beta} : (U_{ \alpha} \cap U_{ \beta}) \to G$ we can construct a principal $ G$-bundle as follows
		 \begin{align*}
			P_E = \bigsqcup_{ \alpha} U_{ \alpha} \times G / \sim
		\end{align*}
		when $ (x,g) \in U_{ \alpha} \sim (x',g') \in U_{ \beta} \times G \iff x=x'$ and $ \tau_{ \beta \alpha}(x) g= g'$. Exercise: show this is a principal $ G$-bundle.

		If $ E$ is a vector bundle then  $ P_E$ is a principal  $ \text{GL}_{ n}( \rr) $-bundle. It is called the \allbold{frame bundle} because you can think of points in the fibers of $ P_E$ as frames for the fibers of  $ E$. Exercise: think through this. We denote this $ \mathcal{ F}(E)$. Note $O(n) \simeq \text{GL}_{ n}(\rr) $ so we could take $ \mathcal{ F}(E)$ to be a principal $ O(n)$-bundle.
	\item diagram is a principal  $ S^{1}$-bundle.
	\item Regular covering spaces are of a manifold are principal bundles. Exercise: check this and what are fibers? Can an irregular cover be a principal bundle?
\end{enumerate}
\end{eg}

Exercise:
\begin{enumerate}[label=(\arabic*)]
	\item Show a prinicpal $ G$-bundle is trivial iff it admits a section.
	\item If  $ E$ is a vector bundle, the sections of  $ E$ are the same as  $ \text{GL}_{ n}(\rr) $-equivariant maps $ v: \mathcal{ F}(E) \to \rr^{n}$, \emph{i.e.} $ v(y.g) = g^{-1} v(y)$. Hint: given $ s: B \to E$ then for each $ y \in \mathcal{ F}(E)$, let $ v(y) = s(p(y))$ expressed in the frame  $ y$. Then $ p: \mathcal{ F}(E) \to B$. This allows us to turn sections into functions which are easier to work with.
\end{enumerate}

Given $ P \to B$ a principal $ G$-bundle, and  $ \rho: G \to G'$ is a homomorphism, where $ G' \subseteq \text{Homeo}( F) $. Then we can construct an $ F$-bundle with structure group  $ G'$ as follows
 \begin{align*}
	P \times _{\rho} F = P \times F / (p.g, f) \sim (p,\rho(g)f)
\end{align*}
\end{document}
