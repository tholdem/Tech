\documentclass[12pt,class=article,crop=false]{standalone} 
%Fall 2022
% Some basic packages
\usepackage{standalone}[subpreambles=true]
\usepackage[utf8]{inputenc}
\usepackage[T1]{fontenc}
\usepackage{textcomp}
\usepackage[english]{babel}
\usepackage{url}
\usepackage{graphicx}
%\usepackage{quiver}
\usepackage{float}
\usepackage{enumitem}
\usepackage{lmodern}
\usepackage{comment}
\usepackage{hyperref}
\usepackage[usenames,svgnames,dvipsnames]{xcolor}
\usepackage[margin=1in]{geometry}
\usepackage{pdfpages}

\pdfminorversion=7

% Don't indent paragraphs, leave some space between them
\usepackage{parskip}

% Hide page number when page is empty
\usepackage{emptypage}
\usepackage{subcaption}
\usepackage{multicol}
\usepackage[b]{esvect}

% Math stuff
\usepackage{amsmath, amsfonts, mathtools, amsthm, amssymb}
\usepackage{bbm}
\usepackage{stmaryrd}
\allowdisplaybreaks

% Fancy script capitals
\usepackage{mathrsfs}
\usepackage{cancel}
% Bold math
\usepackage{bm}
% Some shortcuts
\newcommand{\rr}{\ensuremath{\mathbb{R}}}
\newcommand{\zz}{\ensuremath{\mathbb{Z}}}
\newcommand{\qq}{\ensuremath{\mathbb{Q}}}
\newcommand{\nn}{\ensuremath{\mathbb{N}}}
\newcommand{\ff}{\ensuremath{\mathbb{F}}}
\newcommand{\cc}{\ensuremath{\mathbb{C}}}
\newcommand{\ee}{\ensuremath{\mathbb{E}}}
\newcommand{\hh}{\ensuremath{\mathbb{H}}}
\renewcommand\O{\ensuremath{\emptyset}}
\newcommand{\norm}[1]{{\left\lVert{#1}\right\rVert}}
\newcommand{\dbracket}[1]{{\left\llbracket{#1}\right\rrbracket}}
\newcommand{\ve}[1]{{\bm{#1}}}
\newcommand\allbold[1]{{\boldmath\textbf{#1}}}
\DeclareMathOperator{\lcm}{lcm}
\DeclareMathOperator{\im}{im}
\DeclareMathOperator{\coim}{coim}
\DeclareMathOperator{\dom}{dom}
\DeclareMathOperator{\tr}{tr}
\DeclareMathOperator{\rank}{rank}
\DeclareMathOperator*{\var}{Var}
\DeclareMathOperator*{\ev}{E}
\DeclareMathOperator{\dg}{deg}
\DeclareMathOperator{\aff}{aff}
\DeclareMathOperator{\conv}{conv}
\DeclareMathOperator{\inte}{int}
\DeclareMathOperator*{\argmin}{argmin}
\DeclareMathOperator*{\argmax}{argmax}
\DeclareMathOperator{\graph}{graph}
\DeclareMathOperator{\sgn}{sgn}
\DeclareMathOperator*{\Rep}{Rep}
\DeclareMathOperator{\Proj}{Proj}
\DeclareMathOperator{\mat}{mat}
\DeclareMathOperator{\diag}{diag}
\DeclareMathOperator{\aut}{Aut}
\DeclareMathOperator{\gal}{Gal}
\DeclareMathOperator{\inn}{Inn}
\DeclareMathOperator{\edm}{End}
\DeclareMathOperator{\Hom}{Hom}
\DeclareMathOperator{\ext}{Ext}
\DeclareMathOperator{\tor}{Tor}
\DeclareMathOperator{\Span}{Span}
\DeclareMathOperator{\Stab}{Stab}
\DeclareMathOperator{\cont}{cont}
\DeclareMathOperator{\Ann}{Ann}
\DeclareMathOperator{\Div}{div}
\DeclareMathOperator{\curl}{curl}
\DeclareMathOperator{\nat}{Nat}
\DeclareMathOperator{\gr}{Gr}
\DeclareMathOperator{\vect}{Vect}
\DeclareMathOperator{\id}{id}
\DeclareMathOperator{\Mod}{Mod}
\DeclareMathOperator{\sign}{sign}
\DeclareMathOperator{\Surf}{Surf}
\DeclareMathOperator{\fcone}{fcone}
\DeclareMathOperator{\Rot}{Rot}
\DeclareMathOperator{\grad}{grad}
\DeclareMathOperator{\atan2}{atan2}
\DeclareMathOperator{\Ric}{Ric}
\let\vec\relax
\DeclareMathOperator{\vec}{vec}
\let\Re\relax
\DeclareMathOperator{\Re}{Re}
\let\Im\relax
\DeclareMathOperator{\Im}{Im}
% Put x \to \infty below \lim
\let\svlim\lim\def\lim{\svlim\limits}

%wide hat
\usepackage{scalerel,stackengine}
\stackMath
\newcommand*\wh[1]{%
\savestack{\tmpbox}{\stretchto{%
  \scaleto{%
    \scalerel*[\widthof{\ensuremath{#1}}]{\kern-.6pt\bigwedge\kern-.6pt}%
    {\rule[-\textheight/2]{1ex}{\textheight}}%WIDTH-LIMITED BIG WEDGE
  }{\textheight}% 
}{0.5ex}}%
\stackon[1pt]{#1}{\tmpbox}%
}
\parskip 1ex

%Make implies and impliedby shorter
\let\implies\Rightarrow
\let\impliedby\Leftarrow
\let\iff\Leftrightarrow
\let\epsilon\varepsilon

% Add \contra symbol to denote contradiction
\usepackage{stmaryrd} % for \lightning
\newcommand\contra{\scalebox{1.5}{$\lightning$}}

% \let\phi\varphi

% Command for short corrections
% Usage: 1+1=\correct{3}{2}

\definecolor{correct}{HTML}{009900}
\newcommand\correct[2]{\ensuremath{\:}{\color{red}{#1}}\ensuremath{\to }{\color{correct}{#2}}\ensuremath{\:}}
\newcommand\green[1]{{\color{correct}{#1}}}

% horizontal rule
\newcommand\hr{
    \noindent\rule[0.5ex]{\linewidth}{0.5pt}
}

% hide parts
\newcommand\hide[1]{}

% si unitx
\usepackage{siunitx}
\sisetup{locale = FR}

%allows pmatrix to stretch
\makeatletter
\renewcommand*\env@matrix[1][\arraystretch]{%
  \edef\arraystretch{#1}%
  \hskip -\arraycolsep
  \let\@ifnextchar\new@ifnextchar
  \array{*\c@MaxMatrixCols c}}
\makeatother

\renewcommand{\arraystretch}{0.8}

\renewcommand{\baselinestretch}{1.5}

\usepackage{graphics}
\usepackage{epstopdf}

\RequirePackage{hyperref}
%%
%% Add support for color in order to color the hyperlinks.
%% 
\hypersetup{
  colorlinks = true,
  urlcolor = blue,
  citecolor = blue
}
%%fakesection Links
\hypersetup{
    colorlinks,
    linkcolor={red!50!black},
    citecolor={green!50!black},
    urlcolor={blue!80!black}
}
%customization of cleveref
\RequirePackage[capitalize,nameinlink]{cleveref}[0.19]

% Per SIAM Style Manual, "section" should be lowercase
\crefname{section}{section}{sections}
\crefname{subsection}{subsection}{subsections}
\Crefname{section}{Section}{Sections}
\Crefname{subsection}{Subsection}{Subsections}

% Per SIAM Style Manual, "Figure" should be spelled out in references
\Crefname{figure}{Figure}{Figures}

% Per SIAM Style Manual, don't say equation in front on an equation.
\crefformat{equation}{\textup{#2(#1)#3}}
\crefrangeformat{equation}{\textup{#3(#1)#4--#5(#2)#6}}
\crefmultiformat{equation}{\textup{#2(#1)#3}}{ and \textup{#2(#1)#3}}
{, \textup{#2(#1)#3}}{, and \textup{#2(#1)#3}}
\crefrangemultiformat{equation}{\textup{#3(#1)#4--#5(#2)#6}}%
{ and \textup{#3(#1)#4--#5(#2)#6}}{, \textup{#3(#1)#4--#5(#2)#6}}{, and \textup{#3(#1)#4--#5(#2)#6}}

% But spell it out at the beginning of a sentence.
\Crefformat{equation}{#2Equation~\textup{(#1)}#3}
\Crefrangeformat{equation}{Equations~\textup{#3(#1)#4--#5(#2)#6}}
\Crefmultiformat{equation}{Equations~\textup{#2(#1)#3}}{ and \textup{#2(#1)#3}}
{, \textup{#2(#1)#3}}{, and \textup{#2(#1)#3}}
\Crefrangemultiformat{equation}{Equations~\textup{#3(#1)#4--#5(#2)#6}}%
{ and \textup{#3(#1)#4--#5(#2)#6}}{, \textup{#3(#1)#4--#5(#2)#6}}{, and \textup{#3(#1)#4--#5(#2)#6}}

% Make number non-italic in any environment.
\crefdefaultlabelformat{#2\textup{#1}#3}

% Environments
\makeatother
% For box around Definition, Theorem, \ldots
%%fakesection Theorems
\usepackage{thmtools}
\usepackage[framemethod=TikZ]{mdframed}

\theoremstyle{definition}
\mdfdefinestyle{mdbluebox}{%
	roundcorner = 10pt,
	linewidth=1pt,
	skipabove=12pt,
	innerbottommargin=9pt,
	skipbelow=2pt,
	nobreak=true,
	linecolor=blue,
	backgroundcolor=TealBlue!5,
}
\declaretheoremstyle[
	headfont=\sffamily\bfseries\color{MidnightBlue},
	mdframed={style=mdbluebox},
	headpunct={\\[3pt]},
	postheadspace={0pt}
]{thmbluebox}

\mdfdefinestyle{mdredbox}{%
	linewidth=0.5pt,
	skipabove=12pt,
	frametitleaboveskip=5pt,
	frametitlebelowskip=0pt,
	skipbelow=2pt,
	frametitlefont=\bfseries,
	innertopmargin=4pt,
	innerbottommargin=8pt,
	nobreak=false,
	linecolor=RawSienna,
	backgroundcolor=Salmon!5,
}
\declaretheoremstyle[
	headfont=\bfseries\color{RawSienna},
	mdframed={style=mdredbox},
	headpunct={\\[3pt]},
	postheadspace={0pt},
]{thmredbox}

\declaretheorem[%
style=thmbluebox,name=Theorem,numberwithin=section]{thm}
\declaretheorem[style=thmbluebox,name=Lemma,sibling=thm]{lem}
\declaretheorem[style=thmbluebox,name=Proposition,sibling=thm]{prop}
\declaretheorem[style=thmbluebox,name=Corollary,sibling=thm]{coro}
\declaretheorem[style=thmredbox,name=Example,sibling=thm]{eg}

\mdfdefinestyle{mdgreenbox}{%
	roundcorner = 10pt,
	linewidth=1pt,
	skipabove=12pt,
	innerbottommargin=9pt,
	skipbelow=2pt,
	nobreak=true,
	linecolor=ForestGreen,
	backgroundcolor=ForestGreen!5,
}

\declaretheoremstyle[
	headfont=\bfseries\sffamily\color{ForestGreen!70!black},
	bodyfont=\normalfont,
	spaceabove=2pt,
	spacebelow=1pt,
	mdframed={style=mdgreenbox},
	headpunct={ --- },
]{thmgreenbox}

\declaretheorem[style=thmgreenbox,name=Definition,sibling=thm]{defn}

\mdfdefinestyle{mdgreenboxsq}{%
	linewidth=1pt,
	skipabove=12pt,
	innerbottommargin=9pt,
	skipbelow=2pt,
	nobreak=true,
	linecolor=ForestGreen,
	backgroundcolor=ForestGreen!5,
}
\declaretheoremstyle[
	headfont=\bfseries\sffamily\color{ForestGreen!70!black},
	bodyfont=\normalfont,
	spaceabove=2pt,
	spacebelow=1pt,
	mdframed={style=mdgreenboxsq},
	headpunct={},
]{thmgreenboxsq}
\declaretheoremstyle[
	headfont=\bfseries\sffamily\color{ForestGreen!70!black},
	bodyfont=\normalfont,
	spaceabove=2pt,
	spacebelow=1pt,
	mdframed={style=mdgreenboxsq},
	headpunct={},
]{thmgreenboxsq*}

\mdfdefinestyle{mdblackbox}{%
	skipabove=8pt,
	linewidth=3pt,
	rightline=false,
	leftline=true,
	topline=false,
	bottomline=false,
	linecolor=black,
	backgroundcolor=RedViolet!5!gray!5,
}
\declaretheoremstyle[
	headfont=\bfseries,
	bodyfont=\normalfont\small,
	spaceabove=0pt,
	spacebelow=0pt,
	mdframed={style=mdblackbox}
]{thmblackbox}

\theoremstyle{plain}
\declaretheorem[name=Question,sibling=thm,style=thmblackbox]{ques}
\declaretheorem[name=Remark,sibling=thm,style=thmgreenboxsq]{remark}
\declaretheorem[name=Remark,sibling=thm,style=thmgreenboxsq*]{remark*}
\newtheorem{ass}[thm]{Assumptions}

\theoremstyle{definition}
\newtheorem*{problem}{Problem}
\newtheorem{claim}[thm]{Claim}
\theoremstyle{remark}
\newtheorem*{case}{Case}
\newtheorem*{notation}{Notation}
\newtheorem*{note}{Note}
\newtheorem*{motivation}{Motivation}
\newtheorem*{intuition}{Intuition}
\newtheorem*{conjecture}{Conjecture}

% Make section starts with 1 for report type
%\renewcommand\thesection{\arabic{section}}

% End example and intermezzo environments with a small diamond (just like proof
% environments end with a small square)
\usepackage{etoolbox}
\AtEndEnvironment{vb}{\null\hfill$\diamond$}%
\AtEndEnvironment{intermezzo}{\null\hfill$\diamond$}%
% \AtEndEnvironment{opmerking}{\null\hfill$\diamond$}%

% Fix some spacing
% http://tex.stackexchange.com/questions/22119/how-can-i-change-the-spacing-before-theorems-with-amsthm
\makeatletter
\def\thm@space@setup{%
  \thm@preskip=\parskip \thm@postskip=0pt
}

% Fix some stuff
% %http://tex.stackexchange.com/questions/76273/multiple-pdfs-with-page-group-included-in-a-single-page-warning
\pdfsuppresswarningpagegroup=1


% My name
\author{Jaden Wang}



\begin{document}
\section{Homotopy groups}
Recall that $ \pi_n(X) = [S^{n}, X]_0$. Since $ S^{n}$ is an H'-space (by successive suspension), there is a multiplication on $ \pi_n(X)$. What is this product?

Define $ f,g:(S^{n},p) \to (X,x_0)$.
~\begin{figure}[H]
	\centering
	\includegraphics[width=0.8\textwidth]{./figures/product1.png}
\end{figure}

Sometimes it is useful to see $ \pi_n(X)$ as $ [(D^{n}, \partial D^{n}),(X,x_0)]$. If $ \kappa: D^{n} \to S^{n}$ collapses $ \partial D^{n}$ to $ p \in S^{n}$, then  $\pi_n(X) \to [(D^{n},\partial D^{n}),(X,x_0)], [f] \mapsto [f \circ \kappa ]$ is well-defined and injective. It is surjective since any $ f:(D^{n},\partial D^{n}) \to (X,x_0)$ factors through $ (S^{n},x_0)$ (universal property of quotients of pairs?).

Note that we can see $ \pi_n(X)$ is abelian for $ n \geq 2$ using the multiplication structure.

What is the multiplication structure in $ [(D^{n} , \partial D^{n}),(X,x_0)]$? We think $ D^{n}$ as $ I \times D^{n-1}$. Given  $ g,f: I \times D^{n-1} \to X$,
\begin{align*}
	f \cdot g(t,x) = \begin{cases}
		f(2t,x) & t \in [0, \frac{1}{2}]\\
		g(2t-1,x) & t \in (\frac{1}{2},1]\\
	\end{cases}
\end{align*}
Now we just move the puzzle pieces to swap them.

\begin{defn}
	We can also define \allbold{relative homotopy groups}.
Given space $ X$, subspace  $ A$ and  $ x_0 \in A$, define
\begin{align*}
	\pi_n(X,A) = [(D^{n}, \partial D^{n},s_0),(X,A,x_0)]
\end{align*}
with $ s_0 \in \partial D^{n}$.
\end{defn}

This multiplication does not make sense for $ \pi_1(X,A)$. So $ \pi_1(X,A)$ is just a set as it doesn't have to preserve base point to get a group structure.
~\begin{figure}[H]
	\centering
	\includegraphics[width=0.8\textwidth]{./figures/pi1_bad.png}
\end{figure}


This definition doesn't help us showing inverses and associativity. So we provide an alternative definition:
\begin{defn}
	Let $ D^{n} = I^{n}$ and $ J = \overline{\partial D^{n}-(D^{n-1} \times \{1\} )} = (D^{n-1} \times \{0\} ) \cup (\partial D^{n-1} \times I)$. That is, $ J$ is three edges of a square.
\end{defn}
	Exercise: show $ [(D^{n}, \partial D^{n}, s_0),(X,A,x_0)]$ is in 1-1 correspondence with $ [(D^{n}, \partial D^{n},J),(X,A,x_0)]$. Note $ (D^{n}, \partial D^{n}, J) / J \cong(D^{n}, \partial D^{n}, s_0)$.

Define a multiplication: $ f,g \in \pi_n(X,A)$.
\begin{align*}
	f \cdot g (x_1,\ldots,x_n) = \begin{cases}
		f(2x_1,x_2,\ldots,x_n) & x_1 \in [0, \frac{1}{2}]\\
		g(2x_1-1,x_2,\ldots,x_n) & x_2 \in (\frac{1}{2},1]\\
	\end{cases}
\end{align*}
	Let $ f ^{-1} (x_1,\ldots,x_n) = f(1-x_1,\ldots,x_n)$.

	Exercise:
	\begin{enumerate}[label=(\arabic*)]
		\item Show $ \pi_n(X,A)$ is a group with identity the constant map for $ n \geq 2$.
		\item Show  $ \pi_n(X,A)$ is abelian for $ n \geq 3$ (need the 3rd dimension to move around).
		\item  $ \pi_n(X,x_0) = \pi_n(X)$.
	\end{enumerate}

	\begin{lem}
	$ f: (D^{n}, \partial D^{n},s_0) \to (X,A,x_0)$ is $ 0$ in  $ \pi_n(X,A)$ iff it is homotopic rel $ \partial D^{n}$ and $ s_0$ to a map whose image is in $ A$.
	\end{lem}
	\begin{proof}
	$ (\impliedby):$ suppose $ f$ is homotopic to  $ g$ with  $ g$ having image in  $ A$. We know  $ D^{n}$ deformation retracts to $ s_0$:
	\begin{align*}
		H: D^{n} \times I \to D^{n}, H(x,0) = x, H(x,1) = s_0, H(s_0,t) = s_0 \ \forall \ t.
	\end{align*}
	Now $ g \circ H$ is a homotopy from $ g$ to constant map (rel  $ A$). Therefore  $ f = 0 \in \pi_n(X,A)$.
	\end{proof}

	$ (\implies):$ assume $ f=0$. So there exists a homotopy  $ H:D^{n} \times I \to X, H(x,0) = f, H(x,1)=x_0, H(x,t) \in A \ \forall \ x \in \partial D^{n}$. Note $ H|_{D^{n} \times \{1\} \cup \partial D^{n} \times I}$ is a map $ D^{n} \to A$ ($ H|_{D^{n} \times \{0\} } = f$. We can use $ H$ on  $ D^{n} \times I$ to give a homotopy $ f$ to a map with image in  $ A$. Here is the idea: there exists a homeomorphism  $ D^{n} \times I \xrightarrow{ \phi}  D^{n+1}$. There is also a continuous map $ D^{n} \times I \xrightarrow{ \psi} D^{n+1} $ that collapses $ ( \partial D^{n} \times I)$ to the equator. Now $ H \circ \phi ^{-1} \circ \psi: D^{n} \times I \to X$ is the homotopy.

\begin{note}
\begin{enumerate}[label=(\arabic*)]
	\item The inclusion maps $ (A,x_0) \subseteq (X,x_0) \subseteq (X,A)$ yield
		\begin{align*}
			\pi_n(A,x_0) \xrightarrow{ i_*} \pi_n(X,x_0) \xrightarrow{ j_*} \pi_n(X,A).  
		\end{align*}
	\item if $ f:(D^{n},\partial D^{n},J ) \to (X,A,x_0)$ then define $ \partial f: (\partial D^{n},J) \to (A,x_0)$. This induces a map $ \pi_n(X,A) \to \pi_{n-1}(X,A)$. Note $ \pi_{n-1}(A) = [(\partial D^{n},(A,x_0), Y]_0$. Exercise: show it's well-defined.
\end{enumerate}
\end{note}

\begin{thm}
Given $ (X,A,x_0)$ we have a long exact sequence
\begin{align*}
	\cdots \to \pi_n(A) \xrightarrow{ i_*} \pi_n(X) \xrightarrow{ j_*} \pi_n(X,A) \xrightarrow{ \partial } \pi_{n-1}(A) \to \cdots
\end{align*}
is equivariant under $ \pi_1(A)$ action.
\end{thm}
\begin{proof}
	First $ j_* i_* = 0$ by lemma 16. If  $ [f] \in \ker j_*$ then $ f:(D^{n}, \partial D^{n})\to (X,x_0)$ and homotopy $ H: D^{n} \times I \to X$ s.t.\ 
	\begin{enumerate}[label=(\arabic*)]
		\item $ H(x,0) = j \circ f(x)= f(x)$
		\item $ H(x,1) \in A$ by lemma 16.
		\item $ H(x,t) \in A \ \forall \ t$ and $ x \in \partial D^{n}$.
		\item $ H(s_0,t) = x_0 \ \forall \ t$.
	\end{enumerate}
	Note $ D' := (D^{n} \times \{1\} ) \cup  (\partial D^{n} \times I)$ is a disk and $ H|_{D'}: D' \to A$ s.t.\ $ H( \partial D') = x_0$. So $ g = H|_{D'}: D^{n} \to A$ this is in $ \pi_n(A)$ and as in the proof of lemma 16, $ f$ is homotopic to $ g$, so  $ i_*([g]) = [f]$ so  $ \im i_* = \ker j_*$ yielding exactness.

	Now we show $ \im j_* \subseteq \ker \partial $. Given $ [f] \in \pi_n(X,A)$, s.t.\ $ \partial f =0$ in $ \pi_{n-1}(A)$. There exists homotopy $ H: S^{n-1} \times I \to A$ s.t.\ $ H(x,0) =f(x)$, $ H(x,1) = x_0$, and $ H(s_0,t) = x_0$. Let $ D' = D^{n} \cup (S^{n-1} \times I)$ is a disk.

	$ f': D' \to X: x\mapsto \begin{cases}
		f(x) & x \in D^{n}\\
		H(x) & x \in S^{n-1} \times I
	\end{cases}$.
	Note $ [f'] \in \pi_n(X)$. Exercise $ f' \sim f $ in $ \pi_n(X,A)$ (because it can't see what's in $ A$ intuitively). Then $ j_*([f']) = [f]$ so  $ \ker \partial \subseteq \im j_*$. 

	Exercise: show $ \im \partial  = \ker i_*$.
\end{proof}

\begin{thm}
Let $ p: \widetilde{ X} \to X$ be a covering space, then $ p_*: \pi_n(\widetilde{ X}, \widetilde{ x_0}) \to \pi_n(X,p(\widetilde{ x_0}))$ is an isomorphism $ \ \forall \ n \geq 2$.
\end{thm}
\begin{proof}
Recall the lifting criterion: given a $ f:Y \to X$ s.t.\ $ f(y_0) = x_0$, $ f$ lifts to a map  $ \widetilde{ f}: Y \to \widetilde{ X}$ with $ \widetilde{ f} (y_0) = \widetilde{ x_0}$ iff $ f_*(\pi_1(Y,y_0)) \subseteq p_*(\pi_1(\widetilde{ X},\widetilde{ x_0}))$.

From this, we see that $ p_*$ is surjective for  $ n\geq 2$. Given  $ [f] \in \pi_n(X,x_0)$, then
\begin{align*}
	f_*(\pi_1(S^{n})) = \{e\} < p_*(pi_1(\widetilde{ X})) .
\end{align*}
So there exists a lift $ \widetilde{ f} : S^{n} \to \widetilde{ X}$ s.t.\ $ p_*([\widetilde{ f}]) =p \circ \widetilde{ f} = f$. 

It is also injective for $ n \geq 2$. Given $ [f] \in \pi_n(\widetilde{ X})$, suppose  $ p_*([f]) = 0$ in  $ \pi_n(X)$. So there exists a homotopy $ H : S^{n} \times I \to X$ s.t.\ $ H(x,0) = p \circ f(x)$, $ H(x,1) = x_0$, and $ H(s_0,t) = x_0$. Recall covering spacess satisfying the homotopy lifting property. So there exists $ \widetilde{ H}: S^{n} \times I \to \widetilde{ X}$ s.t.\ $ \widetilde{ H}(x,0) = f(x)$, $ \widetilde{ H}(x,1) = \widetilde{ x_0}$, and $ \widetilde{ H}(s_0,t) = \widetilde{ x_0}$. So $ [f] = 0$ in  $ \pi_n(\widetilde{ X})$.
\end{proof}
\end{document}
