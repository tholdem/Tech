\documentclass[12pt,class=article,crop=false]{standalone} 
\newcommand{\alert}[1]{{\bf \color{red} [Alert:] #1}}
\newcommand{\todo}[1]{{\bf \color{orange} [TODO:] #1}}
\newcommand{\real}[1][]{\mathbb{R}^{#1}}
\newcommand{\myeqn}[1]{(\ref{#1})}
\newcommand{\myex}[1]{Example \ref{#1}}
\newcommand{\defeq}{\stackrel{\mathrm{def}}{=}}
\newcommand{\parder}[2]{\frac{\partial #1}{\partial #2}}
\newcommand{\Lie}[3][]{\mathsf{L}_{#3}^{#1} #2}
\newcommand{\LieA}[1]{\mathsf{Lie}(#1)}
\newcommand{\lieder}[2]{\mathcal{L}_{#2} #1}
\renewcommand{\t}{^{\mbox{\tiny\sf T}}}
\newcommand{\trans}{^{\mbox{\tiny\sf T}}}
\newcommand{\markup}[1]{\{\textbf{#1}\}}
\newcommand{\msub}[1]{_\mathrm{#1}}
\newcommand{\msup}[1]{^\mathrm{#1}}
\newcommand{\inv}[1]{#1^{-1}}
\newcommand{\pinv}[1]{{#1}^{+}}
\newcommand{\myfracA}[2]{\displaystyle{\frac{#1}{#2}}}
\newcommand{\myfracB}[2]{{#1}/{#2}}
\newcommand{\mydiffA}[1]{\dot{#1}}
\newcommand{\mydiffB}[2]{\myfracA{\mathrm{d}{#1}}{\mathrm{d}{#2}}}
\newcommand{\ball}[2]{\mathcal{B}_{#1}\left(#2\right)}
\newcommand{\acos}[1]{\cos^{-1}\left(#1\right)}
\newcommand{\asin}[1]{\sin^{-1}\left(#1\right)}
\newcommand{\mani}{\mathcal{M}}
\newcommand{\tang}[2]{\mathsf{T}_{#1} #2}
\newcommand{\LieB}[2]{[ #1, #2 ]}
\newcommand{\LieBad}[3][]{\mathsf{ad}_{#2}^{#1} #3}
\newcommand{\ReachVT}{\mathcal{R}^V_T}
\newcommand{\ReachVt}{\mathcal{R}^V_t}
\newcommand{\ReachVTe}{\mathcal{R}^V_{\le T}}
\newcommand{\ReachT}{\mathcal{R}_T}
\newcommand{\Reacht}{\mathcal{R}_t}
\newcommand{\ReachTe}{\mathcal{R}_{\le T}}
\newcommand{\accLA}[1]{\mathsf{Lie}(#1)}
\newcommand{\accD}{\Delta_{\mathcal{F}}}
\newcommand{\accSA}{\mathsf{Lie}(\mathcal{G},f)}
\newcommand{\accDS}{\Delta_{\mathcal{G}}}
\newcommand{\eval}[3]{\mathsf{Ev}^{#2}_{#1}\left( #3 \right)}
\newcommand{\stlc}{\textsc{stlc}}
\newcommand{\clf}{\textsc{clf}}
\newcommand{\jqlf}{\textsc{jqlf}}
\newcommand{\dlas}{\textsc{dlas}}
\newcommand{\Ad}[2]{\mathsf{Ad}_{#1} #2}
\newcommand{\xe}{\ensuremath{x_e}}
\newcommand{\lebg}[1]{\mathcal{L}_{#1}}
\newcommand{\lebgx}[1]{\mathcal{L}_{#1 \mathrm{e}}}
\newcommand{\dom}{D}
\newcommand{\domT}{[t_0,\infty) \times D}
\newcommand{\rarrow}{\rightarrow}
\renewcommand{\d}{\mathrm{d}}
\renewcommand{\Re}{\mathbb{R}}
\newcommand{\C}{\mathrm{C}}

\newcommand{\QED}{{\unskip\nobreak\hfil\penalty50\hskip2em\vadjust{}
		\nobreak\hfil$\Box$\parfillskip=0pt\finalhyphendemerits=0\par}\vspace{0.1cm}}
\newcommand{\eoEx}{{\unskip\nobreak\hfil\penalty50\hskip0em\vadjust{}
		\nobreak\hfil$\Large\Diamond$\parfillskip=0pt\finalhyphendemerits=0\par}\vspace{0.1cm}}

\newcommand{\sgn}{\ensuremath{\operatorname{sgn}}}
\newcommand{\sat}{\ensuremath{\operatorname{sat}}}

\newcommand{\half}{\frac{1}{2}}
\newcommand{\shalf}{\mbox{$\frac{1}{2}$}}
\newcommand{\marcom}[1]{\marginpar{\footnotesize #1}}
\newcommand{\der}{\mathrm{D}}
\newcommand{\e}{\mathrm{e}}
\newcommand{\dt}{\mathrm{d}t}

\newcommand{\cA}{\ensuremath{\mathcal{A}}}
\newcommand{\cB}{\ensuremath{\mathcal{B}}}
\newcommand{\cG}{\ensuremath{\mathcal{G}}}
\newcommand{\cK}{\ensuremath{\mathcal{K}}}
\newcommand{\cW}{\ensuremath{\mathcal{W}}}
\newcommand{\cZ}{\ensuremath{\mathcal{Z}}}
\newcommand{\cS}{\ensuremath{\mathcal{S}}}
\newcommand{\cD}{\ensuremath{\mathcal{D}}}
\newcommand{\cP}{\ensuremath{\mathcal{P}}}
\newcommand{\cV}{\ensuremath{\mathcal{V}}}
\newcommand{\cL}{\ensuremath{\mathcal{L}}}
\newcommand{\cN}{\ensuremath{\mathcal{N}}}
\newcommand{\cI}{\ensuremath{\mathcal{I}}}
\newcommand{\cR}{\ensuremath{\mathcal{R}}}
\newcommand{\cM}{\ensuremath{\mathcal{M}}}
\newcommand{\cC}{\ensuremath{\mathcal{C}}}
\newcommand{\cF}{\ensuremath{\mathcal{F}}}
\newcommand{\cH}{\ensuremath{\mathcal{H}}}
\newcommand{\cO}{\ensuremath{\mathcal{O}}}
\newcommand{\cX}{\ensuremath{\mathcal{X}}}
\newcommand{\cY}{\ensuremath{\mathcal{Y}}}
\newcommand{\Ci}{\ensuremath{\mathcal{C}^\infty}}
\newcommand{\ISS}{\textsc{iss}}
\newcommand{\LISS}{\textsc{liss}}
\newcommand{\GAS}{\textsc{gas}}
\newcommand{\GS}{\textsc{gs}}
\newcommand{\LES}{\textsc{les}}
\newcommand{\GUAS}{\textsc{guas}}
\newcommand{\BIBO}{\textsc{bibo}}
\newcommand{\spec}{\ensuremath{\operatorname{spec}}}
\newcommand{\spn}{\ensuremath{\operatorname{span}}}
\renewcommand{\i}{\mathrm{i\,}}

\renewcommand{\implies}{\Rightarrow}

\renewcommand{\theenumi}{$\roman{enumi})$}
\renewcommand{\labelenumi}{\theenumi}

\font\ptmten=zptmcmrm scaled 1200
\newcommand{\w}{\mbox{{\ptmten w}}}
\newcommand{\z}{\mbox{{\ptmten z}}}
\renewcommand{\Re}{\mathbb{R}}

\newcommand{\cl}{\operatorname{cl}}
\newcommand{\intr}{\operatorname{int}}
\newcommand{\rank}{\operatorname{rank}}
\newcommand{\co}{\operatorname{co}}
\newcommand{\aff}{\operatorname{aff}}

\theoremstyle{plain}
\newtheorem{theorem}{Theorem}[chapter]
\newtheorem{claim}[theorem]{Claim}
\newtheorem{corollary}[theorem]{Corollary}
\newtheorem{prop}[theorem]{Proposition}
\newtheorem{fact}[theorem]{Fact}
\newtheorem{lemma}[theorem]{Lemma}

\newtheorem{remark}{Remark}[chapter]

\theoremstyle{definition}
\newtheorem{assume}[theorem]{Assumption}
\newtheorem{defn}[theorem]{Definition}
\newtheorem{problem}[theorem]{Problem}
\newtheorem{exercise}{Exercise}
\newtheorem{example}[theorem]{Example}


\begin{document}
\section{Homotopy groups}
Recall that $ \pi_n(X) = [S^{n}, X]_0$. Since $ S^{n}$ is an H'-space (by successive suspension), there is a multiplication on $ \pi_n(X)$. What is this product?

Define $ f,g:(S^{n},p) \to (X,x_0)$.
~\begin{figure}[H]
	\centering
	\includegraphics[width=0.8\textwidth]{./figures/product1.png}
\end{figure}

Sometimes it is useful to see $ \pi_n(X)$ as $ [(D^{n}, \partial D^{n}),(X,x_0)]$. If $ \kappa: D^{n} \to S^{n}$ collapses $ \partial D^{n}$ to $ p \in S^{n}$, then  $\pi_n(X) \to [(D^{n},\partial D^{n}),(X,x_0)], [f] \mapsto [f \circ \kappa ]$ is well-defined and injective. It is surjective since any $ f:(D^{n},\partial D^{n}) \to (X,x_0)$ factors through $ (S^{n},x_0)$ (universal property of quotients of pairs?).

Note that we can see $ \pi_n(X)$ is abelian for $ n \geq 2$ using the multiplication structure.

What is the multiplication structure in $ [(D^{n} , \partial D^{n}),(X,x_0)]$? We think $ D^{n}$ as $ I \times D^{n-1}$. Given  $ g,f: I \times D^{n-1} \to X$,
\begin{align*}
	f \cdot g(t,x) = \begin{cases}
		f(2t,x) & t \in [0, \frac{1}{2}]\\
		g(2t-1,x) & t \in (\frac{1}{2},1]\\
	\end{cases}
\end{align*}
Now we just move the puzzle pieces to swap them.

\begin{defn}
	We can also define \allbold{relative homotopy groups}.
Given space $ X$, subspace  $ A$ and  $ x_0 \in A$, define
\begin{align*}
	\pi_n(X,A) = [(D^{n}, \partial D^{n},s_0),(X,A,x_0)]
\end{align*}
with $ s_0 \in \partial D^{n}$.
\end{defn}

This multiplication does not make sense for $ \pi_1(X,A)$. So $ \pi_1(X,A)$ is just a set as it doesn't have to preserve base point to get a group structure.
~\begin{figure}[H]
	\centering
	\includegraphics[width=0.8\textwidth]{./figures/pi1_bad.png}
\end{figure}


This definition doesn't help us showing inverses and associativity. So we provide an alternative definition:
\begin{defn}
	Let $ D^{n} = I^{n}$ and $ J = \overline{\partial D^{n}-(D^{n-1} \times \{1\} )} = (D^{n-1} \times \{0\} ) \cup (\partial D^{n-1} \times I)$. That is, $ J$ is three edges of a square.
\end{defn}
	Exercise: show $ [(D^{n}, \partial D^{n}, s_0),(X,A,x_0)]$ is in 1-1 correspondence with $ [(D^{n}, \partial D^{n},J),(X,A,x_0)]$. Note $ (D^{n}, \partial D^{n}, J) / J \cong(D^{n}, \partial D^{n}, s_0)$.

Define a multiplication: $ f,g \in \pi_n(X,A)$.
\begin{align*}
	f \cdot g (x_1,\ldots,x_n) = \begin{cases}
		f(2x_1,x_2,\ldots,x_n) & x_1 \in [0, \frac{1}{2}]\\
		g(2x_1-1,x_2,\ldots,x_n) & x_2 \in (\frac{1}{2},1]\\
	\end{cases}
\end{align*}
	Let $ f ^{-1} (x_1,\ldots,x_n) = f(1-x_1,\ldots,x_n)$.

	Exercise:
	\begin{enumerate}[label=(\arabic*)]
		\item Show $ \pi_n(X,A)$ is a group with identity the constant map for $ n \geq 2$.
		\item Show  $ \pi_n(X,A)$ is abelian for $ n \geq 3$ (need the 3rd dimension to move around).
		\item  $ \pi_n(X,x_0) = \pi_n(X)$.
	\end{enumerate}

	\begin{lem}
	$ f: (D^{n}, \partial D^{n},s_0) \to (X,A,x_0)$ is $ 0$ in  $ \pi_n(X,A)$ iff it is homotopic rel $ \partial D^{n}$ and $ s_0$ to a map whose image is in $ A$.
	\end{lem}
	\begin{proof}
	$ (\impliedby):$ suppose $ f$ is homotopic to  $ g$ with  $ g$ having image in  $ A$. We know  $ D^{n}$ deformation retracts to $ s_0$:
	\begin{align*}
		H: D^{n} \times I \to D^{n}, H(x,0) = x, H(x,1) = s_0, H(s_0,t) = s_0 \ \forall \ t.
	\end{align*}
	Now $ g \circ H$ is a homotopy from $ g$ to constant map (rel  $ A$). Therefore  $ f = 0 \in \pi_n(X,A)$.
	\end{proof}

	$ (\implies):$ assume $ f=0$. So there exists a homotopy  $ H:D^{n} \times I \to X, H(x,0) = f, H(x,1)=x_0, H(x,t) \in A \ \forall \ x \in \partial D^{n}$. Note $ H|_{D^{n} \times \{1\} \cup \partial D^{n} \times I}$ is a map $ D^{n} \to A$ ($ H|_{D^{n} \times \{0\} } = f$. We can use $ H$ on  $ D^{n} \times I$ to give a homotopy $ f$ to a map with image in  $ A$. Here is the idea: there exists a homeomorphism  $ D^{n} \times I \xrightarrow{ \phi}  D^{n+1}$. There is also a continuous map $ D^{n} \times I \xrightarrow{ \psi} D^{n+1} $ that collapses $ ( \partial D^{n} \times I)$ to the equator. Now $ H \circ \phi ^{-1} \circ \psi: D^{n} \times I \to X$ is the homotopy.

\begin{note}
\begin{enumerate}[label=(\arabic*)]
	\item The inclusion maps $ (A,x_0) \subseteq (X,x_0) \subseteq (X,A)$ yield
		\begin{align*}
			\pi_n(A,x_0) \xrightarrow{ i_*} \pi_n(X,x_0) \xrightarrow{ j_*} \pi_n(X,A).  
		\end{align*}
	\item if $ f:(D^{n},\partial D^{n},J ) \to (X,A,x_0)$ then define $ \partial f: (\partial D^{n},J) \to (A,x_0)$. This induces a map $ \pi_n(X,A) \to \pi_{n-1}(X,A)$. Note $ \pi_{n-1}(A) = [(\partial D^{n},(A,x_0), Y]_0$. Exercise: show it's well-defined.
\end{enumerate}
\end{note}

\begin{thm}
Given $ (X,A,x_0)$ we have a long exact sequence
\begin{align*}
	\cdots \to \pi_n(A) \xrightarrow{ i_*} \pi_n(X) \xrightarrow{ j_*} \pi_n(X,A) \xrightarrow{ \partial } \pi_{n-1}(A) \to \cdots
\end{align*}
is equivariant under $ \pi_1(A)$ action.
\end{thm}
\begin{proof}
	First $ j_* i_* = 0$ by lemma 16. If  $ [f] \in \ker j_*$ then $ f:(D^{n}, \partial D^{n})\to (X,x_0)$ and homotopy $ H: D^{n} \times I \to X$ s.t.\ 
	\begin{enumerate}[label=(\arabic*)]
		\item $ H(x,0) = j \circ f(x)= f(x)$
		\item $ H(x,1) \in A$ by lemma 16.
		\item $ H(x,t) \in A \ \forall \ t$ and $ x \in \partial D^{n}$.
		\item $ H(s_0,t) = x_0 \ \forall \ t$.
	\end{enumerate}
	Note $ D' := (D^{n} \times \{1\} ) \cup  (\partial D^{n} \times I)$ is a disk and $ H|_{D'}: D' \to A$ s.t.\ $ H( \partial D') = x_0$. So $ g = H|_{D'}: D^{n} \to A$ this is in $ \pi_n(A)$ and as in the proof of lemma 16, $ f$ is homotopic to $ g$, so  $ i_*([g]) = [f]$ so  $ \im i_* = \ker j_*$ yielding exactness.

	Now we show $ \im j_* \subseteq \ker \partial $. Given $ [f] \in \pi_n(X,A)$, s.t.\ $ \partial f =0$ in $ \pi_{n-1}(A)$. There exists homotopy $ H: S^{n-1} \times I \to A$ s.t.\ $ H(x,0) =f(x)$, $ H(x,1) = x_0$, and $ H(s_0,t) = x_0$. Let $ D' = D^{n} \cup (S^{n-1} \times I)$ is a disk.

	$ f': D' \to X: x\mapsto \begin{cases}
		f(x) & x \in D^{n}\\
		H(x) & x \in S^{n-1} \times I
	\end{cases}$.
	Note $ [f'] \in \pi_n(X)$. Exercise $ f' \sim f $ in $ \pi_n(X,A)$ (because it can't see what's in $ A$ intuitively). Then $ j_*([f']) = [f]$ so  $ \ker \partial \subseteq \im j_*$. 

	Exercise: show $ \im \partial  = \ker i_*$.
\end{proof}

\begin{thm}
Let $ p: \widetilde{ X} \to X$ be a covering space, then $ p_*: \pi_n(\widetilde{ X}, \widetilde{ x_0}) \to \pi_n(X,p(\widetilde{ x_0}))$ is an isomorphism $ \ \forall \ n \geq 2$.
\end{thm}
\begin{proof}
Recall the lifting criterion: given a $ f:Y \to X$ s.t.\ $ f(y_0) = x_0$, $ f$ lifts to a map  $ \widetilde{ f}: Y \to \widetilde{ X}$ with $ \widetilde{ f} (y_0) = \widetilde{ x_0}$ iff $ f_*(\pi_1(Y,y_0)) \subseteq p_*(\pi_1(\widetilde{ X},\widetilde{ x_0}))$.

From this, we see that $ p_*$ is surjective for  $ n\geq 2$. Given  $ [f] \in \pi_n(X,x_0)$, then
\begin{align*}
	f_*(\pi_1(S^{n})) = \{e\} < p_*(pi_1(\widetilde{ X})) .
\end{align*}
So there exists a lift $ \widetilde{ f} : S^{n} \to \widetilde{ X}$ s.t.\ $ p_*([\widetilde{ f}]) =p \circ \widetilde{ f} = f$. 

It is also injective for $ n \geq 2$. Given $ [f] \in \pi_n(\widetilde{ X})$, suppose  $ p_*([f]) = 0$ in  $ \pi_n(X)$. So there exists a homotopy $ H : S^{n} \times I \to X$ s.t.\ $ H(x,0) = p \circ f(x)$, $ H(x,1) = x_0$, and $ H(s_0,t) = x_0$. Recall covering spacess satisfying the homotopy lifting property. So there exists $ \widetilde{ H}: S^{n} \times I \to \widetilde{ X}$ s.t.\ $ \widetilde{ H}(x,0) = f(x)$, $ \widetilde{ H}(x,1) = \widetilde{ x_0}$, and $ \widetilde{ H}(s_0,t) = \widetilde{ x_0}$. So $ [f] = 0$ in  $ \pi_n(\widetilde{ X})$.
\end{proof}
\end{document}
