\documentclass[12pt,class=article,crop=false]{standalone} 
%Fall 2022
% Some basic packages
\usepackage{standalone}[subpreambles=true]
\usepackage[utf8]{inputenc}
\usepackage[T1]{fontenc}
\usepackage{textcomp}
\usepackage[english]{babel}
\usepackage{url}
\usepackage{graphicx}
%\usepackage{quiver}
\usepackage{float}
\usepackage{enumitem}
\usepackage{lmodern}
\usepackage{comment}
\usepackage{hyperref}
\usepackage[usenames,svgnames,dvipsnames]{xcolor}
\usepackage[margin=1in]{geometry}
\usepackage{pdfpages}

\pdfminorversion=7

% Don't indent paragraphs, leave some space between them
\usepackage{parskip}

% Hide page number when page is empty
\usepackage{emptypage}
\usepackage{subcaption}
\usepackage{multicol}
\usepackage[b]{esvect}

% Math stuff
\usepackage{amsmath, amsfonts, mathtools, amsthm, amssymb}
\usepackage{bbm}
\usepackage{stmaryrd}
\allowdisplaybreaks

% Fancy script capitals
\usepackage{mathrsfs}
\usepackage{cancel}
% Bold math
\usepackage{bm}
% Some shortcuts
\newcommand{\rr}{\ensuremath{\mathbb{R}}}
\newcommand{\zz}{\ensuremath{\mathbb{Z}}}
\newcommand{\qq}{\ensuremath{\mathbb{Q}}}
\newcommand{\nn}{\ensuremath{\mathbb{N}}}
\newcommand{\ff}{\ensuremath{\mathbb{F}}}
\newcommand{\cc}{\ensuremath{\mathbb{C}}}
\newcommand{\ee}{\ensuremath{\mathbb{E}}}
\newcommand{\hh}{\ensuremath{\mathbb{H}}}
\renewcommand\O{\ensuremath{\emptyset}}
\newcommand{\norm}[1]{{\left\lVert{#1}\right\rVert}}
\newcommand{\dbracket}[1]{{\left\llbracket{#1}\right\rrbracket}}
\newcommand{\ve}[1]{{\bm{#1}}}
\newcommand\allbold[1]{{\boldmath\textbf{#1}}}
\DeclareMathOperator{\lcm}{lcm}
\DeclareMathOperator{\im}{im}
\DeclareMathOperator{\coim}{coim}
\DeclareMathOperator{\dom}{dom}
\DeclareMathOperator{\tr}{tr}
\DeclareMathOperator{\rank}{rank}
\DeclareMathOperator*{\var}{Var}
\DeclareMathOperator*{\ev}{E}
\DeclareMathOperator{\dg}{deg}
\DeclareMathOperator{\aff}{aff}
\DeclareMathOperator{\conv}{conv}
\DeclareMathOperator{\inte}{int}
\DeclareMathOperator*{\argmin}{argmin}
\DeclareMathOperator*{\argmax}{argmax}
\DeclareMathOperator{\graph}{graph}
\DeclareMathOperator{\sgn}{sgn}
\DeclareMathOperator*{\Rep}{Rep}
\DeclareMathOperator{\Proj}{Proj}
\DeclareMathOperator{\mat}{mat}
\DeclareMathOperator{\diag}{diag}
\DeclareMathOperator{\aut}{Aut}
\DeclareMathOperator{\gal}{Gal}
\DeclareMathOperator{\inn}{Inn}
\DeclareMathOperator{\edm}{End}
\DeclareMathOperator{\Hom}{Hom}
\DeclareMathOperator{\ext}{Ext}
\DeclareMathOperator{\tor}{Tor}
\DeclareMathOperator{\Span}{Span}
\DeclareMathOperator{\Stab}{Stab}
\DeclareMathOperator{\cont}{cont}
\DeclareMathOperator{\Ann}{Ann}
\DeclareMathOperator{\Div}{div}
\DeclareMathOperator{\curl}{curl}
\DeclareMathOperator{\nat}{Nat}
\DeclareMathOperator{\gr}{Gr}
\DeclareMathOperator{\vect}{Vect}
\DeclareMathOperator{\id}{id}
\DeclareMathOperator{\Mod}{Mod}
\DeclareMathOperator{\sign}{sign}
\DeclareMathOperator{\Surf}{Surf}
\DeclareMathOperator{\fcone}{fcone}
\DeclareMathOperator{\Rot}{Rot}
\DeclareMathOperator{\grad}{grad}
\DeclareMathOperator{\atan2}{atan2}
\DeclareMathOperator{\Ric}{Ric}
\let\vec\relax
\DeclareMathOperator{\vec}{vec}
\let\Re\relax
\DeclareMathOperator{\Re}{Re}
\let\Im\relax
\DeclareMathOperator{\Im}{Im}
% Put x \to \infty below \lim
\let\svlim\lim\def\lim{\svlim\limits}

%wide hat
\usepackage{scalerel,stackengine}
\stackMath
\newcommand*\wh[1]{%
\savestack{\tmpbox}{\stretchto{%
  \scaleto{%
    \scalerel*[\widthof{\ensuremath{#1}}]{\kern-.6pt\bigwedge\kern-.6pt}%
    {\rule[-\textheight/2]{1ex}{\textheight}}%WIDTH-LIMITED BIG WEDGE
  }{\textheight}% 
}{0.5ex}}%
\stackon[1pt]{#1}{\tmpbox}%
}
\parskip 1ex

%Make implies and impliedby shorter
\let\implies\Rightarrow
\let\impliedby\Leftarrow
\let\iff\Leftrightarrow
\let\epsilon\varepsilon

% Add \contra symbol to denote contradiction
\usepackage{stmaryrd} % for \lightning
\newcommand\contra{\scalebox{1.5}{$\lightning$}}

% \let\phi\varphi

% Command for short corrections
% Usage: 1+1=\correct{3}{2}

\definecolor{correct}{HTML}{009900}
\newcommand\correct[2]{\ensuremath{\:}{\color{red}{#1}}\ensuremath{\to }{\color{correct}{#2}}\ensuremath{\:}}
\newcommand\green[1]{{\color{correct}{#1}}}

% horizontal rule
\newcommand\hr{
    \noindent\rule[0.5ex]{\linewidth}{0.5pt}
}

% hide parts
\newcommand\hide[1]{}

% si unitx
\usepackage{siunitx}
\sisetup{locale = FR}

%allows pmatrix to stretch
\makeatletter
\renewcommand*\env@matrix[1][\arraystretch]{%
  \edef\arraystretch{#1}%
  \hskip -\arraycolsep
  \let\@ifnextchar\new@ifnextchar
  \array{*\c@MaxMatrixCols c}}
\makeatother

\renewcommand{\arraystretch}{0.8}

\renewcommand{\baselinestretch}{1.5}

\usepackage{graphics}
\usepackage{epstopdf}

\RequirePackage{hyperref}
%%
%% Add support for color in order to color the hyperlinks.
%% 
\hypersetup{
  colorlinks = true,
  urlcolor = blue,
  citecolor = blue
}
%%fakesection Links
\hypersetup{
    colorlinks,
    linkcolor={red!50!black},
    citecolor={green!50!black},
    urlcolor={blue!80!black}
}
%customization of cleveref
\RequirePackage[capitalize,nameinlink]{cleveref}[0.19]

% Per SIAM Style Manual, "section" should be lowercase
\crefname{section}{section}{sections}
\crefname{subsection}{subsection}{subsections}
\Crefname{section}{Section}{Sections}
\Crefname{subsection}{Subsection}{Subsections}

% Per SIAM Style Manual, "Figure" should be spelled out in references
\Crefname{figure}{Figure}{Figures}

% Per SIAM Style Manual, don't say equation in front on an equation.
\crefformat{equation}{\textup{#2(#1)#3}}
\crefrangeformat{equation}{\textup{#3(#1)#4--#5(#2)#6}}
\crefmultiformat{equation}{\textup{#2(#1)#3}}{ and \textup{#2(#1)#3}}
{, \textup{#2(#1)#3}}{, and \textup{#2(#1)#3}}
\crefrangemultiformat{equation}{\textup{#3(#1)#4--#5(#2)#6}}%
{ and \textup{#3(#1)#4--#5(#2)#6}}{, \textup{#3(#1)#4--#5(#2)#6}}{, and \textup{#3(#1)#4--#5(#2)#6}}

% But spell it out at the beginning of a sentence.
\Crefformat{equation}{#2Equation~\textup{(#1)}#3}
\Crefrangeformat{equation}{Equations~\textup{#3(#1)#4--#5(#2)#6}}
\Crefmultiformat{equation}{Equations~\textup{#2(#1)#3}}{ and \textup{#2(#1)#3}}
{, \textup{#2(#1)#3}}{, and \textup{#2(#1)#3}}
\Crefrangemultiformat{equation}{Equations~\textup{#3(#1)#4--#5(#2)#6}}%
{ and \textup{#3(#1)#4--#5(#2)#6}}{, \textup{#3(#1)#4--#5(#2)#6}}{, and \textup{#3(#1)#4--#5(#2)#6}}

% Make number non-italic in any environment.
\crefdefaultlabelformat{#2\textup{#1}#3}

% Environments
\makeatother
% For box around Definition, Theorem, \ldots
%%fakesection Theorems
\usepackage{thmtools}
\usepackage[framemethod=TikZ]{mdframed}

\theoremstyle{definition}
\mdfdefinestyle{mdbluebox}{%
	roundcorner = 10pt,
	linewidth=1pt,
	skipabove=12pt,
	innerbottommargin=9pt,
	skipbelow=2pt,
	nobreak=true,
	linecolor=blue,
	backgroundcolor=TealBlue!5,
}
\declaretheoremstyle[
	headfont=\sffamily\bfseries\color{MidnightBlue},
	mdframed={style=mdbluebox},
	headpunct={\\[3pt]},
	postheadspace={0pt}
]{thmbluebox}

\mdfdefinestyle{mdredbox}{%
	linewidth=0.5pt,
	skipabove=12pt,
	frametitleaboveskip=5pt,
	frametitlebelowskip=0pt,
	skipbelow=2pt,
	frametitlefont=\bfseries,
	innertopmargin=4pt,
	innerbottommargin=8pt,
	nobreak=false,
	linecolor=RawSienna,
	backgroundcolor=Salmon!5,
}
\declaretheoremstyle[
	headfont=\bfseries\color{RawSienna},
	mdframed={style=mdredbox},
	headpunct={\\[3pt]},
	postheadspace={0pt},
]{thmredbox}

\declaretheorem[%
style=thmbluebox,name=Theorem,numberwithin=section]{thm}
\declaretheorem[style=thmbluebox,name=Lemma,sibling=thm]{lem}
\declaretheorem[style=thmbluebox,name=Proposition,sibling=thm]{prop}
\declaretheorem[style=thmbluebox,name=Corollary,sibling=thm]{coro}
\declaretheorem[style=thmredbox,name=Example,sibling=thm]{eg}

\mdfdefinestyle{mdgreenbox}{%
	roundcorner = 10pt,
	linewidth=1pt,
	skipabove=12pt,
	innerbottommargin=9pt,
	skipbelow=2pt,
	nobreak=true,
	linecolor=ForestGreen,
	backgroundcolor=ForestGreen!5,
}

\declaretheoremstyle[
	headfont=\bfseries\sffamily\color{ForestGreen!70!black},
	bodyfont=\normalfont,
	spaceabove=2pt,
	spacebelow=1pt,
	mdframed={style=mdgreenbox},
	headpunct={ --- },
]{thmgreenbox}

\declaretheorem[style=thmgreenbox,name=Definition,sibling=thm]{defn}

\mdfdefinestyle{mdgreenboxsq}{%
	linewidth=1pt,
	skipabove=12pt,
	innerbottommargin=9pt,
	skipbelow=2pt,
	nobreak=true,
	linecolor=ForestGreen,
	backgroundcolor=ForestGreen!5,
}
\declaretheoremstyle[
	headfont=\bfseries\sffamily\color{ForestGreen!70!black},
	bodyfont=\normalfont,
	spaceabove=2pt,
	spacebelow=1pt,
	mdframed={style=mdgreenboxsq},
	headpunct={},
]{thmgreenboxsq}
\declaretheoremstyle[
	headfont=\bfseries\sffamily\color{ForestGreen!70!black},
	bodyfont=\normalfont,
	spaceabove=2pt,
	spacebelow=1pt,
	mdframed={style=mdgreenboxsq},
	headpunct={},
]{thmgreenboxsq*}

\mdfdefinestyle{mdblackbox}{%
	skipabove=8pt,
	linewidth=3pt,
	rightline=false,
	leftline=true,
	topline=false,
	bottomline=false,
	linecolor=black,
	backgroundcolor=RedViolet!5!gray!5,
}
\declaretheoremstyle[
	headfont=\bfseries,
	bodyfont=\normalfont\small,
	spaceabove=0pt,
	spacebelow=0pt,
	mdframed={style=mdblackbox}
]{thmblackbox}

\theoremstyle{plain}
\declaretheorem[name=Question,sibling=thm,style=thmblackbox]{ques}
\declaretheorem[name=Remark,sibling=thm,style=thmgreenboxsq]{remark}
\declaretheorem[name=Remark,sibling=thm,style=thmgreenboxsq*]{remark*}
\newtheorem{ass}[thm]{Assumptions}

\theoremstyle{definition}
\newtheorem*{problem}{Problem}
\newtheorem{claim}[thm]{Claim}
\theoremstyle{remark}
\newtheorem*{case}{Case}
\newtheorem*{notation}{Notation}
\newtheorem*{note}{Note}
\newtheorem*{motivation}{Motivation}
\newtheorem*{intuition}{Intuition}
\newtheorem*{conjecture}{Conjecture}

% Make section starts with 1 for report type
%\renewcommand\thesection{\arabic{section}}

% End example and intermezzo environments with a small diamond (just like proof
% environments end with a small square)
\usepackage{etoolbox}
\AtEndEnvironment{vb}{\null\hfill$\diamond$}%
\AtEndEnvironment{intermezzo}{\null\hfill$\diamond$}%
% \AtEndEnvironment{opmerking}{\null\hfill$\diamond$}%

% Fix some spacing
% http://tex.stackexchange.com/questions/22119/how-can-i-change-the-spacing-before-theorems-with-amsthm
\makeatletter
\def\thm@space@setup{%
  \thm@preskip=\parskip \thm@postskip=0pt
}

% Fix some stuff
% %http://tex.stackexchange.com/questions/76273/multiple-pdfs-with-page-group-included-in-a-single-page-warning
\pdfsuppresswarningpagegroup=1


% My name
\author{Jaden Wang}



\begin{document}
\section{Serre fibration}
\begin{defn}
A continuous map $ p: E \to B$ is called a \allbold{fibration} (or a \allbold{Serre fibration}) if it has the homotopy lifting property (HLP). That is, given a function $ \widetilde{ g}: Y \to E$ and a homotopy $ G: Y \times I \to B$ of $ p \circ \widetilde{ g}$. Then there exists a homotopy $ \widetilde{ G}: Y \times I \to E$ s.t.\ $ p \circ \widetilde{ G} = G$. In other words, the following diagram commutes:
% https://q.uiver.app/?q=WzAsNCxbMSwwLCJFIl0sWzEsMSwiQiJdLFswLDEsIlkgXFx0aW1lcyBJIl0sWzAsMCwiWSBcXHRpbWVzIFxcezBcXH0iXSxbMCwxLCJwIl0sWzIsMCwiXFx0aWxkZXtHfSIsMix7InN0eWxlIjp7ImJvZHkiOnsibmFtZSI6ImRhc2hlZCJ9fX1dLFsyLDEsIkciLDJdLFszLDAsIlxcdGlsZGV7Z30iXSxbMywyLCIiLDAseyJzdHlsZSI6eyJ0YWlsIjp7Im5hbWUiOiJob29rIiwic2lkZSI6InRvcCJ9fX1dXQ==
\[\begin{tikzcd}
	{Y \times \{0\}} & E \\
	{Y \times I} & B
	\arrow["p", from=1-2, to=2-2]
	\arrow["{\tilde{G}}"', dashed, from=2-1, to=1-2]
	\arrow["G"', from=2-1, to=2-2]
	\arrow["{\tilde{g}}", from=1-1, to=1-2]
	\arrow[hook, from=1-1, to=2-1]
\end{tikzcd}\]  
\end{defn}
\begin{remark}
A locally trivial fibration is a fibration because $ Y \to Y, y\mapsto y$ is a bundle with fiber a point.
% https://q.uiver.app/?q=WzAsNCxbMCwwLCJFIl0sWzAsMSwiQiJdLFsxLDAsIlkiXSxbMSwxLCJZIl0sWzAsMiwiXFx0aWxkZXtofV8wIl0sWzIsMywiXFx0ZXh0e2lkfV9ZIl0sWzAsMSwicCIsMl0sWzEsMywiaF8wIl1d
\[\begin{tikzcd}
	E & Y \\
	B & Y
	\arrow["{\tilde{h}_0}", from=1-1, to=1-2]
	\arrow["{\text{id}_Y}", from=1-2, to=2-2]
	\arrow["p"', from=1-1, to=2-1]
	\arrow["{h_0}", from=2-1, to=2-2]
\end{tikzcd}\]
So we get a homotopy lifting by Theorem 2.
\end{remark}

\begin{thm}
If $ p: E \to B$ is a Serre fibration, and $ x_0,x_1 \in B$ are in the same path component, then $ p^{-1}(x_0) \simeq p ^{-1}(x_1)$.
\end{thm}
\begin{proof}
Let $ F_i= p ^{-1}(x_i)$ and $ \gamma$ a path from $ x_0 $ to $ x_1$. Diagram. So HLP gives a homotopy $ A^{ \gamma} : F_0 \times I \to E$ and $ A_1^{ \gamma} : F_0 \to F_1$.
\begin{claim}
	If $ \gamma_0 , \gamma_1$ are homotopic rel end points, then $ A^{ \gamma_0}$ and $ A^{ \gamma_1}$ are homotopic and hence $ A_1^{ \gamma_0} \simeq A_1^{ \gamma_1}$. 
\end{claim}

Let $ H: I \times I \to B$ be homotopy $ \gamma_0$ to $ \gamma_1$. Consider $ \Lambda: F_0 \times I \times I \to B, ( e,s,t)\mapsto H(s,t)$. 
~\begin{figure}[H]
	\centering
	\includegraphics[width=0.8\textwidth]{./figures/homotopy_claim.png}
\end{figure}

Define $ F_0 \times I \{0\} = A^{ \gamma_0} $, $ F_0 \times I \times \{1\} = A^{ \gamma_1} $, and $ F_0 \times \{0\} \times I = (e,0,s)\mapsto e $. Let $ C = (I \times \{0,1\} ) \cup (\{0\} \times I) \subseteq I \times I$, there exists a homeo taking $ C$ to  $ I \times \{0\} $.  
Diagram. Compose with $ \text{id}_{ F_0} \times f^{-1}$ to get $ \widetilde{ \Lambda}$. Diagrams. So $ \widetilde{ \Lambda}$ is a homotopy from $ A^{ \gamma_0}$ to $ A^{ \gamma_1}$. Thus $ \widetilde{ \Lambda}|_{F_0 \times \{1\} \times I}$ is a homotopy $ A_1^{ \gamma_0}$ to $ A_1 ^{ \gamma_1}$.

Now consider $A_1^{ \gamma}, A_1^{ \gamma^{-1}}: F_1 \to F_0$. Note that $ A_1^{ \gamma} \circ A_1^{ \gamma ^{-1}}: F_1 \to F_1$ is a lifting of homotopy $ \gamma * \overline{\gamma} \simeq$ constant path rel end points. Hence
	\begin{align*}
		A_1^{ \gamma} \circ A_1^{ \overline{ \gamma}} \simeq \text{id}_{ F_0}.
	\end{align*}
	The other direction follows similarly so we prove the theorem.
\end{proof}
\begin{remark}
This theorem says that although the lifted homotopies aren't unique, they are homotopic.
\end{remark}

\begin{eg}
Let $ (X,x_0)$ be a based topological space. Set $ P(X) = C((I, \{0\}) , (X,x_0))$ (all paths that starts with $ x_0$ ), often called the \allbold{path space}, and $ p: P(X) \to X, \gamma\mapsto \gamma(1)$. 
\end{eg}
\begin{lem}
In the case above, $p:P(X) \to X$ is a fibration and $ P(X)$ is contractible.
\end{lem}
\begin{proof}
We need to check HLP so the diagram commutes.
% https://q.uiver.app/?q=WzAsNCxbMSwwLCJQKFgpIl0sWzEsMSwiWCJdLFswLDEsIlkgXFx0aW1lcyBJIl0sWzAsMCwiWSBcXHRpbWVzIFxcezBcXH0iXSxbMCwxLCJwIl0sWzIsMCwiXFx0aWxkZXtGfSIsMix7InN0eWxlIjp7ImJvZHkiOnsibmFtZSI6ImRhc2hlZCJ9fX1dLFsyLDEsIkYiLDJdLFszLDAsImZfMCJdLFszLDIsIiIsMCx7InN0eWxlIjp7InRhaWwiOnsibmFtZSI6Imhvb2siLCJzaWRlIjoidG9wIn19fV1d
\[\begin{tikzcd}
	{Y \times \{0\}} & {P(X)} \\
	{Y \times I} & X
	\arrow["p", from=1-2, to=2-2]
	\arrow["{\tilde{F}}"', dashed, from=2-1, to=1-2]
	\arrow["F"', from=2-1, to=2-2]
	\arrow["{f_0}", from=1-1, to=1-2]
	\arrow[hook, from=1-1, to=2-1]
\end{tikzcd}\]

We need to define $ \widetilde{ F}: Y \times I \to P(X)$.

For $ (y,s) \in Y \times I$, 
\begin{align*}
\widetilde{ F}(y,s) : I \to X, t \mapsto \begin{cases}
	f_0(y) \left( \frac{2t}{2-s } \right) & t \in \left[ 0, \frac{2-s}{ 2} \right]\\
	F(y,2t-2+s) & t \in \left[ \frac{2-s}{ 2} ,1\right] 
\end{cases}
\end{align*}
\begin{enumerate}[label=(\arabic*)]
	\item Note this path is well-defined:
\begin{align*}
	f_0(y) \left( \frac{2(2-s) /2}{2-s } \right) &= f_0(y)(1) \\
	F(y,2(2-s) /2 -2+s) &= F(y,0)
\end{align*}
and since $ p \circ f_0 = F$ they are the same.
\item $ \widetilde{ F}(y,0) (t) = f_0(y)(t)$.
\item $ \widetilde{ F}(y,s)(0) = f_0(y)(0) = x_0$.
\item $ p \circ \widetilde{ F}(y,s) = \widetilde{ F}(y,s)(1) = F(y,s)$. 
\end{enumerate}
So $ \widetilde{ F}$ is a lift of $ F$. Now for contractibility, we have
\begin{align*}
	H:P(X) \times I \to P(X), (\gamma,s) \mapsto \gamma((1-s)t).
\end{align*}
This is a strong deformation retraction to one point.
\end{proof}
\begin{remark}
Since $ p ^{-1}(x_0)$ is all paths that also end with $ x_0$, $ p ^{-1}(x_0) = \Omega(X)$ the loop space. So by Theorem 3, $ p ^{-1}(x) \simeq \Omega(X) \ \forall \ x \in X$ if $ X$ is path-connected. Diagram.
\end{remark}
\begin{eg}
	Given $ f: X \to Y$, we saw earlier that $ f$ is homotopic to an inclusion. Recall if  $ C_f = X\times I \cup Y / (x,0) \sim f(x)$ the mapping cylinder, then $ Y \sim C_f$. And diagram. So up to homotopy we can assume $ X \subseteq Y$. Now let $ E = (C(I, \{0\} ),(Y,X))$ which are all paths in $ Y$ that starts in  $ X$. Let  $ B = C(\{0,1\}, \{0\} ,(Y,X) ) = X \times Y$.

	Exercise: show that $ E \to Y, \gamma \mapsto \gamma(1)$ is a fibration (almost the same as lemma 5). Note $ E \simeq X$ (same as $ P(X)$ contractible). So the diagram holds. Hence  $ f \simeq j \simeq p$ a fibration. Hence we have the slogan:
	
	Any map is a fibration upto homotopy.
\end{eg}

\begin{lem}
If $ (E,B,F,p)$ is a fibration, then  $ \pi_n(E,F) \cong \pi_n(B)$.
\end{lem}
\begin{proof}
Let $ b_0$ be a base point in $ B$ where  $ F=p ^{-1}(b_0)$ and $ e_0 \in F \subseteq E$. Given $ f:(D^{n}, \partial D^{n}) \to (E,F)$, we have $ p \circ f: (D^{n}, \partial D^{n}) \to (B,b_0)$. So $ p$ induces a map  $ p_*: \pi_n(E,F) \to \pi_n(B)$. Exercise: $ p_*$ is well-defined and a homomorphism.
 \begin{claim}
$ p_*$ is surjective.
\end{claim}
Given $ g \in \pi_n(B)$, think of $ D^{n} = D^{n-1} \times I$. Define
\begin{align*}
	\widetilde{ g}_0: (D^{n-1} \times \{0\} ) \to E, x\mapsto e_0
\end{align*}
So $ g$ is a homotopy of  $ p \circ \widetilde{ g}_0$ so HLP implies there exists $ \widetilde{ g}: D^{n-1} \times I \to E$ lifting $ g$. Since  $ p \circ \widetilde{ g}( \partial (D^{n-1} \times I)) = \{b_0\} $, so $ \widetilde{ g}(\partial (D^{n-1} \times I)) \subseteq F = p^{-1}(b_0)$. So $ \widetilde{ g} \in \pi_n(E,F)$. Clearly $ p \circ \widetilde{ g} = g$.
\begin{claim}
$ p_*$ is injective.
\end{claim}
Suppose $ p_*([f]) =[0] \in \pi_n(B)$, \emph{i.e.} $ p \circ f \simeq$ constant $ b_0$ map. Let $ H:(D^{n}, \partial D^{n}) \times I \to (B,b_0)$ be the homotopy where $ H(x,0)= p \circ f(x)$. So by HLP, there exists $ \widetilde{ H}: (D^{n}, \partial D^{n}) \times I to E$. As previous, $ \widetilde{ H}(\partial D^{n-1}\times I) \subseteq F$ and $ \widetilde{ H}(D^{n} \times \{1\} ) \subseteq F$. So $ \widetilde{ H}$ is a homotopy from $ f$ to a map with image in  $ F$, so  $ [f] = [0] \in \pi_n(E,F)$ by lemma I.16.
\end{proof}

\begin{coro}
If $ (E,B,F,p)$ is a fibration, then we get a long exact sequence
 \begin{align*}
	\cdots \to \pi_n(F) \xrightarrow{ i_*} \pi_n(E) \xrightarrow{ p_*} \pi_n(B) \xrightarrow{ \partial } \pi_{n-1} (F) \to \cdots   
\end{align*}
where $ i$ is inclusion and  $ \pi_n(B) \cong \pi_n(E,F) \xrightarrow{ \partial } \pi_{n-1}(F) $.
\end{coro}
\begin{proof}
Theorem I.17 gives the long exact sequence and we simply replace $ \pi_n(E,F)$ with $ \pi_n(B)$.
\end{proof}

\begin{coro}
$ \pi_k(S^{2n+1}) \cong \pi_k( \cc P^{n})$ for $ k>2$. In particular,  $ \pi_3(S^2= \cc P^{1}) \cong \pi_3(S^3) \cong \zz$.
\end{coro}
\begin{proof}
Recall we have the Hopf fibrations. So
\begin{align*}
	\pi_k(S^{1}) \to \pi_k(S^{2n+1}) \to \pi_k( \cc P^{n}) \to \pi_{k-1}(S^{1})
\end{align*}
Since $ \rr$ is the universal cover of $ S^{1}$, we know $ \pi_k(S^{1}) \cong \pi_k(\rr) = 0$ for $ k\geq 2$. So for  $ k>0$ we have  $ k-1 >1$ so
 \begin{align*}
	0 \to \pi_k(S^{2n+1}) \to \pi_k( \cc P^{n}) \to 0
\end{align*}
\end{proof}
Note
\begin{align*}
	0= \pi_2(S^3) \to \pi_2(S^2) \to \pi_1(S^{1}) = \zz \to \pi_1(S^3) =0
\end{align*}
So we know this without Hurewicz.

\begin{coro}
$ X$ is path connected, then
 \begin{align*}
	\pi_k(X) \cong \pi_{k-1}( \Omega(X)).
\end{align*}
\end{coro}
Note: we already know this from Cor I.8.
\begin{proof}
Since $ (P(X),X, \Omega(X),)$ is a fibration and $ P(X)$ is contractible so  $ \pi_k(P(X)) =0$. Hence
\begin{align*}
	\to \pi_k(P(X)) \to \pi_k(X) \to \pi_{k-1}(\Omega(X)) \to \pi_{k-1} (P(X))
\end{align*}
\end{proof}
\begin{coro}
$ \pi_k(O(n-1)) \cong \pi_k(O(n))$ for $ k<n-2$.
 $ \pi_k(U(n)) \cong \pi_k(U(n-1))$ for $ k<2n-2$.
\end{coro}
\begin{proof}
Recall $ (O(n),S^{n-1},O(n-1))$ is a fibration.
\begin{align*}
	\pi_{k+1} (S^{n-1}) \to \pi_{k} (O(n-1)) \to \pi_{k} (O(n)) \to \pi_{k} (S^{n-1})
\end{align*}
since $ k+1 <n-1$ so we have iso. Similar for  $ U(n)$.
\end{proof}
\begin{remark}
This corollary implies that for large $ n$,  $ \pi_k(O(n))$ is independent of $ k$ for  $ k$ small. Can we compute this?

We have inclusions  $ O(1) \to O(2) \to \cdots$. Let $ O = \lim_{ n \to \infty} O(n) = \bigcup_{ n =1}^{\infty} O(n)$. Similar for $ U$. Then the corollary yields  $ \pi_k(O) \cong \pi_k(O(n))$ if $ n > k+2$ and  $ \pi_k(U) \cong \pi_k(U(n))$ if $ n> k+2 /2$.
\end{remark}

\begin{thm}[Bott Perodicity]
$ \pi_k(O) \cong \pi_{k+8}(O)$.
$ \pi_k(U) \cong \pi_{k+2}(U)$.
\end{thm}

\begin{remark}
Use $ (O(n),\{\pm 1\}, \text{SO}( n) , \det)$ is a bundle so $ \pi_k( \text{SO}( n) ) \cong \pi_k(O(n)) \ \forall \ k >0$. Similarly $ (U(n),S^{1}, \text{SU}( n) ,)$. So $ \pi_k( \text{SU}( n) ) \cong \pi_k(U(n)) \ \forall \ k>1$.
\end{remark}

Recall $ V_{n,k} \cong O(n) /O(n-k)$ are the $ k$-frames in  $ \rr^{n}$ and $ V_{n,k}( \cc) \cong U(n) / U(n-k)$.
\begin{coro}
\begin{align*}
 \pi_j(V_{n,k}) \cong \begin{cases}
	 0 & j<n-k\\
	 \zz & j=n-k \text{ even or } k=1\\
	 \zz /2 & j=n-k \text{ odd}\\ 
 \end{cases}
 \pi_jV_{n,k}( \cc) \cong \begin{cases}
	 0& j \leq 2(n-k)\\
	 \zz & j=2(n-k)+1\\
 \end{cases}
\end{align*}
\end{coro}
\begin{proof}
Recall $ V_{n+1,k+1} = O(n+1) / O(n-k) = \text{SO}((n+1) / \text{SO}(n-k)  $. Since $ \text{SO}( n) \subseteq \text{SO}(n+1) $, we have $ V_{n,k} \subseteq V_{n+1,k+1}$ as quotient groups. Diagram.

Let's start with $ k=1$. Diagram.
 \begin{align*}
	\pi_j(S^{n}) \xrightarrow{ \partial } \pi_{j-1} (S^{n-1}) \to \pi_{j-1} (V_{n+1,2}) \to \pi_{j-1}(S^{n})
\end{align*}
If $ j \leq n-1$ then  $ \pi_j(S^{n}) =0 = \pi_{j-1}(S^{n})$ so $ \pi_{j-1}(V_{n+1,2}) \cong \pi_{j-1}(S^{n-1}) = 0$. For $ j=n$ we get
 \begin{align*}
	\pi_n(S^{n}) \cong \zz \xrightarrow{ \partial } \pi_{n-1} (S^{n-1}) \cong \zz \to \pi_{n-1}(V_{n+1,2}) \to 0
\end{align*}
So $ \pi_{n-1}(V_{n+1,2}) \cong \pi_{n-1}(S^{n-1}) / \im \partial $. Recall we define  $ \partial $ by taking $ f:(D^{n}, \partial D^{n}) \to (S^{n},s_0) \in \pi_n(S^{n})$ lfiting to get $ \widetilde{ f}: (D^{n}, \partial D^{n}) \to (V_{n+1,2},F)$ taking $ \widetilde{ f}|_{\partial D^{n}}: \partial D^{n} \to S^{n-1}$. So we have
\begin{align*}
	\partial ([f]) = \widetilde{ f} |_{ \partial D^{n}}: \partial D^{n} \to S^{n-1}.
\end{align*}

Fact:
\begin{enumerate}[label=(\arabic*)]
	\item There exists a vector field $ v$ on  $ S^{n}$ with a single zero at $ s_0$, its index is 0 if $ n$ odd and  $ 2$ if  $ n$ even. Index: for an isolated zero of a vector field $ v$, take a small sphere  $ S_{ \epsilon}^{n-1}$. Then we have a map $ S_{ \epsilon}^{n-1} \to S^{n-1}, x\mapsto \frac{v(x)}{ |v(x)|}$. Then the index is just the degree of this map.
	\item If  $ f:(D^{n},\partial D^{n}) \to S^{n}$ is the quotient map, then it generates $ \pi_n(S^{n})$, and $ \widetilde{ f}: S^{n} - \{s_0\} \to V_{n+1,2} $,
		\begin{align*}
			\widetilde{ f}(x) = \left( x, \frac{v(x)}{ |v(x)|} \right) 
		\end{align*}
		is a lift of $ f $ to $ V_{n+1,2}$. Note $ p \circ \widetilde{ f} = f$.
	\item index of $ v$ is the degree of  $ \widetilde{ f}|_{ \partial D^{n}}: \partial D^{n} \to S^{n-1}$, so
		\begin{align*}
			\partial [f] = \deg (\widetilde{ f}|_{ \partial D^{n}}) [g]
		\end{align*}
		where $ [g]$ is generator of  $ \pi_{n-1}(S^{n-1})$.
\end{enumerate}
Hence we prove $ k=1$ case.

Assume this is true for  $ k$ and we show  $ k+1$.

So
\begin{align*}
	\pi_{j+1}(S^{n}) \to \pi_j(V_{n,k}) \to \pi_j(V_{n+1,k+1}) \to \pi_j(S^{n})
\end{align*}
for $ j<n-k$ we have  $ \pi$
for $ j=n-k$,
 \begin{align*}
 	\pi_j(V_{n+1,k+1}) \cong \pi_j(V_{n,k})
 \end{align*}
 $ V_{n,k}( \cc)$ is similar.
\end{proof}

\end{document}
