\documentclass[12pt,class=article,crop=false]{standalone} 
\newcommand{\alert}[1]{{\bf \color{red} [Alert:] #1}}
\newcommand{\todo}[1]{{\bf \color{orange} [TODO:] #1}}
\newcommand{\real}[1][]{\mathbb{R}^{#1}}
\newcommand{\myeqn}[1]{(\ref{#1})}
\newcommand{\myex}[1]{Example \ref{#1}}
\newcommand{\defeq}{\stackrel{\mathrm{def}}{=}}
\newcommand{\parder}[2]{\frac{\partial #1}{\partial #2}}
\newcommand{\Lie}[3][]{\mathsf{L}_{#3}^{#1} #2}
\newcommand{\LieA}[1]{\mathsf{Lie}(#1)}
\newcommand{\lieder}[2]{\mathcal{L}_{#2} #1}
\renewcommand{\t}{^{\mbox{\tiny\sf T}}}
\newcommand{\trans}{^{\mbox{\tiny\sf T}}}
\newcommand{\markup}[1]{\{\textbf{#1}\}}
\newcommand{\msub}[1]{_\mathrm{#1}}
\newcommand{\msup}[1]{^\mathrm{#1}}
\newcommand{\inv}[1]{#1^{-1}}
\newcommand{\pinv}[1]{{#1}^{+}}
\newcommand{\myfracA}[2]{\displaystyle{\frac{#1}{#2}}}
\newcommand{\myfracB}[2]{{#1}/{#2}}
\newcommand{\mydiffA}[1]{\dot{#1}}
\newcommand{\mydiffB}[2]{\myfracA{\mathrm{d}{#1}}{\mathrm{d}{#2}}}
\newcommand{\ball}[2]{\mathcal{B}_{#1}\left(#2\right)}
\newcommand{\acos}[1]{\cos^{-1}\left(#1\right)}
\newcommand{\asin}[1]{\sin^{-1}\left(#1\right)}
\newcommand{\mani}{\mathcal{M}}
\newcommand{\tang}[2]{\mathsf{T}_{#1} #2}
\newcommand{\LieB}[2]{[ #1, #2 ]}
\newcommand{\LieBad}[3][]{\mathsf{ad}_{#2}^{#1} #3}
\newcommand{\ReachVT}{\mathcal{R}^V_T}
\newcommand{\ReachVt}{\mathcal{R}^V_t}
\newcommand{\ReachVTe}{\mathcal{R}^V_{\le T}}
\newcommand{\ReachT}{\mathcal{R}_T}
\newcommand{\Reacht}{\mathcal{R}_t}
\newcommand{\ReachTe}{\mathcal{R}_{\le T}}
\newcommand{\accLA}[1]{\mathsf{Lie}(#1)}
\newcommand{\accD}{\Delta_{\mathcal{F}}}
\newcommand{\accSA}{\mathsf{Lie}(\mathcal{G},f)}
\newcommand{\accDS}{\Delta_{\mathcal{G}}}
\newcommand{\eval}[3]{\mathsf{Ev}^{#2}_{#1}\left( #3 \right)}
\newcommand{\stlc}{\textsc{stlc}}
\newcommand{\clf}{\textsc{clf}}
\newcommand{\jqlf}{\textsc{jqlf}}
\newcommand{\dlas}{\textsc{dlas}}
\newcommand{\Ad}[2]{\mathsf{Ad}_{#1} #2}
\newcommand{\xe}{\ensuremath{x_e}}
\newcommand{\lebg}[1]{\mathcal{L}_{#1}}
\newcommand{\lebgx}[1]{\mathcal{L}_{#1 \mathrm{e}}}
\newcommand{\dom}{D}
\newcommand{\domT}{[t_0,\infty) \times D}
\newcommand{\rarrow}{\rightarrow}
\renewcommand{\d}{\mathrm{d}}
\renewcommand{\Re}{\mathbb{R}}
\newcommand{\C}{\mathrm{C}}

\newcommand{\QED}{{\unskip\nobreak\hfil\penalty50\hskip2em\vadjust{}
		\nobreak\hfil$\Box$\parfillskip=0pt\finalhyphendemerits=0\par}\vspace{0.1cm}}
\newcommand{\eoEx}{{\unskip\nobreak\hfil\penalty50\hskip0em\vadjust{}
		\nobreak\hfil$\Large\Diamond$\parfillskip=0pt\finalhyphendemerits=0\par}\vspace{0.1cm}}

\newcommand{\sgn}{\ensuremath{\operatorname{sgn}}}
\newcommand{\sat}{\ensuremath{\operatorname{sat}}}

\newcommand{\half}{\frac{1}{2}}
\newcommand{\shalf}{\mbox{$\frac{1}{2}$}}
\newcommand{\marcom}[1]{\marginpar{\footnotesize #1}}
\newcommand{\der}{\mathrm{D}}
\newcommand{\e}{\mathrm{e}}
\newcommand{\dt}{\mathrm{d}t}

\newcommand{\cA}{\ensuremath{\mathcal{A}}}
\newcommand{\cB}{\ensuremath{\mathcal{B}}}
\newcommand{\cG}{\ensuremath{\mathcal{G}}}
\newcommand{\cK}{\ensuremath{\mathcal{K}}}
\newcommand{\cW}{\ensuremath{\mathcal{W}}}
\newcommand{\cZ}{\ensuremath{\mathcal{Z}}}
\newcommand{\cS}{\ensuremath{\mathcal{S}}}
\newcommand{\cD}{\ensuremath{\mathcal{D}}}
\newcommand{\cP}{\ensuremath{\mathcal{P}}}
\newcommand{\cV}{\ensuremath{\mathcal{V}}}
\newcommand{\cL}{\ensuremath{\mathcal{L}}}
\newcommand{\cN}{\ensuremath{\mathcal{N}}}
\newcommand{\cI}{\ensuremath{\mathcal{I}}}
\newcommand{\cR}{\ensuremath{\mathcal{R}}}
\newcommand{\cM}{\ensuremath{\mathcal{M}}}
\newcommand{\cC}{\ensuremath{\mathcal{C}}}
\newcommand{\cF}{\ensuremath{\mathcal{F}}}
\newcommand{\cH}{\ensuremath{\mathcal{H}}}
\newcommand{\cO}{\ensuremath{\mathcal{O}}}
\newcommand{\cX}{\ensuremath{\mathcal{X}}}
\newcommand{\cY}{\ensuremath{\mathcal{Y}}}
\newcommand{\Ci}{\ensuremath{\mathcal{C}^\infty}}
\newcommand{\ISS}{\textsc{iss}}
\newcommand{\LISS}{\textsc{liss}}
\newcommand{\GAS}{\textsc{gas}}
\newcommand{\GS}{\textsc{gs}}
\newcommand{\LES}{\textsc{les}}
\newcommand{\GUAS}{\textsc{guas}}
\newcommand{\BIBO}{\textsc{bibo}}
\newcommand{\spec}{\ensuremath{\operatorname{spec}}}
\newcommand{\spn}{\ensuremath{\operatorname{span}}}
\renewcommand{\i}{\mathrm{i\,}}

\renewcommand{\implies}{\Rightarrow}

\renewcommand{\theenumi}{$\roman{enumi})$}
\renewcommand{\labelenumi}{\theenumi}

\font\ptmten=zptmcmrm scaled 1200
\newcommand{\w}{\mbox{{\ptmten w}}}
\newcommand{\z}{\mbox{{\ptmten z}}}
\renewcommand{\Re}{\mathbb{R}}

\newcommand{\cl}{\operatorname{cl}}
\newcommand{\intr}{\operatorname{int}}
\newcommand{\rank}{\operatorname{rank}}
\newcommand{\co}{\operatorname{co}}
\newcommand{\aff}{\operatorname{aff}}

\theoremstyle{plain}
\newtheorem{theorem}{Theorem}[chapter]
\newtheorem{claim}[theorem]{Claim}
\newtheorem{corollary}[theorem]{Corollary}
\newtheorem{prop}[theorem]{Proposition}
\newtheorem{fact}[theorem]{Fact}
\newtheorem{lemma}[theorem]{Lemma}

\newtheorem{remark}{Remark}[chapter]

\theoremstyle{definition}
\newtheorem{assume}[theorem]{Assumption}
\newtheorem{defn}[theorem]{Definition}
\newtheorem{problem}[theorem]{Problem}
\newtheorem{exercise}{Exercise}
\newtheorem{example}[theorem]{Example}


\begin{document}
\section{Characteristic Classes}
There is another way to think of Steifel-Whitney classes:
\begin{thm}
There exists a unique function $ w_i: \text{Vect}(M) \to H^{i}(M;\zz /2 )$, where $ \text{Vect}(M) $ is the set of vector bundles over $ M$, for all  $ M$ and $ i$ satisfying
 \begin{enumerate}[label=(\arabic*)]
	\item $ w_i(f^* E) = f^* (w_i(E)) \ \forall \ f: M \to N$
	\item $ w_0(E) =1, w_1(E) = 0 \ \forall \ i > $ fiber dimension of $ E$.
	\item  $ w(E_1 \oplus E_2) = w(E_1) \smile w(E_2)$ where $ w(E)=1+w_2(E)+w_2(E)+ \ldots$. $ E_1 \oplus E_2$ has fibers the direct sum of fibers of $ E_1$ and $ E_2$. That is, given $(E,M,\rr^{m}),(E,N,\rr^{n})$ we get $ (E_1 \times E_2,M \times N, \rr^{m} \times \rr^{n})$. If $ M=N$, let $ \Delta: M \to M \times M$ be the diagonal map. Define $ E_1 \oplus E_2 = \Delta^* (E_1 \times E_2)$.
	\item $ w_1( \gamma_n) \neq 0$ where $ \gamma_n$ is the universal line bundle over $ \rr P^{n}$.
\end{enumerate}
\end{thm}
For 4 recall
\begin{align*}
	\gamma_n=\{(\ell,v) \in \rr P^{n}+ \rr^{n+1}: v \in\ell\} .
\end{align*}
Exercise: $ \gamma_n$ is a line bundle over $ \rr P^{n}$.

Recall $ H^{i}(\rr P^{n}; \zz /2) \cong \zz /2 \ \forall \ 0 \leq i \leq n$. So 4) implies  $ w_1( \gamma_n)$ generates $ H^{1}(\rr P^{n}l \zz /2)$.

get $ \gamma$ over $ \rr P^{\infty}$ and 4) implies $ w_i( \gamma) \neq 0$.

Exercise:
\begin{enumerate}[label=(\arabic*)]
	\item $ i: \rr P^{n} \to \rr P^{m}$ we get $ i^* ( \gamma_m) = \gamma_n$. If $ w_1( \gamma_1) \neq 0$ then true for $ \gamma_n \ \forall \ n$.
	\item Show $ \rr P^{1} \cong S^{1}$ and $ \gamma_1$ is infinite Mobius band. This is non-orientable, so from above $ w_1( \gamma_1) \neq 0$.
\end{enumerate}
It remains to prove part 3 and uniqueness in the theorem. First we look at some consequences.

Easy consequences:
\begin{enumerate}[label=(\arabic*)]
	\item If $ E_1 \cong E_2$ then $ w_i(E_1) = w_i(E_2) \ \forall \ i$ by 1.
	\item If $ E$ is a trivial bundle, then  $ w_i(E) = 0 \ \forall \ i>0$. This follows from obstruction theory. Let's check using Theorem 5. 
		If $ E \to M$ is trivial, let $ f: M to \{x_0\} $, then $ f^* (\{x_0\} \times \rr^{n}) = E$. So $ w_i(E) = f^* (w_i(\{x_0\} \times \rr^{n} )) = 0$.
	\item If $ E'$ is a trivial bundle and  $ E$ any vector bundle then
		 \begin{align*}
			w_i(E \oplus E') = w_i(E)
		\end{align*}
		from 3.
\end{enumerate}
Recall Whitney showed that any $ n$-manifold embeds in  $ \rr^{2n}$ and immerses in $ \rr^{2n-1}$.
\begin{thm}
If $ \rr P^{2^{r}}$ is immersed in $ \rr^{2^{r}+k}$ then $ k$ must be at least  $ 2^{r}-1$.
\end{thm}
That is, Whitney's Theorem cannot be improved for all manifolds.

Note: if $ f: M^{n} \to \rr^{k}$ is an embedding, then we have the normal bundle of $ f(M^{n})$ :
\begin{align*}
	\nu(M) = \{v \in T_x \rr^{k}: v \perp T_x M \ \forall \ x \in f(M)\} .
\end{align*}
And $ TM \oplus \nu(M) = T\rr^{k}|_M = M \times \rr^{k}$.

Exercise: show $ \nu(M)$ is well-defined if $ f$ is just an immersion and we still have
 \begin{align*}
	TM \oplus \nu(M) = f^* T\rr^{k} = M \times \rr^{k}.
\end{align*}

\begin{thm}
If $ M$ is the boundary of a compact manifold  $ W$, then all Stiefel-Whitney numbers are zero.
\end{thm}

\begin{remark}
The converse is also true by Thom.
\end{remark}

\begin{defn}
Given two unoriented manifolds $ M_1$ and $ M_2$, we say they are \allbold{unoriented cobordant} if there exists a compact manifold $ W$  s.t.\ $ \partial W = M_1 \cup M_2$. 
\end{defn}

\begin{coro}
Two closed, connected manifolds $ M_1$ and $ M_2$ are unoriented cobordant iff they have the same Stiefel-Whitney numbers. 
\end{coro}


\end{document}
