\documentclass[12pt,class=article,crop=false]{standalone} 
\newcommand{\alert}[1]{{\bf \color{red} [Alert:] #1}}
\newcommand{\todo}[1]{{\bf \color{orange} [TODO:] #1}}
\newcommand{\real}[1][]{\mathbb{R}^{#1}}
\newcommand{\myeqn}[1]{(\ref{#1})}
\newcommand{\myex}[1]{Example \ref{#1}}
\newcommand{\defeq}{\stackrel{\mathrm{def}}{=}}
\newcommand{\parder}[2]{\frac{\partial #1}{\partial #2}}
\newcommand{\Lie}[3][]{\mathsf{L}_{#3}^{#1} #2}
\newcommand{\LieA}[1]{\mathsf{Lie}(#1)}
\newcommand{\lieder}[2]{\mathcal{L}_{#2} #1}
\renewcommand{\t}{^{\mbox{\tiny\sf T}}}
\newcommand{\trans}{^{\mbox{\tiny\sf T}}}
\newcommand{\markup}[1]{\{\textbf{#1}\}}
\newcommand{\msub}[1]{_\mathrm{#1}}
\newcommand{\msup}[1]{^\mathrm{#1}}
\newcommand{\inv}[1]{#1^{-1}}
\newcommand{\pinv}[1]{{#1}^{+}}
\newcommand{\myfracA}[2]{\displaystyle{\frac{#1}{#2}}}
\newcommand{\myfracB}[2]{{#1}/{#2}}
\newcommand{\mydiffA}[1]{\dot{#1}}
\newcommand{\mydiffB}[2]{\myfracA{\mathrm{d}{#1}}{\mathrm{d}{#2}}}
\newcommand{\ball}[2]{\mathcal{B}_{#1}\left(#2\right)}
\newcommand{\acos}[1]{\cos^{-1}\left(#1\right)}
\newcommand{\asin}[1]{\sin^{-1}\left(#1\right)}
\newcommand{\mani}{\mathcal{M}}
\newcommand{\tang}[2]{\mathsf{T}_{#1} #2}
\newcommand{\LieB}[2]{[ #1, #2 ]}
\newcommand{\LieBad}[3][]{\mathsf{ad}_{#2}^{#1} #3}
\newcommand{\ReachVT}{\mathcal{R}^V_T}
\newcommand{\ReachVt}{\mathcal{R}^V_t}
\newcommand{\ReachVTe}{\mathcal{R}^V_{\le T}}
\newcommand{\ReachT}{\mathcal{R}_T}
\newcommand{\Reacht}{\mathcal{R}_t}
\newcommand{\ReachTe}{\mathcal{R}_{\le T}}
\newcommand{\accLA}[1]{\mathsf{Lie}(#1)}
\newcommand{\accD}{\Delta_{\mathcal{F}}}
\newcommand{\accSA}{\mathsf{Lie}(\mathcal{G},f)}
\newcommand{\accDS}{\Delta_{\mathcal{G}}}
\newcommand{\eval}[3]{\mathsf{Ev}^{#2}_{#1}\left( #3 \right)}
\newcommand{\stlc}{\textsc{stlc}}
\newcommand{\clf}{\textsc{clf}}
\newcommand{\jqlf}{\textsc{jqlf}}
\newcommand{\dlas}{\textsc{dlas}}
\newcommand{\Ad}[2]{\mathsf{Ad}_{#1} #2}
\newcommand{\xe}{\ensuremath{x_e}}
\newcommand{\lebg}[1]{\mathcal{L}_{#1}}
\newcommand{\lebgx}[1]{\mathcal{L}_{#1 \mathrm{e}}}
\newcommand{\dom}{D}
\newcommand{\domT}{[t_0,\infty) \times D}
\newcommand{\rarrow}{\rightarrow}
\renewcommand{\d}{\mathrm{d}}
\renewcommand{\Re}{\mathbb{R}}
\newcommand{\C}{\mathrm{C}}

\newcommand{\QED}{{\unskip\nobreak\hfil\penalty50\hskip2em\vadjust{}
		\nobreak\hfil$\Box$\parfillskip=0pt\finalhyphendemerits=0\par}\vspace{0.1cm}}
\newcommand{\eoEx}{{\unskip\nobreak\hfil\penalty50\hskip0em\vadjust{}
		\nobreak\hfil$\Large\Diamond$\parfillskip=0pt\finalhyphendemerits=0\par}\vspace{0.1cm}}

\newcommand{\sgn}{\ensuremath{\operatorname{sgn}}}
\newcommand{\sat}{\ensuremath{\operatorname{sat}}}

\newcommand{\half}{\frac{1}{2}}
\newcommand{\shalf}{\mbox{$\frac{1}{2}$}}
\newcommand{\marcom}[1]{\marginpar{\footnotesize #1}}
\newcommand{\der}{\mathrm{D}}
\newcommand{\e}{\mathrm{e}}
\newcommand{\dt}{\mathrm{d}t}

\newcommand{\cA}{\ensuremath{\mathcal{A}}}
\newcommand{\cB}{\ensuremath{\mathcal{B}}}
\newcommand{\cG}{\ensuremath{\mathcal{G}}}
\newcommand{\cK}{\ensuremath{\mathcal{K}}}
\newcommand{\cW}{\ensuremath{\mathcal{W}}}
\newcommand{\cZ}{\ensuremath{\mathcal{Z}}}
\newcommand{\cS}{\ensuremath{\mathcal{S}}}
\newcommand{\cD}{\ensuremath{\mathcal{D}}}
\newcommand{\cP}{\ensuremath{\mathcal{P}}}
\newcommand{\cV}{\ensuremath{\mathcal{V}}}
\newcommand{\cL}{\ensuremath{\mathcal{L}}}
\newcommand{\cN}{\ensuremath{\mathcal{N}}}
\newcommand{\cI}{\ensuremath{\mathcal{I}}}
\newcommand{\cR}{\ensuremath{\mathcal{R}}}
\newcommand{\cM}{\ensuremath{\mathcal{M}}}
\newcommand{\cC}{\ensuremath{\mathcal{C}}}
\newcommand{\cF}{\ensuremath{\mathcal{F}}}
\newcommand{\cH}{\ensuremath{\mathcal{H}}}
\newcommand{\cO}{\ensuremath{\mathcal{O}}}
\newcommand{\cX}{\ensuremath{\mathcal{X}}}
\newcommand{\cY}{\ensuremath{\mathcal{Y}}}
\newcommand{\Ci}{\ensuremath{\mathcal{C}^\infty}}
\newcommand{\ISS}{\textsc{iss}}
\newcommand{\LISS}{\textsc{liss}}
\newcommand{\GAS}{\textsc{gas}}
\newcommand{\GS}{\textsc{gs}}
\newcommand{\LES}{\textsc{les}}
\newcommand{\GUAS}{\textsc{guas}}
\newcommand{\BIBO}{\textsc{bibo}}
\newcommand{\spec}{\ensuremath{\operatorname{spec}}}
\newcommand{\spn}{\ensuremath{\operatorname{span}}}
\renewcommand{\i}{\mathrm{i\,}}

\renewcommand{\implies}{\Rightarrow}

\renewcommand{\theenumi}{$\roman{enumi})$}
\renewcommand{\labelenumi}{\theenumi}

\font\ptmten=zptmcmrm scaled 1200
\newcommand{\w}{\mbox{{\ptmten w}}}
\newcommand{\z}{\mbox{{\ptmten z}}}
\renewcommand{\Re}{\mathbb{R}}

\newcommand{\cl}{\operatorname{cl}}
\newcommand{\intr}{\operatorname{int}}
\newcommand{\rank}{\operatorname{rank}}
\newcommand{\co}{\operatorname{co}}
\newcommand{\aff}{\operatorname{aff}}

\theoremstyle{plain}
\newtheorem{theorem}{Theorem}[chapter]
\newtheorem{claim}[theorem]{Claim}
\newtheorem{corollary}[theorem]{Corollary}
\newtheorem{prop}[theorem]{Proposition}
\newtheorem{fact}[theorem]{Fact}
\newtheorem{lemma}[theorem]{Lemma}

\newtheorem{remark}{Remark}[chapter]

\theoremstyle{definition}
\newtheorem{assume}[theorem]{Assumption}
\newtheorem{defn}[theorem]{Definition}
\newtheorem{problem}[theorem]{Problem}
\newtheorem{exercise}{Exercise}
\newtheorem{example}[theorem]{Example}


\begin{document}
\begin{eg}
Recall a vector bundle $ (E,B,\rr^{n})$ has a $ k$-frame iff the structure group reduces to $ \text{GL}_{ n-k}(\rr) $. If we choose a metric on $ E$, then the structure group of  $ E$ is  $ O(n)$.  $ E$ has an orthonormal  $ k$-frame iff structure group reduces to  $ O( n-k) $. In terms of principal bundles. Let $ \mathcal{ F}(E)$ be the orthonormal frame bundle associated to $ E$. Now  $ E$ has an orthonormal  $ k$-frame iff  $ \mathcal{ F}(E) / O(n-k)$ has a section. The fibers of $ F(E) / O(n-k)$ are  $ O(n) /O(n-k) \cong V_{n,k}$. In Cor II.11, we have $ \pi_i(V_{n,k})$. Unfortunately, $ \pi_1(M)$ does not necessarily act trivially on $ \pi_{n-k}(V_{n,k})$ if it is $ \zz$. But if we take $ \pi_{n-k}(V_{n,k}) \bmod 2$ then action is trivial (since any action on $ \zz /2$ is trivial. So we have a primary obstruction to a $ k$-frame over the  $ (n-k+1)$-skeleton of  $ M$. So  $ \gamma^{n-k+1}(E) \in H^{n-k+1}(M; \pi_{n-k}(V_{n,k}) \bmod 2)$. We set $ w_{\ell}(E) := \gamma^{\ell}(E) \in H^{\ell}(M; \zz /2)$. This is called the  \allbold{$ \ell$th- Steifel-Witney class of $ E$}. When $ \ell$ is even, this is the primary obstruction to existence of an $ n-\ell+1$-frame on $ M^{(\ell-1)}$ that extends to $ M^{(\ell)}$. If $ \ell$ odd then $ w_{\ell}(E)$ is the $ \bmod 2$ reduction of the primary obstruction.

Fact: $ w_\ell$ determines primary obstruction for all $ \ell$.

Exercise: given $ (E,M,\rr^{n})$,
\begin{enumerate}[label=(\arabic*)]
	\item $ w_\ell(E) 0 \iff$ there exists $ n$-frame over  $ M^{(0)}$ that extends over $ M^{(1)}$ iff there is an orientation on $ E$, \emph{i.e.} structure group reduces to $ \text{SO}( n) $. 
	\item If $ E$ is orientable, then  $ w_2(E) = 0 \iff$ there exists an $ (n-1)$-frame over  $ M^{(1)}$ that extends to $ M^{(2)}$ iff there exists an $ n$-frame on  $ M^{(1)}$ that extends to $ M^{(2)}$ (since there is a canonical unit vector with positive orientation orthogonal to $ (n-1)$-frame ). This is called a \allbold{spin} structure on $ E$. 
\end{enumerate}

\end{eg}
\begin{eg}
If $ (E,M,\rr^{n})$ is oriented, then $ \pi_1(M)$ acts trivially on $ \pi_{n-1}(V_{n,1}) \cong \zz$. Exercise: check this. So we get a primary obstruction $ e(E) \in H^{n}(M;\zz)$ to the existence of a non-zero section of $ E$ over  $ M^{(n)}$. $ e(E)$ is called the  \allbold{Euler class}. 

Exercise:
\begin{enumerate}[label=(\arabic*)]
	\item If $ s:M \to E$ is a section and $ M$ a manifold. Then we can isotop  $ s$ so it is transverse to zero section  $Z= \{0 \in E_x: x \in M\} $. Then $ e(E)= P.D.[s^{-1}(Z)]$ (Poincare duality of homology class of $ s ^{-1}(Z)$).
	\item $ e(TM) ([M]) = \Chi(M)$.
\end{enumerate}
\end{eg}
\begin{eg}
Let $(E,M, \cc^{n})$ be a vector bundle. Structure group is $ \text{GL}_{ n}(\cc) $ so this is a complex vector bundle. As discussed above, we can take structure group to be $ U(n)$. Let  $ \mathcal{ F}(E)$ be the frame bundle which is a $ U(n)$ bundle over  $ M$.

Then  $ E$ has a complex  $ k$-frame iff structure group reduces to  $ U(n) /U(n-k) \cong V_{n,k}( \cc)\iff \mathcal{ F}(E) / U(n-k)$ has a section. We have $ \pi_i(V_{n,k}( \cc))$ by Cor II.11.

Exercise: $ \pi_1(M)$ acts trivially on $ \pi_{2(n-k)+1}(V_{n,k}( \cc)$ as the fiber of $ \mathcal{ E} / U(n-k)$. Thus we get a primary obstruction to a $ M^{2(n-k)+2}$ :
\begin{align*}
	\gamma^{2(n-k)+2}(E) \in H^{2(n-k)+2}(M; \zz = \pi_{2(n-k)+1}(V_{n,k}( \cc))).
\end{align*}
Define $ c_k(E) = \gamma^{2k}(E) \in H^{2k}(M;\zz)$. This is called the \allbold{$ k$th Chern class of $ E$}. Then  $ c_k(E)$ is the obstruction to a complex $ (n-k+1)$ frame on  $ M^{(2k-1)}$ that extends to $ M^{(2k)}$.

Exercise:
\begin{enumerate}[label=(\arabic*)]
	\item $ c_n(E) = e(E)$.
	\item $ w_{2i+1}(E) = 0$, which implies complex bundles are orientable.
	\item $ w_{2i}(E) = c_i(E) \bmod 2$.
	\item $ c_1(E) = 0 \iff$ structure group of $ E$ reduces to  $ \text{SU}((n)) $, ``complex orientation".
	\item if $ \overline{E}$ is $ E$ with the conjugate complex structure,  \emph{i.e.} $ z \in \cc$ multiply by $ \overline{z}$, then $ c_i( \overline{E}) = (-1)^{i}c_i(E)$. Hint: easy for $ c_n(E)$, reduce to this case (see Milnor-Stasheff).
\end{enumerate}
\end{eg}

\begin{eg}
	$ (E,M,\rr^{n})$, then $ E \otimes_\rr \cc$ is a complex vector bundle. Then the \allbold{ $ i$th Pontrjagin class of $ E$ } is
	\begin{align*}
		p_1(E) = (-1)^{i} c_{2i}(E \otimes \cc) \in H^{4i}(M;\zz)
	\end{align*}
	Exercise:
	\begin{enumerate}[label=(\arabic*)]
		\item show $ E \otimes \cc$ and $ \overline{E \otimes \cc}$ are isomorphic. Use this to show $ c_{2i+1}(E \otimes \cc)$ is 2-torsion.
		\item If $ E$ is an oriented  $ \rr^{2n}$-bundle, then
			\begin{align*}
				p_n(E) = e(E) \smile e(E)
			\end{align*}
		\item If $ E$ is a complex bundle and and  $ E^{\rr}$ denotes the underlying real bundle. Then
			\begin{align*}
				E^{\rr} \otimes \cc \cong E \oplus \overline{E}.
			\end{align*}
		\item use 3) to show for complex $ \cc^{n}$-bundle $ E$ we have
			 \begin{align*}
				1-P_1(E)+p_2(E)-\ldots\pm p_n(E) = (1+ c_1(E) + \ldots c_n(E)) \smile (1-c_1(E)+c_2(E)-\ldots \pm c_n(E))
				p_1(E) = c_1(E) \smile c_1(E) - 2c_2(E)
			\end{align*}
	\end{enumerate}
\end{eg}

Characteristic class, in general, do not determine a bundle. But we have
\begin{enumerate}[label=(\arabic*)]
	\item complex line bundles are determined by $ c_1$, and any $ \alpha \in H^2(M)$ is $ c_1$ of some $ \cc$-bundle.
	\item $ c^2$-bundles are determined by $ c_1$ and $ c_2$, and any $ (\alpha , \beta) \in H^2(M) \times H^{4}(M)$ is $ (c_1,c_2)$ of some $ \cc^2$-bundle.
	\item $ \text{SO}(3) $-bundles are isomorphic iff $ w_2,p_1$ agree.
	\item $ \text{SO}(4) $-bundles are isomorphic $ \iff$ $ w_2,p_1,e$ agree.
\end{enumerate}
Exercise: prove them.
\end{document}
