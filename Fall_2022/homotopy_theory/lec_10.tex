\documentclass[12pt,class=article,crop=false]{standalone} 
%Fall 2022
% Some basic packages
\usepackage{standalone}[subpreambles=true]
\usepackage[utf8]{inputenc}
\usepackage[T1]{fontenc}
\usepackage{textcomp}
\usepackage[english]{babel}
\usepackage{url}
\usepackage{graphicx}
%\usepackage{quiver}
\usepackage{float}
\usepackage{enumitem}
\usepackage{lmodern}
\usepackage{comment}
\usepackage{hyperref}
\usepackage[usenames,svgnames,dvipsnames]{xcolor}
\usepackage[margin=1in]{geometry}
\usepackage{pdfpages}

\pdfminorversion=7

% Don't indent paragraphs, leave some space between them
\usepackage{parskip}

% Hide page number when page is empty
\usepackage{emptypage}
\usepackage{subcaption}
\usepackage{multicol}
\usepackage[b]{esvect}

% Math stuff
\usepackage{amsmath, amsfonts, mathtools, amsthm, amssymb}
\usepackage{bbm}
\usepackage{stmaryrd}
\allowdisplaybreaks

% Fancy script capitals
\usepackage{mathrsfs}
\usepackage{cancel}
% Bold math
\usepackage{bm}
% Some shortcuts
\newcommand{\rr}{\ensuremath{\mathbb{R}}}
\newcommand{\zz}{\ensuremath{\mathbb{Z}}}
\newcommand{\qq}{\ensuremath{\mathbb{Q}}}
\newcommand{\nn}{\ensuremath{\mathbb{N}}}
\newcommand{\ff}{\ensuremath{\mathbb{F}}}
\newcommand{\cc}{\ensuremath{\mathbb{C}}}
\newcommand{\ee}{\ensuremath{\mathbb{E}}}
\newcommand{\hh}{\ensuremath{\mathbb{H}}}
\renewcommand\O{\ensuremath{\emptyset}}
\newcommand{\norm}[1]{{\left\lVert{#1}\right\rVert}}
\newcommand{\dbracket}[1]{{\left\llbracket{#1}\right\rrbracket}}
\newcommand{\ve}[1]{{\bm{#1}}}
\newcommand\allbold[1]{{\boldmath\textbf{#1}}}
\DeclareMathOperator{\lcm}{lcm}
\DeclareMathOperator{\im}{im}
\DeclareMathOperator{\coim}{coim}
\DeclareMathOperator{\dom}{dom}
\DeclareMathOperator{\tr}{tr}
\DeclareMathOperator{\rank}{rank}
\DeclareMathOperator*{\var}{Var}
\DeclareMathOperator*{\ev}{E}
\DeclareMathOperator{\dg}{deg}
\DeclareMathOperator{\aff}{aff}
\DeclareMathOperator{\conv}{conv}
\DeclareMathOperator{\inte}{int}
\DeclareMathOperator*{\argmin}{argmin}
\DeclareMathOperator*{\argmax}{argmax}
\DeclareMathOperator{\graph}{graph}
\DeclareMathOperator{\sgn}{sgn}
\DeclareMathOperator*{\Rep}{Rep}
\DeclareMathOperator{\Proj}{Proj}
\DeclareMathOperator{\mat}{mat}
\DeclareMathOperator{\diag}{diag}
\DeclareMathOperator{\aut}{Aut}
\DeclareMathOperator{\gal}{Gal}
\DeclareMathOperator{\inn}{Inn}
\DeclareMathOperator{\edm}{End}
\DeclareMathOperator{\Hom}{Hom}
\DeclareMathOperator{\ext}{Ext}
\DeclareMathOperator{\tor}{Tor}
\DeclareMathOperator{\Span}{Span}
\DeclareMathOperator{\Stab}{Stab}
\DeclareMathOperator{\cont}{cont}
\DeclareMathOperator{\Ann}{Ann}
\DeclareMathOperator{\Div}{div}
\DeclareMathOperator{\curl}{curl}
\DeclareMathOperator{\nat}{Nat}
\DeclareMathOperator{\gr}{Gr}
\DeclareMathOperator{\vect}{Vect}
\DeclareMathOperator{\id}{id}
\DeclareMathOperator{\Mod}{Mod}
\DeclareMathOperator{\sign}{sign}
\DeclareMathOperator{\Surf}{Surf}
\DeclareMathOperator{\fcone}{fcone}
\DeclareMathOperator{\Rot}{Rot}
\DeclareMathOperator{\grad}{grad}
\DeclareMathOperator{\atan2}{atan2}
\DeclareMathOperator{\Ric}{Ric}
\let\vec\relax
\DeclareMathOperator{\vec}{vec}
\let\Re\relax
\DeclareMathOperator{\Re}{Re}
\let\Im\relax
\DeclareMathOperator{\Im}{Im}
% Put x \to \infty below \lim
\let\svlim\lim\def\lim{\svlim\limits}

%wide hat
\usepackage{scalerel,stackengine}
\stackMath
\newcommand*\wh[1]{%
\savestack{\tmpbox}{\stretchto{%
  \scaleto{%
    \scalerel*[\widthof{\ensuremath{#1}}]{\kern-.6pt\bigwedge\kern-.6pt}%
    {\rule[-\textheight/2]{1ex}{\textheight}}%WIDTH-LIMITED BIG WEDGE
  }{\textheight}% 
}{0.5ex}}%
\stackon[1pt]{#1}{\tmpbox}%
}
\parskip 1ex

%Make implies and impliedby shorter
\let\implies\Rightarrow
\let\impliedby\Leftarrow
\let\iff\Leftrightarrow
\let\epsilon\varepsilon

% Add \contra symbol to denote contradiction
\usepackage{stmaryrd} % for \lightning
\newcommand\contra{\scalebox{1.5}{$\lightning$}}

% \let\phi\varphi

% Command for short corrections
% Usage: 1+1=\correct{3}{2}

\definecolor{correct}{HTML}{009900}
\newcommand\correct[2]{\ensuremath{\:}{\color{red}{#1}}\ensuremath{\to }{\color{correct}{#2}}\ensuremath{\:}}
\newcommand\green[1]{{\color{correct}{#1}}}

% horizontal rule
\newcommand\hr{
    \noindent\rule[0.5ex]{\linewidth}{0.5pt}
}

% hide parts
\newcommand\hide[1]{}

% si unitx
\usepackage{siunitx}
\sisetup{locale = FR}

%allows pmatrix to stretch
\makeatletter
\renewcommand*\env@matrix[1][\arraystretch]{%
  \edef\arraystretch{#1}%
  \hskip -\arraycolsep
  \let\@ifnextchar\new@ifnextchar
  \array{*\c@MaxMatrixCols c}}
\makeatother

\renewcommand{\arraystretch}{0.8}

\renewcommand{\baselinestretch}{1.5}

\usepackage{graphics}
\usepackage{epstopdf}

\RequirePackage{hyperref}
%%
%% Add support for color in order to color the hyperlinks.
%% 
\hypersetup{
  colorlinks = true,
  urlcolor = blue,
  citecolor = blue
}
%%fakesection Links
\hypersetup{
    colorlinks,
    linkcolor={red!50!black},
    citecolor={green!50!black},
    urlcolor={blue!80!black}
}
%customization of cleveref
\RequirePackage[capitalize,nameinlink]{cleveref}[0.19]

% Per SIAM Style Manual, "section" should be lowercase
\crefname{section}{section}{sections}
\crefname{subsection}{subsection}{subsections}
\Crefname{section}{Section}{Sections}
\Crefname{subsection}{Subsection}{Subsections}

% Per SIAM Style Manual, "Figure" should be spelled out in references
\Crefname{figure}{Figure}{Figures}

% Per SIAM Style Manual, don't say equation in front on an equation.
\crefformat{equation}{\textup{#2(#1)#3}}
\crefrangeformat{equation}{\textup{#3(#1)#4--#5(#2)#6}}
\crefmultiformat{equation}{\textup{#2(#1)#3}}{ and \textup{#2(#1)#3}}
{, \textup{#2(#1)#3}}{, and \textup{#2(#1)#3}}
\crefrangemultiformat{equation}{\textup{#3(#1)#4--#5(#2)#6}}%
{ and \textup{#3(#1)#4--#5(#2)#6}}{, \textup{#3(#1)#4--#5(#2)#6}}{, and \textup{#3(#1)#4--#5(#2)#6}}

% But spell it out at the beginning of a sentence.
\Crefformat{equation}{#2Equation~\textup{(#1)}#3}
\Crefrangeformat{equation}{Equations~\textup{#3(#1)#4--#5(#2)#6}}
\Crefmultiformat{equation}{Equations~\textup{#2(#1)#3}}{ and \textup{#2(#1)#3}}
{, \textup{#2(#1)#3}}{, and \textup{#2(#1)#3}}
\Crefrangemultiformat{equation}{Equations~\textup{#3(#1)#4--#5(#2)#6}}%
{ and \textup{#3(#1)#4--#5(#2)#6}}{, \textup{#3(#1)#4--#5(#2)#6}}{, and \textup{#3(#1)#4--#5(#2)#6}}

% Make number non-italic in any environment.
\crefdefaultlabelformat{#2\textup{#1}#3}

% Environments
\makeatother
% For box around Definition, Theorem, \ldots
%%fakesection Theorems
\usepackage{thmtools}
\usepackage[framemethod=TikZ]{mdframed}

\theoremstyle{definition}
\mdfdefinestyle{mdbluebox}{%
	roundcorner = 10pt,
	linewidth=1pt,
	skipabove=12pt,
	innerbottommargin=9pt,
	skipbelow=2pt,
	nobreak=true,
	linecolor=blue,
	backgroundcolor=TealBlue!5,
}
\declaretheoremstyle[
	headfont=\sffamily\bfseries\color{MidnightBlue},
	mdframed={style=mdbluebox},
	headpunct={\\[3pt]},
	postheadspace={0pt}
]{thmbluebox}

\mdfdefinestyle{mdredbox}{%
	linewidth=0.5pt,
	skipabove=12pt,
	frametitleaboveskip=5pt,
	frametitlebelowskip=0pt,
	skipbelow=2pt,
	frametitlefont=\bfseries,
	innertopmargin=4pt,
	innerbottommargin=8pt,
	nobreak=false,
	linecolor=RawSienna,
	backgroundcolor=Salmon!5,
}
\declaretheoremstyle[
	headfont=\bfseries\color{RawSienna},
	mdframed={style=mdredbox},
	headpunct={\\[3pt]},
	postheadspace={0pt},
]{thmredbox}

\declaretheorem[%
style=thmbluebox,name=Theorem,numberwithin=section]{thm}
\declaretheorem[style=thmbluebox,name=Lemma,sibling=thm]{lem}
\declaretheorem[style=thmbluebox,name=Proposition,sibling=thm]{prop}
\declaretheorem[style=thmbluebox,name=Corollary,sibling=thm]{coro}
\declaretheorem[style=thmredbox,name=Example,sibling=thm]{eg}

\mdfdefinestyle{mdgreenbox}{%
	roundcorner = 10pt,
	linewidth=1pt,
	skipabove=12pt,
	innerbottommargin=9pt,
	skipbelow=2pt,
	nobreak=true,
	linecolor=ForestGreen,
	backgroundcolor=ForestGreen!5,
}

\declaretheoremstyle[
	headfont=\bfseries\sffamily\color{ForestGreen!70!black},
	bodyfont=\normalfont,
	spaceabove=2pt,
	spacebelow=1pt,
	mdframed={style=mdgreenbox},
	headpunct={ --- },
]{thmgreenbox}

\declaretheorem[style=thmgreenbox,name=Definition,sibling=thm]{defn}

\mdfdefinestyle{mdgreenboxsq}{%
	linewidth=1pt,
	skipabove=12pt,
	innerbottommargin=9pt,
	skipbelow=2pt,
	nobreak=true,
	linecolor=ForestGreen,
	backgroundcolor=ForestGreen!5,
}
\declaretheoremstyle[
	headfont=\bfseries\sffamily\color{ForestGreen!70!black},
	bodyfont=\normalfont,
	spaceabove=2pt,
	spacebelow=1pt,
	mdframed={style=mdgreenboxsq},
	headpunct={},
]{thmgreenboxsq}
\declaretheoremstyle[
	headfont=\bfseries\sffamily\color{ForestGreen!70!black},
	bodyfont=\normalfont,
	spaceabove=2pt,
	spacebelow=1pt,
	mdframed={style=mdgreenboxsq},
	headpunct={},
]{thmgreenboxsq*}

\mdfdefinestyle{mdblackbox}{%
	skipabove=8pt,
	linewidth=3pt,
	rightline=false,
	leftline=true,
	topline=false,
	bottomline=false,
	linecolor=black,
	backgroundcolor=RedViolet!5!gray!5,
}
\declaretheoremstyle[
	headfont=\bfseries,
	bodyfont=\normalfont\small,
	spaceabove=0pt,
	spacebelow=0pt,
	mdframed={style=mdblackbox}
]{thmblackbox}

\theoremstyle{plain}
\declaretheorem[name=Question,sibling=thm,style=thmblackbox]{ques}
\declaretheorem[name=Remark,sibling=thm,style=thmgreenboxsq]{remark}
\declaretheorem[name=Remark,sibling=thm,style=thmgreenboxsq*]{remark*}
\newtheorem{ass}[thm]{Assumptions}

\theoremstyle{definition}
\newtheorem*{problem}{Problem}
\newtheorem{claim}[thm]{Claim}
\theoremstyle{remark}
\newtheorem*{case}{Case}
\newtheorem*{notation}{Notation}
\newtheorem*{note}{Note}
\newtheorem*{motivation}{Motivation}
\newtheorem*{intuition}{Intuition}
\newtheorem*{conjecture}{Conjecture}

% Make section starts with 1 for report type
%\renewcommand\thesection{\arabic{section}}

% End example and intermezzo environments with a small diamond (just like proof
% environments end with a small square)
\usepackage{etoolbox}
\AtEndEnvironment{vb}{\null\hfill$\diamond$}%
\AtEndEnvironment{intermezzo}{\null\hfill$\diamond$}%
% \AtEndEnvironment{opmerking}{\null\hfill$\diamond$}%

% Fix some spacing
% http://tex.stackexchange.com/questions/22119/how-can-i-change-the-spacing-before-theorems-with-amsthm
\makeatletter
\def\thm@space@setup{%
  \thm@preskip=\parskip \thm@postskip=0pt
}

% Fix some stuff
% %http://tex.stackexchange.com/questions/76273/multiple-pdfs-with-page-group-included-in-a-single-page-warning
\pdfsuppresswarningpagegroup=1


% My name
\author{Jaden Wang}



\begin{document}
\begin{eg}
Recall a vector bundle $ (E,B,\rr^{n})$ has a $ k$-frame iff the structure group reduces to $ \text{GL}_{ n-k}(\rr) $. If we choose a metric on $ E$, then the structure group of  $ E$ is  $ O(n)$.  $ E$ has an orthonormal  $ k$-frame iff structure group reduces to  $ O( n-k) $. In terms of principal bundles. Let $ \mathcal{ F}(E)$ be the orthonormal frame bundle associated to $ E$. Now  $ E$ has an orthonormal  $ k$-frame iff  $ \mathcal{ F}(E) / O(n-k)$ has a section. The fibers of $ F(E) / O(n-k)$ are  $ O(n) /O(n-k) \cong V_{n,k}$. In Cor II.11, we have $ \pi_i(V_{n,k})$. Unfortunately, $ \pi_1(M)$ does not necessarily act trivially on $ \pi_{n-k}(V_{n,k})$ if it is $ \zz$. But if we take $ \pi_{n-k}(V_{n,k}) \bmod 2$ then action is trivial (since any action on $ \zz /2$ is trivial. So we have a primary obstruction to a $ k$-frame over the  $ (n-k+1)$-skeleton of  $ M$. So  $ \gamma^{n-k+1}(E) \in H^{n-k+1}(M; \pi_{n-k}(V_{n,k}) \bmod 2)$. We set $ w_{\ell}(E) := \gamma^{\ell}(E) \in H^{\ell}(M; \zz /2)$. This is called the  \allbold{$ \ell$th- Steifel-Witney class of $ E$}. When $ \ell$ is even, this is the primary obstruction to existence of an $ n-\ell+1$-frame on $ M^{(\ell-1)}$ that extends to $ M^{(\ell)}$. If $ \ell$ odd then $ w_{\ell}(E)$ is the $ \bmod 2$ reduction of the primary obstruction.

Fact: $ w_\ell$ determines primary obstruction for all $ \ell$.

Exercise: given $ (E,M,\rr^{n})$,
\begin{enumerate}[label=(\arabic*)]
	\item $ w_\ell(E) 0 \iff$ there exists $ n$-frame over  $ M^{(0)}$ that extends over $ M^{(1)}$ iff there is an orientation on $ E$, \emph{i.e.} structure group reduces to $ \text{SO}( n) $. 
	\item If $ E$ is orientable, then  $ w_2(E) = 0 \iff$ there exists an $ (n-1)$-frame over  $ M^{(1)}$ that extends to $ M^{(2)}$ iff there exists an $ n$-frame on  $ M^{(1)}$ that extends to $ M^{(2)}$ (since there is a canonical unit vector with positive orientation orthogonal to $ (n-1)$-frame ). This is called a \allbold{spin} structure on $ E$. 
\end{enumerate}

\end{eg}
\begin{eg}
If $ (E,M,\rr^{n})$ is oriented, then $ \pi_1(M)$ acts trivially on $ \pi_{n-1}(V_{n,1}) \cong \zz$. Exercise: check this. So we get a primary obstruction $ e(E) \in H^{n}(M;\zz)$ to the existence of a non-zero section of $ E$ over  $ M^{(n)}$. $ e(E)$ is called the  \allbold{Euler class}. 

Exercise:
\begin{enumerate}[label=(\arabic*)]
	\item If $ s:M \to E$ is a section and $ M$ a manifold. Then we can isotop  $ s$ so it is transverse to zero section  $Z= \{0 \in E_x: x \in M\} $. Then $ e(E)= P.D.[s^{-1}(Z)]$ (Poincare duality of homology class of $ s ^{-1}(Z)$).
	\item $ e(TM) ([M]) = \Chi(M)$.
\end{enumerate}
\end{eg}
\begin{eg}
Let $(E,M, \cc^{n})$ be a vector bundle. Structure group is $ \text{GL}_{ n}(\cc) $ so this is a complex vector bundle. As discussed above, we can take structure group to be $ U(n)$. Let  $ \mathcal{ F}(E)$ be the frame bundle which is a $ U(n)$ bundle over  $ M$.

Then  $ E$ has a complex  $ k$-frame iff structure group reduces to  $ U(n) /U(n-k) \cong V_{n,k}( \cc)\iff \mathcal{ F}(E) / U(n-k)$ has a section. We have $ \pi_i(V_{n,k}( \cc))$ by Cor II.11.

Exercise: $ \pi_1(M)$ acts trivially on $ \pi_{2(n-k)+1}(V_{n,k}( \cc)$ as the fiber of $ \mathcal{ E} / U(n-k)$. Thus we get a primary obstruction to a $ M^{2(n-k)+2}$ :
\begin{align*}
	\gamma^{2(n-k)+2}(E) \in H^{2(n-k)+2}(M; \zz = \pi_{2(n-k)+1}(V_{n,k}( \cc))).
\end{align*}
Define $ c_k(E) = \gamma^{2k}(E) \in H^{2k}(M;\zz)$. This is called the \allbold{$ k$th Chern class of $ E$}. Then  $ c_k(E)$ is the obstruction to a complex $ (n-k+1)$ frame on  $ M^{(2k-1)}$ that extends to $ M^{(2k)}$.

Exercise:
\begin{enumerate}[label=(\arabic*)]
	\item $ c_n(E) = e(E)$.
	\item $ w_{2i+1}(E) = 0$, which implies complex bundles are orientable.
	\item $ w_{2i}(E) = c_i(E) \bmod 2$.
	\item $ c_1(E) = 0 \iff$ structure group of $ E$ reduces to  $ \text{SU}((n)) $, ``complex orientation".
	\item if $ \overline{E}$ is $ E$ with the conjugate complex structure,  \emph{i.e.} $ z \in \cc$ multiply by $ \overline{z}$, then $ c_i( \overline{E}) = (-1)^{i}c_i(E)$. Hint: easy for $ c_n(E)$, reduce to this case (see Milnor-Stasheff).
\end{enumerate}
\end{eg}

\begin{eg}
	$ (E,M,\rr^{n})$, then $ E \otimes_\rr \cc$ is a complex vector bundle. Then the \allbold{ $ i$th Pontrjagin class of $ E$ } is
	\begin{align*}
		p_1(E) = (-1)^{i} c_{2i}(E \otimes \cc) \in H^{4i}(M;\zz)
	\end{align*}
	Exercise:
	\begin{enumerate}[label=(\arabic*)]
		\item show $ E \otimes \cc$ and $ \overline{E \otimes \cc}$ are isomorphic. Use this to show $ c_{2i+1}(E \otimes \cc)$ is 2-torsion.
		\item If $ E$ is an oriented  $ \rr^{2n}$-bundle, then
			\begin{align*}
				p_n(E) = e(E) \smile e(E)
			\end{align*}
		\item If $ E$ is a complex bundle and and  $ E^{\rr}$ denotes the underlying real bundle. Then
			\begin{align*}
				E^{\rr} \otimes \cc \cong E \oplus \overline{E}.
			\end{align*}
		\item use 3) to show for complex $ \cc^{n}$-bundle $ E$ we have
			 \begin{align*}
				1-P_1(E)+p_2(E)-\ldots\pm p_n(E) = (1+ c_1(E) + \ldots c_n(E)) \smile (1-c_1(E)+c_2(E)-\ldots \pm c_n(E))
				p_1(E) = c_1(E) \smile c_1(E) - 2c_2(E)
			\end{align*}
	\end{enumerate}
\end{eg}

Characteristic class, in general, do not determine a bundle. But we have
\begin{enumerate}[label=(\arabic*)]
	\item complex line bundles are determined by $ c_1$, and any $ \alpha \in H^2(M)$ is $ c_1$ of some $ \cc$-bundle.
	\item $ c^2$-bundles are determined by $ c_1$ and $ c_2$, and any $ (\alpha , \beta) \in H^2(M) \times H^{4}(M)$ is $ (c_1,c_2)$ of some $ \cc^2$-bundle.
	\item $ \text{SO}(3) $-bundles are isomorphic iff $ w_2,p_1$ agree.
	\item $ \text{SO}(4) $-bundles are isomorphic $ \iff$ $ w_2,p_1,e$ agree.
\end{enumerate}
Exercise: prove them.
\end{document}
