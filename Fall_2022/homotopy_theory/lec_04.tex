\documentclass[12pt,class=article,crop=false]{standalone} 
%Fall 2022
% Some basic packages
\usepackage{standalone}[subpreambles=true]
\usepackage[utf8]{inputenc}
\usepackage[T1]{fontenc}
\usepackage{textcomp}
\usepackage[english]{babel}
\usepackage{url}
\usepackage{graphicx}
%\usepackage{quiver}
\usepackage{float}
\usepackage{enumitem}
\usepackage{lmodern}
\usepackage{comment}
\usepackage{hyperref}
\usepackage[usenames,svgnames,dvipsnames]{xcolor}
\usepackage[margin=1in]{geometry}
\usepackage{pdfpages}

\pdfminorversion=7

% Don't indent paragraphs, leave some space between them
\usepackage{parskip}

% Hide page number when page is empty
\usepackage{emptypage}
\usepackage{subcaption}
\usepackage{multicol}
\usepackage[b]{esvect}

% Math stuff
\usepackage{amsmath, amsfonts, mathtools, amsthm, amssymb}
\usepackage{bbm}
\usepackage{stmaryrd}
\allowdisplaybreaks

% Fancy script capitals
\usepackage{mathrsfs}
\usepackage{cancel}
% Bold math
\usepackage{bm}
% Some shortcuts
\newcommand{\rr}{\ensuremath{\mathbb{R}}}
\newcommand{\zz}{\ensuremath{\mathbb{Z}}}
\newcommand{\qq}{\ensuremath{\mathbb{Q}}}
\newcommand{\nn}{\ensuremath{\mathbb{N}}}
\newcommand{\ff}{\ensuremath{\mathbb{F}}}
\newcommand{\cc}{\ensuremath{\mathbb{C}}}
\newcommand{\ee}{\ensuremath{\mathbb{E}}}
\newcommand{\hh}{\ensuremath{\mathbb{H}}}
\renewcommand\O{\ensuremath{\emptyset}}
\newcommand{\norm}[1]{{\left\lVert{#1}\right\rVert}}
\newcommand{\dbracket}[1]{{\left\llbracket{#1}\right\rrbracket}}
\newcommand{\ve}[1]{{\bm{#1}}}
\newcommand\allbold[1]{{\boldmath\textbf{#1}}}
\DeclareMathOperator{\lcm}{lcm}
\DeclareMathOperator{\im}{im}
\DeclareMathOperator{\coim}{coim}
\DeclareMathOperator{\dom}{dom}
\DeclareMathOperator{\tr}{tr}
\DeclareMathOperator{\rank}{rank}
\DeclareMathOperator*{\var}{Var}
\DeclareMathOperator*{\ev}{E}
\DeclareMathOperator{\dg}{deg}
\DeclareMathOperator{\aff}{aff}
\DeclareMathOperator{\conv}{conv}
\DeclareMathOperator{\inte}{int}
\DeclareMathOperator*{\argmin}{argmin}
\DeclareMathOperator*{\argmax}{argmax}
\DeclareMathOperator{\graph}{graph}
\DeclareMathOperator{\sgn}{sgn}
\DeclareMathOperator*{\Rep}{Rep}
\DeclareMathOperator{\Proj}{Proj}
\DeclareMathOperator{\mat}{mat}
\DeclareMathOperator{\diag}{diag}
\DeclareMathOperator{\aut}{Aut}
\DeclareMathOperator{\gal}{Gal}
\DeclareMathOperator{\inn}{Inn}
\DeclareMathOperator{\edm}{End}
\DeclareMathOperator{\Hom}{Hom}
\DeclareMathOperator{\ext}{Ext}
\DeclareMathOperator{\tor}{Tor}
\DeclareMathOperator{\Span}{Span}
\DeclareMathOperator{\Stab}{Stab}
\DeclareMathOperator{\cont}{cont}
\DeclareMathOperator{\Ann}{Ann}
\DeclareMathOperator{\Div}{div}
\DeclareMathOperator{\curl}{curl}
\DeclareMathOperator{\nat}{Nat}
\DeclareMathOperator{\gr}{Gr}
\DeclareMathOperator{\vect}{Vect}
\DeclareMathOperator{\id}{id}
\DeclareMathOperator{\Mod}{Mod}
\DeclareMathOperator{\sign}{sign}
\DeclareMathOperator{\Surf}{Surf}
\DeclareMathOperator{\fcone}{fcone}
\DeclareMathOperator{\Rot}{Rot}
\DeclareMathOperator{\grad}{grad}
\DeclareMathOperator{\atan2}{atan2}
\DeclareMathOperator{\Ric}{Ric}
\let\vec\relax
\DeclareMathOperator{\vec}{vec}
\let\Re\relax
\DeclareMathOperator{\Re}{Re}
\let\Im\relax
\DeclareMathOperator{\Im}{Im}
% Put x \to \infty below \lim
\let\svlim\lim\def\lim{\svlim\limits}

%wide hat
\usepackage{scalerel,stackengine}
\stackMath
\newcommand*\wh[1]{%
\savestack{\tmpbox}{\stretchto{%
  \scaleto{%
    \scalerel*[\widthof{\ensuremath{#1}}]{\kern-.6pt\bigwedge\kern-.6pt}%
    {\rule[-\textheight/2]{1ex}{\textheight}}%WIDTH-LIMITED BIG WEDGE
  }{\textheight}% 
}{0.5ex}}%
\stackon[1pt]{#1}{\tmpbox}%
}
\parskip 1ex

%Make implies and impliedby shorter
\let\implies\Rightarrow
\let\impliedby\Leftarrow
\let\iff\Leftrightarrow
\let\epsilon\varepsilon

% Add \contra symbol to denote contradiction
\usepackage{stmaryrd} % for \lightning
\newcommand\contra{\scalebox{1.5}{$\lightning$}}

% \let\phi\varphi

% Command for short corrections
% Usage: 1+1=\correct{3}{2}

\definecolor{correct}{HTML}{009900}
\newcommand\correct[2]{\ensuremath{\:}{\color{red}{#1}}\ensuremath{\to }{\color{correct}{#2}}\ensuremath{\:}}
\newcommand\green[1]{{\color{correct}{#1}}}

% horizontal rule
\newcommand\hr{
    \noindent\rule[0.5ex]{\linewidth}{0.5pt}
}

% hide parts
\newcommand\hide[1]{}

% si unitx
\usepackage{siunitx}
\sisetup{locale = FR}

%allows pmatrix to stretch
\makeatletter
\renewcommand*\env@matrix[1][\arraystretch]{%
  \edef\arraystretch{#1}%
  \hskip -\arraycolsep
  \let\@ifnextchar\new@ifnextchar
  \array{*\c@MaxMatrixCols c}}
\makeatother

\renewcommand{\arraystretch}{0.8}

\renewcommand{\baselinestretch}{1.5}

\usepackage{graphics}
\usepackage{epstopdf}

\RequirePackage{hyperref}
%%
%% Add support for color in order to color the hyperlinks.
%% 
\hypersetup{
  colorlinks = true,
  urlcolor = blue,
  citecolor = blue
}
%%fakesection Links
\hypersetup{
    colorlinks,
    linkcolor={red!50!black},
    citecolor={green!50!black},
    urlcolor={blue!80!black}
}
%customization of cleveref
\RequirePackage[capitalize,nameinlink]{cleveref}[0.19]

% Per SIAM Style Manual, "section" should be lowercase
\crefname{section}{section}{sections}
\crefname{subsection}{subsection}{subsections}
\Crefname{section}{Section}{Sections}
\Crefname{subsection}{Subsection}{Subsections}

% Per SIAM Style Manual, "Figure" should be spelled out in references
\Crefname{figure}{Figure}{Figures}

% Per SIAM Style Manual, don't say equation in front on an equation.
\crefformat{equation}{\textup{#2(#1)#3}}
\crefrangeformat{equation}{\textup{#3(#1)#4--#5(#2)#6}}
\crefmultiformat{equation}{\textup{#2(#1)#3}}{ and \textup{#2(#1)#3}}
{, \textup{#2(#1)#3}}{, and \textup{#2(#1)#3}}
\crefrangemultiformat{equation}{\textup{#3(#1)#4--#5(#2)#6}}%
{ and \textup{#3(#1)#4--#5(#2)#6}}{, \textup{#3(#1)#4--#5(#2)#6}}{, and \textup{#3(#1)#4--#5(#2)#6}}

% But spell it out at the beginning of a sentence.
\Crefformat{equation}{#2Equation~\textup{(#1)}#3}
\Crefrangeformat{equation}{Equations~\textup{#3(#1)#4--#5(#2)#6}}
\Crefmultiformat{equation}{Equations~\textup{#2(#1)#3}}{ and \textup{#2(#1)#3}}
{, \textup{#2(#1)#3}}{, and \textup{#2(#1)#3}}
\Crefrangemultiformat{equation}{Equations~\textup{#3(#1)#4--#5(#2)#6}}%
{ and \textup{#3(#1)#4--#5(#2)#6}}{, \textup{#3(#1)#4--#5(#2)#6}}{, and \textup{#3(#1)#4--#5(#2)#6}}

% Make number non-italic in any environment.
\crefdefaultlabelformat{#2\textup{#1}#3}

% Environments
\makeatother
% For box around Definition, Theorem, \ldots
%%fakesection Theorems
\usepackage{thmtools}
\usepackage[framemethod=TikZ]{mdframed}

\theoremstyle{definition}
\mdfdefinestyle{mdbluebox}{%
	roundcorner = 10pt,
	linewidth=1pt,
	skipabove=12pt,
	innerbottommargin=9pt,
	skipbelow=2pt,
	nobreak=true,
	linecolor=blue,
	backgroundcolor=TealBlue!5,
}
\declaretheoremstyle[
	headfont=\sffamily\bfseries\color{MidnightBlue},
	mdframed={style=mdbluebox},
	headpunct={\\[3pt]},
	postheadspace={0pt}
]{thmbluebox}

\mdfdefinestyle{mdredbox}{%
	linewidth=0.5pt,
	skipabove=12pt,
	frametitleaboveskip=5pt,
	frametitlebelowskip=0pt,
	skipbelow=2pt,
	frametitlefont=\bfseries,
	innertopmargin=4pt,
	innerbottommargin=8pt,
	nobreak=false,
	linecolor=RawSienna,
	backgroundcolor=Salmon!5,
}
\declaretheoremstyle[
	headfont=\bfseries\color{RawSienna},
	mdframed={style=mdredbox},
	headpunct={\\[3pt]},
	postheadspace={0pt},
]{thmredbox}

\declaretheorem[%
style=thmbluebox,name=Theorem,numberwithin=section]{thm}
\declaretheorem[style=thmbluebox,name=Lemma,sibling=thm]{lem}
\declaretheorem[style=thmbluebox,name=Proposition,sibling=thm]{prop}
\declaretheorem[style=thmbluebox,name=Corollary,sibling=thm]{coro}
\declaretheorem[style=thmredbox,name=Example,sibling=thm]{eg}

\mdfdefinestyle{mdgreenbox}{%
	roundcorner = 10pt,
	linewidth=1pt,
	skipabove=12pt,
	innerbottommargin=9pt,
	skipbelow=2pt,
	nobreak=true,
	linecolor=ForestGreen,
	backgroundcolor=ForestGreen!5,
}

\declaretheoremstyle[
	headfont=\bfseries\sffamily\color{ForestGreen!70!black},
	bodyfont=\normalfont,
	spaceabove=2pt,
	spacebelow=1pt,
	mdframed={style=mdgreenbox},
	headpunct={ --- },
]{thmgreenbox}

\declaretheorem[style=thmgreenbox,name=Definition,sibling=thm]{defn}

\mdfdefinestyle{mdgreenboxsq}{%
	linewidth=1pt,
	skipabove=12pt,
	innerbottommargin=9pt,
	skipbelow=2pt,
	nobreak=true,
	linecolor=ForestGreen,
	backgroundcolor=ForestGreen!5,
}
\declaretheoremstyle[
	headfont=\bfseries\sffamily\color{ForestGreen!70!black},
	bodyfont=\normalfont,
	spaceabove=2pt,
	spacebelow=1pt,
	mdframed={style=mdgreenboxsq},
	headpunct={},
]{thmgreenboxsq}
\declaretheoremstyle[
	headfont=\bfseries\sffamily\color{ForestGreen!70!black},
	bodyfont=\normalfont,
	spaceabove=2pt,
	spacebelow=1pt,
	mdframed={style=mdgreenboxsq},
	headpunct={},
]{thmgreenboxsq*}

\mdfdefinestyle{mdblackbox}{%
	skipabove=8pt,
	linewidth=3pt,
	rightline=false,
	leftline=true,
	topline=false,
	bottomline=false,
	linecolor=black,
	backgroundcolor=RedViolet!5!gray!5,
}
\declaretheoremstyle[
	headfont=\bfseries,
	bodyfont=\normalfont\small,
	spaceabove=0pt,
	spacebelow=0pt,
	mdframed={style=mdblackbox}
]{thmblackbox}

\theoremstyle{plain}
\declaretheorem[name=Question,sibling=thm,style=thmblackbox]{ques}
\declaretheorem[name=Remark,sibling=thm,style=thmgreenboxsq]{remark}
\declaretheorem[name=Remark,sibling=thm,style=thmgreenboxsq*]{remark*}
\newtheorem{ass}[thm]{Assumptions}

\theoremstyle{definition}
\newtheorem*{problem}{Problem}
\newtheorem{claim}[thm]{Claim}
\theoremstyle{remark}
\newtheorem*{case}{Case}
\newtheorem*{notation}{Notation}
\newtheorem*{note}{Note}
\newtheorem*{motivation}{Motivation}
\newtheorem*{intuition}{Intuition}
\newtheorem*{conjecture}{Conjecture}

% Make section starts with 1 for report type
%\renewcommand\thesection{\arabic{section}}

% End example and intermezzo environments with a small diamond (just like proof
% environments end with a small square)
\usepackage{etoolbox}
\AtEndEnvironment{vb}{\null\hfill$\diamond$}%
\AtEndEnvironment{intermezzo}{\null\hfill$\diamond$}%
% \AtEndEnvironment{opmerking}{\null\hfill$\diamond$}%

% Fix some spacing
% http://tex.stackexchange.com/questions/22119/how-can-i-change-the-spacing-before-theorems-with-amsthm
\makeatletter
\def\thm@space@setup{%
  \thm@preskip=\parskip \thm@postskip=0pt
}

% Fix some stuff
% %http://tex.stackexchange.com/questions/76273/multiple-pdfs-with-page-group-included-in-a-single-page-warning
\pdfsuppresswarningpagegroup=1


% My name
\author{Jaden Wang}



\begin{document}
\section{Homotopy group and CW-complexes}

Recall if $ A$ is a top space, and  $ f: \bigsqcup_{ i \in I} S^{n-1} \to A$ then $ X = A \cup _f ( \bigsqcup_{ D^{n}}) = A \bigsqcup ( \bigsqcup D^{n}) / \sim$ where $ x \in \partial ( \bigsqcup_{ D^{n}})$ is identified with $ f(x) \in A$ is said to be obtained from $ A$ by attaching  $ n$ cells.

A  \allbold{relative CW-pair} is a pair $ (X,A)$  s.t.\ 
\begin{enumerate}[label=(\arabic*)]
	\item $ X$ is a top space.
	\item  $ A$ is a closed subspace.
	\item There exists a sequence of spaces  $ X^{(n)}, n=-1,0,1,\ldots$ called $ n$-skeleton  s.t.\
		\begin{enumerate}[label=(\alph*)]
			\item $ X^{(-1)} = A$.
			\item $ X^{(n)}$ is obtained from $ X^{(n-1)}$ by attaching $ n$-cells.
			\item  $ X = \bigcup_{ i =1}^{\infty} X^{(i)}$.
			\item $ B \subseteq X$ is closed iff $ B \cap X^{(n)}$ closed for all $ n$.
		\end{enumerate}
	If $ X^{(n)}$ for some $ n$ then we say  $ (X,A)$ is an  \allbold{$ n$-dimensional CW-pair}. Otherwise infinite. If $ A = \O$, then $ X$ is a CW-complex. If  $ X$ has a finite number of cells then  (d) is automatically ignored. 
\end{enumerate}

exercise: $ (X,A)$ a CW-pair then  $ X /A$ is a CW complex.

 \begin{eg}
\begin{enumerate}[label=(\arabic*)]
	\item A 1-dimensional CW-complex is a graph.
	\item any surface as a 2-dimensional CW-complex. Any $ n$-manifold is a CW-complex.
	\item If  $ X,Y$ are CW-complexes, then so is  $ X \times Y$. Exercise: work out the CW structure on $ X \times Y$ from the CW structure on $ X$ and  $ Y$.
\end{enumerate}
\end{eg}

A map $ f: X \to Y$ between CW-complexes is \allbold{cellular} if $ f(X^{(n)}) \subseteq Y^{(n)} \ \forall \ n$. 

\begin{thm}[cellular approximation]
If $ f: X \to Y$ is a map between CW-complexes and $ f$ is cellular on  $ A \subseteq X$ a sub CW-complex. Then $ f$ is homotopic rel  $ A$ to a map  $ g: X \to Y$ that is cellular on all of $ X$.
\end{thm}

\begin{prop}
$ \pi_k(S^{n}) = 0 \ \forall \ k <n$.
\end{prop}
\begin{proof}
Given $ f : (S^{k},s_0) \to (S^{n},x_0)$ where $ s_0,x_0$ part of $ 0$-skeleton. We can homotop  $ f$ to  $ g$  s.t.\ $ g((S^{k})^{(k)}) \subseteq (k$-skeleton of $ S^{n}) = \{x_0\} $. So $ f \simeq 0$ in $ \pi_n(S^{n})$.
\end{proof}

What about $ \pi_k(S^{n}) $ for $ k>n$. This is very hard in general.
 \begin{eg}
$ \pi_3(S^2) \neq 0$. To see this let $ f: S^3 \to S^2$ be the Hopf map. That is, think $ S^3 \subseteq \cc^2$, $ S^{1} \subseteq \cc$ the unit spheres. $ S^{1}$ acts on $ S^3$ by multiplication, \emph{i.e.} $ \lamda \in S^{1}$, then $ \lambda(z_1,z_2) = (\lambda z_1, \lambda z_2) \in S^3$. In fact $ S^3 /S^{1} = \cc P^{1} \cong S^2$. So the Hopf map is this quotient map. Exercise: $ \cc P^2  \cong \cc P^{1} \cup _f D^{4}$ (glue a 4-cell to $ S^2$ by the Hopf map).

If $ f \simeq $ const, then $ \cc P^2 \cong S^2 \vee S^{4}$. Easy to see generator $ [s^2] \in H^2(S^2 \vee S^{4})$. $ [s^2] \smile [s^2] = 0$ in $ H^{4}(S^2 \vee S^{4})$. Poincare duality says $ g \in H^2( \cc P^2)$ s.t.\ $ g \smile g \neq 0$ in $ H^{4}( \cc P^2)$. So $ f$ cannot be trivial in  $ \pi_3(S^2)$.
\end{eg}

\begin{lem}
$ X$ a CW-complex. Let  $ i : X^{(n)} \to X$ be inclusion then $ i$ induces an isomorphism  $ i_*: \pi_k(X^{(n)}) \to \pi_k(X)$ for $ k <n$ and a surjection for  $ k=n$.
\end{lem}

\begin{proof}
	$ i_*$ is surjective for  $ k = n$ by similar argument to previous proposition. Given $ [f] \in \pi_n(X)$, we have $ f: S^{n} \to X$. By cellular approximation theorem, we can homotop $ f$ to a cellular map  $ g$  s.t.\ $ g(S^{n}) \subseteq X^{(n)}$. Then viewing  $ g$ as a map from  $ S^{n}$ to $ X^{(n)}$, we see that $ [g] \in \pi_n(X^{(n)})$ is the element that maps to $ [f]$ under $ i_*$. 

	If $ k<n$ then  $ i_*$ is injective. suppose  $ f: S^{k} \to X^{(n)}, g : S^{k} \to X^{(n)}$ and $ [f] = [g]$ in  $ \pi_k(X)$. By cellular approximation, we can assume $ f,g$ map into  $ X^{(k)}$. Let $ H:S^{k} \times I \to X$ be the homotopy. Note: $ H$ is cellular on  $ (S^{k} \times I) \cup (s_0 \times I)$. Exercise: $ S^{k} \times I$ has a CW structure of $ \dim k+1$. Cellular approximation says we can homotop  $ H$ and  $ S^{k} \times I$ and $ s_0 \times I$ so its image is in $ X^{(k+1)} \subseteq X^{(n)}$. Therefore, $ f \simeq g$ in $ X^{(n)}$.
\end{proof}

\begin{lem}[Homotopy extension theorem]
Given a relative CW-complex $ (X,A)$ a map  $ f: X \to Y$ and a homotopy $ H:A \times I \to Y$ of $ f|_A$, then there exists an extension of  $ H $ to $ G: X \times I \to Y$ s.t.\ $ G(x,t) = H(x,t)$ on  $ A \times I$ and $ G(x,0)=f(x)$.
\end{lem}
Exercise: prove theorem 21 and 24 directly using this lemma.

\begin{proof}
For any $ D^{n}$ there is a deformation retraction of $ D^{n} \times I $ to $ D^{n} \times \{0\} \cup (\partial D^{n} \times I)=:B$. To see this, $ D^{n} \subseteq \rr^{n} = \rr^{n} \times \{0\} \subseteq \rr^{n+1} $. Also $ D^{n} \times I \subseteq \rr^{n+1}$. Let $ p=(0,\ldots,0,2)$. For any $ x \in D^{n} \times I$, let $ \ell_x$ be the line through $ p,x$ and it is going to intersect $ B$ at a unique point $ \widetilde{ r}(x)$. Then we have a deformation retract $ \widetilde{ r}_t(x) = t \widetilde{ r}(x) + (1-t) x$.

Now suppose $ X-A$ has one cell  $ D^{n}$. We know $ \partial D^{n} \subseteq A$, by hypothesis of the lemma, we have a map $ \overline{H}: X \times \{0\}  \cup (A \times I)=:C \to Y, (x,0)\mapsto f(x),(x,t)\mapsto H(x,t)$. Now let
\begin{align*}
	G: X \times I \to Y, G(x,t) = \begin{cases}
		\overline{H}(x,t) & x \in C\\
		\overline{H} \circ \widetilde{ r}(x,t) & x \in D^{n} \times I\\
	\end{cases}
\end{align*}
This is an extension, we can do this cell by cell.
\end{proof}

\begin{lem}
If $ (X,A)$ a relative CW-complex and  $ A$ contractible, then  $ X / A \simeq X$.
\end{lem}

\begin{proof}
Since $ A$ is contractible, we have a homotopy $ f: A \times I \to A$ s.t.\ $ f(x,0)=x, f_1$ is constant, $ f_t(x):=f(x,t)$. Note that $ f_0 = F_0|_A$ where $ F_0 = \text{id}_{ X}$. So HET yields a homotopy $ F: X \times I \to X$ by extending $ f$. Note that  $ F_t(A) \subseteq A$. Therefore, there are induced maps $ \overline{ F}_t: X /A \to X /A$ since everything in $ A$ gets sent to the same equivalence class, and everything outside  $ A$ is untouched by  $ F_t$ so the diagram commutes. Also $ F_1(A)=$pt. So $ F_1$ also induces a map $ h: X /A \to X$. By commutative diagram, $ h \circ q = F_1$, $ q \circ h = \overline{ F}_1$. But $ h \circ q = F_1 \simeq F_0 = \text{id}_{ X}$ and $ q \circ h = \overline{F}_1 \simeq \overline{F}_0 = \text{id}_{ X /A}$ so $ h,q$ are homotopy equivalences.
\end{proof}

\begin{defn}
	A space $ X$ is  \allbold{$ k$-connected} if $ \pi_{\ell} (X) =0 \ \forall \ \ell\leq k$.
\end{defn}

\begin{thm}
If $ X$ is a  $ k$-connected  CW-complex, then  $ X \simeq X'$ where $ X'$ is a CW-complex containing a single vertex and no cells of dimension 1 through  $ k$.
\end{thm}
\begin{proof}
Let $ x_0$ be a vertex, and $ v_1,\ldots,v_\ell$ be all the vertices. Since $ k>0$, $ \pi_0(X) = 0$ so $ X$ is path-connected, so there exists a path  $ \gamma_i$ from $ x_0$ to $ v_i$. By cellular approximation we can assume $ \im \gamma_i \subseteq X^{(1)}$. Attach $ D^2$ to $ X$ as follows:
~\begin{figure}[H]
	\centering
	\includegraphics[width=0.8\textwidth]{./figures/kconnect_disk.png}
\end{figure}
Call result $ \widetilde{ X}'$. Note: $ \widetilde{ X}'$ is a CW-complex where for each $ i$ we add a 1-cell and a 2-cell. Also $ \widetilde{ X}' \simeq X$ since we can just push the disk down into the boundary. Let $ e = \overline{ \widetilde{ X}' - X} $. Note that $ e$ is a contractible subcomplex of  $ \widetilde{ X}'$ (push down to path and then retract along the paths to $ x_0$). Now set $ \widetilde{ X} = \widetilde{ X}' / e$ then lemma 20 says $ \widetilde{ X}  \simeq \widetilde{ X}'$ since $ e$ is contractible. So  $ X \simeq \widetilde{ X}$ which has one vertex. More generally, let $ T$ be a tree in  $ X^{(1)}$ so $ \widetilde{ X} = X /T \simeq X$.

Assume $ X \simeq \wh{ X}$ where $ \wh{ X}$ is a CW-complex with one vertex and no cells of dim $ 1,\ldots,\ell$ for $ \ell < k$. For each $ \ell+1$ cell, $ e^{\ell+1}$, the attaching map is $ \partial e^{\ell+1} \xrightarrow{ f} X^{( \ell)} = \{e_0\} $. This attaches a $ \ell+1$-sphere to $ \wh{ X}$. So $ e^{\ell+1}$ is an element of $ \pi_{\ell+1}(\wh{ X})=0$, so there must exist a disk $ \alpha: D^{\ell+2} \to \wh{ X}$ s.t.\ $ \alpha( \partial D^{\ell+2}) = e^{\ell+1}$ ???. We can assume $ \alpha(D^{\ell+2}) \subseteq \wh{ X} ^{(\ell+2)}$ by cellular approximation. Now glue $ D^{\ell+3}$ to $ \wh{ X}$ by 
~\begin{figure}[H]
	\centering
	\includegraphics[width=0.8\textwidth]{./figures/kconnect_disk2.png}
\end{figure}
call result $ \widetilde{ X} := \wh{ X}$ with a $ \ell+2$ cell $ e$ and a  $ \ell+3$ cell $ e'$.

Since  $ e'$ is homotopic to  $ \overline{\partial e' - e}$ so $ \widetilde{ X} ' \simeq \wh{ X}$. Since $ e$ is contractible,  $ \wh{ X}' = \widetilde{ X}' / e \simeq \widetilde{ X}' \simeq \wh{ X}$. NOw $ \wh{ tX}'$ has one less $ \ell+1$ cells and we repeat to get rid of all of them.
\end{proof}

\begin{coro}
If $ X$ is a CW-complex with $ \pi_{i} (X) = 0 \ \forall \ i$, then $ X$ is contractible.
\end{coro}
\begin{proof}
If $ X$ is a finite dimensional CW-complex, then theorem above says  $ X \simeq \{ \text{pt} \} $. If $ X$ is infinite, use weak topology.
\end{proof}

\begin{coro}
If $ X$ is a  $ k$-connected CW-complex, then  $ \widetilde{ H}_\ell(X) =0 \ \forall \ \ell \leq k$.
\end{coro}
That is, $ \pi_{\ell}(X)=0$ for all $ \ell \leq k$ implies that $ \widetilde{ H}_\ell (X) =0 \ \forall \ \ell \leq k$. Recall that we remove a $ \zz$ from 0th homology to get reduced homology.

\begin{proof}
Compute $ \widetilde{ H}_\ell(X)$ using cellular homology. Recall $ C_\ell^{ \text{CW} }(X)$ is the free abelian group generated by the $ \ell$-cells. We can assume no $ \ell$-cells for $ \ell=1,\ldots,k$ and for $ \ell=0$. So $ H_\ell(X) = 0 \ \forall \ \ell=1,\ldots,k$. Also $ H_0(X) = \zz$ since it is path-connected so $ \widetilde{ H}_0(X) = 0$.
\end{proof}

\begin{thm}
If $ (X,A)$ is a CW pair and  $ \pi_n(X,A) = 0 \ \forall \ n $ then $ X$ deformation retracts to  $ A$,  \emph{i.e.} $ X \simeq A$.
\end{thm}

\begin{proof}
Exercise. Much like 21 and 22.
\end{proof}

\begin{thm}[Whitehead]
If $ X,Y$ are CW complexes, with base points $ x_0 \in X^{(0)}, y_0 \in Y^{(0)}$ with $ Y$ connected, and  $ f:(X,x_0) \to  (Y,y_0)$ is a map s.t.\ $ f_* \pi_k(X,x_0) \to \pi_k(Y,y_0)$ is an isomorphism for all $ k$, then  $ f: X \to Y$ is a homotopy equivalence.
\end{thm}

\begin{remark}
\begin{enumerate}[label=(\arabic*)]
	\item $ f$ satisfying the hypothesis is called a  \allbold{weak homotopy equivalence}. So theorem says for CW-complexes, a weak homotopy equivalence is a homotopy equivalence.
	\item 2 spaces can have isomorphic $ \pi_n \ \forall \ n$ but not be homotopy equivalence. We do need this map.
		\begin{eg}
		Let $ X = \rr P^2 \times S^3$, $ Y = S^2 \times \rr P^2$. Note $ S^2 \times S^3$ is the universal cover of $ X$ and  $ Y$, by lemma 18,  $ \pi_n(X) \cong\pi_n(S^2 \times S^3) \cong\pi_n(Y) \ \forall \ k\geq 2$. So $ \pi_1(X)= \zz /2 = \pi_1(Y)$. They are path-connected so they have isomorphic $ \pi_0$. But $ X$ is not homotopy equivalence to  $ Y$, because $ X$ is not orientable but $ Y$ is so $ H_5(X)=0, H_5(Y) \cong \zz$.
		\end{eg}
	\item If $ X,Y$ are not CW-complexes, then  $ f:X to Y$ inducing isomorphisms on all homotopy groups, then  $ f$ doesn't not need to be a homotopy equivalence. Consider topologist's comb and a point at the top of first bar.
\end{enumerate}
\end{remark}

\begin{proof}
Given $ f: X \to Y$ we can make it cellular, consider the mapping cylinder $ C_f= (X \times I) \sqcup Y /(x,0) \sim f(x) $. Exercise: $ C_f$ has the structure of a CW-complex where  $ X \times \{1\} $ is a subcomplex. Recall $ C_f \simeq Y$ given by $ j$ which has a homotopy inverse  $ i:Y \to C_f$. Let $ i_x: X \to C_f,x\mapsto (x,1)$, then $ j \circ i_X \simeq f$. Since $ f_*: \pi_n(X) \to \pi_n(Y)$ is an iso for all $ n$, so is  $ (i_X)_*$. By long exact sequence in lemma 17, 

By Theorem 24,  $ C_f \simeq X$.
\end{proof}

Let's go back to computing $ \pi_k$. Recall by lemma 10, $ \pi_1(X,x_0)$ acts on $ \pi_n(X,x_0)$. Given $ [ \gamma] \in \pi_1(X,x_0), [f] \in \pi_n(X,x_0)$. Define $ [ \gamma]. [f]$ by

Exercise: this makes $ \pi_n(X,x_0)$ into a $\zz [\pi_1(X,x_0)]$-module (group ring).

\begin{thm}
	Given $ (X,x_0)$, $ f: \partial D^{n}\to X$ a map s.t.\ $ f(y_0) = x_0$. Let $ \wh{ X}= X \cup _f D^{n}$. Let $ i: X \to \wh{ X}$ be inclusion. Then $ i_*: \pi_k(X,x_0) \to \pi_k( \wh{ X} ,x_0)$ is an isomorphism for  $ k<n-1$ and surjective for $ k=n-1$ with kernel generated by  $ [f]$ and  $ [ \gamma] . [f]$ for all $ [ \gamma] \in \pi_1(X,x_0)$.
\end{thm}

\begin{proof}
	Given $ g: S^{k} \to \wh{ X}$ s.t.\ $ [g] \in \pi_k(\wh{ X})$, we want to find an element in $ \pi_k(X)$ that maps to it. Consider $ \int(D^{n})$ this is a smooth open manifold. So $ g^{-1}(\int D^{n})$ is a smooth open submanifold of $ S^{k}$ (open subset of smooth manifold). We can homotop $ g|_{g^{-1}(\int D^{n})}$ to be smooth. Choose a regular value $ p$ of  $ g|_{g^{-1}(\int D^{n})}$ by Sard's Theorem. If $ k<n$ then  $ g^{-1}(p) = 0$ by dimension $ <0$. Since $ D^{n} - p$ deformation retracts to $ \partial D^{n}$, we can homotop $ g$ to $ \wh{ g}$ s.t.\ $ \im \wh{ g} \cap \int D^{n} = \O$. So $ \wh{ g} \in \pi_n(X)$ and $ i_*([\wh{ g}]) = [g]$. So $ i_*$ is surjective if  $ k \leq n-1$.

	Suppose  $ [g_0],[g_1] \in \pi_k(X)$ s.t.\ $ i_*([g_0]) = i_*([g_1])$, that is, there exists $ H: S^{k} \times I \to \wh{ X}$ between $  g_0$ and $ g_1$. Note $ S^{k} \times I$ is a smooth manifold of $ \dim k+1$. So if  $ k+1 \leq n-1$, then the argument above (for surjectivity) says we can homotop  $ H$ to  $ \wh{ H}$ s.t.\ $ \wh{ H}: S^{k} \times I \to X$ is a homotop of $ g_0$ to $ g_1$ in $ X$. So  $ i_*$ is injective for  $ k \leq n-2$. 

	Now for $ i_*: \pi_{n-1}(X) \to \pi_{n-1}(\wh{ X})$, clearly $ [f]$ and  $ [ \gamma] \cdot [f]$ are in $ \ker i_*$. So it remains to show $ [g] \in \ker i_*$ is in the subgroup generated by $ [f]$ and  $ [ \gamma] \cdot [f]$. We have $ G: D^{n} \to \wh{ X}$ s.t.\ $ G|_{\partial D^{n}} =g$. We can assume there exists $ p \in \int(D^{n})$ (the cell we added to get $ \wh{ X}$) s.t.\ $ G^{-1}(p) = \{p_1,\ldots,p_\ell\} $ by codimension. So there exists open balls $ N_i$ around $ p_i$ s.t.\ $ G|_{N_i}$ embeds $ N_i$ into $ \int D^{n}$. Note that $ G|_{D^{n} - \cup N_i}$ misses $ p$ so we can deformation retract to boundary, so homotopic to  $ G'$ with image in  $ X$ and each boundary component of  $ \partial [(D^{n}- \cup N_i)-\partial D^{n}]$ has image equal to $ f$. So there exists  $ p_i \in \partial N_i$ s.t.\ $ G'(p_i)=x_0$. Let $ \alpha_i: I \to D^{n}-\cup N_i$ be a path from $ p_i'$ to  $ x_0$.
\end{proof}

\begin{thm}
Any topological space is weakly homotopy equivalent to a CW-complex.
\end{thm}

\begin{proof}
Given a topological space $ X$, WLOG path-connected with base point $ x_0$,  set $ Y_0 = \{e^{0}\} $ and $ f_0: Y_0 \to X, e^{0} \mapsto x_0$. We see that $ f_0$ is an isomorphism on $ \pi_0$.

Let $ \alpha_1,\ldots, \alpha_k: I \to X$ generate $ \pi_1(X,x_0)$. Set $ Y_1 = Y_0 \cup e_1^{1} \cup \cdots \cup e_k^{1}$ which is a wedge of circles. Extend $ f_0$ to $ f_1': Y_1' \to X$ by $ \alpha_i$ on each $ e_i^{1}$. Clearly $ f_1'$ is an isomorphism on $ \pi_n$ for $ n<1$ and surjective on  $ \pi_1$. Let $ \beta_1, \beta_\ell$ generate $ \ker (f_1')_*$ on $ \pi_1$, \emph{i.e.} $ f_1' \circ \beta_i: I \to X$ are null-homotopic. So we have $ F_i: D^2 \to X$ s.t.\ $ F_2|_{ \partial D^2} = f_1' \circ \beta_i$ ??? Let $ Y_1 = Y_1' \cup \bigsqcup_{ i=1}^{\ell} \overline{e}_1^2$, glue $ \overline{e}_i^2$ to $ Y_1'$ by $ \beta_i$. Extend $ f_1'$ to $ f_1: Y_1 \to X$ by $ F_i$ on $ \overline{e}_i^2$.

Exercise: $ \pi_1(Y_1) \cong \pi_1(Y_1') / \langle \beta_1, \ldots, \beta_\ell \rangle$ so $ f_1 $ is an isomorphism on $ \pi_n$ for $ n \leq 1$.

Now let  $ \alpha_1,\ldots, \alpha_k: D^2 \to X$, generate $ \pi_2(X)$, $ Y_2' = Y_1 \cup  e_1^2 \cup \cdots \cup e_k^2$ which each 2-cell is attached by the constant map (a wedge of spheres). Extend $ f_1$ on $ Y_1$ to $ f_2': Y_2' \to X$ by $ \alpha_i$ on each $ e_i^2$. Clearly $ f_2'$ induces an isomorphism on $ \pi_n$ for $ n\leq 1$ and a surjection on  $ \pi_2$. Let $ \beta_1,\ldots, \beta_\ell$ generate $ \ker( f_2')_*$ on $ \pi_2(Y_2')$ (the smallest normal subgroup and $ \pi_1$ acting on it). As above $ f_2' \circ  \beta_i: D^2 \to X$ is null-homotopic, so there exists $ F_i: D^3 \to X$ s.t.\ $ F_i|_{ \partial D^3} = f_2' \circ \beta_i$. Set $ Y_2 = Y_2' \cup \bigsqcup_{ i=1}^{\ell} \overline{e}_i^3$ where $ \overline{e}_i^3$ is attached to $ Y_2'$ by $ \beta_i$. Extend $ f_2'$ to $ f_2: Y_2 to X$ by $ F_i$ on $ \overline{e}_i^3$ using Theorem 27. So $ f_2$ induces an isomorphism on $ \pi_n, n \leq 2$. The induction step is clear.
\end{proof}
\end{document}

