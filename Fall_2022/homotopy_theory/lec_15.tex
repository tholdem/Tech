\documentclass[12pt,class=article,crop=false]{standalone} 
\newcommand{\alert}[1]{{\bf \color{red} [Alert:] #1}}
\newcommand{\todo}[1]{{\bf \color{orange} [TODO:] #1}}
\newcommand{\real}[1][]{\mathbb{R}^{#1}}
\newcommand{\myeqn}[1]{(\ref{#1})}
\newcommand{\myex}[1]{Example \ref{#1}}
\newcommand{\defeq}{\stackrel{\mathrm{def}}{=}}
\newcommand{\parder}[2]{\frac{\partial #1}{\partial #2}}
\newcommand{\Lie}[3][]{\mathsf{L}_{#3}^{#1} #2}
\newcommand{\LieA}[1]{\mathsf{Lie}(#1)}
\newcommand{\lieder}[2]{\mathcal{L}_{#2} #1}
\renewcommand{\t}{^{\mbox{\tiny\sf T}}}
\newcommand{\trans}{^{\mbox{\tiny\sf T}}}
\newcommand{\markup}[1]{\{\textbf{#1}\}}
\newcommand{\msub}[1]{_\mathrm{#1}}
\newcommand{\msup}[1]{^\mathrm{#1}}
\newcommand{\inv}[1]{#1^{-1}}
\newcommand{\pinv}[1]{{#1}^{+}}
\newcommand{\myfracA}[2]{\displaystyle{\frac{#1}{#2}}}
\newcommand{\myfracB}[2]{{#1}/{#2}}
\newcommand{\mydiffA}[1]{\dot{#1}}
\newcommand{\mydiffB}[2]{\myfracA{\mathrm{d}{#1}}{\mathrm{d}{#2}}}
\newcommand{\ball}[2]{\mathcal{B}_{#1}\left(#2\right)}
\newcommand{\acos}[1]{\cos^{-1}\left(#1\right)}
\newcommand{\asin}[1]{\sin^{-1}\left(#1\right)}
\newcommand{\mani}{\mathcal{M}}
\newcommand{\tang}[2]{\mathsf{T}_{#1} #2}
\newcommand{\LieB}[2]{[ #1, #2 ]}
\newcommand{\LieBad}[3][]{\mathsf{ad}_{#2}^{#1} #3}
\newcommand{\ReachVT}{\mathcal{R}^V_T}
\newcommand{\ReachVt}{\mathcal{R}^V_t}
\newcommand{\ReachVTe}{\mathcal{R}^V_{\le T}}
\newcommand{\ReachT}{\mathcal{R}_T}
\newcommand{\Reacht}{\mathcal{R}_t}
\newcommand{\ReachTe}{\mathcal{R}_{\le T}}
\newcommand{\accLA}[1]{\mathsf{Lie}(#1)}
\newcommand{\accD}{\Delta_{\mathcal{F}}}
\newcommand{\accSA}{\mathsf{Lie}(\mathcal{G},f)}
\newcommand{\accDS}{\Delta_{\mathcal{G}}}
\newcommand{\eval}[3]{\mathsf{Ev}^{#2}_{#1}\left( #3 \right)}
\newcommand{\stlc}{\textsc{stlc}}
\newcommand{\clf}{\textsc{clf}}
\newcommand{\jqlf}{\textsc{jqlf}}
\newcommand{\dlas}{\textsc{dlas}}
\newcommand{\Ad}[2]{\mathsf{Ad}_{#1} #2}
\newcommand{\xe}{\ensuremath{x_e}}
\newcommand{\lebg}[1]{\mathcal{L}_{#1}}
\newcommand{\lebgx}[1]{\mathcal{L}_{#1 \mathrm{e}}}
\newcommand{\dom}{D}
\newcommand{\domT}{[t_0,\infty) \times D}
\newcommand{\rarrow}{\rightarrow}
\renewcommand{\d}{\mathrm{d}}
\renewcommand{\Re}{\mathbb{R}}
\newcommand{\C}{\mathrm{C}}

\newcommand{\QED}{{\unskip\nobreak\hfil\penalty50\hskip2em\vadjust{}
		\nobreak\hfil$\Box$\parfillskip=0pt\finalhyphendemerits=0\par}\vspace{0.1cm}}
\newcommand{\eoEx}{{\unskip\nobreak\hfil\penalty50\hskip0em\vadjust{}
		\nobreak\hfil$\Large\Diamond$\parfillskip=0pt\finalhyphendemerits=0\par}\vspace{0.1cm}}

\newcommand{\sgn}{\ensuremath{\operatorname{sgn}}}
\newcommand{\sat}{\ensuremath{\operatorname{sat}}}

\newcommand{\half}{\frac{1}{2}}
\newcommand{\shalf}{\mbox{$\frac{1}{2}$}}
\newcommand{\marcom}[1]{\marginpar{\footnotesize #1}}
\newcommand{\der}{\mathrm{D}}
\newcommand{\e}{\mathrm{e}}
\newcommand{\dt}{\mathrm{d}t}

\newcommand{\cA}{\ensuremath{\mathcal{A}}}
\newcommand{\cB}{\ensuremath{\mathcal{B}}}
\newcommand{\cG}{\ensuremath{\mathcal{G}}}
\newcommand{\cK}{\ensuremath{\mathcal{K}}}
\newcommand{\cW}{\ensuremath{\mathcal{W}}}
\newcommand{\cZ}{\ensuremath{\mathcal{Z}}}
\newcommand{\cS}{\ensuremath{\mathcal{S}}}
\newcommand{\cD}{\ensuremath{\mathcal{D}}}
\newcommand{\cP}{\ensuremath{\mathcal{P}}}
\newcommand{\cV}{\ensuremath{\mathcal{V}}}
\newcommand{\cL}{\ensuremath{\mathcal{L}}}
\newcommand{\cN}{\ensuremath{\mathcal{N}}}
\newcommand{\cI}{\ensuremath{\mathcal{I}}}
\newcommand{\cR}{\ensuremath{\mathcal{R}}}
\newcommand{\cM}{\ensuremath{\mathcal{M}}}
\newcommand{\cC}{\ensuremath{\mathcal{C}}}
\newcommand{\cF}{\ensuremath{\mathcal{F}}}
\newcommand{\cH}{\ensuremath{\mathcal{H}}}
\newcommand{\cO}{\ensuremath{\mathcal{O}}}
\newcommand{\cX}{\ensuremath{\mathcal{X}}}
\newcommand{\cY}{\ensuremath{\mathcal{Y}}}
\newcommand{\Ci}{\ensuremath{\mathcal{C}^\infty}}
\newcommand{\ISS}{\textsc{iss}}
\newcommand{\LISS}{\textsc{liss}}
\newcommand{\GAS}{\textsc{gas}}
\newcommand{\GS}{\textsc{gs}}
\newcommand{\LES}{\textsc{les}}
\newcommand{\GUAS}{\textsc{guas}}
\newcommand{\BIBO}{\textsc{bibo}}
\newcommand{\spec}{\ensuremath{\operatorname{spec}}}
\newcommand{\spn}{\ensuremath{\operatorname{span}}}
\renewcommand{\i}{\mathrm{i\,}}

\renewcommand{\implies}{\Rightarrow}

\renewcommand{\theenumi}{$\roman{enumi})$}
\renewcommand{\labelenumi}{\theenumi}

\font\ptmten=zptmcmrm scaled 1200
\newcommand{\w}{\mbox{{\ptmten w}}}
\newcommand{\z}{\mbox{{\ptmten z}}}
\renewcommand{\Re}{\mathbb{R}}

\newcommand{\cl}{\operatorname{cl}}
\newcommand{\intr}{\operatorname{int}}
\newcommand{\rank}{\operatorname{rank}}
\newcommand{\co}{\operatorname{co}}
\newcommand{\aff}{\operatorname{aff}}

\theoremstyle{plain}
\newtheorem{theorem}{Theorem}[chapter]
\newtheorem{claim}[theorem]{Claim}
\newtheorem{corollary}[theorem]{Corollary}
\newtheorem{prop}[theorem]{Proposition}
\newtheorem{fact}[theorem]{Fact}
\newtheorem{lemma}[theorem]{Lemma}

\newtheorem{remark}{Remark}[chapter]

\theoremstyle{definition}
\newtheorem{assume}[theorem]{Assumption}
\newtheorem{defn}[theorem]{Definition}
\newtheorem{problem}[theorem]{Problem}
\newtheorem{exercise}{Exercise}
\newtheorem{example}[theorem]{Example}


\begin{document}
\section{Fibrations}
\begin{thm}[Leray-Serre spectral sequence]
If $ E \xrightarrow{ p} B $ is a Serre fibration and $ B$ is a simply connected CW complex (only need $ \pi_1(B)$ acts trivially on $ H_*(F)$). Then there exists a spectral sequence that converges to $ G(H_*(E))$ with  $ E_{s,t}^2 = H_s(B;H_t(F))$.
\end{thm}
\begin{remark}
There is a similar cohomology spectral sequence with $ E_2^{s,t} = H^{s}(B;H^{t}(F))$.
\end{remark}
\begin{proof}
Exercise: show $ d^{1}$ is the boundary map for chain complex $ C_*^{CW}(B; H_t(F))$.
\end{proof}
\begin{lem}
$H_{s+t}(E^{s} , E^{s-1}) \cong C_s(B;H_t(F))$.
\end{lem}
\begin{proof}
Exercise: for $ p>0$,  $ H_p(Y)  \cong H_{p+1}(\Sigma Y)$. 

Recall $ \Sigma Y \cong S^{1} \wedge Y = S^{1} \times Y / S^{1} \vee Y$. Also $ S^{p} = S^{1} \wedge \cdots \wedge S^{1}$ $ p$ times. There exists an $ S^{s}$  in $ D^{s} \times F /S^{s-1} \times F$ by taking a point in $ F$.

Exercise: for any spaces  $ X,Y,Z$ with  $ Y \subseteq x$
\begin{align*}
	\frac{X\times Z}{ (Y \times Z) \cup (X \times \{*\} )} \cong \frac{X /Y \times Z}{(\{*\} \times Z ) \vee (X \times \{*\} / Y \times \{*\} ) }.
\end{align*}
Exercise: think about $ t=0,1$.


\end{proof}
\begin{thm}
For $ k \geq 2$,
\begin{align*}
	H_q(\Omega S^{k}) = \begin{cases}
		\zz& q = a(k-1) , a \geq 0\\
		0& \text{ else}\\ 
	\end{cases}
\end{align*}
\end{thm}
\begin{proof}
Apply Theorem 4 and Lemma 1.
\end{proof}

\begin{thm}[Gysin sequence]
Let $ E \xrightarrow{ p} B $ be a fibration with fiber $ S^{n}$ and $ B$ a CW complex. Assume $ \pi_1(B)$ acts trivially on $ H_*(S^{n})$, there exists an exact sequence
\begin{align*}
	\ldots H_r(E) \xrightarrow{ p_*} H_r(B) \to H_{r-n-1} (B) \to H_{r-1} (E) \xrightarrow{ p_*} H_{r-1}(B) \ldots  
\end{align*}
for $ k \geq n+1$.
\end{thm}
\begin{proof}
Apply Theorem 4 and Lemma 1.
\end{proof}

Exercise: If $ E\xrightarrow{ p} S^{n}$ is a fibration with fiber $ F$, show there exists an exact sequence
 \begin{align*}
	\ldots H_r(F) \to H_r(E) \to H_{r-n}(F) \to H_{r-1}(F) \to\ldots
\end{align*}
called the Wang sequence.

Let's consider a cohomology version:
\begin{thm}[Leray-Serre for cohomology]
Let $ E\xrightarrow{ p} B$ be a Serre fibration with $ B$ a CW complex where  $ \pi_1(B)$ acts trivially on  $ H^{*}(F)$. There exists a spectral sequence converging to $ G(H^* (F))^{s,t}$ with $ E_2^{s,t} = H^{s}(B;H^{t}(F))$ and
\begin{enumerate}[label=(\arabic*)]
	\item $ \{E_r^{s,t}\} $ is a bigraded algebra, \emph{i.e.} there exists a product $ E_r^{s,t} \times E_r^{p,q} \to E_r^{s+p,t+q}$ 
	\item $ d_r: E_r \to E_r$ is a derivation $ (r,-r+1)$, \emph{i.e.}
		\begin{align*}
			d_r(a \cdot b) = (d_r a) \cdot b + (-1)^{p+q} a \cdot d_r b
		\end{align*}
	\item $ E_2^{*,0} \cong H^* (B)$ as rings and $ E_2^{0,*} \cong H^* (F)$ as rings.
\end{enumerate}
\end{thm}
\begin{remark}
The product structure on $ E_2^{s,t}$ is
\begin{align*}
	H^{p}(B; H^{q}(F)) \times H^{s}(B; H^{t}(F)) \to H^{p+s}(B; H^{q}(F) \otimes H^{t}(F))
\end{align*}
and compose with the cup product on $ H^{q}(F) \otimes H^{t}(F)$.
\end{remark}

\begin{eg}
$ \cc P^n$.
\end{eg}
\begin{thm}
$ H^* (U(n)) \cong \Lambda (x_1,x_3,\ldots,x_{2n+1})$ with $ \deg x_i = i$.
\end{thm}
\begin{remark}
	From this we can compate $ H^* (BU(n)) \cong \zz[c_1,\ldots,c_n]$ where $ c_i$ has degree $ 2i$ (this is Theorem II.17).
\end{remark}
\end{document}
