\documentclass[12pt,class=article,crop=false]{standalone} 
\newcommand{\alert}[1]{{\bf \color{red} [Alert:] #1}}
\newcommand{\todo}[1]{{\bf \color{orange} [TODO:] #1}}
\newcommand{\real}[1][]{\mathbb{R}^{#1}}
\newcommand{\myeqn}[1]{(\ref{#1})}
\newcommand{\myex}[1]{Example \ref{#1}}
\newcommand{\defeq}{\stackrel{\mathrm{def}}{=}}
\newcommand{\parder}[2]{\frac{\partial #1}{\partial #2}}
\newcommand{\Lie}[3][]{\mathsf{L}_{#3}^{#1} #2}
\newcommand{\LieA}[1]{\mathsf{Lie}(#1)}
\newcommand{\lieder}[2]{\mathcal{L}_{#2} #1}
\renewcommand{\t}{^{\mbox{\tiny\sf T}}}
\newcommand{\trans}{^{\mbox{\tiny\sf T}}}
\newcommand{\markup}[1]{\{\textbf{#1}\}}
\newcommand{\msub}[1]{_\mathrm{#1}}
\newcommand{\msup}[1]{^\mathrm{#1}}
\newcommand{\inv}[1]{#1^{-1}}
\newcommand{\pinv}[1]{{#1}^{+}}
\newcommand{\myfracA}[2]{\displaystyle{\frac{#1}{#2}}}
\newcommand{\myfracB}[2]{{#1}/{#2}}
\newcommand{\mydiffA}[1]{\dot{#1}}
\newcommand{\mydiffB}[2]{\myfracA{\mathrm{d}{#1}}{\mathrm{d}{#2}}}
\newcommand{\ball}[2]{\mathcal{B}_{#1}\left(#2\right)}
\newcommand{\acos}[1]{\cos^{-1}\left(#1\right)}
\newcommand{\asin}[1]{\sin^{-1}\left(#1\right)}
\newcommand{\mani}{\mathcal{M}}
\newcommand{\tang}[2]{\mathsf{T}_{#1} #2}
\newcommand{\LieB}[2]{[ #1, #2 ]}
\newcommand{\LieBad}[3][]{\mathsf{ad}_{#2}^{#1} #3}
\newcommand{\ReachVT}{\mathcal{R}^V_T}
\newcommand{\ReachVt}{\mathcal{R}^V_t}
\newcommand{\ReachVTe}{\mathcal{R}^V_{\le T}}
\newcommand{\ReachT}{\mathcal{R}_T}
\newcommand{\Reacht}{\mathcal{R}_t}
\newcommand{\ReachTe}{\mathcal{R}_{\le T}}
\newcommand{\accLA}[1]{\mathsf{Lie}(#1)}
\newcommand{\accD}{\Delta_{\mathcal{F}}}
\newcommand{\accSA}{\mathsf{Lie}(\mathcal{G},f)}
\newcommand{\accDS}{\Delta_{\mathcal{G}}}
\newcommand{\eval}[3]{\mathsf{Ev}^{#2}_{#1}\left( #3 \right)}
\newcommand{\stlc}{\textsc{stlc}}
\newcommand{\clf}{\textsc{clf}}
\newcommand{\jqlf}{\textsc{jqlf}}
\newcommand{\dlas}{\textsc{dlas}}
\newcommand{\Ad}[2]{\mathsf{Ad}_{#1} #2}
\newcommand{\xe}{\ensuremath{x_e}}
\newcommand{\lebg}[1]{\mathcal{L}_{#1}}
\newcommand{\lebgx}[1]{\mathcal{L}_{#1 \mathrm{e}}}
\newcommand{\dom}{D}
\newcommand{\domT}{[t_0,\infty) \times D}
\newcommand{\rarrow}{\rightarrow}
\renewcommand{\d}{\mathrm{d}}
\renewcommand{\Re}{\mathbb{R}}
\newcommand{\C}{\mathrm{C}}

\newcommand{\QED}{{\unskip\nobreak\hfil\penalty50\hskip2em\vadjust{}
		\nobreak\hfil$\Box$\parfillskip=0pt\finalhyphendemerits=0\par}\vspace{0.1cm}}
\newcommand{\eoEx}{{\unskip\nobreak\hfil\penalty50\hskip0em\vadjust{}
		\nobreak\hfil$\Large\Diamond$\parfillskip=0pt\finalhyphendemerits=0\par}\vspace{0.1cm}}

\newcommand{\sgn}{\ensuremath{\operatorname{sgn}}}
\newcommand{\sat}{\ensuremath{\operatorname{sat}}}

\newcommand{\half}{\frac{1}{2}}
\newcommand{\shalf}{\mbox{$\frac{1}{2}$}}
\newcommand{\marcom}[1]{\marginpar{\footnotesize #1}}
\newcommand{\der}{\mathrm{D}}
\newcommand{\e}{\mathrm{e}}
\newcommand{\dt}{\mathrm{d}t}

\newcommand{\cA}{\ensuremath{\mathcal{A}}}
\newcommand{\cB}{\ensuremath{\mathcal{B}}}
\newcommand{\cG}{\ensuremath{\mathcal{G}}}
\newcommand{\cK}{\ensuremath{\mathcal{K}}}
\newcommand{\cW}{\ensuremath{\mathcal{W}}}
\newcommand{\cZ}{\ensuremath{\mathcal{Z}}}
\newcommand{\cS}{\ensuremath{\mathcal{S}}}
\newcommand{\cD}{\ensuremath{\mathcal{D}}}
\newcommand{\cP}{\ensuremath{\mathcal{P}}}
\newcommand{\cV}{\ensuremath{\mathcal{V}}}
\newcommand{\cL}{\ensuremath{\mathcal{L}}}
\newcommand{\cN}{\ensuremath{\mathcal{N}}}
\newcommand{\cI}{\ensuremath{\mathcal{I}}}
\newcommand{\cR}{\ensuremath{\mathcal{R}}}
\newcommand{\cM}{\ensuremath{\mathcal{M}}}
\newcommand{\cC}{\ensuremath{\mathcal{C}}}
\newcommand{\cF}{\ensuremath{\mathcal{F}}}
\newcommand{\cH}{\ensuremath{\mathcal{H}}}
\newcommand{\cO}{\ensuremath{\mathcal{O}}}
\newcommand{\cX}{\ensuremath{\mathcal{X}}}
\newcommand{\cY}{\ensuremath{\mathcal{Y}}}
\newcommand{\Ci}{\ensuremath{\mathcal{C}^\infty}}
\newcommand{\ISS}{\textsc{iss}}
\newcommand{\LISS}{\textsc{liss}}
\newcommand{\GAS}{\textsc{gas}}
\newcommand{\GS}{\textsc{gs}}
\newcommand{\LES}{\textsc{les}}
\newcommand{\GUAS}{\textsc{guas}}
\newcommand{\BIBO}{\textsc{bibo}}
\newcommand{\spec}{\ensuremath{\operatorname{spec}}}
\newcommand{\spn}{\ensuremath{\operatorname{span}}}
\renewcommand{\i}{\mathrm{i\,}}

\renewcommand{\implies}{\Rightarrow}

\renewcommand{\theenumi}{$\roman{enumi})$}
\renewcommand{\labelenumi}{\theenumi}

\font\ptmten=zptmcmrm scaled 1200
\newcommand{\w}{\mbox{{\ptmten w}}}
\newcommand{\z}{\mbox{{\ptmten z}}}
\renewcommand{\Re}{\mathbb{R}}

\newcommand{\cl}{\operatorname{cl}}
\newcommand{\intr}{\operatorname{int}}
\newcommand{\rank}{\operatorname{rank}}
\newcommand{\co}{\operatorname{co}}
\newcommand{\aff}{\operatorname{aff}}

\theoremstyle{plain}
\newtheorem{theorem}{Theorem}[chapter]
\newtheorem{claim}[theorem]{Claim}
\newtheorem{corollary}[theorem]{Corollary}
\newtheorem{prop}[theorem]{Proposition}
\newtheorem{fact}[theorem]{Fact}
\newtheorem{lemma}[theorem]{Lemma}

\newtheorem{remark}{Remark}[chapter]

\theoremstyle{definition}
\newtheorem{assume}[theorem]{Assumption}
\newtheorem{defn}[theorem]{Definition}
\newtheorem{problem}[theorem]{Problem}
\newtheorem{exercise}{Exercise}
\newtheorem{example}[theorem]{Example}


\begin{document}
How does $ [X, Y]_0$ depend on the base point?
\begin{defn}
Given $ f_0, f_1: X \to Y$ and a path $ u : I \to Y$, if there is a homotopy $ H: X \times I \to Y$ s.t.\ $ H(x,0) = f_0(x)$, $ H(x,1) = f_1(x)$, and $ H(x_0,t) = u(t)$, then we say $ H$ is a  \allbold{homotopy along $ u$}, denote $ f_0 \simeq_u f_1$. 
\end{defn}

If $ f_0, f_1$ are base-point preserving, then $ u$ is a loop in  $ Y$. To be able to say anything about moving base point we read  $ (X,x_0)$ to be ``non-degenerate" by which we mean that  $ (X,x_0)$ is an \allbold{NDR-pair} (neighborhood deformation retract).

\begin{defn}
$ A \subseteq X$ is an \allbold{NDR-pair} if there are maps $ u: X \to I$ and $ h:X \times I \to X$ s.t.\
\begin{enumerate}[label=(\arabic*)]
	\item $ A= u ^{-1}(0)$;
	\item $ h(x,0) = x \ \forall \ x \in X$;
	\item $ h(a,t) = a \ \forall \ a \in A$;
	\item $ h(x,1) \in A \ \forall \ x \in X$ with $ u(x)<1$.
\end{enumerate}
\end{defn}
Note: $ u^{-1}([0,1))$ is an open neighborhood of $ A \in X$ that retracts to $ A$.

\begin{eg}[NDR-pairs]
We have
\begin{enumerate}[label=(\arabic*)]
	\item sub CW-complex of a CW-complex (think that sphere can contract to boundary if we remove a point).
	\item submanifold of a manifold.
\end{enumerate}
\end{eg}

\begin{lem}
If $ (X,A)$ is an NDR-pair, then  $ (X \times \{0\}) \cup  (A \times I) $ is a retract of $ X \times I$.
\end{lem}

\begin{proof}
Define $ R: X \times I \to (X \times \{0\}) \cup (A \times I), (x,t) \mapsto \begin{cases}
	(x,t) & x \in A \text{ or } t=0\\
	(h(x,1),t - u(x) ) & t \geq u(x), t>0\\
	(h(x,\frac{t}{u(x)}),0) & u(x) \geq t \text{ and } u(x)>0 
\end{cases} $
Exercise: this is a retract.
\end{proof}

\begin{lem}
If $ (X,x_0)$ is an NDR-pair and  $ f_0: X \to Y$, $ f_0(x_0)=y_0$, $ \gamma: I \to Y$ path from $ y_0$ to $ y$, then there exists  $ f_1: X \to Y$ s.t.\  $ f_1(x_0) = y_1$ and $ f_0 \simeq_\gamma f_1$. We denote $ f_1$ by $ \gamma \cdot f_0$ (well-defined once $ R$ from previous lemma is fixed).
\end{lem}
\begin{proof}
Let $ R$ be from previous lemma for  $ (X,x_0)$. Let $ H : X \times I \to Y$ to be
\begin{align*}
	H(x,t) = \begin{cases}
		f_0(R(x,t)) & R(x,t) \in X \times \{0\} \\
		\gamma(R(x,t)) & R(x,t) \in A \times I\\
	\end{cases}
\end{align*}
	Let $ f_1(x) = H(x,1)$. Then $ H$ yields  $ f_0 \simeq _{ \gamma} f_1$.
\end{proof}

\begin{lem}
Suppose $  f_0,f_1,f_2: X \to Y, (X,x_0)$ an NDR pair if  $ f_0 \simeq _{ \gamma} f_1, f_0 \simeq _{ \gamma'} f_2$ with $ \gamma \simeq \gamma'$ rel boundary, then $ f_1 \simeq f_2$ rel base point.
\end{lem}
\begin{proof}
Since $ (X,x_0)$ is an NDR-pair, we can show that $ (X \times I, (X \times \{0,1\} \cup \{x_0\} \times I ))$ is also an NDR-pair. Exercise: show this ($ \epsilon$ tubes retract, just need to be compatible on overlaps). So lemma yields a retraction $ R: (X \times I) \times I \to ((X \times I) \times \{0\}) \cup ((X \times \{0,1\}) \cup (\{x_0\} \times I  )) \times I$.

Let $ H$ be homotopy  $ f_0 \simeq_{ \gamma} f_1$, $ G$ be homotopy  $ f_0 \simeq_{ \gamma'} f_2$, $ K$ be homotopy  $ \gamma \simeq \gamma'$ rel boundary. Let $ \overline{R} = () \circ R$. Check $ \overline{R}|_{ X \times I \times \{1\} }$ is a homotopy $ f_1$ to $ f_2$ rel base point.
\end{proof}

\begin{lem}
Suppose $ f_0,f_1, f_2 : X \to Y$, $ f_0 \simeq_{ \gamma_1} f_1$, $ f1 \simeq_{ \gamma_2} f_2$ with $ \gamma_1(1) = \gamma_2(0)$, then $ f_0 \simeq_{ \gamma_1 \times \gamma_2} f_2$.
\end{lem}
\begin{proof}
Concatenate homotopies, $ H $ for $ f_0 \simeq _{ \gamma_1} f_1$ and $ G$ for  $ f_1 \simeq _{ \gamma_2} f_2$.
\end{proof}

\begin{thm}
	If $ x_0$ is a non-degenerate base point of $ X$, then  $ \pi_1(Y,y_0)$ acts on $ [X, Y]_0$. Moreover, $ [X, Y]$ is the quotient of $ [X, Y]_0$ by the $ \pi_1(Y,y_0)$ action if $ Y$ is path-connected.
\end{thm}

\begin{proof}
	Take $ [ \gamma] \in \pi_1(Y,y_0), [f] \in [X, Y]_0$, lemma yields $ \gamma \cdot f: X \to Y$ and $ [ \gamma \cdot f]$ clearly $ [X, Y]_0$.
	\begin{claim}
		$ [ \gamma \cdot f]$ is well-defined. 
	\end{claim}
		If $ f,g \in [f]$ so $ f \simeq g$ rel base point. By lemma, we get $ f,g$  s.t.\ $ f \simeq _{ \gamma} f_1$ and $ g \simeq _{ \gamma} g_1$. Thus $ f_1 \simeq _{ \gamma ^{-1}} f \simeq g \simeq _{ \gamma} g_1$. Lemma says $ f_1 \simeq \gamma^{-1} \times \text{const} \times \gamma g_1 \simeq \text{const}  $. So Lemma 11 says $ f_1 \simeq g_1$ rel base points.

		That is, $ [ \gamma \cdot f]$ does not depend on choice of $ f$. By lemma 11,  $ [ \gamma \cdot f]$ does not depend on $ \gamma$. Let $ \Phi: [X, Y]_0 \to [X, Y]$ by just forgetting base point. Clearly $ \Phi( [ \gamma] \cdot [ f]) = \Phi([f])$. Hence we obtain an induced map
		\begin{align*}
			\Phi: [X, Y]_0 / \pi_1(Y,y_0) \to [X, Y].
		\end{align*}
\end{proof}
If $ \Phi([f]) = \Phi([g])$, then let $ H$ be the free homotopy from  $ f$ to  $ g$. Set  $ \gamma(t) = H(x_0,t)$. Both $ f,g$ take  $ x_0$ to $ y_0$ so $ [ \gamma] \in \pi_1(Y,y_0)$ and $ [ \gamma \cdot f] = [g]$. So $ \Phi$ is injective.

$ \Phi$ is surjective by lemma 10 if $ Y$ is path-connected.

\begin{coro}
A based map is null-homotopic iff it is based null-homotopic.
\end{coro}
\begin{proof}
Based null-homotopy clearly implies null-homotopic. If $ f \simeq e$, WLOG $ e(x)=y_0$, by homotopy $ H$, let  $ \gamma(t) = H(x_0,t)$ so $ f \simeq _{ \gamma} e$ so $ e \simeq _{ \gamma ^{-1}} f$. By lemma 11, $ f \simeq e$ rel base point.
\end{proof}
\end{document}
