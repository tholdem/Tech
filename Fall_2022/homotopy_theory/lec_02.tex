\documentclass[12pt,class=article,crop=false]{standalone} 
%Fall 2022
% Some basic packages
\usepackage{standalone}[subpreambles=true]
\usepackage[utf8]{inputenc}
\usepackage[T1]{fontenc}
\usepackage{textcomp}
\usepackage[english]{babel}
\usepackage{url}
\usepackage{graphicx}
%\usepackage{quiver}
\usepackage{float}
\usepackage{enumitem}
\usepackage{lmodern}
\usepackage{comment}
\usepackage{hyperref}
\usepackage[usenames,svgnames,dvipsnames]{xcolor}
\usepackage[margin=1in]{geometry}
\usepackage{pdfpages}

\pdfminorversion=7

% Don't indent paragraphs, leave some space between them
\usepackage{parskip}

% Hide page number when page is empty
\usepackage{emptypage}
\usepackage{subcaption}
\usepackage{multicol}
\usepackage[b]{esvect}

% Math stuff
\usepackage{amsmath, amsfonts, mathtools, amsthm, amssymb}
\usepackage{bbm}
\usepackage{stmaryrd}
\allowdisplaybreaks

% Fancy script capitals
\usepackage{mathrsfs}
\usepackage{cancel}
% Bold math
\usepackage{bm}
% Some shortcuts
\newcommand{\rr}{\ensuremath{\mathbb{R}}}
\newcommand{\zz}{\ensuremath{\mathbb{Z}}}
\newcommand{\qq}{\ensuremath{\mathbb{Q}}}
\newcommand{\nn}{\ensuremath{\mathbb{N}}}
\newcommand{\ff}{\ensuremath{\mathbb{F}}}
\newcommand{\cc}{\ensuremath{\mathbb{C}}}
\newcommand{\ee}{\ensuremath{\mathbb{E}}}
\newcommand{\hh}{\ensuremath{\mathbb{H}}}
\renewcommand\O{\ensuremath{\emptyset}}
\newcommand{\norm}[1]{{\left\lVert{#1}\right\rVert}}
\newcommand{\dbracket}[1]{{\left\llbracket{#1}\right\rrbracket}}
\newcommand{\ve}[1]{{\bm{#1}}}
\newcommand\allbold[1]{{\boldmath\textbf{#1}}}
\DeclareMathOperator{\lcm}{lcm}
\DeclareMathOperator{\im}{im}
\DeclareMathOperator{\coim}{coim}
\DeclareMathOperator{\dom}{dom}
\DeclareMathOperator{\tr}{tr}
\DeclareMathOperator{\rank}{rank}
\DeclareMathOperator*{\var}{Var}
\DeclareMathOperator*{\ev}{E}
\DeclareMathOperator{\dg}{deg}
\DeclareMathOperator{\aff}{aff}
\DeclareMathOperator{\conv}{conv}
\DeclareMathOperator{\inte}{int}
\DeclareMathOperator*{\argmin}{argmin}
\DeclareMathOperator*{\argmax}{argmax}
\DeclareMathOperator{\graph}{graph}
\DeclareMathOperator{\sgn}{sgn}
\DeclareMathOperator*{\Rep}{Rep}
\DeclareMathOperator{\Proj}{Proj}
\DeclareMathOperator{\mat}{mat}
\DeclareMathOperator{\diag}{diag}
\DeclareMathOperator{\aut}{Aut}
\DeclareMathOperator{\gal}{Gal}
\DeclareMathOperator{\inn}{Inn}
\DeclareMathOperator{\edm}{End}
\DeclareMathOperator{\Hom}{Hom}
\DeclareMathOperator{\ext}{Ext}
\DeclareMathOperator{\tor}{Tor}
\DeclareMathOperator{\Span}{Span}
\DeclareMathOperator{\Stab}{Stab}
\DeclareMathOperator{\cont}{cont}
\DeclareMathOperator{\Ann}{Ann}
\DeclareMathOperator{\Div}{div}
\DeclareMathOperator{\curl}{curl}
\DeclareMathOperator{\nat}{Nat}
\DeclareMathOperator{\gr}{Gr}
\DeclareMathOperator{\vect}{Vect}
\DeclareMathOperator{\id}{id}
\DeclareMathOperator{\Mod}{Mod}
\DeclareMathOperator{\sign}{sign}
\DeclareMathOperator{\Surf}{Surf}
\DeclareMathOperator{\fcone}{fcone}
\DeclareMathOperator{\Rot}{Rot}
\DeclareMathOperator{\grad}{grad}
\DeclareMathOperator{\atan2}{atan2}
\DeclareMathOperator{\Ric}{Ric}
\let\vec\relax
\DeclareMathOperator{\vec}{vec}
\let\Re\relax
\DeclareMathOperator{\Re}{Re}
\let\Im\relax
\DeclareMathOperator{\Im}{Im}
% Put x \to \infty below \lim
\let\svlim\lim\def\lim{\svlim\limits}

%wide hat
\usepackage{scalerel,stackengine}
\stackMath
\newcommand*\wh[1]{%
\savestack{\tmpbox}{\stretchto{%
  \scaleto{%
    \scalerel*[\widthof{\ensuremath{#1}}]{\kern-.6pt\bigwedge\kern-.6pt}%
    {\rule[-\textheight/2]{1ex}{\textheight}}%WIDTH-LIMITED BIG WEDGE
  }{\textheight}% 
}{0.5ex}}%
\stackon[1pt]{#1}{\tmpbox}%
}
\parskip 1ex

%Make implies and impliedby shorter
\let\implies\Rightarrow
\let\impliedby\Leftarrow
\let\iff\Leftrightarrow
\let\epsilon\varepsilon

% Add \contra symbol to denote contradiction
\usepackage{stmaryrd} % for \lightning
\newcommand\contra{\scalebox{1.5}{$\lightning$}}

% \let\phi\varphi

% Command for short corrections
% Usage: 1+1=\correct{3}{2}

\definecolor{correct}{HTML}{009900}
\newcommand\correct[2]{\ensuremath{\:}{\color{red}{#1}}\ensuremath{\to }{\color{correct}{#2}}\ensuremath{\:}}
\newcommand\green[1]{{\color{correct}{#1}}}

% horizontal rule
\newcommand\hr{
    \noindent\rule[0.5ex]{\linewidth}{0.5pt}
}

% hide parts
\newcommand\hide[1]{}

% si unitx
\usepackage{siunitx}
\sisetup{locale = FR}

%allows pmatrix to stretch
\makeatletter
\renewcommand*\env@matrix[1][\arraystretch]{%
  \edef\arraystretch{#1}%
  \hskip -\arraycolsep
  \let\@ifnextchar\new@ifnextchar
  \array{*\c@MaxMatrixCols c}}
\makeatother

\renewcommand{\arraystretch}{0.8}

\renewcommand{\baselinestretch}{1.5}

\usepackage{graphics}
\usepackage{epstopdf}

\RequirePackage{hyperref}
%%
%% Add support for color in order to color the hyperlinks.
%% 
\hypersetup{
  colorlinks = true,
  urlcolor = blue,
  citecolor = blue
}
%%fakesection Links
\hypersetup{
    colorlinks,
    linkcolor={red!50!black},
    citecolor={green!50!black},
    urlcolor={blue!80!black}
}
%customization of cleveref
\RequirePackage[capitalize,nameinlink]{cleveref}[0.19]

% Per SIAM Style Manual, "section" should be lowercase
\crefname{section}{section}{sections}
\crefname{subsection}{subsection}{subsections}
\Crefname{section}{Section}{Sections}
\Crefname{subsection}{Subsection}{Subsections}

% Per SIAM Style Manual, "Figure" should be spelled out in references
\Crefname{figure}{Figure}{Figures}

% Per SIAM Style Manual, don't say equation in front on an equation.
\crefformat{equation}{\textup{#2(#1)#3}}
\crefrangeformat{equation}{\textup{#3(#1)#4--#5(#2)#6}}
\crefmultiformat{equation}{\textup{#2(#1)#3}}{ and \textup{#2(#1)#3}}
{, \textup{#2(#1)#3}}{, and \textup{#2(#1)#3}}
\crefrangemultiformat{equation}{\textup{#3(#1)#4--#5(#2)#6}}%
{ and \textup{#3(#1)#4--#5(#2)#6}}{, \textup{#3(#1)#4--#5(#2)#6}}{, and \textup{#3(#1)#4--#5(#2)#6}}

% But spell it out at the beginning of a sentence.
\Crefformat{equation}{#2Equation~\textup{(#1)}#3}
\Crefrangeformat{equation}{Equations~\textup{#3(#1)#4--#5(#2)#6}}
\Crefmultiformat{equation}{Equations~\textup{#2(#1)#3}}{ and \textup{#2(#1)#3}}
{, \textup{#2(#1)#3}}{, and \textup{#2(#1)#3}}
\Crefrangemultiformat{equation}{Equations~\textup{#3(#1)#4--#5(#2)#6}}%
{ and \textup{#3(#1)#4--#5(#2)#6}}{, \textup{#3(#1)#4--#5(#2)#6}}{, and \textup{#3(#1)#4--#5(#2)#6}}

% Make number non-italic in any environment.
\crefdefaultlabelformat{#2\textup{#1}#3}

% Environments
\makeatother
% For box around Definition, Theorem, \ldots
%%fakesection Theorems
\usepackage{thmtools}
\usepackage[framemethod=TikZ]{mdframed}

\theoremstyle{definition}
\mdfdefinestyle{mdbluebox}{%
	roundcorner = 10pt,
	linewidth=1pt,
	skipabove=12pt,
	innerbottommargin=9pt,
	skipbelow=2pt,
	nobreak=true,
	linecolor=blue,
	backgroundcolor=TealBlue!5,
}
\declaretheoremstyle[
	headfont=\sffamily\bfseries\color{MidnightBlue},
	mdframed={style=mdbluebox},
	headpunct={\\[3pt]},
	postheadspace={0pt}
]{thmbluebox}

\mdfdefinestyle{mdredbox}{%
	linewidth=0.5pt,
	skipabove=12pt,
	frametitleaboveskip=5pt,
	frametitlebelowskip=0pt,
	skipbelow=2pt,
	frametitlefont=\bfseries,
	innertopmargin=4pt,
	innerbottommargin=8pt,
	nobreak=false,
	linecolor=RawSienna,
	backgroundcolor=Salmon!5,
}
\declaretheoremstyle[
	headfont=\bfseries\color{RawSienna},
	mdframed={style=mdredbox},
	headpunct={\\[3pt]},
	postheadspace={0pt},
]{thmredbox}

\declaretheorem[%
style=thmbluebox,name=Theorem,numberwithin=section]{thm}
\declaretheorem[style=thmbluebox,name=Lemma,sibling=thm]{lem}
\declaretheorem[style=thmbluebox,name=Proposition,sibling=thm]{prop}
\declaretheorem[style=thmbluebox,name=Corollary,sibling=thm]{coro}
\declaretheorem[style=thmredbox,name=Example,sibling=thm]{eg}

\mdfdefinestyle{mdgreenbox}{%
	roundcorner = 10pt,
	linewidth=1pt,
	skipabove=12pt,
	innerbottommargin=9pt,
	skipbelow=2pt,
	nobreak=true,
	linecolor=ForestGreen,
	backgroundcolor=ForestGreen!5,
}

\declaretheoremstyle[
	headfont=\bfseries\sffamily\color{ForestGreen!70!black},
	bodyfont=\normalfont,
	spaceabove=2pt,
	spacebelow=1pt,
	mdframed={style=mdgreenbox},
	headpunct={ --- },
]{thmgreenbox}

\declaretheorem[style=thmgreenbox,name=Definition,sibling=thm]{defn}

\mdfdefinestyle{mdgreenboxsq}{%
	linewidth=1pt,
	skipabove=12pt,
	innerbottommargin=9pt,
	skipbelow=2pt,
	nobreak=true,
	linecolor=ForestGreen,
	backgroundcolor=ForestGreen!5,
}
\declaretheoremstyle[
	headfont=\bfseries\sffamily\color{ForestGreen!70!black},
	bodyfont=\normalfont,
	spaceabove=2pt,
	spacebelow=1pt,
	mdframed={style=mdgreenboxsq},
	headpunct={},
]{thmgreenboxsq}
\declaretheoremstyle[
	headfont=\bfseries\sffamily\color{ForestGreen!70!black},
	bodyfont=\normalfont,
	spaceabove=2pt,
	spacebelow=1pt,
	mdframed={style=mdgreenboxsq},
	headpunct={},
]{thmgreenboxsq*}

\mdfdefinestyle{mdblackbox}{%
	skipabove=8pt,
	linewidth=3pt,
	rightline=false,
	leftline=true,
	topline=false,
	bottomline=false,
	linecolor=black,
	backgroundcolor=RedViolet!5!gray!5,
}
\declaretheoremstyle[
	headfont=\bfseries,
	bodyfont=\normalfont\small,
	spaceabove=0pt,
	spacebelow=0pt,
	mdframed={style=mdblackbox}
]{thmblackbox}

\theoremstyle{plain}
\declaretheorem[name=Question,sibling=thm,style=thmblackbox]{ques}
\declaretheorem[name=Remark,sibling=thm,style=thmgreenboxsq]{remark}
\declaretheorem[name=Remark,sibling=thm,style=thmgreenboxsq*]{remark*}
\newtheorem{ass}[thm]{Assumptions}

\theoremstyle{definition}
\newtheorem*{problem}{Problem}
\newtheorem{claim}[thm]{Claim}
\theoremstyle{remark}
\newtheorem*{case}{Case}
\newtheorem*{notation}{Notation}
\newtheorem*{note}{Note}
\newtheorem*{motivation}{Motivation}
\newtheorem*{intuition}{Intuition}
\newtheorem*{conjecture}{Conjecture}

% Make section starts with 1 for report type
%\renewcommand\thesection{\arabic{section}}

% End example and intermezzo environments with a small diamond (just like proof
% environments end with a small square)
\usepackage{etoolbox}
\AtEndEnvironment{vb}{\null\hfill$\diamond$}%
\AtEndEnvironment{intermezzo}{\null\hfill$\diamond$}%
% \AtEndEnvironment{opmerking}{\null\hfill$\diamond$}%

% Fix some spacing
% http://tex.stackexchange.com/questions/22119/how-can-i-change-the-spacing-before-theorems-with-amsthm
\makeatletter
\def\thm@space@setup{%
  \thm@preskip=\parskip \thm@postskip=0pt
}

% Fix some stuff
% %http://tex.stackexchange.com/questions/76273/multiple-pdfs-with-page-group-included-in-a-single-page-warning
\pdfsuppresswarningpagegroup=1


% My name
\author{Jaden Wang}



\begin{document}
How does $ [X, Y]_0$ depend on the base point?
\begin{defn}
Given $ f_0, f_1: X \to Y$ and a path $ u : I \to Y$, if there is a homotopy $ H: X \times I \to Y$ s.t.\ $ H(x,0) = f_0(x)$, $ H(x,1) = f_1(x)$, and $ H(x_0,t) = u(t)$, then we say $ H$ is a  \allbold{homotopy along $ u$}, denote $ f_0 \simeq_u f_1$. 
\end{defn}

If $ f_0, f_1$ are base-point preserving, then $ u$ is a loop in  $ Y$. To be able to say anything about moving base point we read  $ (X,x_0)$ to be ``non-degenerate" by which we mean that  $ (X,x_0)$ is an \allbold{NDR-pair} (neighborhood deformation retract).

\begin{defn}
$ A \subseteq X$ is an \allbold{NDR-pair} if there are maps $ u: X \to I$ and $ h:X \times I \to X$ s.t.\
\begin{enumerate}[label=(\arabic*)]
	\item $ A= u ^{-1}(0)$;
	\item $ h(x,0) = x \ \forall \ x \in X$;
	\item $ h(a,t) = a \ \forall \ a \in A$;
	\item $ h(x,1) \in A \ \forall \ x \in X$ with $ u(x)<1$.
\end{enumerate}
\end{defn}
Note: $ u^{-1}([0,1))$ is an open neighborhood of $ A \in X$ that retracts to $ A$.

\begin{eg}[NDR-pairs]
We have
\begin{enumerate}[label=(\arabic*)]
	\item sub CW-complex of a CW-complex (think that sphere can contract to boundary if we remove a point).
	\item submanifold of a manifold.
\end{enumerate}
\end{eg}

\begin{lem}
If $ (X,A)$ is an NDR-pair, then  $ (X \times \{0\}) \cup  (A \times I) $ is a retract of $ X \times I$.
\end{lem}

\begin{proof}
Define $ R: X \times I \to (X \times \{0\}) \cup (A \times I), (x,t) \mapsto \begin{cases}
	(x,t) & x \in A \text{ or } t=0\\
	(h(x,1),t - u(x) ) & t \geq u(x), t>0\\
	(h(x,\frac{t}{u(x)}),0) & u(x) \geq t \text{ and } u(x)>0 
\end{cases} $
Exercise: this is a retract.
\end{proof}

\begin{lem}
If $ (X,x_0)$ is an NDR-pair and  $ f_0: X \to Y$, $ f_0(x_0)=y_0$, $ \gamma: I \to Y$ path from $ y_0$ to $ y$, then there exists  $ f_1: X \to Y$ s.t.\  $ f_1(x_0) = y_1$ and $ f_0 \simeq_\gamma f_1$. We denote $ f_1$ by $ \gamma \cdot f_0$ (well-defined once $ R$ from previous lemma is fixed).
\end{lem}
\begin{proof}
Let $ R$ be from previous lemma for  $ (X,x_0)$. Let $ H : X \times I \to Y$ to be
\begin{align*}
	H(x,t) = \begin{cases}
		f_0(R(x,t)) & R(x,t) \in X \times \{0\} \\
		\gamma(R(x,t)) & R(x,t) \in A \times I\\
	\end{cases}
\end{align*}
	Let $ f_1(x) = H(x,1)$. Then $ H$ yields  $ f_0 \simeq _{ \gamma} f_1$.
\end{proof}

\begin{lem}
Suppose $  f_0,f_1,f_2: X \to Y, (X,x_0)$ an NDR pair if  $ f_0 \simeq _{ \gamma} f_1, f_0 \simeq _{ \gamma'} f_2$ with $ \gamma \simeq \gamma'$ rel boundary, then $ f_1 \simeq f_2$ rel base point.
\end{lem}
\begin{proof}
Since $ (X,x_0)$ is an NDR-pair, we can show that $ (X \times I, (X \times \{0,1\} \cup \{x_0\} \times I ))$ is also an NDR-pair. Exercise: show this ($ \epsilon$ tubes retract, just need to be compatible on overlaps). So lemma yields a retraction $ R: (X \times I) \times I \to ((X \times I) \times \{0\}) \cup ((X \times \{0,1\}) \cup (\{x_0\} \times I  )) \times I$.

Let $ H$ be homotopy  $ f_0 \simeq_{ \gamma} f_1$, $ G$ be homotopy  $ f_0 \simeq_{ \gamma'} f_2$, $ K$ be homotopy  $ \gamma \simeq \gamma'$ rel boundary. Let $ \overline{R} = () \circ R$. Check $ \overline{R}|_{ X \times I \times \{1\} }$ is a homotopy $ f_1$ to $ f_2$ rel base point.
\end{proof}

\begin{lem}
Suppose $ f_0,f_1, f_2 : X \to Y$, $ f_0 \simeq_{ \gamma_1} f_1$, $ f1 \simeq_{ \gamma_2} f_2$ with $ \gamma_1(1) = \gamma_2(0)$, then $ f_0 \simeq_{ \gamma_1 \times \gamma_2} f_2$.
\end{lem}
\begin{proof}
Concatenate homotopies, $ H $ for $ f_0 \simeq _{ \gamma_1} f_1$ and $ G$ for  $ f_1 \simeq _{ \gamma_2} f_2$.
\end{proof}

\begin{thm}
	If $ x_0$ is a non-degenerate base point of $ X$, then  $ \pi_1(Y,y_0)$ acts on $ [X, Y]_0$. Moreover, $ [X, Y]$ is the quotient of $ [X, Y]_0$ by the $ \pi_1(Y,y_0)$ action if $ Y$ is path-connected.
\end{thm}

\begin{proof}
	Take $ [ \gamma] \in \pi_1(Y,y_0), [f] \in [X, Y]_0$, lemma yields $ \gamma \cdot f: X \to Y$ and $ [ \gamma \cdot f]$ clearly $ [X, Y]_0$.
	\begin{claim}
		$ [ \gamma \cdot f]$ is well-defined. 
	\end{claim}
		If $ f,g \in [f]$ so $ f \simeq g$ rel base point. By lemma, we get $ f,g$  s.t.\ $ f \simeq _{ \gamma} f_1$ and $ g \simeq _{ \gamma} g_1$. Thus $ f_1 \simeq _{ \gamma ^{-1}} f \simeq g \simeq _{ \gamma} g_1$. Lemma says $ f_1 \simeq \gamma^{-1} \times \text{const} \times \gamma g_1 \simeq \text{const}  $. So Lemma 11 says $ f_1 \simeq g_1$ rel base points.

		That is, $ [ \gamma \cdot f]$ does not depend on choice of $ f$. By lemma 11,  $ [ \gamma \cdot f]$ does not depend on $ \gamma$. Let $ \Phi: [X, Y]_0 \to [X, Y]$ by just forgetting base point. Clearly $ \Phi( [ \gamma] \cdot [ f]) = \Phi([f])$. Hence we obtain an induced map
		\begin{align*}
			\Phi: [X, Y]_0 / \pi_1(Y,y_0) \to [X, Y].
		\end{align*}
\end{proof}
If $ \Phi([f]) = \Phi([g])$, then let $ H$ be the free homotopy from  $ f$ to  $ g$. Set  $ \gamma(t) = H(x_0,t)$. Both $ f,g$ take  $ x_0$ to $ y_0$ so $ [ \gamma] \in \pi_1(Y,y_0)$ and $ [ \gamma \cdot f] = [g]$. So $ \Phi$ is injective.

$ \Phi$ is surjective by lemma 10 if $ Y$ is path-connected.

\begin{coro}
A based map is null-homotopic iff it is based null-homotopic.
\end{coro}
\begin{proof}
Based null-homotopy clearly implies null-homotopic. If $ f \simeq e$, WLOG $ e(x)=y_0$, by homotopy $ H$, let  $ \gamma(t) = H(x_0,t)$ so $ f \simeq _{ \gamma} e$ so $ e \simeq _{ \gamma ^{-1}} f$. By lemma 11, $ f \simeq e$ rel base point.
\end{proof}
\end{document}
