\documentclass[12pt,class=article,crop=false]{standalone} 
\newcommand{\alert}[1]{{\bf \color{red} [Alert:] #1}}
\newcommand{\todo}[1]{{\bf \color{orange} [TODO:] #1}}
\newcommand{\real}[1][]{\mathbb{R}^{#1}}
\newcommand{\myeqn}[1]{(\ref{#1})}
\newcommand{\myex}[1]{Example \ref{#1}}
\newcommand{\defeq}{\stackrel{\mathrm{def}}{=}}
\newcommand{\parder}[2]{\frac{\partial #1}{\partial #2}}
\newcommand{\Lie}[3][]{\mathsf{L}_{#3}^{#1} #2}
\newcommand{\LieA}[1]{\mathsf{Lie}(#1)}
\newcommand{\lieder}[2]{\mathcal{L}_{#2} #1}
\renewcommand{\t}{^{\mbox{\tiny\sf T}}}
\newcommand{\trans}{^{\mbox{\tiny\sf T}}}
\newcommand{\markup}[1]{\{\textbf{#1}\}}
\newcommand{\msub}[1]{_\mathrm{#1}}
\newcommand{\msup}[1]{^\mathrm{#1}}
\newcommand{\inv}[1]{#1^{-1}}
\newcommand{\pinv}[1]{{#1}^{+}}
\newcommand{\myfracA}[2]{\displaystyle{\frac{#1}{#2}}}
\newcommand{\myfracB}[2]{{#1}/{#2}}
\newcommand{\mydiffA}[1]{\dot{#1}}
\newcommand{\mydiffB}[2]{\myfracA{\mathrm{d}{#1}}{\mathrm{d}{#2}}}
\newcommand{\ball}[2]{\mathcal{B}_{#1}\left(#2\right)}
\newcommand{\acos}[1]{\cos^{-1}\left(#1\right)}
\newcommand{\asin}[1]{\sin^{-1}\left(#1\right)}
\newcommand{\mani}{\mathcal{M}}
\newcommand{\tang}[2]{\mathsf{T}_{#1} #2}
\newcommand{\LieB}[2]{[ #1, #2 ]}
\newcommand{\LieBad}[3][]{\mathsf{ad}_{#2}^{#1} #3}
\newcommand{\ReachVT}{\mathcal{R}^V_T}
\newcommand{\ReachVt}{\mathcal{R}^V_t}
\newcommand{\ReachVTe}{\mathcal{R}^V_{\le T}}
\newcommand{\ReachT}{\mathcal{R}_T}
\newcommand{\Reacht}{\mathcal{R}_t}
\newcommand{\ReachTe}{\mathcal{R}_{\le T}}
\newcommand{\accLA}[1]{\mathsf{Lie}(#1)}
\newcommand{\accD}{\Delta_{\mathcal{F}}}
\newcommand{\accSA}{\mathsf{Lie}(\mathcal{G},f)}
\newcommand{\accDS}{\Delta_{\mathcal{G}}}
\newcommand{\eval}[3]{\mathsf{Ev}^{#2}_{#1}\left( #3 \right)}
\newcommand{\stlc}{\textsc{stlc}}
\newcommand{\clf}{\textsc{clf}}
\newcommand{\jqlf}{\textsc{jqlf}}
\newcommand{\dlas}{\textsc{dlas}}
\newcommand{\Ad}[2]{\mathsf{Ad}_{#1} #2}
\newcommand{\xe}{\ensuremath{x_e}}
\newcommand{\lebg}[1]{\mathcal{L}_{#1}}
\newcommand{\lebgx}[1]{\mathcal{L}_{#1 \mathrm{e}}}
\newcommand{\dom}{D}
\newcommand{\domT}{[t_0,\infty) \times D}
\newcommand{\rarrow}{\rightarrow}
\renewcommand{\d}{\mathrm{d}}
\renewcommand{\Re}{\mathbb{R}}
\newcommand{\C}{\mathrm{C}}

\newcommand{\QED}{{\unskip\nobreak\hfil\penalty50\hskip2em\vadjust{}
		\nobreak\hfil$\Box$\parfillskip=0pt\finalhyphendemerits=0\par}\vspace{0.1cm}}
\newcommand{\eoEx}{{\unskip\nobreak\hfil\penalty50\hskip0em\vadjust{}
		\nobreak\hfil$\Large\Diamond$\parfillskip=0pt\finalhyphendemerits=0\par}\vspace{0.1cm}}

\newcommand{\sgn}{\ensuremath{\operatorname{sgn}}}
\newcommand{\sat}{\ensuremath{\operatorname{sat}}}

\newcommand{\half}{\frac{1}{2}}
\newcommand{\shalf}{\mbox{$\frac{1}{2}$}}
\newcommand{\marcom}[1]{\marginpar{\footnotesize #1}}
\newcommand{\der}{\mathrm{D}}
\newcommand{\e}{\mathrm{e}}
\newcommand{\dt}{\mathrm{d}t}

\newcommand{\cA}{\ensuremath{\mathcal{A}}}
\newcommand{\cB}{\ensuremath{\mathcal{B}}}
\newcommand{\cG}{\ensuremath{\mathcal{G}}}
\newcommand{\cK}{\ensuremath{\mathcal{K}}}
\newcommand{\cW}{\ensuremath{\mathcal{W}}}
\newcommand{\cZ}{\ensuremath{\mathcal{Z}}}
\newcommand{\cS}{\ensuremath{\mathcal{S}}}
\newcommand{\cD}{\ensuremath{\mathcal{D}}}
\newcommand{\cP}{\ensuremath{\mathcal{P}}}
\newcommand{\cV}{\ensuremath{\mathcal{V}}}
\newcommand{\cL}{\ensuremath{\mathcal{L}}}
\newcommand{\cN}{\ensuremath{\mathcal{N}}}
\newcommand{\cI}{\ensuremath{\mathcal{I}}}
\newcommand{\cR}{\ensuremath{\mathcal{R}}}
\newcommand{\cM}{\ensuremath{\mathcal{M}}}
\newcommand{\cC}{\ensuremath{\mathcal{C}}}
\newcommand{\cF}{\ensuremath{\mathcal{F}}}
\newcommand{\cH}{\ensuremath{\mathcal{H}}}
\newcommand{\cO}{\ensuremath{\mathcal{O}}}
\newcommand{\cX}{\ensuremath{\mathcal{X}}}
\newcommand{\cY}{\ensuremath{\mathcal{Y}}}
\newcommand{\Ci}{\ensuremath{\mathcal{C}^\infty}}
\newcommand{\ISS}{\textsc{iss}}
\newcommand{\LISS}{\textsc{liss}}
\newcommand{\GAS}{\textsc{gas}}
\newcommand{\GS}{\textsc{gs}}
\newcommand{\LES}{\textsc{les}}
\newcommand{\GUAS}{\textsc{guas}}
\newcommand{\BIBO}{\textsc{bibo}}
\newcommand{\spec}{\ensuremath{\operatorname{spec}}}
\newcommand{\spn}{\ensuremath{\operatorname{span}}}
\renewcommand{\i}{\mathrm{i\,}}

\renewcommand{\implies}{\Rightarrow}

\renewcommand{\theenumi}{$\roman{enumi})$}
\renewcommand{\labelenumi}{\theenumi}

\font\ptmten=zptmcmrm scaled 1200
\newcommand{\w}{\mbox{{\ptmten w}}}
\newcommand{\z}{\mbox{{\ptmten z}}}
\renewcommand{\Re}{\mathbb{R}}

\newcommand{\cl}{\operatorname{cl}}
\newcommand{\intr}{\operatorname{int}}
\newcommand{\rank}{\operatorname{rank}}
\newcommand{\co}{\operatorname{co}}
\newcommand{\aff}{\operatorname{aff}}

\theoremstyle{plain}
\newtheorem{theorem}{Theorem}[chapter]
\newtheorem{claim}[theorem]{Claim}
\newtheorem{corollary}[theorem]{Corollary}
\newtheorem{prop}[theorem]{Proposition}
\newtheorem{fact}[theorem]{Fact}
\newtheorem{lemma}[theorem]{Lemma}

\newtheorem{remark}{Remark}[chapter]

\theoremstyle{definition}
\newtheorem{assume}[theorem]{Assumption}
\newtheorem{defn}[theorem]{Definition}
\newtheorem{problem}[theorem]{Problem}
\newtheorem{exercise}{Exercise}
\newtheorem{example}[theorem]{Example}


\begin{document}
\section{Introduction to Spectral Sequences}

\begin{defn}
A \allbold{bigraded module} is an indexed collection of modules $ E_{s,t}$ for every pair of integers  $ s,t$.

A  \allbold{differential of bidegree $ (-r,r-1)$}  is a collection of homomorphisms $ d: E_{s,t} \to E_{s-r,t+r-1}$ s.t.\ $ d^2 = 0$ (where composition makes sense).

The \allbold{homology of $ d$} is
\begin{align*}
	H_{s,t}(E,d) : = \frac{ \ker :E_{x,t} \to E_{s-r,t+r-1}}{ \im (d: E_{s+r,t-r+1} \toE_{s,t})}
\end{align*}
\end{defn}
If $ E_q = \bigoplus_{ s+t=q} E_{s,t}$ then $ d$ induces a homomorphism  $ \partial :E_q \to E_{q-1}).$ so $ (E_q, \partial )$ is a chain complex. Moreover, $ H_q(E_*, \partial ) = \bigoplus_{ s+t=q} H_{s,t}(E,d) $.
\begin{defn}
An  \allbold{$ E^{k}$-spectral sequence} is a sequence $ \{E^{r},d^{r}\} $  for $ r \geq k$  s.t.\ 
\begin{enumerate}[label=(\alph*)]
	\item $ E^{r}$ is a bigraded module and $ d^{r}$ is a differential of bidegree $ (-r,r-1)$,
	\item  $ E^{r+1} = H(E^{r},d^{r}) \ \forall \ r>k$.
\end{enumerate}
\end{defn}

\begin{eg}
$ E^{1}$.
\end{eg}

\begin{defn}
Suppose for every $ s,t$ there exists a  $ \# r(s,t)$  s.t.\ $ \ \forall \ r > r(s,t)$
\begin{align*}
	d^{r}:E_{s,t}^{r} \to E_{s-r,t+r-1}^{r}
\end{align*}
is the zero map, then $ E_{s,t}^{r+1}$ is just a quotient of $ E_{s,t}^{r}$ we can define $ E_{s,t}^{\infty}$ to be the direct limit of $ E_{s,t}^{r}$. In this situation we say the spectral sequence \allbold{coverges} to $ E_{s,t}^{\infty}$ .
\end{defn} 
\begin{remark}
If $ E_{s,t}^{r} = 0, s<0,t<0$ (first quadrant ss), then for each $ s,t$ there is some  $ r$  s.t.\ $ E_{s,t}^{r}$ is constant in $ r$.
\end{remark}

So where do spectral sequences come from? Filtrations.

\begin{defn}
A \allbold{filtration $ F$ on a module  $ A$}  is a sequence of submodules $ \{F_sA\} $ of $ A$  s.t.\ 
\begin{align*}
	A \supseteq \ldots \supseteq F_{s+1} A \supseteq F_s A \supseteq \ldots
\end{align*}
$ s$ is the  \allbold{filtration degree}, $ t$ is the  \allbold{complementary degree}, and $ s+t$ is the  \allbold{total degree}.   
A filtration is \allbold{convergent}  if
\begin{align*}
	\bigcap_{ s} F_sA = 0 
\end{align*}
and
\begin{align*}
	\bigcup_{ s} F_s A =A 
\end{align*}
We will usually have a finite filtration where $ F_{-1}(A) = 0$.

If $ A$ is graded  $ \{A_k\} $ and $ F$ respects the grading,  then the filtration inherits a grading $ F_sA = \{F_s A_k\} $. The \allbold{associated graded module} is
\begin{align*}
	G(A)_s = F_sA / F_{s-1}A\\
	G(A) = \bigoplus_{ s} G(A)_s 
\end{align*}
If $ A$ is graded then  $ G(A)$ is bigraded
 \begin{align*}
	G(A)_{s,t} = F_s A_{s+t} / F_{s-1} A_{s+t}
\end{align*}
\end{defn}

\begin{eg}
\begin{enumerate}[label=(\arabic*)]
	\item $ A= F_1A = \zz /4$, $ F_0A= \zz /2$, $ F_{-1}A=0$, so
		\begin{align*}
			G(A)_s = \begin{cases}
				\zz /2 & s=1,0\\
				0 & s \neq 0,1\\
			\end{cases}
		\end{align*}
		So $ G(A) = \bigoplus_{ s}G(A) = \zz /2 \oplus  \zz /2 $.
	\item $ A = F_1A = \zz_2 \oplus \zz_2$, $ F_0A = \zz_2$, $ F_{-1}A = 0$. Then
		\begin{align*}
			G(A)_s = \begin{cases}
				\zz /2 & i = 1,0\\
				0 & i \neq 0,1\\
			\end{cases}
		\end{align*}
	\item $ A = A_1 = \zz$, $ A_0 = 2 \zz$, $ A_{-1}=0$, then
		\begin{align*}
			G(A) = A_1 / A_0 \oplus  A_0 / A_{-1}\\
			\cong \zz_2 \oplus \zz \not \cong A
		\end{align*}
		But $ G(A)$ is ``close'' to  $ A$.
\end{enumerate}
\end{eg}
\begin{lem}
If $ F$ is a finite filtration of  $ A_1$ then
\begin{enumerate}[label=(\arabic*)]
	\item If $ G(A)_s$ is free for all  $ s$ then  $ G(A) \cong A$.
	\item If $ G(A)_s$ is a vector space over a field then  $ G(A) \cong A$.
	\item If $ G(A)_s$ is finite for all  $ s$, then  $ A$ is finite and order $ (A) = $order $ G(A)$. 
	\item If  $ G(A)_s$ is finitely generated  $ \ \forall \ s$, then $ A$ is finitely generated and  $ \rank A = \rank (G(A))$.
	\item If  $ G(A)_s =0$ for all but one  $ s$, then  $ A \cong G(A)$.
	\item If $ G(A)_s = 0$ for all but two  $ s$, say  $ G(A)_k, G(A)_\ell$ with $ k < \ell$ then
		\begin{align*}
			0 \to G(A)_k \to A \to G(A)_{\ell} \to 0
		\end{align*}
is exact.
\end{enumerate}
\end{lem}
\begin{proof}
\begin{enumerate}[label=(\arabic*)]
	\item We have
		\begin{align*}
			0 \to F_{n-1} A \to F_nA = A \to F_n A / F_{n-1} A = G(A)_n \to 0
		\end{align*}
		Exercise: if $ 0 \to A \to B \to C \to 0$ is exact and $ C$ is free, then  $ B = A \oplus C$. So $ A \cong G(A)_n \oplus F_{n-1}(A)$. Similarly,
		\begin{align*}
			$ F_{n-1}(A) \cong G(A)_{n-1} \oplus F_{n-2}(A)$ and so on. So $ A \cong \bigoplus_{ s} G(A)_s $.
		\end{align*}
	\item this is a corollary of 1.
	\item 
	\item
	\item 
	\begin{align*}
	G(A)_s = \begin{cases}
		F_s(A) & s=k\\
		0& s \neq k\\
	\end{cases}
\end{align*}
So $ G(A) = G(A)_k$ and
 \begin{align*}
	A = F_n(A) = F_{n-1}(A) = \ldots= F_k(A) \supseteq  F_{k-1}(A) = \ldots = F_{-1}A
\end{align*}
so $ A = F_k(A) = F_k(A) / F_{k-1}(A) = G(A)_k = G(A)$.

6): If
\begin{align*}
	G(A)_s = \begin{cases}
		C,& s=\ell\\
		B, & s = k\\
		0, & \text{ else} 
	\end{cases}
\end{align*}
Then
\begin{align*}
	A = F_n(A) = \ldots F_{\ell}(A) \supseteq F_{\ell-1}(A) = \ldots F_k(A) \supseteq F_{k-1}(A) = \ldots=F_{-1}(A) = 0
\end{align*}
So $ G(A)_{\ell} = A /F_{\ell-1}(A) \cong C$.
and
\begin{align*}
	B = G(A)_k = (F_{l-1}A = F_k(A)) / (F_{k-1}A = 0) = F_{\ell-1}A
\end{align*}
So $ A  /B = C$ iff
 \begin{align*}
	0 \to B \to A \to C \to 0
\end{align*}

\end{enumerate}

Exercise: prove 3-4.
\end{proof}

If $ F$ is a filtration of a chain complex $ \{C_*, \partial \} $ s.t.\ $ (F_s C_*, \partial )$ is a subcomplex of $ (C_*, \partial )$, then $ F$ induces a filtration on the homology  $ H_*(C_*, \partial )$
\begin{align*}
	F_s H(C_*, \partial ) = \im  (H(F_s C_*, \partial ) \to H_*(C_*,\partial ))
\end{align*}
If $ F$ is finite, then it is easy to see
 \begin{align*}
	H(C_*, \partial ) = \bigcup F_s H(C_*, \partial )\\\
	\bigcap (F_s H(C_*, \partial )) =0
\end{align*}
\begin{thm}
Let $ F$ be a finite filtration of a chain complex  $ (c_*, \partial )$, then there is an $ E^{1}$ spectral sequence with
\begin{enumerate}[label=(\arabic*)]
	\item $ E^{1}_{s,t} = H_{s+t}(F_s C / F_{s-1}C)$ 
	\item $ d^{1}$ is the connecting homomorphism of the triple $ (F_s C, F_{s-1}C,F_{s-2}C)$.
	\item $ G(H(C_*, \partial ))_{s,t} = E_{s,t}^{\infty}$.
\end{enumerate}
\end{thm}
\begin{proof}
\begin{align*}
	E_{s,t}^{r} = \frac{ \{ c \in F_s(C_{s,t}): \partial c \in F_{s-r} (C_{s+t+1})\}}{F_{s-1}C_{s+t} + \partial (F_{s+r-1} C_{s+t-1}) }
\end{align*}
and $ d^{n} = \partial $ applied to representatives of $ E_{s,t}^{r}$.
\end{proof}
For 2, recall
\begin{align*}
	0 \to F_{s-1}C / F_{s-2}C \to F_{s}C / F_{s-2} c \to F_s C / F_{s-1} C \to 0
\end{align*}
Third isomorphism theorem, induces a long exact sequence on homology.
\begin{align*}
	 H_k(F_s C / F_{s-1}C) \xrightarrow{ d^{1}} H_{k-1} (F_{s-1}C / F_{s-2}C) 
\end{align*}

\begin{eg}
Computing the homology of $ T^2$.
\end{eg}
\begin{eg}
Let $ A \subseteq X$ a subspace we get a filtration of $ C_*(X)$
 \begin{align*}
	 F_1C &= C_*(X)\\
	F_0C &= C_*(A)\\
	F_{-1}C &= 0 
\end{align*}
So we get a spectral sequence with
\begin{align*}
	E_{s,t}^{1} = H_{s+t}\left( \frac{F_s C_*}{ F_{s-1}C_*} \right)  \cong \begin{cases}
		H_{s+t}(X,A) & s=1\\
		H_{s+t} (A) & s=1\\
		0 & s \neq 0,1\\
	\end{cases}
\end{align*}
Note $ E^2 = E^{2+k}  = E^{\infty}$. So $ G(H(C_*(X)))_k = \bigoplus_{ s+t=k} E_{s,t}^{\infty} = E_{0,k}^{\infty} \oplus E_{1,k-1}^{\infty}$. Lemma 1 says we get
\begin{align*}
	0 \to H_k(A) / \im \partial \to H_k(X) \to \ker \partial \to 0
\end{align*}
Note this is equivalent to
\begin{align*}
	0 \to \im (\partial : H_{k+1}(X,A) \to H_k(A)) \to H_k(A) \xrightarrow{ i_8}  H_k(X) \to \ker (\partial :H_{k}(X,A) \to H_{k-1}(A)) \to 0
\end{align*}
If $ i: A \to X$ and $ j: X \to (X,A)$ inclusions. The above says $ \im \partial = \ker i_*$ and $ \ker \partial  = \im j_*$. This proves the hard part of the long exact sequence for the pair $ (X,A)$. So the spectral sequence generalizes long exact sequence.
\end{eg}
\begin{eg}
We will show for $ X$ CW complex , $H_*^{CW}(X) \cong H_*^{Sing}(X) $.

Let $ F_kC_*(X) = C_*(X^{(k)})$. This is a finite filtration so there exists a spectral sequence that converges to $ E_{s,t}^{\infty}$.

\begin{align*}
	E_{s,t}^{1} = H_{s+t}(F_s C / F_{s-1}C)\\
	&= H_{s+t}(C_*(X^{(s)})/ C_*(X^{(s-1)})) \\
	&= H_{s+t}(X^{(s)}, X^{(s-1)}) & \text{ by defn} \\
	&=\con\widetilde{ H}_{s,t}(X^{(s)} / X^{(s-1)})\\
	&= \widetilde{ H}_{s+t} ( \text{wedge of s-spheres} ) \\
	&= \begin{cases}
		\bigoplus_{ s- \text{cells} } \zz& t=0\\
		0& t \neq 0\\
	\end{cases} \\
	&= \begin{cases}
		C_s^{CW}(X) & t=0\\
		0& t\neq 0\\
	\end{cases} 
\end{align*}
and in sequence $ d^{1} = \partial $ map of the LES of $ (F_s ,F_{s-1}, F_{s-2})$ \emph{i.e.} $ H_s(X^{(s)} , X^{(s-1)}) \to H_{s-1} (X^{(s-1)} , X^{(s-2)})$. We know $ d^{1} = \partial ^{CW}$. So
\begin{align*}
	E^2 = E_{s,t}^{\infty} = \begin{cases}
		H_s^{CW}(X) & t=0\\
		0& t \neq 0\\
	\end{cases}
\end{align*}
By Lemma 1 part 5, we have $ H_p(X) \cong H_p ^{CW}(X)$.
\begin{remark}
All the above works for cohomology too and the $ \partial $ respect the product structure.
\end{remark}
\end{eg}
\end{document}
