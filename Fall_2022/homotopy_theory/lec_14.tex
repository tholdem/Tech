\documentclass[12pt,class=article,crop=false]{standalone} 
%Fall 2022
% Some basic packages
\usepackage{standalone}[subpreambles=true]
\usepackage[utf8]{inputenc}
\usepackage[T1]{fontenc}
\usepackage{textcomp}
\usepackage[english]{babel}
\usepackage{url}
\usepackage{graphicx}
%\usepackage{quiver}
\usepackage{float}
\usepackage{enumitem}
\usepackage{lmodern}
\usepackage{comment}
\usepackage{hyperref}
\usepackage[usenames,svgnames,dvipsnames]{xcolor}
\usepackage[margin=1in]{geometry}
\usepackage{pdfpages}

\pdfminorversion=7

% Don't indent paragraphs, leave some space between them
\usepackage{parskip}

% Hide page number when page is empty
\usepackage{emptypage}
\usepackage{subcaption}
\usepackage{multicol}
\usepackage[b]{esvect}

% Math stuff
\usepackage{amsmath, amsfonts, mathtools, amsthm, amssymb}
\usepackage{bbm}
\usepackage{stmaryrd}
\allowdisplaybreaks

% Fancy script capitals
\usepackage{mathrsfs}
\usepackage{cancel}
% Bold math
\usepackage{bm}
% Some shortcuts
\newcommand{\rr}{\ensuremath{\mathbb{R}}}
\newcommand{\zz}{\ensuremath{\mathbb{Z}}}
\newcommand{\qq}{\ensuremath{\mathbb{Q}}}
\newcommand{\nn}{\ensuremath{\mathbb{N}}}
\newcommand{\ff}{\ensuremath{\mathbb{F}}}
\newcommand{\cc}{\ensuremath{\mathbb{C}}}
\newcommand{\ee}{\ensuremath{\mathbb{E}}}
\newcommand{\hh}{\ensuremath{\mathbb{H}}}
\renewcommand\O{\ensuremath{\emptyset}}
\newcommand{\norm}[1]{{\left\lVert{#1}\right\rVert}}
\newcommand{\dbracket}[1]{{\left\llbracket{#1}\right\rrbracket}}
\newcommand{\ve}[1]{{\bm{#1}}}
\newcommand\allbold[1]{{\boldmath\textbf{#1}}}
\DeclareMathOperator{\lcm}{lcm}
\DeclareMathOperator{\im}{im}
\DeclareMathOperator{\coim}{coim}
\DeclareMathOperator{\dom}{dom}
\DeclareMathOperator{\tr}{tr}
\DeclareMathOperator{\rank}{rank}
\DeclareMathOperator*{\var}{Var}
\DeclareMathOperator*{\ev}{E}
\DeclareMathOperator{\dg}{deg}
\DeclareMathOperator{\aff}{aff}
\DeclareMathOperator{\conv}{conv}
\DeclareMathOperator{\inte}{int}
\DeclareMathOperator*{\argmin}{argmin}
\DeclareMathOperator*{\argmax}{argmax}
\DeclareMathOperator{\graph}{graph}
\DeclareMathOperator{\sgn}{sgn}
\DeclareMathOperator*{\Rep}{Rep}
\DeclareMathOperator{\Proj}{Proj}
\DeclareMathOperator{\mat}{mat}
\DeclareMathOperator{\diag}{diag}
\DeclareMathOperator{\aut}{Aut}
\DeclareMathOperator{\gal}{Gal}
\DeclareMathOperator{\inn}{Inn}
\DeclareMathOperator{\edm}{End}
\DeclareMathOperator{\Hom}{Hom}
\DeclareMathOperator{\ext}{Ext}
\DeclareMathOperator{\tor}{Tor}
\DeclareMathOperator{\Span}{Span}
\DeclareMathOperator{\Stab}{Stab}
\DeclareMathOperator{\cont}{cont}
\DeclareMathOperator{\Ann}{Ann}
\DeclareMathOperator{\Div}{div}
\DeclareMathOperator{\curl}{curl}
\DeclareMathOperator{\nat}{Nat}
\DeclareMathOperator{\gr}{Gr}
\DeclareMathOperator{\vect}{Vect}
\DeclareMathOperator{\id}{id}
\DeclareMathOperator{\Mod}{Mod}
\DeclareMathOperator{\sign}{sign}
\DeclareMathOperator{\Surf}{Surf}
\DeclareMathOperator{\fcone}{fcone}
\DeclareMathOperator{\Rot}{Rot}
\DeclareMathOperator{\grad}{grad}
\DeclareMathOperator{\atan2}{atan2}
\DeclareMathOperator{\Ric}{Ric}
\let\vec\relax
\DeclareMathOperator{\vec}{vec}
\let\Re\relax
\DeclareMathOperator{\Re}{Re}
\let\Im\relax
\DeclareMathOperator{\Im}{Im}
% Put x \to \infty below \lim
\let\svlim\lim\def\lim{\svlim\limits}

%wide hat
\usepackage{scalerel,stackengine}
\stackMath
\newcommand*\wh[1]{%
\savestack{\tmpbox}{\stretchto{%
  \scaleto{%
    \scalerel*[\widthof{\ensuremath{#1}}]{\kern-.6pt\bigwedge\kern-.6pt}%
    {\rule[-\textheight/2]{1ex}{\textheight}}%WIDTH-LIMITED BIG WEDGE
  }{\textheight}% 
}{0.5ex}}%
\stackon[1pt]{#1}{\tmpbox}%
}
\parskip 1ex

%Make implies and impliedby shorter
\let\implies\Rightarrow
\let\impliedby\Leftarrow
\let\iff\Leftrightarrow
\let\epsilon\varepsilon

% Add \contra symbol to denote contradiction
\usepackage{stmaryrd} % for \lightning
\newcommand\contra{\scalebox{1.5}{$\lightning$}}

% \let\phi\varphi

% Command for short corrections
% Usage: 1+1=\correct{3}{2}

\definecolor{correct}{HTML}{009900}
\newcommand\correct[2]{\ensuremath{\:}{\color{red}{#1}}\ensuremath{\to }{\color{correct}{#2}}\ensuremath{\:}}
\newcommand\green[1]{{\color{correct}{#1}}}

% horizontal rule
\newcommand\hr{
    \noindent\rule[0.5ex]{\linewidth}{0.5pt}
}

% hide parts
\newcommand\hide[1]{}

% si unitx
\usepackage{siunitx}
\sisetup{locale = FR}

%allows pmatrix to stretch
\makeatletter
\renewcommand*\env@matrix[1][\arraystretch]{%
  \edef\arraystretch{#1}%
  \hskip -\arraycolsep
  \let\@ifnextchar\new@ifnextchar
  \array{*\c@MaxMatrixCols c}}
\makeatother

\renewcommand{\arraystretch}{0.8}

\renewcommand{\baselinestretch}{1.5}

\usepackage{graphics}
\usepackage{epstopdf}

\RequirePackage{hyperref}
%%
%% Add support for color in order to color the hyperlinks.
%% 
\hypersetup{
  colorlinks = true,
  urlcolor = blue,
  citecolor = blue
}
%%fakesection Links
\hypersetup{
    colorlinks,
    linkcolor={red!50!black},
    citecolor={green!50!black},
    urlcolor={blue!80!black}
}
%customization of cleveref
\RequirePackage[capitalize,nameinlink]{cleveref}[0.19]

% Per SIAM Style Manual, "section" should be lowercase
\crefname{section}{section}{sections}
\crefname{subsection}{subsection}{subsections}
\Crefname{section}{Section}{Sections}
\Crefname{subsection}{Subsection}{Subsections}

% Per SIAM Style Manual, "Figure" should be spelled out in references
\Crefname{figure}{Figure}{Figures}

% Per SIAM Style Manual, don't say equation in front on an equation.
\crefformat{equation}{\textup{#2(#1)#3}}
\crefrangeformat{equation}{\textup{#3(#1)#4--#5(#2)#6}}
\crefmultiformat{equation}{\textup{#2(#1)#3}}{ and \textup{#2(#1)#3}}
{, \textup{#2(#1)#3}}{, and \textup{#2(#1)#3}}
\crefrangemultiformat{equation}{\textup{#3(#1)#4--#5(#2)#6}}%
{ and \textup{#3(#1)#4--#5(#2)#6}}{, \textup{#3(#1)#4--#5(#2)#6}}{, and \textup{#3(#1)#4--#5(#2)#6}}

% But spell it out at the beginning of a sentence.
\Crefformat{equation}{#2Equation~\textup{(#1)}#3}
\Crefrangeformat{equation}{Equations~\textup{#3(#1)#4--#5(#2)#6}}
\Crefmultiformat{equation}{Equations~\textup{#2(#1)#3}}{ and \textup{#2(#1)#3}}
{, \textup{#2(#1)#3}}{, and \textup{#2(#1)#3}}
\Crefrangemultiformat{equation}{Equations~\textup{#3(#1)#4--#5(#2)#6}}%
{ and \textup{#3(#1)#4--#5(#2)#6}}{, \textup{#3(#1)#4--#5(#2)#6}}{, and \textup{#3(#1)#4--#5(#2)#6}}

% Make number non-italic in any environment.
\crefdefaultlabelformat{#2\textup{#1}#3}

% Environments
\makeatother
% For box around Definition, Theorem, \ldots
%%fakesection Theorems
\usepackage{thmtools}
\usepackage[framemethod=TikZ]{mdframed}

\theoremstyle{definition}
\mdfdefinestyle{mdbluebox}{%
	roundcorner = 10pt,
	linewidth=1pt,
	skipabove=12pt,
	innerbottommargin=9pt,
	skipbelow=2pt,
	nobreak=true,
	linecolor=blue,
	backgroundcolor=TealBlue!5,
}
\declaretheoremstyle[
	headfont=\sffamily\bfseries\color{MidnightBlue},
	mdframed={style=mdbluebox},
	headpunct={\\[3pt]},
	postheadspace={0pt}
]{thmbluebox}

\mdfdefinestyle{mdredbox}{%
	linewidth=0.5pt,
	skipabove=12pt,
	frametitleaboveskip=5pt,
	frametitlebelowskip=0pt,
	skipbelow=2pt,
	frametitlefont=\bfseries,
	innertopmargin=4pt,
	innerbottommargin=8pt,
	nobreak=false,
	linecolor=RawSienna,
	backgroundcolor=Salmon!5,
}
\declaretheoremstyle[
	headfont=\bfseries\color{RawSienna},
	mdframed={style=mdredbox},
	headpunct={\\[3pt]},
	postheadspace={0pt},
]{thmredbox}

\declaretheorem[%
style=thmbluebox,name=Theorem,numberwithin=section]{thm}
\declaretheorem[style=thmbluebox,name=Lemma,sibling=thm]{lem}
\declaretheorem[style=thmbluebox,name=Proposition,sibling=thm]{prop}
\declaretheorem[style=thmbluebox,name=Corollary,sibling=thm]{coro}
\declaretheorem[style=thmredbox,name=Example,sibling=thm]{eg}

\mdfdefinestyle{mdgreenbox}{%
	roundcorner = 10pt,
	linewidth=1pt,
	skipabove=12pt,
	innerbottommargin=9pt,
	skipbelow=2pt,
	nobreak=true,
	linecolor=ForestGreen,
	backgroundcolor=ForestGreen!5,
}

\declaretheoremstyle[
	headfont=\bfseries\sffamily\color{ForestGreen!70!black},
	bodyfont=\normalfont,
	spaceabove=2pt,
	spacebelow=1pt,
	mdframed={style=mdgreenbox},
	headpunct={ --- },
]{thmgreenbox}

\declaretheorem[style=thmgreenbox,name=Definition,sibling=thm]{defn}

\mdfdefinestyle{mdgreenboxsq}{%
	linewidth=1pt,
	skipabove=12pt,
	innerbottommargin=9pt,
	skipbelow=2pt,
	nobreak=true,
	linecolor=ForestGreen,
	backgroundcolor=ForestGreen!5,
}
\declaretheoremstyle[
	headfont=\bfseries\sffamily\color{ForestGreen!70!black},
	bodyfont=\normalfont,
	spaceabove=2pt,
	spacebelow=1pt,
	mdframed={style=mdgreenboxsq},
	headpunct={},
]{thmgreenboxsq}
\declaretheoremstyle[
	headfont=\bfseries\sffamily\color{ForestGreen!70!black},
	bodyfont=\normalfont,
	spaceabove=2pt,
	spacebelow=1pt,
	mdframed={style=mdgreenboxsq},
	headpunct={},
]{thmgreenboxsq*}

\mdfdefinestyle{mdblackbox}{%
	skipabove=8pt,
	linewidth=3pt,
	rightline=false,
	leftline=true,
	topline=false,
	bottomline=false,
	linecolor=black,
	backgroundcolor=RedViolet!5!gray!5,
}
\declaretheoremstyle[
	headfont=\bfseries,
	bodyfont=\normalfont\small,
	spaceabove=0pt,
	spacebelow=0pt,
	mdframed={style=mdblackbox}
]{thmblackbox}

\theoremstyle{plain}
\declaretheorem[name=Question,sibling=thm,style=thmblackbox]{ques}
\declaretheorem[name=Remark,sibling=thm,style=thmgreenboxsq]{remark}
\declaretheorem[name=Remark,sibling=thm,style=thmgreenboxsq*]{remark*}
\newtheorem{ass}[thm]{Assumptions}

\theoremstyle{definition}
\newtheorem*{problem}{Problem}
\newtheorem{claim}[thm]{Claim}
\theoremstyle{remark}
\newtheorem*{case}{Case}
\newtheorem*{notation}{Notation}
\newtheorem*{note}{Note}
\newtheorem*{motivation}{Motivation}
\newtheorem*{intuition}{Intuition}
\newtheorem*{conjecture}{Conjecture}

% Make section starts with 1 for report type
%\renewcommand\thesection{\arabic{section}}

% End example and intermezzo environments with a small diamond (just like proof
% environments end with a small square)
\usepackage{etoolbox}
\AtEndEnvironment{vb}{\null\hfill$\diamond$}%
\AtEndEnvironment{intermezzo}{\null\hfill$\diamond$}%
% \AtEndEnvironment{opmerking}{\null\hfill$\diamond$}%

% Fix some spacing
% http://tex.stackexchange.com/questions/22119/how-can-i-change-the-spacing-before-theorems-with-amsthm
\makeatletter
\def\thm@space@setup{%
  \thm@preskip=\parskip \thm@postskip=0pt
}

% Fix some stuff
% %http://tex.stackexchange.com/questions/76273/multiple-pdfs-with-page-group-included-in-a-single-page-warning
\pdfsuppresswarningpagegroup=1


% My name
\author{Jaden Wang}



\begin{document}
\section{Introduction to Spectral Sequences}

\begin{defn}
A \allbold{bigraded module} is an indexed collection of modules $ E_{s,t}$ for every pair of integers  $ s,t$.

A  \allbold{differential of bidegree $ (-r,r-1)$}  is a collection of homomorphisms $ d: E_{s,t} \to E_{s-r,t+r-1}$ s.t.\ $ d^2 = 0$ (where composition makes sense).

The \allbold{homology of $ d$} is
\begin{align*}
	H_{s,t}(E,d) : = \frac{ \ker :E_{x,t} \to E_{s-r,t+r-1}}{ \im (d: E_{s+r,t-r+1} \toE_{s,t})}
\end{align*}
\end{defn}
If $ E_q = \bigoplus_{ s+t=q} E_{s,t}$ then $ d$ induces a homomorphism  $ \partial :E_q \to E_{q-1}).$ so $ (E_q, \partial )$ is a chain complex. Moreover, $ H_q(E_*, \partial ) = \bigoplus_{ s+t=q} H_{s,t}(E,d) $.
\begin{defn}
An  \allbold{$ E^{k}$-spectral sequence} is a sequence $ \{E^{r},d^{r}\} $  for $ r \geq k$  s.t.\ 
\begin{enumerate}[label=(\alph*)]
	\item $ E^{r}$ is a bigraded module and $ d^{r}$ is a differential of bidegree $ (-r,r-1)$,
	\item  $ E^{r+1} = H(E^{r},d^{r}) \ \forall \ r>k$.
\end{enumerate}
\end{defn}

\begin{eg}
$ E^{1}$.
\end{eg}

\begin{defn}
Suppose for every $ s,t$ there exists a  $ \# r(s,t)$  s.t.\ $ \ \forall \ r > r(s,t)$
\begin{align*}
	d^{r}:E_{s,t}^{r} \to E_{s-r,t+r-1}^{r}
\end{align*}
is the zero map, then $ E_{s,t}^{r+1}$ is just a quotient of $ E_{s,t}^{r}$ we can define $ E_{s,t}^{\infty}$ to be the direct limit of $ E_{s,t}^{r}$. In this situation we say the spectral sequence \allbold{coverges} to $ E_{s,t}^{\infty}$ .
\end{defn} 
\begin{remark}
If $ E_{s,t}^{r} = 0, s<0,t<0$ (first quadrant ss), then for each $ s,t$ there is some  $ r$  s.t.\ $ E_{s,t}^{r}$ is constant in $ r$.
\end{remark}

So where do spectral sequences come from? Filtrations.

\begin{defn}
A \allbold{filtration $ F$ on a module  $ A$}  is a sequence of submodules $ \{F_sA\} $ of $ A$  s.t.\ 
\begin{align*}
	A \supseteq \ldots \supseteq F_{s+1} A \supseteq F_s A \supseteq \ldots
\end{align*}
$ s$ is the  \allbold{filtration degree}, $ t$ is the  \allbold{complementary degree}, and $ s+t$ is the  \allbold{total degree}.   
A filtration is \allbold{convergent}  if
\begin{align*}
	\bigcap_{ s} F_sA = 0 
\end{align*}
and
\begin{align*}
	\bigcup_{ s} F_s A =A 
\end{align*}
We will usually have a finite filtration where $ F_{-1}(A) = 0$.

If $ A$ is graded  $ \{A_k\} $ and $ F$ respects the grading,  then the filtration inherits a grading $ F_sA = \{F_s A_k\} $. The \allbold{associated graded module} is
\begin{align*}
	G(A)_s = F_sA / F_{s-1}A\\
	G(A) = \bigoplus_{ s} G(A)_s 
\end{align*}
If $ A$ is graded then  $ G(A)$ is bigraded
 \begin{align*}
	G(A)_{s,t} = F_s A_{s+t} / F_{s-1} A_{s+t}
\end{align*}
\end{defn}

\begin{eg}
\begin{enumerate}[label=(\arabic*)]
	\item $ A= F_1A = \zz /4$, $ F_0A= \zz /2$, $ F_{-1}A=0$, so
		\begin{align*}
			G(A)_s = \begin{cases}
				\zz /2 & s=1,0\\
				0 & s \neq 0,1\\
			\end{cases}
		\end{align*}
		So $ G(A) = \bigoplus_{ s}G(A) = \zz /2 \oplus  \zz /2 $.
	\item $ A = F_1A = \zz_2 \oplus \zz_2$, $ F_0A = \zz_2$, $ F_{-1}A = 0$. Then
		\begin{align*}
			G(A)_s = \begin{cases}
				\zz /2 & i = 1,0\\
				0 & i \neq 0,1\\
			\end{cases}
		\end{align*}
	\item $ A = A_1 = \zz$, $ A_0 = 2 \zz$, $ A_{-1}=0$, then
		\begin{align*}
			G(A) = A_1 / A_0 \oplus  A_0 / A_{-1}\\
			\cong \zz_2 \oplus \zz \not \cong A
		\end{align*}
		But $ G(A)$ is ``close'' to  $ A$.
\end{enumerate}
\end{eg}
\begin{lem}
If $ F$ is a finite filtration of  $ A_1$ then
\begin{enumerate}[label=(\arabic*)]
	\item If $ G(A)_s$ is free for all  $ s$ then  $ G(A) \cong A$.
	\item If $ G(A)_s$ is a vector space over a field then  $ G(A) \cong A$.
	\item If $ G(A)_s$ is finite for all  $ s$, then  $ A$ is finite and order $ (A) = $order $ G(A)$. 
	\item If  $ G(A)_s$ is finitely generated  $ \ \forall \ s$, then $ A$ is finitely generated and  $ \rank A = \rank (G(A))$.
	\item If  $ G(A)_s =0$ for all but one  $ s$, then  $ A \cong G(A)$.
	\item If $ G(A)_s = 0$ for all but two  $ s$, say  $ G(A)_k, G(A)_\ell$ with $ k < \ell$ then
		\begin{align*}
			0 \to G(A)_k \to A \to G(A)_{\ell} \to 0
		\end{align*}
is exact.
\end{enumerate}
\end{lem}
\begin{proof}
\begin{enumerate}[label=(\arabic*)]
	\item We have
		\begin{align*}
			0 \to F_{n-1} A \to F_nA = A \to F_n A / F_{n-1} A = G(A)_n \to 0
		\end{align*}
		Exercise: if $ 0 \to A \to B \to C \to 0$ is exact and $ C$ is free, then  $ B = A \oplus C$. So $ A \cong G(A)_n \oplus F_{n-1}(A)$. Similarly,
		\begin{align*}
			$ F_{n-1}(A) \cong G(A)_{n-1} \oplus F_{n-2}(A)$ and so on. So $ A \cong \bigoplus_{ s} G(A)_s $.
		\end{align*}
	\item this is a corollary of 1.
	\item 
	\item
	\item 
	\begin{align*}
	G(A)_s = \begin{cases}
		F_s(A) & s=k\\
		0& s \neq k\\
	\end{cases}
\end{align*}
So $ G(A) = G(A)_k$ and
 \begin{align*}
	A = F_n(A) = F_{n-1}(A) = \ldots= F_k(A) \supseteq  F_{k-1}(A) = \ldots = F_{-1}A
\end{align*}
so $ A = F_k(A) = F_k(A) / F_{k-1}(A) = G(A)_k = G(A)$.

6): If
\begin{align*}
	G(A)_s = \begin{cases}
		C,& s=\ell\\
		B, & s = k\\
		0, & \text{ else} 
	\end{cases}
\end{align*}
Then
\begin{align*}
	A = F_n(A) = \ldots F_{\ell}(A) \supseteq F_{\ell-1}(A) = \ldots F_k(A) \supseteq F_{k-1}(A) = \ldots=F_{-1}(A) = 0
\end{align*}
So $ G(A)_{\ell} = A /F_{\ell-1}(A) \cong C$.
and
\begin{align*}
	B = G(A)_k = (F_{l-1}A = F_k(A)) / (F_{k-1}A = 0) = F_{\ell-1}A
\end{align*}
So $ A  /B = C$ iff
 \begin{align*}
	0 \to B \to A \to C \to 0
\end{align*}

\end{enumerate}

Exercise: prove 3-4.
\end{proof}

If $ F$ is a filtration of a chain complex $ \{C_*, \partial \} $ s.t.\ $ (F_s C_*, \partial )$ is a subcomplex of $ (C_*, \partial )$, then $ F$ induces a filtration on the homology  $ H_*(C_*, \partial )$
\begin{align*}
	F_s H(C_*, \partial ) = \im  (H(F_s C_*, \partial ) \to H_*(C_*,\partial ))
\end{align*}
If $ F$ is finite, then it is easy to see
 \begin{align*}
	H(C_*, \partial ) = \bigcup F_s H(C_*, \partial )\\\
	\bigcap (F_s H(C_*, \partial )) =0
\end{align*}
\begin{thm}
Let $ F$ be a finite filtration of a chain complex  $ (c_*, \partial )$, then there is an $ E^{1}$ spectral sequence with
\begin{enumerate}[label=(\arabic*)]
	\item $ E^{1}_{s,t} = H_{s+t}(F_s C / F_{s-1}C)$ 
	\item $ d^{1}$ is the connecting homomorphism of the triple $ (F_s C, F_{s-1}C,F_{s-2}C)$.
	\item $ G(H(C_*, \partial ))_{s,t} = E_{s,t}^{\infty}$.
\end{enumerate}
\end{thm}
\begin{proof}
\begin{align*}
	E_{s,t}^{r} = \frac{ \{ c \in F_s(C_{s,t}): \partial c \in F_{s-r} (C_{s+t+1})\}}{F_{s-1}C_{s+t} + \partial (F_{s+r-1} C_{s+t-1}) }
\end{align*}
and $ d^{n} = \partial $ applied to representatives of $ E_{s,t}^{r}$.
\end{proof}
For 2, recall
\begin{align*}
	0 \to F_{s-1}C / F_{s-2}C \to F_{s}C / F_{s-2} c \to F_s C / F_{s-1} C \to 0
\end{align*}
Third isomorphism theorem, induces a long exact sequence on homology.
\begin{align*}
	 H_k(F_s C / F_{s-1}C) \xrightarrow{ d^{1}} H_{k-1} (F_{s-1}C / F_{s-2}C) 
\end{align*}

\begin{eg}
Computing the homology of $ T^2$.
\end{eg}
\begin{eg}
Let $ A \subseteq X$ a subspace we get a filtration of $ C_*(X)$
 \begin{align*}
	 F_1C &= C_*(X)\\
	F_0C &= C_*(A)\\
	F_{-1}C &= 0 
\end{align*}
So we get a spectral sequence with
\begin{align*}
	E_{s,t}^{1} = H_{s+t}\left( \frac{F_s C_*}{ F_{s-1}C_*} \right)  \cong \begin{cases}
		H_{s+t}(X,A) & s=1\\
		H_{s+t} (A) & s=1\\
		0 & s \neq 0,1\\
	\end{cases}
\end{align*}
Note $ E^2 = E^{2+k}  = E^{\infty}$. So $ G(H(C_*(X)))_k = \bigoplus_{ s+t=k} E_{s,t}^{\infty} = E_{0,k}^{\infty} \oplus E_{1,k-1}^{\infty}$. Lemma 1 says we get
\begin{align*}
	0 \to H_k(A) / \im \partial \to H_k(X) \to \ker \partial \to 0
\end{align*}
Note this is equivalent to
\begin{align*}
	0 \to \im (\partial : H_{k+1}(X,A) \to H_k(A)) \to H_k(A) \xrightarrow{ i_8}  H_k(X) \to \ker (\partial :H_{k}(X,A) \to H_{k-1}(A)) \to 0
\end{align*}
If $ i: A \to X$ and $ j: X \to (X,A)$ inclusions. The above says $ \im \partial = \ker i_*$ and $ \ker \partial  = \im j_*$. This proves the hard part of the long exact sequence for the pair $ (X,A)$. So the spectral sequence generalizes long exact sequence.
\end{eg}
\begin{eg}
We will show for $ X$ CW complex , $H_*^{CW}(X) \cong H_*^{Sing}(X) $.

Let $ F_kC_*(X) = C_*(X^{(k)})$. This is a finite filtration so there exists a spectral sequence that converges to $ E_{s,t}^{\infty}$.

\begin{align*}
	E_{s,t}^{1} = H_{s+t}(F_s C / F_{s-1}C)\\
	&= H_{s+t}(C_*(X^{(s)})/ C_*(X^{(s-1)})) \\
	&= H_{s+t}(X^{(s)}, X^{(s-1)}) & \text{ by defn} \\
	&=\con\widetilde{ H}_{s,t}(X^{(s)} / X^{(s-1)})\\
	&= \widetilde{ H}_{s+t} ( \text{wedge of s-spheres} ) \\
	&= \begin{cases}
		\bigoplus_{ s- \text{cells} } \zz& t=0\\
		0& t \neq 0\\
	\end{cases} \\
	&= \begin{cases}
		C_s^{CW}(X) & t=0\\
		0& t\neq 0\\
	\end{cases} 
\end{align*}
and in sequence $ d^{1} = \partial $ map of the LES of $ (F_s ,F_{s-1}, F_{s-2})$ \emph{i.e.} $ H_s(X^{(s)} , X^{(s-1)}) \to H_{s-1} (X^{(s-1)} , X^{(s-2)})$. We know $ d^{1} = \partial ^{CW}$. So
\begin{align*}
	E^2 = E_{s,t}^{\infty} = \begin{cases}
		H_s^{CW}(X) & t=0\\
		0& t \neq 0\\
	\end{cases}
\end{align*}
By Lemma 1 part 5, we have $ H_p(X) \cong H_p ^{CW}(X)$.
\begin{remark}
All the above works for cohomology too and the $ \partial $ respect the product structure.
\end{remark}
\end{eg}
\end{document}
