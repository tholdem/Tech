\documentclass[12pt,class=article,crop=false]{standalone} 
\newcommand{\alert}[1]{{\bf \color{red} [Alert:] #1}}
\newcommand{\todo}[1]{{\bf \color{orange} [TODO:] #1}}
\newcommand{\real}[1][]{\mathbb{R}^{#1}}
\newcommand{\myeqn}[1]{(\ref{#1})}
\newcommand{\myex}[1]{Example \ref{#1}}
\newcommand{\defeq}{\stackrel{\mathrm{def}}{=}}
\newcommand{\parder}[2]{\frac{\partial #1}{\partial #2}}
\newcommand{\Lie}[3][]{\mathsf{L}_{#3}^{#1} #2}
\newcommand{\LieA}[1]{\mathsf{Lie}(#1)}
\newcommand{\lieder}[2]{\mathcal{L}_{#2} #1}
\renewcommand{\t}{^{\mbox{\tiny\sf T}}}
\newcommand{\trans}{^{\mbox{\tiny\sf T}}}
\newcommand{\markup}[1]{\{\textbf{#1}\}}
\newcommand{\msub}[1]{_\mathrm{#1}}
\newcommand{\msup}[1]{^\mathrm{#1}}
\newcommand{\inv}[1]{#1^{-1}}
\newcommand{\pinv}[1]{{#1}^{+}}
\newcommand{\myfracA}[2]{\displaystyle{\frac{#1}{#2}}}
\newcommand{\myfracB}[2]{{#1}/{#2}}
\newcommand{\mydiffA}[1]{\dot{#1}}
\newcommand{\mydiffB}[2]{\myfracA{\mathrm{d}{#1}}{\mathrm{d}{#2}}}
\newcommand{\ball}[2]{\mathcal{B}_{#1}\left(#2\right)}
\newcommand{\acos}[1]{\cos^{-1}\left(#1\right)}
\newcommand{\asin}[1]{\sin^{-1}\left(#1\right)}
\newcommand{\mani}{\mathcal{M}}
\newcommand{\tang}[2]{\mathsf{T}_{#1} #2}
\newcommand{\LieB}[2]{[ #1, #2 ]}
\newcommand{\LieBad}[3][]{\mathsf{ad}_{#2}^{#1} #3}
\newcommand{\ReachVT}{\mathcal{R}^V_T}
\newcommand{\ReachVt}{\mathcal{R}^V_t}
\newcommand{\ReachVTe}{\mathcal{R}^V_{\le T}}
\newcommand{\ReachT}{\mathcal{R}_T}
\newcommand{\Reacht}{\mathcal{R}_t}
\newcommand{\ReachTe}{\mathcal{R}_{\le T}}
\newcommand{\accLA}[1]{\mathsf{Lie}(#1)}
\newcommand{\accD}{\Delta_{\mathcal{F}}}
\newcommand{\accSA}{\mathsf{Lie}(\mathcal{G},f)}
\newcommand{\accDS}{\Delta_{\mathcal{G}}}
\newcommand{\eval}[3]{\mathsf{Ev}^{#2}_{#1}\left( #3 \right)}
\newcommand{\stlc}{\textsc{stlc}}
\newcommand{\clf}{\textsc{clf}}
\newcommand{\jqlf}{\textsc{jqlf}}
\newcommand{\dlas}{\textsc{dlas}}
\newcommand{\Ad}[2]{\mathsf{Ad}_{#1} #2}
\newcommand{\xe}{\ensuremath{x_e}}
\newcommand{\lebg}[1]{\mathcal{L}_{#1}}
\newcommand{\lebgx}[1]{\mathcal{L}_{#1 \mathrm{e}}}
\newcommand{\dom}{D}
\newcommand{\domT}{[t_0,\infty) \times D}
\newcommand{\rarrow}{\rightarrow}
\renewcommand{\d}{\mathrm{d}}
\renewcommand{\Re}{\mathbb{R}}
\newcommand{\C}{\mathrm{C}}

\newcommand{\QED}{{\unskip\nobreak\hfil\penalty50\hskip2em\vadjust{}
		\nobreak\hfil$\Box$\parfillskip=0pt\finalhyphendemerits=0\par}\vspace{0.1cm}}
\newcommand{\eoEx}{{\unskip\nobreak\hfil\penalty50\hskip0em\vadjust{}
		\nobreak\hfil$\Large\Diamond$\parfillskip=0pt\finalhyphendemerits=0\par}\vspace{0.1cm}}

\newcommand{\sgn}{\ensuremath{\operatorname{sgn}}}
\newcommand{\sat}{\ensuremath{\operatorname{sat}}}

\newcommand{\half}{\frac{1}{2}}
\newcommand{\shalf}{\mbox{$\frac{1}{2}$}}
\newcommand{\marcom}[1]{\marginpar{\footnotesize #1}}
\newcommand{\der}{\mathrm{D}}
\newcommand{\e}{\mathrm{e}}
\newcommand{\dt}{\mathrm{d}t}

\newcommand{\cA}{\ensuremath{\mathcal{A}}}
\newcommand{\cB}{\ensuremath{\mathcal{B}}}
\newcommand{\cG}{\ensuremath{\mathcal{G}}}
\newcommand{\cK}{\ensuremath{\mathcal{K}}}
\newcommand{\cW}{\ensuremath{\mathcal{W}}}
\newcommand{\cZ}{\ensuremath{\mathcal{Z}}}
\newcommand{\cS}{\ensuremath{\mathcal{S}}}
\newcommand{\cD}{\ensuremath{\mathcal{D}}}
\newcommand{\cP}{\ensuremath{\mathcal{P}}}
\newcommand{\cV}{\ensuremath{\mathcal{V}}}
\newcommand{\cL}{\ensuremath{\mathcal{L}}}
\newcommand{\cN}{\ensuremath{\mathcal{N}}}
\newcommand{\cI}{\ensuremath{\mathcal{I}}}
\newcommand{\cR}{\ensuremath{\mathcal{R}}}
\newcommand{\cM}{\ensuremath{\mathcal{M}}}
\newcommand{\cC}{\ensuremath{\mathcal{C}}}
\newcommand{\cF}{\ensuremath{\mathcal{F}}}
\newcommand{\cH}{\ensuremath{\mathcal{H}}}
\newcommand{\cO}{\ensuremath{\mathcal{O}}}
\newcommand{\cX}{\ensuremath{\mathcal{X}}}
\newcommand{\cY}{\ensuremath{\mathcal{Y}}}
\newcommand{\Ci}{\ensuremath{\mathcal{C}^\infty}}
\newcommand{\ISS}{\textsc{iss}}
\newcommand{\LISS}{\textsc{liss}}
\newcommand{\GAS}{\textsc{gas}}
\newcommand{\GS}{\textsc{gs}}
\newcommand{\LES}{\textsc{les}}
\newcommand{\GUAS}{\textsc{guas}}
\newcommand{\BIBO}{\textsc{bibo}}
\newcommand{\spec}{\ensuremath{\operatorname{spec}}}
\newcommand{\spn}{\ensuremath{\operatorname{span}}}
\renewcommand{\i}{\mathrm{i\,}}

\renewcommand{\implies}{\Rightarrow}

\renewcommand{\theenumi}{$\roman{enumi})$}
\renewcommand{\labelenumi}{\theenumi}

\font\ptmten=zptmcmrm scaled 1200
\newcommand{\w}{\mbox{{\ptmten w}}}
\newcommand{\z}{\mbox{{\ptmten z}}}
\renewcommand{\Re}{\mathbb{R}}

\newcommand{\cl}{\operatorname{cl}}
\newcommand{\intr}{\operatorname{int}}
\newcommand{\rank}{\operatorname{rank}}
\newcommand{\co}{\operatorname{co}}
\newcommand{\aff}{\operatorname{aff}}

\theoremstyle{plain}
\newtheorem{theorem}{Theorem}[chapter]
\newtheorem{claim}[theorem]{Claim}
\newtheorem{corollary}[theorem]{Corollary}
\newtheorem{prop}[theorem]{Proposition}
\newtheorem{fact}[theorem]{Fact}
\newtheorem{lemma}[theorem]{Lemma}

\newtheorem{remark}{Remark}[chapter]

\theoremstyle{definition}
\newtheorem{assume}[theorem]{Assumption}
\newtheorem{defn}[theorem]{Definition}
\newtheorem{problem}[theorem]{Problem}
\newtheorem{exercise}{Exercise}
\newtheorem{example}[theorem]{Example}


\begin{document}
When does a bundle have a section?

\begin{lem}
$ \delta \widetilde{ \sigma}(s_k) = 0$.
\end{lem}
\begin{proof}
Recall $ \partial ^{CW}: C_{k+1}(M) \to C_k(M)$ is as follows: for $ e_i^{k+1}$,
\begin{align*}
	\partial e_i^{k+1} = S^{k} \xrightarrow{ a_i} M^{(k)}  \to M^{k+1} / M^{k} \cong \bigvee S^{k} \xrightarrow{ p_j} S^{k} \\
	\partial ^{CW} e_i^{k+1} = \sum_i (\deg g_{ij}) e_j^{k}
\end{align*}
where $ g_{ij}$ is the composition map above. Then
\begin{align*}
	\delta: \Hom( C_{k-1}( M);G) \to \Hom( C_k(M);G )\\
	(\delta h)(e_i^{k}) := h ( \partial ^{CW} e_i^{k})
\end{align*}

$ \delta( \widetilde{ \sigma}(s_k)): C_{k+2}(M) \to \pi_k(F), e^{k+2} \mapsto \widetilde{ \sigma}(s_k)( \partial e^{k+2})$. Let $ a: \partial e^{k+2} \to M^{(k+1)}$ be the attaching map and $ I:e^{k+2} \to M$ be the ``inclusion". We homotop $ a$ as in exercise so
 \begin{align*}
	 (\delta \widetilde{ \sigma}(s_k))(e^{k+2}) = \widetilde{ \sigma}(s_k)( \sum d_j [e_j^{k+1}])
\end{align*}
where $ d_j$ are degrees of maps. As above $ I^* E \cong e^{k+2} \times F$ arrows into $ e^{k+2}$ and $ s_k$ gives a section above $ a(\partial d^{k+2} - \bigcup_{ n}D_n^{k+1})$ from exercise. And $ a|_{ \partial D_n^{k+1}}$ is the attaching map for some $e_i^{k+1} $. We can use $ p_2 \circ a|_{ \pm \partial D_n^{k+2}}$ to define $ \widetilde{ \sigma}(s_k) (e_i^{k+1})$. Hence the maps used in definition of $ \sum d_j \widetilde{ \sigma}(s_k)(e_j^{k+1})$ can be extended over $ \partial e^{k+2} - \bigcup D_n^{k+1} $.

Exercise: show this means $ \sum d_j \widetilde{ \sigma}(s_k)(e_j^{k+1})$ is 0 in $ \pi_k(F)$. Hint: first consider the case when there is only one $ D_n^{k+1}$. So this finishes the proof.
\end{proof}

Now suppose $ s_k,s_k'$ are two sections over $ M^{(k)}$ that agree on $ M^{(k-1)}$, then their \allbold{difference class} in $ C^{k}(M, \pi_k(F))$ is defined as follows:
\begin{align*}
	D(s_k,s_k'): C_k(M) \to \pi_k(F)
\end{align*}
Let $ I_i: e_i^{k} \to M$ be inclusion of a $ k$-cell. Then  $ I_i^* E \cong e_i^{k} \times F$. Now $ s_k|_{\partial e_i^{k}} = s_k'|_{\partial e_i^{k}}$. So putting them together as upper and lower hemispheres, we have a map $ p_2 \circ (s_k|_{e_i^{k}} - s_k'|_{e_i^{k}}): S^{k} \to F$. Define $ D(s_k,s_k')(e_i^{k}):= p_2 \circ (s_k|_{e_i^{k}} - s_k'|_{e_i^{k}}) \in \pi_k(F)$.

\begin{lem}
\begin{enumerate}[label=(\arabic*)]
	\item $ \delta(D(s_k,s_k'))= \widetilde{ \sigma}(s_k) - \widetilde{ \sigma}(s_k')$.
	\item given any $ s_k$ and $ h \in C^{k}(M;\pi_k(F))$, there exists $ s_k'$ s.t.\ $ D(s_k,s_k') = h$.
\end{enumerate}
\end{lem}
\begin{proof}
1 is similar to proof of lemma 1. Exercise.

Let $ s_k$ be a section over $ M$, for fix a $ k$-cell  $ e^{k}$, define
\begin{align*}
	h(e^{k}) := [g] \in \pi_k(F)
\end{align*}
and $ h=0$ on other  $ k$-cells. If we can find  $ s_k'$ s.t.\ $ D(s_k,s_k') = h$, then we are done (by doing it cell by cell). Let $ I:e^{k} \to M$ be the inclusion so pullback bundle is trivial. We choose a disk $ D^{k} \subseteq \int e^{k}$ and homotop $ s_k$ on $ e^{k} $ so $ p \circ s_k(D^{k}) = x_0 \in F$.

Let $ s_k'=s_k$ on $ M^{(k)} - D^{k}$ and on $ D^{k}$ let it be $ -g \in \pi_k(F)$. Clearly $ D(s_k,s_k') = h$.
\end{proof}

The two lemmas above give
\begin{thm}
Given a bundle $ (E,M,F,p)$ satisfying the three assumptions and a section  $ s_k:M^{(k)} \to E$ then $ s_k|_{M^{(k-1)}}$ extends to $ M^{(k+1)} \iff$ 
\begin{align*}
	\sigma(s_k) = [\widetilde{ \sigma}(s_k)] = 0 \in H^{k+1}(M; \pi_k(F)).
\end{align*}
\end{thm}

\begin{remark}
If $ \pi_k(F) = 0$ for all $ k< \dim M$, then the above shows there exists a section of  $ (E,M,F)$. In particular, if  $ F$ is contractible then any bundle with fiber  $ F$ has a section.
\end{remark}
\begin{remark}
$ \sigma(s_k)$ depends on $ s_k|_{M^{(k-1)}}$ so it is not an obstruction to the existence of a section of $ E$ over  $ M^{(k+1)}$ but only an obstruction to the existence of an extension of $ s_k|_{M^{(k-1)}}$ to $ M^{(k+1)}$.
\end{remark}
But the ``first obstruction" is independent of any choices and is ``natural".

\begin{thm}
Given a bundle satisfying the three assumptions, if $ \pi_k(F) = 0 $ for $ k<n$, then there exists a section  $ s_n:M^{(n)} \to E$ and the obstruction $ \sigma(s_n)$ does not depend on $ s_n$, \emph{i.e.} it is well-defined independent of choices. Denote $ \sigma(s_n)$ by $ \gamma^{n+1}(E)$, called the \allbold{primary obstruction}. And if $ f:N \to M$ is a map then
\begin{align*}
	\gamma^{n+1}(f^* E) = f^* ( \gamma^{n+1}(E)).
\end{align*}
\end{thm}
\begin{defn}
	$ \gamma^{n+1}$ is called a \allbold{characteristic class}.
\end{defn}
\begin{proof}
The discussion above says $ s_n$ exists since all obstructions vanish. You can develop an obstruction theory to homotoping one section to another: given $ s,s'$ agreeing on  $ k-1$-skeleton, then  $ s|_{M^{(k)}}$ is homotopic to $ s'|_{M^{(k)}} \iff \sigma(s,s') \in H^{k}(M;\pi_k(F))$ vanishes (by the same argument before). Exercise.

So there is a unique (up to homotopy) section of $ E$ over  $ M^{(n-1)}$. Thus $ \sigma(s_n)$ is independent of $ s_n$.

For naturality, WLOG suppose $ f:N \to M$ is a cellular map. Now a section $ s$ of  $ E \to M$ gives a section $ f^* (S)$ of $ f^* (E)$. Exercise: check this. For any $ \Phi:(D^{n+1}, \partial D^{n+1}) \to (N^{(n+1),N^{(n)}}$ we see
\begin{align*}
	\pi_{n+1}(N^{(n+1)},N^{(n)}) \xrightarrow{ f_*} \pi_{n+1} (M^{(n+1)}, M^{(n)})  \to \pi_n(F)\\
	[\Phi] \mapsto [f \circ \Phi] \mapsto [p_2 \circ s \circ f \circ \Phi_{ \partial D^{n+1}}]
\end{align*}
is essentially both $ \sigma(f^* s)(\Phi)$ and $ (f^* \sigma(s))(\Phi)$. Now $ pi_{n+1}(N^{(n+1)},N^{(n)} \cong H_{n+1} (N^{(n+1)},N^{(n)}) \cong C_{n+1}^{CW}(N)$. Similarly for $ M$. So the cocycle  $ \gamma^{n+1}(f^* E)$ is $ f^* ( \gamma^{n+1}(E))$.
\end{proof}
\end{document}
