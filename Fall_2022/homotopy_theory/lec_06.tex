\documentclass[12pt,class=article,crop=false]{standalone} 
\newcommand{\alert}[1]{{\bf \color{red} [Alert:] #1}}
\newcommand{\todo}[1]{{\bf \color{orange} [TODO:] #1}}
\newcommand{\real}[1][]{\mathbb{R}^{#1}}
\newcommand{\myeqn}[1]{(\ref{#1})}
\newcommand{\myex}[1]{Example \ref{#1}}
\newcommand{\defeq}{\stackrel{\mathrm{def}}{=}}
\newcommand{\parder}[2]{\frac{\partial #1}{\partial #2}}
\newcommand{\Lie}[3][]{\mathsf{L}_{#3}^{#1} #2}
\newcommand{\LieA}[1]{\mathsf{Lie}(#1)}
\newcommand{\lieder}[2]{\mathcal{L}_{#2} #1}
\renewcommand{\t}{^{\mbox{\tiny\sf T}}}
\newcommand{\trans}{^{\mbox{\tiny\sf T}}}
\newcommand{\markup}[1]{\{\textbf{#1}\}}
\newcommand{\msub}[1]{_\mathrm{#1}}
\newcommand{\msup}[1]{^\mathrm{#1}}
\newcommand{\inv}[1]{#1^{-1}}
\newcommand{\pinv}[1]{{#1}^{+}}
\newcommand{\myfracA}[2]{\displaystyle{\frac{#1}{#2}}}
\newcommand{\myfracB}[2]{{#1}/{#2}}
\newcommand{\mydiffA}[1]{\dot{#1}}
\newcommand{\mydiffB}[2]{\myfracA{\mathrm{d}{#1}}{\mathrm{d}{#2}}}
\newcommand{\ball}[2]{\mathcal{B}_{#1}\left(#2\right)}
\newcommand{\acos}[1]{\cos^{-1}\left(#1\right)}
\newcommand{\asin}[1]{\sin^{-1}\left(#1\right)}
\newcommand{\mani}{\mathcal{M}}
\newcommand{\tang}[2]{\mathsf{T}_{#1} #2}
\newcommand{\LieB}[2]{[ #1, #2 ]}
\newcommand{\LieBad}[3][]{\mathsf{ad}_{#2}^{#1} #3}
\newcommand{\ReachVT}{\mathcal{R}^V_T}
\newcommand{\ReachVt}{\mathcal{R}^V_t}
\newcommand{\ReachVTe}{\mathcal{R}^V_{\le T}}
\newcommand{\ReachT}{\mathcal{R}_T}
\newcommand{\Reacht}{\mathcal{R}_t}
\newcommand{\ReachTe}{\mathcal{R}_{\le T}}
\newcommand{\accLA}[1]{\mathsf{Lie}(#1)}
\newcommand{\accD}{\Delta_{\mathcal{F}}}
\newcommand{\accSA}{\mathsf{Lie}(\mathcal{G},f)}
\newcommand{\accDS}{\Delta_{\mathcal{G}}}
\newcommand{\eval}[3]{\mathsf{Ev}^{#2}_{#1}\left( #3 \right)}
\newcommand{\stlc}{\textsc{stlc}}
\newcommand{\clf}{\textsc{clf}}
\newcommand{\jqlf}{\textsc{jqlf}}
\newcommand{\dlas}{\textsc{dlas}}
\newcommand{\Ad}[2]{\mathsf{Ad}_{#1} #2}
\newcommand{\xe}{\ensuremath{x_e}}
\newcommand{\lebg}[1]{\mathcal{L}_{#1}}
\newcommand{\lebgx}[1]{\mathcal{L}_{#1 \mathrm{e}}}
\newcommand{\dom}{D}
\newcommand{\domT}{[t_0,\infty) \times D}
\newcommand{\rarrow}{\rightarrow}
\renewcommand{\d}{\mathrm{d}}
\renewcommand{\Re}{\mathbb{R}}
\newcommand{\C}{\mathrm{C}}

\newcommand{\QED}{{\unskip\nobreak\hfil\penalty50\hskip2em\vadjust{}
		\nobreak\hfil$\Box$\parfillskip=0pt\finalhyphendemerits=0\par}\vspace{0.1cm}}
\newcommand{\eoEx}{{\unskip\nobreak\hfil\penalty50\hskip0em\vadjust{}
		\nobreak\hfil$\Large\Diamond$\parfillskip=0pt\finalhyphendemerits=0\par}\vspace{0.1cm}}

\newcommand{\sgn}{\ensuremath{\operatorname{sgn}}}
\newcommand{\sat}{\ensuremath{\operatorname{sat}}}

\newcommand{\half}{\frac{1}{2}}
\newcommand{\shalf}{\mbox{$\frac{1}{2}$}}
\newcommand{\marcom}[1]{\marginpar{\footnotesize #1}}
\newcommand{\der}{\mathrm{D}}
\newcommand{\e}{\mathrm{e}}
\newcommand{\dt}{\mathrm{d}t}

\newcommand{\cA}{\ensuremath{\mathcal{A}}}
\newcommand{\cB}{\ensuremath{\mathcal{B}}}
\newcommand{\cG}{\ensuremath{\mathcal{G}}}
\newcommand{\cK}{\ensuremath{\mathcal{K}}}
\newcommand{\cW}{\ensuremath{\mathcal{W}}}
\newcommand{\cZ}{\ensuremath{\mathcal{Z}}}
\newcommand{\cS}{\ensuremath{\mathcal{S}}}
\newcommand{\cD}{\ensuremath{\mathcal{D}}}
\newcommand{\cP}{\ensuremath{\mathcal{P}}}
\newcommand{\cV}{\ensuremath{\mathcal{V}}}
\newcommand{\cL}{\ensuremath{\mathcal{L}}}
\newcommand{\cN}{\ensuremath{\mathcal{N}}}
\newcommand{\cI}{\ensuremath{\mathcal{I}}}
\newcommand{\cR}{\ensuremath{\mathcal{R}}}
\newcommand{\cM}{\ensuremath{\mathcal{M}}}
\newcommand{\cC}{\ensuremath{\mathcal{C}}}
\newcommand{\cF}{\ensuremath{\mathcal{F}}}
\newcommand{\cH}{\ensuremath{\mathcal{H}}}
\newcommand{\cO}{\ensuremath{\mathcal{O}}}
\newcommand{\cX}{\ensuremath{\mathcal{X}}}
\newcommand{\cY}{\ensuremath{\mathcal{Y}}}
\newcommand{\Ci}{\ensuremath{\mathcal{C}^\infty}}
\newcommand{\ISS}{\textsc{iss}}
\newcommand{\LISS}{\textsc{liss}}
\newcommand{\GAS}{\textsc{gas}}
\newcommand{\GS}{\textsc{gs}}
\newcommand{\LES}{\textsc{les}}
\newcommand{\GUAS}{\textsc{guas}}
\newcommand{\BIBO}{\textsc{bibo}}
\newcommand{\spec}{\ensuremath{\operatorname{spec}}}
\newcommand{\spn}{\ensuremath{\operatorname{span}}}
\renewcommand{\i}{\mathrm{i\,}}

\renewcommand{\implies}{\Rightarrow}

\renewcommand{\theenumi}{$\roman{enumi})$}
\renewcommand{\labelenumi}{\theenumi}

\font\ptmten=zptmcmrm scaled 1200
\newcommand{\w}{\mbox{{\ptmten w}}}
\newcommand{\z}{\mbox{{\ptmten z}}}
\renewcommand{\Re}{\mathbb{R}}

\newcommand{\cl}{\operatorname{cl}}
\newcommand{\intr}{\operatorname{int}}
\newcommand{\rank}{\operatorname{rank}}
\newcommand{\co}{\operatorname{co}}
\newcommand{\aff}{\operatorname{aff}}

\theoremstyle{plain}
\newtheorem{theorem}{Theorem}[chapter]
\newtheorem{claim}[theorem]{Claim}
\newtheorem{corollary}[theorem]{Corollary}
\newtheorem{prop}[theorem]{Proposition}
\newtheorem{fact}[theorem]{Fact}
\newtheorem{lemma}[theorem]{Lemma}

\newtheorem{remark}{Remark}[chapter]

\theoremstyle{definition}
\newtheorem{assume}[theorem]{Assumption}
\newtheorem{defn}[theorem]{Definition}
\newtheorem{problem}[theorem]{Problem}
\newtheorem{exercise}{Exercise}
\newtheorem{example}[theorem]{Example}


\begin{document}
\section{Locally trivial fibrations}
\begin{defn}
A \allbold{fiber bundle} (or a \allbold{locally trivial fibration} or a \allbold{twisted product} or a \allbold{fibration}) is a 4-tuple $ (E,B,F,p)$ where $ E,B,F$ are topological spaces and  $ p: E \to B$ continuous s.t.\ $ \ \forall \ x \in B$, there exists an open set $ U \subseteq B$ containing $ x$ and a homeomorphism
 \begin{align*}
	\phi: p^{-1}(U) \to U \times F
\end{align*}
s.t.\ $ \pi_1 \circ \phi = p$ where $ \pi_1$ is projection onto first factor. Here $ E$ is called the  \allbold{total space}, $ B$ is  \allbold{base space}, $ F$ is the  \allbold{fiber}, $ p$ is the  \allbold{projection}, $ \phi$ is a \allbold{local trivialization}.     
\end{defn}
\begin{eg}
~\begin{enumerate}[label=(\arabic*)]
	\item $ E=B \times F$.
	\item Mobius band: $ M= \rr^2 / (x,y) \sim (x+1,-y)$. Let $ q: \rr^2 \to M$ be the quotient map.
	\item $ S^{2n-1}$ be the unit sphere in $ \cc^{n}$. Recall $ S^{1}$ the unit circle in $ \cc$ acts on $ \cc^{2n-1}$ by
		\begin{align*}
			S^{1} \times S^{2n-1} \to S^{2n-1}, (\lambda (z_1,\ldots,z_n)) \mapsto (\lambda z_1,\ldots,\lambda z_n).
		\end{align*}
		Exercise: $ S^{2n-1} / S^{1} \cong \cc P^{n-1}$.

		Exercise: show $ (S^{2n-1}, \cc P^{n-1}, S^{1},p)$ is a fiber bundle.
	\item If $ G$ is a Lie group, and $ H$ a compact subgroup of  $ G$, then  $ (G,G /H,H,p)$ is a fiber bundle where $ p$ is canonical projection. Exercise.
\end{enumerate}
\end{eg}
\begin{eg}
\begin{enumerate}[label=(\arabic*)]
	\item $ O(n) = \{A \in GL(n, \rr): \langle Av, Aw \rangle = \langle v,w \rangle \} = \{A \in GL(n,\rr): A^{T} = A^{-1}\} $. And $ SO(n) = \{A \in O(n) : \det A = 1\} $. Recall from diff top that they are smooth manifold of dimension $ n(n-1) /2$.  $ O(n)$ has two components and  $ SO(n)$ is the component containing the identity. Exercise:  $ SO(1) = \{1\} $. $ SO(2) \cong S^{1}$. $ SO(3) \cong \rr P^3$.

		Notice $ SO(n) \leq SO(n+1)$. Exercise: prove that  $ SO(n+1) / SO(n) \cong S^{n}$. Hint: note the 1st column of $ B \in SO(n+1)$ is a unit vector in $ \rr^{n+1}$.
	\item Let $ V_{n,k}$ be orthonormal $ k$-frames (ordered $ k$ vectors) in  $ \rr^{n}$. Exercise: Steifel manifold $ V_{n,k} = O(n) / O(n-k)$. So $ V_{n,n} \cong O(n)$. $ V_{n,1} = S^{n-1}$. $ V_{n,n-1} \cong SO(n)$. Exercise: if $ k<n$, then  $ V_{n,k} \cong SO(n) / SO(n-k)$.
	\item $ G_{n,k}$ is the $ k$-dimensional subspaces in  $ \rr^{n}$. Exercise: $ G_{n,k} = O(n) / O(n-k) \times O(k)$.
	\item Recall the unitary group $ U(n) = \{A \in GL_{n, \cc}: \langle Av,Au \rangle = \langle v,u \rangle\} $ where $ \langle v,u \rangle = \overline{v} \cdot u$. Alternatively, $ U(n) = \{A \in GL(n, \cc) : \overline{A}^{T} = A^{-1}\} $. The special unitary group is $ SU(n) = \{A \in U(n): \det A =1\} $. From diff top, these are manifolds and $ \dim U(n) = n^2$ and $ \dim SU(n) = n^2-1$. Exercise: $ U(n) / SU(n) \cong S^{1}$. Exercise: $ U(1) \cong S^{1}$, $ SU(2) \cong S^3$, $ U(2) \cong S^3 \times S^{1}$. $ SU(n+1) / SU(n) \cong S^{2n+1}$. 
	\item $ V_{n,k}( \cc) \cong U(n) / U(n-k)$.
	\item $ G_{n,k}( \cc) \cong = U(n) / U(k) \times U(n-k)$.
	\item If $ f: M \to N$ a smooth map, s.t.\ 
		\begin{enumerate}[label=(\roman*)]
			\item $ f$ is surjective
			\item  $ f$ is a submersion
			\item  $ f$ is proper  \emph{i.e.} preimage of compact set is compact.
		\end{enumerate}
		Then $ f^{-1}(p)$ where $ p$ is any point is a fiber bundle. This is Ehresmann's Theorem.
	\item vector bundles are fiber bundles with fiber $ \rr^{k}$ or $ \cc^{k}$ with extra structure on the fibers. This includes the tangent bundles, cotangent bundles, normal bundles.
	\item covering space is a bundle with discrete fiber.
\end{enumerate}
\end{eg}

\begin{defn}
Given a fiber bundle $ E \xrightarrow{ p} B$ and a map $ f: A \to B$, the \allbold{pull-back} of $ E$ to A is
 \begin{align*}
	f^* (E) = \{(a,e) \in A \times E: f(a) = p(e)\} . 
\end{align*}
\end{defn}
\begin{align*}
		p: f^* E \to A: (a,e) \mapsto a.
\end{align*}

Exercise:
\begin{enumerate}[label=(\arabic*)]
	\item Show $ f^* E \to A$ is a fiber bundle with the same fiber as $ E \to B$.
	\item If $ A$ is a subset of  $ B$ and  $ f: A \to B$ is inclusion, then show $ f^* (E) \cong E|_A$ \emph{i.e.} $ E|_A = p^{-1}(A)$.
	\item $ \widetilde{ f} : f^* E \to E, (a,e)\mapsto e$ is a bundle map so the diagram commutes.
	\item If $ f: A \to B$ is constant and fiber of $ E$ is  $ F$, then  $ f^* E \cong A \times F$.
	\item If $ E = B \times F$ then $ f^* E \cong A \times F$. 
\end{enumerate}
\begin{defn}
If $ E \xrightarrow{ p} B $ are $ E' \xrightarrow{ p'} B $ are bundles, we say they are \allbold{bundle isomorphic} if there exists a homeomorphism $ h: E \to E'$ s.t.\  the diagram commutes. We denote $ E \cong E'$.
\end{defn}

\begin{thm}
If $ f_i: A \to B$, $ i=0,1$ are homotopic and  $ A$ is locally compact and normal (\emph{e.g.} a CW complex), then  $ f_0^* E \cong f_1^* E$.
\end{thm}
\begin{proof}
Let $ f_i: A \to B$ and homotopy $H: A \times I \to B$. Diagrams. Theorem 2 says there exists a homotopy $ \widetilde{ H}:$

$ H^* E = \{(x,t,e) \in A \times I \times E: H(x,t) = p(e)\} $. Define 
 \begin{align*}
	\overline{H} ((x,e),t) = (x,t, \widetilde{ H}(x,e,t)).
\end{align*}
% https://q.uiver.app/?q=WzAsNyxbMCwwLCJmXzBeKkUgXFx0aW1lcyBJIl0sWzAsMSwiQSBcXHRpbWVzIEkiXSxbMSwwLCJFIl0sWzEsMSwiQiJdLFsyLDAsImZfMF4qRSBcXHRpbWVzIEkgIl0sWzIsMSwiQSBcXHRpbWVzIEkiXSxbMywwLCJIXipFIl0sWzAsMSwiXFxwaV8xIFxcdGltZXMgXFx0ZXh0e2lkfV9JIiwyXSxbMCwyLCJcXHRpbGRle0h9Il0sWzEsMywiSCJdLFsyLDMsInAiLDJdLFs0LDUsIlxccGlfMSBcXHRpbWVzIFxcdGV4dHtpZH1fSSIsMl0sWzQsNiwiXFxvdmVybGluZXtIfSJdLFs2LDUsIlxccGlfMSIsMV1d
\[\begin{tikzcd}
	{f_0^*E \times I} & E & {f_0^*E \times I } & {H^*E} \\
	{A \times I} & B & {A \times I}
	\arrow["{\pi_1 \times \text{id}_I}"', from=1-1, to=2-1]
	\arrow["{\tilde{H}}", from=1-1, to=1-2]
	\arrow["H", from=2-1, to=2-2]
	\arrow["p"', from=1-2, to=2-2]
	\arrow["{\pi_1 \times \text{id}_I}"', from=1-3, to=2-3]
	\arrow["{\overline{H}}", from=1-3, to=1-4]
	\arrow["{\pi_1}"{description}, from=1-4, to=2-3]
\end{tikzcd}\]
Exercise: $ \overline{H}$ is a bundle isomorphism.


Restricting $ \overline{H}$ to $ f_0^* E \times \{1\}$ yields a bundle isomorphism. Notice $ \overline{H}(f_0^* E \times \{1\} ) = \{(x,1,e) \in A \times I \times E: H(x,1) = f_1(x) = p(e)\} =f_1^* E$. Hence $ f_0^* E \cong f_1^* E$.
\end{proof}

\begin{thm}[covering homotopy property]
Let $ p: E \to B$ and $ q: Z \to Y$ be fiber bundles with the same fiber. Suppose $ B $ is locally compact and normal. Given  $ \widetilde{ h}_0: E \to Z, h_0: B \to Y$ s.t.\  the following diagram commutes,
% https://q.uiver.app/?q=WzAsNCxbMCwwLCJFIl0sWzAsMSwiQiJdLFsxLDAsIloiXSxbMSwxLCJZIl0sWzAsMiwiXFx0aWxkZXtofV8wIl0sWzIsMywicSJdLFswLDEsInAiLDJdLFsxLDMsImhfMCJdXQ==
\[\begin{tikzcd}
	E & Z \\
	B & Y
	\arrow["{\tilde{h}_0}", from=1-1, to=1-2]
	\arrow["q", from=1-2, to=2-2]
	\arrow["p"', from=1-1, to=2-1]
	\arrow["{h_0}", from=2-1, to=2-2]
\end{tikzcd}\]
and a homotopy $ H: B \times I \to Y$ of $ h_0$, then there exists a homotopy $ \widetilde{ H}: E \times I \to Z$ of bundle maps covering $ H$.
\end{thm}
\begin{proof}
We assume $ B$ is compact (locally compact case is an exercise). Idea: break  $ Z$ into pieces where bundle is trivial  $ U \times F$. Here the theorem is clear. Then we put the homotopies together.

Let $ \{V_{ \beta}\} $ be a cover of $ Y$ by locally trivializing charts so we have an isomorphism
 \begin{align*}
	q^{-1}(V_{ \beta}) \xrightarrow{ \phi _{ \beta}}  V_{ \beta} \times F
\end{align*}
$ \{H ^{-1}(V _{ \beta}\}) $ is an open cover of $ B \times I$ since $ B \times I$ is compact, we have a finite subcover $ \{U _{ \alpha} \times I_j\} $ covering $ B \times I$ s.t.\ $ H(U_{ \alpha} \times I_j) \subseteq V_{ \beta}$ for some $ \beta$. Note: $ H^* Z$ is trivial over $ U_{ \alpha} \times I_j$ since $ Z$ is trivial over  $ V_{ \beta}$. We can take the $ I_j$ to be segements. We will inductively lift $ H$ to  $ \widetilde{ H}: E \times [0,t_{k}] \to Z$. For each $ x \in B$ there exists neighborhoods  $ W,W'$  s.t.\ $ x \in W \subseteq \overline{W} \subseteq W'$ and $ \overline{W}' \subseteq U_i$ for some $ i$ by normal. There are finite number of $ \{W_i,W_i'\}_{i=1}^{s} $ s.t.\ $ \{W_i\} $ cover $ B$. By Urysohn's lemma, there exist maps  $ u_i: B \to [t_k,t_{k+1}]$ s.t.\ $ u_i(\overline{W}_i) = t_{k+1}$ and $ u_i(B - W_i') = t_k$. Set $ \tau_0(x) = t_k \ \forall \ x$ and $ \tau_i(x) = \max \{u_1(x),\ldots,u_i(x)\} $. So $ t_k = \tau_0(x) \leq \tau_1(x) \leq \ldots\leq \tau_s(x) = t_{k+1}$. Set $ B_i = \{(x,t) \in B \times I: t_k \leq t \leq \tau_i(x)\} $. Let $ E_i$ be the part of $ E \times I$ above $ B_i$. So $ E_0= E \times \{t_k\} \subseteq E_1 \subseteq \ldots \subseteq E_s = E \times [t_k, t_{k+1}]$. Assume we have $ \widetilde{ H}$ defined on $ E \times [0,t_k]$ so it is defined on $ E \times \{t_k\} = E_0$. We inductively extend $ \widetilde{ H}$ over $ E_i$. Note if $ (x,t) \in B_i - B_{i-1}$, then $ \tau_{i-1} < t \leq \tau_i(x)$. So $ u_i(x) > \tau_{i-1}(x)$. Thus $ x(,t) \in W_i' \times \{t_k,t_{k+1}\} $. By definition $ W_i' \times [t_k,t_{k+1}] \subseteq U_{ \alpha} \times I_j$. So $ H(B_i-B_{i-1}) \subseteq V_ \beta$ for some $ \beta$ and $ q^{-1}(V_ \beta) \xrightarrow{ \phi_{ \beta}} V_{ \beta} \times F$. Let $ \rho_{ \beta}: q^{-1}(V_{ \beta}) \to F$ be $ \phi_{ \beta}$ composed with projection. For $ (e,t) \in E_i - E_{i-1}$, let $ p(e) = x \in B$. Set $ \widetilde{ H}(e,t) = \phi_{ \beta}^{-1}(H(x,t), \rho_{ \beta}(\widetilde{ H}(e, \tau_{i-1}(x))))$. Exercise: show this extends $ \widetilde{ H}$ over $ E_i$.
\end{proof}

\begin{coro}
If $ X$ is contractible and locally compact and normal, then any fiber bundle over  $ X$ is trivial,  \emph{i.e.} $ E \cong X \times F$.
\end{coro}
\begin{proof}
X contractible means the identity map $ f_0$ is homotopy to the constant map $ f_1$. Therefore, $ f_0^* E \cong E \cong f_1^* E \cong X \times F $.
\end{proof}


\end{document}
