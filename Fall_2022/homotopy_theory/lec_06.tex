\documentclass[12pt,class=article,crop=false]{standalone} 
%Fall 2022
% Some basic packages
\usepackage{standalone}[subpreambles=true]
\usepackage[utf8]{inputenc}
\usepackage[T1]{fontenc}
\usepackage{textcomp}
\usepackage[english]{babel}
\usepackage{url}
\usepackage{graphicx}
%\usepackage{quiver}
\usepackage{float}
\usepackage{enumitem}
\usepackage{lmodern}
\usepackage{comment}
\usepackage{hyperref}
\usepackage[usenames,svgnames,dvipsnames]{xcolor}
\usepackage[margin=1in]{geometry}
\usepackage{pdfpages}

\pdfminorversion=7

% Don't indent paragraphs, leave some space between them
\usepackage{parskip}

% Hide page number when page is empty
\usepackage{emptypage}
\usepackage{subcaption}
\usepackage{multicol}
\usepackage[b]{esvect}

% Math stuff
\usepackage{amsmath, amsfonts, mathtools, amsthm, amssymb}
\usepackage{bbm}
\usepackage{stmaryrd}
\allowdisplaybreaks

% Fancy script capitals
\usepackage{mathrsfs}
\usepackage{cancel}
% Bold math
\usepackage{bm}
% Some shortcuts
\newcommand{\rr}{\ensuremath{\mathbb{R}}}
\newcommand{\zz}{\ensuremath{\mathbb{Z}}}
\newcommand{\qq}{\ensuremath{\mathbb{Q}}}
\newcommand{\nn}{\ensuremath{\mathbb{N}}}
\newcommand{\ff}{\ensuremath{\mathbb{F}}}
\newcommand{\cc}{\ensuremath{\mathbb{C}}}
\newcommand{\ee}{\ensuremath{\mathbb{E}}}
\newcommand{\hh}{\ensuremath{\mathbb{H}}}
\renewcommand\O{\ensuremath{\emptyset}}
\newcommand{\norm}[1]{{\left\lVert{#1}\right\rVert}}
\newcommand{\dbracket}[1]{{\left\llbracket{#1}\right\rrbracket}}
\newcommand{\ve}[1]{{\bm{#1}}}
\newcommand\allbold[1]{{\boldmath\textbf{#1}}}
\DeclareMathOperator{\lcm}{lcm}
\DeclareMathOperator{\im}{im}
\DeclareMathOperator{\coim}{coim}
\DeclareMathOperator{\dom}{dom}
\DeclareMathOperator{\tr}{tr}
\DeclareMathOperator{\rank}{rank}
\DeclareMathOperator*{\var}{Var}
\DeclareMathOperator*{\ev}{E}
\DeclareMathOperator{\dg}{deg}
\DeclareMathOperator{\aff}{aff}
\DeclareMathOperator{\conv}{conv}
\DeclareMathOperator{\inte}{int}
\DeclareMathOperator*{\argmin}{argmin}
\DeclareMathOperator*{\argmax}{argmax}
\DeclareMathOperator{\graph}{graph}
\DeclareMathOperator{\sgn}{sgn}
\DeclareMathOperator*{\Rep}{Rep}
\DeclareMathOperator{\Proj}{Proj}
\DeclareMathOperator{\mat}{mat}
\DeclareMathOperator{\diag}{diag}
\DeclareMathOperator{\aut}{Aut}
\DeclareMathOperator{\gal}{Gal}
\DeclareMathOperator{\inn}{Inn}
\DeclareMathOperator{\edm}{End}
\DeclareMathOperator{\Hom}{Hom}
\DeclareMathOperator{\ext}{Ext}
\DeclareMathOperator{\tor}{Tor}
\DeclareMathOperator{\Span}{Span}
\DeclareMathOperator{\Stab}{Stab}
\DeclareMathOperator{\cont}{cont}
\DeclareMathOperator{\Ann}{Ann}
\DeclareMathOperator{\Div}{div}
\DeclareMathOperator{\curl}{curl}
\DeclareMathOperator{\nat}{Nat}
\DeclareMathOperator{\gr}{Gr}
\DeclareMathOperator{\vect}{Vect}
\DeclareMathOperator{\id}{id}
\DeclareMathOperator{\Mod}{Mod}
\DeclareMathOperator{\sign}{sign}
\DeclareMathOperator{\Surf}{Surf}
\DeclareMathOperator{\fcone}{fcone}
\DeclareMathOperator{\Rot}{Rot}
\DeclareMathOperator{\grad}{grad}
\DeclareMathOperator{\atan2}{atan2}
\DeclareMathOperator{\Ric}{Ric}
\let\vec\relax
\DeclareMathOperator{\vec}{vec}
\let\Re\relax
\DeclareMathOperator{\Re}{Re}
\let\Im\relax
\DeclareMathOperator{\Im}{Im}
% Put x \to \infty below \lim
\let\svlim\lim\def\lim{\svlim\limits}

%wide hat
\usepackage{scalerel,stackengine}
\stackMath
\newcommand*\wh[1]{%
\savestack{\tmpbox}{\stretchto{%
  \scaleto{%
    \scalerel*[\widthof{\ensuremath{#1}}]{\kern-.6pt\bigwedge\kern-.6pt}%
    {\rule[-\textheight/2]{1ex}{\textheight}}%WIDTH-LIMITED BIG WEDGE
  }{\textheight}% 
}{0.5ex}}%
\stackon[1pt]{#1}{\tmpbox}%
}
\parskip 1ex

%Make implies and impliedby shorter
\let\implies\Rightarrow
\let\impliedby\Leftarrow
\let\iff\Leftrightarrow
\let\epsilon\varepsilon

% Add \contra symbol to denote contradiction
\usepackage{stmaryrd} % for \lightning
\newcommand\contra{\scalebox{1.5}{$\lightning$}}

% \let\phi\varphi

% Command for short corrections
% Usage: 1+1=\correct{3}{2}

\definecolor{correct}{HTML}{009900}
\newcommand\correct[2]{\ensuremath{\:}{\color{red}{#1}}\ensuremath{\to }{\color{correct}{#2}}\ensuremath{\:}}
\newcommand\green[1]{{\color{correct}{#1}}}

% horizontal rule
\newcommand\hr{
    \noindent\rule[0.5ex]{\linewidth}{0.5pt}
}

% hide parts
\newcommand\hide[1]{}

% si unitx
\usepackage{siunitx}
\sisetup{locale = FR}

%allows pmatrix to stretch
\makeatletter
\renewcommand*\env@matrix[1][\arraystretch]{%
  \edef\arraystretch{#1}%
  \hskip -\arraycolsep
  \let\@ifnextchar\new@ifnextchar
  \array{*\c@MaxMatrixCols c}}
\makeatother

\renewcommand{\arraystretch}{0.8}

\renewcommand{\baselinestretch}{1.5}

\usepackage{graphics}
\usepackage{epstopdf}

\RequirePackage{hyperref}
%%
%% Add support for color in order to color the hyperlinks.
%% 
\hypersetup{
  colorlinks = true,
  urlcolor = blue,
  citecolor = blue
}
%%fakesection Links
\hypersetup{
    colorlinks,
    linkcolor={red!50!black},
    citecolor={green!50!black},
    urlcolor={blue!80!black}
}
%customization of cleveref
\RequirePackage[capitalize,nameinlink]{cleveref}[0.19]

% Per SIAM Style Manual, "section" should be lowercase
\crefname{section}{section}{sections}
\crefname{subsection}{subsection}{subsections}
\Crefname{section}{Section}{Sections}
\Crefname{subsection}{Subsection}{Subsections}

% Per SIAM Style Manual, "Figure" should be spelled out in references
\Crefname{figure}{Figure}{Figures}

% Per SIAM Style Manual, don't say equation in front on an equation.
\crefformat{equation}{\textup{#2(#1)#3}}
\crefrangeformat{equation}{\textup{#3(#1)#4--#5(#2)#6}}
\crefmultiformat{equation}{\textup{#2(#1)#3}}{ and \textup{#2(#1)#3}}
{, \textup{#2(#1)#3}}{, and \textup{#2(#1)#3}}
\crefrangemultiformat{equation}{\textup{#3(#1)#4--#5(#2)#6}}%
{ and \textup{#3(#1)#4--#5(#2)#6}}{, \textup{#3(#1)#4--#5(#2)#6}}{, and \textup{#3(#1)#4--#5(#2)#6}}

% But spell it out at the beginning of a sentence.
\Crefformat{equation}{#2Equation~\textup{(#1)}#3}
\Crefrangeformat{equation}{Equations~\textup{#3(#1)#4--#5(#2)#6}}
\Crefmultiformat{equation}{Equations~\textup{#2(#1)#3}}{ and \textup{#2(#1)#3}}
{, \textup{#2(#1)#3}}{, and \textup{#2(#1)#3}}
\Crefrangemultiformat{equation}{Equations~\textup{#3(#1)#4--#5(#2)#6}}%
{ and \textup{#3(#1)#4--#5(#2)#6}}{, \textup{#3(#1)#4--#5(#2)#6}}{, and \textup{#3(#1)#4--#5(#2)#6}}

% Make number non-italic in any environment.
\crefdefaultlabelformat{#2\textup{#1}#3}

% Environments
\makeatother
% For box around Definition, Theorem, \ldots
%%fakesection Theorems
\usepackage{thmtools}
\usepackage[framemethod=TikZ]{mdframed}

\theoremstyle{definition}
\mdfdefinestyle{mdbluebox}{%
	roundcorner = 10pt,
	linewidth=1pt,
	skipabove=12pt,
	innerbottommargin=9pt,
	skipbelow=2pt,
	nobreak=true,
	linecolor=blue,
	backgroundcolor=TealBlue!5,
}
\declaretheoremstyle[
	headfont=\sffamily\bfseries\color{MidnightBlue},
	mdframed={style=mdbluebox},
	headpunct={\\[3pt]},
	postheadspace={0pt}
]{thmbluebox}

\mdfdefinestyle{mdredbox}{%
	linewidth=0.5pt,
	skipabove=12pt,
	frametitleaboveskip=5pt,
	frametitlebelowskip=0pt,
	skipbelow=2pt,
	frametitlefont=\bfseries,
	innertopmargin=4pt,
	innerbottommargin=8pt,
	nobreak=false,
	linecolor=RawSienna,
	backgroundcolor=Salmon!5,
}
\declaretheoremstyle[
	headfont=\bfseries\color{RawSienna},
	mdframed={style=mdredbox},
	headpunct={\\[3pt]},
	postheadspace={0pt},
]{thmredbox}

\declaretheorem[%
style=thmbluebox,name=Theorem,numberwithin=section]{thm}
\declaretheorem[style=thmbluebox,name=Lemma,sibling=thm]{lem}
\declaretheorem[style=thmbluebox,name=Proposition,sibling=thm]{prop}
\declaretheorem[style=thmbluebox,name=Corollary,sibling=thm]{coro}
\declaretheorem[style=thmredbox,name=Example,sibling=thm]{eg}

\mdfdefinestyle{mdgreenbox}{%
	roundcorner = 10pt,
	linewidth=1pt,
	skipabove=12pt,
	innerbottommargin=9pt,
	skipbelow=2pt,
	nobreak=true,
	linecolor=ForestGreen,
	backgroundcolor=ForestGreen!5,
}

\declaretheoremstyle[
	headfont=\bfseries\sffamily\color{ForestGreen!70!black},
	bodyfont=\normalfont,
	spaceabove=2pt,
	spacebelow=1pt,
	mdframed={style=mdgreenbox},
	headpunct={ --- },
]{thmgreenbox}

\declaretheorem[style=thmgreenbox,name=Definition,sibling=thm]{defn}

\mdfdefinestyle{mdgreenboxsq}{%
	linewidth=1pt,
	skipabove=12pt,
	innerbottommargin=9pt,
	skipbelow=2pt,
	nobreak=true,
	linecolor=ForestGreen,
	backgroundcolor=ForestGreen!5,
}
\declaretheoremstyle[
	headfont=\bfseries\sffamily\color{ForestGreen!70!black},
	bodyfont=\normalfont,
	spaceabove=2pt,
	spacebelow=1pt,
	mdframed={style=mdgreenboxsq},
	headpunct={},
]{thmgreenboxsq}
\declaretheoremstyle[
	headfont=\bfseries\sffamily\color{ForestGreen!70!black},
	bodyfont=\normalfont,
	spaceabove=2pt,
	spacebelow=1pt,
	mdframed={style=mdgreenboxsq},
	headpunct={},
]{thmgreenboxsq*}

\mdfdefinestyle{mdblackbox}{%
	skipabove=8pt,
	linewidth=3pt,
	rightline=false,
	leftline=true,
	topline=false,
	bottomline=false,
	linecolor=black,
	backgroundcolor=RedViolet!5!gray!5,
}
\declaretheoremstyle[
	headfont=\bfseries,
	bodyfont=\normalfont\small,
	spaceabove=0pt,
	spacebelow=0pt,
	mdframed={style=mdblackbox}
]{thmblackbox}

\theoremstyle{plain}
\declaretheorem[name=Question,sibling=thm,style=thmblackbox]{ques}
\declaretheorem[name=Remark,sibling=thm,style=thmgreenboxsq]{remark}
\declaretheorem[name=Remark,sibling=thm,style=thmgreenboxsq*]{remark*}
\newtheorem{ass}[thm]{Assumptions}

\theoremstyle{definition}
\newtheorem*{problem}{Problem}
\newtheorem{claim}[thm]{Claim}
\theoremstyle{remark}
\newtheorem*{case}{Case}
\newtheorem*{notation}{Notation}
\newtheorem*{note}{Note}
\newtheorem*{motivation}{Motivation}
\newtheorem*{intuition}{Intuition}
\newtheorem*{conjecture}{Conjecture}

% Make section starts with 1 for report type
%\renewcommand\thesection{\arabic{section}}

% End example and intermezzo environments with a small diamond (just like proof
% environments end with a small square)
\usepackage{etoolbox}
\AtEndEnvironment{vb}{\null\hfill$\diamond$}%
\AtEndEnvironment{intermezzo}{\null\hfill$\diamond$}%
% \AtEndEnvironment{opmerking}{\null\hfill$\diamond$}%

% Fix some spacing
% http://tex.stackexchange.com/questions/22119/how-can-i-change-the-spacing-before-theorems-with-amsthm
\makeatletter
\def\thm@space@setup{%
  \thm@preskip=\parskip \thm@postskip=0pt
}

% Fix some stuff
% %http://tex.stackexchange.com/questions/76273/multiple-pdfs-with-page-group-included-in-a-single-page-warning
\pdfsuppresswarningpagegroup=1


% My name
\author{Jaden Wang}



\begin{document}
\section{Locally trivial fibrations}
\begin{defn}
A \allbold{fiber bundle} (or a \allbold{locally trivial fibration} or a \allbold{twisted product} or a \allbold{fibration}) is a 4-tuple $ (E,B,F,p)$ where $ E,B,F$ are topological spaces and  $ p: E \to B$ continuous s.t.\ $ \ \forall \ x \in B$, there exists an open set $ U \subseteq B$ containing $ x$ and a homeomorphism
 \begin{align*}
	\phi: p^{-1}(U) \to U \times F
\end{align*}
s.t.\ $ \pi_1 \circ \phi = p$ where $ \pi_1$ is projection onto first factor. Here $ E$ is called the  \allbold{total space}, $ B$ is  \allbold{base space}, $ F$ is the  \allbold{fiber}, $ p$ is the  \allbold{projection}, $ \phi$ is a \allbold{local trivialization}.     
\end{defn}
\begin{eg}
~\begin{enumerate}[label=(\arabic*)]
	\item $ E=B \times F$.
	\item Mobius band: $ M= \rr^2 / (x,y) \sim (x+1,-y)$. Let $ q: \rr^2 \to M$ be the quotient map.
	\item $ S^{2n-1}$ be the unit sphere in $ \cc^{n}$. Recall $ S^{1}$ the unit circle in $ \cc$ acts on $ \cc^{2n-1}$ by
		\begin{align*}
			S^{1} \times S^{2n-1} \to S^{2n-1}, (\lambda (z_1,\ldots,z_n)) \mapsto (\lambda z_1,\ldots,\lambda z_n).
		\end{align*}
		Exercise: $ S^{2n-1} / S^{1} \cong \cc P^{n-1}$.

		Exercise: show $ (S^{2n-1}, \cc P^{n-1}, S^{1},p)$ is a fiber bundle.
	\item If $ G$ is a Lie group, and $ H$ a compact subgroup of  $ G$, then  $ (G,G /H,H,p)$ is a fiber bundle where $ p$ is canonical projection. Exercise.
\end{enumerate}
\end{eg}
\begin{eg}
\begin{enumerate}[label=(\arabic*)]
	\item $ O(n) = \{A \in GL(n, \rr): \langle Av, Aw \rangle = \langle v,w \rangle \} = \{A \in GL(n,\rr): A^{T} = A^{-1}\} $. And $ SO(n) = \{A \in O(n) : \det A = 1\} $. Recall from diff top that they are smooth manifold of dimension $ n(n-1) /2$.  $ O(n)$ has two components and  $ SO(n)$ is the component containing the identity. Exercise:  $ SO(1) = \{1\} $. $ SO(2) \cong S^{1}$. $ SO(3) \cong \rr P^3$.

		Notice $ SO(n) \leq SO(n+1)$. Exercise: prove that  $ SO(n+1) / SO(n) \cong S^{n}$. Hint: note the 1st column of $ B \in SO(n+1)$ is a unit vector in $ \rr^{n+1}$.
	\item Let $ V_{n,k}$ be orthonormal $ k$-frames (ordered $ k$ vectors) in  $ \rr^{n}$. Exercise: Steifel manifold $ V_{n,k} = O(n) / O(n-k)$. So $ V_{n,n} \cong O(n)$. $ V_{n,1} = S^{n-1}$. $ V_{n,n-1} \cong SO(n)$. Exercise: if $ k<n$, then  $ V_{n,k} \cong SO(n) / SO(n-k)$.
	\item $ G_{n,k}$ is the $ k$-dimensional subspaces in  $ \rr^{n}$. Exercise: $ G_{n,k} = O(n) / O(n-k) \times O(k)$.
	\item Recall the unitary group $ U(n) = \{A \in GL_{n, \cc}: \langle Av,Au \rangle = \langle v,u \rangle\} $ where $ \langle v,u \rangle = \overline{v} \cdot u$. Alternatively, $ U(n) = \{A \in GL(n, \cc) : \overline{A}^{T} = A^{-1}\} $. The special unitary group is $ SU(n) = \{A \in U(n): \det A =1\} $. From diff top, these are manifolds and $ \dim U(n) = n^2$ and $ \dim SU(n) = n^2-1$. Exercise: $ U(n) / SU(n) \cong S^{1}$. Exercise: $ U(1) \cong S^{1}$, $ SU(2) \cong S^3$, $ U(2) \cong S^3 \times S^{1}$. $ SU(n+1) / SU(n) \cong S^{2n+1}$. 
	\item $ V_{n,k}( \cc) \cong U(n) / U(n-k)$.
	\item $ G_{n,k}( \cc) \cong = U(n) / U(k) \times U(n-k)$.
	\item If $ f: M \to N$ a smooth map, s.t.\ 
		\begin{enumerate}[label=(\roman*)]
			\item $ f$ is surjective
			\item  $ f$ is a submersion
			\item  $ f$ is proper  \emph{i.e.} preimage of compact set is compact.
		\end{enumerate}
		Then $ f^{-1}(p)$ where $ p$ is any point is a fiber bundle. This is Ehresmann's Theorem.
	\item vector bundles are fiber bundles with fiber $ \rr^{k}$ or $ \cc^{k}$ with extra structure on the fibers. This includes the tangent bundles, cotangent bundles, normal bundles.
	\item covering space is a bundle with discrete fiber.
\end{enumerate}
\end{eg}

\begin{defn}
Given a fiber bundle $ E \xrightarrow{ p} B$ and a map $ f: A \to B$, the \allbold{pull-back} of $ E$ to A is
 \begin{align*}
	f^* (E) = \{(a,e) \in A \times E: f(a) = p(e)\} . 
\end{align*}
\end{defn}
\begin{align*}
		p: f^* E \to A: (a,e) \mapsto a.
\end{align*}

Exercise:
\begin{enumerate}[label=(\arabic*)]
	\item Show $ f^* E \to A$ is a fiber bundle with the same fiber as $ E \to B$.
	\item If $ A$ is a subset of  $ B$ and  $ f: A \to B$ is inclusion, then show $ f^* (E) \cong E|_A$ \emph{i.e.} $ E|_A = p^{-1}(A)$.
	\item $ \widetilde{ f} : f^* E \to E, (a,e)\mapsto e$ is a bundle map so the diagram commutes.
	\item If $ f: A \to B$ is constant and fiber of $ E$ is  $ F$, then  $ f^* E \cong A \times F$.
	\item If $ E = B \times F$ then $ f^* E \cong A \times F$. 
\end{enumerate}
\begin{defn}
If $ E \xrightarrow{ p} B $ are $ E' \xrightarrow{ p'} B $ are bundles, we say they are \allbold{bundle isomorphic} if there exists a homeomorphism $ h: E \to E'$ s.t.\  the diagram commutes. We denote $ E \cong E'$.
\end{defn}

\begin{thm}
If $ f_i: A \to B$, $ i=0,1$ are homotopic and  $ A$ is locally compact and normal (\emph{e.g.} a CW complex), then  $ f_0^* E \cong f_1^* E$.
\end{thm}
\begin{proof}
Let $ f_i: A \to B$ and homotopy $H: A \times I \to B$. Diagrams. Theorem 2 says there exists a homotopy $ \widetilde{ H}:$

$ H^* E = \{(x,t,e) \in A \times I \times E: H(x,t) = p(e)\} $. Define 
 \begin{align*}
	\overline{H} ((x,e),t) = (x,t, \widetilde{ H}(x,e,t)).
\end{align*}
% https://q.uiver.app/?q=WzAsNyxbMCwwLCJmXzBeKkUgXFx0aW1lcyBJIl0sWzAsMSwiQSBcXHRpbWVzIEkiXSxbMSwwLCJFIl0sWzEsMSwiQiJdLFsyLDAsImZfMF4qRSBcXHRpbWVzIEkgIl0sWzIsMSwiQSBcXHRpbWVzIEkiXSxbMywwLCJIXipFIl0sWzAsMSwiXFxwaV8xIFxcdGltZXMgXFx0ZXh0e2lkfV9JIiwyXSxbMCwyLCJcXHRpbGRle0h9Il0sWzEsMywiSCJdLFsyLDMsInAiLDJdLFs0LDUsIlxccGlfMSBcXHRpbWVzIFxcdGV4dHtpZH1fSSIsMl0sWzQsNiwiXFxvdmVybGluZXtIfSJdLFs2LDUsIlxccGlfMSIsMV1d
\[\begin{tikzcd}
	{f_0^*E \times I} & E & {f_0^*E \times I } & {H^*E} \\
	{A \times I} & B & {A \times I}
	\arrow["{\pi_1 \times \text{id}_I}"', from=1-1, to=2-1]
	\arrow["{\tilde{H}}", from=1-1, to=1-2]
	\arrow["H", from=2-1, to=2-2]
	\arrow["p"', from=1-2, to=2-2]
	\arrow["{\pi_1 \times \text{id}_I}"', from=1-3, to=2-3]
	\arrow["{\overline{H}}", from=1-3, to=1-4]
	\arrow["{\pi_1}"{description}, from=1-4, to=2-3]
\end{tikzcd}\]
Exercise: $ \overline{H}$ is a bundle isomorphism.


Restricting $ \overline{H}$ to $ f_0^* E \times \{1\}$ yields a bundle isomorphism. Notice $ \overline{H}(f_0^* E \times \{1\} ) = \{(x,1,e) \in A \times I \times E: H(x,1) = f_1(x) = p(e)\} =f_1^* E$. Hence $ f_0^* E \cong f_1^* E$.
\end{proof}

\begin{thm}[covering homotopy property]
Let $ p: E \to B$ and $ q: Z \to Y$ be fiber bundles with the same fiber. Suppose $ B $ is locally compact and normal. Given  $ \widetilde{ h}_0: E \to Z, h_0: B \to Y$ s.t.\  the following diagram commutes,
% https://q.uiver.app/?q=WzAsNCxbMCwwLCJFIl0sWzAsMSwiQiJdLFsxLDAsIloiXSxbMSwxLCJZIl0sWzAsMiwiXFx0aWxkZXtofV8wIl0sWzIsMywicSJdLFswLDEsInAiLDJdLFsxLDMsImhfMCJdXQ==
\[\begin{tikzcd}
	E & Z \\
	B & Y
	\arrow["{\tilde{h}_0}", from=1-1, to=1-2]
	\arrow["q", from=1-2, to=2-2]
	\arrow["p"', from=1-1, to=2-1]
	\arrow["{h_0}", from=2-1, to=2-2]
\end{tikzcd}\]
and a homotopy $ H: B \times I \to Y$ of $ h_0$, then there exists a homotopy $ \widetilde{ H}: E \times I \to Z$ of bundle maps covering $ H$.
\end{thm}
\begin{proof}
We assume $ B$ is compact (locally compact case is an exercise). Idea: break  $ Z$ into pieces where bundle is trivial  $ U \times F$. Here the theorem is clear. Then we put the homotopies together.

Let $ \{V_{ \beta}\} $ be a cover of $ Y$ by locally trivializing charts so we have an isomorphism
 \begin{align*}
	q^{-1}(V_{ \beta}) \xrightarrow{ \phi _{ \beta}}  V_{ \beta} \times F
\end{align*}
$ \{H ^{-1}(V _{ \beta}\}) $ is an open cover of $ B \times I$ since $ B \times I$ is compact, we have a finite subcover $ \{U _{ \alpha} \times I_j\} $ covering $ B \times I$ s.t.\ $ H(U_{ \alpha} \times I_j) \subseteq V_{ \beta}$ for some $ \beta$. Note: $ H^* Z$ is trivial over $ U_{ \alpha} \times I_j$ since $ Z$ is trivial over  $ V_{ \beta}$. We can take the $ I_j$ to be segements. We will inductively lift $ H$ to  $ \widetilde{ H}: E \times [0,t_{k}] \to Z$. For each $ x \in B$ there exists neighborhoods  $ W,W'$  s.t.\ $ x \in W \subseteq \overline{W} \subseteq W'$ and $ \overline{W}' \subseteq U_i$ for some $ i$ by normal. There are finite number of $ \{W_i,W_i'\}_{i=1}^{s} $ s.t.\ $ \{W_i\} $ cover $ B$. By Urysohn's lemma, there exist maps  $ u_i: B \to [t_k,t_{k+1}]$ s.t.\ $ u_i(\overline{W}_i) = t_{k+1}$ and $ u_i(B - W_i') = t_k$. Set $ \tau_0(x) = t_k \ \forall \ x$ and $ \tau_i(x) = \max \{u_1(x),\ldots,u_i(x)\} $. So $ t_k = \tau_0(x) \leq \tau_1(x) \leq \ldots\leq \tau_s(x) = t_{k+1}$. Set $ B_i = \{(x,t) \in B \times I: t_k \leq t \leq \tau_i(x)\} $. Let $ E_i$ be the part of $ E \times I$ above $ B_i$. So $ E_0= E \times \{t_k\} \subseteq E_1 \subseteq \ldots \subseteq E_s = E \times [t_k, t_{k+1}]$. Assume we have $ \widetilde{ H}$ defined on $ E \times [0,t_k]$ so it is defined on $ E \times \{t_k\} = E_0$. We inductively extend $ \widetilde{ H}$ over $ E_i$. Note if $ (x,t) \in B_i - B_{i-1}$, then $ \tau_{i-1} < t \leq \tau_i(x)$. So $ u_i(x) > \tau_{i-1}(x)$. Thus $ x(,t) \in W_i' \times \{t_k,t_{k+1}\} $. By definition $ W_i' \times [t_k,t_{k+1}] \subseteq U_{ \alpha} \times I_j$. So $ H(B_i-B_{i-1}) \subseteq V_ \beta$ for some $ \beta$ and $ q^{-1}(V_ \beta) \xrightarrow{ \phi_{ \beta}} V_{ \beta} \times F$. Let $ \rho_{ \beta}: q^{-1}(V_{ \beta}) \to F$ be $ \phi_{ \beta}$ composed with projection. For $ (e,t) \in E_i - E_{i-1}$, let $ p(e) = x \in B$. Set $ \widetilde{ H}(e,t) = \phi_{ \beta}^{-1}(H(x,t), \rho_{ \beta}(\widetilde{ H}(e, \tau_{i-1}(x))))$. Exercise: show this extends $ \widetilde{ H}$ over $ E_i$.
\end{proof}

\begin{coro}
If $ X$ is contractible and locally compact and normal, then any fiber bundle over  $ X$ is trivial,  \emph{i.e.} $ E \cong X \times F$.
\end{coro}
\begin{proof}
X contractible means the identity map $ f_0$ is homotopy to the constant map $ f_1$. Therefore, $ f_0^* E \cong E \cong f_1^* E \cong X \times F $.
\end{proof}


\end{document}
