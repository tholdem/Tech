\documentclass[12pt]{article}
\newcommand{\alert}[1]{{\bf \color{red} [Alert:] #1}}
\newcommand{\todo}[1]{{\bf \color{orange} [TODO:] #1}}
\newcommand{\real}[1][]{\mathbb{R}^{#1}}
\newcommand{\myeqn}[1]{(\ref{#1})}
\newcommand{\myex}[1]{Example \ref{#1}}
\newcommand{\defeq}{\stackrel{\mathrm{def}}{=}}
\newcommand{\parder}[2]{\frac{\partial #1}{\partial #2}}
\newcommand{\Lie}[3][]{\mathsf{L}_{#3}^{#1} #2}
\newcommand{\LieA}[1]{\mathsf{Lie}(#1)}
\newcommand{\lieder}[2]{\mathcal{L}_{#2} #1}
\renewcommand{\t}{^{\mbox{\tiny\sf T}}}
\newcommand{\trans}{^{\mbox{\tiny\sf T}}}
\newcommand{\markup}[1]{\{\textbf{#1}\}}
\newcommand{\msub}[1]{_\mathrm{#1}}
\newcommand{\msup}[1]{^\mathrm{#1}}
\newcommand{\inv}[1]{#1^{-1}}
\newcommand{\pinv}[1]{{#1}^{+}}
\newcommand{\myfracA}[2]{\displaystyle{\frac{#1}{#2}}}
\newcommand{\myfracB}[2]{{#1}/{#2}}
\newcommand{\mydiffA}[1]{\dot{#1}}
\newcommand{\mydiffB}[2]{\myfracA{\mathrm{d}{#1}}{\mathrm{d}{#2}}}
\newcommand{\ball}[2]{\mathcal{B}_{#1}\left(#2\right)}
\newcommand{\acos}[1]{\cos^{-1}\left(#1\right)}
\newcommand{\asin}[1]{\sin^{-1}\left(#1\right)}
\newcommand{\mani}{\mathcal{M}}
\newcommand{\tang}[2]{\mathsf{T}_{#1} #2}
\newcommand{\LieB}[2]{[ #1, #2 ]}
\newcommand{\LieBad}[3][]{\mathsf{ad}_{#2}^{#1} #3}
\newcommand{\ReachVT}{\mathcal{R}^V_T}
\newcommand{\ReachVt}{\mathcal{R}^V_t}
\newcommand{\ReachVTe}{\mathcal{R}^V_{\le T}}
\newcommand{\ReachT}{\mathcal{R}_T}
\newcommand{\Reacht}{\mathcal{R}_t}
\newcommand{\ReachTe}{\mathcal{R}_{\le T}}
\newcommand{\accLA}[1]{\mathsf{Lie}(#1)}
\newcommand{\accD}{\Delta_{\mathcal{F}}}
\newcommand{\accSA}{\mathsf{Lie}(\mathcal{G},f)}
\newcommand{\accDS}{\Delta_{\mathcal{G}}}
\newcommand{\eval}[3]{\mathsf{Ev}^{#2}_{#1}\left( #3 \right)}
\newcommand{\stlc}{\textsc{stlc}}
\newcommand{\clf}{\textsc{clf}}
\newcommand{\jqlf}{\textsc{jqlf}}
\newcommand{\dlas}{\textsc{dlas}}
\newcommand{\Ad}[2]{\mathsf{Ad}_{#1} #2}
\newcommand{\xe}{\ensuremath{x_e}}
\newcommand{\lebg}[1]{\mathcal{L}_{#1}}
\newcommand{\lebgx}[1]{\mathcal{L}_{#1 \mathrm{e}}}
\newcommand{\dom}{D}
\newcommand{\domT}{[t_0,\infty) \times D}
\newcommand{\rarrow}{\rightarrow}
\renewcommand{\d}{\mathrm{d}}
\renewcommand{\Re}{\mathbb{R}}
\newcommand{\C}{\mathrm{C}}

\newcommand{\QED}{{\unskip\nobreak\hfil\penalty50\hskip2em\vadjust{}
		\nobreak\hfil$\Box$\parfillskip=0pt\finalhyphendemerits=0\par}\vspace{0.1cm}}
\newcommand{\eoEx}{{\unskip\nobreak\hfil\penalty50\hskip0em\vadjust{}
		\nobreak\hfil$\Large\Diamond$\parfillskip=0pt\finalhyphendemerits=0\par}\vspace{0.1cm}}

\newcommand{\sgn}{\ensuremath{\operatorname{sgn}}}
\newcommand{\sat}{\ensuremath{\operatorname{sat}}}

\newcommand{\half}{\frac{1}{2}}
\newcommand{\shalf}{\mbox{$\frac{1}{2}$}}
\newcommand{\marcom}[1]{\marginpar{\footnotesize #1}}
\newcommand{\der}{\mathrm{D}}
\newcommand{\e}{\mathrm{e}}
\newcommand{\dt}{\mathrm{d}t}

\newcommand{\cA}{\ensuremath{\mathcal{A}}}
\newcommand{\cB}{\ensuremath{\mathcal{B}}}
\newcommand{\cG}{\ensuremath{\mathcal{G}}}
\newcommand{\cK}{\ensuremath{\mathcal{K}}}
\newcommand{\cW}{\ensuremath{\mathcal{W}}}
\newcommand{\cZ}{\ensuremath{\mathcal{Z}}}
\newcommand{\cS}{\ensuremath{\mathcal{S}}}
\newcommand{\cD}{\ensuremath{\mathcal{D}}}
\newcommand{\cP}{\ensuremath{\mathcal{P}}}
\newcommand{\cV}{\ensuremath{\mathcal{V}}}
\newcommand{\cL}{\ensuremath{\mathcal{L}}}
\newcommand{\cN}{\ensuremath{\mathcal{N}}}
\newcommand{\cI}{\ensuremath{\mathcal{I}}}
\newcommand{\cR}{\ensuremath{\mathcal{R}}}
\newcommand{\cM}{\ensuremath{\mathcal{M}}}
\newcommand{\cC}{\ensuremath{\mathcal{C}}}
\newcommand{\cF}{\ensuremath{\mathcal{F}}}
\newcommand{\cH}{\ensuremath{\mathcal{H}}}
\newcommand{\cO}{\ensuremath{\mathcal{O}}}
\newcommand{\cX}{\ensuremath{\mathcal{X}}}
\newcommand{\cY}{\ensuremath{\mathcal{Y}}}
\newcommand{\Ci}{\ensuremath{\mathcal{C}^\infty}}
\newcommand{\ISS}{\textsc{iss}}
\newcommand{\LISS}{\textsc{liss}}
\newcommand{\GAS}{\textsc{gas}}
\newcommand{\GS}{\textsc{gs}}
\newcommand{\LES}{\textsc{les}}
\newcommand{\GUAS}{\textsc{guas}}
\newcommand{\BIBO}{\textsc{bibo}}
\newcommand{\spec}{\ensuremath{\operatorname{spec}}}
\newcommand{\spn}{\ensuremath{\operatorname{span}}}
\renewcommand{\i}{\mathrm{i\,}}

\renewcommand{\implies}{\Rightarrow}

\renewcommand{\theenumi}{$\roman{enumi})$}
\renewcommand{\labelenumi}{\theenumi}

\font\ptmten=zptmcmrm scaled 1200
\newcommand{\w}{\mbox{{\ptmten w}}}
\newcommand{\z}{\mbox{{\ptmten z}}}
\renewcommand{\Re}{\mathbb{R}}

\newcommand{\cl}{\operatorname{cl}}
\newcommand{\intr}{\operatorname{int}}
\newcommand{\rank}{\operatorname{rank}}
\newcommand{\co}{\operatorname{co}}
\newcommand{\aff}{\operatorname{aff}}

\theoremstyle{plain}
\newtheorem{theorem}{Theorem}[chapter]
\newtheorem{claim}[theorem]{Claim}
\newtheorem{corollary}[theorem]{Corollary}
\newtheorem{prop}[theorem]{Proposition}
\newtheorem{fact}[theorem]{Fact}
\newtheorem{lemma}[theorem]{Lemma}

\newtheorem{remark}{Remark}[chapter]

\theoremstyle{definition}
\newtheorem{assume}[theorem]{Assumption}
\newtheorem{defn}[theorem]{Definition}
\newtheorem{problem}[theorem]{Problem}
\newtheorem{exercise}{Exercise}
\newtheorem{example}[theorem]{Example}


\begin{document}
\centerline {\textsf{\textbf{\LARGE{Homework 10}}}}
\centerline {Jaden Wang}
\vspace{.15in}

\begin{problem}[2.5.1]
Since $ F(x)\neq z \ \forall \ x \in D$, we see that $ u:X \to S^{n-1}$ can be extended to all of $ D$. Hence by the boundary theorem,  $ W_2(f,z) = \deg_2(u) = 0$.
\end{problem}

\begin{problem}[2.5.2]
Since $ y_i \in \inte B_i$, $ u_i: \partial B_i \to S^{n-1}, x\mapsto \frac{f_i(x)-z}{\norm{ f_i(x)-z}}$ is well-defined. And $ W_2(f_i,z) = \deg_2(u_i)$. Now consider the compact manifold (removing disjoint open balls remains closed and hence compact as a subspace) with boundary $ \widetilde{ D} := D - \bigcup_{ i= 1}^{\ell} \inte B_i$. Let $ \widetilde{ f}: \partial \widetilde{ D} = X \sqcup \bigsqcup_{ i=1}^{\ell} \partial B_i \to \rr^{n}$ be defined by $ f$ and  $ f_i$. Then define $ \widetilde{ u}$ as usual using $ \widetilde{ f}$ and let $ \widetilde{ F} = F|_{\widetilde{ D}}$. Since $ \widetilde{ F}$ clearly doesn't hit $ z$, we have
\begin{align*}
	W_2(\widetilde{ f},z) = \deg_2(\widetilde{ u}) = I_2(\widetilde{ u},z)=0
\end{align*}
Recall that $  I_2(\widetilde{ u},z)$ is the number mod 2 of points in $ \widetilde{ u}^{-1} (z)$. Since  $ \partial \widetilde{ D}$ is a disjoint unions of boundaries, we can compute the intersection number by summing the intersection numbers on each disjoint boundary (mod 2), as 
\begin{align*}
 \widetilde{ u}^{-1}(z) = (\widetilde{ u}|_{X})^{-1}(z) \sqcup \bigsqcup_{ i=1}^{n} (\widetilde{ u}|_{\partial B_i})^{-1}(z) = u^{-1}(z) \sqcup \bigsqcup_{ i=1}^{\ell} u_i^{-1}(z)
\end{align*}
Thus,
\begin{align*}
	I_2(u,z) + \sum_{ i= 1}^{ \ell} I_2(u_i,z) &= 0 \bmod 2\\ 
	I_2(u,z)&=  - \sum_{ i= 1}^{ \ell} I_2(u_i,z)  \bmod 2\\
	I_2(u,z) &= \sum_{ i= 1}^{ \ell}I_2(u_i,z)  \bmod 2\\
	W_2(f,z) &= \sum_{ i= 1}^{ \ell}W_2(f_i,z)  \bmod 2
\end{align*}
\end{problem}

\begin{problem}[2.5.12]
By 8 we have a relation $ W_2(X,z_0) = W_2(X,z_1) + \ell \bmod 2$ where $ \ell$ is the intersection number of the ray from $ z_0$ passing through $ z_1$. By 9 we know that the winding number mod 2 of $ X$ around  $ z$ completely determines the component that  $ z$ is in. By 10 we know that the outside component has  $ W_2(X,z) = 0$. This implies that the points inside $ X$ has  $ W_2(X,z) = 1$. For any $ z$ and a ray from $ z$ transversal to $ X$, let $ z'$ be a point on the ray super far away that it is certainly outside $ X$ and there is no intersection with $ X$ from  $ z'$ to infinity. This allows $ \ell$ to be exactly the number of intersections of the ray with $ X$.

$ (\implies):$ If $ z$ is inside  $ X$, then . We see that $ \ell = W_2(X,z) - W_2(X,z') = 1-0 = 1 \bmod 2$. So $ \ell$ is odd.

$ (\impliedby):$ If $ \ell$ is odd, then $ W_2(X,z) = W_2(X,z') + \ell \bmod 2 = 0+ \ell \bmod 2 = 1$. So $ z$ is inside.
\end{problem}

\begin{problem}[2.6.1]
Recall that  $ u:S^{k} \to S^{k}, \frac{f(x)}{ \norm{ f(x)} }$ is well-defined iff $ f:S^{k} \to \rr^{k+1}$ doesn't hit the origin. Moreover, $ f(-x) = -f(x)$ iff $ u(-x) = -u(x)$ \emph{i.e.} $ u$ carries antipodal points to antipodal points. Then $ W_2(f,0) = \deg_2(u) =1$. So they are equivalent statements.
\end{problem}
\begin{problem}[2.6.2]
Let $ f:S^{1} \to S^{1}$ satisfy $ f(-x) = -f(x)$. So
\begin{align*}
	(\cos g(t+ \pi), \sin g(t+\pi)) &=  f(\cos (t+ \pi), \sin(t+ \pi))\\
					&= f(-\cos t, -\sin t)\\ 
					&= -f(\cos t, \sin t)\\ 
					&= (-\cos g(t), - \sin g(t))
\end{align*}
Hence we have $ \cos g(t+\pi) = -\cos g(t) = \cos (g(t)+q\pi)$ where $ q$ is odd. Therefore,  $ g(t+\pi) = g(t) + q\pi$. The sin part yields the same relation. We see that  $ g(t+2\pi) = g(t+\pi + \pi) = g(t+\pi) + q\pi = g(t)+ 2q \pi$ so this is exactly the $ q$ described by Exercise 2.4.8. It follows that $ \deg_2(f) = q \bmod 2 =1$ since $ q$ is odd.
\end{problem}

\begin{problem}[2.6.3]
Since $ p_1,\ldots,p_n$ are homogeneous polynomials of odd degree, their polynomial functions $ f_1,\ldots,f_n: \rr^{n+1} \to \rr$ are odd functions, \emph{i.e.} $ f_i(-x) = -f_i(x)$. Define $ g_i:= f_i|_{S^{n}}$. Then by corollary of Boruk-Ulam,  $ g_i$'s must possess a common root, say $ v$, in $ S^{n}$. But if $ v$ is a root of $ g_i$, it is a root of $ f_i$, and any scalar multiple $ tv$ of $ v$ is also a root of $ f_i$ due to homogeneous degrees. Hence they share a line of roots $ tv$.
\end{problem}
\end{document}
