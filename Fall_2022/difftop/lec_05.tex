\documentclass[12pt,class=article,crop=false]{standalone} 
\newcommand{\alert}[1]{{\bf \color{red} [Alert:] #1}}
\newcommand{\todo}[1]{{\bf \color{orange} [TODO:] #1}}
\newcommand{\real}[1][]{\mathbb{R}^{#1}}
\newcommand{\myeqn}[1]{(\ref{#1})}
\newcommand{\myex}[1]{Example \ref{#1}}
\newcommand{\defeq}{\stackrel{\mathrm{def}}{=}}
\newcommand{\parder}[2]{\frac{\partial #1}{\partial #2}}
\newcommand{\Lie}[3][]{\mathsf{L}_{#3}^{#1} #2}
\newcommand{\LieA}[1]{\mathsf{Lie}(#1)}
\newcommand{\lieder}[2]{\mathcal{L}_{#2} #1}
\renewcommand{\t}{^{\mbox{\tiny\sf T}}}
\newcommand{\trans}{^{\mbox{\tiny\sf T}}}
\newcommand{\markup}[1]{\{\textbf{#1}\}}
\newcommand{\msub}[1]{_\mathrm{#1}}
\newcommand{\msup}[1]{^\mathrm{#1}}
\newcommand{\inv}[1]{#1^{-1}}
\newcommand{\pinv}[1]{{#1}^{+}}
\newcommand{\myfracA}[2]{\displaystyle{\frac{#1}{#2}}}
\newcommand{\myfracB}[2]{{#1}/{#2}}
\newcommand{\mydiffA}[1]{\dot{#1}}
\newcommand{\mydiffB}[2]{\myfracA{\mathrm{d}{#1}}{\mathrm{d}{#2}}}
\newcommand{\ball}[2]{\mathcal{B}_{#1}\left(#2\right)}
\newcommand{\acos}[1]{\cos^{-1}\left(#1\right)}
\newcommand{\asin}[1]{\sin^{-1}\left(#1\right)}
\newcommand{\mani}{\mathcal{M}}
\newcommand{\tang}[2]{\mathsf{T}_{#1} #2}
\newcommand{\LieB}[2]{[ #1, #2 ]}
\newcommand{\LieBad}[3][]{\mathsf{ad}_{#2}^{#1} #3}
\newcommand{\ReachVT}{\mathcal{R}^V_T}
\newcommand{\ReachVt}{\mathcal{R}^V_t}
\newcommand{\ReachVTe}{\mathcal{R}^V_{\le T}}
\newcommand{\ReachT}{\mathcal{R}_T}
\newcommand{\Reacht}{\mathcal{R}_t}
\newcommand{\ReachTe}{\mathcal{R}_{\le T}}
\newcommand{\accLA}[1]{\mathsf{Lie}(#1)}
\newcommand{\accD}{\Delta_{\mathcal{F}}}
\newcommand{\accSA}{\mathsf{Lie}(\mathcal{G},f)}
\newcommand{\accDS}{\Delta_{\mathcal{G}}}
\newcommand{\eval}[3]{\mathsf{Ev}^{#2}_{#1}\left( #3 \right)}
\newcommand{\stlc}{\textsc{stlc}}
\newcommand{\clf}{\textsc{clf}}
\newcommand{\jqlf}{\textsc{jqlf}}
\newcommand{\dlas}{\textsc{dlas}}
\newcommand{\Ad}[2]{\mathsf{Ad}_{#1} #2}
\newcommand{\xe}{\ensuremath{x_e}}
\newcommand{\lebg}[1]{\mathcal{L}_{#1}}
\newcommand{\lebgx}[1]{\mathcal{L}_{#1 \mathrm{e}}}
\newcommand{\dom}{D}
\newcommand{\domT}{[t_0,\infty) \times D}
\newcommand{\rarrow}{\rightarrow}
\renewcommand{\d}{\mathrm{d}}
\renewcommand{\Re}{\mathbb{R}}
\newcommand{\C}{\mathrm{C}}

\newcommand{\QED}{{\unskip\nobreak\hfil\penalty50\hskip2em\vadjust{}
		\nobreak\hfil$\Box$\parfillskip=0pt\finalhyphendemerits=0\par}\vspace{0.1cm}}
\newcommand{\eoEx}{{\unskip\nobreak\hfil\penalty50\hskip0em\vadjust{}
		\nobreak\hfil$\Large\Diamond$\parfillskip=0pt\finalhyphendemerits=0\par}\vspace{0.1cm}}

\newcommand{\sgn}{\ensuremath{\operatorname{sgn}}}
\newcommand{\sat}{\ensuremath{\operatorname{sat}}}

\newcommand{\half}{\frac{1}{2}}
\newcommand{\shalf}{\mbox{$\frac{1}{2}$}}
\newcommand{\marcom}[1]{\marginpar{\footnotesize #1}}
\newcommand{\der}{\mathrm{D}}
\newcommand{\e}{\mathrm{e}}
\newcommand{\dt}{\mathrm{d}t}

\newcommand{\cA}{\ensuremath{\mathcal{A}}}
\newcommand{\cB}{\ensuremath{\mathcal{B}}}
\newcommand{\cG}{\ensuremath{\mathcal{G}}}
\newcommand{\cK}{\ensuremath{\mathcal{K}}}
\newcommand{\cW}{\ensuremath{\mathcal{W}}}
\newcommand{\cZ}{\ensuremath{\mathcal{Z}}}
\newcommand{\cS}{\ensuremath{\mathcal{S}}}
\newcommand{\cD}{\ensuremath{\mathcal{D}}}
\newcommand{\cP}{\ensuremath{\mathcal{P}}}
\newcommand{\cV}{\ensuremath{\mathcal{V}}}
\newcommand{\cL}{\ensuremath{\mathcal{L}}}
\newcommand{\cN}{\ensuremath{\mathcal{N}}}
\newcommand{\cI}{\ensuremath{\mathcal{I}}}
\newcommand{\cR}{\ensuremath{\mathcal{R}}}
\newcommand{\cM}{\ensuremath{\mathcal{M}}}
\newcommand{\cC}{\ensuremath{\mathcal{C}}}
\newcommand{\cF}{\ensuremath{\mathcal{F}}}
\newcommand{\cH}{\ensuremath{\mathcal{H}}}
\newcommand{\cO}{\ensuremath{\mathcal{O}}}
\newcommand{\cX}{\ensuremath{\mathcal{X}}}
\newcommand{\cY}{\ensuremath{\mathcal{Y}}}
\newcommand{\Ci}{\ensuremath{\mathcal{C}^\infty}}
\newcommand{\ISS}{\textsc{iss}}
\newcommand{\LISS}{\textsc{liss}}
\newcommand{\GAS}{\textsc{gas}}
\newcommand{\GS}{\textsc{gs}}
\newcommand{\LES}{\textsc{les}}
\newcommand{\GUAS}{\textsc{guas}}
\newcommand{\BIBO}{\textsc{bibo}}
\newcommand{\spec}{\ensuremath{\operatorname{spec}}}
\newcommand{\spn}{\ensuremath{\operatorname{span}}}
\renewcommand{\i}{\mathrm{i\,}}

\renewcommand{\implies}{\Rightarrow}

\renewcommand{\theenumi}{$\roman{enumi})$}
\renewcommand{\labelenumi}{\theenumi}

\font\ptmten=zptmcmrm scaled 1200
\newcommand{\w}{\mbox{{\ptmten w}}}
\newcommand{\z}{\mbox{{\ptmten z}}}
\renewcommand{\Re}{\mathbb{R}}

\newcommand{\cl}{\operatorname{cl}}
\newcommand{\intr}{\operatorname{int}}
\newcommand{\rank}{\operatorname{rank}}
\newcommand{\co}{\operatorname{co}}
\newcommand{\aff}{\operatorname{aff}}

\theoremstyle{plain}
\newtheorem{theorem}{Theorem}[chapter]
\newtheorem{claim}[theorem]{Claim}
\newtheorem{corollary}[theorem]{Corollary}
\newtheorem{prop}[theorem]{Proposition}
\newtheorem{fact}[theorem]{Fact}
\newtheorem{lemma}[theorem]{Lemma}

\newtheorem{remark}{Remark}[chapter]

\theoremstyle{definition}
\newtheorem{assume}[theorem]{Assumption}
\newtheorem{defn}[theorem]{Definition}
\newtheorem{problem}[theorem]{Problem}
\newtheorem{exercise}{Exercise}
\newtheorem{example}[theorem]{Example}


\begin{document}
\section{Brouwer's Fixed Point Theorem}
\begin{thm}
Let $ f: B^{n} \to B^{n}$ be a continuous map, then there exists a $ p \in B^{n}$ s.t.\ $ f(p) = p$.
\end{thm}

\begin{defn}
A \allbold{manifold with boundary} $ M$ models after halfspaces $ H^{n}: \{(x_1,\ldots,x_n) \in \rr^{n}| x_n \geq 0\} $. 
\end{defn}

Exercise: If $ \dim(M) =n$, then  $ \partial M$ is an $ (n-1)$-manifold (therefore $ \partial ( \partial M) = \O$.

The tangent space are one-sided derivatives of half-curves and has the same dimension as $ M$.

Every  $ n$-manifold with  $ \partial $ is diffeomorphic to a subset of a $ n$-manifold without  $ \partial $. (Doubling of a manifold by identifying boundaries of two manifolds). 

Exercise: $ f : M^{m} \to \rr$ and $ 0$ is a regular value of  $ f$, then  $ f^{-1}([0,\infty))$ is an $ m$-manifold with boundary and  $ \partial (f^{-1}[0,\infty)) = f^{-1}(0)$.

\begin{thm}
$ f: M^{m} \to N^{n}$, $ M$ is a manifold with  $ \partial $, $ q \in N$ is a regular value of both $ f$ and  $ f|_{ \partial M}$ and $ f^{-1}(q)\neq \O$, then $ f^{-1}(q)$ is $ (m-n)$-manifold with  $ \partial $, and
\begin{align*}
	\partial (f^{-1}(q)) = f^{-1}(q) \cap \partial M
\end{align*}
\end{thm}

Exercise 9: $ T_p H^{m} \cong T_p \rr^{m}$ since we can extend any half-curve to a curve.

Exercise 13: take $ v \in T_p f^{-1}(q)$, $ v = \alpha'(0), \alpha:(- \epsilon, \epsilon) \to f^{-1}(q), \alpha(0)=p$. Since $ f \circ \alpha(t) = q$ so $ df_p(v) = (f' \circ \alpha)'(0)=0$.
\begin{proof}
	WLOG $ M = H^{m}$. Take $ p f^{-1}(q) \cap \partial M$. Take a neighborhood $ V$ of  $ p$, by smoothness of  $ f$ we can extend it to  $ \widetilde{ f}: V \to N$ where $ \widetilde{ f} = f$ on $ H^{m} \cap V$. By $ d\widetilde{ f}_p = df_p$, we know that $ p$ is a regular point of  $ \widetilde{ f}$. This implies that $ q$ is a regular value of  $ \widetilde{ f}$ as we can make $ V$ small so $ \{p\} = \widetilde{ f}^{-1}(q)$. So $ \widetilde{ f}^{-1}(q)$ is a manifold in $ V$. Define  $ g:\widetilde{ f}^{-1}(q) \to \rr, (x_1,\ldots,x_n)\mapsto x_n$. Then $ g(p)=0$ and  $ V \cap f^{-1}(q) = H \cap \widetilde{ f}^{-1}(q) = g^{-1}([0,\infty))$. Suppose $ 0$ is not a regular value, then rank is 0 so  $ T_p \widetilde{ f}^{-1}(q) = \ker dg _p \subseteq T_p \partial H$. Since both $ \ker dg_p $ and $ T_p \partial H$ are dimension $ m-1$ we have equality. It follows that $ \ker df_p \supseteq T_p \partial H$, a contradiction since $ p$ is a regular value of  $ f_{\partial M}$.
\end{proof}
\begin{thm}[Sard's]
Let $ f: M \to N$ be a smooth map, almost every  (except for a set of measure zero)$ g \in N$ is a regular value of $ f$.
\end{thm}

\begin{proof}[Proof of Brouwer]
First we may assume that $ f$ is smooth by approximation theorem. Suppose to the contrary that  $ f: B^{n} \to B^{n}$ has no fixed point. Then there exists a smooth retraction $ r: B^{n} \to \partial B^{n} = S^{n-1}$ by a ray at $ f(p)$ through $ p$. Notice that $ r(p)=p \ \forall \ p \in \partial B^{n}$. By Sard's Theorem, there exists a $ q \in S^{n-1}$ which is a regular value of $ r$ and $ r^{-1}(q) \neq \O$. By the regular value theorem, $ r^{-1}(q)$ is a 1-dim manifold with $ \partial $. Recall $ \partial (r^{-1}(q)) = r^{-1}(q) \cap S^{n-1}$. Since $ q \in r^{-1}(q)$. So $ r^{-1}(q)$ must be an interval with distinct endpoints on the boundary. But this says that $r(q')=q $ yet  $ r(q')=q'$, a contradiction.
\end{proof}

\end{document}
