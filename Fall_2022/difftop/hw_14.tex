\documentclass[12pt,class=article,crop=false]{standalone} 
\newcommand{\alert}[1]{{\bf \color{red} [Alert:] #1}}
\newcommand{\todo}[1]{{\bf \color{orange} [TODO:] #1}}
\newcommand{\real}[1][]{\mathbb{R}^{#1}}
\newcommand{\myeqn}[1]{(\ref{#1})}
\newcommand{\myex}[1]{Example \ref{#1}}
\newcommand{\defeq}{\stackrel{\mathrm{def}}{=}}
\newcommand{\parder}[2]{\frac{\partial #1}{\partial #2}}
\newcommand{\Lie}[3][]{\mathsf{L}_{#3}^{#1} #2}
\newcommand{\LieA}[1]{\mathsf{Lie}(#1)}
\newcommand{\lieder}[2]{\mathcal{L}_{#2} #1}
\renewcommand{\t}{^{\mbox{\tiny\sf T}}}
\newcommand{\trans}{^{\mbox{\tiny\sf T}}}
\newcommand{\markup}[1]{\{\textbf{#1}\}}
\newcommand{\msub}[1]{_\mathrm{#1}}
\newcommand{\msup}[1]{^\mathrm{#1}}
\newcommand{\inv}[1]{#1^{-1}}
\newcommand{\pinv}[1]{{#1}^{+}}
\newcommand{\myfracA}[2]{\displaystyle{\frac{#1}{#2}}}
\newcommand{\myfracB}[2]{{#1}/{#2}}
\newcommand{\mydiffA}[1]{\dot{#1}}
\newcommand{\mydiffB}[2]{\myfracA{\mathrm{d}{#1}}{\mathrm{d}{#2}}}
\newcommand{\ball}[2]{\mathcal{B}_{#1}\left(#2\right)}
\newcommand{\acos}[1]{\cos^{-1}\left(#1\right)}
\newcommand{\asin}[1]{\sin^{-1}\left(#1\right)}
\newcommand{\mani}{\mathcal{M}}
\newcommand{\tang}[2]{\mathsf{T}_{#1} #2}
\newcommand{\LieB}[2]{[ #1, #2 ]}
\newcommand{\LieBad}[3][]{\mathsf{ad}_{#2}^{#1} #3}
\newcommand{\ReachVT}{\mathcal{R}^V_T}
\newcommand{\ReachVt}{\mathcal{R}^V_t}
\newcommand{\ReachVTe}{\mathcal{R}^V_{\le T}}
\newcommand{\ReachT}{\mathcal{R}_T}
\newcommand{\Reacht}{\mathcal{R}_t}
\newcommand{\ReachTe}{\mathcal{R}_{\le T}}
\newcommand{\accLA}[1]{\mathsf{Lie}(#1)}
\newcommand{\accD}{\Delta_{\mathcal{F}}}
\newcommand{\accSA}{\mathsf{Lie}(\mathcal{G},f)}
\newcommand{\accDS}{\Delta_{\mathcal{G}}}
\newcommand{\eval}[3]{\mathsf{Ev}^{#2}_{#1}\left( #3 \right)}
\newcommand{\stlc}{\textsc{stlc}}
\newcommand{\clf}{\textsc{clf}}
\newcommand{\jqlf}{\textsc{jqlf}}
\newcommand{\dlas}{\textsc{dlas}}
\newcommand{\Ad}[2]{\mathsf{Ad}_{#1} #2}
\newcommand{\xe}{\ensuremath{x_e}}
\newcommand{\lebg}[1]{\mathcal{L}_{#1}}
\newcommand{\lebgx}[1]{\mathcal{L}_{#1 \mathrm{e}}}
\newcommand{\dom}{D}
\newcommand{\domT}{[t_0,\infty) \times D}
\newcommand{\rarrow}{\rightarrow}
\renewcommand{\d}{\mathrm{d}}
\renewcommand{\Re}{\mathbb{R}}
\newcommand{\C}{\mathrm{C}}

\newcommand{\QED}{{\unskip\nobreak\hfil\penalty50\hskip2em\vadjust{}
		\nobreak\hfil$\Box$\parfillskip=0pt\finalhyphendemerits=0\par}\vspace{0.1cm}}
\newcommand{\eoEx}{{\unskip\nobreak\hfil\penalty50\hskip0em\vadjust{}
		\nobreak\hfil$\Large\Diamond$\parfillskip=0pt\finalhyphendemerits=0\par}\vspace{0.1cm}}

\newcommand{\sgn}{\ensuremath{\operatorname{sgn}}}
\newcommand{\sat}{\ensuremath{\operatorname{sat}}}

\newcommand{\half}{\frac{1}{2}}
\newcommand{\shalf}{\mbox{$\frac{1}{2}$}}
\newcommand{\marcom}[1]{\marginpar{\footnotesize #1}}
\newcommand{\der}{\mathrm{D}}
\newcommand{\e}{\mathrm{e}}
\newcommand{\dt}{\mathrm{d}t}

\newcommand{\cA}{\ensuremath{\mathcal{A}}}
\newcommand{\cB}{\ensuremath{\mathcal{B}}}
\newcommand{\cG}{\ensuremath{\mathcal{G}}}
\newcommand{\cK}{\ensuremath{\mathcal{K}}}
\newcommand{\cW}{\ensuremath{\mathcal{W}}}
\newcommand{\cZ}{\ensuremath{\mathcal{Z}}}
\newcommand{\cS}{\ensuremath{\mathcal{S}}}
\newcommand{\cD}{\ensuremath{\mathcal{D}}}
\newcommand{\cP}{\ensuremath{\mathcal{P}}}
\newcommand{\cV}{\ensuremath{\mathcal{V}}}
\newcommand{\cL}{\ensuremath{\mathcal{L}}}
\newcommand{\cN}{\ensuremath{\mathcal{N}}}
\newcommand{\cI}{\ensuremath{\mathcal{I}}}
\newcommand{\cR}{\ensuremath{\mathcal{R}}}
\newcommand{\cM}{\ensuremath{\mathcal{M}}}
\newcommand{\cC}{\ensuremath{\mathcal{C}}}
\newcommand{\cF}{\ensuremath{\mathcal{F}}}
\newcommand{\cH}{\ensuremath{\mathcal{H}}}
\newcommand{\cO}{\ensuremath{\mathcal{O}}}
\newcommand{\cX}{\ensuremath{\mathcal{X}}}
\newcommand{\cY}{\ensuremath{\mathcal{Y}}}
\newcommand{\Ci}{\ensuremath{\mathcal{C}^\infty}}
\newcommand{\ISS}{\textsc{iss}}
\newcommand{\LISS}{\textsc{liss}}
\newcommand{\GAS}{\textsc{gas}}
\newcommand{\GS}{\textsc{gs}}
\newcommand{\LES}{\textsc{les}}
\newcommand{\GUAS}{\textsc{guas}}
\newcommand{\BIBO}{\textsc{bibo}}
\newcommand{\spec}{\ensuremath{\operatorname{spec}}}
\newcommand{\spn}{\ensuremath{\operatorname{span}}}
\renewcommand{\i}{\mathrm{i\,}}

\renewcommand{\implies}{\Rightarrow}

\renewcommand{\theenumi}{$\roman{enumi})$}
\renewcommand{\labelenumi}{\theenumi}

\font\ptmten=zptmcmrm scaled 1200
\newcommand{\w}{\mbox{{\ptmten w}}}
\newcommand{\z}{\mbox{{\ptmten z}}}
\renewcommand{\Re}{\mathbb{R}}

\newcommand{\cl}{\operatorname{cl}}
\newcommand{\intr}{\operatorname{int}}
\newcommand{\rank}{\operatorname{rank}}
\newcommand{\co}{\operatorname{co}}
\newcommand{\aff}{\operatorname{aff}}

\theoremstyle{plain}
\newtheorem{theorem}{Theorem}[chapter]
\newtheorem{claim}[theorem]{Claim}
\newtheorem{corollary}[theorem]{Corollary}
\newtheorem{prop}[theorem]{Proposition}
\newtheorem{fact}[theorem]{Fact}
\newtheorem{lemma}[theorem]{Lemma}

\newtheorem{remark}{Remark}[chapter]

\theoremstyle{definition}
\newtheorem{assume}[theorem]{Assumption}
\newtheorem{defn}[theorem]{Definition}
\newtheorem{problem}[theorem]{Problem}
\newtheorem{exercise}{Exercise}
\newtheorem{example}[theorem]{Example}


\begin{document}
\begin{problem}[4.7.1]
	Since $ \partial [a,b] = b -a$, we see that by Stokes Theorem,
\begin{align*}
	\int_{[a,b]} df &= \int_{ \partial [a,b]} f \\
	&= \int_{b-a} f \\
	&= f(b) - f(a) 
\end{align*}
\end{problem}
\begin{problem}[4.7.2]
First note that
\begin{align*}
	d(fdx+gdy) &= d(fdx) + d(gdy) \\
	&= df \wedge dx +0 + dg \wedge dy+0 \\
	&= \frac{\partial f}{\partial x} dx \wedge dx + \frac{\partial f}{\partial y} dy \wedge dx + \frac{\partial g}{\partial x} dx \wedge dy + \frac{\partial g}{\partial y} dy \wedge dy\\
	&= \left( \frac{\partial g}{\partial x} - \frac{\partial f}{\partial y}  \right) dx \wedge dy 
\end{align*}
Therefore, by Stokes Theorem,
\begin{align*}
	\int_{ \gamma} fdx + gdy &= \int_{W} d(fdx + gdy) \\
	&= \int_W \left( \frac{\partial g}{\partial x} - \frac{\partial f}{\partial y}  \right) dx dy 
\end{align*}
\end{problem}

\begin{problem}[4.7.3]
Let $ \omega$ be the one from Exercise 4.4.14, then
\begin{align*}
	d \omega &= df_1 \wedge dx_2 \wedge dx_3 + 0 + 0 + df_2 \wedge dx_3 \wedge dx_1 + 0 + 0 + df_3 \wedge dx_1 \wedge dx_2 +0+0\\
	&= \frac{\partial f_1}{\partial x_1} dx_1 \wedge dx_2 \wedge dx_3 + \frac{\partial f_2}{\partial x_2} dx_2 \wedge dx_3 \wedge  d x_1 + \frac{\partial f_3}{\partial x_3} dx_3 \wedge dx_1 \wedge dx_2 \\
	&= \left( \frac{\partial f_1}{\partial x_1}+ \frac{\partial f_2}{\partial x_2} + \frac{\partial f_3}{\partial x_3}   \right) dx_1 \wedge dx_2 \wedge dx_3\\
	&= \Div \ve{F} dx_1 \wedge dx_2 \wedge dx_3 
\end{align*}
By Exercise 4.4.14, we immediately have
\begin{align*}
d(\ve{F} \cdot \ve{n}\ dA) = d\omega.
\end{align*}
By Stokes Theorem,
\begin{align*}
	\int_{\partial W } (\ve{F} \cdot \ve{ n}) dA &= \int_W d \omega \\
	&= \int_{W} \Div \ve{ F} dx_1 dx_2 dx_3 
\end{align*}
\end{problem}
\begin{problem}[4.7.4]
Recall that
\begin{align*}
	\curl \ve{ F} = \left( \frac{\partial f_2}{\partial x_3} - \frac{\partial f_3}{\partial x_2}, \frac{\partial f_3}{\partial x_1} -\frac{\partial f_1}{\partial x_3}, \frac{\partial f_1}{\partial x_2}- \frac{\partial f_2}{\partial x_1} \right) =: (g_1,g_2,g_3)
\end{align*}
Again by Exercise 4.4.14,
\begin{align*}
	(\curl \ve{ F} \cdot \ve{ n}) dA = g_1 dx_2 \wedge dx_3+ g_2 dx_3 \wedge dx_1 \wedge g_3 dx_1 \wedge dx_2 =: \omega.
\end{align*}
Moreover, we see that
\begin{align*}
	d(f_1 dx_1 + f_2 dx_2 + f_3 dx_3) &= df_1 \wedge dx_1 + df_2 \wedge dx_2 + df_3 \wedge dx_3 \\
	&= \frac{\partial f_1}{\partial x_2} dx_2 \wedge dx_1 + \frac{\partial f_1}{\partial x_3} dx_3 \wedge dx_1 + \cdots  \\
	&= \omega 
\end{align*}
Therefore by Stokes,
\begin{align*}
	\int_S (\curl \ve{ F} \cdot \ve{ n} ) dA &= \int_S \omega \\
	&= \int_{ \partial S} f_1 dx_1 + f_2 dx_2 + f_3 dx_3 
\end{align*}
\end{problem}

\begin{problem}[4.7.7]
Since $ \omega$ is exact, there exists a $ (k-1)$-form $ \omega'$ s.t.\ $ d \omega' = \omega$. By Stokes Theorem,
\begin{align*}
	\int_{X} \omega = \int_{ \partial X} \omega' = \int_{ \O} \omega' = 0.
\end{align*}
\end{problem}
\begin{problem}[4.7.8]
Since $ \omega$ is a closed $ k$-form,  $ d \omega = 0$. Let $ F: W \to Y$ be the extension of $ f$. Then $ \partial F(W) = F(\partial W) = f(X)$. Thus we have
 \begin{align*}
	\int_X f^* \omega &= \int_{f(X)} \omega \\
	&= \int_{ \partial F(W)} \omega \\
	&= \int_{F(W)} d\omega && \text{Stokes} \\
	&= \int_{F(W)} 0 =0
\end{align*}
\end{problem}

\begin{problem}[4.7.9]
Let $ H: X \times I \to Y$ be the homotopy between $ f_0$ and $ f_1$. Since $ X$ is boundaryless,  $ \partial (X \times I) = X \times \{1\} - X \times \{0\}=: X_1 - X_0 $. Define $ f= f_0$ on $ X_0 $ and $ f= f_1$ on $ X_1$. Since $ f$ extends to  $ H$, by Exercise 8 we have
\begin{align*}
	\int_{\partial (X \times I)} f^*\omega &= 0 \\
	\int_{X_1 - X_0} f^* \omega &= 0 \\
	\int_{X_1} f^* \omega - \int_{X_0} f^* \omega&= 0 && X_0,X_1 \text{ are disjoint} \\
	\int_{X_1} f^* \omega &= \int_{X_0} f^* \omega\\
	\int_X f_1^* \omega &= \int_{X} f_0^* \omega  
\end{align*}
\end{problem}
\end{document}
