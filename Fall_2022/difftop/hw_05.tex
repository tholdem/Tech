\documentclass[12pt]{article}
%Fall 2022
% Some basic packages
\usepackage{standalone}[subpreambles=true]
\usepackage[utf8]{inputenc}
\usepackage[T1]{fontenc}
\usepackage{textcomp}
\usepackage[english]{babel}
\usepackage{url}
\usepackage{graphicx}
%\usepackage{quiver}
\usepackage{float}
\usepackage{enumitem}
\usepackage{lmodern}
\usepackage{comment}
\usepackage{hyperref}
\usepackage[usenames,svgnames,dvipsnames]{xcolor}
\usepackage[margin=1in]{geometry}
\usepackage{pdfpages}

\pdfminorversion=7

% Don't indent paragraphs, leave some space between them
\usepackage{parskip}

% Hide page number when page is empty
\usepackage{emptypage}
\usepackage{subcaption}
\usepackage{multicol}
\usepackage[b]{esvect}

% Math stuff
\usepackage{amsmath, amsfonts, mathtools, amsthm, amssymb}
\usepackage{bbm}
\usepackage{stmaryrd}
\allowdisplaybreaks

% Fancy script capitals
\usepackage{mathrsfs}
\usepackage{cancel}
% Bold math
\usepackage{bm}
% Some shortcuts
\newcommand{\rr}{\ensuremath{\mathbb{R}}}
\newcommand{\zz}{\ensuremath{\mathbb{Z}}}
\newcommand{\qq}{\ensuremath{\mathbb{Q}}}
\newcommand{\nn}{\ensuremath{\mathbb{N}}}
\newcommand{\ff}{\ensuremath{\mathbb{F}}}
\newcommand{\cc}{\ensuremath{\mathbb{C}}}
\newcommand{\ee}{\ensuremath{\mathbb{E}}}
\newcommand{\hh}{\ensuremath{\mathbb{H}}}
\renewcommand\O{\ensuremath{\emptyset}}
\newcommand{\norm}[1]{{\left\lVert{#1}\right\rVert}}
\newcommand{\dbracket}[1]{{\left\llbracket{#1}\right\rrbracket}}
\newcommand{\ve}[1]{{\bm{#1}}}
\newcommand\allbold[1]{{\boldmath\textbf{#1}}}
\DeclareMathOperator{\lcm}{lcm}
\DeclareMathOperator{\im}{im}
\DeclareMathOperator{\coim}{coim}
\DeclareMathOperator{\dom}{dom}
\DeclareMathOperator{\tr}{tr}
\DeclareMathOperator{\rank}{rank}
\DeclareMathOperator*{\var}{Var}
\DeclareMathOperator*{\ev}{E}
\DeclareMathOperator{\dg}{deg}
\DeclareMathOperator{\aff}{aff}
\DeclareMathOperator{\conv}{conv}
\DeclareMathOperator{\inte}{int}
\DeclareMathOperator*{\argmin}{argmin}
\DeclareMathOperator*{\argmax}{argmax}
\DeclareMathOperator{\graph}{graph}
\DeclareMathOperator{\sgn}{sgn}
\DeclareMathOperator*{\Rep}{Rep}
\DeclareMathOperator{\Proj}{Proj}
\DeclareMathOperator{\mat}{mat}
\DeclareMathOperator{\diag}{diag}
\DeclareMathOperator{\aut}{Aut}
\DeclareMathOperator{\gal}{Gal}
\DeclareMathOperator{\inn}{Inn}
\DeclareMathOperator{\edm}{End}
\DeclareMathOperator{\Hom}{Hom}
\DeclareMathOperator{\ext}{Ext}
\DeclareMathOperator{\tor}{Tor}
\DeclareMathOperator{\Span}{Span}
\DeclareMathOperator{\Stab}{Stab}
\DeclareMathOperator{\cont}{cont}
\DeclareMathOperator{\Ann}{Ann}
\DeclareMathOperator{\Div}{div}
\DeclareMathOperator{\curl}{curl}
\DeclareMathOperator{\nat}{Nat}
\DeclareMathOperator{\gr}{Gr}
\DeclareMathOperator{\vect}{Vect}
\DeclareMathOperator{\id}{id}
\DeclareMathOperator{\Mod}{Mod}
\DeclareMathOperator{\sign}{sign}
\DeclareMathOperator{\Surf}{Surf}
\DeclareMathOperator{\fcone}{fcone}
\DeclareMathOperator{\Rot}{Rot}
\DeclareMathOperator{\grad}{grad}
\DeclareMathOperator{\atan2}{atan2}
\DeclareMathOperator{\Ric}{Ric}
\let\vec\relax
\DeclareMathOperator{\vec}{vec}
\let\Re\relax
\DeclareMathOperator{\Re}{Re}
\let\Im\relax
\DeclareMathOperator{\Im}{Im}
% Put x \to \infty below \lim
\let\svlim\lim\def\lim{\svlim\limits}

%wide hat
\usepackage{scalerel,stackengine}
\stackMath
\newcommand*\wh[1]{%
\savestack{\tmpbox}{\stretchto{%
  \scaleto{%
    \scalerel*[\widthof{\ensuremath{#1}}]{\kern-.6pt\bigwedge\kern-.6pt}%
    {\rule[-\textheight/2]{1ex}{\textheight}}%WIDTH-LIMITED BIG WEDGE
  }{\textheight}% 
}{0.5ex}}%
\stackon[1pt]{#1}{\tmpbox}%
}
\parskip 1ex

%Make implies and impliedby shorter
\let\implies\Rightarrow
\let\impliedby\Leftarrow
\let\iff\Leftrightarrow
\let\epsilon\varepsilon

% Add \contra symbol to denote contradiction
\usepackage{stmaryrd} % for \lightning
\newcommand\contra{\scalebox{1.5}{$\lightning$}}

% \let\phi\varphi

% Command for short corrections
% Usage: 1+1=\correct{3}{2}

\definecolor{correct}{HTML}{009900}
\newcommand\correct[2]{\ensuremath{\:}{\color{red}{#1}}\ensuremath{\to }{\color{correct}{#2}}\ensuremath{\:}}
\newcommand\green[1]{{\color{correct}{#1}}}

% horizontal rule
\newcommand\hr{
    \noindent\rule[0.5ex]{\linewidth}{0.5pt}
}

% hide parts
\newcommand\hide[1]{}

% si unitx
\usepackage{siunitx}
\sisetup{locale = FR}

%allows pmatrix to stretch
\makeatletter
\renewcommand*\env@matrix[1][\arraystretch]{%
  \edef\arraystretch{#1}%
  \hskip -\arraycolsep
  \let\@ifnextchar\new@ifnextchar
  \array{*\c@MaxMatrixCols c}}
\makeatother

\renewcommand{\arraystretch}{0.8}

\renewcommand{\baselinestretch}{1.5}

\usepackage{graphics}
\usepackage{epstopdf}

\RequirePackage{hyperref}
%%
%% Add support for color in order to color the hyperlinks.
%% 
\hypersetup{
  colorlinks = true,
  urlcolor = blue,
  citecolor = blue
}
%%fakesection Links
\hypersetup{
    colorlinks,
    linkcolor={red!50!black},
    citecolor={green!50!black},
    urlcolor={blue!80!black}
}
%customization of cleveref
\RequirePackage[capitalize,nameinlink]{cleveref}[0.19]

% Per SIAM Style Manual, "section" should be lowercase
\crefname{section}{section}{sections}
\crefname{subsection}{subsection}{subsections}
\Crefname{section}{Section}{Sections}
\Crefname{subsection}{Subsection}{Subsections}

% Per SIAM Style Manual, "Figure" should be spelled out in references
\Crefname{figure}{Figure}{Figures}

% Per SIAM Style Manual, don't say equation in front on an equation.
\crefformat{equation}{\textup{#2(#1)#3}}
\crefrangeformat{equation}{\textup{#3(#1)#4--#5(#2)#6}}
\crefmultiformat{equation}{\textup{#2(#1)#3}}{ and \textup{#2(#1)#3}}
{, \textup{#2(#1)#3}}{, and \textup{#2(#1)#3}}
\crefrangemultiformat{equation}{\textup{#3(#1)#4--#5(#2)#6}}%
{ and \textup{#3(#1)#4--#5(#2)#6}}{, \textup{#3(#1)#4--#5(#2)#6}}{, and \textup{#3(#1)#4--#5(#2)#6}}

% But spell it out at the beginning of a sentence.
\Crefformat{equation}{#2Equation~\textup{(#1)}#3}
\Crefrangeformat{equation}{Equations~\textup{#3(#1)#4--#5(#2)#6}}
\Crefmultiformat{equation}{Equations~\textup{#2(#1)#3}}{ and \textup{#2(#1)#3}}
{, \textup{#2(#1)#3}}{, and \textup{#2(#1)#3}}
\Crefrangemultiformat{equation}{Equations~\textup{#3(#1)#4--#5(#2)#6}}%
{ and \textup{#3(#1)#4--#5(#2)#6}}{, \textup{#3(#1)#4--#5(#2)#6}}{, and \textup{#3(#1)#4--#5(#2)#6}}

% Make number non-italic in any environment.
\crefdefaultlabelformat{#2\textup{#1}#3}

% Environments
\makeatother
% For box around Definition, Theorem, \ldots
%%fakesection Theorems
\usepackage{thmtools}
\usepackage[framemethod=TikZ]{mdframed}

\theoremstyle{definition}
\mdfdefinestyle{mdbluebox}{%
	roundcorner = 10pt,
	linewidth=1pt,
	skipabove=12pt,
	innerbottommargin=9pt,
	skipbelow=2pt,
	nobreak=true,
	linecolor=blue,
	backgroundcolor=TealBlue!5,
}
\declaretheoremstyle[
	headfont=\sffamily\bfseries\color{MidnightBlue},
	mdframed={style=mdbluebox},
	headpunct={\\[3pt]},
	postheadspace={0pt}
]{thmbluebox}

\mdfdefinestyle{mdredbox}{%
	linewidth=0.5pt,
	skipabove=12pt,
	frametitleaboveskip=5pt,
	frametitlebelowskip=0pt,
	skipbelow=2pt,
	frametitlefont=\bfseries,
	innertopmargin=4pt,
	innerbottommargin=8pt,
	nobreak=false,
	linecolor=RawSienna,
	backgroundcolor=Salmon!5,
}
\declaretheoremstyle[
	headfont=\bfseries\color{RawSienna},
	mdframed={style=mdredbox},
	headpunct={\\[3pt]},
	postheadspace={0pt},
]{thmredbox}

\declaretheorem[%
style=thmbluebox,name=Theorem,numberwithin=section]{thm}
\declaretheorem[style=thmbluebox,name=Lemma,sibling=thm]{lem}
\declaretheorem[style=thmbluebox,name=Proposition,sibling=thm]{prop}
\declaretheorem[style=thmbluebox,name=Corollary,sibling=thm]{coro}
\declaretheorem[style=thmredbox,name=Example,sibling=thm]{eg}

\mdfdefinestyle{mdgreenbox}{%
	roundcorner = 10pt,
	linewidth=1pt,
	skipabove=12pt,
	innerbottommargin=9pt,
	skipbelow=2pt,
	nobreak=true,
	linecolor=ForestGreen,
	backgroundcolor=ForestGreen!5,
}

\declaretheoremstyle[
	headfont=\bfseries\sffamily\color{ForestGreen!70!black},
	bodyfont=\normalfont,
	spaceabove=2pt,
	spacebelow=1pt,
	mdframed={style=mdgreenbox},
	headpunct={ --- },
]{thmgreenbox}

\declaretheorem[style=thmgreenbox,name=Definition,sibling=thm]{defn}

\mdfdefinestyle{mdgreenboxsq}{%
	linewidth=1pt,
	skipabove=12pt,
	innerbottommargin=9pt,
	skipbelow=2pt,
	nobreak=true,
	linecolor=ForestGreen,
	backgroundcolor=ForestGreen!5,
}
\declaretheoremstyle[
	headfont=\bfseries\sffamily\color{ForestGreen!70!black},
	bodyfont=\normalfont,
	spaceabove=2pt,
	spacebelow=1pt,
	mdframed={style=mdgreenboxsq},
	headpunct={},
]{thmgreenboxsq}
\declaretheoremstyle[
	headfont=\bfseries\sffamily\color{ForestGreen!70!black},
	bodyfont=\normalfont,
	spaceabove=2pt,
	spacebelow=1pt,
	mdframed={style=mdgreenboxsq},
	headpunct={},
]{thmgreenboxsq*}

\mdfdefinestyle{mdblackbox}{%
	skipabove=8pt,
	linewidth=3pt,
	rightline=false,
	leftline=true,
	topline=false,
	bottomline=false,
	linecolor=black,
	backgroundcolor=RedViolet!5!gray!5,
}
\declaretheoremstyle[
	headfont=\bfseries,
	bodyfont=\normalfont\small,
	spaceabove=0pt,
	spacebelow=0pt,
	mdframed={style=mdblackbox}
]{thmblackbox}

\theoremstyle{plain}
\declaretheorem[name=Question,sibling=thm,style=thmblackbox]{ques}
\declaretheorem[name=Remark,sibling=thm,style=thmgreenboxsq]{remark}
\declaretheorem[name=Remark,sibling=thm,style=thmgreenboxsq*]{remark*}
\newtheorem{ass}[thm]{Assumptions}

\theoremstyle{definition}
\newtheorem*{problem}{Problem}
\newtheorem{claim}[thm]{Claim}
\theoremstyle{remark}
\newtheorem*{case}{Case}
\newtheorem*{notation}{Notation}
\newtheorem*{note}{Note}
\newtheorem*{motivation}{Motivation}
\newtheorem*{intuition}{Intuition}
\newtheorem*{conjecture}{Conjecture}

% Make section starts with 1 for report type
%\renewcommand\thesection{\arabic{section}}

% End example and intermezzo environments with a small diamond (just like proof
% environments end with a small square)
\usepackage{etoolbox}
\AtEndEnvironment{vb}{\null\hfill$\diamond$}%
\AtEndEnvironment{intermezzo}{\null\hfill$\diamond$}%
% \AtEndEnvironment{opmerking}{\null\hfill$\diamond$}%

% Fix some spacing
% http://tex.stackexchange.com/questions/22119/how-can-i-change-the-spacing-before-theorems-with-amsthm
\makeatletter
\def\thm@space@setup{%
  \thm@preskip=\parskip \thm@postskip=0pt
}

% Fix some stuff
% %http://tex.stackexchange.com/questions/76273/multiple-pdfs-with-page-group-included-in-a-single-page-warning
\pdfsuppresswarningpagegroup=1


% My name
\author{Jaden Wang}



\begin{document}
\centerline {\textsf{\textbf{\LARGE{Homework 5}}}}
\centerline {Jaden Wang}
\vspace{.15in}
\begin{problem}[9.1]
Suppose $ df_p $ has rank  $ k$, this means that  $ df_p$ maps  $ T_pM$ to a  $ k$-dimensional subspace of  $ T_pN$. Then the matrix representation of  $ df_p$ would have $ k$-pivots after row reduction so the  $ k \times k$ upper left submatrix has nonzero determinant. So take an open interval $ I$ between this number and 0.

Let $ (U,\phi)$ be a chart around $ p$ and $ (V,\psi)$ be a chart around $ f(p)$. Then $F:= \psi \circ f \circ \phi ^{-1}: \rr^{m} \to \rr^{n}$. Then $ dF_p: T_p \rr^{m} \to T_p \rr^{n}$ is just the derivative of $ F$ at  $ p$ and can be identified as an element of $ L(\rr^{m}, \rr^{n})$. It's easy to see that $\rank dF_p = \rank df_p = k$. Since $ f,\phi ^{-1},\psi$ are smooth, $ F$ is smooth and so are all of its derivatives. In particular, $ F': \rr^{m} \to L(\rr^{m}, \rr^{n}), p \mapsto dF_p$ is continuous. Then by identifying $ dF_p$ with a matrix representation where the first $ k$-columns are linearly independent (since it has rank $ k$), we can take the determinant of its upper $ k\times k$ minor. Denote this function by $ \det_k $. Now we have a chain of composing continuous functions, denoted by $ g: U \to \rr$:
\begin{align*}
	M \supseteq U \xrightarrow{ \phi} \rr^{m} \xrightarrow{ F'} L(\rr^{m}, \rr^{n}) \xrightarrow{ \det_k}   \rr .
\end{align*}
Since $ dF_p$ has rank  $ k$, $ \det_k(dF_p) = a \neq 0$. WLOG suppose $ a>0$ and let  $ I:=(0,a)$. Therefore, $ g^{-1}(I)$ is open in $ U$. That is, there exists an open set $ V \subseteq M$ s.t.\ $ V \cap U = g^{-1}(I)$. Since intersection of open sets is open, $ g^{-1}(I)$ is open in $ M$. By the way we define $ g$, we see that the differential has rank  $ k$ everywhere in $ g^{-1}(I)$, as desired.
\end{problem}


\begin{problem}[9.3]
Since $ q$ is a regular value of  $ f$,  by the regular value theorem, $ f^{-1}(q)$ is a submanifold of dimension $ \dim M - \dim N = 0$ so it is a set of points. Since N is Hausdorff, $ \{q\} $ is closed in $ N$ so by continuity of  $ f$,  $ f^{-1}(q)$ is also closed. Since $ M$ is compact, the closed subspace $ f^{-1}(q)$ must also be compact (Theorem 26.2 of Munkres). Take any $ p \in f^{-1}(q)$, since $ \rank df_p = \dim N = \dim M$, $ df_p$ is a linear isomorphism so  $ f$ is a local diffeomorphism at  $ p$. That is, each $ p$ has a neighborhood that is diffeomorphic to a neighborhood of $ q$. We see that each of these neighborhood contains exactly one point (otherwise the diffeomorphism wouldn't be 1-1) and all of them together form a cover of $ f^{-1}(q)$. Since $ f^{-1}(q)$ is compact, we can find a finite subcover. Since each neighborhood contains exactly one point, $ f^{-1}(q)$ must only have finite number of points.

Denote these points as $ \{p_i\}_{i=1}^{n} $ and the respective disjoint neighborhoods as $ \{V_i\}_{i=1}^{n}$. Let $ \{U_i\}_{i=1}^{n} $ be the corresponding diffeomorphic neighborhood of $ q$ and set $U:= \bigcap_{ i= 1}^{ n}  U_i$, which is still an open neighborhood of $ q$. Now denote $C := M - \bigcup_{ i= 1}^{ n} V_i$ which is closed. Since $ M$ is compact,  $ C$ is also compact. Then  $ f(C)$ is compact and therefore closed since  $ N$ is Hausdorff. Hence  $ N - C$ is open so  $ \widetilde{ U} := U \cap (N-C)$ is open. Take any $ q' \in \widetilde{ U}$, we see that $ f^{-1}(q')$ must have one point in each $ V_i$ by local diffeomorphism and nowhere else outside $ V_i$. Therefore we have $ \#f^{-1}(q') = \#f^{-1}(q) = n$.
\end{problem}

\begin{problem}[9.5]
In exercise 8 of Lecture Notes 5, we have already shown that  $ dP_z(w) = \theta_z ^{-1}(P'(z) \theta_z(w))$ where $ P'(z)$ can be expressed as the Jacobian of  $ P$ at  $ z$.

$ (\implies):$ if $ z$ is a singular point of  $ P$, then  $\rank dP_z=0,1$. But by the Cauchy-Riemann equations, $ \det dP_z = \left( \frac{\partial u}{ \partial x} \right)^2 + \left( \frac{ \partial u}{\partial y } \right)^2$, which is positive definite. Hence $ \det dP_z = 0$ iff  $ dP_z = 0$ which is equivalent to  $ P'(z) = 0$ so  $ z$ is a root of  $ P'(z)$.

$ (\impliedby):$ suppose $ z$ is a root of  $ P'(z)$. Then  $ dP_z(w) = P'(z)w = 0 w = 0$ so  $ \rank dP_z = 0 \neq 2$, so  $ z$ is a singular point.
\end{problem}

\begin{problem}[9.6]
Since any boundary point $ p$ of $ M$ cannot be mapped to the interior of  $ H^{m}$, $ p$ must have a neighborhood $ U_p$ that is mapped homeomorphically by $ \phi$ into the boundary $ \rr^{m-1} \times \{0\} \cong \rr^{m-1}$. This yields a well-defined homeomorphism $ \phi|_{ U_p}: U_p \to \phi(U_p) \subseteq \rr^{m-1}$. That is, $ \partial M$ is locally homeomorphic to an open subset of $ \rr^{m-1}$. Since $ \partial M$ as a subspace of $ M$ inherits Hausdorff and second-countable, we show that $ \partial M$ is a $ (m-1)$-manifold. Since no point in $ \partial M$ is locally homeomorphic to the boundary of $ H^{m-1}$, we see that $ \partial M$ has no boundary point and therefore no boundary.
\end{problem}

\begin{problem}[9.9]
	Notice that any half-curve $ \gamma: [0, \epsilon) \to M$ or $ \gamma:(- \epsilon,0]$ with $ \gamma(0) = p$ can be smoothly extended to a curve $ \widetilde{ \gamma}:(- \epsilon, \epsilon) \to M$ with $ \widetilde{ \gamma}(0) =p$ (by just going straight along the tangent vector direction). It's therefore easy to see that equivalent classes of half-curves (elements of $ T_p H^{m}$) yield the same tangent space at $ p$ as the equivalent classes of curves (elements of $ T_p U$), \emph{i.e.} $ T_p H^{m} = T_p U$. Observe $ df_p:T_p H^{m}= T_p U \to T_pM, \frac{d}{dt}\bigg|_{t=0} p+tv \mapsto \frac{d}{dt}\bigg|_{t=0} f(p+tv)$ and $ d \widetilde{ f}_p: T_pU \to T_pM, \frac{d}{dt}\bigg|_{t=0} p+tv \mapsto \frac{d}{dt}\bigg|_{t=0} \widetilde{ f}(p+tv)$. Since we already know that $ f = \widetilde{ f}$ on $ U \cap H^{m}$ of $ p$, by taking $ \epsilon$ small enough s.t.\ $ p+tv \in U \cap H^{m}$, $ \widetilde{ f}(p+tv) = f(p+tv)$ $ \ \forall \ t \in [0, \epsilon)$, so they have the same derivative at $ p$ (by extension of half-curve this is well-defined). Thus we have $ df_p([v]) = d \widetilde{ f}_p([v])$ and they thus equal as functions.
\end{problem}

\begin{problem}[9.13]
First, $ f^{-1}(q)$ is a submanifold of $ \dim (m-n)$. $ df_p: T_pM \to T_pN$ has rank $ n$ so its null space has dimension  $ (m-n)$.  Since every point in $ f^{-1}(q)$ maps to $ q$, this is a constant map so its derivative is 0,  \emph{i.e.} $ df_p|_{f^{-1}(q)} = 0$. Thus $ T_p f^{-1}(q) \subseteq \ker df_p$. But since they have the same dimension, they must be isomorphic so we achieve equality $ T_p f^{-1}(q) = \ker df_p$.
\end{problem}
\end{document}
