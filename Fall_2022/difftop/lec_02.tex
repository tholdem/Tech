\documentclass[12pt,class=article,crop=false]{standalone} 
\newcommand{\alert}[1]{{\bf \color{red} [Alert:] #1}}
\newcommand{\todo}[1]{{\bf \color{orange} [TODO:] #1}}
\newcommand{\real}[1][]{\mathbb{R}^{#1}}
\newcommand{\myeqn}[1]{(\ref{#1})}
\newcommand{\myex}[1]{Example \ref{#1}}
\newcommand{\defeq}{\stackrel{\mathrm{def}}{=}}
\newcommand{\parder}[2]{\frac{\partial #1}{\partial #2}}
\newcommand{\Lie}[3][]{\mathsf{L}_{#3}^{#1} #2}
\newcommand{\LieA}[1]{\mathsf{Lie}(#1)}
\newcommand{\lieder}[2]{\mathcal{L}_{#2} #1}
\renewcommand{\t}{^{\mbox{\tiny\sf T}}}
\newcommand{\trans}{^{\mbox{\tiny\sf T}}}
\newcommand{\markup}[1]{\{\textbf{#1}\}}
\newcommand{\msub}[1]{_\mathrm{#1}}
\newcommand{\msup}[1]{^\mathrm{#1}}
\newcommand{\inv}[1]{#1^{-1}}
\newcommand{\pinv}[1]{{#1}^{+}}
\newcommand{\myfracA}[2]{\displaystyle{\frac{#1}{#2}}}
\newcommand{\myfracB}[2]{{#1}/{#2}}
\newcommand{\mydiffA}[1]{\dot{#1}}
\newcommand{\mydiffB}[2]{\myfracA{\mathrm{d}{#1}}{\mathrm{d}{#2}}}
\newcommand{\ball}[2]{\mathcal{B}_{#1}\left(#2\right)}
\newcommand{\acos}[1]{\cos^{-1}\left(#1\right)}
\newcommand{\asin}[1]{\sin^{-1}\left(#1\right)}
\newcommand{\mani}{\mathcal{M}}
\newcommand{\tang}[2]{\mathsf{T}_{#1} #2}
\newcommand{\LieB}[2]{[ #1, #2 ]}
\newcommand{\LieBad}[3][]{\mathsf{ad}_{#2}^{#1} #3}
\newcommand{\ReachVT}{\mathcal{R}^V_T}
\newcommand{\ReachVt}{\mathcal{R}^V_t}
\newcommand{\ReachVTe}{\mathcal{R}^V_{\le T}}
\newcommand{\ReachT}{\mathcal{R}_T}
\newcommand{\Reacht}{\mathcal{R}_t}
\newcommand{\ReachTe}{\mathcal{R}_{\le T}}
\newcommand{\accLA}[1]{\mathsf{Lie}(#1)}
\newcommand{\accD}{\Delta_{\mathcal{F}}}
\newcommand{\accSA}{\mathsf{Lie}(\mathcal{G},f)}
\newcommand{\accDS}{\Delta_{\mathcal{G}}}
\newcommand{\eval}[3]{\mathsf{Ev}^{#2}_{#1}\left( #3 \right)}
\newcommand{\stlc}{\textsc{stlc}}
\newcommand{\clf}{\textsc{clf}}
\newcommand{\jqlf}{\textsc{jqlf}}
\newcommand{\dlas}{\textsc{dlas}}
\newcommand{\Ad}[2]{\mathsf{Ad}_{#1} #2}
\newcommand{\xe}{\ensuremath{x_e}}
\newcommand{\lebg}[1]{\mathcal{L}_{#1}}
\newcommand{\lebgx}[1]{\mathcal{L}_{#1 \mathrm{e}}}
\newcommand{\dom}{D}
\newcommand{\domT}{[t_0,\infty) \times D}
\newcommand{\rarrow}{\rightarrow}
\renewcommand{\d}{\mathrm{d}}
\renewcommand{\Re}{\mathbb{R}}
\newcommand{\C}{\mathrm{C}}

\newcommand{\QED}{{\unskip\nobreak\hfil\penalty50\hskip2em\vadjust{}
		\nobreak\hfil$\Box$\parfillskip=0pt\finalhyphendemerits=0\par}\vspace{0.1cm}}
\newcommand{\eoEx}{{\unskip\nobreak\hfil\penalty50\hskip0em\vadjust{}
		\nobreak\hfil$\Large\Diamond$\parfillskip=0pt\finalhyphendemerits=0\par}\vspace{0.1cm}}

\newcommand{\sgn}{\ensuremath{\operatorname{sgn}}}
\newcommand{\sat}{\ensuremath{\operatorname{sat}}}

\newcommand{\half}{\frac{1}{2}}
\newcommand{\shalf}{\mbox{$\frac{1}{2}$}}
\newcommand{\marcom}[1]{\marginpar{\footnotesize #1}}
\newcommand{\der}{\mathrm{D}}
\newcommand{\e}{\mathrm{e}}
\newcommand{\dt}{\mathrm{d}t}

\newcommand{\cA}{\ensuremath{\mathcal{A}}}
\newcommand{\cB}{\ensuremath{\mathcal{B}}}
\newcommand{\cG}{\ensuremath{\mathcal{G}}}
\newcommand{\cK}{\ensuremath{\mathcal{K}}}
\newcommand{\cW}{\ensuremath{\mathcal{W}}}
\newcommand{\cZ}{\ensuremath{\mathcal{Z}}}
\newcommand{\cS}{\ensuremath{\mathcal{S}}}
\newcommand{\cD}{\ensuremath{\mathcal{D}}}
\newcommand{\cP}{\ensuremath{\mathcal{P}}}
\newcommand{\cV}{\ensuremath{\mathcal{V}}}
\newcommand{\cL}{\ensuremath{\mathcal{L}}}
\newcommand{\cN}{\ensuremath{\mathcal{N}}}
\newcommand{\cI}{\ensuremath{\mathcal{I}}}
\newcommand{\cR}{\ensuremath{\mathcal{R}}}
\newcommand{\cM}{\ensuremath{\mathcal{M}}}
\newcommand{\cC}{\ensuremath{\mathcal{C}}}
\newcommand{\cF}{\ensuremath{\mathcal{F}}}
\newcommand{\cH}{\ensuremath{\mathcal{H}}}
\newcommand{\cO}{\ensuremath{\mathcal{O}}}
\newcommand{\cX}{\ensuremath{\mathcal{X}}}
\newcommand{\cY}{\ensuremath{\mathcal{Y}}}
\newcommand{\Ci}{\ensuremath{\mathcal{C}^\infty}}
\newcommand{\ISS}{\textsc{iss}}
\newcommand{\LISS}{\textsc{liss}}
\newcommand{\GAS}{\textsc{gas}}
\newcommand{\GS}{\textsc{gs}}
\newcommand{\LES}{\textsc{les}}
\newcommand{\GUAS}{\textsc{guas}}
\newcommand{\BIBO}{\textsc{bibo}}
\newcommand{\spec}{\ensuremath{\operatorname{spec}}}
\newcommand{\spn}{\ensuremath{\operatorname{span}}}
\renewcommand{\i}{\mathrm{i\,}}

\renewcommand{\implies}{\Rightarrow}

\renewcommand{\theenumi}{$\roman{enumi})$}
\renewcommand{\labelenumi}{\theenumi}

\font\ptmten=zptmcmrm scaled 1200
\newcommand{\w}{\mbox{{\ptmten w}}}
\newcommand{\z}{\mbox{{\ptmten z}}}
\renewcommand{\Re}{\mathbb{R}}

\newcommand{\cl}{\operatorname{cl}}
\newcommand{\intr}{\operatorname{int}}
\newcommand{\rank}{\operatorname{rank}}
\newcommand{\co}{\operatorname{co}}
\newcommand{\aff}{\operatorname{aff}}

\theoremstyle{plain}
\newtheorem{theorem}{Theorem}[chapter]
\newtheorem{claim}[theorem]{Claim}
\newtheorem{corollary}[theorem]{Corollary}
\newtheorem{prop}[theorem]{Proposition}
\newtheorem{fact}[theorem]{Fact}
\newtheorem{lemma}[theorem]{Lemma}

\newtheorem{remark}{Remark}[chapter]

\theoremstyle{definition}
\newtheorem{assume}[theorem]{Assumption}
\newtheorem{defn}[theorem]{Definition}
\newtheorem{problem}[theorem]{Problem}
\newtheorem{exercise}{Exercise}
\newtheorem{example}[theorem]{Example}


\begin{document}
\section{Tangent Spaces}
There is no linear structure on manifolds.

\begin{eg}[tangent space on $ \rr^n$]
$ T_p \rr^{n} = \{(p,q): q \in \rr^{n}\}$. This is a vector space. Note that $ T_p \rr^{n} \cong \rr^{n}, (p,q) \mapsto q-p$. This is canonical. We lose this canonical isomorphism in other manifolds.
\end{eg}

From the submanifold perspective:
\begin{claim}
$ \dim(T_pM) = n $.
\end{claim}
\begin{proof}
\begin{align*}
	\alpha &= \phi ^{-1} \circ  (\phi \circ  \alpha)\\
	\alpha'(0)&= D_0 \phi ^{-1} (\phi \circ \alpha)'(0) 
\end{align*}
\begin{claim}
$ \rank D_0 \phi ^{-1} = n$.
\end{claim}
Since $ \overline{\phi} \circ \phi ^{-1} = \text{id}_{ \phi(U)} $, by chain rule,
\begin{align*}
	D \overline{\phi} \circ D \phi ^{-1} = I
\end{align*}
which yields that $ \rank D \overline{\phi} \circ D \phi ^{-1} =  n$. It follows that $ \rank D \phi_^{-1} \geq n$ which yields equality.
\end{proof}

From the intrinsic perspective, we define $ d f_p: T_pM \to T_{f(p)}N$. Given $  v \in T_pM$, where $ v = \alpha'(0)$ for some curve $ \alpha$ that passes through $ p$. Then define
 \begin{align*}
	d f_p (v) := (f \circ \alpha)'(0)
\end{align*}
Exercise: If $ M = \rr^{n}$ and $ N = \rr^{m}$, then $ d f_p(v) $ reduces to the Jacobian at  $ p$ multiplying by  $ v$ (directional derivative).

Define $ \gamma'(0)$ by operating on functions $ M \to \rr$:
\begin{align*}
	[\gamma'(0)](f) := (f \circ \gamma)'(0)
\end{align*}
So $ \gamma \sim \widetilde{ \gamma}$ iff $ [ \gamma(0)] f = [ \gamma'(0)]f$ for all $ f \in \mathcal{ F}(M)$ space of smooth functions on $ M$.

 \begin{prop}
$ T_pM$ is a vector space with natural operations and has the same dimension as that of the manifold.
\end{prop}
\begin{proof}
\begin{align*}
	[ \alpha'(0)] f &= (f \circ \alpha)'(0) \\
	&= (f \circ x \circ x^{-1} \circ \alpha)'(0) \\
	&= \sum_{ i= 1}^{ n} \frac{ \partial (f \circ x)}{ \partial x_i} ((x^{-1} \circ \alpha)^{i} )'(0)
\end{align*}
Define local coordinates via $ x: \rr^{n} \to M. x_i(t) = x(0,\ldots,0,t,0,\ldots0)$ at $ i$th entry. Then
 \begin{align*}
	 [x_i'(0)] f = (f \circ  x_i)'(0) = \frac{ \partial (f \circ x)}{ \partial x_i} (0)= \sum \lambda_i [x_i'(0)](f).
\end{align*}
So $ \alpha'(0)$ is a linear combination of $ (x_i)'(0)$. It remains to show they are linearly independent. Suppose $ \sum \lamdba_i x_i'(0) =0$.
\end{proof}

\begin{defn}
Let $ f: M \to N$, $ p \in M$, define $ df_p ( \alpha'(0)) = (f \circ \alpha)'(0)$.
\end{defn}
Exercise: this is linear. If $ M = \rr^{n}$, $ N = \rr^{m}$, $ df_p$ is the Jacobian.

\begin{thm}[Chain Rule]
Let $ f: M \to N$ and $ g: N \to L$. Then
\begin{align*}
	d(g \circ f) _p = dg_{f(p)} \circ df_p.
\end{align*}
\end{thm}
\begin{proof}
\begin{align*}
	d(g \circ f) ( \alpha'(0)) &= (g \circ f \circ \alpha)'(0) \\
	&= dg( (f\circ \alpha)'(0)) \\
	&= dg \circ df( \alpha'(0)) 
\end{align*}
\end{proof}

\begin{coro}
If $ M,N$ are diffeomorphic, then  $ \dim M = \dim N$.
\end{coro}
\begin{proof}
Let $ f:M \to N$ be the diffeomorphism, then
\begin{align*}
	f \circ f^{-1} &= \text{id}_M  \\
	df \circ df^{-1} &= I 
\end{align*}
Thus $ df$ is invertible and  $\dim M = \dim T_pM = \dim T_{f(p)}N = \dim N$.
\end{proof}
\end{document}
