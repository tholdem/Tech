\documentclass[12pt]{article}
\newcommand{\alert}[1]{{\bf \color{red} [Alert:] #1}}
\newcommand{\todo}[1]{{\bf \color{orange} [TODO:] #1}}
\newcommand{\real}[1][]{\mathbb{R}^{#1}}
\newcommand{\myeqn}[1]{(\ref{#1})}
\newcommand{\myex}[1]{Example \ref{#1}}
\newcommand{\defeq}{\stackrel{\mathrm{def}}{=}}
\newcommand{\parder}[2]{\frac{\partial #1}{\partial #2}}
\newcommand{\Lie}[3][]{\mathsf{L}_{#3}^{#1} #2}
\newcommand{\LieA}[1]{\mathsf{Lie}(#1)}
\newcommand{\lieder}[2]{\mathcal{L}_{#2} #1}
\renewcommand{\t}{^{\mbox{\tiny\sf T}}}
\newcommand{\trans}{^{\mbox{\tiny\sf T}}}
\newcommand{\markup}[1]{\{\textbf{#1}\}}
\newcommand{\msub}[1]{_\mathrm{#1}}
\newcommand{\msup}[1]{^\mathrm{#1}}
\newcommand{\inv}[1]{#1^{-1}}
\newcommand{\pinv}[1]{{#1}^{+}}
\newcommand{\myfracA}[2]{\displaystyle{\frac{#1}{#2}}}
\newcommand{\myfracB}[2]{{#1}/{#2}}
\newcommand{\mydiffA}[1]{\dot{#1}}
\newcommand{\mydiffB}[2]{\myfracA{\mathrm{d}{#1}}{\mathrm{d}{#2}}}
\newcommand{\ball}[2]{\mathcal{B}_{#1}\left(#2\right)}
\newcommand{\acos}[1]{\cos^{-1}\left(#1\right)}
\newcommand{\asin}[1]{\sin^{-1}\left(#1\right)}
\newcommand{\mani}{\mathcal{M}}
\newcommand{\tang}[2]{\mathsf{T}_{#1} #2}
\newcommand{\LieB}[2]{[ #1, #2 ]}
\newcommand{\LieBad}[3][]{\mathsf{ad}_{#2}^{#1} #3}
\newcommand{\ReachVT}{\mathcal{R}^V_T}
\newcommand{\ReachVt}{\mathcal{R}^V_t}
\newcommand{\ReachVTe}{\mathcal{R}^V_{\le T}}
\newcommand{\ReachT}{\mathcal{R}_T}
\newcommand{\Reacht}{\mathcal{R}_t}
\newcommand{\ReachTe}{\mathcal{R}_{\le T}}
\newcommand{\accLA}[1]{\mathsf{Lie}(#1)}
\newcommand{\accD}{\Delta_{\mathcal{F}}}
\newcommand{\accSA}{\mathsf{Lie}(\mathcal{G},f)}
\newcommand{\accDS}{\Delta_{\mathcal{G}}}
\newcommand{\eval}[3]{\mathsf{Ev}^{#2}_{#1}\left( #3 \right)}
\newcommand{\stlc}{\textsc{stlc}}
\newcommand{\clf}{\textsc{clf}}
\newcommand{\jqlf}{\textsc{jqlf}}
\newcommand{\dlas}{\textsc{dlas}}
\newcommand{\Ad}[2]{\mathsf{Ad}_{#1} #2}
\newcommand{\xe}{\ensuremath{x_e}}
\newcommand{\lebg}[1]{\mathcal{L}_{#1}}
\newcommand{\lebgx}[1]{\mathcal{L}_{#1 \mathrm{e}}}
\newcommand{\dom}{D}
\newcommand{\domT}{[t_0,\infty) \times D}
\newcommand{\rarrow}{\rightarrow}
\renewcommand{\d}{\mathrm{d}}
\renewcommand{\Re}{\mathbb{R}}
\newcommand{\C}{\mathrm{C}}

\newcommand{\QED}{{\unskip\nobreak\hfil\penalty50\hskip2em\vadjust{}
		\nobreak\hfil$\Box$\parfillskip=0pt\finalhyphendemerits=0\par}\vspace{0.1cm}}
\newcommand{\eoEx}{{\unskip\nobreak\hfil\penalty50\hskip0em\vadjust{}
		\nobreak\hfil$\Large\Diamond$\parfillskip=0pt\finalhyphendemerits=0\par}\vspace{0.1cm}}

\newcommand{\sgn}{\ensuremath{\operatorname{sgn}}}
\newcommand{\sat}{\ensuremath{\operatorname{sat}}}

\newcommand{\half}{\frac{1}{2}}
\newcommand{\shalf}{\mbox{$\frac{1}{2}$}}
\newcommand{\marcom}[1]{\marginpar{\footnotesize #1}}
\newcommand{\der}{\mathrm{D}}
\newcommand{\e}{\mathrm{e}}
\newcommand{\dt}{\mathrm{d}t}

\newcommand{\cA}{\ensuremath{\mathcal{A}}}
\newcommand{\cB}{\ensuremath{\mathcal{B}}}
\newcommand{\cG}{\ensuremath{\mathcal{G}}}
\newcommand{\cK}{\ensuremath{\mathcal{K}}}
\newcommand{\cW}{\ensuremath{\mathcal{W}}}
\newcommand{\cZ}{\ensuremath{\mathcal{Z}}}
\newcommand{\cS}{\ensuremath{\mathcal{S}}}
\newcommand{\cD}{\ensuremath{\mathcal{D}}}
\newcommand{\cP}{\ensuremath{\mathcal{P}}}
\newcommand{\cV}{\ensuremath{\mathcal{V}}}
\newcommand{\cL}{\ensuremath{\mathcal{L}}}
\newcommand{\cN}{\ensuremath{\mathcal{N}}}
\newcommand{\cI}{\ensuremath{\mathcal{I}}}
\newcommand{\cR}{\ensuremath{\mathcal{R}}}
\newcommand{\cM}{\ensuremath{\mathcal{M}}}
\newcommand{\cC}{\ensuremath{\mathcal{C}}}
\newcommand{\cF}{\ensuremath{\mathcal{F}}}
\newcommand{\cH}{\ensuremath{\mathcal{H}}}
\newcommand{\cO}{\ensuremath{\mathcal{O}}}
\newcommand{\cX}{\ensuremath{\mathcal{X}}}
\newcommand{\cY}{\ensuremath{\mathcal{Y}}}
\newcommand{\Ci}{\ensuremath{\mathcal{C}^\infty}}
\newcommand{\ISS}{\textsc{iss}}
\newcommand{\LISS}{\textsc{liss}}
\newcommand{\GAS}{\textsc{gas}}
\newcommand{\GS}{\textsc{gs}}
\newcommand{\LES}{\textsc{les}}
\newcommand{\GUAS}{\textsc{guas}}
\newcommand{\BIBO}{\textsc{bibo}}
\newcommand{\spec}{\ensuremath{\operatorname{spec}}}
\newcommand{\spn}{\ensuremath{\operatorname{span}}}
\renewcommand{\i}{\mathrm{i\,}}

\renewcommand{\implies}{\Rightarrow}

\renewcommand{\theenumi}{$\roman{enumi})$}
\renewcommand{\labelenumi}{\theenumi}

\font\ptmten=zptmcmrm scaled 1200
\newcommand{\w}{\mbox{{\ptmten w}}}
\newcommand{\z}{\mbox{{\ptmten z}}}
\renewcommand{\Re}{\mathbb{R}}

\newcommand{\cl}{\operatorname{cl}}
\newcommand{\intr}{\operatorname{int}}
\newcommand{\rank}{\operatorname{rank}}
\newcommand{\co}{\operatorname{co}}
\newcommand{\aff}{\operatorname{aff}}

\theoremstyle{plain}
\newtheorem{theorem}{Theorem}[chapter]
\newtheorem{claim}[theorem]{Claim}
\newtheorem{corollary}[theorem]{Corollary}
\newtheorem{prop}[theorem]{Proposition}
\newtheorem{fact}[theorem]{Fact}
\newtheorem{lemma}[theorem]{Lemma}

\newtheorem{remark}{Remark}[chapter]

\theoremstyle{definition}
\newtheorem{assume}[theorem]{Assumption}
\newtheorem{defn}[theorem]{Definition}
\newtheorem{problem}[theorem]{Problem}
\newtheorem{exercise}{Exercise}
\newtheorem{example}[theorem]{Example}


\begin{document}
\centerline {\textsf{\textbf{\LARGE{Homework 2}}}}
\centerline {Jaden Wang}
\vspace{.15in}
\begin{problem}[4.4]
There is a canonical homeomorphism $ \phi: \rr^{m \times n} \to \rr^{mn}$ by concatenating the columns together. This is a linear operation as it is clearly closed under addition and scalar multiplication and therefore smooth. Hence $ (\phi, \rr^{m \times n})$ is a global chart. Hausdorff and second-countable follows from homeomorphism and therefore $ \rr^{m \times n}$ is a smooth manifold.

Notice that the general linear group $ \text{GL}_{ n}( R) = \{M \in \rr^{m \times n} : \det (M) \neq 0\} = \det ^{-1}( \rr \setminus \{0\}) $. Since $ \{0\} $ is closed ($ \rr$ is Hausdorff and hence T1), $ \rr \setminus \{0\} $ is open. The determinant function is a polynomial of entries and therefore continuous, so $ \text{GL}_{n}( R) $ as the preimage of an open set of $ \rr$ via a continuous function is open in $ \rr^{m \times n}$. By theorem any open subset of a smooth manifold is a smooth manifold.
\end{problem}

\begin{problem}[4.6]
Given $ (U_1,\phi_1),(V_1,\psi_1)$ and $ (U_2,\phi_2),(V_2,\phi_2)$ be two sets of local charts that satisfy the assumptions. Notice that since $ \rr^{m}$ and $ \rr^{n}$ are smooth manifolds, the transition maps $ \phi_1 \circ \phi_2 ^{-1}: U_1 \cap U_2 \to \rr^{m}$ and $ \psi_2 \circ \psi_1 ^{-1} : V_1 \cap V_2 \to \rr^{n}$ are also smooth. Thus if $\psi \circ  f \circ \phi ^{-1}$ is smooth, then
\begin{align*}
	\psi_2 \circ f \circ \phi_2 ^{-1} = (\psi_2 \circ \psi_1 ^{-1}) \circ (\psi_1 \circ f \circ \phi_1 ^{-1}) \circ (\phi_1 \circ \phi_2 ^{-1})
\end{align*}
is a composition of smooth functions and therefore smooth. Thus $ f$ is smooth regardless of the choice of local charts.
\end{problem}

\begin{problem}[5.8]
	Recall that for any smooth map $ f: \rr^{n} \to \rr^{m}$, $ d_pf : T_p \rr^{n} \to T_{f(p)} \rr^{m}, [ \alpha] \mapsto [f \circ \alpha]$. By the identification, we can also think $ d_pf: \rr^{n} \to \rr^{m}, \alpha'(0) \mapsto (f \circ \alpha)'(0)$. But by the chain rule, $ (f \circ \alpha)'(0) = Df(p) \circ \alpha'(0)$. That is, $ d_p f = Df(p)$ which is exactly the Jacobian of $ f$ at $ p$.
\end{problem}
\begin{problem}[5.9]
	Let $ f:M \to N$ and $ g; N \to L$ be smooth maps. Then $ d_p(g \circ f): T_p \rr^{m} \to T_p \rr^{ \ell}, [ \alpha] \mapsto [g \circ f \circ \alpha]$. Choose any chart $ (U,\phi),(V, \psi),(W, \theta)$ containing $ p,f(p), g \circ f(p)$. Then
\begin{align*}
	d_p(g \circ f)([ \alpha]) &=(\theta \circ g \circ f \circ \alpha)'(0) \\
	&= (\theta \circ g \circ  \psi ^{-1}) \circ (\psi \circ f \circ \phi ^{-1}) \circ (\phi \circ \alpha))'(0) \\
	&= D (\theta \circ g \circ \psi ^{-1}) \circ D( \psi \circ  f \circ \phi ^{-1}) \circ ( \phi \circ \alpha)'(0) \\
	&= d_{f(p)} g \circ d_pf ( [ \alpha]) 
\end{align*}
\end{problem}

\begin{problem}[5.10]
Let $ f: M \to N$ be a diffeomorphism. That is, given $ p \in M$ and any charts $ (U,\phi)$ of $ \rr^{m}$ containing $ p$ and $ (V, \psi)$ of $ \rr^{n}$ containing $ f(p)$, the transition map $ \psi \circ f \circ \phi ^{-1}: \rr^{m} \to \rr^{n}$ (restricted to well-defined domain) is smooth and has smooth inverse. This forces the Jacobian of the transition map to be a linear isomorphism by the inverse function theorem. Then
\begin{align*}
	d_pf ([ \alpha]) &= (\psi \circ f \circ \circ \alpha)'(0) \\
	&= ((\psi \circ f \circ \phi ^{-1} )\circ (\phi \circ \alpha))'(0) \\
	&= D(\psi \circ f \circ \phi ^{-1}) \circ (\phi \circ \alpha)'(0) \\
	&= D(\psi \circ f \circ \phi ^{-1})[ \alpha] .
\end{align*}
So we see that $ d_pf$ can be identified with the Jacobian of the transition map which must be a linear isomorphism. Since the domain and codomain of $ d_pf$ can be identified with  $ \rr^{m}$ and $ \rr^{n}$, it follows that $ \rr^{m} = \rr^{n}$ so $ m=n$. That is,  $ \dim M = \dim N$.
\end{problem}
\end{document}


