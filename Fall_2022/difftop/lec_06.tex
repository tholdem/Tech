\documentclass[12pt,class=article,crop=false]{standalone} 
\newcommand{\alert}[1]{{\bf \color{red} [Alert:] #1}}
\newcommand{\todo}[1]{{\bf \color{orange} [TODO:] #1}}
\newcommand{\real}[1][]{\mathbb{R}^{#1}}
\newcommand{\myeqn}[1]{(\ref{#1})}
\newcommand{\myex}[1]{Example \ref{#1}}
\newcommand{\defeq}{\stackrel{\mathrm{def}}{=}}
\newcommand{\parder}[2]{\frac{\partial #1}{\partial #2}}
\newcommand{\Lie}[3][]{\mathsf{L}_{#3}^{#1} #2}
\newcommand{\LieA}[1]{\mathsf{Lie}(#1)}
\newcommand{\lieder}[2]{\mathcal{L}_{#2} #1}
\renewcommand{\t}{^{\mbox{\tiny\sf T}}}
\newcommand{\trans}{^{\mbox{\tiny\sf T}}}
\newcommand{\markup}[1]{\{\textbf{#1}\}}
\newcommand{\msub}[1]{_\mathrm{#1}}
\newcommand{\msup}[1]{^\mathrm{#1}}
\newcommand{\inv}[1]{#1^{-1}}
\newcommand{\pinv}[1]{{#1}^{+}}
\newcommand{\myfracA}[2]{\displaystyle{\frac{#1}{#2}}}
\newcommand{\myfracB}[2]{{#1}/{#2}}
\newcommand{\mydiffA}[1]{\dot{#1}}
\newcommand{\mydiffB}[2]{\myfracA{\mathrm{d}{#1}}{\mathrm{d}{#2}}}
\newcommand{\ball}[2]{\mathcal{B}_{#1}\left(#2\right)}
\newcommand{\acos}[1]{\cos^{-1}\left(#1\right)}
\newcommand{\asin}[1]{\sin^{-1}\left(#1\right)}
\newcommand{\mani}{\mathcal{M}}
\newcommand{\tang}[2]{\mathsf{T}_{#1} #2}
\newcommand{\LieB}[2]{[ #1, #2 ]}
\newcommand{\LieBad}[3][]{\mathsf{ad}_{#2}^{#1} #3}
\newcommand{\ReachVT}{\mathcal{R}^V_T}
\newcommand{\ReachVt}{\mathcal{R}^V_t}
\newcommand{\ReachVTe}{\mathcal{R}^V_{\le T}}
\newcommand{\ReachT}{\mathcal{R}_T}
\newcommand{\Reacht}{\mathcal{R}_t}
\newcommand{\ReachTe}{\mathcal{R}_{\le T}}
\newcommand{\accLA}[1]{\mathsf{Lie}(#1)}
\newcommand{\accD}{\Delta_{\mathcal{F}}}
\newcommand{\accSA}{\mathsf{Lie}(\mathcal{G},f)}
\newcommand{\accDS}{\Delta_{\mathcal{G}}}
\newcommand{\eval}[3]{\mathsf{Ev}^{#2}_{#1}\left( #3 \right)}
\newcommand{\stlc}{\textsc{stlc}}
\newcommand{\clf}{\textsc{clf}}
\newcommand{\jqlf}{\textsc{jqlf}}
\newcommand{\dlas}{\textsc{dlas}}
\newcommand{\Ad}[2]{\mathsf{Ad}_{#1} #2}
\newcommand{\xe}{\ensuremath{x_e}}
\newcommand{\lebg}[1]{\mathcal{L}_{#1}}
\newcommand{\lebgx}[1]{\mathcal{L}_{#1 \mathrm{e}}}
\newcommand{\dom}{D}
\newcommand{\domT}{[t_0,\infty) \times D}
\newcommand{\rarrow}{\rightarrow}
\renewcommand{\d}{\mathrm{d}}
\renewcommand{\Re}{\mathbb{R}}
\newcommand{\C}{\mathrm{C}}

\newcommand{\QED}{{\unskip\nobreak\hfil\penalty50\hskip2em\vadjust{}
		\nobreak\hfil$\Box$\parfillskip=0pt\finalhyphendemerits=0\par}\vspace{0.1cm}}
\newcommand{\eoEx}{{\unskip\nobreak\hfil\penalty50\hskip0em\vadjust{}
		\nobreak\hfil$\Large\Diamond$\parfillskip=0pt\finalhyphendemerits=0\par}\vspace{0.1cm}}

\newcommand{\sgn}{\ensuremath{\operatorname{sgn}}}
\newcommand{\sat}{\ensuremath{\operatorname{sat}}}

\newcommand{\half}{\frac{1}{2}}
\newcommand{\shalf}{\mbox{$\frac{1}{2}$}}
\newcommand{\marcom}[1]{\marginpar{\footnotesize #1}}
\newcommand{\der}{\mathrm{D}}
\newcommand{\e}{\mathrm{e}}
\newcommand{\dt}{\mathrm{d}t}

\newcommand{\cA}{\ensuremath{\mathcal{A}}}
\newcommand{\cB}{\ensuremath{\mathcal{B}}}
\newcommand{\cG}{\ensuremath{\mathcal{G}}}
\newcommand{\cK}{\ensuremath{\mathcal{K}}}
\newcommand{\cW}{\ensuremath{\mathcal{W}}}
\newcommand{\cZ}{\ensuremath{\mathcal{Z}}}
\newcommand{\cS}{\ensuremath{\mathcal{S}}}
\newcommand{\cD}{\ensuremath{\mathcal{D}}}
\newcommand{\cP}{\ensuremath{\mathcal{P}}}
\newcommand{\cV}{\ensuremath{\mathcal{V}}}
\newcommand{\cL}{\ensuremath{\mathcal{L}}}
\newcommand{\cN}{\ensuremath{\mathcal{N}}}
\newcommand{\cI}{\ensuremath{\mathcal{I}}}
\newcommand{\cR}{\ensuremath{\mathcal{R}}}
\newcommand{\cM}{\ensuremath{\mathcal{M}}}
\newcommand{\cC}{\ensuremath{\mathcal{C}}}
\newcommand{\cF}{\ensuremath{\mathcal{F}}}
\newcommand{\cH}{\ensuremath{\mathcal{H}}}
\newcommand{\cO}{\ensuremath{\mathcal{O}}}
\newcommand{\cX}{\ensuremath{\mathcal{X}}}
\newcommand{\cY}{\ensuremath{\mathcal{Y}}}
\newcommand{\Ci}{\ensuremath{\mathcal{C}^\infty}}
\newcommand{\ISS}{\textsc{iss}}
\newcommand{\LISS}{\textsc{liss}}
\newcommand{\GAS}{\textsc{gas}}
\newcommand{\GS}{\textsc{gs}}
\newcommand{\LES}{\textsc{les}}
\newcommand{\GUAS}{\textsc{guas}}
\newcommand{\BIBO}{\textsc{bibo}}
\newcommand{\spec}{\ensuremath{\operatorname{spec}}}
\newcommand{\spn}{\ensuremath{\operatorname{span}}}
\renewcommand{\i}{\mathrm{i\,}}

\renewcommand{\implies}{\Rightarrow}

\renewcommand{\theenumi}{$\roman{enumi})$}
\renewcommand{\labelenumi}{\theenumi}

\font\ptmten=zptmcmrm scaled 1200
\newcommand{\w}{\mbox{{\ptmten w}}}
\newcommand{\z}{\mbox{{\ptmten z}}}
\renewcommand{\Re}{\mathbb{R}}

\newcommand{\cl}{\operatorname{cl}}
\newcommand{\intr}{\operatorname{int}}
\newcommand{\rank}{\operatorname{rank}}
\newcommand{\co}{\operatorname{co}}
\newcommand{\aff}{\operatorname{aff}}

\theoremstyle{plain}
\newtheorem{theorem}{Theorem}[chapter]
\newtheorem{claim}[theorem]{Claim}
\newtheorem{corollary}[theorem]{Corollary}
\newtheorem{prop}[theorem]{Proposition}
\newtheorem{fact}[theorem]{Fact}
\newtheorem{lemma}[theorem]{Lemma}

\newtheorem{remark}{Remark}[chapter]

\theoremstyle{definition}
\newtheorem{assume}[theorem]{Assumption}
\newtheorem{defn}[theorem]{Definition}
\newtheorem{problem}[theorem]{Problem}
\newtheorem{exercise}{Exercise}
\newtheorem{example}[theorem]{Example}


\begin{document}
\section{Sard's Theorem}
\begin{eg}
Consider $ T: P \to S^{1}$, take $ p \in S^{1}$, by Sard's Theorem and rank theorem, $ T^{-1}(q)$ is finite points.
\end{eg}
\begin{lem}[Fubini for measure zero]
Let $ A \subseteq \rr^{n}$, closed, $ \rr^{n} = \rr \times \rr^{n-1}$, $ S_x := \{x\} \times \rr^{n-1}, x \in \rr $. If $ \mu(A \cap S_x) =0 \ \forall \ x \in \rr$, then $ \mu(A) = 0$.
\end{lem}
\begin{proof}
We may assume $ A$ is compact (by countable). If  $ A \cap S_x$ lies in an open set $ U \subseteq S_x$, denote $ S_{(x- \epsilon,x+ \epsilon)}:= \bigcup_{ x' \in (x- \epsilon, x+\epsilon) S_{x'}}^{ n}$ (thickened slice), then $ A \cap S_{x- \epsilon,x+ \epsilon} \subseteq U \times (x- \epsilon, x+ \epsilon)$. Then
\begin{align*}
	A \subseteq \bigcup_{ i= 1}^{ n} U_i \times (x_i - \epsilon_i,x_i+ \epsilon_i) = : N.
\end{align*}
We may also assume that $ \sum_{ i= 1}^{ n} 2 \epsilon_i \leq 2(b-a)$. Then by subadditivity, $ $
\begin{align*}
	\mu(N) &\leq \sum_{ i= 1}^{ n} \mu(U_i \times (x_i - \epsilon_i, x_i+ \epsilon_i))\\ 
	       &\leq \sum_{ i= 1}^{ n} \mu(U_i) \cdot 2 \epsilon_i\\
	       &= \sum_{ i= 1}^{ n} \delta 2 \epsilon_i \\
	       &= 2 \delta (b-a) \to 0 
	\item 
\end{align*}
\end{proof}
\begin{proof}
Baby case: $ f: M \to N$, $ \dim M \leq \dim  N$. The case when $ \dim M = \dim N$ is implicitly shown in Lecture 8 (lemma 4). The rest is shown in Lemma 7.

May assume that $ M = \rr^{n}$ and $ N = \rr^{p}$ because we are proving a local property and countable union of measure zero sets is measure zero. When $ n=0$, it has measure zero. Assume theorem holds for $ n-1$. $ f: \rr^{n} \to \rr^{p}$. Let $ C$ be the set of critical points in  $ \rr^{n}$. Define $ C_i := \{x \in U: \text{ all partial derivatives up to order }k =0  \} $. Then $ C \supseteq C_1 \supseteq C_2 \supseteq \ldots$. Then it suffices to show that
\begin{enumerate}[label=(\arabic*)]
	\item $ \mu(f(C-C_1)) = 0$
	\item $ \mu(f(C_k - C_{k+1})) = 0$ 
	\item $ \mu(f(C_k)) = 0$ for some large $ k$.
\end{enumerate}
For 1, it suffices to show that there exists an open neighborhood $ V$ of  $ x \in \rr^{n}$ s.t.\ $ \mu(f(V \cap C)) = 0$ by countable basis of $ \rr^{n}$. Suppose $ x \not\in  C_1$, then WLOG assume $ \frac{ \partial f^{1}}{ \partial x_1 }  \neq 0$. Define $ h : U \subseteq \rr^{n} \to \rr^{n}$ by $ h(x) = (f^{1}(x),x_2,\ldots,x_n)$ so $ \rank dh_x = n$ since the Jacobian is
 \begin{align*}
	dh_x = \begin{pmatrix} \frac{ \partial f^{1}}{ \partial x_1 } & 0\\ 0& I \end{pmatrix} 
\end{align*}

So $ h$ is a local diffeomorphism by IFT. Let $ g : = f \circ  h^{-1}$. I claim that locally $ g$ and  $ f$ share the same critical points. Moreover, $ g$ fixes the first coordinate of any point by definition of  $ h$ and  $ g$. Thus  $ t \times \rr^{n-1} \xrightarrow{ g} t \times \rr^{p-1} $. Define $ g^{t} = g|_{t \times \rr^{n-1}}$. Then
\begin{align*}
	dg = \begin{pmatrix} 1&0\\ 0& dg^{t}\end{pmatrix} 
\end{align*}
Then critical points of $ g^{t}$ are also critical points of $ g$ since the matrix has 0 determinant iff  $ \det dg^{t} =0$. Now we've reduced the dimension and can apply induction hypothesis. So the critical points of $ g$ has measure zero and thus same goes for $ f$.

2 is very similar.


\end{proof}

\begin{eg}
Application: $ S^{n}$ is simply connected for $ n \geq 2$. Any curve will miss a point by Sard's.
\end{eg}
\begin{eg}
	Let $ M$ be a smooth closed (compact, connected, without boundary) hypersurface in  $ \rr^{n}$ so $ \dim M = n-1$. Then we have the Gauss map  $ \nu: M to S^{n-1}$, where we map each point to its outward unit normal vector. Then $ \# \nu^{-1}(u)< \infty$ for almost every $ u \in S^{n-1}$. Since regular point would have codimension 0 and surface is compact. This implies that for almost every $ u \in S^{n-1}$, there exists finitely many tangent hyperplanes $ H$ of  $ M$ that are orthogonal to  $ u$. Any hyperplane $ H \subseteq \rr^{n}$ will be transversal to $ M$, after a perturbation. The set of transversal hyperplanes to $ M$ is open and dense. This is a metric space with distance of unit normal vectors and the offset from origin as the metric. It is diffeomorphic to  $ S^{n-1} \times [0, \infty)$.
\end{eg}
\end{document}
