\documentclass[12pt]{article}
%Fall 2022
% Some basic packages
\usepackage{standalone}[subpreambles=true]
\usepackage[utf8]{inputenc}
\usepackage[T1]{fontenc}
\usepackage{textcomp}
\usepackage[english]{babel}
\usepackage{url}
\usepackage{graphicx}
%\usepackage{quiver}
\usepackage{float}
\usepackage{enumitem}
\usepackage{lmodern}
\usepackage{comment}
\usepackage{hyperref}
\usepackage[usenames,svgnames,dvipsnames]{xcolor}
\usepackage[margin=1in]{geometry}
\usepackage{pdfpages}

\pdfminorversion=7

% Don't indent paragraphs, leave some space between them
\usepackage{parskip}

% Hide page number when page is empty
\usepackage{emptypage}
\usepackage{subcaption}
\usepackage{multicol}
\usepackage[b]{esvect}

% Math stuff
\usepackage{amsmath, amsfonts, mathtools, amsthm, amssymb}
\usepackage{bbm}
\usepackage{stmaryrd}
\allowdisplaybreaks

% Fancy script capitals
\usepackage{mathrsfs}
\usepackage{cancel}
% Bold math
\usepackage{bm}
% Some shortcuts
\newcommand{\rr}{\ensuremath{\mathbb{R}}}
\newcommand{\zz}{\ensuremath{\mathbb{Z}}}
\newcommand{\qq}{\ensuremath{\mathbb{Q}}}
\newcommand{\nn}{\ensuremath{\mathbb{N}}}
\newcommand{\ff}{\ensuremath{\mathbb{F}}}
\newcommand{\cc}{\ensuremath{\mathbb{C}}}
\newcommand{\ee}{\ensuremath{\mathbb{E}}}
\newcommand{\hh}{\ensuremath{\mathbb{H}}}
\renewcommand\O{\ensuremath{\emptyset}}
\newcommand{\norm}[1]{{\left\lVert{#1}\right\rVert}}
\newcommand{\dbracket}[1]{{\left\llbracket{#1}\right\rrbracket}}
\newcommand{\ve}[1]{{\bm{#1}}}
\newcommand\allbold[1]{{\boldmath\textbf{#1}}}
\DeclareMathOperator{\lcm}{lcm}
\DeclareMathOperator{\im}{im}
\DeclareMathOperator{\coim}{coim}
\DeclareMathOperator{\dom}{dom}
\DeclareMathOperator{\tr}{tr}
\DeclareMathOperator{\rank}{rank}
\DeclareMathOperator*{\var}{Var}
\DeclareMathOperator*{\ev}{E}
\DeclareMathOperator{\dg}{deg}
\DeclareMathOperator{\aff}{aff}
\DeclareMathOperator{\conv}{conv}
\DeclareMathOperator{\inte}{int}
\DeclareMathOperator*{\argmin}{argmin}
\DeclareMathOperator*{\argmax}{argmax}
\DeclareMathOperator{\graph}{graph}
\DeclareMathOperator{\sgn}{sgn}
\DeclareMathOperator*{\Rep}{Rep}
\DeclareMathOperator{\Proj}{Proj}
\DeclareMathOperator{\mat}{mat}
\DeclareMathOperator{\diag}{diag}
\DeclareMathOperator{\aut}{Aut}
\DeclareMathOperator{\gal}{Gal}
\DeclareMathOperator{\inn}{Inn}
\DeclareMathOperator{\edm}{End}
\DeclareMathOperator{\Hom}{Hom}
\DeclareMathOperator{\ext}{Ext}
\DeclareMathOperator{\tor}{Tor}
\DeclareMathOperator{\Span}{Span}
\DeclareMathOperator{\Stab}{Stab}
\DeclareMathOperator{\cont}{cont}
\DeclareMathOperator{\Ann}{Ann}
\DeclareMathOperator{\Div}{div}
\DeclareMathOperator{\curl}{curl}
\DeclareMathOperator{\nat}{Nat}
\DeclareMathOperator{\gr}{Gr}
\DeclareMathOperator{\vect}{Vect}
\DeclareMathOperator{\id}{id}
\DeclareMathOperator{\Mod}{Mod}
\DeclareMathOperator{\sign}{sign}
\DeclareMathOperator{\Surf}{Surf}
\DeclareMathOperator{\fcone}{fcone}
\DeclareMathOperator{\Rot}{Rot}
\DeclareMathOperator{\grad}{grad}
\DeclareMathOperator{\atan2}{atan2}
\DeclareMathOperator{\Ric}{Ric}
\let\vec\relax
\DeclareMathOperator{\vec}{vec}
\let\Re\relax
\DeclareMathOperator{\Re}{Re}
\let\Im\relax
\DeclareMathOperator{\Im}{Im}
% Put x \to \infty below \lim
\let\svlim\lim\def\lim{\svlim\limits}

%wide hat
\usepackage{scalerel,stackengine}
\stackMath
\newcommand*\wh[1]{%
\savestack{\tmpbox}{\stretchto{%
  \scaleto{%
    \scalerel*[\widthof{\ensuremath{#1}}]{\kern-.6pt\bigwedge\kern-.6pt}%
    {\rule[-\textheight/2]{1ex}{\textheight}}%WIDTH-LIMITED BIG WEDGE
  }{\textheight}% 
}{0.5ex}}%
\stackon[1pt]{#1}{\tmpbox}%
}
\parskip 1ex

%Make implies and impliedby shorter
\let\implies\Rightarrow
\let\impliedby\Leftarrow
\let\iff\Leftrightarrow
\let\epsilon\varepsilon

% Add \contra symbol to denote contradiction
\usepackage{stmaryrd} % for \lightning
\newcommand\contra{\scalebox{1.5}{$\lightning$}}

% \let\phi\varphi

% Command for short corrections
% Usage: 1+1=\correct{3}{2}

\definecolor{correct}{HTML}{009900}
\newcommand\correct[2]{\ensuremath{\:}{\color{red}{#1}}\ensuremath{\to }{\color{correct}{#2}}\ensuremath{\:}}
\newcommand\green[1]{{\color{correct}{#1}}}

% horizontal rule
\newcommand\hr{
    \noindent\rule[0.5ex]{\linewidth}{0.5pt}
}

% hide parts
\newcommand\hide[1]{}

% si unitx
\usepackage{siunitx}
\sisetup{locale = FR}

%allows pmatrix to stretch
\makeatletter
\renewcommand*\env@matrix[1][\arraystretch]{%
  \edef\arraystretch{#1}%
  \hskip -\arraycolsep
  \let\@ifnextchar\new@ifnextchar
  \array{*\c@MaxMatrixCols c}}
\makeatother

\renewcommand{\arraystretch}{0.8}

\renewcommand{\baselinestretch}{1.5}

\usepackage{graphics}
\usepackage{epstopdf}

\RequirePackage{hyperref}
%%
%% Add support for color in order to color the hyperlinks.
%% 
\hypersetup{
  colorlinks = true,
  urlcolor = blue,
  citecolor = blue
}
%%fakesection Links
\hypersetup{
    colorlinks,
    linkcolor={red!50!black},
    citecolor={green!50!black},
    urlcolor={blue!80!black}
}
%customization of cleveref
\RequirePackage[capitalize,nameinlink]{cleveref}[0.19]

% Per SIAM Style Manual, "section" should be lowercase
\crefname{section}{section}{sections}
\crefname{subsection}{subsection}{subsections}
\Crefname{section}{Section}{Sections}
\Crefname{subsection}{Subsection}{Subsections}

% Per SIAM Style Manual, "Figure" should be spelled out in references
\Crefname{figure}{Figure}{Figures}

% Per SIAM Style Manual, don't say equation in front on an equation.
\crefformat{equation}{\textup{#2(#1)#3}}
\crefrangeformat{equation}{\textup{#3(#1)#4--#5(#2)#6}}
\crefmultiformat{equation}{\textup{#2(#1)#3}}{ and \textup{#2(#1)#3}}
{, \textup{#2(#1)#3}}{, and \textup{#2(#1)#3}}
\crefrangemultiformat{equation}{\textup{#3(#1)#4--#5(#2)#6}}%
{ and \textup{#3(#1)#4--#5(#2)#6}}{, \textup{#3(#1)#4--#5(#2)#6}}{, and \textup{#3(#1)#4--#5(#2)#6}}

% But spell it out at the beginning of a sentence.
\Crefformat{equation}{#2Equation~\textup{(#1)}#3}
\Crefrangeformat{equation}{Equations~\textup{#3(#1)#4--#5(#2)#6}}
\Crefmultiformat{equation}{Equations~\textup{#2(#1)#3}}{ and \textup{#2(#1)#3}}
{, \textup{#2(#1)#3}}{, and \textup{#2(#1)#3}}
\Crefrangemultiformat{equation}{Equations~\textup{#3(#1)#4--#5(#2)#6}}%
{ and \textup{#3(#1)#4--#5(#2)#6}}{, \textup{#3(#1)#4--#5(#2)#6}}{, and \textup{#3(#1)#4--#5(#2)#6}}

% Make number non-italic in any environment.
\crefdefaultlabelformat{#2\textup{#1}#3}

% Environments
\makeatother
% For box around Definition, Theorem, \ldots
%%fakesection Theorems
\usepackage{thmtools}
\usepackage[framemethod=TikZ]{mdframed}

\theoremstyle{definition}
\mdfdefinestyle{mdbluebox}{%
	roundcorner = 10pt,
	linewidth=1pt,
	skipabove=12pt,
	innerbottommargin=9pt,
	skipbelow=2pt,
	nobreak=true,
	linecolor=blue,
	backgroundcolor=TealBlue!5,
}
\declaretheoremstyle[
	headfont=\sffamily\bfseries\color{MidnightBlue},
	mdframed={style=mdbluebox},
	headpunct={\\[3pt]},
	postheadspace={0pt}
]{thmbluebox}

\mdfdefinestyle{mdredbox}{%
	linewidth=0.5pt,
	skipabove=12pt,
	frametitleaboveskip=5pt,
	frametitlebelowskip=0pt,
	skipbelow=2pt,
	frametitlefont=\bfseries,
	innertopmargin=4pt,
	innerbottommargin=8pt,
	nobreak=false,
	linecolor=RawSienna,
	backgroundcolor=Salmon!5,
}
\declaretheoremstyle[
	headfont=\bfseries\color{RawSienna},
	mdframed={style=mdredbox},
	headpunct={\\[3pt]},
	postheadspace={0pt},
]{thmredbox}

\declaretheorem[%
style=thmbluebox,name=Theorem,numberwithin=section]{thm}
\declaretheorem[style=thmbluebox,name=Lemma,sibling=thm]{lem}
\declaretheorem[style=thmbluebox,name=Proposition,sibling=thm]{prop}
\declaretheorem[style=thmbluebox,name=Corollary,sibling=thm]{coro}
\declaretheorem[style=thmredbox,name=Example,sibling=thm]{eg}

\mdfdefinestyle{mdgreenbox}{%
	roundcorner = 10pt,
	linewidth=1pt,
	skipabove=12pt,
	innerbottommargin=9pt,
	skipbelow=2pt,
	nobreak=true,
	linecolor=ForestGreen,
	backgroundcolor=ForestGreen!5,
}

\declaretheoremstyle[
	headfont=\bfseries\sffamily\color{ForestGreen!70!black},
	bodyfont=\normalfont,
	spaceabove=2pt,
	spacebelow=1pt,
	mdframed={style=mdgreenbox},
	headpunct={ --- },
]{thmgreenbox}

\declaretheorem[style=thmgreenbox,name=Definition,sibling=thm]{defn}

\mdfdefinestyle{mdgreenboxsq}{%
	linewidth=1pt,
	skipabove=12pt,
	innerbottommargin=9pt,
	skipbelow=2pt,
	nobreak=true,
	linecolor=ForestGreen,
	backgroundcolor=ForestGreen!5,
}
\declaretheoremstyle[
	headfont=\bfseries\sffamily\color{ForestGreen!70!black},
	bodyfont=\normalfont,
	spaceabove=2pt,
	spacebelow=1pt,
	mdframed={style=mdgreenboxsq},
	headpunct={},
]{thmgreenboxsq}
\declaretheoremstyle[
	headfont=\bfseries\sffamily\color{ForestGreen!70!black},
	bodyfont=\normalfont,
	spaceabove=2pt,
	spacebelow=1pt,
	mdframed={style=mdgreenboxsq},
	headpunct={},
]{thmgreenboxsq*}

\mdfdefinestyle{mdblackbox}{%
	skipabove=8pt,
	linewidth=3pt,
	rightline=false,
	leftline=true,
	topline=false,
	bottomline=false,
	linecolor=black,
	backgroundcolor=RedViolet!5!gray!5,
}
\declaretheoremstyle[
	headfont=\bfseries,
	bodyfont=\normalfont\small,
	spaceabove=0pt,
	spacebelow=0pt,
	mdframed={style=mdblackbox}
]{thmblackbox}

\theoremstyle{plain}
\declaretheorem[name=Question,sibling=thm,style=thmblackbox]{ques}
\declaretheorem[name=Remark,sibling=thm,style=thmgreenboxsq]{remark}
\declaretheorem[name=Remark,sibling=thm,style=thmgreenboxsq*]{remark*}
\newtheorem{ass}[thm]{Assumptions}

\theoremstyle{definition}
\newtheorem*{problem}{Problem}
\newtheorem{claim}[thm]{Claim}
\theoremstyle{remark}
\newtheorem*{case}{Case}
\newtheorem*{notation}{Notation}
\newtheorem*{note}{Note}
\newtheorem*{motivation}{Motivation}
\newtheorem*{intuition}{Intuition}
\newtheorem*{conjecture}{Conjecture}

% Make section starts with 1 for report type
%\renewcommand\thesection{\arabic{section}}

% End example and intermezzo environments with a small diamond (just like proof
% environments end with a small square)
\usepackage{etoolbox}
\AtEndEnvironment{vb}{\null\hfill$\diamond$}%
\AtEndEnvironment{intermezzo}{\null\hfill$\diamond$}%
% \AtEndEnvironment{opmerking}{\null\hfill$\diamond$}%

% Fix some spacing
% http://tex.stackexchange.com/questions/22119/how-can-i-change-the-spacing-before-theorems-with-amsthm
\makeatletter
\def\thm@space@setup{%
  \thm@preskip=\parskip \thm@postskip=0pt
}

% Fix some stuff
% %http://tex.stackexchange.com/questions/76273/multiple-pdfs-with-page-group-included-in-a-single-page-warning
\pdfsuppresswarningpagegroup=1


% My name
\author{Jaden Wang}



\begin{document}
\centerline {\textsf{\textbf{\LARGE{Homework 4}}}}
\centerline {Jaden Wang}
\vspace{.15in}
\begin{problem}[1]
$ f$ is 1-1 continuous so it is an immersion and is bijective onto its image $ f(X)$. By Theorem 26.6 of Munkres, $ f$ is a homeomorphism onto $ f(X)$ so it is an embedding.
\end{problem}
\begin{problem}[4]
	Since $ f$ is a product map of polynomials, it is continuous. Let $ \widetilde{ f}$ denotes the map with domain restricted to the unit sphere in  $ \rr^3$ and codomain restricted to its image, which remains continuous. Suppose $ (x_1y_1,y_1z_1,x_1z_1,x_1^2+2y_1^2+3z_1^2) =(x_2y_2,y_2z_2,x_2z_2,x_2^2+2y_2^2+3z_2^2)$. In the case when none of the entries is zero, since $ x_1 = \frac{x_2y_2}{y_1 } = \frac{x_2z_2}{ z_1}$, combining with $ y_1 z_1 = y_2 z_2$  we obtain that $ z_1^2 = z_2^2$ so $ z_1 = \pm z_2$. Then $ y_1 = \pm y_2$ and $ x_1 = \pm x_2$ (the signs must be all + or all -). These solutions are consistent with the 4th entry so they are all valid. Whenever there is at least one entry that is zero, WLOG assume $ x_1 = 0$, this forces that either $ x_2$ or $ y_2$ to be zero and either $ x_2$ or $ z_2$ to be zero. Suppose $ x_2 \neq 0$, then $ y_2 = z_2 =0$, which forces that either $ y_1$ or $ z_1$ to be zero. WLOG suppose $ y_1=0$, then by the relation that $ x^2+y^2+z^2=1$, we obtain $ z_1=x_2=1$. Thus in the 4th entry we have
\begin{align*}
	0^2+2 \cdot 0^2+ 3 z_1^2 &= x_2^2 + 2 \cdot 0^2 + 3 \cdot 0^2\\
	3 &= 1,
\end{align*}
a contradiction. Thus $ x_1=0$ forces $ x_2 = 0 = \pm x_1$. 

	Taken both cases together, for any $ q \in \widetilde{ f}(S^2)$, $ \widetilde{ f}^{-1}(q) = \{ \pm p\}$ for some $ p$ in  $ S^2$. That is, if we quotient $ S^2$ by the antipodal action of $ \zz_2$, we obtain a map $ g: S^2 / \sim \to \widetilde{ f}(S^2)$, which again is continuous by the universal property of quotient topology. But $ \rr P^2 = S^2 / \sim$ so we have found a well-defined continuous map from $ \rr P^2$ into $ \rr^{4}$. It is clearly injective and surjective onto its image per the argument above, so it remains to show that $ g$ is a homeomorphism. By Corollary 22.3 of Munkres, it suffices to show that $ \widetilde{ f}$ is a quotient map. By Exercise 22.2(b), it suffices to show that there exists a continuous map $ h:\widetilde{ f}(S^2) \to S^2$ s.t.\ $ \widetilde{ f} \circ h = \text{id}_{ \widetilde{ f}(S^2)} $.  Given $ (a,b,c,d) \in \widetilde{ f}(S^2)$, we want to recover $ x,y,z$ in terms of  $ a,b,c,d$. Define  $ A=\sqrt{d-1+2\sqrt{2} b} $ and $ B = \sqrt{3-d+2\sqrt{2}a } $. Then we can check that
	\begin{align*}
x &=\frac{\sqrt{2}(d-2) }{2(A-B) } - \frac{A-B}{ 2\sqrt{2} } \\
y &= \frac{a}{x} \\
		z &= \frac{A-B}{2\sqrt{2}  }+\frac{\sqrt{2}(d-2) }{2(A-B) }
	\end{align*}
	does the job (using the relation that $ x^2+y^2+z^2=1$). Then the map $ h:(a,b,c,d)\mapsto (x,y,z)$ is clearly continuous as the composition of continuous functions and gives us the identity (no need to worry about dividing by 0 since we know such $ x,y,z$ exist as $ (a,b,c,d)$ is in the image). It follows that $ g$ is a homemomorphism onto its image so it is an embedding.

	Alternatively, we could use the fact that since $ \rr P^2$ is the quotient of a compact space and therefore compact, and $ \rr^{4}$ is Hausdorff so any subspace is Hausdorff, since $ g$ is already bijective continuous, it is a homeomorphism by Munkres Theorem 26.6.

\end{problem}
\begin{problem}[7.16]
I claim that $ 1$ is a regular value of  $ g:TM \to \rr$. Given $ p \in g^{-1}(1)$, $ \rank dg_p = 0$ or $1 $. Suppose $ \rank dg_p = 0$, then it has to send everything to 0, but since $ g_p$ is an inner product, it is positive definite and cannot be constant so the derivative cannot be 0. Hence $ \rank dg_p =1$ and $ 1$ is a regular value of  $ g$. Thus  $ g^{-1}(1)$ is a $ 2n-1$ manifold which is the unit tangent bundle.
\end{problem}
\begin{problem}[8.5]
Suppose $ X \subseteq M$ is measure zero. Then for any $ p \in M$, denote the chart given by the definition $ (U_p, \phi_p)$ s.t.\ $ \phi_p(U_p \cap X)$ has measure zero. Clearly these charts form an atlas for $ M$. By Exercise 8, we can obtain a countable atlas  $ (U_i,\phi_i)$. Given any arbitrary chart $(V,\psi) $ of $ M$, $ V = \bigcup_{ i \in \nn} U_i$. Then $ \psi \circ \phi_i ^{-1}: \phi_i(U_i) \cap \psi(V) \subseteq \rr^{n} \to \rr^{n}$ is smooth. Thus we can apply Lemma 4 and obtain:
\begin{align*}
	 \psi \circ \phi_i ^{-1} (\phi_i(U_i \cap X) \cap \phi_i(U_i \cap V)) &= \psi (U_i \cap X \cap V) 
\end{align*}
has measure zero. Then we see that
\begin{align*}
	\psi(V \cap X) \subseteq \bigcup_{ i \in \nn} \psi(U_i \cap X \cap V) 
\end{align*}
Since countable union of measure zero sets are measure zero, $ \psi(V \cap X)$ is also measure zero by monotonicty.
\end{problem}
\begin{problem}[8.8]
	Given an arbitrary atlas $ \mathcal{ A} = \{(U_{ \alpha}, \phi_{ \alpha}): \alpha \in J\} $ of a manifold $M$, since $ M$ is second-countable, by Theorem 30.3 of Munkres, it is a Lindelof space. Therefore, there exists a countable subcover $ \mathcal{ A'}$ of $ \mathcal{ A}$ that serves as a countable atlas.
\end{problem}
\begin{problem}[8.11]
$ (\implies)$: The derivative of $ \pi_u(x)$ at $ p$ is the unique linear operator $ d_{\pi_u}: T_p M \to \rr^{n-1}$ satisfying the following
\begin{align*}
	\lim_{ t \to 0} \frac{\pi_u(x+tv) -(\pi_u(x)+ td_{\pi_u}(v))}{ t} = 0
\end{align*}
where $ v \in T_pM \cong \rr^{n} \subseteq \rr^{N}$. Notice
\begin{align*}
	\frac{1}{t}(x+tv - \langle x+tv,u \rangle u - x - \langle x,u \rangle u ) &= \frac{x}{t}+v- \langle \frac{x}{t},u \rangle u - \langle v,u \rangle u - \frac{x}{t} -\langle \frac{x}{t},u \rangle u \\
	&= v- \langle v,u \rangle u 
\end{align*}
Alternatively, projection is a linear operator so its derivative is just itself. Therefore, it's easy to see that $ d_{\pi_u}(v) = v- \langle v,u \rangle u$. Now we have $ u \not\in T_pM$ iff $ u$ and  $ v$ are linearly independent iff $ t u -v \neq 0 , v \neq 0, \ \forall \ t \in \rr$ iff $ d_{\pi_u}$ doesn't send any nonzero vector in $ T_pM$ to 0 iff $ \ker d_{\pi_u}$ iff it is not singular at $ p$ onto its image.
$ (\impliedby):$ 
\end{problem}
\end{document}
