\documentclass[12pt]{article}
\newcommand{\alert}[1]{{\bf \color{red} [Alert:] #1}}
\newcommand{\todo}[1]{{\bf \color{orange} [TODO:] #1}}
\newcommand{\real}[1][]{\mathbb{R}^{#1}}
\newcommand{\myeqn}[1]{(\ref{#1})}
\newcommand{\myex}[1]{Example \ref{#1}}
\newcommand{\defeq}{\stackrel{\mathrm{def}}{=}}
\newcommand{\parder}[2]{\frac{\partial #1}{\partial #2}}
\newcommand{\Lie}[3][]{\mathsf{L}_{#3}^{#1} #2}
\newcommand{\LieA}[1]{\mathsf{Lie}(#1)}
\newcommand{\lieder}[2]{\mathcal{L}_{#2} #1}
\renewcommand{\t}{^{\mbox{\tiny\sf T}}}
\newcommand{\trans}{^{\mbox{\tiny\sf T}}}
\newcommand{\markup}[1]{\{\textbf{#1}\}}
\newcommand{\msub}[1]{_\mathrm{#1}}
\newcommand{\msup}[1]{^\mathrm{#1}}
\newcommand{\inv}[1]{#1^{-1}}
\newcommand{\pinv}[1]{{#1}^{+}}
\newcommand{\myfracA}[2]{\displaystyle{\frac{#1}{#2}}}
\newcommand{\myfracB}[2]{{#1}/{#2}}
\newcommand{\mydiffA}[1]{\dot{#1}}
\newcommand{\mydiffB}[2]{\myfracA{\mathrm{d}{#1}}{\mathrm{d}{#2}}}
\newcommand{\ball}[2]{\mathcal{B}_{#1}\left(#2\right)}
\newcommand{\acos}[1]{\cos^{-1}\left(#1\right)}
\newcommand{\asin}[1]{\sin^{-1}\left(#1\right)}
\newcommand{\mani}{\mathcal{M}}
\newcommand{\tang}[2]{\mathsf{T}_{#1} #2}
\newcommand{\LieB}[2]{[ #1, #2 ]}
\newcommand{\LieBad}[3][]{\mathsf{ad}_{#2}^{#1} #3}
\newcommand{\ReachVT}{\mathcal{R}^V_T}
\newcommand{\ReachVt}{\mathcal{R}^V_t}
\newcommand{\ReachVTe}{\mathcal{R}^V_{\le T}}
\newcommand{\ReachT}{\mathcal{R}_T}
\newcommand{\Reacht}{\mathcal{R}_t}
\newcommand{\ReachTe}{\mathcal{R}_{\le T}}
\newcommand{\accLA}[1]{\mathsf{Lie}(#1)}
\newcommand{\accD}{\Delta_{\mathcal{F}}}
\newcommand{\accSA}{\mathsf{Lie}(\mathcal{G},f)}
\newcommand{\accDS}{\Delta_{\mathcal{G}}}
\newcommand{\eval}[3]{\mathsf{Ev}^{#2}_{#1}\left( #3 \right)}
\newcommand{\stlc}{\textsc{stlc}}
\newcommand{\clf}{\textsc{clf}}
\newcommand{\jqlf}{\textsc{jqlf}}
\newcommand{\dlas}{\textsc{dlas}}
\newcommand{\Ad}[2]{\mathsf{Ad}_{#1} #2}
\newcommand{\xe}{\ensuremath{x_e}}
\newcommand{\lebg}[1]{\mathcal{L}_{#1}}
\newcommand{\lebgx}[1]{\mathcal{L}_{#1 \mathrm{e}}}
\newcommand{\dom}{D}
\newcommand{\domT}{[t_0,\infty) \times D}
\newcommand{\rarrow}{\rightarrow}
\renewcommand{\d}{\mathrm{d}}
\renewcommand{\Re}{\mathbb{R}}
\newcommand{\C}{\mathrm{C}}

\newcommand{\QED}{{\unskip\nobreak\hfil\penalty50\hskip2em\vadjust{}
		\nobreak\hfil$\Box$\parfillskip=0pt\finalhyphendemerits=0\par}\vspace{0.1cm}}
\newcommand{\eoEx}{{\unskip\nobreak\hfil\penalty50\hskip0em\vadjust{}
		\nobreak\hfil$\Large\Diamond$\parfillskip=0pt\finalhyphendemerits=0\par}\vspace{0.1cm}}

\newcommand{\sgn}{\ensuremath{\operatorname{sgn}}}
\newcommand{\sat}{\ensuremath{\operatorname{sat}}}

\newcommand{\half}{\frac{1}{2}}
\newcommand{\shalf}{\mbox{$\frac{1}{2}$}}
\newcommand{\marcom}[1]{\marginpar{\footnotesize #1}}
\newcommand{\der}{\mathrm{D}}
\newcommand{\e}{\mathrm{e}}
\newcommand{\dt}{\mathrm{d}t}

\newcommand{\cA}{\ensuremath{\mathcal{A}}}
\newcommand{\cB}{\ensuremath{\mathcal{B}}}
\newcommand{\cG}{\ensuremath{\mathcal{G}}}
\newcommand{\cK}{\ensuremath{\mathcal{K}}}
\newcommand{\cW}{\ensuremath{\mathcal{W}}}
\newcommand{\cZ}{\ensuremath{\mathcal{Z}}}
\newcommand{\cS}{\ensuremath{\mathcal{S}}}
\newcommand{\cD}{\ensuremath{\mathcal{D}}}
\newcommand{\cP}{\ensuremath{\mathcal{P}}}
\newcommand{\cV}{\ensuremath{\mathcal{V}}}
\newcommand{\cL}{\ensuremath{\mathcal{L}}}
\newcommand{\cN}{\ensuremath{\mathcal{N}}}
\newcommand{\cI}{\ensuremath{\mathcal{I}}}
\newcommand{\cR}{\ensuremath{\mathcal{R}}}
\newcommand{\cM}{\ensuremath{\mathcal{M}}}
\newcommand{\cC}{\ensuremath{\mathcal{C}}}
\newcommand{\cF}{\ensuremath{\mathcal{F}}}
\newcommand{\cH}{\ensuremath{\mathcal{H}}}
\newcommand{\cO}{\ensuremath{\mathcal{O}}}
\newcommand{\cX}{\ensuremath{\mathcal{X}}}
\newcommand{\cY}{\ensuremath{\mathcal{Y}}}
\newcommand{\Ci}{\ensuremath{\mathcal{C}^\infty}}
\newcommand{\ISS}{\textsc{iss}}
\newcommand{\LISS}{\textsc{liss}}
\newcommand{\GAS}{\textsc{gas}}
\newcommand{\GS}{\textsc{gs}}
\newcommand{\LES}{\textsc{les}}
\newcommand{\GUAS}{\textsc{guas}}
\newcommand{\BIBO}{\textsc{bibo}}
\newcommand{\spec}{\ensuremath{\operatorname{spec}}}
\newcommand{\spn}{\ensuremath{\operatorname{span}}}
\renewcommand{\i}{\mathrm{i\,}}

\renewcommand{\implies}{\Rightarrow}

\renewcommand{\theenumi}{$\roman{enumi})$}
\renewcommand{\labelenumi}{\theenumi}

\font\ptmten=zptmcmrm scaled 1200
\newcommand{\w}{\mbox{{\ptmten w}}}
\newcommand{\z}{\mbox{{\ptmten z}}}
\renewcommand{\Re}{\mathbb{R}}

\newcommand{\cl}{\operatorname{cl}}
\newcommand{\intr}{\operatorname{int}}
\newcommand{\rank}{\operatorname{rank}}
\newcommand{\co}{\operatorname{co}}
\newcommand{\aff}{\operatorname{aff}}

\theoremstyle{plain}
\newtheorem{theorem}{Theorem}[chapter]
\newtheorem{claim}[theorem]{Claim}
\newtheorem{corollary}[theorem]{Corollary}
\newtheorem{prop}[theorem]{Proposition}
\newtheorem{fact}[theorem]{Fact}
\newtheorem{lemma}[theorem]{Lemma}

\newtheorem{remark}{Remark}[chapter]

\theoremstyle{definition}
\newtheorem{assume}[theorem]{Assumption}
\newtheorem{defn}[theorem]{Definition}
\newtheorem{problem}[theorem]{Problem}
\newtheorem{exercise}{Exercise}
\newtheorem{example}[theorem]{Example}


\begin{document}
\centerline {\textsf{\textbf{\LARGE{Homework 4}}}}
\centerline {Jaden Wang}
\vspace{.15in}
\begin{problem}[1]
$ f$ is 1-1 continuous so it is an immersion and is bijective onto its image $ f(X)$. By Theorem 26.6 of Munkres, $ f$ is a homeomorphism onto $ f(X)$ so it is an embedding.
\end{problem}
\begin{problem}[4]
	Since $ f$ is a product map of polynomials, it is continuous. Let $ \widetilde{ f}$ denotes the map with domain restricted to the unit sphere in  $ \rr^3$ and codomain restricted to its image, which remains continuous. Suppose $ (x_1y_1,y_1z_1,x_1z_1,x_1^2+2y_1^2+3z_1^2) =(x_2y_2,y_2z_2,x_2z_2,x_2^2+2y_2^2+3z_2^2)$. In the case when none of the entries is zero, since $ x_1 = \frac{x_2y_2}{y_1 } = \frac{x_2z_2}{ z_1}$, combining with $ y_1 z_1 = y_2 z_2$  we obtain that $ z_1^2 = z_2^2$ so $ z_1 = \pm z_2$. Then $ y_1 = \pm y_2$ and $ x_1 = \pm x_2$ (the signs must be all + or all -). These solutions are consistent with the 4th entry so they are all valid. Whenever there is at least one entry that is zero, WLOG assume $ x_1 = 0$, this forces that either $ x_2$ or $ y_2$ to be zero and either $ x_2$ or $ z_2$ to be zero. Suppose $ x_2 \neq 0$, then $ y_2 = z_2 =0$, which forces that either $ y_1$ or $ z_1$ to be zero. WLOG suppose $ y_1=0$, then by the relation that $ x^2+y^2+z^2=1$, we obtain $ z_1=x_2=1$. Thus in the 4th entry we have
\begin{align*}
	0^2+2 \cdot 0^2+ 3 z_1^2 &= x_2^2 + 2 \cdot 0^2 + 3 \cdot 0^2\\
	3 &= 1,
\end{align*}
a contradiction. Thus $ x_1=0$ forces $ x_2 = 0 = \pm x_1$. 

	Taken both cases together, for any $ q \in \widetilde{ f}(S^2)$, $ \widetilde{ f}^{-1}(q) = \{ \pm p\}$ for some $ p$ in  $ S^2$. That is, if we quotient $ S^2$ by the antipodal action of $ \zz_2$, we obtain a map $ g: S^2 / \sim \to \widetilde{ f}(S^2)$, which again is continuous by the universal property of quotient topology. But $ \rr P^2 = S^2 / \sim$ so we have found a well-defined continuous map from $ \rr P^2$ into $ \rr^{4}$. It is clearly injective and surjective onto its image per the argument above, so it remains to show that $ g$ is a homeomorphism. By Corollary 22.3 of Munkres, it suffices to show that $ \widetilde{ f}$ is a quotient map. By Exercise 22.2(b), it suffices to show that there exists a continuous map $ h:\widetilde{ f}(S^2) \to S^2$ s.t.\ $ \widetilde{ f} \circ h = \text{id}_{ \widetilde{ f}(S^2)} $.  Given $ (a,b,c,d) \in \widetilde{ f}(S^2)$, we want to recover $ x,y,z$ in terms of  $ a,b,c,d$. Define  $ A=\sqrt{d-1+2\sqrt{2} b} $ and $ B = \sqrt{3-d+2\sqrt{2}a } $. Then we can check that
	\begin{align*}
x &=\frac{\sqrt{2}(d-2) }{2(A-B) } - \frac{A-B}{ 2\sqrt{2} } \\
y &= \frac{a}{x} \\
		z &= \frac{A-B}{2\sqrt{2}  }+\frac{\sqrt{2}(d-2) }{2(A-B) }
	\end{align*}
	does the job (using the relation that $ x^2+y^2+z^2=1$). Then the map $ h:(a,b,c,d)\mapsto (x,y,z)$ is clearly continuous as the composition of continuous functions and gives us the identity (no need to worry about dividing by 0 since we know such $ x,y,z$ exist as $ (a,b,c,d)$ is in the image). It follows that $ g$ is a homemomorphism onto its image so it is an embedding.

	Alternatively, we could use the fact that since $ \rr P^2$ is the quotient of a compact space and therefore compact, and $ \rr^{4}$ is Hausdorff so any subspace is Hausdorff, since $ g$ is already bijective continuous, it is a homeomorphism by Munkres Theorem 26.6.

\end{problem}
\begin{problem}[7.16]
I claim that $ 1$ is a regular value of  $ g:TM \to \rr$. Given $ p \in g^{-1}(1)$, $ \rank dg_p = 0$ or $1 $. Suppose $ \rank dg_p = 0$, then it has to send everything to 0, but since $ g_p$ is an inner product, it is positive definite and cannot be constant so the derivative cannot be 0. Hence $ \rank dg_p =1$ and $ 1$ is a regular value of  $ g$. Thus  $ g^{-1}(1)$ is a $ 2n-1$ manifold which is the unit tangent bundle.
\end{problem}
\begin{problem}[8.5]
Suppose $ X \subseteq M$ is measure zero. Then for any $ p \in M$, denote the chart given by the definition $ (U_p, \phi_p)$ s.t.\ $ \phi_p(U_p \cap X)$ has measure zero. Clearly these charts form an atlas for $ M$. By Exercise 8, we can obtain a countable atlas  $ (U_i,\phi_i)$. Given any arbitrary chart $(V,\psi) $ of $ M$, $ V = \bigcup_{ i \in \nn} U_i$. Then $ \psi \circ \phi_i ^{-1}: \phi_i(U_i) \cap \psi(V) \subseteq \rr^{n} \to \rr^{n}$ is smooth. Thus we can apply Lemma 4 and obtain:
\begin{align*}
	 \psi \circ \phi_i ^{-1} (\phi_i(U_i \cap X) \cap \phi_i(U_i \cap V)) &= \psi (U_i \cap X \cap V) 
\end{align*}
has measure zero. Then we see that
\begin{align*}
	\psi(V \cap X) \subseteq \bigcup_{ i \in \nn} \psi(U_i \cap X \cap V) 
\end{align*}
Since countable union of measure zero sets are measure zero, $ \psi(V \cap X)$ is also measure zero by monotonicty.
\end{problem}
\begin{problem}[8.8]
	Given an arbitrary atlas $ \mathcal{ A} = \{(U_{ \alpha}, \phi_{ \alpha}): \alpha \in J\} $ of a manifold $M$, since $ M$ is second-countable, by Theorem 30.3 of Munkres, it is a Lindelof space. Therefore, there exists a countable subcover $ \mathcal{ A'}$ of $ \mathcal{ A}$ that serves as a countable atlas.
\end{problem}
\begin{problem}[8.11]
$ (\implies)$: The derivative of $ \pi_u(x)$ at $ p$ is the unique linear operator $ d_{\pi_u}: T_p M \to \rr^{n-1}$ satisfying the following
\begin{align*}
	\lim_{ t \to 0} \frac{\pi_u(x+tv) -(\pi_u(x)+ td_{\pi_u}(v))}{ t} = 0
\end{align*}
where $ v \in T_pM \cong \rr^{n} \subseteq \rr^{N}$. Notice
\begin{align*}
	\frac{1}{t}(x+tv - \langle x+tv,u \rangle u - x - \langle x,u \rangle u ) &= \frac{x}{t}+v- \langle \frac{x}{t},u \rangle u - \langle v,u \rangle u - \frac{x}{t} -\langle \frac{x}{t},u \rangle u \\
	&= v- \langle v,u \rangle u 
\end{align*}
Alternatively, projection is a linear operator so its derivative is just itself. Therefore, it's easy to see that $ d_{\pi_u}(v) = v- \langle v,u \rangle u$. Now we have $ u \not\in T_pM$ iff $ u$ and  $ v$ are linearly independent iff $ t u -v \neq 0 , v \neq 0, \ \forall \ t \in \rr$ iff $ d_{\pi_u}$ doesn't send any nonzero vector in $ T_pM$ to 0 iff $ \ker d_{\pi_u}$ iff it is not singular at $ p$ onto its image.
$ (\impliedby):$ 
\end{problem}
\end{document}
