\documentclass[12pt]{article}
\newcommand{\alert}[1]{{\bf \color{red} [Alert:] #1}}
\newcommand{\todo}[1]{{\bf \color{orange} [TODO:] #1}}
\newcommand{\real}[1][]{\mathbb{R}^{#1}}
\newcommand{\myeqn}[1]{(\ref{#1})}
\newcommand{\myex}[1]{Example \ref{#1}}
\newcommand{\defeq}{\stackrel{\mathrm{def}}{=}}
\newcommand{\parder}[2]{\frac{\partial #1}{\partial #2}}
\newcommand{\Lie}[3][]{\mathsf{L}_{#3}^{#1} #2}
\newcommand{\LieA}[1]{\mathsf{Lie}(#1)}
\newcommand{\lieder}[2]{\mathcal{L}_{#2} #1}
\renewcommand{\t}{^{\mbox{\tiny\sf T}}}
\newcommand{\trans}{^{\mbox{\tiny\sf T}}}
\newcommand{\markup}[1]{\{\textbf{#1}\}}
\newcommand{\msub}[1]{_\mathrm{#1}}
\newcommand{\msup}[1]{^\mathrm{#1}}
\newcommand{\inv}[1]{#1^{-1}}
\newcommand{\pinv}[1]{{#1}^{+}}
\newcommand{\myfracA}[2]{\displaystyle{\frac{#1}{#2}}}
\newcommand{\myfracB}[2]{{#1}/{#2}}
\newcommand{\mydiffA}[1]{\dot{#1}}
\newcommand{\mydiffB}[2]{\myfracA{\mathrm{d}{#1}}{\mathrm{d}{#2}}}
\newcommand{\ball}[2]{\mathcal{B}_{#1}\left(#2\right)}
\newcommand{\acos}[1]{\cos^{-1}\left(#1\right)}
\newcommand{\asin}[1]{\sin^{-1}\left(#1\right)}
\newcommand{\mani}{\mathcal{M}}
\newcommand{\tang}[2]{\mathsf{T}_{#1} #2}
\newcommand{\LieB}[2]{[ #1, #2 ]}
\newcommand{\LieBad}[3][]{\mathsf{ad}_{#2}^{#1} #3}
\newcommand{\ReachVT}{\mathcal{R}^V_T}
\newcommand{\ReachVt}{\mathcal{R}^V_t}
\newcommand{\ReachVTe}{\mathcal{R}^V_{\le T}}
\newcommand{\ReachT}{\mathcal{R}_T}
\newcommand{\Reacht}{\mathcal{R}_t}
\newcommand{\ReachTe}{\mathcal{R}_{\le T}}
\newcommand{\accLA}[1]{\mathsf{Lie}(#1)}
\newcommand{\accD}{\Delta_{\mathcal{F}}}
\newcommand{\accSA}{\mathsf{Lie}(\mathcal{G},f)}
\newcommand{\accDS}{\Delta_{\mathcal{G}}}
\newcommand{\eval}[3]{\mathsf{Ev}^{#2}_{#1}\left( #3 \right)}
\newcommand{\stlc}{\textsc{stlc}}
\newcommand{\clf}{\textsc{clf}}
\newcommand{\jqlf}{\textsc{jqlf}}
\newcommand{\dlas}{\textsc{dlas}}
\newcommand{\Ad}[2]{\mathsf{Ad}_{#1} #2}
\newcommand{\xe}{\ensuremath{x_e}}
\newcommand{\lebg}[1]{\mathcal{L}_{#1}}
\newcommand{\lebgx}[1]{\mathcal{L}_{#1 \mathrm{e}}}
\newcommand{\dom}{D}
\newcommand{\domT}{[t_0,\infty) \times D}
\newcommand{\rarrow}{\rightarrow}
\renewcommand{\d}{\mathrm{d}}
\renewcommand{\Re}{\mathbb{R}}
\newcommand{\C}{\mathrm{C}}

\newcommand{\QED}{{\unskip\nobreak\hfil\penalty50\hskip2em\vadjust{}
		\nobreak\hfil$\Box$\parfillskip=0pt\finalhyphendemerits=0\par}\vspace{0.1cm}}
\newcommand{\eoEx}{{\unskip\nobreak\hfil\penalty50\hskip0em\vadjust{}
		\nobreak\hfil$\Large\Diamond$\parfillskip=0pt\finalhyphendemerits=0\par}\vspace{0.1cm}}

\newcommand{\sgn}{\ensuremath{\operatorname{sgn}}}
\newcommand{\sat}{\ensuremath{\operatorname{sat}}}

\newcommand{\half}{\frac{1}{2}}
\newcommand{\shalf}{\mbox{$\frac{1}{2}$}}
\newcommand{\marcom}[1]{\marginpar{\footnotesize #1}}
\newcommand{\der}{\mathrm{D}}
\newcommand{\e}{\mathrm{e}}
\newcommand{\dt}{\mathrm{d}t}

\newcommand{\cA}{\ensuremath{\mathcal{A}}}
\newcommand{\cB}{\ensuremath{\mathcal{B}}}
\newcommand{\cG}{\ensuremath{\mathcal{G}}}
\newcommand{\cK}{\ensuremath{\mathcal{K}}}
\newcommand{\cW}{\ensuremath{\mathcal{W}}}
\newcommand{\cZ}{\ensuremath{\mathcal{Z}}}
\newcommand{\cS}{\ensuremath{\mathcal{S}}}
\newcommand{\cD}{\ensuremath{\mathcal{D}}}
\newcommand{\cP}{\ensuremath{\mathcal{P}}}
\newcommand{\cV}{\ensuremath{\mathcal{V}}}
\newcommand{\cL}{\ensuremath{\mathcal{L}}}
\newcommand{\cN}{\ensuremath{\mathcal{N}}}
\newcommand{\cI}{\ensuremath{\mathcal{I}}}
\newcommand{\cR}{\ensuremath{\mathcal{R}}}
\newcommand{\cM}{\ensuremath{\mathcal{M}}}
\newcommand{\cC}{\ensuremath{\mathcal{C}}}
\newcommand{\cF}{\ensuremath{\mathcal{F}}}
\newcommand{\cH}{\ensuremath{\mathcal{H}}}
\newcommand{\cO}{\ensuremath{\mathcal{O}}}
\newcommand{\cX}{\ensuremath{\mathcal{X}}}
\newcommand{\cY}{\ensuremath{\mathcal{Y}}}
\newcommand{\Ci}{\ensuremath{\mathcal{C}^\infty}}
\newcommand{\ISS}{\textsc{iss}}
\newcommand{\LISS}{\textsc{liss}}
\newcommand{\GAS}{\textsc{gas}}
\newcommand{\GS}{\textsc{gs}}
\newcommand{\LES}{\textsc{les}}
\newcommand{\GUAS}{\textsc{guas}}
\newcommand{\BIBO}{\textsc{bibo}}
\newcommand{\spec}{\ensuremath{\operatorname{spec}}}
\newcommand{\spn}{\ensuremath{\operatorname{span}}}
\renewcommand{\i}{\mathrm{i\,}}

\renewcommand{\implies}{\Rightarrow}

\renewcommand{\theenumi}{$\roman{enumi})$}
\renewcommand{\labelenumi}{\theenumi}

\font\ptmten=zptmcmrm scaled 1200
\newcommand{\w}{\mbox{{\ptmten w}}}
\newcommand{\z}{\mbox{{\ptmten z}}}
\renewcommand{\Re}{\mathbb{R}}

\newcommand{\cl}{\operatorname{cl}}
\newcommand{\intr}{\operatorname{int}}
\newcommand{\rank}{\operatorname{rank}}
\newcommand{\co}{\operatorname{co}}
\newcommand{\aff}{\operatorname{aff}}

\theoremstyle{plain}
\newtheorem{theorem}{Theorem}[chapter]
\newtheorem{claim}[theorem]{Claim}
\newtheorem{corollary}[theorem]{Corollary}
\newtheorem{prop}[theorem]{Proposition}
\newtheorem{fact}[theorem]{Fact}
\newtheorem{lemma}[theorem]{Lemma}

\newtheorem{remark}{Remark}[chapter]

\theoremstyle{definition}
\newtheorem{assume}[theorem]{Assumption}
\newtheorem{defn}[theorem]{Definition}
\newtheorem{problem}[theorem]{Problem}
\newtheorem{exercise}{Exercise}
\newtheorem{example}[theorem]{Example}


\begin{document}
\centerline {\textsf{\textbf{\LARGE{Homework 3}}}}
\centerline {Jaden Wang}
\vspace{.15in}

\begin{problem}[6.5]
No. Since $ f: \rr \to \rr, x \mapsto \sqrt{x^2+1}$, notice $ \sqrt{x^2+1} > x, \sqrt{y^2+1} >y$, and $ x-y$ and  $ \sqrt{x^2+1} + \sqrt{y^2+1}$ has the same sign so we can drop the absolute value, then
\begin{align*}
	\frac{(x-y)}{ \sqrt{x^2+1} - \sqrt{y^2+1} } &= \frac{(x-y)(\sqrt{x^2+1}+ \sqrt{y^2+1})}{ x^2+1 - y^2 - 1} \\
	&= \frac{\sqrt{x^2+1} + \sqrt{y^2+1}}{ x+y} \\
	&> 1
\end{align*}
So we establish that $| f(x) - f(y) | < |x-y|$. However, suppose $\sqrt{x^2+1}  = x $ this forces $ 0=1$ a contradiction so  $ f$ has no fix-point. Since  $ \rr$ is a complete metric space, we have a counterexample.
\end{problem}

\begin{problem}[6.7]
	Define $ \gamma: [0,1] \to \rr^{n}, t \mapsto (1-t)p+tq$. Then $ f \circ \gamma: [0,1] \to \rr$. Notice that $ \gamma'(t) = q-p$. Applying the 1D mean value theorem to this function yields a $ t \in (0,1)$ s.t.\ 
	\begin{align*}
		\frac{f \circ \gamma (1) - g \circ \gamma (0) }{ 1-0} &= (f \circ \gamma)'(t)\\
		f(q) - f(p)&= Df(s) \circ \gamma'(t) & s:= (1-t)p+tq \\
		f(q) - f(p) &= Df(s) (q-p).
	\end{align*}
\end{problem}

\begin{problem}[6.10]
\begin{thm}
Let $ f: \rr \to \rr$ be a smooth map. Suppose that $ f'$ is nonzero at some $ p \in \rr$. Then $ f^{-1}$ exists in some open interval around $ p$ and is also smooth. Moreover,
\begin{align*}
	(f^{-1})'(f(p)) = \frac{1}{f'(p)}.
\end{align*}
\end{thm}
\begin{proof}
Since $ f'(p)$ is nonzero, it is either positive or negative. WLOG suppose $ f'(p)>0$, then since $ f'$ is continuous, $ f$ is monotone on  some interval $ (c, d)$ containing $ p$. Hence $ f^{-1}$ exists on this interval. Smoothness of $ f ^{-1}$ on $ (c,d)$ is the result that $ f \circ f ^{-1} = \text{id} $ and $ f$ and  $ \text{id} $ are both smooth so $ f^{-1}$ must be smooth. By chain rule,
\begin{align*}
	(f \circ f^{-1})'(f(p)) &= \text{id}'(f(p))  \\
	f'(p) (f^{-1})'(f(p))&= 1 \\
	(f^{-1})'(f(p)) &= \frac{1}{f'(p)} 
\end{align*}
\end{proof}
\end{problem}

\begin{problem}[6.12]
Suppose the theorem is true when $ M = \rr^{m}$ and $ N = \rr^{n}$. Then for a smooth map of general manifolds (with dim $ m,n$) $ f:M \to N$, take any local charts $ (U,\phi),(V,\psi)$ of $ p$ and  $ f(p)$, we see that $ \psi \circ f \circ \phi ^{-1}: \rr^{m} \to \rr^n$ by assumption yields local charts (linear isomorphisms) $(U', \phi')$ and $ (V', \psi')$ for $ \rr^{m}$ and $ \rr^{n}$ respectively s.t.\ 
\begin{align*}
	\psi' \circ (\psi \circ f \circ \phi^{-1}) \circ \phi' ^{-1}(x_1,\ldots,x_n) &= \Psi \circ f \circ \Phi^{-1} (x_1,\ldots,x_n)\\
	&= (x_1,\ldots,x_k,0\ldots,0) 
\end{align*}
Then $\Phi := \phi' \circ  \phi$ and $\Psi:= \psi' \circ \psi$ are the local charts we seek for $ f$.

Now assume $ f: \rr^{m} \to \rr^{n}$. If $ p \neq 0$, then we can set  $ \wh{ f}(x) = f(x-p) $ which has the same Jacobian as $ f$ so it doesn't change the proof. Similarly, if $ f(p) \neq 0$, we can set  $ \widetilde{ f}(x) = f(x) - f(p)$ and again it doesn't change the Jacobian. Finally, by assumption $\rank Df =k$, so  $ Df(0)$ has  $ k$ linearly-independent columns and $ k$ linearly independent rows. So by a series of permutation matrices we can swap all the linearly-independent columns to the first $ k$-columns, and then swap all the linearly-independent rows to the first $ k$-columns. Then the first $ k \times k$ submatrix has $ k$ pivots so it is nonsingular. These permutations are nonsingular and smooth which still allow us to use the inverse function theorem.
\end{problem}

\begin{problem}[6.13]
Suppose $ f: \rr^2 \to \rr$ is $ C^{1}$ and 1-1. Then $ \rank d_pf \leq \dim \rr = 1$ so it is either 0 or 1. Suppose $ \rank d_pf = 0$ for all $ p \in \rr^2$, then $ f$ is clearly not 1-1  as it is constant. Suppose there exists a $ p \in \rr^2$ s.t.\ $ \rank d_pf = 1$. Then we know that  $ f(p)$ is a regular value by definition, but since  $ f$ is 1-1, the preimage $f^{-1}(f(p)) = \{p\} $ which is a submanifold. But $ p$ is clearly not a manifold, a contradiction. Hence in both cases no such function can exist.
\end{problem}

\begin{problem}[1.3.5]
Recall that $ f:X \to Y$ is a local diffeomorphism if for all $ x \in X$, $ f$  maps a neighborhood of $ x$ diffeomorphically to a neighborhood of  $ f(x)$. 

Since $ f$ is clearly surjective onto its image, and $ f$ is 1-1 by assumption, we have  $ f:X \to f(X)$ is a bijection so $ f^{-1}$ is well-defined as a set map. Denote each neighborhood of $ x$ locally diffeomorphic to open set of $ Y$ as  $ U_x$. Let $ V_y = f(U_x)$ with  $ y = f(x)$. Then each $ V_y$ is open by local diffeomorphism so $ W:=\bigcup_{ x \in X} V_y $ is an open subset of $ Y$. Since $ W \subseteq f(X)$ but $ f(x) \in W \ \forall \ x \in X$, we have $ W = f(X)$. Moreover, $ (f^{-1})|_{V_y} = (f|_{U_x})^{-1}$ so $ f^{-1}$ is locally smooth as well. Smoothness of $ f$ follows from that $ f$ is locally smooth so any transition map restricted to such neighborhood is smooth. These restricted neighborhood for all points in $ X$ still form an atlas of $ X$ so  any transition map from this atlas is smooth. Likewise $ f^{-1}: W \to X$ is smooth. Therefore, $ f$ is a diffeomorphism from  $ X$ to  $ W$.
\end{problem}
\end{document}
