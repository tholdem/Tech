\documentclass[12pt]{article}
%Fall 2022
% Some basic packages
\usepackage{standalone}[subpreambles=true]
\usepackage[utf8]{inputenc}
\usepackage[T1]{fontenc}
\usepackage{textcomp}
\usepackage[english]{babel}
\usepackage{url}
\usepackage{graphicx}
%\usepackage{quiver}
\usepackage{float}
\usepackage{enumitem}
\usepackage{lmodern}
\usepackage{comment}
\usepackage{hyperref}
\usepackage[usenames,svgnames,dvipsnames]{xcolor}
\usepackage[margin=1in]{geometry}
\usepackage{pdfpages}

\pdfminorversion=7

% Don't indent paragraphs, leave some space between them
\usepackage{parskip}

% Hide page number when page is empty
\usepackage{emptypage}
\usepackage{subcaption}
\usepackage{multicol}
\usepackage[b]{esvect}

% Math stuff
\usepackage{amsmath, amsfonts, mathtools, amsthm, amssymb}
\usepackage{bbm}
\usepackage{stmaryrd}
\allowdisplaybreaks

% Fancy script capitals
\usepackage{mathrsfs}
\usepackage{cancel}
% Bold math
\usepackage{bm}
% Some shortcuts
\newcommand{\rr}{\ensuremath{\mathbb{R}}}
\newcommand{\zz}{\ensuremath{\mathbb{Z}}}
\newcommand{\qq}{\ensuremath{\mathbb{Q}}}
\newcommand{\nn}{\ensuremath{\mathbb{N}}}
\newcommand{\ff}{\ensuremath{\mathbb{F}}}
\newcommand{\cc}{\ensuremath{\mathbb{C}}}
\newcommand{\ee}{\ensuremath{\mathbb{E}}}
\newcommand{\hh}{\ensuremath{\mathbb{H}}}
\renewcommand\O{\ensuremath{\emptyset}}
\newcommand{\norm}[1]{{\left\lVert{#1}\right\rVert}}
\newcommand{\dbracket}[1]{{\left\llbracket{#1}\right\rrbracket}}
\newcommand{\ve}[1]{{\bm{#1}}}
\newcommand\allbold[1]{{\boldmath\textbf{#1}}}
\DeclareMathOperator{\lcm}{lcm}
\DeclareMathOperator{\im}{im}
\DeclareMathOperator{\coim}{coim}
\DeclareMathOperator{\dom}{dom}
\DeclareMathOperator{\tr}{tr}
\DeclareMathOperator{\rank}{rank}
\DeclareMathOperator*{\var}{Var}
\DeclareMathOperator*{\ev}{E}
\DeclareMathOperator{\dg}{deg}
\DeclareMathOperator{\aff}{aff}
\DeclareMathOperator{\conv}{conv}
\DeclareMathOperator{\inte}{int}
\DeclareMathOperator*{\argmin}{argmin}
\DeclareMathOperator*{\argmax}{argmax}
\DeclareMathOperator{\graph}{graph}
\DeclareMathOperator{\sgn}{sgn}
\DeclareMathOperator*{\Rep}{Rep}
\DeclareMathOperator{\Proj}{Proj}
\DeclareMathOperator{\mat}{mat}
\DeclareMathOperator{\diag}{diag}
\DeclareMathOperator{\aut}{Aut}
\DeclareMathOperator{\gal}{Gal}
\DeclareMathOperator{\inn}{Inn}
\DeclareMathOperator{\edm}{End}
\DeclareMathOperator{\Hom}{Hom}
\DeclareMathOperator{\ext}{Ext}
\DeclareMathOperator{\tor}{Tor}
\DeclareMathOperator{\Span}{Span}
\DeclareMathOperator{\Stab}{Stab}
\DeclareMathOperator{\cont}{cont}
\DeclareMathOperator{\Ann}{Ann}
\DeclareMathOperator{\Div}{div}
\DeclareMathOperator{\curl}{curl}
\DeclareMathOperator{\nat}{Nat}
\DeclareMathOperator{\gr}{Gr}
\DeclareMathOperator{\vect}{Vect}
\DeclareMathOperator{\id}{id}
\DeclareMathOperator{\Mod}{Mod}
\DeclareMathOperator{\sign}{sign}
\DeclareMathOperator{\Surf}{Surf}
\DeclareMathOperator{\fcone}{fcone}
\DeclareMathOperator{\Rot}{Rot}
\DeclareMathOperator{\grad}{grad}
\DeclareMathOperator{\atan2}{atan2}
\DeclareMathOperator{\Ric}{Ric}
\let\vec\relax
\DeclareMathOperator{\vec}{vec}
\let\Re\relax
\DeclareMathOperator{\Re}{Re}
\let\Im\relax
\DeclareMathOperator{\Im}{Im}
% Put x \to \infty below \lim
\let\svlim\lim\def\lim{\svlim\limits}

%wide hat
\usepackage{scalerel,stackengine}
\stackMath
\newcommand*\wh[1]{%
\savestack{\tmpbox}{\stretchto{%
  \scaleto{%
    \scalerel*[\widthof{\ensuremath{#1}}]{\kern-.6pt\bigwedge\kern-.6pt}%
    {\rule[-\textheight/2]{1ex}{\textheight}}%WIDTH-LIMITED BIG WEDGE
  }{\textheight}% 
}{0.5ex}}%
\stackon[1pt]{#1}{\tmpbox}%
}
\parskip 1ex

%Make implies and impliedby shorter
\let\implies\Rightarrow
\let\impliedby\Leftarrow
\let\iff\Leftrightarrow
\let\epsilon\varepsilon

% Add \contra symbol to denote contradiction
\usepackage{stmaryrd} % for \lightning
\newcommand\contra{\scalebox{1.5}{$\lightning$}}

% \let\phi\varphi

% Command for short corrections
% Usage: 1+1=\correct{3}{2}

\definecolor{correct}{HTML}{009900}
\newcommand\correct[2]{\ensuremath{\:}{\color{red}{#1}}\ensuremath{\to }{\color{correct}{#2}}\ensuremath{\:}}
\newcommand\green[1]{{\color{correct}{#1}}}

% horizontal rule
\newcommand\hr{
    \noindent\rule[0.5ex]{\linewidth}{0.5pt}
}

% hide parts
\newcommand\hide[1]{}

% si unitx
\usepackage{siunitx}
\sisetup{locale = FR}

%allows pmatrix to stretch
\makeatletter
\renewcommand*\env@matrix[1][\arraystretch]{%
  \edef\arraystretch{#1}%
  \hskip -\arraycolsep
  \let\@ifnextchar\new@ifnextchar
  \array{*\c@MaxMatrixCols c}}
\makeatother

\renewcommand{\arraystretch}{0.8}

\renewcommand{\baselinestretch}{1.5}

\usepackage{graphics}
\usepackage{epstopdf}

\RequirePackage{hyperref}
%%
%% Add support for color in order to color the hyperlinks.
%% 
\hypersetup{
  colorlinks = true,
  urlcolor = blue,
  citecolor = blue
}
%%fakesection Links
\hypersetup{
    colorlinks,
    linkcolor={red!50!black},
    citecolor={green!50!black},
    urlcolor={blue!80!black}
}
%customization of cleveref
\RequirePackage[capitalize,nameinlink]{cleveref}[0.19]

% Per SIAM Style Manual, "section" should be lowercase
\crefname{section}{section}{sections}
\crefname{subsection}{subsection}{subsections}
\Crefname{section}{Section}{Sections}
\Crefname{subsection}{Subsection}{Subsections}

% Per SIAM Style Manual, "Figure" should be spelled out in references
\Crefname{figure}{Figure}{Figures}

% Per SIAM Style Manual, don't say equation in front on an equation.
\crefformat{equation}{\textup{#2(#1)#3}}
\crefrangeformat{equation}{\textup{#3(#1)#4--#5(#2)#6}}
\crefmultiformat{equation}{\textup{#2(#1)#3}}{ and \textup{#2(#1)#3}}
{, \textup{#2(#1)#3}}{, and \textup{#2(#1)#3}}
\crefrangemultiformat{equation}{\textup{#3(#1)#4--#5(#2)#6}}%
{ and \textup{#3(#1)#4--#5(#2)#6}}{, \textup{#3(#1)#4--#5(#2)#6}}{, and \textup{#3(#1)#4--#5(#2)#6}}

% But spell it out at the beginning of a sentence.
\Crefformat{equation}{#2Equation~\textup{(#1)}#3}
\Crefrangeformat{equation}{Equations~\textup{#3(#1)#4--#5(#2)#6}}
\Crefmultiformat{equation}{Equations~\textup{#2(#1)#3}}{ and \textup{#2(#1)#3}}
{, \textup{#2(#1)#3}}{, and \textup{#2(#1)#3}}
\Crefrangemultiformat{equation}{Equations~\textup{#3(#1)#4--#5(#2)#6}}%
{ and \textup{#3(#1)#4--#5(#2)#6}}{, \textup{#3(#1)#4--#5(#2)#6}}{, and \textup{#3(#1)#4--#5(#2)#6}}

% Make number non-italic in any environment.
\crefdefaultlabelformat{#2\textup{#1}#3}

% Environments
\makeatother
% For box around Definition, Theorem, \ldots
%%fakesection Theorems
\usepackage{thmtools}
\usepackage[framemethod=TikZ]{mdframed}

\theoremstyle{definition}
\mdfdefinestyle{mdbluebox}{%
	roundcorner = 10pt,
	linewidth=1pt,
	skipabove=12pt,
	innerbottommargin=9pt,
	skipbelow=2pt,
	nobreak=true,
	linecolor=blue,
	backgroundcolor=TealBlue!5,
}
\declaretheoremstyle[
	headfont=\sffamily\bfseries\color{MidnightBlue},
	mdframed={style=mdbluebox},
	headpunct={\\[3pt]},
	postheadspace={0pt}
]{thmbluebox}

\mdfdefinestyle{mdredbox}{%
	linewidth=0.5pt,
	skipabove=12pt,
	frametitleaboveskip=5pt,
	frametitlebelowskip=0pt,
	skipbelow=2pt,
	frametitlefont=\bfseries,
	innertopmargin=4pt,
	innerbottommargin=8pt,
	nobreak=false,
	linecolor=RawSienna,
	backgroundcolor=Salmon!5,
}
\declaretheoremstyle[
	headfont=\bfseries\color{RawSienna},
	mdframed={style=mdredbox},
	headpunct={\\[3pt]},
	postheadspace={0pt},
]{thmredbox}

\declaretheorem[%
style=thmbluebox,name=Theorem,numberwithin=section]{thm}
\declaretheorem[style=thmbluebox,name=Lemma,sibling=thm]{lem}
\declaretheorem[style=thmbluebox,name=Proposition,sibling=thm]{prop}
\declaretheorem[style=thmbluebox,name=Corollary,sibling=thm]{coro}
\declaretheorem[style=thmredbox,name=Example,sibling=thm]{eg}

\mdfdefinestyle{mdgreenbox}{%
	roundcorner = 10pt,
	linewidth=1pt,
	skipabove=12pt,
	innerbottommargin=9pt,
	skipbelow=2pt,
	nobreak=true,
	linecolor=ForestGreen,
	backgroundcolor=ForestGreen!5,
}

\declaretheoremstyle[
	headfont=\bfseries\sffamily\color{ForestGreen!70!black},
	bodyfont=\normalfont,
	spaceabove=2pt,
	spacebelow=1pt,
	mdframed={style=mdgreenbox},
	headpunct={ --- },
]{thmgreenbox}

\declaretheorem[style=thmgreenbox,name=Definition,sibling=thm]{defn}

\mdfdefinestyle{mdgreenboxsq}{%
	linewidth=1pt,
	skipabove=12pt,
	innerbottommargin=9pt,
	skipbelow=2pt,
	nobreak=true,
	linecolor=ForestGreen,
	backgroundcolor=ForestGreen!5,
}
\declaretheoremstyle[
	headfont=\bfseries\sffamily\color{ForestGreen!70!black},
	bodyfont=\normalfont,
	spaceabove=2pt,
	spacebelow=1pt,
	mdframed={style=mdgreenboxsq},
	headpunct={},
]{thmgreenboxsq}
\declaretheoremstyle[
	headfont=\bfseries\sffamily\color{ForestGreen!70!black},
	bodyfont=\normalfont,
	spaceabove=2pt,
	spacebelow=1pt,
	mdframed={style=mdgreenboxsq},
	headpunct={},
]{thmgreenboxsq*}

\mdfdefinestyle{mdblackbox}{%
	skipabove=8pt,
	linewidth=3pt,
	rightline=false,
	leftline=true,
	topline=false,
	bottomline=false,
	linecolor=black,
	backgroundcolor=RedViolet!5!gray!5,
}
\declaretheoremstyle[
	headfont=\bfseries,
	bodyfont=\normalfont\small,
	spaceabove=0pt,
	spacebelow=0pt,
	mdframed={style=mdblackbox}
]{thmblackbox}

\theoremstyle{plain}
\declaretheorem[name=Question,sibling=thm,style=thmblackbox]{ques}
\declaretheorem[name=Remark,sibling=thm,style=thmgreenboxsq]{remark}
\declaretheorem[name=Remark,sibling=thm,style=thmgreenboxsq*]{remark*}
\newtheorem{ass}[thm]{Assumptions}

\theoremstyle{definition}
\newtheorem*{problem}{Problem}
\newtheorem{claim}[thm]{Claim}
\theoremstyle{remark}
\newtheorem*{case}{Case}
\newtheorem*{notation}{Notation}
\newtheorem*{note}{Note}
\newtheorem*{motivation}{Motivation}
\newtheorem*{intuition}{Intuition}
\newtheorem*{conjecture}{Conjecture}

% Make section starts with 1 for report type
%\renewcommand\thesection{\arabic{section}}

% End example and intermezzo environments with a small diamond (just like proof
% environments end with a small square)
\usepackage{etoolbox}
\AtEndEnvironment{vb}{\null\hfill$\diamond$}%
\AtEndEnvironment{intermezzo}{\null\hfill$\diamond$}%
% \AtEndEnvironment{opmerking}{\null\hfill$\diamond$}%

% Fix some spacing
% http://tex.stackexchange.com/questions/22119/how-can-i-change-the-spacing-before-theorems-with-amsthm
\makeatletter
\def\thm@space@setup{%
  \thm@preskip=\parskip \thm@postskip=0pt
}

% Fix some stuff
% %http://tex.stackexchange.com/questions/76273/multiple-pdfs-with-page-group-included-in-a-single-page-warning
\pdfsuppresswarningpagegroup=1


% My name
\author{Jaden Wang}



\begin{document}
\centerline {\textsf{\textbf{\LARGE{Homework 3}}}}
\centerline {Jaden Wang}
\vspace{.15in}

\begin{problem}[6.5]
No. Since $ f: \rr \to \rr, x \mapsto \sqrt{x^2+1}$, notice $ \sqrt{x^2+1} > x, \sqrt{y^2+1} >y$, and $ x-y$ and  $ \sqrt{x^2+1} + \sqrt{y^2+1}$ has the same sign so we can drop the absolute value, then
\begin{align*}
	\frac{(x-y)}{ \sqrt{x^2+1} - \sqrt{y^2+1} } &= \frac{(x-y)(\sqrt{x^2+1}+ \sqrt{y^2+1})}{ x^2+1 - y^2 - 1} \\
	&= \frac{\sqrt{x^2+1} + \sqrt{y^2+1}}{ x+y} \\
	&> 1
\end{align*}
So we establish that $| f(x) - f(y) | < |x-y|$. However, suppose $\sqrt{x^2+1}  = x $ this forces $ 0=1$ a contradiction so  $ f$ has no fix-point. Since  $ \rr$ is a complete metric space, we have a counterexample.
\end{problem}

\begin{problem}[6.7]
	Define $ \gamma: [0,1] \to \rr^{n}, t \mapsto (1-t)p+tq$. Then $ f \circ \gamma: [0,1] \to \rr$. Notice that $ \gamma'(t) = q-p$. Applying the 1D mean value theorem to this function yields a $ t \in (0,1)$ s.t.\ 
	\begin{align*}
		\frac{f \circ \gamma (1) - g \circ \gamma (0) }{ 1-0} &= (f \circ \gamma)'(t)\\
		f(q) - f(p)&= Df(s) \circ \gamma'(t) & s:= (1-t)p+tq \\
		f(q) - f(p) &= Df(s) (q-p).
	\end{align*}
\end{problem}

\begin{problem}[6.10]
\begin{thm}
Let $ f: \rr \to \rr$ be a smooth map. Suppose that $ f'$ is nonzero at some $ p \in \rr$. Then $ f^{-1}$ exists in some open interval around $ p$ and is also smooth. Moreover,
\begin{align*}
	(f^{-1})'(f(p)) = \frac{1}{f'(p)}.
\end{align*}
\end{thm}
\begin{proof}
Since $ f'(p)$ is nonzero, it is either positive or negative. WLOG suppose $ f'(p)>0$, then since $ f'$ is continuous, $ f$ is monotone on  some interval $ (c, d)$ containing $ p$. Hence $ f^{-1}$ exists on this interval. Smoothness of $ f ^{-1}$ on $ (c,d)$ is the result that $ f \circ f ^{-1} = \text{id} $ and $ f$ and  $ \text{id} $ are both smooth so $ f^{-1}$ must be smooth. By chain rule,
\begin{align*}
	(f \circ f^{-1})'(f(p)) &= \text{id}'(f(p))  \\
	f'(p) (f^{-1})'(f(p))&= 1 \\
	(f^{-1})'(f(p)) &= \frac{1}{f'(p)} 
\end{align*}
\end{proof}
\end{problem}

\begin{problem}[6.12]
Suppose the theorem is true when $ M = \rr^{m}$ and $ N = \rr^{n}$. Then for a smooth map of general manifolds (with dim $ m,n$) $ f:M \to N$, take any local charts $ (U,\phi),(V,\psi)$ of $ p$ and  $ f(p)$, we see that $ \psi \circ f \circ \phi ^{-1}: \rr^{m} \to \rr^n$ by assumption yields local charts (linear isomorphisms) $(U', \phi')$ and $ (V', \psi')$ for $ \rr^{m}$ and $ \rr^{n}$ respectively s.t.\ 
\begin{align*}
	\psi' \circ (\psi \circ f \circ \phi^{-1}) \circ \phi' ^{-1}(x_1,\ldots,x_n) &= \Psi \circ f \circ \Phi^{-1} (x_1,\ldots,x_n)\\
	&= (x_1,\ldots,x_k,0\ldots,0) 
\end{align*}
Then $\Phi := \phi' \circ  \phi$ and $\Psi:= \psi' \circ \psi$ are the local charts we seek for $ f$.

Now assume $ f: \rr^{m} \to \rr^{n}$. If $ p \neq 0$, then we can set  $ \wh{ f}(x) = f(x-p) $ which has the same Jacobian as $ f$ so it doesn't change the proof. Similarly, if $ f(p) \neq 0$, we can set  $ \widetilde{ f}(x) = f(x) - f(p)$ and again it doesn't change the Jacobian. Finally, by assumption $\rank Df =k$, so  $ Df(0)$ has  $ k$ linearly-independent columns and $ k$ linearly independent rows. So by a series of permutation matrices we can swap all the linearly-independent columns to the first $ k$-columns, and then swap all the linearly-independent rows to the first $ k$-columns. Then the first $ k \times k$ submatrix has $ k$ pivots so it is nonsingular. These permutations are nonsingular and smooth which still allow us to use the inverse function theorem.
\end{problem}

\begin{problem}[6.13]
Suppose $ f: \rr^2 \to \rr$ is $ C^{1}$ and 1-1. Then $ \rank d_pf \leq \dim \rr = 1$ so it is either 0 or 1. Suppose $ \rank d_pf = 0$ for all $ p \in \rr^2$, then $ f$ is clearly not 1-1  as it is constant. Suppose there exists a $ p \in \rr^2$ s.t.\ $ \rank d_pf = 1$. Then we know that  $ f(p)$ is a regular value by definition, but since  $ f$ is 1-1, the preimage $f^{-1}(f(p)) = \{p\} $ which is a submanifold. But $ p$ is clearly not a manifold, a contradiction. Hence in both cases no such function can exist.
\end{problem}

\begin{problem}[1.3.5]
Recall that $ f:X \to Y$ is a local diffeomorphism if for all $ x \in X$, $ f$  maps a neighborhood of $ x$ diffeomorphically to a neighborhood of  $ f(x)$. 

Since $ f$ is clearly surjective onto its image, and $ f$ is 1-1 by assumption, we have  $ f:X \to f(X)$ is a bijection so $ f^{-1}$ is well-defined as a set map. Denote each neighborhood of $ x$ locally diffeomorphic to open set of $ Y$ as  $ U_x$. Let $ V_y = f(U_x)$ with  $ y = f(x)$. Then each $ V_y$ is open by local diffeomorphism so $ W:=\bigcup_{ x \in X} V_y $ is an open subset of $ Y$. Since $ W \subseteq f(X)$ but $ f(x) \in W \ \forall \ x \in X$, we have $ W = f(X)$. Moreover, $ (f^{-1})|_{V_y} = (f|_{U_x})^{-1}$ so $ f^{-1}$ is locally smooth as well. Smoothness of $ f$ follows from that $ f$ is locally smooth so any transition map restricted to such neighborhood is smooth. These restricted neighborhood for all points in $ X$ still form an atlas of $ X$ so  any transition map from this atlas is smooth. Likewise $ f^{-1}: W \to X$ is smooth. Therefore, $ f$ is a diffeomorphism from  $ X$ to  $ W$.
\end{problem}
\end{document}
