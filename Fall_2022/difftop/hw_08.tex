\documentclass[12pt]{article}
%Fall 2022
% Some basic packages
\usepackage{standalone}[subpreambles=true]
\usepackage[utf8]{inputenc}
\usepackage[T1]{fontenc}
\usepackage{textcomp}
\usepackage[english]{babel}
\usepackage{url}
\usepackage{graphicx}
%\usepackage{quiver}
\usepackage{float}
\usepackage{enumitem}
\usepackage{lmodern}
\usepackage{comment}
\usepackage{hyperref}
\usepackage[usenames,svgnames,dvipsnames]{xcolor}
\usepackage[margin=1in]{geometry}
\usepackage{pdfpages}

\pdfminorversion=7

% Don't indent paragraphs, leave some space between them
\usepackage{parskip}

% Hide page number when page is empty
\usepackage{emptypage}
\usepackage{subcaption}
\usepackage{multicol}
\usepackage[b]{esvect}

% Math stuff
\usepackage{amsmath, amsfonts, mathtools, amsthm, amssymb}
\usepackage{bbm}
\usepackage{stmaryrd}
\allowdisplaybreaks

% Fancy script capitals
\usepackage{mathrsfs}
\usepackage{cancel}
% Bold math
\usepackage{bm}
% Some shortcuts
\newcommand{\rr}{\ensuremath{\mathbb{R}}}
\newcommand{\zz}{\ensuremath{\mathbb{Z}}}
\newcommand{\qq}{\ensuremath{\mathbb{Q}}}
\newcommand{\nn}{\ensuremath{\mathbb{N}}}
\newcommand{\ff}{\ensuremath{\mathbb{F}}}
\newcommand{\cc}{\ensuremath{\mathbb{C}}}
\newcommand{\ee}{\ensuremath{\mathbb{E}}}
\newcommand{\hh}{\ensuremath{\mathbb{H}}}
\renewcommand\O{\ensuremath{\emptyset}}
\newcommand{\norm}[1]{{\left\lVert{#1}\right\rVert}}
\newcommand{\dbracket}[1]{{\left\llbracket{#1}\right\rrbracket}}
\newcommand{\ve}[1]{{\bm{#1}}}
\newcommand\allbold[1]{{\boldmath\textbf{#1}}}
\DeclareMathOperator{\lcm}{lcm}
\DeclareMathOperator{\im}{im}
\DeclareMathOperator{\coim}{coim}
\DeclareMathOperator{\dom}{dom}
\DeclareMathOperator{\tr}{tr}
\DeclareMathOperator{\rank}{rank}
\DeclareMathOperator*{\var}{Var}
\DeclareMathOperator*{\ev}{E}
\DeclareMathOperator{\dg}{deg}
\DeclareMathOperator{\aff}{aff}
\DeclareMathOperator{\conv}{conv}
\DeclareMathOperator{\inte}{int}
\DeclareMathOperator*{\argmin}{argmin}
\DeclareMathOperator*{\argmax}{argmax}
\DeclareMathOperator{\graph}{graph}
\DeclareMathOperator{\sgn}{sgn}
\DeclareMathOperator*{\Rep}{Rep}
\DeclareMathOperator{\Proj}{Proj}
\DeclareMathOperator{\mat}{mat}
\DeclareMathOperator{\diag}{diag}
\DeclareMathOperator{\aut}{Aut}
\DeclareMathOperator{\gal}{Gal}
\DeclareMathOperator{\inn}{Inn}
\DeclareMathOperator{\edm}{End}
\DeclareMathOperator{\Hom}{Hom}
\DeclareMathOperator{\ext}{Ext}
\DeclareMathOperator{\tor}{Tor}
\DeclareMathOperator{\Span}{Span}
\DeclareMathOperator{\Stab}{Stab}
\DeclareMathOperator{\cont}{cont}
\DeclareMathOperator{\Ann}{Ann}
\DeclareMathOperator{\Div}{div}
\DeclareMathOperator{\curl}{curl}
\DeclareMathOperator{\nat}{Nat}
\DeclareMathOperator{\gr}{Gr}
\DeclareMathOperator{\vect}{Vect}
\DeclareMathOperator{\id}{id}
\DeclareMathOperator{\Mod}{Mod}
\DeclareMathOperator{\sign}{sign}
\DeclareMathOperator{\Surf}{Surf}
\DeclareMathOperator{\fcone}{fcone}
\DeclareMathOperator{\Rot}{Rot}
\DeclareMathOperator{\grad}{grad}
\DeclareMathOperator{\atan2}{atan2}
\DeclareMathOperator{\Ric}{Ric}
\let\vec\relax
\DeclareMathOperator{\vec}{vec}
\let\Re\relax
\DeclareMathOperator{\Re}{Re}
\let\Im\relax
\DeclareMathOperator{\Im}{Im}
% Put x \to \infty below \lim
\let\svlim\lim\def\lim{\svlim\limits}

%wide hat
\usepackage{scalerel,stackengine}
\stackMath
\newcommand*\wh[1]{%
\savestack{\tmpbox}{\stretchto{%
  \scaleto{%
    \scalerel*[\widthof{\ensuremath{#1}}]{\kern-.6pt\bigwedge\kern-.6pt}%
    {\rule[-\textheight/2]{1ex}{\textheight}}%WIDTH-LIMITED BIG WEDGE
  }{\textheight}% 
}{0.5ex}}%
\stackon[1pt]{#1}{\tmpbox}%
}
\parskip 1ex

%Make implies and impliedby shorter
\let\implies\Rightarrow
\let\impliedby\Leftarrow
\let\iff\Leftrightarrow
\let\epsilon\varepsilon

% Add \contra symbol to denote contradiction
\usepackage{stmaryrd} % for \lightning
\newcommand\contra{\scalebox{1.5}{$\lightning$}}

% \let\phi\varphi

% Command for short corrections
% Usage: 1+1=\correct{3}{2}

\definecolor{correct}{HTML}{009900}
\newcommand\correct[2]{\ensuremath{\:}{\color{red}{#1}}\ensuremath{\to }{\color{correct}{#2}}\ensuremath{\:}}
\newcommand\green[1]{{\color{correct}{#1}}}

% horizontal rule
\newcommand\hr{
    \noindent\rule[0.5ex]{\linewidth}{0.5pt}
}

% hide parts
\newcommand\hide[1]{}

% si unitx
\usepackage{siunitx}
\sisetup{locale = FR}

%allows pmatrix to stretch
\makeatletter
\renewcommand*\env@matrix[1][\arraystretch]{%
  \edef\arraystretch{#1}%
  \hskip -\arraycolsep
  \let\@ifnextchar\new@ifnextchar
  \array{*\c@MaxMatrixCols c}}
\makeatother

\renewcommand{\arraystretch}{0.8}

\renewcommand{\baselinestretch}{1.5}

\usepackage{graphics}
\usepackage{epstopdf}

\RequirePackage{hyperref}
%%
%% Add support for color in order to color the hyperlinks.
%% 
\hypersetup{
  colorlinks = true,
  urlcolor = blue,
  citecolor = blue
}
%%fakesection Links
\hypersetup{
    colorlinks,
    linkcolor={red!50!black},
    citecolor={green!50!black},
    urlcolor={blue!80!black}
}
%customization of cleveref
\RequirePackage[capitalize,nameinlink]{cleveref}[0.19]

% Per SIAM Style Manual, "section" should be lowercase
\crefname{section}{section}{sections}
\crefname{subsection}{subsection}{subsections}
\Crefname{section}{Section}{Sections}
\Crefname{subsection}{Subsection}{Subsections}

% Per SIAM Style Manual, "Figure" should be spelled out in references
\Crefname{figure}{Figure}{Figures}

% Per SIAM Style Manual, don't say equation in front on an equation.
\crefformat{equation}{\textup{#2(#1)#3}}
\crefrangeformat{equation}{\textup{#3(#1)#4--#5(#2)#6}}
\crefmultiformat{equation}{\textup{#2(#1)#3}}{ and \textup{#2(#1)#3}}
{, \textup{#2(#1)#3}}{, and \textup{#2(#1)#3}}
\crefrangemultiformat{equation}{\textup{#3(#1)#4--#5(#2)#6}}%
{ and \textup{#3(#1)#4--#5(#2)#6}}{, \textup{#3(#1)#4--#5(#2)#6}}{, and \textup{#3(#1)#4--#5(#2)#6}}

% But spell it out at the beginning of a sentence.
\Crefformat{equation}{#2Equation~\textup{(#1)}#3}
\Crefrangeformat{equation}{Equations~\textup{#3(#1)#4--#5(#2)#6}}
\Crefmultiformat{equation}{Equations~\textup{#2(#1)#3}}{ and \textup{#2(#1)#3}}
{, \textup{#2(#1)#3}}{, and \textup{#2(#1)#3}}
\Crefrangemultiformat{equation}{Equations~\textup{#3(#1)#4--#5(#2)#6}}%
{ and \textup{#3(#1)#4--#5(#2)#6}}{, \textup{#3(#1)#4--#5(#2)#6}}{, and \textup{#3(#1)#4--#5(#2)#6}}

% Make number non-italic in any environment.
\crefdefaultlabelformat{#2\textup{#1}#3}

% Environments
\makeatother
% For box around Definition, Theorem, \ldots
%%fakesection Theorems
\usepackage{thmtools}
\usepackage[framemethod=TikZ]{mdframed}

\theoremstyle{definition}
\mdfdefinestyle{mdbluebox}{%
	roundcorner = 10pt,
	linewidth=1pt,
	skipabove=12pt,
	innerbottommargin=9pt,
	skipbelow=2pt,
	nobreak=true,
	linecolor=blue,
	backgroundcolor=TealBlue!5,
}
\declaretheoremstyle[
	headfont=\sffamily\bfseries\color{MidnightBlue},
	mdframed={style=mdbluebox},
	headpunct={\\[3pt]},
	postheadspace={0pt}
]{thmbluebox}

\mdfdefinestyle{mdredbox}{%
	linewidth=0.5pt,
	skipabove=12pt,
	frametitleaboveskip=5pt,
	frametitlebelowskip=0pt,
	skipbelow=2pt,
	frametitlefont=\bfseries,
	innertopmargin=4pt,
	innerbottommargin=8pt,
	nobreak=false,
	linecolor=RawSienna,
	backgroundcolor=Salmon!5,
}
\declaretheoremstyle[
	headfont=\bfseries\color{RawSienna},
	mdframed={style=mdredbox},
	headpunct={\\[3pt]},
	postheadspace={0pt},
]{thmredbox}

\declaretheorem[%
style=thmbluebox,name=Theorem,numberwithin=section]{thm}
\declaretheorem[style=thmbluebox,name=Lemma,sibling=thm]{lem}
\declaretheorem[style=thmbluebox,name=Proposition,sibling=thm]{prop}
\declaretheorem[style=thmbluebox,name=Corollary,sibling=thm]{coro}
\declaretheorem[style=thmredbox,name=Example,sibling=thm]{eg}

\mdfdefinestyle{mdgreenbox}{%
	roundcorner = 10pt,
	linewidth=1pt,
	skipabove=12pt,
	innerbottommargin=9pt,
	skipbelow=2pt,
	nobreak=true,
	linecolor=ForestGreen,
	backgroundcolor=ForestGreen!5,
}

\declaretheoremstyle[
	headfont=\bfseries\sffamily\color{ForestGreen!70!black},
	bodyfont=\normalfont,
	spaceabove=2pt,
	spacebelow=1pt,
	mdframed={style=mdgreenbox},
	headpunct={ --- },
]{thmgreenbox}

\declaretheorem[style=thmgreenbox,name=Definition,sibling=thm]{defn}

\mdfdefinestyle{mdgreenboxsq}{%
	linewidth=1pt,
	skipabove=12pt,
	innerbottommargin=9pt,
	skipbelow=2pt,
	nobreak=true,
	linecolor=ForestGreen,
	backgroundcolor=ForestGreen!5,
}
\declaretheoremstyle[
	headfont=\bfseries\sffamily\color{ForestGreen!70!black},
	bodyfont=\normalfont,
	spaceabove=2pt,
	spacebelow=1pt,
	mdframed={style=mdgreenboxsq},
	headpunct={},
]{thmgreenboxsq}
\declaretheoremstyle[
	headfont=\bfseries\sffamily\color{ForestGreen!70!black},
	bodyfont=\normalfont,
	spaceabove=2pt,
	spacebelow=1pt,
	mdframed={style=mdgreenboxsq},
	headpunct={},
]{thmgreenboxsq*}

\mdfdefinestyle{mdblackbox}{%
	skipabove=8pt,
	linewidth=3pt,
	rightline=false,
	leftline=true,
	topline=false,
	bottomline=false,
	linecolor=black,
	backgroundcolor=RedViolet!5!gray!5,
}
\declaretheoremstyle[
	headfont=\bfseries,
	bodyfont=\normalfont\small,
	spaceabove=0pt,
	spacebelow=0pt,
	mdframed={style=mdblackbox}
]{thmblackbox}

\theoremstyle{plain}
\declaretheorem[name=Question,sibling=thm,style=thmblackbox]{ques}
\declaretheorem[name=Remark,sibling=thm,style=thmgreenboxsq]{remark}
\declaretheorem[name=Remark,sibling=thm,style=thmgreenboxsq*]{remark*}
\newtheorem{ass}[thm]{Assumptions}

\theoremstyle{definition}
\newtheorem*{problem}{Problem}
\newtheorem{claim}[thm]{Claim}
\theoremstyle{remark}
\newtheorem*{case}{Case}
\newtheorem*{notation}{Notation}
\newtheorem*{note}{Note}
\newtheorem*{motivation}{Motivation}
\newtheorem*{intuition}{Intuition}
\newtheorem*{conjecture}{Conjecture}

% Make section starts with 1 for report type
%\renewcommand\thesection{\arabic{section}}

% End example and intermezzo environments with a small diamond (just like proof
% environments end with a small square)
\usepackage{etoolbox}
\AtEndEnvironment{vb}{\null\hfill$\diamond$}%
\AtEndEnvironment{intermezzo}{\null\hfill$\diamond$}%
% \AtEndEnvironment{opmerking}{\null\hfill$\diamond$}%

% Fix some spacing
% http://tex.stackexchange.com/questions/22119/how-can-i-change-the-spacing-before-theorems-with-amsthm
\makeatletter
\def\thm@space@setup{%
  \thm@preskip=\parskip \thm@postskip=0pt
}

% Fix some stuff
% %http://tex.stackexchange.com/questions/76273/multiple-pdfs-with-page-group-included-in-a-single-page-warning
\pdfsuppresswarningpagegroup=1


% My name
\author{Jaden Wang}



\begin{document}
\centerline {\textsf{\textbf{\LARGE{Homework 8}}}}
\centerline {Jaden Wang}
\vspace{.15in}
\begin{problem}[2.1.3]
Let $ s$ be a corner of the square and let  $ f$ be the diffeomorphism that maps a neighborhood  $ U$ of  $ s$ to  $ H^2$, with the boundary(two edges) mapped to the boundary of $ H^2$ which is the real line. Then $ v_1, v_2$ depicted in Figure 2-4 are two smooth half curves based at $ s$ so they are in  $ T_s S$. They are also clearly linearly independent. However, since they are along the boundary, they are mapped into the real line, becoming linearly dependent. Thus  $ df_s$ is singular so $ f$ cannot be a diffeomorphism, a contradiction. Thus  $ S$ is not a manifold with boundary.
\end{problem}
\begin{problem}[2.1.10]
Let $ h: H^{n} \to \rr$ be the height function which is clearly smooth as it is linear. It is easy to see that $ h(z) = 0$ iff $ z \in \partial H^{n}$ which is the real line. Now since $ X$ is a manifold with boundary, take  $ x \in \partial X$, take a chart $(U, \phi)$ around $ x$. Then define  $ f= h \circ \phi $. Since $ \phi$ maps boundary to boundary, we see that for any boundary point $ z \in \partial U$, $ \phi(z) \in \partial H^{n}$, and $ f(z) = h ( \phi(z)) =0$. Moreover, if $ f(z) = h(\phi(z)) = 0$, then $ \phi(z)$ must be a boundary point in $ H^{n}$, so $ z = \phi ^{-1} \phi(z) \in \partial U$.

If $ z \in \partial U$, then the outward normal $ n(z)$. It corresponds to a vector $ v=(0,\ldots,0,-a)$ where $ a >0$ pointing straight down in  $ H^{n}$, \emph{i.e.} then the curve is $ vt$. So
\begin{align*}
	df_z(n(z)) &= (f \circ \phi ^{-1} (vt))'(0) \\
	&= (h (vt))'(0) \\
	&= (t(-a))'(0) \\
	&= -a <0.
\end{align*}
\end{problem}

\begin{problem}[2.2.3]
You can simply rotate the solid torus by an angle that isn't a multiple of $ 2 \pi$. The proof fails at the fact that the ray from $ f(x)$ to $ x$ can hit the boundary more than once as the solid torus isn't convex, so $ g(x)$ isn't well-defined. Even if we arbitrarily pick a point of intersection, some boundary points can be mapped to other boundary points, so  $ g(x)$ may not be identity on the boundary so we cannot apply the no-retract theorem. Moreover, $ g(x)$ won't be continuous.
\end{problem}

\begin{problem}[2.2.7]
Assume $ A$ is nonsingular with nonnegative entries. Consider the map $ f: S^{n-1} \to S^{n-1}, v \mapsto Av / \norm{ Av} $. Suppose $ v$ is in $ Q$ (\emph{i.e.} unit vector with nonnegative entries), then clearly $ Av$ with each entry being the sum of nonnegative numbers would remain in the first quadrant. Hence  $ Av / \norm{ Av} $ is in $ Q$. So $ f|_Q : Q \to Q$. Since $ Q \cong B^{n-1}$, we obtain a continuous map $ g: B^{n-1} \to B^{n-1}$. By Brouwer Fixed-Point Theorem for continuous maps, $ g$ has a fix point: there exists an $ x \in B^{n-1}$ s.t.\  $g(x) = x$. The homemorphism yields that there exists a $ v \in Q$ s.t.\ $ f|_Q(v) = Av / \norm{ Av} = v$. So $ Av = \norm{ Av} v $. By positive definiteness of norm, $ A$ has an nonnegative eigenvalue  $ \norm{ Av} $.
\end{problem}

\begin{problem}[2.3.4]
Consider $ F:X \times \rr^{n} \to \rr^{n}, (x,a)\mapsto x+a$. I claim that $ F$ is transversal to  $ Y$. First notice that  $ F$ is linear so $ dF$ is just  $ F$. Furthermore notice that  $ dF$ is surjective since given any  $ v \in \rr^{n}$, just choose $ x=0$ and  $ a=v$ and we have  $ dF(0,v) = v$. This yields $ dF_p(T_p (X \times \rr^{n})) = \rr^{n}$. Thus we have
\begin{align*}
	dF_p(T_p(X \times \rr^{n})) + T_{F(p)}Y = T_{F(p)} \rr^{n} =\rr^{n}.
\end{align*}

By the Transversality Theorem, for almost every $ a \in \rr^{n}$, $ f_a: X \times \{a\} \to x+a$ is transversal to $ Y$. Since  $ f_a(X \times \{a\} ) = X+a$, we show that $ X+a$ is transversal to  $ Y$.
\end{problem}

\begin{problem}[2.3.5]
Given $ \epsilon>0$. Consider the inclusion map $ i: X \to Y$. It is an embedding of $ X$. By the corollary Tubular Neighborhood Theorem, there exists a $ F: X \times B^{n} \to Y$ where $ n= \dim Y$  s.t.\ $ F_0 = i(x)$ and  $ F$ is a submersion,  \emph{i.e.} $ F \pitchfork Z$. Then by Transversality Theorem, for almost all $ s \in B^{n}$, $F_s \pitchfork Z$. Since $ X$ is compact, by the generalized stability theorem of Exercise 1.6.11, there exists an $ \epsilon_1>0$ s.t.\ $ F_s$ is also an embedding if $ |s|< \epsilon_1$. Moreover, since $ F$ is continuous, we can take the closure of $ B^{n}$ to compactify the domain, so $ F$ is uniformly continuous. That is, there exists a  $ \delta>0$ s.t.\ if $ |s|< \delta$, for every $ x \in X$ we have $|F(x,0)-F(x,s)| = |x-i_s(x)|< \epsilon$. Finally, we set $ \epsilon_2 = \min \{ \delta, \epsilon_1\}$ and choose any $ |s_1|< \epsilon_2$ that makes $ F_{s_1} \pitchfork Z$. This means that $ X_s$ is a manifold and is transversal to  $ Z$, but since  $ \dim X = \dim X_s$ and $ \dim X+ \dim  Z < \dim Y$, we see that $ X_s$ and  $ Z$ can be transversal only if they do not intersect. Hence  $ X_s$ is the deformation of  $ X$ that doesn't intersect  $ Z$.
\end{problem}

\begin{problem}[2.3.9]
Consider $ F: \rr^{k} \times  \rr^{k} \to \rr^{k}, (x,a) \mapsto  \left(  \frac{\partial f}{\partial x_1} +a_1, \ldots, \frac{\partial f}{\partial x_k} +a_k  \right) $. I claim that $ F$ is a submersion. Notice $ F$ is linear in the second argument so $ dF$ with a fixed  $ x$ is just  $ F$ with a fixed  $ x$. Given $ v \in \rr^{k}$, we can set $ x=0$ and  $ a=v$ so that  $ dF(0,v) = 0+v=v$. So $ F$ is a submersion. Hence  $ F \pitchfork \{0\}$. Then by Transversality Theorem, $ F_a: \rr^{k} \times \{a\} \to \rr^{k} $ is transversal to $ \{0\} $ as well. Denote the hessian of $ f_a$ as  $ H$. Notice that $ df_a = F_a$ so $ H = dF_a$. Since $ F_a$ is transversal to  $ \{0\} $, $ dF_a$ is surjective so does $ H$. Then $ H$ as a $ k\times k$ matrix must be invertible and hence is nondegenerate. Therefore, any critical point of $ f_a$ must be nondegenerate so $ f_a$ is a Morse function.
\end{problem}
\end{document}
