\documentclass[12pt]{article}
\newcommand{\alert}[1]{{\bf \color{red} [Alert:] #1}}
\newcommand{\todo}[1]{{\bf \color{orange} [TODO:] #1}}
\newcommand{\real}[1][]{\mathbb{R}^{#1}}
\newcommand{\myeqn}[1]{(\ref{#1})}
\newcommand{\myex}[1]{Example \ref{#1}}
\newcommand{\defeq}{\stackrel{\mathrm{def}}{=}}
\newcommand{\parder}[2]{\frac{\partial #1}{\partial #2}}
\newcommand{\Lie}[3][]{\mathsf{L}_{#3}^{#1} #2}
\newcommand{\LieA}[1]{\mathsf{Lie}(#1)}
\newcommand{\lieder}[2]{\mathcal{L}_{#2} #1}
\renewcommand{\t}{^{\mbox{\tiny\sf T}}}
\newcommand{\trans}{^{\mbox{\tiny\sf T}}}
\newcommand{\markup}[1]{\{\textbf{#1}\}}
\newcommand{\msub}[1]{_\mathrm{#1}}
\newcommand{\msup}[1]{^\mathrm{#1}}
\newcommand{\inv}[1]{#1^{-1}}
\newcommand{\pinv}[1]{{#1}^{+}}
\newcommand{\myfracA}[2]{\displaystyle{\frac{#1}{#2}}}
\newcommand{\myfracB}[2]{{#1}/{#2}}
\newcommand{\mydiffA}[1]{\dot{#1}}
\newcommand{\mydiffB}[2]{\myfracA{\mathrm{d}{#1}}{\mathrm{d}{#2}}}
\newcommand{\ball}[2]{\mathcal{B}_{#1}\left(#2\right)}
\newcommand{\acos}[1]{\cos^{-1}\left(#1\right)}
\newcommand{\asin}[1]{\sin^{-1}\left(#1\right)}
\newcommand{\mani}{\mathcal{M}}
\newcommand{\tang}[2]{\mathsf{T}_{#1} #2}
\newcommand{\LieB}[2]{[ #1, #2 ]}
\newcommand{\LieBad}[3][]{\mathsf{ad}_{#2}^{#1} #3}
\newcommand{\ReachVT}{\mathcal{R}^V_T}
\newcommand{\ReachVt}{\mathcal{R}^V_t}
\newcommand{\ReachVTe}{\mathcal{R}^V_{\le T}}
\newcommand{\ReachT}{\mathcal{R}_T}
\newcommand{\Reacht}{\mathcal{R}_t}
\newcommand{\ReachTe}{\mathcal{R}_{\le T}}
\newcommand{\accLA}[1]{\mathsf{Lie}(#1)}
\newcommand{\accD}{\Delta_{\mathcal{F}}}
\newcommand{\accSA}{\mathsf{Lie}(\mathcal{G},f)}
\newcommand{\accDS}{\Delta_{\mathcal{G}}}
\newcommand{\eval}[3]{\mathsf{Ev}^{#2}_{#1}\left( #3 \right)}
\newcommand{\stlc}{\textsc{stlc}}
\newcommand{\clf}{\textsc{clf}}
\newcommand{\jqlf}{\textsc{jqlf}}
\newcommand{\dlas}{\textsc{dlas}}
\newcommand{\Ad}[2]{\mathsf{Ad}_{#1} #2}
\newcommand{\xe}{\ensuremath{x_e}}
\newcommand{\lebg}[1]{\mathcal{L}_{#1}}
\newcommand{\lebgx}[1]{\mathcal{L}_{#1 \mathrm{e}}}
\newcommand{\dom}{D}
\newcommand{\domT}{[t_0,\infty) \times D}
\newcommand{\rarrow}{\rightarrow}
\renewcommand{\d}{\mathrm{d}}
\renewcommand{\Re}{\mathbb{R}}
\newcommand{\C}{\mathrm{C}}

\newcommand{\QED}{{\unskip\nobreak\hfil\penalty50\hskip2em\vadjust{}
		\nobreak\hfil$\Box$\parfillskip=0pt\finalhyphendemerits=0\par}\vspace{0.1cm}}
\newcommand{\eoEx}{{\unskip\nobreak\hfil\penalty50\hskip0em\vadjust{}
		\nobreak\hfil$\Large\Diamond$\parfillskip=0pt\finalhyphendemerits=0\par}\vspace{0.1cm}}

\newcommand{\sgn}{\ensuremath{\operatorname{sgn}}}
\newcommand{\sat}{\ensuremath{\operatorname{sat}}}

\newcommand{\half}{\frac{1}{2}}
\newcommand{\shalf}{\mbox{$\frac{1}{2}$}}
\newcommand{\marcom}[1]{\marginpar{\footnotesize #1}}
\newcommand{\der}{\mathrm{D}}
\newcommand{\e}{\mathrm{e}}
\newcommand{\dt}{\mathrm{d}t}

\newcommand{\cA}{\ensuremath{\mathcal{A}}}
\newcommand{\cB}{\ensuremath{\mathcal{B}}}
\newcommand{\cG}{\ensuremath{\mathcal{G}}}
\newcommand{\cK}{\ensuremath{\mathcal{K}}}
\newcommand{\cW}{\ensuremath{\mathcal{W}}}
\newcommand{\cZ}{\ensuremath{\mathcal{Z}}}
\newcommand{\cS}{\ensuremath{\mathcal{S}}}
\newcommand{\cD}{\ensuremath{\mathcal{D}}}
\newcommand{\cP}{\ensuremath{\mathcal{P}}}
\newcommand{\cV}{\ensuremath{\mathcal{V}}}
\newcommand{\cL}{\ensuremath{\mathcal{L}}}
\newcommand{\cN}{\ensuremath{\mathcal{N}}}
\newcommand{\cI}{\ensuremath{\mathcal{I}}}
\newcommand{\cR}{\ensuremath{\mathcal{R}}}
\newcommand{\cM}{\ensuremath{\mathcal{M}}}
\newcommand{\cC}{\ensuremath{\mathcal{C}}}
\newcommand{\cF}{\ensuremath{\mathcal{F}}}
\newcommand{\cH}{\ensuremath{\mathcal{H}}}
\newcommand{\cO}{\ensuremath{\mathcal{O}}}
\newcommand{\cX}{\ensuremath{\mathcal{X}}}
\newcommand{\cY}{\ensuremath{\mathcal{Y}}}
\newcommand{\Ci}{\ensuremath{\mathcal{C}^\infty}}
\newcommand{\ISS}{\textsc{iss}}
\newcommand{\LISS}{\textsc{liss}}
\newcommand{\GAS}{\textsc{gas}}
\newcommand{\GS}{\textsc{gs}}
\newcommand{\LES}{\textsc{les}}
\newcommand{\GUAS}{\textsc{guas}}
\newcommand{\BIBO}{\textsc{bibo}}
\newcommand{\spec}{\ensuremath{\operatorname{spec}}}
\newcommand{\spn}{\ensuremath{\operatorname{span}}}
\renewcommand{\i}{\mathrm{i\,}}

\renewcommand{\implies}{\Rightarrow}

\renewcommand{\theenumi}{$\roman{enumi})$}
\renewcommand{\labelenumi}{\theenumi}

\font\ptmten=zptmcmrm scaled 1200
\newcommand{\w}{\mbox{{\ptmten w}}}
\newcommand{\z}{\mbox{{\ptmten z}}}
\renewcommand{\Re}{\mathbb{R}}

\newcommand{\cl}{\operatorname{cl}}
\newcommand{\intr}{\operatorname{int}}
\newcommand{\rank}{\operatorname{rank}}
\newcommand{\co}{\operatorname{co}}
\newcommand{\aff}{\operatorname{aff}}

\theoremstyle{plain}
\newtheorem{theorem}{Theorem}[chapter]
\newtheorem{claim}[theorem]{Claim}
\newtheorem{corollary}[theorem]{Corollary}
\newtheorem{prop}[theorem]{Proposition}
\newtheorem{fact}[theorem]{Fact}
\newtheorem{lemma}[theorem]{Lemma}

\newtheorem{remark}{Remark}[chapter]

\theoremstyle{definition}
\newtheorem{assume}[theorem]{Assumption}
\newtheorem{defn}[theorem]{Definition}
\newtheorem{problem}[theorem]{Problem}
\newtheorem{exercise}{Exercise}
\newtheorem{example}[theorem]{Example}


\begin{document}
\centerline {\textsf{\textbf{\LARGE{Homework 13}}}}
\centerline {Jaden Wang}
\vspace{.15in}
\begin{problem}[4.4.3]
\begin{align*}
	\int_{ a}^{ b} c^* \omega &= \int_{ a}^{ b} c^* df  \\ 
	&= \int_{ a}^{ b} d (c^* f)  \\
	&= \int_{ a}^{ b} d(f \circ c)  \\
	&= \int_{ a}^{ b} \frac{d(f \circ c)}{ dx} dx \\
	&= f \circ c(b) - f \circ c(a) &&\text{ FTC}  \\
	&= f(q) - f(p) 
\end{align*}
\end{problem}

\begin{problem}[4.4.5]
Let $ \gamma(t) = ( \gamma_1(t), \gamma_2(t), \gamma_3(t))$ where we view $ S^{1} \cong I / \partial I$. Since $ \omega$ is a 1-form, we have $ \omega = \sum_{ i= 1}^{ k} f_i d x_i$, where $ f_i: X \to \rr$. Therefore,
\begin{align*}
	\gamma^* \omega &= \sum_{ i= 1}^{ k} (f_i \circ \gamma) \gamma^* d x_i\\
	&= \sum_{ i= 1}^{ k} (f_i \circ \gamma) d(x_i \circ \gamma) \\
	&= \sum_{ i= 1}^{ k} (f_i \circ \gamma) d \gamma_i \\
	&= \sum_{ i= 1}^{ k} (f_i \circ \gamma) \frac{d \gamma_i}{ dt} dt 
\end{align*}
Therefore,
\begin{align*}
	\oint_{ \gamma} \omega &= \int_{S^{1}} \sum_{ i= 1}^{ k} (f_i \circ \gamma(t)) \frac{d \gamma_i}{ dt} dt \\
&= \sum_{ i= 1}^{ k} \int_{S^{1}} (f_i \circ \gamma(t)) \frac{d \gamma_i}{ dt} dt 
\end{align*}
\end{problem}
\begin{problem}[4.4.7]
By Problem 3, since the endpoints $ p$ of a closed curve are the same, we immediately obtain
\begin{align*}
	\oint_{ \gamma} \omega &= f(p) - f(p) = 0 
\end{align*}
\end{problem}

\begin{problem}[4.4.8]
~\begin{enumerate}[label=(\alph*)]
	\item First note that under the polar coordinates, $ x = r \cos \theta, y = r \sin \theta$, and $ r^2 = x^2 + y^2$. Thus $ dx = \frac{\partial x}{\partial r} dr + \frac{\partial x}{\partial \theta} d \theta  = \cos \theta dr - r \sin \theta d \theta $ and $ dy = \frac{\partial y}{\partial r} dr + \frac{\partial y}{\partial \theta} d \theta = \sin \theta dr + r \cos \theta d \theta $. Thus we have
\begin{align*}
	\int_{C_a} \omega &= \int_{C_a} \frac{-y}{ x^2+y^2} dx + \int_{C_a} \frac{x}{ x^2+y^2} dy\\
	&= \int_{a}^{a} \frac{- \sin \theta \cos \theta}{ r} dr + \int_{0}^{2 \pi} (\sin \theta)^2 d \theta + \int_{ a}^{ a} \frac{ \cos \theta \sin \theta }{ r} dr + \int_{ 0}^{ 2\pi} (\cos \theta)^2 d \theta \\
	&= 0+ \int_{ 0}^{ 2 \pi} (\sin^2 \theta + \cos^2 \theta) d \theta  + 0  \\
	&= \theta|_{0}^{2\pi} \\
	&= 2 \pi
\end{align*}
\item Note that when $ x > 0$,  $ \frac{y}{x}$ is well-defined. Thus we have
	\begin{align*}
		d \left( \arctan \frac{y}{x}\right) &=  \frac{\partial \arctan \frac{y}{x}}{\partial x} dx + \frac{\partial \arctan \frac{y}{x}}{\partial y} dy \\
		&= \frac{1}{1+\left( \frac{y}{x} \right)^2 } \left( - \frac{y}{x^2} \right) dx + \frac{1}{1+ \left( \frac{y}{x} \right) ^2} \frac{1}{x} dy\\
		&= \frac{-y}{ x^2+y^2} dx + \frac{x}{ x^2+y^2} dy 
	\end{align*}
\item Suppose $ w = \partial f$ for some $ f: \rr^2 - \{0\} \to \rr$, then by 7 its contour integral would be 0, but that contradicts with part a.
\end{enumerate}
\end{problem}

\begin{problem}[4.4.13]
Suppose $ S$ is the graph of $ G: \rr^2 \to \rr$, then $ S$ can be parameterized by $ h(x_1,x_2) = (x_1,x_2, G(x_1,x_2))$. Therefore,
\begin{align*}
	d x_3 = d G &= \frac{\partial G}{\partial x_1} dx_1 + \frac{\partial G}{\partial x_2} dx_2  .
\end{align*}
Then the 2-form becomes
\begin{align*}
	d A &= - n_1 \frac{\partial G}{\partial x_1} dx_1 \wedge dx_2 - n_2 \frac{\partial G}{\partial x_2} dx_1 \wedge dx_2  + n_3 dx_1 \wedge dx_2 \\
	    &= \left( -\frac{\partial G}{\partial x_1} n_1, -\frac{\partial G}{\partial x_2} n_2, n_3  \right) dx_1 \wedge dx_2=: \ve{v} dx_1 \wedge dx_2 && n_i\text{ are linearly independent} \\
	    &= \begin{pmatrix} - \frac{\partial G}{\partial x_1}&0&0\\0& - \frac{\partial G}{\partial x_2} &0\\ 0&0&1   \end{pmatrix} \begin{pmatrix} n_1\\n_2\\n_3 \end{pmatrix} dx_1 \wedge dx_2 =: D \ve{n} dx_1 \wedge dx_2\\
	    &= \norm{ D} \ve{n} dx_1 \wedge dx_2  && \ve{n}  \text{ unit vec} \\
	    &= \sqrt{ \left( \frac{\partial G}{\partial x_1} \right)^2 + \left( \frac{\partial G}{\partial x_2}  \right)^2 + 1 }\quad  \ve{n} dx_1 \wedge dx_2 \\
	    &= \norm{ \ve{v}} \ve{n} dx_1 \wedge dx_2
\end{align*}
which coincides with the other definition of $ dA$ after doting with $ \ve{n}$.
\end{problem}

\begin{problem}[4.5.1]
\begin{enumerate}[label=(\alph*)]
Denote the given form $ \omega$.
	\item
		\begin{align*}
			d(\omega)& = d(z^2 dx \wedge dy) + d(z^2 dx \wedge dz) + d(2y dx \wedge dz)\\
			&= d(z^2) \wedge dx \wedge dy + d(z^2) dx \wedge dz + d(2y) \wedge dx \wedge dz \\
			&= 2z dz \wedge dx \wedge dy + 2z dz \wedge dx \wedge dz + 2 dy \wedge dx \wedge dz \\
			&= (2z- 2) dx \wedge dy \wedge dz 
		\end{align*}
	\item 
		\begin{align*}
			d\omega &= 13 dx \wedge dx + 2y dy \wedge dy +d(xyz) dz \\
			&= 0+ 0+ (yzdx+ xzdy+xydz) \wedge dz \\
			&= yz dx \wedge dz + xz dy \wedge dz
		\end{align*}
	\item 
		\begin{align*}
			d(f dg) &= df \wedge d g + 0 \qquad \text{  product rule}  \\
			&= \left( \frac{\partial f}{\partial x} dx + \frac{\partial f}{\partial y} dy + \frac{\partial f}{\partial z} dz \right) \wedge  \left( \frac{\partial g}{\partial x} dx + \frac{\partial g}{\partial y} dy + \frac{\partial g}{\partial z} dz \right)\\
			&= \frac{\partial f}{\partial x} \frac{\partial g}{\partial y}  dx  \wedge dy  + \frac{\partial f}{\partial x} \frac{\partial g}{\partial z}  dx  \wedge dz + \frac{\partial f}{\partial y} \frac{\partial g}{\partial x}  dy  \wedge dx \\
			&\quad + \frac{\partial f}{\partial y} \frac{\partial g}{\partial z}  dy\wedge dz + \frac{\partial f}{\partial z} \frac{\partial g}{\partial x}  dz  \wedge dx+\frac{\partial f}{\partial z} \frac{\partial g}{\partial y}  dz  \wedge dy\\
			&= \left(\frac{\partial f}{\partial x} \frac{\partial g}{\partial y} -  \frac{\partial f}{\partial y} \frac{\partial g}{\partial x}   \right) dx \wedge dy + \left(\frac{\partial f}{\partial x} \frac{\partial g}{\partial z} -  \frac{\partial f}{\partial z} \frac{\partial g}{\partial x}   \right) dx \wedge dz + \left(\frac{\partial f}{\partial y} \frac{\partial g}{\partial z} -  \frac{\partial f}{\partial z} \frac{\partial g}{\partial y} \right) dy \wedge dz
		\end{align*}
	\item 
		\begin{align*}
			d \omega = 0+0+ 6y^2 dy \wedge dz \wedge dx + 0 = 6y^2 dy \wedge dz \wedge dx
		\end{align*}
\end{enumerate}
\end{problem}
\begin{problem}[4.5.2]
\begin{align*}
	\nabla \times F &= \frac{\partial F_2}{\partial x} - \frac{\partial F_1}{\partial y}  \\
	&= \frac{x^2+y^2-2x^2}{ (x^2+y^2)^2} + \frac{x^2+y^2-2y^2}{ (x^2+y^2)^2} \\
	&= 0 
\end{align*}
By Problem 4.4.8, we know that $ F$ is not the differential of any function, thus it cannot be written as the gradient of any function. Otherwise, the gradient expression immediately leads to the 1-form aka differential.
\end{problem}
\end{document}
