\documentclass[12pt]{article}
%Fall 2022
% Some basic packages
\usepackage{standalone}[subpreambles=true]
\usepackage[utf8]{inputenc}
\usepackage[T1]{fontenc}
\usepackage{textcomp}
\usepackage[english]{babel}
\usepackage{url}
\usepackage{graphicx}
%\usepackage{quiver}
\usepackage{float}
\usepackage{enumitem}
\usepackage{lmodern}
\usepackage{comment}
\usepackage{hyperref}
\usepackage[usenames,svgnames,dvipsnames]{xcolor}
\usepackage[margin=1in]{geometry}
\usepackage{pdfpages}

\pdfminorversion=7

% Don't indent paragraphs, leave some space between them
\usepackage{parskip}

% Hide page number when page is empty
\usepackage{emptypage}
\usepackage{subcaption}
\usepackage{multicol}
\usepackage[b]{esvect}

% Math stuff
\usepackage{amsmath, amsfonts, mathtools, amsthm, amssymb}
\usepackage{bbm}
\usepackage{stmaryrd}
\allowdisplaybreaks

% Fancy script capitals
\usepackage{mathrsfs}
\usepackage{cancel}
% Bold math
\usepackage{bm}
% Some shortcuts
\newcommand{\rr}{\ensuremath{\mathbb{R}}}
\newcommand{\zz}{\ensuremath{\mathbb{Z}}}
\newcommand{\qq}{\ensuremath{\mathbb{Q}}}
\newcommand{\nn}{\ensuremath{\mathbb{N}}}
\newcommand{\ff}{\ensuremath{\mathbb{F}}}
\newcommand{\cc}{\ensuremath{\mathbb{C}}}
\newcommand{\ee}{\ensuremath{\mathbb{E}}}
\newcommand{\hh}{\ensuremath{\mathbb{H}}}
\renewcommand\O{\ensuremath{\emptyset}}
\newcommand{\norm}[1]{{\left\lVert{#1}\right\rVert}}
\newcommand{\dbracket}[1]{{\left\llbracket{#1}\right\rrbracket}}
\newcommand{\ve}[1]{{\bm{#1}}}
\newcommand\allbold[1]{{\boldmath\textbf{#1}}}
\DeclareMathOperator{\lcm}{lcm}
\DeclareMathOperator{\im}{im}
\DeclareMathOperator{\coim}{coim}
\DeclareMathOperator{\dom}{dom}
\DeclareMathOperator{\tr}{tr}
\DeclareMathOperator{\rank}{rank}
\DeclareMathOperator*{\var}{Var}
\DeclareMathOperator*{\ev}{E}
\DeclareMathOperator{\dg}{deg}
\DeclareMathOperator{\aff}{aff}
\DeclareMathOperator{\conv}{conv}
\DeclareMathOperator{\inte}{int}
\DeclareMathOperator*{\argmin}{argmin}
\DeclareMathOperator*{\argmax}{argmax}
\DeclareMathOperator{\graph}{graph}
\DeclareMathOperator{\sgn}{sgn}
\DeclareMathOperator*{\Rep}{Rep}
\DeclareMathOperator{\Proj}{Proj}
\DeclareMathOperator{\mat}{mat}
\DeclareMathOperator{\diag}{diag}
\DeclareMathOperator{\aut}{Aut}
\DeclareMathOperator{\gal}{Gal}
\DeclareMathOperator{\inn}{Inn}
\DeclareMathOperator{\edm}{End}
\DeclareMathOperator{\Hom}{Hom}
\DeclareMathOperator{\ext}{Ext}
\DeclareMathOperator{\tor}{Tor}
\DeclareMathOperator{\Span}{Span}
\DeclareMathOperator{\Stab}{Stab}
\DeclareMathOperator{\cont}{cont}
\DeclareMathOperator{\Ann}{Ann}
\DeclareMathOperator{\Div}{div}
\DeclareMathOperator{\curl}{curl}
\DeclareMathOperator{\nat}{Nat}
\DeclareMathOperator{\gr}{Gr}
\DeclareMathOperator{\vect}{Vect}
\DeclareMathOperator{\id}{id}
\DeclareMathOperator{\Mod}{Mod}
\DeclareMathOperator{\sign}{sign}
\DeclareMathOperator{\Surf}{Surf}
\DeclareMathOperator{\fcone}{fcone}
\DeclareMathOperator{\Rot}{Rot}
\DeclareMathOperator{\grad}{grad}
\DeclareMathOperator{\atan2}{atan2}
\DeclareMathOperator{\Ric}{Ric}
\let\vec\relax
\DeclareMathOperator{\vec}{vec}
\let\Re\relax
\DeclareMathOperator{\Re}{Re}
\let\Im\relax
\DeclareMathOperator{\Im}{Im}
% Put x \to \infty below \lim
\let\svlim\lim\def\lim{\svlim\limits}

%wide hat
\usepackage{scalerel,stackengine}
\stackMath
\newcommand*\wh[1]{%
\savestack{\tmpbox}{\stretchto{%
  \scaleto{%
    \scalerel*[\widthof{\ensuremath{#1}}]{\kern-.6pt\bigwedge\kern-.6pt}%
    {\rule[-\textheight/2]{1ex}{\textheight}}%WIDTH-LIMITED BIG WEDGE
  }{\textheight}% 
}{0.5ex}}%
\stackon[1pt]{#1}{\tmpbox}%
}
\parskip 1ex

%Make implies and impliedby shorter
\let\implies\Rightarrow
\let\impliedby\Leftarrow
\let\iff\Leftrightarrow
\let\epsilon\varepsilon

% Add \contra symbol to denote contradiction
\usepackage{stmaryrd} % for \lightning
\newcommand\contra{\scalebox{1.5}{$\lightning$}}

% \let\phi\varphi

% Command for short corrections
% Usage: 1+1=\correct{3}{2}

\definecolor{correct}{HTML}{009900}
\newcommand\correct[2]{\ensuremath{\:}{\color{red}{#1}}\ensuremath{\to }{\color{correct}{#2}}\ensuremath{\:}}
\newcommand\green[1]{{\color{correct}{#1}}}

% horizontal rule
\newcommand\hr{
    \noindent\rule[0.5ex]{\linewidth}{0.5pt}
}

% hide parts
\newcommand\hide[1]{}

% si unitx
\usepackage{siunitx}
\sisetup{locale = FR}

%allows pmatrix to stretch
\makeatletter
\renewcommand*\env@matrix[1][\arraystretch]{%
  \edef\arraystretch{#1}%
  \hskip -\arraycolsep
  \let\@ifnextchar\new@ifnextchar
  \array{*\c@MaxMatrixCols c}}
\makeatother

\renewcommand{\arraystretch}{0.8}

\renewcommand{\baselinestretch}{1.5}

\usepackage{graphics}
\usepackage{epstopdf}

\RequirePackage{hyperref}
%%
%% Add support for color in order to color the hyperlinks.
%% 
\hypersetup{
  colorlinks = true,
  urlcolor = blue,
  citecolor = blue
}
%%fakesection Links
\hypersetup{
    colorlinks,
    linkcolor={red!50!black},
    citecolor={green!50!black},
    urlcolor={blue!80!black}
}
%customization of cleveref
\RequirePackage[capitalize,nameinlink]{cleveref}[0.19]

% Per SIAM Style Manual, "section" should be lowercase
\crefname{section}{section}{sections}
\crefname{subsection}{subsection}{subsections}
\Crefname{section}{Section}{Sections}
\Crefname{subsection}{Subsection}{Subsections}

% Per SIAM Style Manual, "Figure" should be spelled out in references
\Crefname{figure}{Figure}{Figures}

% Per SIAM Style Manual, don't say equation in front on an equation.
\crefformat{equation}{\textup{#2(#1)#3}}
\crefrangeformat{equation}{\textup{#3(#1)#4--#5(#2)#6}}
\crefmultiformat{equation}{\textup{#2(#1)#3}}{ and \textup{#2(#1)#3}}
{, \textup{#2(#1)#3}}{, and \textup{#2(#1)#3}}
\crefrangemultiformat{equation}{\textup{#3(#1)#4--#5(#2)#6}}%
{ and \textup{#3(#1)#4--#5(#2)#6}}{, \textup{#3(#1)#4--#5(#2)#6}}{, and \textup{#3(#1)#4--#5(#2)#6}}

% But spell it out at the beginning of a sentence.
\Crefformat{equation}{#2Equation~\textup{(#1)}#3}
\Crefrangeformat{equation}{Equations~\textup{#3(#1)#4--#5(#2)#6}}
\Crefmultiformat{equation}{Equations~\textup{#2(#1)#3}}{ and \textup{#2(#1)#3}}
{, \textup{#2(#1)#3}}{, and \textup{#2(#1)#3}}
\Crefrangemultiformat{equation}{Equations~\textup{#3(#1)#4--#5(#2)#6}}%
{ and \textup{#3(#1)#4--#5(#2)#6}}{, \textup{#3(#1)#4--#5(#2)#6}}{, and \textup{#3(#1)#4--#5(#2)#6}}

% Make number non-italic in any environment.
\crefdefaultlabelformat{#2\textup{#1}#3}

% Environments
\makeatother
% For box around Definition, Theorem, \ldots
%%fakesection Theorems
\usepackage{thmtools}
\usepackage[framemethod=TikZ]{mdframed}

\theoremstyle{definition}
\mdfdefinestyle{mdbluebox}{%
	roundcorner = 10pt,
	linewidth=1pt,
	skipabove=12pt,
	innerbottommargin=9pt,
	skipbelow=2pt,
	nobreak=true,
	linecolor=blue,
	backgroundcolor=TealBlue!5,
}
\declaretheoremstyle[
	headfont=\sffamily\bfseries\color{MidnightBlue},
	mdframed={style=mdbluebox},
	headpunct={\\[3pt]},
	postheadspace={0pt}
]{thmbluebox}

\mdfdefinestyle{mdredbox}{%
	linewidth=0.5pt,
	skipabove=12pt,
	frametitleaboveskip=5pt,
	frametitlebelowskip=0pt,
	skipbelow=2pt,
	frametitlefont=\bfseries,
	innertopmargin=4pt,
	innerbottommargin=8pt,
	nobreak=false,
	linecolor=RawSienna,
	backgroundcolor=Salmon!5,
}
\declaretheoremstyle[
	headfont=\bfseries\color{RawSienna},
	mdframed={style=mdredbox},
	headpunct={\\[3pt]},
	postheadspace={0pt},
]{thmredbox}

\declaretheorem[%
style=thmbluebox,name=Theorem,numberwithin=section]{thm}
\declaretheorem[style=thmbluebox,name=Lemma,sibling=thm]{lem}
\declaretheorem[style=thmbluebox,name=Proposition,sibling=thm]{prop}
\declaretheorem[style=thmbluebox,name=Corollary,sibling=thm]{coro}
\declaretheorem[style=thmredbox,name=Example,sibling=thm]{eg}

\mdfdefinestyle{mdgreenbox}{%
	roundcorner = 10pt,
	linewidth=1pt,
	skipabove=12pt,
	innerbottommargin=9pt,
	skipbelow=2pt,
	nobreak=true,
	linecolor=ForestGreen,
	backgroundcolor=ForestGreen!5,
}

\declaretheoremstyle[
	headfont=\bfseries\sffamily\color{ForestGreen!70!black},
	bodyfont=\normalfont,
	spaceabove=2pt,
	spacebelow=1pt,
	mdframed={style=mdgreenbox},
	headpunct={ --- },
]{thmgreenbox}

\declaretheorem[style=thmgreenbox,name=Definition,sibling=thm]{defn}

\mdfdefinestyle{mdgreenboxsq}{%
	linewidth=1pt,
	skipabove=12pt,
	innerbottommargin=9pt,
	skipbelow=2pt,
	nobreak=true,
	linecolor=ForestGreen,
	backgroundcolor=ForestGreen!5,
}
\declaretheoremstyle[
	headfont=\bfseries\sffamily\color{ForestGreen!70!black},
	bodyfont=\normalfont,
	spaceabove=2pt,
	spacebelow=1pt,
	mdframed={style=mdgreenboxsq},
	headpunct={},
]{thmgreenboxsq}
\declaretheoremstyle[
	headfont=\bfseries\sffamily\color{ForestGreen!70!black},
	bodyfont=\normalfont,
	spaceabove=2pt,
	spacebelow=1pt,
	mdframed={style=mdgreenboxsq},
	headpunct={},
]{thmgreenboxsq*}

\mdfdefinestyle{mdblackbox}{%
	skipabove=8pt,
	linewidth=3pt,
	rightline=false,
	leftline=true,
	topline=false,
	bottomline=false,
	linecolor=black,
	backgroundcolor=RedViolet!5!gray!5,
}
\declaretheoremstyle[
	headfont=\bfseries,
	bodyfont=\normalfont\small,
	spaceabove=0pt,
	spacebelow=0pt,
	mdframed={style=mdblackbox}
]{thmblackbox}

\theoremstyle{plain}
\declaretheorem[name=Question,sibling=thm,style=thmblackbox]{ques}
\declaretheorem[name=Remark,sibling=thm,style=thmgreenboxsq]{remark}
\declaretheorem[name=Remark,sibling=thm,style=thmgreenboxsq*]{remark*}
\newtheorem{ass}[thm]{Assumptions}

\theoremstyle{definition}
\newtheorem*{problem}{Problem}
\newtheorem{claim}[thm]{Claim}
\theoremstyle{remark}
\newtheorem*{case}{Case}
\newtheorem*{notation}{Notation}
\newtheorem*{note}{Note}
\newtheorem*{motivation}{Motivation}
\newtheorem*{intuition}{Intuition}
\newtheorem*{conjecture}{Conjecture}

% Make section starts with 1 for report type
%\renewcommand\thesection{\arabic{section}}

% End example and intermezzo environments with a small diamond (just like proof
% environments end with a small square)
\usepackage{etoolbox}
\AtEndEnvironment{vb}{\null\hfill$\diamond$}%
\AtEndEnvironment{intermezzo}{\null\hfill$\diamond$}%
% \AtEndEnvironment{opmerking}{\null\hfill$\diamond$}%

% Fix some spacing
% http://tex.stackexchange.com/questions/22119/how-can-i-change-the-spacing-before-theorems-with-amsthm
\makeatletter
\def\thm@space@setup{%
  \thm@preskip=\parskip \thm@postskip=0pt
}

% Fix some stuff
% %http://tex.stackexchange.com/questions/76273/multiple-pdfs-with-page-group-included-in-a-single-page-warning
\pdfsuppresswarningpagegroup=1


% My name
\author{Jaden Wang}



\begin{document}
\centerline {\textsf{\textbf{\LARGE{Homework 13}}}}
\centerline {Jaden Wang}
\vspace{.15in}
\begin{problem}[4.4.3]
\begin{align*}
	\int_{ a}^{ b} c^* \omega &= \int_{ a}^{ b} c^* df  \\ 
	&= \int_{ a}^{ b} d (c^* f)  \\
	&= \int_{ a}^{ b} d(f \circ c)  \\
	&= \int_{ a}^{ b} \frac{d(f \circ c)}{ dx} dx \\
	&= f \circ c(b) - f \circ c(a) &&\text{ FTC}  \\
	&= f(q) - f(p) 
\end{align*}
\end{problem}

\begin{problem}[4.4.5]
Let $ \gamma(t) = ( \gamma_1(t), \gamma_2(t), \gamma_3(t))$ where we view $ S^{1} \cong I / \partial I$. Since $ \omega$ is a 1-form, we have $ \omega = \sum_{ i= 1}^{ k} f_i d x_i$, where $ f_i: X \to \rr$. Therefore,
\begin{align*}
	\gamma^* \omega &= \sum_{ i= 1}^{ k} (f_i \circ \gamma) \gamma^* d x_i\\
	&= \sum_{ i= 1}^{ k} (f_i \circ \gamma) d(x_i \circ \gamma) \\
	&= \sum_{ i= 1}^{ k} (f_i \circ \gamma) d \gamma_i \\
	&= \sum_{ i= 1}^{ k} (f_i \circ \gamma) \frac{d \gamma_i}{ dt} dt 
\end{align*}
Therefore,
\begin{align*}
	\oint_{ \gamma} \omega &= \int_{S^{1}} \sum_{ i= 1}^{ k} (f_i \circ \gamma(t)) \frac{d \gamma_i}{ dt} dt \\
&= \sum_{ i= 1}^{ k} \int_{S^{1}} (f_i \circ \gamma(t)) \frac{d \gamma_i}{ dt} dt 
\end{align*}
\end{problem}
\begin{problem}[4.4.7]
By Problem 3, since the endpoints $ p$ of a closed curve are the same, we immediately obtain
\begin{align*}
	\oint_{ \gamma} \omega &= f(p) - f(p) = 0 
\end{align*}
\end{problem}

\begin{problem}[4.4.8]
~\begin{enumerate}[label=(\alph*)]
	\item First note that under the polar coordinates, $ x = r \cos \theta, y = r \sin \theta$, and $ r^2 = x^2 + y^2$. Thus $ dx = \frac{\partial x}{\partial r} dr + \frac{\partial x}{\partial \theta} d \theta  = \cos \theta dr - r \sin \theta d \theta $ and $ dy = \frac{\partial y}{\partial r} dr + \frac{\partial y}{\partial \theta} d \theta = \sin \theta dr + r \cos \theta d \theta $. Thus we have
\begin{align*}
	\int_{C_a} \omega &= \int_{C_a} \frac{-y}{ x^2+y^2} dx + \int_{C_a} \frac{x}{ x^2+y^2} dy\\
	&= \int_{a}^{a} \frac{- \sin \theta \cos \theta}{ r} dr + \int_{0}^{2 \pi} (\sin \theta)^2 d \theta + \int_{ a}^{ a} \frac{ \cos \theta \sin \theta }{ r} dr + \int_{ 0}^{ 2\pi} (\cos \theta)^2 d \theta \\
	&= 0+ \int_{ 0}^{ 2 \pi} (\sin^2 \theta + \cos^2 \theta) d \theta  + 0  \\
	&= \theta|_{0}^{2\pi} \\
	&= 2 \pi
\end{align*}
\item Note that when $ x > 0$,  $ \frac{y}{x}$ is well-defined. Thus we have
	\begin{align*}
		d \left( \arctan \frac{y}{x}\right) &=  \frac{\partial \arctan \frac{y}{x}}{\partial x} dx + \frac{\partial \arctan \frac{y}{x}}{\partial y} dy \\
		&= \frac{1}{1+\left( \frac{y}{x} \right)^2 } \left( - \frac{y}{x^2} \right) dx + \frac{1}{1+ \left( \frac{y}{x} \right) ^2} \frac{1}{x} dy\\
		&= \frac{-y}{ x^2+y^2} dx + \frac{x}{ x^2+y^2} dy 
	\end{align*}
\item Suppose $ w = \partial f$ for some $ f: \rr^2 - \{0\} \to \rr$, then by 7 its contour integral would be 0, but that contradicts with part a.
\end{enumerate}
\end{problem}

\begin{problem}[4.4.13]
Suppose $ S$ is the graph of $ G: \rr^2 \to \rr$, then $ S$ can be parameterized by $ h(x_1,x_2) = (x_1,x_2, G(x_1,x_2))$. Therefore,
\begin{align*}
	d x_3 = d G &= \frac{\partial G}{\partial x_1} dx_1 + \frac{\partial G}{\partial x_2} dx_2  .
\end{align*}
Then the 2-form becomes
\begin{align*}
	d A &= - n_1 \frac{\partial G}{\partial x_1} dx_1 \wedge dx_2 - n_2 \frac{\partial G}{\partial x_2} dx_1 \wedge dx_2  + n_3 dx_1 \wedge dx_2 \\
	    &= \left( -\frac{\partial G}{\partial x_1} n_1, -\frac{\partial G}{\partial x_2} n_2, n_3  \right) dx_1 \wedge dx_2=: \ve{v} dx_1 \wedge dx_2 && n_i\text{ are linearly independent} \\
	    &= \begin{pmatrix} - \frac{\partial G}{\partial x_1}&0&0\\0& - \frac{\partial G}{\partial x_2} &0\\ 0&0&1   \end{pmatrix} \begin{pmatrix} n_1\\n_2\\n_3 \end{pmatrix} dx_1 \wedge dx_2 =: D \ve{n} dx_1 \wedge dx_2\\
	    &= \norm{ D} \ve{n} dx_1 \wedge dx_2  && \ve{n}  \text{ unit vec} \\
	    &= \sqrt{ \left( \frac{\partial G}{\partial x_1} \right)^2 + \left( \frac{\partial G}{\partial x_2}  \right)^2 + 1 }\quad  \ve{n} dx_1 \wedge dx_2 \\
	    &= \norm{ \ve{v}} \ve{n} dx_1 \wedge dx_2
\end{align*}
which coincides with the other definition of $ dA$ after doting with $ \ve{n}$.
\end{problem}

\begin{problem}[4.5.1]
\begin{enumerate}[label=(\alph*)]
Denote the given form $ \omega$.
	\item
		\begin{align*}
			d(\omega)& = d(z^2 dx \wedge dy) + d(z^2 dx \wedge dz) + d(2y dx \wedge dz)\\
			&= d(z^2) \wedge dx \wedge dy + d(z^2) dx \wedge dz + d(2y) \wedge dx \wedge dz \\
			&= 2z dz \wedge dx \wedge dy + 2z dz \wedge dx \wedge dz + 2 dy \wedge dx \wedge dz \\
			&= (2z- 2) dx \wedge dy \wedge dz 
		\end{align*}
	\item 
		\begin{align*}
			d\omega &= 13 dx \wedge dx + 2y dy \wedge dy +d(xyz) dz \\
			&= 0+ 0+ (yzdx+ xzdy+xydz) \wedge dz \\
			&= yz dx \wedge dz + xz dy \wedge dz
		\end{align*}
	\item 
		\begin{align*}
			d(f dg) &= df \wedge d g + 0 \qquad \text{  product rule}  \\
			&= \left( \frac{\partial f}{\partial x} dx + \frac{\partial f}{\partial y} dy + \frac{\partial f}{\partial z} dz \right) \wedge  \left( \frac{\partial g}{\partial x} dx + \frac{\partial g}{\partial y} dy + \frac{\partial g}{\partial z} dz \right)\\
			&= \frac{\partial f}{\partial x} \frac{\partial g}{\partial y}  dx  \wedge dy  + \frac{\partial f}{\partial x} \frac{\partial g}{\partial z}  dx  \wedge dz + \frac{\partial f}{\partial y} \frac{\partial g}{\partial x}  dy  \wedge dx \\
			&\quad + \frac{\partial f}{\partial y} \frac{\partial g}{\partial z}  dy\wedge dz + \frac{\partial f}{\partial z} \frac{\partial g}{\partial x}  dz  \wedge dx+\frac{\partial f}{\partial z} \frac{\partial g}{\partial y}  dz  \wedge dy\\
			&= \left(\frac{\partial f}{\partial x} \frac{\partial g}{\partial y} -  \frac{\partial f}{\partial y} \frac{\partial g}{\partial x}   \right) dx \wedge dy + \left(\frac{\partial f}{\partial x} \frac{\partial g}{\partial z} -  \frac{\partial f}{\partial z} \frac{\partial g}{\partial x}   \right) dx \wedge dz + \left(\frac{\partial f}{\partial y} \frac{\partial g}{\partial z} -  \frac{\partial f}{\partial z} \frac{\partial g}{\partial y} \right) dy \wedge dz
		\end{align*}
	\item 
		\begin{align*}
			d \omega = 0+0+ 6y^2 dy \wedge dz \wedge dx + 0 = 6y^2 dy \wedge dz \wedge dx
		\end{align*}
\end{enumerate}
\end{problem}
\begin{problem}[4.5.2]
\begin{align*}
	\nabla \times F &= \frac{\partial F_2}{\partial x} - \frac{\partial F_1}{\partial y}  \\
	&= \frac{x^2+y^2-2x^2}{ (x^2+y^2)^2} + \frac{x^2+y^2-2y^2}{ (x^2+y^2)^2} \\
	&= 0 
\end{align*}
By Problem 4.4.8, we know that $ F$ is not the differential of any function, thus it cannot be written as the gradient of any function. Otherwise, the gradient expression immediately leads to the 1-form aka differential.
\end{problem}
\end{document}
