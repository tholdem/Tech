\documentclass[12pt]{article}
\newcommand{\alert}[1]{{\bf \color{red} [Alert:] #1}}
\newcommand{\todo}[1]{{\bf \color{orange} [TODO:] #1}}
\newcommand{\real}[1][]{\mathbb{R}^{#1}}
\newcommand{\myeqn}[1]{(\ref{#1})}
\newcommand{\myex}[1]{Example \ref{#1}}
\newcommand{\defeq}{\stackrel{\mathrm{def}}{=}}
\newcommand{\parder}[2]{\frac{\partial #1}{\partial #2}}
\newcommand{\Lie}[3][]{\mathsf{L}_{#3}^{#1} #2}
\newcommand{\LieA}[1]{\mathsf{Lie}(#1)}
\newcommand{\lieder}[2]{\mathcal{L}_{#2} #1}
\renewcommand{\t}{^{\mbox{\tiny\sf T}}}
\newcommand{\trans}{^{\mbox{\tiny\sf T}}}
\newcommand{\markup}[1]{\{\textbf{#1}\}}
\newcommand{\msub}[1]{_\mathrm{#1}}
\newcommand{\msup}[1]{^\mathrm{#1}}
\newcommand{\inv}[1]{#1^{-1}}
\newcommand{\pinv}[1]{{#1}^{+}}
\newcommand{\myfracA}[2]{\displaystyle{\frac{#1}{#2}}}
\newcommand{\myfracB}[2]{{#1}/{#2}}
\newcommand{\mydiffA}[1]{\dot{#1}}
\newcommand{\mydiffB}[2]{\myfracA{\mathrm{d}{#1}}{\mathrm{d}{#2}}}
\newcommand{\ball}[2]{\mathcal{B}_{#1}\left(#2\right)}
\newcommand{\acos}[1]{\cos^{-1}\left(#1\right)}
\newcommand{\asin}[1]{\sin^{-1}\left(#1\right)}
\newcommand{\mani}{\mathcal{M}}
\newcommand{\tang}[2]{\mathsf{T}_{#1} #2}
\newcommand{\LieB}[2]{[ #1, #2 ]}
\newcommand{\LieBad}[3][]{\mathsf{ad}_{#2}^{#1} #3}
\newcommand{\ReachVT}{\mathcal{R}^V_T}
\newcommand{\ReachVt}{\mathcal{R}^V_t}
\newcommand{\ReachVTe}{\mathcal{R}^V_{\le T}}
\newcommand{\ReachT}{\mathcal{R}_T}
\newcommand{\Reacht}{\mathcal{R}_t}
\newcommand{\ReachTe}{\mathcal{R}_{\le T}}
\newcommand{\accLA}[1]{\mathsf{Lie}(#1)}
\newcommand{\accD}{\Delta_{\mathcal{F}}}
\newcommand{\accSA}{\mathsf{Lie}(\mathcal{G},f)}
\newcommand{\accDS}{\Delta_{\mathcal{G}}}
\newcommand{\eval}[3]{\mathsf{Ev}^{#2}_{#1}\left( #3 \right)}
\newcommand{\stlc}{\textsc{stlc}}
\newcommand{\clf}{\textsc{clf}}
\newcommand{\jqlf}{\textsc{jqlf}}
\newcommand{\dlas}{\textsc{dlas}}
\newcommand{\Ad}[2]{\mathsf{Ad}_{#1} #2}
\newcommand{\xe}{\ensuremath{x_e}}
\newcommand{\lebg}[1]{\mathcal{L}_{#1}}
\newcommand{\lebgx}[1]{\mathcal{L}_{#1 \mathrm{e}}}
\newcommand{\dom}{D}
\newcommand{\domT}{[t_0,\infty) \times D}
\newcommand{\rarrow}{\rightarrow}
\renewcommand{\d}{\mathrm{d}}
\renewcommand{\Re}{\mathbb{R}}
\newcommand{\C}{\mathrm{C}}

\newcommand{\QED}{{\unskip\nobreak\hfil\penalty50\hskip2em\vadjust{}
		\nobreak\hfil$\Box$\parfillskip=0pt\finalhyphendemerits=0\par}\vspace{0.1cm}}
\newcommand{\eoEx}{{\unskip\nobreak\hfil\penalty50\hskip0em\vadjust{}
		\nobreak\hfil$\Large\Diamond$\parfillskip=0pt\finalhyphendemerits=0\par}\vspace{0.1cm}}

\newcommand{\sgn}{\ensuremath{\operatorname{sgn}}}
\newcommand{\sat}{\ensuremath{\operatorname{sat}}}

\newcommand{\half}{\frac{1}{2}}
\newcommand{\shalf}{\mbox{$\frac{1}{2}$}}
\newcommand{\marcom}[1]{\marginpar{\footnotesize #1}}
\newcommand{\der}{\mathrm{D}}
\newcommand{\e}{\mathrm{e}}
\newcommand{\dt}{\mathrm{d}t}

\newcommand{\cA}{\ensuremath{\mathcal{A}}}
\newcommand{\cB}{\ensuremath{\mathcal{B}}}
\newcommand{\cG}{\ensuremath{\mathcal{G}}}
\newcommand{\cK}{\ensuremath{\mathcal{K}}}
\newcommand{\cW}{\ensuremath{\mathcal{W}}}
\newcommand{\cZ}{\ensuremath{\mathcal{Z}}}
\newcommand{\cS}{\ensuremath{\mathcal{S}}}
\newcommand{\cD}{\ensuremath{\mathcal{D}}}
\newcommand{\cP}{\ensuremath{\mathcal{P}}}
\newcommand{\cV}{\ensuremath{\mathcal{V}}}
\newcommand{\cL}{\ensuremath{\mathcal{L}}}
\newcommand{\cN}{\ensuremath{\mathcal{N}}}
\newcommand{\cI}{\ensuremath{\mathcal{I}}}
\newcommand{\cR}{\ensuremath{\mathcal{R}}}
\newcommand{\cM}{\ensuremath{\mathcal{M}}}
\newcommand{\cC}{\ensuremath{\mathcal{C}}}
\newcommand{\cF}{\ensuremath{\mathcal{F}}}
\newcommand{\cH}{\ensuremath{\mathcal{H}}}
\newcommand{\cO}{\ensuremath{\mathcal{O}}}
\newcommand{\cX}{\ensuremath{\mathcal{X}}}
\newcommand{\cY}{\ensuremath{\mathcal{Y}}}
\newcommand{\Ci}{\ensuremath{\mathcal{C}^\infty}}
\newcommand{\ISS}{\textsc{iss}}
\newcommand{\LISS}{\textsc{liss}}
\newcommand{\GAS}{\textsc{gas}}
\newcommand{\GS}{\textsc{gs}}
\newcommand{\LES}{\textsc{les}}
\newcommand{\GUAS}{\textsc{guas}}
\newcommand{\BIBO}{\textsc{bibo}}
\newcommand{\spec}{\ensuremath{\operatorname{spec}}}
\newcommand{\spn}{\ensuremath{\operatorname{span}}}
\renewcommand{\i}{\mathrm{i\,}}

\renewcommand{\implies}{\Rightarrow}

\renewcommand{\theenumi}{$\roman{enumi})$}
\renewcommand{\labelenumi}{\theenumi}

\font\ptmten=zptmcmrm scaled 1200
\newcommand{\w}{\mbox{{\ptmten w}}}
\newcommand{\z}{\mbox{{\ptmten z}}}
\renewcommand{\Re}{\mathbb{R}}

\newcommand{\cl}{\operatorname{cl}}
\newcommand{\intr}{\operatorname{int}}
\newcommand{\rank}{\operatorname{rank}}
\newcommand{\co}{\operatorname{co}}
\newcommand{\aff}{\operatorname{aff}}

\theoremstyle{plain}
\newtheorem{theorem}{Theorem}[chapter]
\newtheorem{claim}[theorem]{Claim}
\newtheorem{corollary}[theorem]{Corollary}
\newtheorem{prop}[theorem]{Proposition}
\newtheorem{fact}[theorem]{Fact}
\newtheorem{lemma}[theorem]{Lemma}

\newtheorem{remark}{Remark}[chapter]

\theoremstyle{definition}
\newtheorem{assume}[theorem]{Assumption}
\newtheorem{defn}[theorem]{Definition}
\newtheorem{problem}[theorem]{Problem}
\newtheorem{exercise}{Exercise}
\newtheorem{example}[theorem]{Example}


\begin{document}
\centerline {\textsf{\textbf{\LARGE{Homework 7}}}}
\centerline {Jaden Wang}
\vspace{.15in}
\begin{problem}[1.5.4]
	Since $ X \pitchfork Z$, $ X \cap Z$ is a manifold.
	Let $ y \in X \cap Z$ and given $ [ \gamma] \in T_y(X \cap Z)$, we know that $ \gamma: [- \epsilon, \epsilon] \to X \cap Z$, $ \gamma(0)=y \in X \cap Z$. Since the base point of this class of smooth curves is in both $ X$ and  $ Z$, $ [ \gamma]$ is also an equivalence class of smooth curves in $ X$ and  $ Z$ as well so  $ [ \gamma] \in T_y X$ and $ T_y Z$, \emph{i.e.} their intersection. Given $ [c] \in T_y X \cap T_y Z$, then $ c: [- \epsilon, \epsilon] \to Y$ and $ c: [- \epsilon, \epsilon] \to Z$ and therefore $ c: [- \epsilon, \epsilon] \to Y \cap Z$. Since the base point $ y$ is in both  $ X$ and  $ C$ and therefore  $ y \in X \cap Z$, we have $[c] \in T_y (X \cap Z)$.
\end{problem}
\begin{problem}[7]
We need a fact from Exercise 1.5.5: the tangent space to the preimage of $ Z$ is the preimage of the tangent space of  $ Z$. The proof is self-evident boring set containment argument similar to 1.5.4, so we leave it as an exercise for the undergrad.

$ (\implies):$ Suppose $ f \pitchfork g^{-1}(W)$. Since $ g \pitchfork W$, $ g^{-1}(W)$ is a submanifold of $ Y$ and $ dg_{y}(T_y Y) + T_{g(y)} W = T_{g(y)} Z $. Moreover, $df_x (T_x X) + T_{f(x)}  g^{-1}(W) = T_{f(x)} Y$. Applying $ dg_{f(x)}$ to both sides yields
\begin{align*}
	dg_{f(x)} (df_x (T_x X)) + T_{f(x)} g^{-1}(W)) &= dg_{f(x)} (T_{_f(x)} Y) \\
	dg_{f(x)} (df_x (T_x X)) + dg_{f(x)} (T_{f(x)} g^{-1}(W)) &= dg_{f(x)} (T_{f(x)} Y) && \text{ linearity} \\
	dg_{f(x)} (df_x (T_x X)) + T_{g(f(x))} W &= dg_{f(x)} (T_{f(x)}Y)\\
	d(g \circ f)_x (T_x X)) + T_{g \circ f(x)} W &= dg_{f(x)} (T_{f(x)}Y) + T_{g \circ f(x)} W && + \text{ means span} \\
	d(g \circ f)_x (T_x X)) + T_{g \circ f(x)} W &= T_{g \circ f(x)} Z 
\end{align*}

$ (\impliedby):$ We wish to prove the contrapositive: suppose $ f \not \pitchfork g^{-1}(W)$, that is, there exists a vector $ [v] \in T_{f(x)}Y$ that is not in $ df_x(T_x X) + T_{f(x)}g^{-1}(W)$, then as we apply $ dg_{f(x)} ([v])$ which is an element of $ T_{g \circ f(x)}Z$, then I claim that it is not in $ d(g \circ f)_x (T_x X) + T_{g \circ f(x)} W$. We already know that $ dg_{f(x)}([v]) \not\in T_{g \circ f(x)} W$ because $[v] \not\in T_{f(x)} g^{-1}(W)$. It remains to check that it is not in the first term (since any component $dg_{f(x)} ([v])$ that is in  $ T_{g \circ f(x)}W$ comes from $ T_{f(x)}g^{-1}(W)$ so WLOG we just need to show the other component is not in the first term).

Suppose to the contrary that $ dg_{f(x)}([v]) \in d(g \circ f)_x(T_x X)$, then there exists a vector $ [y] \in df_x(T_x X)$ s.t.\ $ dg_{f(x)}([y]) = dg_{f(x)} ([v])$. That means $ [v] = [y]+ \ker dg_{f(x)}$. But clearly $ \ker dg_{f(x)} \subseteq T_{f(x)}g^{-1}(W)$ since $ T_{g \circ f(x)}W$ contains 0. Hence $ [v] \in df_x(T_x X) + T_{f(x)}g^{-1}(W)$, a contradiction. Therefore, $ dg_{f(x)}([v]) \not\in d(g \circ f)_x (T_x X) + T_{g \circ f(x)}W$ and thus $ g \circ f \not \pitchfork W$.
\end{problem}

\begin{problem}[1.5.11]
Let $ C$ be a closed set of  $ \rr^{k}$, then there exists a smooth function $ f:\rr^{k} \to \rr$ s.t.\ $ C = f^{-1}(0)$. Consider the function $ g: \rr^{k+1} \to \rr, (x_1,\ldots,x_{k},x_{k+1}) \mapsto f(x_1,\ldots,x^{k}) + x_{k+1}$.  I claim that $ 0$ is a regular value of  $ g$. Indeed, the Jacobian of this function is $ (df,1)$ which has full rank for any point. Therefore, $ M:= g^{-1}(0)$ is a submanifold of $ \rr^{k+1}$. Now consider
\begin{align*}
	M \cap \rr^{k} &= \{x \in \rr^{k+1}: x_{k+1} = 0, g(x)= f(x_1,\ldots,x_k)+x_{k+1}=0\}  \\
	&= \{x \in \rr^{k+1} : f(x_1,\ldots,x_k) = 0\}  \\
	&= \{f^{-1}(0)\}  \\
	&= C
\end{align*}
as desired.
\end{problem}

\begin{problem}[1.6.7]
We prove by induction. Denote $ I_n$ to be the $ n \times n$ identity matrix. For $ k=1$, we see that by choosing a basis of $ \rr^2$, the antipodal map  $ S^{1} \to S^{1}, x \mapsto -x$ can be described by $-I$. Then
\begin{align*}
	\begin{pmatrix} \cos(\pi t)& \sin(\pi t)\\ \sin(\pi t)& \cos(\pi t) \end{pmatrix} 
\end{align*}
is a homotopy between the identity map $ I$ and antipodal map $ -I$. 

Now assume that the identity $ I_k$ is homotopic to the antipodal map $ -I_k$ for odd $ k$ via some homotopy $ H: S^{k} \times I \to S^{k}$. Then for the next odd number $ k+2$,
\begin{align*}
	\begin{pmatrix} H_t&0&0\\0& \cos(\pi t)& \sin(\pi t) \\ 0&\sin(\pi t)& \cos(\pi t) \end{pmatrix} 
\end{align*}
is a homotopy between $ I_{k+2}$ and $ -I_{k+2}$. Hence we show that identity and antipodal maps are homotopic for all odd $ k$.
\end{problem}

\begin{problem}[1.6.8]
	First, we consider the case where the manifold is path-connected. Note that since any manifold is locally diffeomorphic to $ \rr^{n}$, it is locally path-connected since $ \rr^{n}$ is. This allows us to use path-connectedness and connectedness interchangably by Munkres Theorem 25.5. Since $ f$ is a diffeomorphism, it is also a local diffeomorphism and an embedding onto  $ N$. Therefore for any homotopy of $ f$, there exist  $ \epsilon_1, \epsilon_2>0$ that preserve these properties under perturbation. Let $ \epsilon= \min( \epsilon_1, \epsilon_2)$, then $ f_t:M \to N$ is a local diffeomorphism and an embedding for any $ t \in [0, \epsilon)$. Given any open set $ U$ of $ M$, let  $ U_m$ denotes the neighborhood of  $ m$ where  $ f_t$ is a local diffeomorphism. Then  $ U=\bigcup_{ m \in U} U \cap U_m$. Notice $ f_t$ is also a diffeomorphism restricted to  $ U \cap U_m$ which is open so $ f_t(U)$ is a union of open sets which is open. That is,  $ f_t$ is an open map. In particular,  $ f_t(M)$ is open. Since $ M$ is compact,  $ f_t(M)$ is also compact. Since  $ N$ is Hausdorff,  $ f_t(M)$ is closed so $ N-f_t(M)$ is open. This implies that $ f_t(M)$ and $ N- f_t(M)$ form a separation of  $ N$. But since $ N$ is connected, and clearly $ f_t(M)$ as an embedding is not empty so this forces $ f_t(M) = N$. That is, $ f_t(M)$is a surjective embedding so it is a diffeomorphism onto  $ N$.

	Now suppose  $ M$ has multiple path components. Then since $ f$ is a diffeomorphism, it must map path components of  $ M$ to path components of  $ N$ in a bijective way. Together with the fact that homotopy can never cross path components, we are allowed to consider one pair of components at a time so the above result applies. Since $ M$ is compact, the path components must be finite (or each component would require at least one open set in any covering and we would not have finite subcover). , it suffices to take the minimum of the $ \epsilon$ yielded from each path component from above and we are done.
\end{problem}

\begin{problem}[1.7.11]
Since $ a$ is a nondegenerate critical point of  $ f$, by Morse Lemma, there exists a local coordinate system $ x=(x_1,\ldots,x_n)^{T}$ s.t.\ 
\begin{align*}
f=f(a) + x^{T}Hx,
\end{align*}
where $ H$ is the nondegenerate Hessian of  $ f$ under this coordinate system. Notice that $ H$ is symmetric so it is also diagonalizable, \emph{i.e.} $ H=P ^{-1}D P = P^{T} D P $ where $ D$ has nonzero eigenvalues $ \lambda_1,\ldots, \lambda_n$ on the diagonal. Let $ \epsilon_i = \sgn (\lambda_i)$. Then
 \begin{align*}
	 H &= P^{T} \begin{pmatrix} \sqrt{ |\lambda_1|} &0& \cdots&0\\0& \sqrt{ |\lambda_2|} & \cdots & 0\\ \vdots &0 &\ddots & 0 \\ 0& 0&\cdots& \sqrt{ \lambda_n} \end{pmatrix} \begin{pmatrix} \epsilon_1 &0& \cdots&0\\0& \epsilon_2 & \cdots & 0\\ \vdots &0 &\ddots & 0 \\ 0& 0&\cdots& \epsilon_n \end{pmatrix} \begin{pmatrix} \sqrt{ |\lambda_1|} &0& \cdots&0\\0& \sqrt{ |\lambda_2|} & \cdots & 0\\ \vdots &0 &\ddots & 0 \\ 0& 0&\cdots& \sqrt{ \lambda_n} \end{pmatrix} P\\
	 &=: (P^{T} \Lambda) E (\Lambda P) \\
	 &= (\Lambda P)^{T} E (\Lambda P) \\
	 &=: \widetilde{ P}^{T} E \widetilde{ P}  
\end{align*}
Since each $ |\lambda_i|>0$, the diagonal matrix $ \Lambda$ and thus $ \widetilde{ P}$ is invertible. Let $ \widetilde{ x} = \widetilde{ P}  x$. Since $ \widetilde{ P}$ is a linear isomorphism, $ \widetilde{ x}$ is also a local coordinate. Under this local coordinate, we see that
\begin{align*}
	f = f(a) + \widetilde{ x}^{T} E \widetilde{ x}
\end{align*}
as desired.
\end{problem}

\begin{problem}[1.7.14]
We wish to use the lemma on page 42 of the book to show that the poles are the only critical points and they are nondegenerate (and therefore the height function $ h$ is a Morse function on $ S^{k-1}$). 

First we consider the south pole $ S$. View $ S^{k-1}$ as a unit sphere inside $ \rr^{k}$ with the south pole at the origin so $ S=0$. Consider the height function restricted to $ S^{k-1} - N$ which is $ h_S: S^{k-1}-N \to \rr$, and the inverse stereographic projection $ g_S: \rr^{k-1} \to S^{k-1}-N$. We see that $ h_S=(h_S \circ g_S) \circ g_S^{-1}$. Since $ g_S^{-1}(0)=0$ (maps the south pole to the origin), by lemma it suffices to consider the critical points of $ h_S \circ g_S: \rr^{k-1} \to \rr, x \mapsto \frac{4+ \norm{ x}^2 }{4- \norm{ x}^2}$. By the quotient rule, its derivative is
\begin{align*}
	d (h_S \circ g_S)(x) &= \frac{2\norm{ x}(4- \norm{ x}^2 ) - (4+ \norm{ x}^2 )(-2 \norm{ x} )}{(4- \norm{ x}^2 )^2 }\\
	&= \frac{16 \norm{ x} }{ (4- \norm{ x}^2 )^2} .
\end{align*}
By basic geometry, $ \norm{ x}^2 = 4 \iff x=(0,\ldots,0,2)=N$, the derivative is well-defined everywhere. It equals zero iff the numerator is zero iff $ \norm{ x} =0$ iff $ x=0=S$ by positive definiteness of the norm. Hence  $ S$ is the unique critical point. Again by the positive definiteness of the norm, the Jacobian of the derivative, \emph{i.e.} the Hessian of $ h_S \circ g$, must also be positive definite. Thus $ S$ is nondegenerate as well. Hence by lemma, $ S$ is the unique critical point of  $ h$ restricted to one chart and it is nondegenerate.

By a similar argument, we can view $ S^{k-1}$ as the unit sphere with normal pole at the origin. Consider $ h_N:S^{k-1} -S \to \rr$ and inverse stereographic projection $ g_N: \rr^{k-1} \to S^{k-1} -S$. The composition $ h_N \circ g_N: \rr^{k-1} \to \rr, x \mapsto \frac{4- \norm{ x}^2 }{4+ \norm{ x}^2}$. This time we have negative definition Hessian. So $ N$ is the unique critical point of  $ h$ restricted to the other chart and it is also nondegenerate. That is all the critical points of  $ h$ and both are nondegenerate so  $ h$ is a Morse function of  $ S^{k-1}$.
\end{problem}
\end{document}
