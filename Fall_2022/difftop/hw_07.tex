\documentclass[12pt]{article}
%Fall 2022
% Some basic packages
\usepackage{standalone}[subpreambles=true]
\usepackage[utf8]{inputenc}
\usepackage[T1]{fontenc}
\usepackage{textcomp}
\usepackage[english]{babel}
\usepackage{url}
\usepackage{graphicx}
%\usepackage{quiver}
\usepackage{float}
\usepackage{enumitem}
\usepackage{lmodern}
\usepackage{comment}
\usepackage{hyperref}
\usepackage[usenames,svgnames,dvipsnames]{xcolor}
\usepackage[margin=1in]{geometry}
\usepackage{pdfpages}

\pdfminorversion=7

% Don't indent paragraphs, leave some space between them
\usepackage{parskip}

% Hide page number when page is empty
\usepackage{emptypage}
\usepackage{subcaption}
\usepackage{multicol}
\usepackage[b]{esvect}

% Math stuff
\usepackage{amsmath, amsfonts, mathtools, amsthm, amssymb}
\usepackage{bbm}
\usepackage{stmaryrd}
\allowdisplaybreaks

% Fancy script capitals
\usepackage{mathrsfs}
\usepackage{cancel}
% Bold math
\usepackage{bm}
% Some shortcuts
\newcommand{\rr}{\ensuremath{\mathbb{R}}}
\newcommand{\zz}{\ensuremath{\mathbb{Z}}}
\newcommand{\qq}{\ensuremath{\mathbb{Q}}}
\newcommand{\nn}{\ensuremath{\mathbb{N}}}
\newcommand{\ff}{\ensuremath{\mathbb{F}}}
\newcommand{\cc}{\ensuremath{\mathbb{C}}}
\newcommand{\ee}{\ensuremath{\mathbb{E}}}
\newcommand{\hh}{\ensuremath{\mathbb{H}}}
\renewcommand\O{\ensuremath{\emptyset}}
\newcommand{\norm}[1]{{\left\lVert{#1}\right\rVert}}
\newcommand{\dbracket}[1]{{\left\llbracket{#1}\right\rrbracket}}
\newcommand{\ve}[1]{{\bm{#1}}}
\newcommand\allbold[1]{{\boldmath\textbf{#1}}}
\DeclareMathOperator{\lcm}{lcm}
\DeclareMathOperator{\im}{im}
\DeclareMathOperator{\coim}{coim}
\DeclareMathOperator{\dom}{dom}
\DeclareMathOperator{\tr}{tr}
\DeclareMathOperator{\rank}{rank}
\DeclareMathOperator*{\var}{Var}
\DeclareMathOperator*{\ev}{E}
\DeclareMathOperator{\dg}{deg}
\DeclareMathOperator{\aff}{aff}
\DeclareMathOperator{\conv}{conv}
\DeclareMathOperator{\inte}{int}
\DeclareMathOperator*{\argmin}{argmin}
\DeclareMathOperator*{\argmax}{argmax}
\DeclareMathOperator{\graph}{graph}
\DeclareMathOperator{\sgn}{sgn}
\DeclareMathOperator*{\Rep}{Rep}
\DeclareMathOperator{\Proj}{Proj}
\DeclareMathOperator{\mat}{mat}
\DeclareMathOperator{\diag}{diag}
\DeclareMathOperator{\aut}{Aut}
\DeclareMathOperator{\gal}{Gal}
\DeclareMathOperator{\inn}{Inn}
\DeclareMathOperator{\edm}{End}
\DeclareMathOperator{\Hom}{Hom}
\DeclareMathOperator{\ext}{Ext}
\DeclareMathOperator{\tor}{Tor}
\DeclareMathOperator{\Span}{Span}
\DeclareMathOperator{\Stab}{Stab}
\DeclareMathOperator{\cont}{cont}
\DeclareMathOperator{\Ann}{Ann}
\DeclareMathOperator{\Div}{div}
\DeclareMathOperator{\curl}{curl}
\DeclareMathOperator{\nat}{Nat}
\DeclareMathOperator{\gr}{Gr}
\DeclareMathOperator{\vect}{Vect}
\DeclareMathOperator{\id}{id}
\DeclareMathOperator{\Mod}{Mod}
\DeclareMathOperator{\sign}{sign}
\DeclareMathOperator{\Surf}{Surf}
\DeclareMathOperator{\fcone}{fcone}
\DeclareMathOperator{\Rot}{Rot}
\DeclareMathOperator{\grad}{grad}
\DeclareMathOperator{\atan2}{atan2}
\DeclareMathOperator{\Ric}{Ric}
\let\vec\relax
\DeclareMathOperator{\vec}{vec}
\let\Re\relax
\DeclareMathOperator{\Re}{Re}
\let\Im\relax
\DeclareMathOperator{\Im}{Im}
% Put x \to \infty below \lim
\let\svlim\lim\def\lim{\svlim\limits}

%wide hat
\usepackage{scalerel,stackengine}
\stackMath
\newcommand*\wh[1]{%
\savestack{\tmpbox}{\stretchto{%
  \scaleto{%
    \scalerel*[\widthof{\ensuremath{#1}}]{\kern-.6pt\bigwedge\kern-.6pt}%
    {\rule[-\textheight/2]{1ex}{\textheight}}%WIDTH-LIMITED BIG WEDGE
  }{\textheight}% 
}{0.5ex}}%
\stackon[1pt]{#1}{\tmpbox}%
}
\parskip 1ex

%Make implies and impliedby shorter
\let\implies\Rightarrow
\let\impliedby\Leftarrow
\let\iff\Leftrightarrow
\let\epsilon\varepsilon

% Add \contra symbol to denote contradiction
\usepackage{stmaryrd} % for \lightning
\newcommand\contra{\scalebox{1.5}{$\lightning$}}

% \let\phi\varphi

% Command for short corrections
% Usage: 1+1=\correct{3}{2}

\definecolor{correct}{HTML}{009900}
\newcommand\correct[2]{\ensuremath{\:}{\color{red}{#1}}\ensuremath{\to }{\color{correct}{#2}}\ensuremath{\:}}
\newcommand\green[1]{{\color{correct}{#1}}}

% horizontal rule
\newcommand\hr{
    \noindent\rule[0.5ex]{\linewidth}{0.5pt}
}

% hide parts
\newcommand\hide[1]{}

% si unitx
\usepackage{siunitx}
\sisetup{locale = FR}

%allows pmatrix to stretch
\makeatletter
\renewcommand*\env@matrix[1][\arraystretch]{%
  \edef\arraystretch{#1}%
  \hskip -\arraycolsep
  \let\@ifnextchar\new@ifnextchar
  \array{*\c@MaxMatrixCols c}}
\makeatother

\renewcommand{\arraystretch}{0.8}

\renewcommand{\baselinestretch}{1.5}

\usepackage{graphics}
\usepackage{epstopdf}

\RequirePackage{hyperref}
%%
%% Add support for color in order to color the hyperlinks.
%% 
\hypersetup{
  colorlinks = true,
  urlcolor = blue,
  citecolor = blue
}
%%fakesection Links
\hypersetup{
    colorlinks,
    linkcolor={red!50!black},
    citecolor={green!50!black},
    urlcolor={blue!80!black}
}
%customization of cleveref
\RequirePackage[capitalize,nameinlink]{cleveref}[0.19]

% Per SIAM Style Manual, "section" should be lowercase
\crefname{section}{section}{sections}
\crefname{subsection}{subsection}{subsections}
\Crefname{section}{Section}{Sections}
\Crefname{subsection}{Subsection}{Subsections}

% Per SIAM Style Manual, "Figure" should be spelled out in references
\Crefname{figure}{Figure}{Figures}

% Per SIAM Style Manual, don't say equation in front on an equation.
\crefformat{equation}{\textup{#2(#1)#3}}
\crefrangeformat{equation}{\textup{#3(#1)#4--#5(#2)#6}}
\crefmultiformat{equation}{\textup{#2(#1)#3}}{ and \textup{#2(#1)#3}}
{, \textup{#2(#1)#3}}{, and \textup{#2(#1)#3}}
\crefrangemultiformat{equation}{\textup{#3(#1)#4--#5(#2)#6}}%
{ and \textup{#3(#1)#4--#5(#2)#6}}{, \textup{#3(#1)#4--#5(#2)#6}}{, and \textup{#3(#1)#4--#5(#2)#6}}

% But spell it out at the beginning of a sentence.
\Crefformat{equation}{#2Equation~\textup{(#1)}#3}
\Crefrangeformat{equation}{Equations~\textup{#3(#1)#4--#5(#2)#6}}
\Crefmultiformat{equation}{Equations~\textup{#2(#1)#3}}{ and \textup{#2(#1)#3}}
{, \textup{#2(#1)#3}}{, and \textup{#2(#1)#3}}
\Crefrangemultiformat{equation}{Equations~\textup{#3(#1)#4--#5(#2)#6}}%
{ and \textup{#3(#1)#4--#5(#2)#6}}{, \textup{#3(#1)#4--#5(#2)#6}}{, and \textup{#3(#1)#4--#5(#2)#6}}

% Make number non-italic in any environment.
\crefdefaultlabelformat{#2\textup{#1}#3}

% Environments
\makeatother
% For box around Definition, Theorem, \ldots
%%fakesection Theorems
\usepackage{thmtools}
\usepackage[framemethod=TikZ]{mdframed}

\theoremstyle{definition}
\mdfdefinestyle{mdbluebox}{%
	roundcorner = 10pt,
	linewidth=1pt,
	skipabove=12pt,
	innerbottommargin=9pt,
	skipbelow=2pt,
	nobreak=true,
	linecolor=blue,
	backgroundcolor=TealBlue!5,
}
\declaretheoremstyle[
	headfont=\sffamily\bfseries\color{MidnightBlue},
	mdframed={style=mdbluebox},
	headpunct={\\[3pt]},
	postheadspace={0pt}
]{thmbluebox}

\mdfdefinestyle{mdredbox}{%
	linewidth=0.5pt,
	skipabove=12pt,
	frametitleaboveskip=5pt,
	frametitlebelowskip=0pt,
	skipbelow=2pt,
	frametitlefont=\bfseries,
	innertopmargin=4pt,
	innerbottommargin=8pt,
	nobreak=false,
	linecolor=RawSienna,
	backgroundcolor=Salmon!5,
}
\declaretheoremstyle[
	headfont=\bfseries\color{RawSienna},
	mdframed={style=mdredbox},
	headpunct={\\[3pt]},
	postheadspace={0pt},
]{thmredbox}

\declaretheorem[%
style=thmbluebox,name=Theorem,numberwithin=section]{thm}
\declaretheorem[style=thmbluebox,name=Lemma,sibling=thm]{lem}
\declaretheorem[style=thmbluebox,name=Proposition,sibling=thm]{prop}
\declaretheorem[style=thmbluebox,name=Corollary,sibling=thm]{coro}
\declaretheorem[style=thmredbox,name=Example,sibling=thm]{eg}

\mdfdefinestyle{mdgreenbox}{%
	roundcorner = 10pt,
	linewidth=1pt,
	skipabove=12pt,
	innerbottommargin=9pt,
	skipbelow=2pt,
	nobreak=true,
	linecolor=ForestGreen,
	backgroundcolor=ForestGreen!5,
}

\declaretheoremstyle[
	headfont=\bfseries\sffamily\color{ForestGreen!70!black},
	bodyfont=\normalfont,
	spaceabove=2pt,
	spacebelow=1pt,
	mdframed={style=mdgreenbox},
	headpunct={ --- },
]{thmgreenbox}

\declaretheorem[style=thmgreenbox,name=Definition,sibling=thm]{defn}

\mdfdefinestyle{mdgreenboxsq}{%
	linewidth=1pt,
	skipabove=12pt,
	innerbottommargin=9pt,
	skipbelow=2pt,
	nobreak=true,
	linecolor=ForestGreen,
	backgroundcolor=ForestGreen!5,
}
\declaretheoremstyle[
	headfont=\bfseries\sffamily\color{ForestGreen!70!black},
	bodyfont=\normalfont,
	spaceabove=2pt,
	spacebelow=1pt,
	mdframed={style=mdgreenboxsq},
	headpunct={},
]{thmgreenboxsq}
\declaretheoremstyle[
	headfont=\bfseries\sffamily\color{ForestGreen!70!black},
	bodyfont=\normalfont,
	spaceabove=2pt,
	spacebelow=1pt,
	mdframed={style=mdgreenboxsq},
	headpunct={},
]{thmgreenboxsq*}

\mdfdefinestyle{mdblackbox}{%
	skipabove=8pt,
	linewidth=3pt,
	rightline=false,
	leftline=true,
	topline=false,
	bottomline=false,
	linecolor=black,
	backgroundcolor=RedViolet!5!gray!5,
}
\declaretheoremstyle[
	headfont=\bfseries,
	bodyfont=\normalfont\small,
	spaceabove=0pt,
	spacebelow=0pt,
	mdframed={style=mdblackbox}
]{thmblackbox}

\theoremstyle{plain}
\declaretheorem[name=Question,sibling=thm,style=thmblackbox]{ques}
\declaretheorem[name=Remark,sibling=thm,style=thmgreenboxsq]{remark}
\declaretheorem[name=Remark,sibling=thm,style=thmgreenboxsq*]{remark*}
\newtheorem{ass}[thm]{Assumptions}

\theoremstyle{definition}
\newtheorem*{problem}{Problem}
\newtheorem{claim}[thm]{Claim}
\theoremstyle{remark}
\newtheorem*{case}{Case}
\newtheorem*{notation}{Notation}
\newtheorem*{note}{Note}
\newtheorem*{motivation}{Motivation}
\newtheorem*{intuition}{Intuition}
\newtheorem*{conjecture}{Conjecture}

% Make section starts with 1 for report type
%\renewcommand\thesection{\arabic{section}}

% End example and intermezzo environments with a small diamond (just like proof
% environments end with a small square)
\usepackage{etoolbox}
\AtEndEnvironment{vb}{\null\hfill$\diamond$}%
\AtEndEnvironment{intermezzo}{\null\hfill$\diamond$}%
% \AtEndEnvironment{opmerking}{\null\hfill$\diamond$}%

% Fix some spacing
% http://tex.stackexchange.com/questions/22119/how-can-i-change-the-spacing-before-theorems-with-amsthm
\makeatletter
\def\thm@space@setup{%
  \thm@preskip=\parskip \thm@postskip=0pt
}

% Fix some stuff
% %http://tex.stackexchange.com/questions/76273/multiple-pdfs-with-page-group-included-in-a-single-page-warning
\pdfsuppresswarningpagegroup=1


% My name
\author{Jaden Wang}



\begin{document}
\centerline {\textsf{\textbf{\LARGE{Homework 7}}}}
\centerline {Jaden Wang}
\vspace{.15in}
\begin{problem}[1.5.4]
	Since $ X \pitchfork Z$, $ X \cap Z$ is a manifold.
	Let $ y \in X \cap Z$ and given $ [ \gamma] \in T_y(X \cap Z)$, we know that $ \gamma: [- \epsilon, \epsilon] \to X \cap Z$, $ \gamma(0)=y \in X \cap Z$. Since the base point of this class of smooth curves is in both $ X$ and  $ Z$, $ [ \gamma]$ is also an equivalence class of smooth curves in $ X$ and  $ Z$ as well so  $ [ \gamma] \in T_y X$ and $ T_y Z$, \emph{i.e.} their intersection. Given $ [c] \in T_y X \cap T_y Z$, then $ c: [- \epsilon, \epsilon] \to Y$ and $ c: [- \epsilon, \epsilon] \to Z$ and therefore $ c: [- \epsilon, \epsilon] \to Y \cap Z$. Since the base point $ y$ is in both  $ X$ and  $ C$ and therefore  $ y \in X \cap Z$, we have $[c] \in T_y (X \cap Z)$.
\end{problem}
\begin{problem}[7]
We need a fact from Exercise 1.5.5: the tangent space to the preimage of $ Z$ is the preimage of the tangent space of  $ Z$. The proof is self-evident boring set containment argument similar to 1.5.4, so we leave it as an exercise for the undergrad.

$ (\implies):$ Suppose $ f \pitchfork g^{-1}(W)$. Since $ g \pitchfork W$, $ g^{-1}(W)$ is a submanifold of $ Y$ and $ dg_{y}(T_y Y) + T_{g(y)} W = T_{g(y)} Z $. Moreover, $df_x (T_x X) + T_{f(x)}  g^{-1}(W) = T_{f(x)} Y$. Applying $ dg_{f(x)}$ to both sides yields
\begin{align*}
	dg_{f(x)} (df_x (T_x X)) + T_{f(x)} g^{-1}(W)) &= dg_{f(x)} (T_{_f(x)} Y) \\
	dg_{f(x)} (df_x (T_x X)) + dg_{f(x)} (T_{f(x)} g^{-1}(W)) &= dg_{f(x)} (T_{f(x)} Y) && \text{ linearity} \\
	dg_{f(x)} (df_x (T_x X)) + T_{g(f(x))} W &= dg_{f(x)} (T_{f(x)}Y)\\
	d(g \circ f)_x (T_x X)) + T_{g \circ f(x)} W &= dg_{f(x)} (T_{f(x)}Y) + T_{g \circ f(x)} W && + \text{ means span} \\
	d(g \circ f)_x (T_x X)) + T_{g \circ f(x)} W &= T_{g \circ f(x)} Z 
\end{align*}

$ (\impliedby):$ We wish to prove the contrapositive: suppose $ f \not \pitchfork g^{-1}(W)$, that is, there exists a vector $ [v] \in T_{f(x)}Y$ that is not in $ df_x(T_x X) + T_{f(x)}g^{-1}(W)$, then as we apply $ dg_{f(x)} ([v])$ which is an element of $ T_{g \circ f(x)}Z$, then I claim that it is not in $ d(g \circ f)_x (T_x X) + T_{g \circ f(x)} W$. We already know that $ dg_{f(x)}([v]) \not\in T_{g \circ f(x)} W$ because $[v] \not\in T_{f(x)} g^{-1}(W)$. It remains to check that it is not in the first term (since any component $dg_{f(x)} ([v])$ that is in  $ T_{g \circ f(x)}W$ comes from $ T_{f(x)}g^{-1}(W)$ so WLOG we just need to show the other component is not in the first term).

Suppose to the contrary that $ dg_{f(x)}([v]) \in d(g \circ f)_x(T_x X)$, then there exists a vector $ [y] \in df_x(T_x X)$ s.t.\ $ dg_{f(x)}([y]) = dg_{f(x)} ([v])$. That means $ [v] = [y]+ \ker dg_{f(x)}$. But clearly $ \ker dg_{f(x)} \subseteq T_{f(x)}g^{-1}(W)$ since $ T_{g \circ f(x)}W$ contains 0. Hence $ [v] \in df_x(T_x X) + T_{f(x)}g^{-1}(W)$, a contradiction. Therefore, $ dg_{f(x)}([v]) \not\in d(g \circ f)_x (T_x X) + T_{g \circ f(x)}W$ and thus $ g \circ f \not \pitchfork W$.
\end{problem}

\begin{problem}[1.5.11]
Let $ C$ be a closed set of  $ \rr^{k}$, then there exists a smooth function $ f:\rr^{k} \to \rr$ s.t.\ $ C = f^{-1}(0)$. Consider the function $ g: \rr^{k+1} \to \rr, (x_1,\ldots,x_{k},x_{k+1}) \mapsto f(x_1,\ldots,x^{k}) + x_{k+1}$.  I claim that $ 0$ is a regular value of  $ g$. Indeed, the Jacobian of this function is $ (df,1)$ which has full rank for any point. Therefore, $ M:= g^{-1}(0)$ is a submanifold of $ \rr^{k+1}$. Now consider
\begin{align*}
	M \cap \rr^{k} &= \{x \in \rr^{k+1}: x_{k+1} = 0, g(x)= f(x_1,\ldots,x_k)+x_{k+1}=0\}  \\
	&= \{x \in \rr^{k+1} : f(x_1,\ldots,x_k) = 0\}  \\
	&= \{f^{-1}(0)\}  \\
	&= C
\end{align*}
as desired.
\end{problem}

\begin{problem}[1.6.7]
We prove by induction. Denote $ I_n$ to be the $ n \times n$ identity matrix. For $ k=1$, we see that by choosing a basis of $ \rr^2$, the antipodal map  $ S^{1} \to S^{1}, x \mapsto -x$ can be described by $-I$. Then
\begin{align*}
	\begin{pmatrix} \cos(\pi t)& \sin(\pi t)\\ \sin(\pi t)& \cos(\pi t) \end{pmatrix} 
\end{align*}
is a homotopy between the identity map $ I$ and antipodal map $ -I$. 

Now assume that the identity $ I_k$ is homotopic to the antipodal map $ -I_k$ for odd $ k$ via some homotopy $ H: S^{k} \times I \to S^{k}$. Then for the next odd number $ k+2$,
\begin{align*}
	\begin{pmatrix} H_t&0&0\\0& \cos(\pi t)& \sin(\pi t) \\ 0&\sin(\pi t)& \cos(\pi t) \end{pmatrix} 
\end{align*}
is a homotopy between $ I_{k+2}$ and $ -I_{k+2}$. Hence we show that identity and antipodal maps are homotopic for all odd $ k$.
\end{problem}

\begin{problem}[1.6.8]
	First, we consider the case where the manifold is path-connected. Note that since any manifold is locally diffeomorphic to $ \rr^{n}$, it is locally path-connected since $ \rr^{n}$ is. This allows us to use path-connectedness and connectedness interchangably by Munkres Theorem 25.5. Since $ f$ is a diffeomorphism, it is also a local diffeomorphism and an embedding onto  $ N$. Therefore for any homotopy of $ f$, there exist  $ \epsilon_1, \epsilon_2>0$ that preserve these properties under perturbation. Let $ \epsilon= \min( \epsilon_1, \epsilon_2)$, then $ f_t:M \to N$ is a local diffeomorphism and an embedding for any $ t \in [0, \epsilon)$. Given any open set $ U$ of $ M$, let  $ U_m$ denotes the neighborhood of  $ m$ where  $ f_t$ is a local diffeomorphism. Then  $ U=\bigcup_{ m \in U} U \cap U_m$. Notice $ f_t$ is also a diffeomorphism restricted to  $ U \cap U_m$ which is open so $ f_t(U)$ is a union of open sets which is open. That is,  $ f_t$ is an open map. In particular,  $ f_t(M)$ is open. Since $ M$ is compact,  $ f_t(M)$ is also compact. Since  $ N$ is Hausdorff,  $ f_t(M)$ is closed so $ N-f_t(M)$ is open. This implies that $ f_t(M)$ and $ N- f_t(M)$ form a separation of  $ N$. But since $ N$ is connected, and clearly $ f_t(M)$ as an embedding is not empty so this forces $ f_t(M) = N$. That is, $ f_t(M)$is a surjective embedding so it is a diffeomorphism onto  $ N$.

	Now suppose  $ M$ has multiple path components. Then since $ f$ is a diffeomorphism, it must map path components of  $ M$ to path components of  $ N$ in a bijective way. Together with the fact that homotopy can never cross path components, we are allowed to consider one pair of components at a time so the above result applies. Since $ M$ is compact, the path components must be finite (or each component would require at least one open set in any covering and we would not have finite subcover). , it suffices to take the minimum of the $ \epsilon$ yielded from each path component from above and we are done.
\end{problem}

\begin{problem}[1.7.11]
Since $ a$ is a nondegenerate critical point of  $ f$, by Morse Lemma, there exists a local coordinate system $ x=(x_1,\ldots,x_n)^{T}$ s.t.\ 
\begin{align*}
f=f(a) + x^{T}Hx,
\end{align*}
where $ H$ is the nondegenerate Hessian of  $ f$ under this coordinate system. Notice that $ H$ is symmetric so it is also diagonalizable, \emph{i.e.} $ H=P ^{-1}D P = P^{T} D P $ where $ D$ has nonzero eigenvalues $ \lambda_1,\ldots, \lambda_n$ on the diagonal. Let $ \epsilon_i = \sgn (\lambda_i)$. Then
 \begin{align*}
	 H &= P^{T} \begin{pmatrix} \sqrt{ |\lambda_1|} &0& \cdots&0\\0& \sqrt{ |\lambda_2|} & \cdots & 0\\ \vdots &0 &\ddots & 0 \\ 0& 0&\cdots& \sqrt{ \lambda_n} \end{pmatrix} \begin{pmatrix} \epsilon_1 &0& \cdots&0\\0& \epsilon_2 & \cdots & 0\\ \vdots &0 &\ddots & 0 \\ 0& 0&\cdots& \epsilon_n \end{pmatrix} \begin{pmatrix} \sqrt{ |\lambda_1|} &0& \cdots&0\\0& \sqrt{ |\lambda_2|} & \cdots & 0\\ \vdots &0 &\ddots & 0 \\ 0& 0&\cdots& \sqrt{ \lambda_n} \end{pmatrix} P\\
	 &=: (P^{T} \Lambda) E (\Lambda P) \\
	 &= (\Lambda P)^{T} E (\Lambda P) \\
	 &=: \widetilde{ P}^{T} E \widetilde{ P}  
\end{align*}
Since each $ |\lambda_i|>0$, the diagonal matrix $ \Lambda$ and thus $ \widetilde{ P}$ is invertible. Let $ \widetilde{ x} = \widetilde{ P}  x$. Since $ \widetilde{ P}$ is a linear isomorphism, $ \widetilde{ x}$ is also a local coordinate. Under this local coordinate, we see that
\begin{align*}
	f = f(a) + \widetilde{ x}^{T} E \widetilde{ x}
\end{align*}
as desired.
\end{problem}

\begin{problem}[1.7.14]
We wish to use the lemma on page 42 of the book to show that the poles are the only critical points and they are nondegenerate (and therefore the height function $ h$ is a Morse function on $ S^{k-1}$). 

First we consider the south pole $ S$. View $ S^{k-1}$ as a unit sphere inside $ \rr^{k}$ with the south pole at the origin so $ S=0$. Consider the height function restricted to $ S^{k-1} - N$ which is $ h_S: S^{k-1}-N \to \rr$, and the inverse stereographic projection $ g_S: \rr^{k-1} \to S^{k-1}-N$. We see that $ h_S=(h_S \circ g_S) \circ g_S^{-1}$. Since $ g_S^{-1}(0)=0$ (maps the south pole to the origin), by lemma it suffices to consider the critical points of $ h_S \circ g_S: \rr^{k-1} \to \rr, x \mapsto \frac{4+ \norm{ x}^2 }{4- \norm{ x}^2}$. By the quotient rule, its derivative is
\begin{align*}
	d (h_S \circ g_S)(x) &= \frac{2\norm{ x}(4- \norm{ x}^2 ) - (4+ \norm{ x}^2 )(-2 \norm{ x} )}{(4- \norm{ x}^2 )^2 }\\
	&= \frac{16 \norm{ x} }{ (4- \norm{ x}^2 )^2} .
\end{align*}
By basic geometry, $ \norm{ x}^2 = 4 \iff x=(0,\ldots,0,2)=N$, the derivative is well-defined everywhere. It equals zero iff the numerator is zero iff $ \norm{ x} =0$ iff $ x=0=S$ by positive definiteness of the norm. Hence  $ S$ is the unique critical point. Again by the positive definiteness of the norm, the Jacobian of the derivative, \emph{i.e.} the Hessian of $ h_S \circ g$, must also be positive definite. Thus $ S$ is nondegenerate as well. Hence by lemma, $ S$ is the unique critical point of  $ h$ restricted to one chart and it is nondegenerate.

By a similar argument, we can view $ S^{k-1}$ as the unit sphere with normal pole at the origin. Consider $ h_N:S^{k-1} -S \to \rr$ and inverse stereographic projection $ g_N: \rr^{k-1} \to S^{k-1} -S$. The composition $ h_N \circ g_N: \rr^{k-1} \to \rr, x \mapsto \frac{4- \norm{ x}^2 }{4+ \norm{ x}^2}$. This time we have negative definition Hessian. So $ N$ is the unique critical point of  $ h$ restricted to the other chart and it is also nondegenerate. That is all the critical points of  $ h$ and both are nondegenerate so  $ h$ is a Morse function of  $ S^{k-1}$.
\end{problem}
\end{document}
