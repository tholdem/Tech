\documentclass[12pt]{article}
\newcommand{\alert}[1]{{\bf \color{red} [Alert:] #1}}
\newcommand{\todo}[1]{{\bf \color{orange} [TODO:] #1}}
\newcommand{\real}[1][]{\mathbb{R}^{#1}}
\newcommand{\myeqn}[1]{(\ref{#1})}
\newcommand{\myex}[1]{Example \ref{#1}}
\newcommand{\defeq}{\stackrel{\mathrm{def}}{=}}
\newcommand{\parder}[2]{\frac{\partial #1}{\partial #2}}
\newcommand{\Lie}[3][]{\mathsf{L}_{#3}^{#1} #2}
\newcommand{\LieA}[1]{\mathsf{Lie}(#1)}
\newcommand{\lieder}[2]{\mathcal{L}_{#2} #1}
\renewcommand{\t}{^{\mbox{\tiny\sf T}}}
\newcommand{\trans}{^{\mbox{\tiny\sf T}}}
\newcommand{\markup}[1]{\{\textbf{#1}\}}
\newcommand{\msub}[1]{_\mathrm{#1}}
\newcommand{\msup}[1]{^\mathrm{#1}}
\newcommand{\inv}[1]{#1^{-1}}
\newcommand{\pinv}[1]{{#1}^{+}}
\newcommand{\myfracA}[2]{\displaystyle{\frac{#1}{#2}}}
\newcommand{\myfracB}[2]{{#1}/{#2}}
\newcommand{\mydiffA}[1]{\dot{#1}}
\newcommand{\mydiffB}[2]{\myfracA{\mathrm{d}{#1}}{\mathrm{d}{#2}}}
\newcommand{\ball}[2]{\mathcal{B}_{#1}\left(#2\right)}
\newcommand{\acos}[1]{\cos^{-1}\left(#1\right)}
\newcommand{\asin}[1]{\sin^{-1}\left(#1\right)}
\newcommand{\mani}{\mathcal{M}}
\newcommand{\tang}[2]{\mathsf{T}_{#1} #2}
\newcommand{\LieB}[2]{[ #1, #2 ]}
\newcommand{\LieBad}[3][]{\mathsf{ad}_{#2}^{#1} #3}
\newcommand{\ReachVT}{\mathcal{R}^V_T}
\newcommand{\ReachVt}{\mathcal{R}^V_t}
\newcommand{\ReachVTe}{\mathcal{R}^V_{\le T}}
\newcommand{\ReachT}{\mathcal{R}_T}
\newcommand{\Reacht}{\mathcal{R}_t}
\newcommand{\ReachTe}{\mathcal{R}_{\le T}}
\newcommand{\accLA}[1]{\mathsf{Lie}(#1)}
\newcommand{\accD}{\Delta_{\mathcal{F}}}
\newcommand{\accSA}{\mathsf{Lie}(\mathcal{G},f)}
\newcommand{\accDS}{\Delta_{\mathcal{G}}}
\newcommand{\eval}[3]{\mathsf{Ev}^{#2}_{#1}\left( #3 \right)}
\newcommand{\stlc}{\textsc{stlc}}
\newcommand{\clf}{\textsc{clf}}
\newcommand{\jqlf}{\textsc{jqlf}}
\newcommand{\dlas}{\textsc{dlas}}
\newcommand{\Ad}[2]{\mathsf{Ad}_{#1} #2}
\newcommand{\xe}{\ensuremath{x_e}}
\newcommand{\lebg}[1]{\mathcal{L}_{#1}}
\newcommand{\lebgx}[1]{\mathcal{L}_{#1 \mathrm{e}}}
\newcommand{\dom}{D}
\newcommand{\domT}{[t_0,\infty) \times D}
\newcommand{\rarrow}{\rightarrow}
\renewcommand{\d}{\mathrm{d}}
\renewcommand{\Re}{\mathbb{R}}
\newcommand{\C}{\mathrm{C}}

\newcommand{\QED}{{\unskip\nobreak\hfil\penalty50\hskip2em\vadjust{}
		\nobreak\hfil$\Box$\parfillskip=0pt\finalhyphendemerits=0\par}\vspace{0.1cm}}
\newcommand{\eoEx}{{\unskip\nobreak\hfil\penalty50\hskip0em\vadjust{}
		\nobreak\hfil$\Large\Diamond$\parfillskip=0pt\finalhyphendemerits=0\par}\vspace{0.1cm}}

\newcommand{\sgn}{\ensuremath{\operatorname{sgn}}}
\newcommand{\sat}{\ensuremath{\operatorname{sat}}}

\newcommand{\half}{\frac{1}{2}}
\newcommand{\shalf}{\mbox{$\frac{1}{2}$}}
\newcommand{\marcom}[1]{\marginpar{\footnotesize #1}}
\newcommand{\der}{\mathrm{D}}
\newcommand{\e}{\mathrm{e}}
\newcommand{\dt}{\mathrm{d}t}

\newcommand{\cA}{\ensuremath{\mathcal{A}}}
\newcommand{\cB}{\ensuremath{\mathcal{B}}}
\newcommand{\cG}{\ensuremath{\mathcal{G}}}
\newcommand{\cK}{\ensuremath{\mathcal{K}}}
\newcommand{\cW}{\ensuremath{\mathcal{W}}}
\newcommand{\cZ}{\ensuremath{\mathcal{Z}}}
\newcommand{\cS}{\ensuremath{\mathcal{S}}}
\newcommand{\cD}{\ensuremath{\mathcal{D}}}
\newcommand{\cP}{\ensuremath{\mathcal{P}}}
\newcommand{\cV}{\ensuremath{\mathcal{V}}}
\newcommand{\cL}{\ensuremath{\mathcal{L}}}
\newcommand{\cN}{\ensuremath{\mathcal{N}}}
\newcommand{\cI}{\ensuremath{\mathcal{I}}}
\newcommand{\cR}{\ensuremath{\mathcal{R}}}
\newcommand{\cM}{\ensuremath{\mathcal{M}}}
\newcommand{\cC}{\ensuremath{\mathcal{C}}}
\newcommand{\cF}{\ensuremath{\mathcal{F}}}
\newcommand{\cH}{\ensuremath{\mathcal{H}}}
\newcommand{\cO}{\ensuremath{\mathcal{O}}}
\newcommand{\cX}{\ensuremath{\mathcal{X}}}
\newcommand{\cY}{\ensuremath{\mathcal{Y}}}
\newcommand{\Ci}{\ensuremath{\mathcal{C}^\infty}}
\newcommand{\ISS}{\textsc{iss}}
\newcommand{\LISS}{\textsc{liss}}
\newcommand{\GAS}{\textsc{gas}}
\newcommand{\GS}{\textsc{gs}}
\newcommand{\LES}{\textsc{les}}
\newcommand{\GUAS}{\textsc{guas}}
\newcommand{\BIBO}{\textsc{bibo}}
\newcommand{\spec}{\ensuremath{\operatorname{spec}}}
\newcommand{\spn}{\ensuremath{\operatorname{span}}}
\renewcommand{\i}{\mathrm{i\,}}

\renewcommand{\implies}{\Rightarrow}

\renewcommand{\theenumi}{$\roman{enumi})$}
\renewcommand{\labelenumi}{\theenumi}

\font\ptmten=zptmcmrm scaled 1200
\newcommand{\w}{\mbox{{\ptmten w}}}
\newcommand{\z}{\mbox{{\ptmten z}}}
\renewcommand{\Re}{\mathbb{R}}

\newcommand{\cl}{\operatorname{cl}}
\newcommand{\intr}{\operatorname{int}}
\newcommand{\rank}{\operatorname{rank}}
\newcommand{\co}{\operatorname{co}}
\newcommand{\aff}{\operatorname{aff}}

\theoremstyle{plain}
\newtheorem{theorem}{Theorem}[chapter]
\newtheorem{claim}[theorem]{Claim}
\newtheorem{corollary}[theorem]{Corollary}
\newtheorem{prop}[theorem]{Proposition}
\newtheorem{fact}[theorem]{Fact}
\newtheorem{lemma}[theorem]{Lemma}

\newtheorem{remark}{Remark}[chapter]

\theoremstyle{definition}
\newtheorem{assume}[theorem]{Assumption}
\newtheorem{defn}[theorem]{Definition}
\newtheorem{problem}[theorem]{Problem}
\newtheorem{exercise}{Exercise}
\newtheorem{example}[theorem]{Example}


\begin{document}
\centerline {\textsf{\textbf{\LARGE{Homework 8}}}}
\centerline {Jaden Wang}
\vspace{.15in}
\begin{problem}[9.14]
	Since $ h_u$ is linear, its derivative  $ {d h_u} _p : T_p \Gamma \to T_p \rr$ is just itself  $ \langle p,u \rangle$. The only critical points of $ h_u$ are those points s.t.\ $ {dh_u}_p = 0 = \langle p,u \rangle$, \emph{i.e.} points that are orthogonal to $ u$. So it suffices to show that there are finitely many points on $ M$ that are orthogonal to  $ u$.

	Let $ f: \Gamma \to S^{1}$ be the function that maps any point on the curve to a vector orthogonal to it (by 90 degree clockwise rotation). Then $ f$ is linear and therefore smooth. Then by Sard's Theorem, almost all  $ u \in S^{1}$ are regular values of $ f$. That is, for almost every  $ u \in S^{1}$, $ f^{-1}(u)$ is a 0-dimensional manifold, \emph{i.e.} a set of points in $ \Gamma$.  Since $ S^{1}$ is compact, so is $ \Gamma$. Then by homework we know that $ f^{-1}(u)$ is a finite number of points. Thus we show that there are only finitely many points of $ \Gamma$ that is orthogonal to $ u$.
\end{problem}
\begin{problem}[10.1]
Sard's Theorem says that all except for measure zero set of elements in $ N$ are regular values of $ f$. But since $ \dim M < \dim N$, $ df_p$ can at most have rank  $ \dim M$ so none of the elements in $ f(M)$  can be regular value. Hence they must be in the measure zero set.
\end{problem}

\begin{problem}[10.2]
	Given $ [f] \in \pi_1(S^2)$, where $ f: S^{1} \to S^2$ is continuous, we can always choose a smooth representative by Weierstrass Approximation Theorem, \emph{i.e.}  perturbing $ f$ into a polynomial  $ \overline{f}: S^{1} \to S^2$, which is smooth. (This is because we can take an $ 2 \epsilon$ tube around the curve $ f(S^{1})$ and we know $ \overline{f}$ is in this $ 2 \epsilon$ tube. Since this $ 2 \epsilon$ tube is an hyper-annulus, it is homotopic to a hyper-circle so we can always homotop $ f$ to  $ \overline{f}$. If $ \overline{f}$ lies outside of $ S^n$, then we just project it to $ S^n$ which is still in the $ 2 \epsilon$ tube. Since projection is smooth, composition of smooth functions are smooth so we get a smooth function on $ S^n$.) So we can assume $ f$ is smooth. Then by 10.1, since $ \dim S^{1} < \dim S^{n}$, $ f(S^{1})$ has measure zero in $ S^n$ and thus misses at least one point in $ S^2$. But $ S^n - \{p\} $ is diffeomorphic to $ \rr^n$ via stereographic projection. Since $ \rr^n$ is contractible, so is $ f(S^{1}) \subseteq S^n - \{p\} $. Therefore, the loop $ f$ is homotopic to the constant loop, \emph{i.e.} $ \pi_1(S^n) = 0$. Since $ S^n$ is path-connected, we see that it is simply connected. 
\end{problem}

\begin{problem}[10.3]
Every hyperplane is completely identified by its outward normal vector and the offset from origin.

Consider the unit normal vector field  $ \nu:M \to S^{n}$. Since $ M$ is a smooth manifold, its unit normal vector field is also smooth. Then by Sard's Theore, almost every  $ u \in S^{n}$ is a regular value of $ f$. That is, almost every $ \nu^{-1}(u)$ is a 0-dimensional manifold (as $ \dim M = \dim S^{n} =n$ ), \emph{i.e.} a set of points. Since $ M$ is compact,  $ \nu^{-1}(u)$ is a finite set of points. That is, only a finite number of points in $ M$ that have tangent planes parallel to  $ H$. If  $ H$ happens to be tangent to $ M$ then a small perturbation would give us a hyperplane that either has a normal vector that is not tangent or an offset that doesn't land on  $ M$. Either way  this hyperplane will be transversal. So almost all hyperplanes are transversal.
\end{problem}

\begin{problem}[10.5]
For any smooth map $ f: M \to N$ and a critical point $ p \in M$, then $ df_p$ has rank less than $ n$. We can choose charts $ (U, \phi)$ of $ M$ around $ p$,  $ (V,\psi)$ of $ N$ around $ q:=f(p)$. Let $ g:= \psi \circ f \circ \phi ^{-1}: \phi (U) \subseteq \rr^{m} \to \rr^{n}$, it is clearly smooth. Then $ dg_{\phi ^{-1}(p)} = d \psi \circ df_p d \phi ^{-1}_p$ also has rank less than $ n$. Thus $ \phi ^{-1}(p)$ is a critical point and $ g(p)$ is a critical value. Therefore, any critical value of $ f$ yields a critical value of  $ g$. Since the set of critical value is measure zero, since $ \psi ^{-1}$ is smooth, it follows that the critical value of $ f$ is also measure zero. Thus almost all points of $ N$ are regular values of $ f$.
\end{problem}
\end{document}
