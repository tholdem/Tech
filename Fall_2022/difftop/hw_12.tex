\documentclass[12pt]{article}
%Fall 2022
% Some basic packages
\usepackage{standalone}[subpreambles=true]
\usepackage[utf8]{inputenc}
\usepackage[T1]{fontenc}
\usepackage{textcomp}
\usepackage[english]{babel}
\usepackage{url}
\usepackage{graphicx}
%\usepackage{quiver}
\usepackage{float}
\usepackage{enumitem}
\usepackage{lmodern}
\usepackage{comment}
\usepackage{hyperref}
\usepackage[usenames,svgnames,dvipsnames]{xcolor}
\usepackage[margin=1in]{geometry}
\usepackage{pdfpages}

\pdfminorversion=7

% Don't indent paragraphs, leave some space between them
\usepackage{parskip}

% Hide page number when page is empty
\usepackage{emptypage}
\usepackage{subcaption}
\usepackage{multicol}
\usepackage[b]{esvect}

% Math stuff
\usepackage{amsmath, amsfonts, mathtools, amsthm, amssymb}
\usepackage{bbm}
\usepackage{stmaryrd}
\allowdisplaybreaks

% Fancy script capitals
\usepackage{mathrsfs}
\usepackage{cancel}
% Bold math
\usepackage{bm}
% Some shortcuts
\newcommand{\rr}{\ensuremath{\mathbb{R}}}
\newcommand{\zz}{\ensuremath{\mathbb{Z}}}
\newcommand{\qq}{\ensuremath{\mathbb{Q}}}
\newcommand{\nn}{\ensuremath{\mathbb{N}}}
\newcommand{\ff}{\ensuremath{\mathbb{F}}}
\newcommand{\cc}{\ensuremath{\mathbb{C}}}
\newcommand{\ee}{\ensuremath{\mathbb{E}}}
\newcommand{\hh}{\ensuremath{\mathbb{H}}}
\renewcommand\O{\ensuremath{\emptyset}}
\newcommand{\norm}[1]{{\left\lVert{#1}\right\rVert}}
\newcommand{\dbracket}[1]{{\left\llbracket{#1}\right\rrbracket}}
\newcommand{\ve}[1]{{\bm{#1}}}
\newcommand\allbold[1]{{\boldmath\textbf{#1}}}
\DeclareMathOperator{\lcm}{lcm}
\DeclareMathOperator{\im}{im}
\DeclareMathOperator{\coim}{coim}
\DeclareMathOperator{\dom}{dom}
\DeclareMathOperator{\tr}{tr}
\DeclareMathOperator{\rank}{rank}
\DeclareMathOperator*{\var}{Var}
\DeclareMathOperator*{\ev}{E}
\DeclareMathOperator{\dg}{deg}
\DeclareMathOperator{\aff}{aff}
\DeclareMathOperator{\conv}{conv}
\DeclareMathOperator{\inte}{int}
\DeclareMathOperator*{\argmin}{argmin}
\DeclareMathOperator*{\argmax}{argmax}
\DeclareMathOperator{\graph}{graph}
\DeclareMathOperator{\sgn}{sgn}
\DeclareMathOperator*{\Rep}{Rep}
\DeclareMathOperator{\Proj}{Proj}
\DeclareMathOperator{\mat}{mat}
\DeclareMathOperator{\diag}{diag}
\DeclareMathOperator{\aut}{Aut}
\DeclareMathOperator{\gal}{Gal}
\DeclareMathOperator{\inn}{Inn}
\DeclareMathOperator{\edm}{End}
\DeclareMathOperator{\Hom}{Hom}
\DeclareMathOperator{\ext}{Ext}
\DeclareMathOperator{\tor}{Tor}
\DeclareMathOperator{\Span}{Span}
\DeclareMathOperator{\Stab}{Stab}
\DeclareMathOperator{\cont}{cont}
\DeclareMathOperator{\Ann}{Ann}
\DeclareMathOperator{\Div}{div}
\DeclareMathOperator{\curl}{curl}
\DeclareMathOperator{\nat}{Nat}
\DeclareMathOperator{\gr}{Gr}
\DeclareMathOperator{\vect}{Vect}
\DeclareMathOperator{\id}{id}
\DeclareMathOperator{\Mod}{Mod}
\DeclareMathOperator{\sign}{sign}
\DeclareMathOperator{\Surf}{Surf}
\DeclareMathOperator{\fcone}{fcone}
\DeclareMathOperator{\Rot}{Rot}
\DeclareMathOperator{\grad}{grad}
\DeclareMathOperator{\atan2}{atan2}
\DeclareMathOperator{\Ric}{Ric}
\let\vec\relax
\DeclareMathOperator{\vec}{vec}
\let\Re\relax
\DeclareMathOperator{\Re}{Re}
\let\Im\relax
\DeclareMathOperator{\Im}{Im}
% Put x \to \infty below \lim
\let\svlim\lim\def\lim{\svlim\limits}

%wide hat
\usepackage{scalerel,stackengine}
\stackMath
\newcommand*\wh[1]{%
\savestack{\tmpbox}{\stretchto{%
  \scaleto{%
    \scalerel*[\widthof{\ensuremath{#1}}]{\kern-.6pt\bigwedge\kern-.6pt}%
    {\rule[-\textheight/2]{1ex}{\textheight}}%WIDTH-LIMITED BIG WEDGE
  }{\textheight}% 
}{0.5ex}}%
\stackon[1pt]{#1}{\tmpbox}%
}
\parskip 1ex

%Make implies and impliedby shorter
\let\implies\Rightarrow
\let\impliedby\Leftarrow
\let\iff\Leftrightarrow
\let\epsilon\varepsilon

% Add \contra symbol to denote contradiction
\usepackage{stmaryrd} % for \lightning
\newcommand\contra{\scalebox{1.5}{$\lightning$}}

% \let\phi\varphi

% Command for short corrections
% Usage: 1+1=\correct{3}{2}

\definecolor{correct}{HTML}{009900}
\newcommand\correct[2]{\ensuremath{\:}{\color{red}{#1}}\ensuremath{\to }{\color{correct}{#2}}\ensuremath{\:}}
\newcommand\green[1]{{\color{correct}{#1}}}

% horizontal rule
\newcommand\hr{
    \noindent\rule[0.5ex]{\linewidth}{0.5pt}
}

% hide parts
\newcommand\hide[1]{}

% si unitx
\usepackage{siunitx}
\sisetup{locale = FR}

%allows pmatrix to stretch
\makeatletter
\renewcommand*\env@matrix[1][\arraystretch]{%
  \edef\arraystretch{#1}%
  \hskip -\arraycolsep
  \let\@ifnextchar\new@ifnextchar
  \array{*\c@MaxMatrixCols c}}
\makeatother

\renewcommand{\arraystretch}{0.8}

\renewcommand{\baselinestretch}{1.5}

\usepackage{graphics}
\usepackage{epstopdf}

\RequirePackage{hyperref}
%%
%% Add support for color in order to color the hyperlinks.
%% 
\hypersetup{
  colorlinks = true,
  urlcolor = blue,
  citecolor = blue
}
%%fakesection Links
\hypersetup{
    colorlinks,
    linkcolor={red!50!black},
    citecolor={green!50!black},
    urlcolor={blue!80!black}
}
%customization of cleveref
\RequirePackage[capitalize,nameinlink]{cleveref}[0.19]

% Per SIAM Style Manual, "section" should be lowercase
\crefname{section}{section}{sections}
\crefname{subsection}{subsection}{subsections}
\Crefname{section}{Section}{Sections}
\Crefname{subsection}{Subsection}{Subsections}

% Per SIAM Style Manual, "Figure" should be spelled out in references
\Crefname{figure}{Figure}{Figures}

% Per SIAM Style Manual, don't say equation in front on an equation.
\crefformat{equation}{\textup{#2(#1)#3}}
\crefrangeformat{equation}{\textup{#3(#1)#4--#5(#2)#6}}
\crefmultiformat{equation}{\textup{#2(#1)#3}}{ and \textup{#2(#1)#3}}
{, \textup{#2(#1)#3}}{, and \textup{#2(#1)#3}}
\crefrangemultiformat{equation}{\textup{#3(#1)#4--#5(#2)#6}}%
{ and \textup{#3(#1)#4--#5(#2)#6}}{, \textup{#3(#1)#4--#5(#2)#6}}{, and \textup{#3(#1)#4--#5(#2)#6}}

% But spell it out at the beginning of a sentence.
\Crefformat{equation}{#2Equation~\textup{(#1)}#3}
\Crefrangeformat{equation}{Equations~\textup{#3(#1)#4--#5(#2)#6}}
\Crefmultiformat{equation}{Equations~\textup{#2(#1)#3}}{ and \textup{#2(#1)#3}}
{, \textup{#2(#1)#3}}{, and \textup{#2(#1)#3}}
\Crefrangemultiformat{equation}{Equations~\textup{#3(#1)#4--#5(#2)#6}}%
{ and \textup{#3(#1)#4--#5(#2)#6}}{, \textup{#3(#1)#4--#5(#2)#6}}{, and \textup{#3(#1)#4--#5(#2)#6}}

% Make number non-italic in any environment.
\crefdefaultlabelformat{#2\textup{#1}#3}

% Environments
\makeatother
% For box around Definition, Theorem, \ldots
%%fakesection Theorems
\usepackage{thmtools}
\usepackage[framemethod=TikZ]{mdframed}

\theoremstyle{definition}
\mdfdefinestyle{mdbluebox}{%
	roundcorner = 10pt,
	linewidth=1pt,
	skipabove=12pt,
	innerbottommargin=9pt,
	skipbelow=2pt,
	nobreak=true,
	linecolor=blue,
	backgroundcolor=TealBlue!5,
}
\declaretheoremstyle[
	headfont=\sffamily\bfseries\color{MidnightBlue},
	mdframed={style=mdbluebox},
	headpunct={\\[3pt]},
	postheadspace={0pt}
]{thmbluebox}

\mdfdefinestyle{mdredbox}{%
	linewidth=0.5pt,
	skipabove=12pt,
	frametitleaboveskip=5pt,
	frametitlebelowskip=0pt,
	skipbelow=2pt,
	frametitlefont=\bfseries,
	innertopmargin=4pt,
	innerbottommargin=8pt,
	nobreak=false,
	linecolor=RawSienna,
	backgroundcolor=Salmon!5,
}
\declaretheoremstyle[
	headfont=\bfseries\color{RawSienna},
	mdframed={style=mdredbox},
	headpunct={\\[3pt]},
	postheadspace={0pt},
]{thmredbox}

\declaretheorem[%
style=thmbluebox,name=Theorem,numberwithin=section]{thm}
\declaretheorem[style=thmbluebox,name=Lemma,sibling=thm]{lem}
\declaretheorem[style=thmbluebox,name=Proposition,sibling=thm]{prop}
\declaretheorem[style=thmbluebox,name=Corollary,sibling=thm]{coro}
\declaretheorem[style=thmredbox,name=Example,sibling=thm]{eg}

\mdfdefinestyle{mdgreenbox}{%
	roundcorner = 10pt,
	linewidth=1pt,
	skipabove=12pt,
	innerbottommargin=9pt,
	skipbelow=2pt,
	nobreak=true,
	linecolor=ForestGreen,
	backgroundcolor=ForestGreen!5,
}

\declaretheoremstyle[
	headfont=\bfseries\sffamily\color{ForestGreen!70!black},
	bodyfont=\normalfont,
	spaceabove=2pt,
	spacebelow=1pt,
	mdframed={style=mdgreenbox},
	headpunct={ --- },
]{thmgreenbox}

\declaretheorem[style=thmgreenbox,name=Definition,sibling=thm]{defn}

\mdfdefinestyle{mdgreenboxsq}{%
	linewidth=1pt,
	skipabove=12pt,
	innerbottommargin=9pt,
	skipbelow=2pt,
	nobreak=true,
	linecolor=ForestGreen,
	backgroundcolor=ForestGreen!5,
}
\declaretheoremstyle[
	headfont=\bfseries\sffamily\color{ForestGreen!70!black},
	bodyfont=\normalfont,
	spaceabove=2pt,
	spacebelow=1pt,
	mdframed={style=mdgreenboxsq},
	headpunct={},
]{thmgreenboxsq}
\declaretheoremstyle[
	headfont=\bfseries\sffamily\color{ForestGreen!70!black},
	bodyfont=\normalfont,
	spaceabove=2pt,
	spacebelow=1pt,
	mdframed={style=mdgreenboxsq},
	headpunct={},
]{thmgreenboxsq*}

\mdfdefinestyle{mdblackbox}{%
	skipabove=8pt,
	linewidth=3pt,
	rightline=false,
	leftline=true,
	topline=false,
	bottomline=false,
	linecolor=black,
	backgroundcolor=RedViolet!5!gray!5,
}
\declaretheoremstyle[
	headfont=\bfseries,
	bodyfont=\normalfont\small,
	spaceabove=0pt,
	spacebelow=0pt,
	mdframed={style=mdblackbox}
]{thmblackbox}

\theoremstyle{plain}
\declaretheorem[name=Question,sibling=thm,style=thmblackbox]{ques}
\declaretheorem[name=Remark,sibling=thm,style=thmgreenboxsq]{remark}
\declaretheorem[name=Remark,sibling=thm,style=thmgreenboxsq*]{remark*}
\newtheorem{ass}[thm]{Assumptions}

\theoremstyle{definition}
\newtheorem*{problem}{Problem}
\newtheorem{claim}[thm]{Claim}
\theoremstyle{remark}
\newtheorem*{case}{Case}
\newtheorem*{notation}{Notation}
\newtheorem*{note}{Note}
\newtheorem*{motivation}{Motivation}
\newtheorem*{intuition}{Intuition}
\newtheorem*{conjecture}{Conjecture}

% Make section starts with 1 for report type
%\renewcommand\thesection{\arabic{section}}

% End example and intermezzo environments with a small diamond (just like proof
% environments end with a small square)
\usepackage{etoolbox}
\AtEndEnvironment{vb}{\null\hfill$\diamond$}%
\AtEndEnvironment{intermezzo}{\null\hfill$\diamond$}%
% \AtEndEnvironment{opmerking}{\null\hfill$\diamond$}%

% Fix some spacing
% http://tex.stackexchange.com/questions/22119/how-can-i-change-the-spacing-before-theorems-with-amsthm
\makeatletter
\def\thm@space@setup{%
  \thm@preskip=\parskip \thm@postskip=0pt
}

% Fix some stuff
% %http://tex.stackexchange.com/questions/76273/multiple-pdfs-with-page-group-included-in-a-single-page-warning
\pdfsuppresswarningpagegroup=1


% My name
\author{Jaden Wang}



\begin{document}
\centerline {\textsf{\textbf{\LARGE{Homework 12}}}}
\centerline {Jaden Wang}
\vspace{.15in}
\begin{problem}[4.2.1]
First note that since $ T$ is alternating,  if any two of the entries coincide \emph{i.e.} $ v_i = v_j$, then
\begin{align*}
	T(v_1,\ldots,v_i,\ldots,v_j,\ldots,v_n) &= - T(v_1,\ldots,v_j,\ldots,v_i,\ldots,v_n) && \text{ alternating} \\
						&= -T(v_1,\ldots,v_i,\ldots,v_j,\ldots,v_n) && v_i = v_j\\
						&= 0 &&  x=-x \implies x=0
\end{align*}
by the definition of alternating. Since $ v_1,\ldots,v_n$ are linearly dependent, there exists $ a_1,\ldots,a_n \in \rr$ not all zeros (WLOG $ a_1 \neq 0$) such that $ a_1 v_1 + \cdots + a_n v_n = 0$. This yields $ v_1 = \frac{a_2}{ a_1} v_2 + \cdots + \frac{a_n}{ a_1}v_n$. Thus we have
\begin{align*}
	T(v_1,\ldots,v_n) &= T\left(\frac{a_2}{ a_1} v_2 + \cdots + \frac{a_n}{ a_1} v_n, v_2,\ldots,v_n\right) \\
	&=\frac{a_2}{ a_1}T\left(v_2,v_2,\ldots,v_n \right) + \cdots + \frac{a_n}{ a_1}T\left( v_n,v_2,\ldots,v_n \right)   \\
	&= 0+ \cdots + 0 = 0 
\end{align*}
\end{problem}
\begin{problem}[4.2.3]
	If $ \phi_i $ are linearly dependent, then the dependence relation would make the matrix $ [\phi_i(v_j)]$ having linearly dependent columns, so $ \det $ is 0, matching the result of Exercise 2. If $ \phi_i$ are linear independent. First consider the standard dual basis $ x_1,\ldots,x_k$ where $ x_i(e_j) = \delta_{ij}$. Let $ M = (v_1,v_2,..,v_n)$. It is easy to see that $ M = [x_i(v_j)]$. Thus we have
	\begin{align*}
		x_1 \wedge \cdots \wedge x_k(M) &= M^* (x_1 \wedge \cdots \wedge x_k(I)) \\
		&= \det M \text{Alt} (x_1 \otimes \cdots \otimes  x_k)(I) \\
		&= \det M \left( \frac{1}{k!} \sum_{ \sigma \in S_k} x_{ \sigma(1)} \otimes \cdots \otimes x_{ \sigma(k)} \right)(I)  \\
		&= \frac{1}{k!} \det M (x_1 \otimes \cdots \otimes x_k)(I) \\
		&= \frac{1}{k!} \det M (1\cdots 1) \\
		&= \frac{1}{k!} \det [x_i(e_j)] 
	\end{align*}
	Now we check that $ \det [\phi_i]$ is an alternating tensor. We have that $ \det $ is multilinear in the columns, and the matrix where each entry $ \phi_i$ is linear is clearly multilinear in all the columns. Thus the composition $ \det [\phi_i]$ is multilinear. Moreover, $ \det $ is alternating; since swapping $ v_j,v_k$ leads to swapping of $ \phi_i(v_j)$ and $ \phi_i(v_k)$, we obtain the negative of the original determinant so $ \det [\phi_i]$ is alternating. Hence  $ \det [\phi_i] \in \Lambda^{k}(\rr^{k*})$. Since its dimension is one, $ \phi_1 \wedge \cdots \wedge \phi_k (v_1,\ldots,v_k) = \lambda \det [\phi_i(v_j)]$. Since $ \phi_i= a_1^{i} x_1 + a_n^{i} x_n$, define $ w_j = \frac{1}{a_1^{j}} e_1 + \cdots \frac{1}{a_n^{j}} e_n$ wherever $ a_k^{j} \neq 0$. Then it is easy to see that $ \phi_i (w_j) = \delta_{ij}$. By the same argument as in the dual basis case, $ \lambda = \frac{1}{k!}$, it follows that 
	\begin{align*}
		\phi_1 \wedge \cdots \wedge \phi_k(v_1,\ldots,v_k) &= \frac{1}{k!} \det [\phi_i(v_j)]
	\end{align*}
\end{problem}

\begin{problem}[4.2.5]
\begin{align*}
	\text{Alt}(\phi_1 \otimes \phi_2 \otimes \phi_3) &= \frac{1}{6} (\phi_1 \otimes \phi_2 \otimes \phi_3 - \phi_1 \otimes \phi_3 \otimes \phi_2 + \phi_3 \otimes \phi_1 \otimes \phi_2 \\
	& \quad - \phi_3 \otimes \phi_2 \otimes \phi_1 +\phi_2 \otimes \phi_3 \otimes \phi_1 - \phi_2 \otimes \phi_1 \otimes \phi_3) \\ 
\end{align*}
\end{problem}

\begin{problem}[4.2.6]
\begin{enumerate}[label=(\alph*)]
	\item Two ordered bases are equivalently oriented iff the linear map defined by mapping between them has positive determinant. Define $ A: V \to V, v_i \mapsto v_i'$. Then notice
		\begin{align*}
			T(v_1',\ldots,v_n') &= T(Av_1,\ldots,Av_n) \\
			&= A^* T(v_1,\ldots,v_n) \\
			&= \det A T(v_1,\ldots,v_n)
		\end{align*}
		It follows that $\det A > 0 \iff T(v_1,\ldots,v_n)$ and $ T(v_1',\ldots,v_n')$ have the same sign.  
	\item By part a), the sign of $ T$ is well-defined independent of the choice of positively oriented basis.
	\item We define the orientation of $ V$ by the orientation of its ordered basis. An ordered basis $ \{v_1,\ldots,v_n\} $ is positively oriented if the sign of $ T(v_1,\ldots,v_n)$ is positive. This is again well-defined by part a). Given an orientation on $ \Lambda^{n}(V^* )$, any two alternating tensors are scalar multiples of each other, so their sign difference will also be passed to the orientation of the vectors $ v_1,\ldots,v_n$, making it well-defined$.

\end{enumerate}
\end{problem}

\begin{problem}[4.2.7]
Define $ D (A):= \det A^{T}$ so $ D \in \Lambda^{k}(\rr^{k*})$. We wish to show that $ D$ is multilinear, alternating, and  $ D(I) = 1$. Since  $ \det $ is multilinear in the row, we see that $ D$ is multilinear in the columns. Let  $ P$ be the permutation matrix that swap  $ i,j$th row of  $ A$. Then
 \begin{align*}
	D(PA) &= \det (PA)^{T} \\
	&= \det A^{T} \det P^{T} \\
	&= \det A^{T} (-1) && \text{ swap row viewed as swap col} \\
	&= - D(A) 
\end{align*}
So $ D$ is alternating. Finally  $ D(I) = \det (I^{T}) = \det I = 1$. Hence by uniqueness of $ \det $, $ D = \det $ so $ D(A) = \det A^{T} = \det A$.
\end{problem}

\begin{problem}[P165 Exercise]
Since $ \phi: Y \to \rr$, $ f^* \phi = \phi \circ f$. Therefore,
\begin{align*}
	f^* (d \phi) &= d \phi \circ df \\
		     &= d (\phi \circ f) && \text{ chain rule}  \\
	&= d(f^* \phi) 
\end{align*}
\end{problem}
\end{document}
