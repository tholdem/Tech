\documentclass[12pt]{article}
\newcommand{\alert}[1]{{\bf \color{red} [Alert:] #1}}
\newcommand{\todo}[1]{{\bf \color{orange} [TODO:] #1}}
\newcommand{\real}[1][]{\mathbb{R}^{#1}}
\newcommand{\myeqn}[1]{(\ref{#1})}
\newcommand{\myex}[1]{Example \ref{#1}}
\newcommand{\defeq}{\stackrel{\mathrm{def}}{=}}
\newcommand{\parder}[2]{\frac{\partial #1}{\partial #2}}
\newcommand{\Lie}[3][]{\mathsf{L}_{#3}^{#1} #2}
\newcommand{\LieA}[1]{\mathsf{Lie}(#1)}
\newcommand{\lieder}[2]{\mathcal{L}_{#2} #1}
\renewcommand{\t}{^{\mbox{\tiny\sf T}}}
\newcommand{\trans}{^{\mbox{\tiny\sf T}}}
\newcommand{\markup}[1]{\{\textbf{#1}\}}
\newcommand{\msub}[1]{_\mathrm{#1}}
\newcommand{\msup}[1]{^\mathrm{#1}}
\newcommand{\inv}[1]{#1^{-1}}
\newcommand{\pinv}[1]{{#1}^{+}}
\newcommand{\myfracA}[2]{\displaystyle{\frac{#1}{#2}}}
\newcommand{\myfracB}[2]{{#1}/{#2}}
\newcommand{\mydiffA}[1]{\dot{#1}}
\newcommand{\mydiffB}[2]{\myfracA{\mathrm{d}{#1}}{\mathrm{d}{#2}}}
\newcommand{\ball}[2]{\mathcal{B}_{#1}\left(#2\right)}
\newcommand{\acos}[1]{\cos^{-1}\left(#1\right)}
\newcommand{\asin}[1]{\sin^{-1}\left(#1\right)}
\newcommand{\mani}{\mathcal{M}}
\newcommand{\tang}[2]{\mathsf{T}_{#1} #2}
\newcommand{\LieB}[2]{[ #1, #2 ]}
\newcommand{\LieBad}[3][]{\mathsf{ad}_{#2}^{#1} #3}
\newcommand{\ReachVT}{\mathcal{R}^V_T}
\newcommand{\ReachVt}{\mathcal{R}^V_t}
\newcommand{\ReachVTe}{\mathcal{R}^V_{\le T}}
\newcommand{\ReachT}{\mathcal{R}_T}
\newcommand{\Reacht}{\mathcal{R}_t}
\newcommand{\ReachTe}{\mathcal{R}_{\le T}}
\newcommand{\accLA}[1]{\mathsf{Lie}(#1)}
\newcommand{\accD}{\Delta_{\mathcal{F}}}
\newcommand{\accSA}{\mathsf{Lie}(\mathcal{G},f)}
\newcommand{\accDS}{\Delta_{\mathcal{G}}}
\newcommand{\eval}[3]{\mathsf{Ev}^{#2}_{#1}\left( #3 \right)}
\newcommand{\stlc}{\textsc{stlc}}
\newcommand{\clf}{\textsc{clf}}
\newcommand{\jqlf}{\textsc{jqlf}}
\newcommand{\dlas}{\textsc{dlas}}
\newcommand{\Ad}[2]{\mathsf{Ad}_{#1} #2}
\newcommand{\xe}{\ensuremath{x_e}}
\newcommand{\lebg}[1]{\mathcal{L}_{#1}}
\newcommand{\lebgx}[1]{\mathcal{L}_{#1 \mathrm{e}}}
\newcommand{\dom}{D}
\newcommand{\domT}{[t_0,\infty) \times D}
\newcommand{\rarrow}{\rightarrow}
\renewcommand{\d}{\mathrm{d}}
\renewcommand{\Re}{\mathbb{R}}
\newcommand{\C}{\mathrm{C}}

\newcommand{\QED}{{\unskip\nobreak\hfil\penalty50\hskip2em\vadjust{}
		\nobreak\hfil$\Box$\parfillskip=0pt\finalhyphendemerits=0\par}\vspace{0.1cm}}
\newcommand{\eoEx}{{\unskip\nobreak\hfil\penalty50\hskip0em\vadjust{}
		\nobreak\hfil$\Large\Diamond$\parfillskip=0pt\finalhyphendemerits=0\par}\vspace{0.1cm}}

\newcommand{\sgn}{\ensuremath{\operatorname{sgn}}}
\newcommand{\sat}{\ensuremath{\operatorname{sat}}}

\newcommand{\half}{\frac{1}{2}}
\newcommand{\shalf}{\mbox{$\frac{1}{2}$}}
\newcommand{\marcom}[1]{\marginpar{\footnotesize #1}}
\newcommand{\der}{\mathrm{D}}
\newcommand{\e}{\mathrm{e}}
\newcommand{\dt}{\mathrm{d}t}

\newcommand{\cA}{\ensuremath{\mathcal{A}}}
\newcommand{\cB}{\ensuremath{\mathcal{B}}}
\newcommand{\cG}{\ensuremath{\mathcal{G}}}
\newcommand{\cK}{\ensuremath{\mathcal{K}}}
\newcommand{\cW}{\ensuremath{\mathcal{W}}}
\newcommand{\cZ}{\ensuremath{\mathcal{Z}}}
\newcommand{\cS}{\ensuremath{\mathcal{S}}}
\newcommand{\cD}{\ensuremath{\mathcal{D}}}
\newcommand{\cP}{\ensuremath{\mathcal{P}}}
\newcommand{\cV}{\ensuremath{\mathcal{V}}}
\newcommand{\cL}{\ensuremath{\mathcal{L}}}
\newcommand{\cN}{\ensuremath{\mathcal{N}}}
\newcommand{\cI}{\ensuremath{\mathcal{I}}}
\newcommand{\cR}{\ensuremath{\mathcal{R}}}
\newcommand{\cM}{\ensuremath{\mathcal{M}}}
\newcommand{\cC}{\ensuremath{\mathcal{C}}}
\newcommand{\cF}{\ensuremath{\mathcal{F}}}
\newcommand{\cH}{\ensuremath{\mathcal{H}}}
\newcommand{\cO}{\ensuremath{\mathcal{O}}}
\newcommand{\cX}{\ensuremath{\mathcal{X}}}
\newcommand{\cY}{\ensuremath{\mathcal{Y}}}
\newcommand{\Ci}{\ensuremath{\mathcal{C}^\infty}}
\newcommand{\ISS}{\textsc{iss}}
\newcommand{\LISS}{\textsc{liss}}
\newcommand{\GAS}{\textsc{gas}}
\newcommand{\GS}{\textsc{gs}}
\newcommand{\LES}{\textsc{les}}
\newcommand{\GUAS}{\textsc{guas}}
\newcommand{\BIBO}{\textsc{bibo}}
\newcommand{\spec}{\ensuremath{\operatorname{spec}}}
\newcommand{\spn}{\ensuremath{\operatorname{span}}}
\renewcommand{\i}{\mathrm{i\,}}

\renewcommand{\implies}{\Rightarrow}

\renewcommand{\theenumi}{$\roman{enumi})$}
\renewcommand{\labelenumi}{\theenumi}

\font\ptmten=zptmcmrm scaled 1200
\newcommand{\w}{\mbox{{\ptmten w}}}
\newcommand{\z}{\mbox{{\ptmten z}}}
\renewcommand{\Re}{\mathbb{R}}

\newcommand{\cl}{\operatorname{cl}}
\newcommand{\intr}{\operatorname{int}}
\newcommand{\rank}{\operatorname{rank}}
\newcommand{\co}{\operatorname{co}}
\newcommand{\aff}{\operatorname{aff}}

\theoremstyle{plain}
\newtheorem{theorem}{Theorem}[chapter]
\newtheorem{claim}[theorem]{Claim}
\newtheorem{corollary}[theorem]{Corollary}
\newtheorem{prop}[theorem]{Proposition}
\newtheorem{fact}[theorem]{Fact}
\newtheorem{lemma}[theorem]{Lemma}

\newtheorem{remark}{Remark}[chapter]

\theoremstyle{definition}
\newtheorem{assume}[theorem]{Assumption}
\newtheorem{defn}[theorem]{Definition}
\newtheorem{problem}[theorem]{Problem}
\newtheorem{exercise}{Exercise}
\newtheorem{example}[theorem]{Example}


\begin{document}
\centerline {\textsf{\textbf{\LARGE{Homework 12}}}}
\centerline {Jaden Wang}
\vspace{.15in}
\begin{problem}[4.2.1]
First note that since $ T$ is alternating,  if any two of the entries coincide \emph{i.e.} $ v_i = v_j$, then
\begin{align*}
	T(v_1,\ldots,v_i,\ldots,v_j,\ldots,v_n) &= - T(v_1,\ldots,v_j,\ldots,v_i,\ldots,v_n) && \text{ alternating} \\
						&= -T(v_1,\ldots,v_i,\ldots,v_j,\ldots,v_n) && v_i = v_j\\
						&= 0 &&  x=-x \implies x=0
\end{align*}
by the definition of alternating. Since $ v_1,\ldots,v_n$ are linearly dependent, there exists $ a_1,\ldots,a_n \in \rr$ not all zeros (WLOG $ a_1 \neq 0$) such that $ a_1 v_1 + \cdots + a_n v_n = 0$. This yields $ v_1 = \frac{a_2}{ a_1} v_2 + \cdots + \frac{a_n}{ a_1}v_n$. Thus we have
\begin{align*}
	T(v_1,\ldots,v_n) &= T\left(\frac{a_2}{ a_1} v_2 + \cdots + \frac{a_n}{ a_1} v_n, v_2,\ldots,v_n\right) \\
	&=\frac{a_2}{ a_1}T\left(v_2,v_2,\ldots,v_n \right) + \cdots + \frac{a_n}{ a_1}T\left( v_n,v_2,\ldots,v_n \right)   \\
	&= 0+ \cdots + 0 = 0 
\end{align*}
\end{problem}
\begin{problem}[4.2.3]
	If $ \phi_i $ are linearly dependent, then the dependence relation would make the matrix $ [\phi_i(v_j)]$ having linearly dependent columns, so $ \det $ is 0, matching the result of Exercise 2. If $ \phi_i$ are linear independent. First consider the standard dual basis $ x_1,\ldots,x_k$ where $ x_i(e_j) = \delta_{ij}$. Let $ M = (v_1,v_2,..,v_n)$. It is easy to see that $ M = [x_i(v_j)]$. Thus we have
	\begin{align*}
		x_1 \wedge \cdots \wedge x_k(M) &= M^* (x_1 \wedge \cdots \wedge x_k(I)) \\
		&= \det M \text{Alt} (x_1 \otimes \cdots \otimes  x_k)(I) \\
		&= \det M \left( \frac{1}{k!} \sum_{ \sigma \in S_k} x_{ \sigma(1)} \otimes \cdots \otimes x_{ \sigma(k)} \right)(I)  \\
		&= \frac{1}{k!} \det M (x_1 \otimes \cdots \otimes x_k)(I) \\
		&= \frac{1}{k!} \det M (1\cdots 1) \\
		&= \frac{1}{k!} \det [x_i(e_j)] 
	\end{align*}
	Now we check that $ \det [\phi_i]$ is an alternating tensor. We have that $ \det $ is multilinear in the columns, and the matrix where each entry $ \phi_i$ is linear is clearly multilinear in all the columns. Thus the composition $ \det [\phi_i]$ is multilinear. Moreover, $ \det $ is alternating; since swapping $ v_j,v_k$ leads to swapping of $ \phi_i(v_j)$ and $ \phi_i(v_k)$, we obtain the negative of the original determinant so $ \det [\phi_i]$ is alternating. Hence  $ \det [\phi_i] \in \Lambda^{k}(\rr^{k*})$. Since its dimension is one, $ \phi_1 \wedge \cdots \wedge \phi_k (v_1,\ldots,v_k) = \lambda \det [\phi_i(v_j)]$. Since $ \phi_i= a_1^{i} x_1 + a_n^{i} x_n$, define $ w_j = \frac{1}{a_1^{j}} e_1 + \cdots \frac{1}{a_n^{j}} e_n$ wherever $ a_k^{j} \neq 0$. Then it is easy to see that $ \phi_i (w_j) = \delta_{ij}$. By the same argument as in the dual basis case, $ \lambda = \frac{1}{k!}$, it follows that 
	\begin{align*}
		\phi_1 \wedge \cdots \wedge \phi_k(v_1,\ldots,v_k) &= \frac{1}{k!} \det [\phi_i(v_j)]
	\end{align*}
\end{problem}

\begin{problem}[4.2.5]
\begin{align*}
	\text{Alt}(\phi_1 \otimes \phi_2 \otimes \phi_3) &= \frac{1}{6} (\phi_1 \otimes \phi_2 \otimes \phi_3 - \phi_1 \otimes \phi_3 \otimes \phi_2 + \phi_3 \otimes \phi_1 \otimes \phi_2 \\
	& \quad - \phi_3 \otimes \phi_2 \otimes \phi_1 +\phi_2 \otimes \phi_3 \otimes \phi_1 - \phi_2 \otimes \phi_1 \otimes \phi_3) \\ 
\end{align*}
\end{problem}

\begin{problem}[4.2.6]
\begin{enumerate}[label=(\alph*)]
	\item Two ordered bases are equivalently oriented iff the linear map defined by mapping between them has positive determinant. Define $ A: V \to V, v_i \mapsto v_i'$. Then notice
		\begin{align*}
			T(v_1',\ldots,v_n') &= T(Av_1,\ldots,Av_n) \\
			&= A^* T(v_1,\ldots,v_n) \\
			&= \det A T(v_1,\ldots,v_n)
		\end{align*}
		It follows that $\det A > 0 \iff T(v_1,\ldots,v_n)$ and $ T(v_1',\ldots,v_n')$ have the same sign.  
	\item By part a), the sign of $ T$ is well-defined independent of the choice of positively oriented basis.
	\item We define the orientation of $ V$ by the orientation of its ordered basis. An ordered basis $ \{v_1,\ldots,v_n\} $ is positively oriented if the sign of $ T(v_1,\ldots,v_n)$ is positive. This is again well-defined by part a). Given an orientation on $ \Lambda^{n}(V^* )$, any two alternating tensors are scalar multiples of each other, so their sign difference will also be passed to the orientation of the vectors $ v_1,\ldots,v_n$, making it well-defined$.

\end{enumerate}
\end{problem}

\begin{problem}[4.2.7]
Define $ D (A):= \det A^{T}$ so $ D \in \Lambda^{k}(\rr^{k*})$. We wish to show that $ D$ is multilinear, alternating, and  $ D(I) = 1$. Since  $ \det $ is multilinear in the row, we see that $ D$ is multilinear in the columns. Let  $ P$ be the permutation matrix that swap  $ i,j$th row of  $ A$. Then
 \begin{align*}
	D(PA) &= \det (PA)^{T} \\
	&= \det A^{T} \det P^{T} \\
	&= \det A^{T} (-1) && \text{ swap row viewed as swap col} \\
	&= - D(A) 
\end{align*}
So $ D$ is alternating. Finally  $ D(I) = \det (I^{T}) = \det I = 1$. Hence by uniqueness of $ \det $, $ D = \det $ so $ D(A) = \det A^{T} = \det A$.
\end{problem}

\begin{problem}[P165 Exercise]
Since $ \phi: Y \to \rr$, $ f^* \phi = \phi \circ f$. Therefore,
\begin{align*}
	f^* (d \phi) &= d \phi \circ df \\
		     &= d (\phi \circ f) && \text{ chain rule}  \\
	&= d(f^* \phi) 
\end{align*}
\end{problem}
\end{document}
