\documentclass[12pt,class=article,crop=false]{standalone} 
%Fall 2022
% Some basic packages
\usepackage{standalone}[subpreambles=true]
\usepackage[utf8]{inputenc}
\usepackage[T1]{fontenc}
\usepackage{textcomp}
\usepackage[english]{babel}
\usepackage{url}
\usepackage{graphicx}
%\usepackage{quiver}
\usepackage{float}
\usepackage{enumitem}
\usepackage{lmodern}
\usepackage{comment}
\usepackage{hyperref}
\usepackage[usenames,svgnames,dvipsnames]{xcolor}
\usepackage[margin=1in]{geometry}
\usepackage{pdfpages}

\pdfminorversion=7

% Don't indent paragraphs, leave some space between them
\usepackage{parskip}

% Hide page number when page is empty
\usepackage{emptypage}
\usepackage{subcaption}
\usepackage{multicol}
\usepackage[b]{esvect}

% Math stuff
\usepackage{amsmath, amsfonts, mathtools, amsthm, amssymb}
\usepackage{bbm}
\usepackage{stmaryrd}
\allowdisplaybreaks

% Fancy script capitals
\usepackage{mathrsfs}
\usepackage{cancel}
% Bold math
\usepackage{bm}
% Some shortcuts
\newcommand{\rr}{\ensuremath{\mathbb{R}}}
\newcommand{\zz}{\ensuremath{\mathbb{Z}}}
\newcommand{\qq}{\ensuremath{\mathbb{Q}}}
\newcommand{\nn}{\ensuremath{\mathbb{N}}}
\newcommand{\ff}{\ensuremath{\mathbb{F}}}
\newcommand{\cc}{\ensuremath{\mathbb{C}}}
\newcommand{\ee}{\ensuremath{\mathbb{E}}}
\newcommand{\hh}{\ensuremath{\mathbb{H}}}
\renewcommand\O{\ensuremath{\emptyset}}
\newcommand{\norm}[1]{{\left\lVert{#1}\right\rVert}}
\newcommand{\dbracket}[1]{{\left\llbracket{#1}\right\rrbracket}}
\newcommand{\ve}[1]{{\bm{#1}}}
\newcommand\allbold[1]{{\boldmath\textbf{#1}}}
\DeclareMathOperator{\lcm}{lcm}
\DeclareMathOperator{\im}{im}
\DeclareMathOperator{\coim}{coim}
\DeclareMathOperator{\dom}{dom}
\DeclareMathOperator{\tr}{tr}
\DeclareMathOperator{\rank}{rank}
\DeclareMathOperator*{\var}{Var}
\DeclareMathOperator*{\ev}{E}
\DeclareMathOperator{\dg}{deg}
\DeclareMathOperator{\aff}{aff}
\DeclareMathOperator{\conv}{conv}
\DeclareMathOperator{\inte}{int}
\DeclareMathOperator*{\argmin}{argmin}
\DeclareMathOperator*{\argmax}{argmax}
\DeclareMathOperator{\graph}{graph}
\DeclareMathOperator{\sgn}{sgn}
\DeclareMathOperator*{\Rep}{Rep}
\DeclareMathOperator{\Proj}{Proj}
\DeclareMathOperator{\mat}{mat}
\DeclareMathOperator{\diag}{diag}
\DeclareMathOperator{\aut}{Aut}
\DeclareMathOperator{\gal}{Gal}
\DeclareMathOperator{\inn}{Inn}
\DeclareMathOperator{\edm}{End}
\DeclareMathOperator{\Hom}{Hom}
\DeclareMathOperator{\ext}{Ext}
\DeclareMathOperator{\tor}{Tor}
\DeclareMathOperator{\Span}{Span}
\DeclareMathOperator{\Stab}{Stab}
\DeclareMathOperator{\cont}{cont}
\DeclareMathOperator{\Ann}{Ann}
\DeclareMathOperator{\Div}{div}
\DeclareMathOperator{\curl}{curl}
\DeclareMathOperator{\nat}{Nat}
\DeclareMathOperator{\gr}{Gr}
\DeclareMathOperator{\vect}{Vect}
\DeclareMathOperator{\id}{id}
\DeclareMathOperator{\Mod}{Mod}
\DeclareMathOperator{\sign}{sign}
\DeclareMathOperator{\Surf}{Surf}
\DeclareMathOperator{\fcone}{fcone}
\DeclareMathOperator{\Rot}{Rot}
\DeclareMathOperator{\grad}{grad}
\DeclareMathOperator{\atan2}{atan2}
\DeclareMathOperator{\Ric}{Ric}
\let\vec\relax
\DeclareMathOperator{\vec}{vec}
\let\Re\relax
\DeclareMathOperator{\Re}{Re}
\let\Im\relax
\DeclareMathOperator{\Im}{Im}
% Put x \to \infty below \lim
\let\svlim\lim\def\lim{\svlim\limits}

%wide hat
\usepackage{scalerel,stackengine}
\stackMath
\newcommand*\wh[1]{%
\savestack{\tmpbox}{\stretchto{%
  \scaleto{%
    \scalerel*[\widthof{\ensuremath{#1}}]{\kern-.6pt\bigwedge\kern-.6pt}%
    {\rule[-\textheight/2]{1ex}{\textheight}}%WIDTH-LIMITED BIG WEDGE
  }{\textheight}% 
}{0.5ex}}%
\stackon[1pt]{#1}{\tmpbox}%
}
\parskip 1ex

%Make implies and impliedby shorter
\let\implies\Rightarrow
\let\impliedby\Leftarrow
\let\iff\Leftrightarrow
\let\epsilon\varepsilon

% Add \contra symbol to denote contradiction
\usepackage{stmaryrd} % for \lightning
\newcommand\contra{\scalebox{1.5}{$\lightning$}}

% \let\phi\varphi

% Command for short corrections
% Usage: 1+1=\correct{3}{2}

\definecolor{correct}{HTML}{009900}
\newcommand\correct[2]{\ensuremath{\:}{\color{red}{#1}}\ensuremath{\to }{\color{correct}{#2}}\ensuremath{\:}}
\newcommand\green[1]{{\color{correct}{#1}}}

% horizontal rule
\newcommand\hr{
    \noindent\rule[0.5ex]{\linewidth}{0.5pt}
}

% hide parts
\newcommand\hide[1]{}

% si unitx
\usepackage{siunitx}
\sisetup{locale = FR}

%allows pmatrix to stretch
\makeatletter
\renewcommand*\env@matrix[1][\arraystretch]{%
  \edef\arraystretch{#1}%
  \hskip -\arraycolsep
  \let\@ifnextchar\new@ifnextchar
  \array{*\c@MaxMatrixCols c}}
\makeatother

\renewcommand{\arraystretch}{0.8}

\renewcommand{\baselinestretch}{1.5}

\usepackage{graphics}
\usepackage{epstopdf}

\RequirePackage{hyperref}
%%
%% Add support for color in order to color the hyperlinks.
%% 
\hypersetup{
  colorlinks = true,
  urlcolor = blue,
  citecolor = blue
}
%%fakesection Links
\hypersetup{
    colorlinks,
    linkcolor={red!50!black},
    citecolor={green!50!black},
    urlcolor={blue!80!black}
}
%customization of cleveref
\RequirePackage[capitalize,nameinlink]{cleveref}[0.19]

% Per SIAM Style Manual, "section" should be lowercase
\crefname{section}{section}{sections}
\crefname{subsection}{subsection}{subsections}
\Crefname{section}{Section}{Sections}
\Crefname{subsection}{Subsection}{Subsections}

% Per SIAM Style Manual, "Figure" should be spelled out in references
\Crefname{figure}{Figure}{Figures}

% Per SIAM Style Manual, don't say equation in front on an equation.
\crefformat{equation}{\textup{#2(#1)#3}}
\crefrangeformat{equation}{\textup{#3(#1)#4--#5(#2)#6}}
\crefmultiformat{equation}{\textup{#2(#1)#3}}{ and \textup{#2(#1)#3}}
{, \textup{#2(#1)#3}}{, and \textup{#2(#1)#3}}
\crefrangemultiformat{equation}{\textup{#3(#1)#4--#5(#2)#6}}%
{ and \textup{#3(#1)#4--#5(#2)#6}}{, \textup{#3(#1)#4--#5(#2)#6}}{, and \textup{#3(#1)#4--#5(#2)#6}}

% But spell it out at the beginning of a sentence.
\Crefformat{equation}{#2Equation~\textup{(#1)}#3}
\Crefrangeformat{equation}{Equations~\textup{#3(#1)#4--#5(#2)#6}}
\Crefmultiformat{equation}{Equations~\textup{#2(#1)#3}}{ and \textup{#2(#1)#3}}
{, \textup{#2(#1)#3}}{, and \textup{#2(#1)#3}}
\Crefrangemultiformat{equation}{Equations~\textup{#3(#1)#4--#5(#2)#6}}%
{ and \textup{#3(#1)#4--#5(#2)#6}}{, \textup{#3(#1)#4--#5(#2)#6}}{, and \textup{#3(#1)#4--#5(#2)#6}}

% Make number non-italic in any environment.
\crefdefaultlabelformat{#2\textup{#1}#3}

% Environments
\makeatother
% For box around Definition, Theorem, \ldots
%%fakesection Theorems
\usepackage{thmtools}
\usepackage[framemethod=TikZ]{mdframed}

\theoremstyle{definition}
\mdfdefinestyle{mdbluebox}{%
	roundcorner = 10pt,
	linewidth=1pt,
	skipabove=12pt,
	innerbottommargin=9pt,
	skipbelow=2pt,
	nobreak=true,
	linecolor=blue,
	backgroundcolor=TealBlue!5,
}
\declaretheoremstyle[
	headfont=\sffamily\bfseries\color{MidnightBlue},
	mdframed={style=mdbluebox},
	headpunct={\\[3pt]},
	postheadspace={0pt}
]{thmbluebox}

\mdfdefinestyle{mdredbox}{%
	linewidth=0.5pt,
	skipabove=12pt,
	frametitleaboveskip=5pt,
	frametitlebelowskip=0pt,
	skipbelow=2pt,
	frametitlefont=\bfseries,
	innertopmargin=4pt,
	innerbottommargin=8pt,
	nobreak=false,
	linecolor=RawSienna,
	backgroundcolor=Salmon!5,
}
\declaretheoremstyle[
	headfont=\bfseries\color{RawSienna},
	mdframed={style=mdredbox},
	headpunct={\\[3pt]},
	postheadspace={0pt},
]{thmredbox}

\declaretheorem[%
style=thmbluebox,name=Theorem,numberwithin=section]{thm}
\declaretheorem[style=thmbluebox,name=Lemma,sibling=thm]{lem}
\declaretheorem[style=thmbluebox,name=Proposition,sibling=thm]{prop}
\declaretheorem[style=thmbluebox,name=Corollary,sibling=thm]{coro}
\declaretheorem[style=thmredbox,name=Example,sibling=thm]{eg}

\mdfdefinestyle{mdgreenbox}{%
	roundcorner = 10pt,
	linewidth=1pt,
	skipabove=12pt,
	innerbottommargin=9pt,
	skipbelow=2pt,
	nobreak=true,
	linecolor=ForestGreen,
	backgroundcolor=ForestGreen!5,
}

\declaretheoremstyle[
	headfont=\bfseries\sffamily\color{ForestGreen!70!black},
	bodyfont=\normalfont,
	spaceabove=2pt,
	spacebelow=1pt,
	mdframed={style=mdgreenbox},
	headpunct={ --- },
]{thmgreenbox}

\declaretheorem[style=thmgreenbox,name=Definition,sibling=thm]{defn}

\mdfdefinestyle{mdgreenboxsq}{%
	linewidth=1pt,
	skipabove=12pt,
	innerbottommargin=9pt,
	skipbelow=2pt,
	nobreak=true,
	linecolor=ForestGreen,
	backgroundcolor=ForestGreen!5,
}
\declaretheoremstyle[
	headfont=\bfseries\sffamily\color{ForestGreen!70!black},
	bodyfont=\normalfont,
	spaceabove=2pt,
	spacebelow=1pt,
	mdframed={style=mdgreenboxsq},
	headpunct={},
]{thmgreenboxsq}
\declaretheoremstyle[
	headfont=\bfseries\sffamily\color{ForestGreen!70!black},
	bodyfont=\normalfont,
	spaceabove=2pt,
	spacebelow=1pt,
	mdframed={style=mdgreenboxsq},
	headpunct={},
]{thmgreenboxsq*}

\mdfdefinestyle{mdblackbox}{%
	skipabove=8pt,
	linewidth=3pt,
	rightline=false,
	leftline=true,
	topline=false,
	bottomline=false,
	linecolor=black,
	backgroundcolor=RedViolet!5!gray!5,
}
\declaretheoremstyle[
	headfont=\bfseries,
	bodyfont=\normalfont\small,
	spaceabove=0pt,
	spacebelow=0pt,
	mdframed={style=mdblackbox}
]{thmblackbox}

\theoremstyle{plain}
\declaretheorem[name=Question,sibling=thm,style=thmblackbox]{ques}
\declaretheorem[name=Remark,sibling=thm,style=thmgreenboxsq]{remark}
\declaretheorem[name=Remark,sibling=thm,style=thmgreenboxsq*]{remark*}
\newtheorem{ass}[thm]{Assumptions}

\theoremstyle{definition}
\newtheorem*{problem}{Problem}
\newtheorem{claim}[thm]{Claim}
\theoremstyle{remark}
\newtheorem*{case}{Case}
\newtheorem*{notation}{Notation}
\newtheorem*{note}{Note}
\newtheorem*{motivation}{Motivation}
\newtheorem*{intuition}{Intuition}
\newtheorem*{conjecture}{Conjecture}

% Make section starts with 1 for report type
%\renewcommand\thesection{\arabic{section}}

% End example and intermezzo environments with a small diamond (just like proof
% environments end with a small square)
\usepackage{etoolbox}
\AtEndEnvironment{vb}{\null\hfill$\diamond$}%
\AtEndEnvironment{intermezzo}{\null\hfill$\diamond$}%
% \AtEndEnvironment{opmerking}{\null\hfill$\diamond$}%

% Fix some spacing
% http://tex.stackexchange.com/questions/22119/how-can-i-change-the-spacing-before-theorems-with-amsthm
\makeatletter
\def\thm@space@setup{%
  \thm@preskip=\parskip \thm@postskip=0pt
}

% Fix some stuff
% %http://tex.stackexchange.com/questions/76273/multiple-pdfs-with-page-group-included-in-a-single-page-warning
\pdfsuppresswarningpagegroup=1


% My name
\author{Jaden Wang}



\begin{document}
\section{Immersions and Embeddings}

How to put one manifold inside another?

\begin{defn}
A \allbold{immersion} $ f: M \to N$ is a mapping with $ \rank (d_pf = \dim M)$ for all $ p \in M$.
\end{defn}

\begin{remark}
$ f$ is always locally 1-1 by the rank theorem.
\end{remark}

\begin{defn}
An \allbold{embedding} is an immersion which is globally 1-1 and a diffeomorphism onto its image (wrt subspace topology). 
\end{defn}
\begin{eg}[immersion]
Given a curve $ f:(a,b) \to \rr^{n}, f'\neq 0$. 
\end{eg}
\begin{eg}
A 1-1 immersion is not always an embedding. Let $ f: (a,b) \to \rr^2$ be a figure-8 curve without intersecting. But it is not an embedding due to open neighborhood around origin. We have 4 connected components if we remove origin in the image.
\end{eg}
\begin{prop}
If $ M$ is compact, then a 1-1 immersion is always an embedding. More generally, let $ X$ be a compact topological space,  $ Y$ be Hausdorff, then any 1-1 map  $ f: X \to Y$ is a homeomorphism onto its image.
\end{prop}
\begin{proof}
It suffices to show that $ f^{-1}$ is continuous or show that $ f$ is open. Given an open set $ U \subseteq X$, so $ X\setminus U$ is closed so it is compact. So $ f(X \setminus U)$ is compact and therefore closed since $ Y$ is Hausdorff. Hence by $ f$ 1-1, $ f(U) = f(X) \setminus f(X\setminus U)$ is open.
\end{proof}

\begin{defn}
	A topological \allbold{immersion} is a locally 1-1 continuous map. A topological \allbold{embedding} is a 1-1 immersion which is a homeomorphism onto the image. 
\end{defn}

\begin{thm}[Whitney Embedding]
Any (smooth) manifold $ M^{n}$ maybe (smoothly) embedded in $ \rr^{2n}$ and immersed in $ \rr^{2n-1}$. 
\end{thm}
\begin{remark}
$ 2n$ is in general a tight bound.
\end{remark}
\begin{eg}
$ \rr P^2$: Exercise 4 of lecture notes 3. $ f: \rr^3 \to \rr^{4}, (x,y,z)\mapsto (xy, yz,xz,x^2+2y^2+3z^2)$. When you restricted $ f$ to  $ S^2$, then antipodal points yield the same thing.

For immersion of $ \rr P^2$ into $ \rr^3$, google Boy's surface.
\end{eg}

\begin{proof}
Any compact manifold $ M^{n}$ may be topologically embedded in $ \rr^{N}$ for $ N$ sufficiently large). Theorem 5 LN3.

Idea: glue $ m$ finite charts together to construct an embedding into $ \rr^{n} \times  \cdots \times \rr^{n}$ $ m+1$ times. There exists $ V_i \subseteq U_i$ s.t.\ $ \{V_i\} $ still covers $ M$. Define  $ \lambda_i: U_i \to \rr$ s.t.\ $ \lambda_i=1$ on $ V_i$ and 0 elsewhere. Define $ f_i: M \to \rr^{n}, p \mapsto \lambda_i(p) \phi_i(p)$. Then $ f(p):=(\lambda_1,\ldots, \lambda_m,f_1,\ldots,f_m)$.

\begin{claim}
$ f$ is 1-1.
\end{claim}
Suppose $ f(p) = f(q)$, that is,  $ \lambda_i(p)= \lambda_i(q), f_i(p) = f_i(q)$. Since $ p \in V_j$ for some $ j$, we have  $ \lambda_j(p) = 1 = \lambda_j(q)$ which implies that $ q \in V_j$. But since $ f$ is 1-1 on each  $ V_i$, so
\begin{align*}
	\lambda_j(p) \phi_j(p) &= f_j(p) = f_j(q) = \lambda_j(q) \phi_j(q) \\
	\phi_j(p) &= \phi_j(q) \\
	p&= q 
\end{align*}
More details: $ V_i:= \phi_^{-1} (\int B^{n})$ unit ball in $ \rr^{n}$. Let $ \lambda: \rr^{n} \to \rr$, $ \lambda \neq 0 $ on $ B^{n}(1)$ and $ \lambda=0$ on $ \rr^{n} - B^{n}(2)$. Then $ \lambda_i: M \to \rr, \lambda_i(p) := \begin{cases}
	\lambda(\phi_i(p)),& p \in U_i\\
	0 & \text{ else}\\ 
\end{cases}$ 
\begin{claim}
$ \lambda_i$ is continuous.
\end{claim}

$ K_i = \phi_i ^{-1}(B^{n}(2))$ so $ K_i$ is compact and therefore closed by Hausdorff. So $ M - K_i$ is open. So $ \{U_i,M - K_i\} $ is an open cover for $ M$,  $ \lambda_i$ is continuous on $ U_i$, is continuous on $ M - K_i$ so $ \lambda_i$ is continuous.

\begin{claim}
$ \rank d_pf =n \ \forall \ p \in M$.
\end{claim}

$ p \in V_i$ for some $ i$,  $ f_i(p) = \lambda_i \phi_i(p) = \phi_i(p)$ which has rank $ n$ so is the derivative. The rank of  $ f$ cannot be less than  $ f_i$, but the dimension of the codomain in $ n$ so the rank has to be  $ n$.

Now for the proof, there exists an embedding $ f: M \to \rr^{N}$, $ N$ large, 
 \begin{claim}
If $ N > 2n+1$, then there exists a unit vector  $  u \in S^{n-1}$ s.t.\ $ \pi_u \circ f: M \to \rr^{n-1}$ is an embedding, where $ \pi_k: \rr^{N} \to H_n$ (hyperplane) be the orthogonal projection.
\end{claim}
Note that $ u$ needs to be chosen  s.t.\ 
\begin{enumerate}[label=(\arabic*)]
	\item For all $ (p,q) \in M$, $ \frac{ f(p) - f(q)}{|f(p)-f(q)| } $ is not parallel to $ u$.
	\item For all  $ p \in M$, $ u \not\in T_pM$.
\end{enumerate}
Let $ \Delta_M$ be the diagonal of $ M$, by Hausdorff  $ \Delta_M$ is closed, so $ M\times M - \Delta_M$ is open so it is a submanifold. $ \dim (M \times M - \Delta_M) = 2n < N-1 = \dim S^{N-1}$ by assumption.  Define $ \sigma: M \times M - \Delta_M \to S^{N-1}, (p,q) \mapsto \frac{f(p)-f(q)}{ \norm{ f(p)-f(q)} }$. Then $ \mu( \sigma(M \times M - \Delta_M)) = 0$ by lemma. Therefore, there exists a $ u \in S^{N-1}$ s.t.\  $ u \not\in \sigma(M \times M- \Delta_M)$. Hence $ \pi_u|_M$ is 1-1.

Exercise: the unit tangent bundle $ \dim T^{1}f(M) = 2n-1$. The norm function is smooth and 1 is a regular value. 
\end{proof}

\begin{lem}
If $ f: M^{m} \to N^{n}$ is a $ C^{1}$ map for $ n>m$, then $ f(M)$ is not surjective. In particular, $ f(M)$ has measure zero in $ N$.
\end{lem}

\begin{proof}

\end{proof}

\end{document}
