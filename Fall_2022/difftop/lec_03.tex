\documentclass[12pt,class=article,crop=false]{standalone} 
\newcommand{\alert}[1]{{\bf \color{red} [Alert:] #1}}
\newcommand{\todo}[1]{{\bf \color{orange} [TODO:] #1}}
\newcommand{\real}[1][]{\mathbb{R}^{#1}}
\newcommand{\myeqn}[1]{(\ref{#1})}
\newcommand{\myex}[1]{Example \ref{#1}}
\newcommand{\defeq}{\stackrel{\mathrm{def}}{=}}
\newcommand{\parder}[2]{\frac{\partial #1}{\partial #2}}
\newcommand{\Lie}[3][]{\mathsf{L}_{#3}^{#1} #2}
\newcommand{\LieA}[1]{\mathsf{Lie}(#1)}
\newcommand{\lieder}[2]{\mathcal{L}_{#2} #1}
\renewcommand{\t}{^{\mbox{\tiny\sf T}}}
\newcommand{\trans}{^{\mbox{\tiny\sf T}}}
\newcommand{\markup}[1]{\{\textbf{#1}\}}
\newcommand{\msub}[1]{_\mathrm{#1}}
\newcommand{\msup}[1]{^\mathrm{#1}}
\newcommand{\inv}[1]{#1^{-1}}
\newcommand{\pinv}[1]{{#1}^{+}}
\newcommand{\myfracA}[2]{\displaystyle{\frac{#1}{#2}}}
\newcommand{\myfracB}[2]{{#1}/{#2}}
\newcommand{\mydiffA}[1]{\dot{#1}}
\newcommand{\mydiffB}[2]{\myfracA{\mathrm{d}{#1}}{\mathrm{d}{#2}}}
\newcommand{\ball}[2]{\mathcal{B}_{#1}\left(#2\right)}
\newcommand{\acos}[1]{\cos^{-1}\left(#1\right)}
\newcommand{\asin}[1]{\sin^{-1}\left(#1\right)}
\newcommand{\mani}{\mathcal{M}}
\newcommand{\tang}[2]{\mathsf{T}_{#1} #2}
\newcommand{\LieB}[2]{[ #1, #2 ]}
\newcommand{\LieBad}[3][]{\mathsf{ad}_{#2}^{#1} #3}
\newcommand{\ReachVT}{\mathcal{R}^V_T}
\newcommand{\ReachVt}{\mathcal{R}^V_t}
\newcommand{\ReachVTe}{\mathcal{R}^V_{\le T}}
\newcommand{\ReachT}{\mathcal{R}_T}
\newcommand{\Reacht}{\mathcal{R}_t}
\newcommand{\ReachTe}{\mathcal{R}_{\le T}}
\newcommand{\accLA}[1]{\mathsf{Lie}(#1)}
\newcommand{\accD}{\Delta_{\mathcal{F}}}
\newcommand{\accSA}{\mathsf{Lie}(\mathcal{G},f)}
\newcommand{\accDS}{\Delta_{\mathcal{G}}}
\newcommand{\eval}[3]{\mathsf{Ev}^{#2}_{#1}\left( #3 \right)}
\newcommand{\stlc}{\textsc{stlc}}
\newcommand{\clf}{\textsc{clf}}
\newcommand{\jqlf}{\textsc{jqlf}}
\newcommand{\dlas}{\textsc{dlas}}
\newcommand{\Ad}[2]{\mathsf{Ad}_{#1} #2}
\newcommand{\xe}{\ensuremath{x_e}}
\newcommand{\lebg}[1]{\mathcal{L}_{#1}}
\newcommand{\lebgx}[1]{\mathcal{L}_{#1 \mathrm{e}}}
\newcommand{\dom}{D}
\newcommand{\domT}{[t_0,\infty) \times D}
\newcommand{\rarrow}{\rightarrow}
\renewcommand{\d}{\mathrm{d}}
\renewcommand{\Re}{\mathbb{R}}
\newcommand{\C}{\mathrm{C}}

\newcommand{\QED}{{\unskip\nobreak\hfil\penalty50\hskip2em\vadjust{}
		\nobreak\hfil$\Box$\parfillskip=0pt\finalhyphendemerits=0\par}\vspace{0.1cm}}
\newcommand{\eoEx}{{\unskip\nobreak\hfil\penalty50\hskip0em\vadjust{}
		\nobreak\hfil$\Large\Diamond$\parfillskip=0pt\finalhyphendemerits=0\par}\vspace{0.1cm}}

\newcommand{\sgn}{\ensuremath{\operatorname{sgn}}}
\newcommand{\sat}{\ensuremath{\operatorname{sat}}}

\newcommand{\half}{\frac{1}{2}}
\newcommand{\shalf}{\mbox{$\frac{1}{2}$}}
\newcommand{\marcom}[1]{\marginpar{\footnotesize #1}}
\newcommand{\der}{\mathrm{D}}
\newcommand{\e}{\mathrm{e}}
\newcommand{\dt}{\mathrm{d}t}

\newcommand{\cA}{\ensuremath{\mathcal{A}}}
\newcommand{\cB}{\ensuremath{\mathcal{B}}}
\newcommand{\cG}{\ensuremath{\mathcal{G}}}
\newcommand{\cK}{\ensuremath{\mathcal{K}}}
\newcommand{\cW}{\ensuremath{\mathcal{W}}}
\newcommand{\cZ}{\ensuremath{\mathcal{Z}}}
\newcommand{\cS}{\ensuremath{\mathcal{S}}}
\newcommand{\cD}{\ensuremath{\mathcal{D}}}
\newcommand{\cP}{\ensuremath{\mathcal{P}}}
\newcommand{\cV}{\ensuremath{\mathcal{V}}}
\newcommand{\cL}{\ensuremath{\mathcal{L}}}
\newcommand{\cN}{\ensuremath{\mathcal{N}}}
\newcommand{\cI}{\ensuremath{\mathcal{I}}}
\newcommand{\cR}{\ensuremath{\mathcal{R}}}
\newcommand{\cM}{\ensuremath{\mathcal{M}}}
\newcommand{\cC}{\ensuremath{\mathcal{C}}}
\newcommand{\cF}{\ensuremath{\mathcal{F}}}
\newcommand{\cH}{\ensuremath{\mathcal{H}}}
\newcommand{\cO}{\ensuremath{\mathcal{O}}}
\newcommand{\cX}{\ensuremath{\mathcal{X}}}
\newcommand{\cY}{\ensuremath{\mathcal{Y}}}
\newcommand{\Ci}{\ensuremath{\mathcal{C}^\infty}}
\newcommand{\ISS}{\textsc{iss}}
\newcommand{\LISS}{\textsc{liss}}
\newcommand{\GAS}{\textsc{gas}}
\newcommand{\GS}{\textsc{gs}}
\newcommand{\LES}{\textsc{les}}
\newcommand{\GUAS}{\textsc{guas}}
\newcommand{\BIBO}{\textsc{bibo}}
\newcommand{\spec}{\ensuremath{\operatorname{spec}}}
\newcommand{\spn}{\ensuremath{\operatorname{span}}}
\renewcommand{\i}{\mathrm{i\,}}

\renewcommand{\implies}{\Rightarrow}

\renewcommand{\theenumi}{$\roman{enumi})$}
\renewcommand{\labelenumi}{\theenumi}

\font\ptmten=zptmcmrm scaled 1200
\newcommand{\w}{\mbox{{\ptmten w}}}
\newcommand{\z}{\mbox{{\ptmten z}}}
\renewcommand{\Re}{\mathbb{R}}

\newcommand{\cl}{\operatorname{cl}}
\newcommand{\intr}{\operatorname{int}}
\newcommand{\rank}{\operatorname{rank}}
\newcommand{\co}{\operatorname{co}}
\newcommand{\aff}{\operatorname{aff}}

\theoremstyle{plain}
\newtheorem{theorem}{Theorem}[chapter]
\newtheorem{claim}[theorem]{Claim}
\newtheorem{corollary}[theorem]{Corollary}
\newtheorem{prop}[theorem]{Proposition}
\newtheorem{fact}[theorem]{Fact}
\newtheorem{lemma}[theorem]{Lemma}

\newtheorem{remark}{Remark}[chapter]

\theoremstyle{definition}
\newtheorem{assume}[theorem]{Assumption}
\newtheorem{defn}[theorem]{Definition}
\newtheorem{problem}[theorem]{Problem}
\newtheorem{exercise}{Exercise}
\newtheorem{example}[theorem]{Example}


\begin{document}
\section{Immersions and Embeddings}

How to put one manifold inside another?

\begin{defn}
A \allbold{immersion} $ f: M \to N$ is a mapping with $ \rank (d_pf = \dim M)$ for all $ p \in M$.
\end{defn}

\begin{remark}
$ f$ is always locally 1-1 by the rank theorem.
\end{remark}

\begin{defn}
An \allbold{embedding} is an immersion which is globally 1-1 and a diffeomorphism onto its image (wrt subspace topology). 
\end{defn}
\begin{eg}[immersion]
Given a curve $ f:(a,b) \to \rr^{n}, f'\neq 0$. 
\end{eg}
\begin{eg}
A 1-1 immersion is not always an embedding. Let $ f: (a,b) \to \rr^2$ be a figure-8 curve without intersecting. But it is not an embedding due to open neighborhood around origin. We have 4 connected components if we remove origin in the image.
\end{eg}
\begin{prop}
If $ M$ is compact, then a 1-1 immersion is always an embedding. More generally, let $ X$ be a compact topological space,  $ Y$ be Hausdorff, then any 1-1 map  $ f: X \to Y$ is a homeomorphism onto its image.
\end{prop}
\begin{proof}
It suffices to show that $ f^{-1}$ is continuous or show that $ f$ is open. Given an open set $ U \subseteq X$, so $ X\setminus U$ is closed so it is compact. So $ f(X \setminus U)$ is compact and therefore closed since $ Y$ is Hausdorff. Hence by $ f$ 1-1, $ f(U) = f(X) \setminus f(X\setminus U)$ is open.
\end{proof}

\begin{defn}
	A topological \allbold{immersion} is a locally 1-1 continuous map. A topological \allbold{embedding} is a 1-1 immersion which is a homeomorphism onto the image. 
\end{defn}

\begin{thm}[Whitney Embedding]
Any (smooth) manifold $ M^{n}$ maybe (smoothly) embedded in $ \rr^{2n}$ and immersed in $ \rr^{2n-1}$. 
\end{thm}
\begin{remark}
$ 2n$ is in general a tight bound.
\end{remark}
\begin{eg}
$ \rr P^2$: Exercise 4 of lecture notes 3. $ f: \rr^3 \to \rr^{4}, (x,y,z)\mapsto (xy, yz,xz,x^2+2y^2+3z^2)$. When you restricted $ f$ to  $ S^2$, then antipodal points yield the same thing.

For immersion of $ \rr P^2$ into $ \rr^3$, google Boy's surface.
\end{eg}

\begin{proof}
Any compact manifold $ M^{n}$ may be topologically embedded in $ \rr^{N}$ for $ N$ sufficiently large). Theorem 5 LN3.

Idea: glue $ m$ finite charts together to construct an embedding into $ \rr^{n} \times  \cdots \times \rr^{n}$ $ m+1$ times. There exists $ V_i \subseteq U_i$ s.t.\ $ \{V_i\} $ still covers $ M$. Define  $ \lambda_i: U_i \to \rr$ s.t.\ $ \lambda_i=1$ on $ V_i$ and 0 elsewhere. Define $ f_i: M \to \rr^{n}, p \mapsto \lambda_i(p) \phi_i(p)$. Then $ f(p):=(\lambda_1,\ldots, \lambda_m,f_1,\ldots,f_m)$.

\begin{claim}
$ f$ is 1-1.
\end{claim}
Suppose $ f(p) = f(q)$, that is,  $ \lambda_i(p)= \lambda_i(q), f_i(p) = f_i(q)$. Since $ p \in V_j$ for some $ j$, we have  $ \lambda_j(p) = 1 = \lambda_j(q)$ which implies that $ q \in V_j$. But since $ f$ is 1-1 on each  $ V_i$, so
\begin{align*}
	\lambda_j(p) \phi_j(p) &= f_j(p) = f_j(q) = \lambda_j(q) \phi_j(q) \\
	\phi_j(p) &= \phi_j(q) \\
	p&= q 
\end{align*}
More details: $ V_i:= \phi_^{-1} (\int B^{n})$ unit ball in $ \rr^{n}$. Let $ \lambda: \rr^{n} \to \rr$, $ \lambda \neq 0 $ on $ B^{n}(1)$ and $ \lambda=0$ on $ \rr^{n} - B^{n}(2)$. Then $ \lambda_i: M \to \rr, \lambda_i(p) := \begin{cases}
	\lambda(\phi_i(p)),& p \in U_i\\
	0 & \text{ else}\\ 
\end{cases}$ 
\begin{claim}
$ \lambda_i$ is continuous.
\end{claim}

$ K_i = \phi_i ^{-1}(B^{n}(2))$ so $ K_i$ is compact and therefore closed by Hausdorff. So $ M - K_i$ is open. So $ \{U_i,M - K_i\} $ is an open cover for $ M$,  $ \lambda_i$ is continuous on $ U_i$, is continuous on $ M - K_i$ so $ \lambda_i$ is continuous.

\begin{claim}
$ \rank d_pf =n \ \forall \ p \in M$.
\end{claim}

$ p \in V_i$ for some $ i$,  $ f_i(p) = \lambda_i \phi_i(p) = \phi_i(p)$ which has rank $ n$ so is the derivative. The rank of  $ f$ cannot be less than  $ f_i$, but the dimension of the codomain in $ n$ so the rank has to be  $ n$.

Now for the proof, there exists an embedding $ f: M \to \rr^{N}$, $ N$ large, 
 \begin{claim}
If $ N > 2n+1$, then there exists a unit vector  $  u \in S^{n-1}$ s.t.\ $ \pi_u \circ f: M \to \rr^{n-1}$ is an embedding, where $ \pi_k: \rr^{N} \to H_n$ (hyperplane) be the orthogonal projection.
\end{claim}
Note that $ u$ needs to be chosen  s.t.\ 
\begin{enumerate}[label=(\arabic*)]
	\item For all $ (p,q) \in M$, $ \frac{ f(p) - f(q)}{|f(p)-f(q)| } $ is not parallel to $ u$.
	\item For all  $ p \in M$, $ u \not\in T_pM$.
\end{enumerate}
Let $ \Delta_M$ be the diagonal of $ M$, by Hausdorff  $ \Delta_M$ is closed, so $ M\times M - \Delta_M$ is open so it is a submanifold. $ \dim (M \times M - \Delta_M) = 2n < N-1 = \dim S^{N-1}$ by assumption.  Define $ \sigma: M \times M - \Delta_M \to S^{N-1}, (p,q) \mapsto \frac{f(p)-f(q)}{ \norm{ f(p)-f(q)} }$. Then $ \mu( \sigma(M \times M - \Delta_M)) = 0$ by lemma. Therefore, there exists a $ u \in S^{N-1}$ s.t.\  $ u \not\in \sigma(M \times M- \Delta_M)$. Hence $ \pi_u|_M$ is 1-1.

Exercise: the unit tangent bundle $ \dim T^{1}f(M) = 2n-1$. The norm function is smooth and 1 is a regular value. 
\end{proof}

\begin{lem}
If $ f: M^{m} \to N^{n}$ is a $ C^{1}$ map for $ n>m$, then $ f(M)$ is not surjective. In particular, $ f(M)$ has measure zero in $ N$.
\end{lem}

\begin{proof}

\end{proof}

\end{document}
