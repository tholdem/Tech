\documentclass[12pt]{article}
%Fall 2022
% Some basic packages
\usepackage{standalone}[subpreambles=true]
\usepackage[utf8]{inputenc}
\usepackage[T1]{fontenc}
\usepackage{textcomp}
\usepackage[english]{babel}
\usepackage{url}
\usepackage{graphicx}
%\usepackage{quiver}
\usepackage{float}
\usepackage{enumitem}
\usepackage{lmodern}
\usepackage{comment}
\usepackage{hyperref}
\usepackage[usenames,svgnames,dvipsnames]{xcolor}
\usepackage[margin=1in]{geometry}
\usepackage{pdfpages}

\pdfminorversion=7

% Don't indent paragraphs, leave some space between them
\usepackage{parskip}

% Hide page number when page is empty
\usepackage{emptypage}
\usepackage{subcaption}
\usepackage{multicol}
\usepackage[b]{esvect}

% Math stuff
\usepackage{amsmath, amsfonts, mathtools, amsthm, amssymb}
\usepackage{bbm}
\usepackage{stmaryrd}
\allowdisplaybreaks

% Fancy script capitals
\usepackage{mathrsfs}
\usepackage{cancel}
% Bold math
\usepackage{bm}
% Some shortcuts
\newcommand{\rr}{\ensuremath{\mathbb{R}}}
\newcommand{\zz}{\ensuremath{\mathbb{Z}}}
\newcommand{\qq}{\ensuremath{\mathbb{Q}}}
\newcommand{\nn}{\ensuremath{\mathbb{N}}}
\newcommand{\ff}{\ensuremath{\mathbb{F}}}
\newcommand{\cc}{\ensuremath{\mathbb{C}}}
\newcommand{\ee}{\ensuremath{\mathbb{E}}}
\newcommand{\hh}{\ensuremath{\mathbb{H}}}
\renewcommand\O{\ensuremath{\emptyset}}
\newcommand{\norm}[1]{{\left\lVert{#1}\right\rVert}}
\newcommand{\dbracket}[1]{{\left\llbracket{#1}\right\rrbracket}}
\newcommand{\ve}[1]{{\bm{#1}}}
\newcommand\allbold[1]{{\boldmath\textbf{#1}}}
\DeclareMathOperator{\lcm}{lcm}
\DeclareMathOperator{\im}{im}
\DeclareMathOperator{\coim}{coim}
\DeclareMathOperator{\dom}{dom}
\DeclareMathOperator{\tr}{tr}
\DeclareMathOperator{\rank}{rank}
\DeclareMathOperator*{\var}{Var}
\DeclareMathOperator*{\ev}{E}
\DeclareMathOperator{\dg}{deg}
\DeclareMathOperator{\aff}{aff}
\DeclareMathOperator{\conv}{conv}
\DeclareMathOperator{\inte}{int}
\DeclareMathOperator*{\argmin}{argmin}
\DeclareMathOperator*{\argmax}{argmax}
\DeclareMathOperator{\graph}{graph}
\DeclareMathOperator{\sgn}{sgn}
\DeclareMathOperator*{\Rep}{Rep}
\DeclareMathOperator{\Proj}{Proj}
\DeclareMathOperator{\mat}{mat}
\DeclareMathOperator{\diag}{diag}
\DeclareMathOperator{\aut}{Aut}
\DeclareMathOperator{\gal}{Gal}
\DeclareMathOperator{\inn}{Inn}
\DeclareMathOperator{\edm}{End}
\DeclareMathOperator{\Hom}{Hom}
\DeclareMathOperator{\ext}{Ext}
\DeclareMathOperator{\tor}{Tor}
\DeclareMathOperator{\Span}{Span}
\DeclareMathOperator{\Stab}{Stab}
\DeclareMathOperator{\cont}{cont}
\DeclareMathOperator{\Ann}{Ann}
\DeclareMathOperator{\Div}{div}
\DeclareMathOperator{\curl}{curl}
\DeclareMathOperator{\nat}{Nat}
\DeclareMathOperator{\gr}{Gr}
\DeclareMathOperator{\vect}{Vect}
\DeclareMathOperator{\id}{id}
\DeclareMathOperator{\Mod}{Mod}
\DeclareMathOperator{\sign}{sign}
\DeclareMathOperator{\Surf}{Surf}
\DeclareMathOperator{\fcone}{fcone}
\DeclareMathOperator{\Rot}{Rot}
\DeclareMathOperator{\grad}{grad}
\DeclareMathOperator{\atan2}{atan2}
\DeclareMathOperator{\Ric}{Ric}
\let\vec\relax
\DeclareMathOperator{\vec}{vec}
\let\Re\relax
\DeclareMathOperator{\Re}{Re}
\let\Im\relax
\DeclareMathOperator{\Im}{Im}
% Put x \to \infty below \lim
\let\svlim\lim\def\lim{\svlim\limits}

%wide hat
\usepackage{scalerel,stackengine}
\stackMath
\newcommand*\wh[1]{%
\savestack{\tmpbox}{\stretchto{%
  \scaleto{%
    \scalerel*[\widthof{\ensuremath{#1}}]{\kern-.6pt\bigwedge\kern-.6pt}%
    {\rule[-\textheight/2]{1ex}{\textheight}}%WIDTH-LIMITED BIG WEDGE
  }{\textheight}% 
}{0.5ex}}%
\stackon[1pt]{#1}{\tmpbox}%
}
\parskip 1ex

%Make implies and impliedby shorter
\let\implies\Rightarrow
\let\impliedby\Leftarrow
\let\iff\Leftrightarrow
\let\epsilon\varepsilon

% Add \contra symbol to denote contradiction
\usepackage{stmaryrd} % for \lightning
\newcommand\contra{\scalebox{1.5}{$\lightning$}}

% \let\phi\varphi

% Command for short corrections
% Usage: 1+1=\correct{3}{2}

\definecolor{correct}{HTML}{009900}
\newcommand\correct[2]{\ensuremath{\:}{\color{red}{#1}}\ensuremath{\to }{\color{correct}{#2}}\ensuremath{\:}}
\newcommand\green[1]{{\color{correct}{#1}}}

% horizontal rule
\newcommand\hr{
    \noindent\rule[0.5ex]{\linewidth}{0.5pt}
}

% hide parts
\newcommand\hide[1]{}

% si unitx
\usepackage{siunitx}
\sisetup{locale = FR}

%allows pmatrix to stretch
\makeatletter
\renewcommand*\env@matrix[1][\arraystretch]{%
  \edef\arraystretch{#1}%
  \hskip -\arraycolsep
  \let\@ifnextchar\new@ifnextchar
  \array{*\c@MaxMatrixCols c}}
\makeatother

\renewcommand{\arraystretch}{0.8}

\renewcommand{\baselinestretch}{1.5}

\usepackage{graphics}
\usepackage{epstopdf}

\RequirePackage{hyperref}
%%
%% Add support for color in order to color the hyperlinks.
%% 
\hypersetup{
  colorlinks = true,
  urlcolor = blue,
  citecolor = blue
}
%%fakesection Links
\hypersetup{
    colorlinks,
    linkcolor={red!50!black},
    citecolor={green!50!black},
    urlcolor={blue!80!black}
}
%customization of cleveref
\RequirePackage[capitalize,nameinlink]{cleveref}[0.19]

% Per SIAM Style Manual, "section" should be lowercase
\crefname{section}{section}{sections}
\crefname{subsection}{subsection}{subsections}
\Crefname{section}{Section}{Sections}
\Crefname{subsection}{Subsection}{Subsections}

% Per SIAM Style Manual, "Figure" should be spelled out in references
\Crefname{figure}{Figure}{Figures}

% Per SIAM Style Manual, don't say equation in front on an equation.
\crefformat{equation}{\textup{#2(#1)#3}}
\crefrangeformat{equation}{\textup{#3(#1)#4--#5(#2)#6}}
\crefmultiformat{equation}{\textup{#2(#1)#3}}{ and \textup{#2(#1)#3}}
{, \textup{#2(#1)#3}}{, and \textup{#2(#1)#3}}
\crefrangemultiformat{equation}{\textup{#3(#1)#4--#5(#2)#6}}%
{ and \textup{#3(#1)#4--#5(#2)#6}}{, \textup{#3(#1)#4--#5(#2)#6}}{, and \textup{#3(#1)#4--#5(#2)#6}}

% But spell it out at the beginning of a sentence.
\Crefformat{equation}{#2Equation~\textup{(#1)}#3}
\Crefrangeformat{equation}{Equations~\textup{#3(#1)#4--#5(#2)#6}}
\Crefmultiformat{equation}{Equations~\textup{#2(#1)#3}}{ and \textup{#2(#1)#3}}
{, \textup{#2(#1)#3}}{, and \textup{#2(#1)#3}}
\Crefrangemultiformat{equation}{Equations~\textup{#3(#1)#4--#5(#2)#6}}%
{ and \textup{#3(#1)#4--#5(#2)#6}}{, \textup{#3(#1)#4--#5(#2)#6}}{, and \textup{#3(#1)#4--#5(#2)#6}}

% Make number non-italic in any environment.
\crefdefaultlabelformat{#2\textup{#1}#3}

% Environments
\makeatother
% For box around Definition, Theorem, \ldots
%%fakesection Theorems
\usepackage{thmtools}
\usepackage[framemethod=TikZ]{mdframed}

\theoremstyle{definition}
\mdfdefinestyle{mdbluebox}{%
	roundcorner = 10pt,
	linewidth=1pt,
	skipabove=12pt,
	innerbottommargin=9pt,
	skipbelow=2pt,
	nobreak=true,
	linecolor=blue,
	backgroundcolor=TealBlue!5,
}
\declaretheoremstyle[
	headfont=\sffamily\bfseries\color{MidnightBlue},
	mdframed={style=mdbluebox},
	headpunct={\\[3pt]},
	postheadspace={0pt}
]{thmbluebox}

\mdfdefinestyle{mdredbox}{%
	linewidth=0.5pt,
	skipabove=12pt,
	frametitleaboveskip=5pt,
	frametitlebelowskip=0pt,
	skipbelow=2pt,
	frametitlefont=\bfseries,
	innertopmargin=4pt,
	innerbottommargin=8pt,
	nobreak=false,
	linecolor=RawSienna,
	backgroundcolor=Salmon!5,
}
\declaretheoremstyle[
	headfont=\bfseries\color{RawSienna},
	mdframed={style=mdredbox},
	headpunct={\\[3pt]},
	postheadspace={0pt},
]{thmredbox}

\declaretheorem[%
style=thmbluebox,name=Theorem,numberwithin=section]{thm}
\declaretheorem[style=thmbluebox,name=Lemma,sibling=thm]{lem}
\declaretheorem[style=thmbluebox,name=Proposition,sibling=thm]{prop}
\declaretheorem[style=thmbluebox,name=Corollary,sibling=thm]{coro}
\declaretheorem[style=thmredbox,name=Example,sibling=thm]{eg}

\mdfdefinestyle{mdgreenbox}{%
	roundcorner = 10pt,
	linewidth=1pt,
	skipabove=12pt,
	innerbottommargin=9pt,
	skipbelow=2pt,
	nobreak=true,
	linecolor=ForestGreen,
	backgroundcolor=ForestGreen!5,
}

\declaretheoremstyle[
	headfont=\bfseries\sffamily\color{ForestGreen!70!black},
	bodyfont=\normalfont,
	spaceabove=2pt,
	spacebelow=1pt,
	mdframed={style=mdgreenbox},
	headpunct={ --- },
]{thmgreenbox}

\declaretheorem[style=thmgreenbox,name=Definition,sibling=thm]{defn}

\mdfdefinestyle{mdgreenboxsq}{%
	linewidth=1pt,
	skipabove=12pt,
	innerbottommargin=9pt,
	skipbelow=2pt,
	nobreak=true,
	linecolor=ForestGreen,
	backgroundcolor=ForestGreen!5,
}
\declaretheoremstyle[
	headfont=\bfseries\sffamily\color{ForestGreen!70!black},
	bodyfont=\normalfont,
	spaceabove=2pt,
	spacebelow=1pt,
	mdframed={style=mdgreenboxsq},
	headpunct={},
]{thmgreenboxsq}
\declaretheoremstyle[
	headfont=\bfseries\sffamily\color{ForestGreen!70!black},
	bodyfont=\normalfont,
	spaceabove=2pt,
	spacebelow=1pt,
	mdframed={style=mdgreenboxsq},
	headpunct={},
]{thmgreenboxsq*}

\mdfdefinestyle{mdblackbox}{%
	skipabove=8pt,
	linewidth=3pt,
	rightline=false,
	leftline=true,
	topline=false,
	bottomline=false,
	linecolor=black,
	backgroundcolor=RedViolet!5!gray!5,
}
\declaretheoremstyle[
	headfont=\bfseries,
	bodyfont=\normalfont\small,
	spaceabove=0pt,
	spacebelow=0pt,
	mdframed={style=mdblackbox}
]{thmblackbox}

\theoremstyle{plain}
\declaretheorem[name=Question,sibling=thm,style=thmblackbox]{ques}
\declaretheorem[name=Remark,sibling=thm,style=thmgreenboxsq]{remark}
\declaretheorem[name=Remark,sibling=thm,style=thmgreenboxsq*]{remark*}
\newtheorem{ass}[thm]{Assumptions}

\theoremstyle{definition}
\newtheorem*{problem}{Problem}
\newtheorem{claim}[thm]{Claim}
\theoremstyle{remark}
\newtheorem*{case}{Case}
\newtheorem*{notation}{Notation}
\newtheorem*{note}{Note}
\newtheorem*{motivation}{Motivation}
\newtheorem*{intuition}{Intuition}
\newtheorem*{conjecture}{Conjecture}

% Make section starts with 1 for report type
%\renewcommand\thesection{\arabic{section}}

% End example and intermezzo environments with a small diamond (just like proof
% environments end with a small square)
\usepackage{etoolbox}
\AtEndEnvironment{vb}{\null\hfill$\diamond$}%
\AtEndEnvironment{intermezzo}{\null\hfill$\diamond$}%
% \AtEndEnvironment{opmerking}{\null\hfill$\diamond$}%

% Fix some spacing
% http://tex.stackexchange.com/questions/22119/how-can-i-change-the-spacing-before-theorems-with-amsthm
\makeatletter
\def\thm@space@setup{%
  \thm@preskip=\parskip \thm@postskip=0pt
}

% Fix some stuff
% %http://tex.stackexchange.com/questions/76273/multiple-pdfs-with-page-group-included-in-a-single-page-warning
\pdfsuppresswarningpagegroup=1


% My name
\author{Jaden Wang}



\begin{document}
\centerline {\textsf{\textbf{\LARGE{Homework 9}}}}
\centerline {Jaden Wang}
\vspace{.15in}
\begin{problem}[2.3.11]
	Move $ X$ around until it satisfies the assumption.  Since $ x$ is a 1-submanifold,  take a neighborhood $ U$ of $ 0$ so that $ \phi: \phi ^{-1}(X) \subseteq \rr \to X \cap U \subseteq \rr^2$ is a diffeomorphism. Let $ \pi: X \cap U \to \rr, (x,y)\mapsto x$ be the projection to $ x$-axis. Then define $f:= \pi \circ \phi: \phi ^{-1}(X) \subseteq \rr \to \rr$. It is easy to see that $ df_p$ has rank 1 and therefore full rank for all  $ p \in \phi ^{-1}(X)$ by chain rule. That is, $ df_0$ is invertible. Consider the map $ F:X \cap U \to \Gamma(f),x\mapsto (x,f(x))$. The Jacobian of $ F$ at $ 0$ is simply $ \begin{pmatrix} I&0\\0&df_0 \end{pmatrix} $ which is invertible. Hence by IFT, there exists a neighborhood $ V$ of  $ 0$  s.t.\ $ F$ is a local diffeomorphism with $ F^{-1}$ being injective. That is, if  $ x_1 = x_2$, then $ f(x_1) = f(x_2)$ so $ f$ is a well-defined function on the Euclidean space, and  $ X= \Gamma(f)$ on $ U \cap V$.

	We wish to find a point $ y$ in $ \rr^2$ s.t.\ some $ q \in h^{-1}(y)$ makes $ dh_q $ not a submersion. First, notice that for any point $\overline{x} \in X \cap U \cap V$, we can represent it by the graph $\overline{x}= (x,f(x))$ for some $ x \in \rr$. The tangent space can be identified as the tangent line to the curve, which is spanned by $ (1,f'(x))$. The set of vectors in $ \rr^2$ normal to $ T_{\overline{x}}X$ is the line spanned by $ (-f'(x),1)$ by analytic geometry. So an element in $ NX$ is  $ ((x,f(x)),c(-f'(x),1))$ for some  $x, c \in \rr$. Then the normal bundle map becomes $ h: ((x,f(x)),c(-f'(x),1)) \mapsto (x-cf'(x),f(x)+c)$. Then $ dh$ is simply the derivative wrt to $ x$, which is $ (1-cf''(x),f'(x))$. Since we are given that $ f''(0)\neq 0$ and $ f'(0)=0$, we quickly see that setting $ x=0$, $c=\frac{1}{f''(0)} $, and $q= ((0,f(0)),\frac{1}{f''(0)}(-f'(0),1))$ makes $ dh_q$ not a submersion as it is the zero map. Hence  $ y=(0,f(0))+\frac{1}{f''(0)}(-f'(0),1) = (0,\frac{1}{ \kappa(p)})$ is a focal point as desired.
\end{problem}

\begin{problem}[2.3.14]
	Given an equivalent class of smooth curves $ [ \gamma]$ based at $ z \in Z$, since $ 0$ is in any subspace, and any curve in $ N(Z;Y)$ is determined by specifying a curve on each factor, we have the equivalence class of smooth curves $ [ (\gamma,0)]$ based at $ (z,0)$. Then $ d \sigma_{(z,0)}:[ ( \gamma,0)] \mapsto [ \sigma \circ ( \gamma,0)] = [ \gamma]$ so $ \sigma$ is a submersion.
	 \begin{align*}
		\sigma^{-1}(z) = \{(z',v) \in N(Z;Y): \sigma(z',v)= z' = z\} = \{(z,v): v \in T_zY, v \in T_zZ\}. 
	\end{align*}
	That is, it is the orthogonal complement of $ T_zZ$.
\end{problem}
\begin{problem}[2.4.4]
First we must assume $ Y$ is connected. As I said before, since manifold is locally path-connected, $ Y$ is also path-connected. Suppose $ f: X \to Y$ is homotopic to a constant function $ e_0: x \mapsto y_0$ for some $ y_0 \in Y$. If $ \dim X>0$, then $ \dim Z < \dim Y$. Hence there exists a $ y \not\in Z$. Let $ \gamma$ be a path between $ y_0$ and $ y$. Then we see that  $ f \simeq e: x \mapsto y$ by composing the homotopy with the path. Since intersection number is invariant under homotopy, and $ y \not\in Z$, we have that $ I_2(f,Z) = I_2(e,Z) = 0$.
\end{problem}

\begin{problem}[2.4.5]
Note that if $ \dim X>0$, it suffices to show that every $ f: X \to Y$ is homotopic to a constant map and then apply problem 4. Since $ Y$ is contractible, we have  $ 1_Y \simeq e_{y_0}$. But notice
\begin{align*}
	1_Y \circ f &\simeq e_{y_0} \circ f\\
	f&\simeq \widetilde{ f}
\end{align*}
where $ \widetilde{ f}: X \to Y, x\mapsto y_0$ is a constant function, as desired.

If $ \dim X=0$, $ X$ can be covered by using local diffeomorphism neighborhoods, so $ f$ maps  $ X$ to a union of open sets so $ \im f$ is open. It is also closed because $ \im f$ is compact and $ Y$ is Hausdorff. Hence $ \im f$ is clopen and by connectedness and nonemptyness (since $ f$ is a function)  $ \im f = Y$. But that means that $ \dim Y= \dim X = 0$, a contradiction.
\end{problem}

\begin{problem}[2.4.6]
Suppose $ Y$ is a compact, contractible manifold. Then let  $ f: Y \to Y$ be the identity map. Let $ Z = \{y\} $ be a single point with  $ \dim Z =0$ which is closed. Then $ I_2(f,Z) = 0$ if $ \dim Y>0$ according to Problem 5. However, since $ Z \cap f(Y) = \{y\} $ so they intersect exactly once, a contradiction. This forces $ \dim Y = 0$. Since $ Y$ is compact,  $ Y$ must be a finite set of points. Since  $ Y$ is contractible, there exists a homotopy that maps finite points to a single point,  \emph{i.e.} finite paths from points to a single point contained in the manifold. Then the paths must be the trivial path or $ Y$ would be at least dimension 1. That is, $ Y$ must be a one-point space.
\end{problem}

\begin{problem}[2.4.10]
Take any two transversal 1-manifolds in $ S^2$, then since $ S^2$ has no boundary, by classification of 1-manifold they must be loops. Homotop them so that one is contained in the north hemisphere and one is contained in the south hemisphere. Their intersection number mod 2 is clearly $ 0$. However, if we take a horizontal circle and a vertical circle in a torus, their intersection number mod 2 is 1. This is a structural difference so they cannot be diffeomorphic.
\end{problem}

\begin{problem}[2.4.17]
We wish to use the Boundary Theorem to prove the No Retraction Theorem.

Suppose $ X$ is a compact manifold with boundary, and suppose to the contrary that there is a smooth map  $ g: X \to \partial X$ s.t.\ $ \partial g: \partial X \to \partial X$ is the identity. In other words, the smooth map $ \partial g$ extends to all of $ X$. Let $ Z = \{z\} $ be a single point of $ \partial X$. Then clearly $ \partial g(\partial X) \cap Z = \{z\} $ so $ I_2(\partial g,Z)=1$. However, $ I_2( \partial g,Z) = 0$ by the Boundary Theorem, a contradiction.
\end{problem}
\end{document}
