\documentclass[12pt,class=article,crop=false]{standalone} 
%Fall 2022
% Some basic packages
\usepackage{standalone}[subpreambles=true]
\usepackage[utf8]{inputenc}
\usepackage[T1]{fontenc}
\usepackage{textcomp}
\usepackage[english]{babel}
\usepackage{url}
\usepackage{graphicx}
%\usepackage{quiver}
\usepackage{float}
\usepackage{enumitem}
\usepackage{lmodern}
\usepackage{comment}
\usepackage{hyperref}
\usepackage[usenames,svgnames,dvipsnames]{xcolor}
\usepackage[margin=1in]{geometry}
\usepackage{pdfpages}

\pdfminorversion=7

% Don't indent paragraphs, leave some space between them
\usepackage{parskip}

% Hide page number when page is empty
\usepackage{emptypage}
\usepackage{subcaption}
\usepackage{multicol}
\usepackage[b]{esvect}

% Math stuff
\usepackage{amsmath, amsfonts, mathtools, amsthm, amssymb}
\usepackage{bbm}
\usepackage{stmaryrd}
\allowdisplaybreaks

% Fancy script capitals
\usepackage{mathrsfs}
\usepackage{cancel}
% Bold math
\usepackage{bm}
% Some shortcuts
\newcommand{\rr}{\ensuremath{\mathbb{R}}}
\newcommand{\zz}{\ensuremath{\mathbb{Z}}}
\newcommand{\qq}{\ensuremath{\mathbb{Q}}}
\newcommand{\nn}{\ensuremath{\mathbb{N}}}
\newcommand{\ff}{\ensuremath{\mathbb{F}}}
\newcommand{\cc}{\ensuremath{\mathbb{C}}}
\newcommand{\ee}{\ensuremath{\mathbb{E}}}
\newcommand{\hh}{\ensuremath{\mathbb{H}}}
\renewcommand\O{\ensuremath{\emptyset}}
\newcommand{\norm}[1]{{\left\lVert{#1}\right\rVert}}
\newcommand{\dbracket}[1]{{\left\llbracket{#1}\right\rrbracket}}
\newcommand{\ve}[1]{{\bm{#1}}}
\newcommand\allbold[1]{{\boldmath\textbf{#1}}}
\DeclareMathOperator{\lcm}{lcm}
\DeclareMathOperator{\im}{im}
\DeclareMathOperator{\coim}{coim}
\DeclareMathOperator{\dom}{dom}
\DeclareMathOperator{\tr}{tr}
\DeclareMathOperator{\rank}{rank}
\DeclareMathOperator*{\var}{Var}
\DeclareMathOperator*{\ev}{E}
\DeclareMathOperator{\dg}{deg}
\DeclareMathOperator{\aff}{aff}
\DeclareMathOperator{\conv}{conv}
\DeclareMathOperator{\inte}{int}
\DeclareMathOperator*{\argmin}{argmin}
\DeclareMathOperator*{\argmax}{argmax}
\DeclareMathOperator{\graph}{graph}
\DeclareMathOperator{\sgn}{sgn}
\DeclareMathOperator*{\Rep}{Rep}
\DeclareMathOperator{\Proj}{Proj}
\DeclareMathOperator{\mat}{mat}
\DeclareMathOperator{\diag}{diag}
\DeclareMathOperator{\aut}{Aut}
\DeclareMathOperator{\gal}{Gal}
\DeclareMathOperator{\inn}{Inn}
\DeclareMathOperator{\edm}{End}
\DeclareMathOperator{\Hom}{Hom}
\DeclareMathOperator{\ext}{Ext}
\DeclareMathOperator{\tor}{Tor}
\DeclareMathOperator{\Span}{Span}
\DeclareMathOperator{\Stab}{Stab}
\DeclareMathOperator{\cont}{cont}
\DeclareMathOperator{\Ann}{Ann}
\DeclareMathOperator{\Div}{div}
\DeclareMathOperator{\curl}{curl}
\DeclareMathOperator{\nat}{Nat}
\DeclareMathOperator{\gr}{Gr}
\DeclareMathOperator{\vect}{Vect}
\DeclareMathOperator{\id}{id}
\DeclareMathOperator{\Mod}{Mod}
\DeclareMathOperator{\sign}{sign}
\DeclareMathOperator{\Surf}{Surf}
\DeclareMathOperator{\fcone}{fcone}
\DeclareMathOperator{\Rot}{Rot}
\DeclareMathOperator{\grad}{grad}
\DeclareMathOperator{\atan2}{atan2}
\DeclareMathOperator{\Ric}{Ric}
\let\vec\relax
\DeclareMathOperator{\vec}{vec}
\let\Re\relax
\DeclareMathOperator{\Re}{Re}
\let\Im\relax
\DeclareMathOperator{\Im}{Im}
% Put x \to \infty below \lim
\let\svlim\lim\def\lim{\svlim\limits}

%wide hat
\usepackage{scalerel,stackengine}
\stackMath
\newcommand*\wh[1]{%
\savestack{\tmpbox}{\stretchto{%
  \scaleto{%
    \scalerel*[\widthof{\ensuremath{#1}}]{\kern-.6pt\bigwedge\kern-.6pt}%
    {\rule[-\textheight/2]{1ex}{\textheight}}%WIDTH-LIMITED BIG WEDGE
  }{\textheight}% 
}{0.5ex}}%
\stackon[1pt]{#1}{\tmpbox}%
}
\parskip 1ex

%Make implies and impliedby shorter
\let\implies\Rightarrow
\let\impliedby\Leftarrow
\let\iff\Leftrightarrow
\let\epsilon\varepsilon

% Add \contra symbol to denote contradiction
\usepackage{stmaryrd} % for \lightning
\newcommand\contra{\scalebox{1.5}{$\lightning$}}

% \let\phi\varphi

% Command for short corrections
% Usage: 1+1=\correct{3}{2}

\definecolor{correct}{HTML}{009900}
\newcommand\correct[2]{\ensuremath{\:}{\color{red}{#1}}\ensuremath{\to }{\color{correct}{#2}}\ensuremath{\:}}
\newcommand\green[1]{{\color{correct}{#1}}}

% horizontal rule
\newcommand\hr{
    \noindent\rule[0.5ex]{\linewidth}{0.5pt}
}

% hide parts
\newcommand\hide[1]{}

% si unitx
\usepackage{siunitx}
\sisetup{locale = FR}

%allows pmatrix to stretch
\makeatletter
\renewcommand*\env@matrix[1][\arraystretch]{%
  \edef\arraystretch{#1}%
  \hskip -\arraycolsep
  \let\@ifnextchar\new@ifnextchar
  \array{*\c@MaxMatrixCols c}}
\makeatother

\renewcommand{\arraystretch}{0.8}

\renewcommand{\baselinestretch}{1.5}

\usepackage{graphics}
\usepackage{epstopdf}

\RequirePackage{hyperref}
%%
%% Add support for color in order to color the hyperlinks.
%% 
\hypersetup{
  colorlinks = true,
  urlcolor = blue,
  citecolor = blue
}
%%fakesection Links
\hypersetup{
    colorlinks,
    linkcolor={red!50!black},
    citecolor={green!50!black},
    urlcolor={blue!80!black}
}
%customization of cleveref
\RequirePackage[capitalize,nameinlink]{cleveref}[0.19]

% Per SIAM Style Manual, "section" should be lowercase
\crefname{section}{section}{sections}
\crefname{subsection}{subsection}{subsections}
\Crefname{section}{Section}{Sections}
\Crefname{subsection}{Subsection}{Subsections}

% Per SIAM Style Manual, "Figure" should be spelled out in references
\Crefname{figure}{Figure}{Figures}

% Per SIAM Style Manual, don't say equation in front on an equation.
\crefformat{equation}{\textup{#2(#1)#3}}
\crefrangeformat{equation}{\textup{#3(#1)#4--#5(#2)#6}}
\crefmultiformat{equation}{\textup{#2(#1)#3}}{ and \textup{#2(#1)#3}}
{, \textup{#2(#1)#3}}{, and \textup{#2(#1)#3}}
\crefrangemultiformat{equation}{\textup{#3(#1)#4--#5(#2)#6}}%
{ and \textup{#3(#1)#4--#5(#2)#6}}{, \textup{#3(#1)#4--#5(#2)#6}}{, and \textup{#3(#1)#4--#5(#2)#6}}

% But spell it out at the beginning of a sentence.
\Crefformat{equation}{#2Equation~\textup{(#1)}#3}
\Crefrangeformat{equation}{Equations~\textup{#3(#1)#4--#5(#2)#6}}
\Crefmultiformat{equation}{Equations~\textup{#2(#1)#3}}{ and \textup{#2(#1)#3}}
{, \textup{#2(#1)#3}}{, and \textup{#2(#1)#3}}
\Crefrangemultiformat{equation}{Equations~\textup{#3(#1)#4--#5(#2)#6}}%
{ and \textup{#3(#1)#4--#5(#2)#6}}{, \textup{#3(#1)#4--#5(#2)#6}}{, and \textup{#3(#1)#4--#5(#2)#6}}

% Make number non-italic in any environment.
\crefdefaultlabelformat{#2\textup{#1}#3}

% Environments
\makeatother
% For box around Definition, Theorem, \ldots
%%fakesection Theorems
\usepackage{thmtools}
\usepackage[framemethod=TikZ]{mdframed}

\theoremstyle{definition}
\mdfdefinestyle{mdbluebox}{%
	roundcorner = 10pt,
	linewidth=1pt,
	skipabove=12pt,
	innerbottommargin=9pt,
	skipbelow=2pt,
	nobreak=true,
	linecolor=blue,
	backgroundcolor=TealBlue!5,
}
\declaretheoremstyle[
	headfont=\sffamily\bfseries\color{MidnightBlue},
	mdframed={style=mdbluebox},
	headpunct={\\[3pt]},
	postheadspace={0pt}
]{thmbluebox}

\mdfdefinestyle{mdredbox}{%
	linewidth=0.5pt,
	skipabove=12pt,
	frametitleaboveskip=5pt,
	frametitlebelowskip=0pt,
	skipbelow=2pt,
	frametitlefont=\bfseries,
	innertopmargin=4pt,
	innerbottommargin=8pt,
	nobreak=false,
	linecolor=RawSienna,
	backgroundcolor=Salmon!5,
}
\declaretheoremstyle[
	headfont=\bfseries\color{RawSienna},
	mdframed={style=mdredbox},
	headpunct={\\[3pt]},
	postheadspace={0pt},
]{thmredbox}

\declaretheorem[%
style=thmbluebox,name=Theorem,numberwithin=section]{thm}
\declaretheorem[style=thmbluebox,name=Lemma,sibling=thm]{lem}
\declaretheorem[style=thmbluebox,name=Proposition,sibling=thm]{prop}
\declaretheorem[style=thmbluebox,name=Corollary,sibling=thm]{coro}
\declaretheorem[style=thmredbox,name=Example,sibling=thm]{eg}

\mdfdefinestyle{mdgreenbox}{%
	roundcorner = 10pt,
	linewidth=1pt,
	skipabove=12pt,
	innerbottommargin=9pt,
	skipbelow=2pt,
	nobreak=true,
	linecolor=ForestGreen,
	backgroundcolor=ForestGreen!5,
}

\declaretheoremstyle[
	headfont=\bfseries\sffamily\color{ForestGreen!70!black},
	bodyfont=\normalfont,
	spaceabove=2pt,
	spacebelow=1pt,
	mdframed={style=mdgreenbox},
	headpunct={ --- },
]{thmgreenbox}

\declaretheorem[style=thmgreenbox,name=Definition,sibling=thm]{defn}

\mdfdefinestyle{mdgreenboxsq}{%
	linewidth=1pt,
	skipabove=12pt,
	innerbottommargin=9pt,
	skipbelow=2pt,
	nobreak=true,
	linecolor=ForestGreen,
	backgroundcolor=ForestGreen!5,
}
\declaretheoremstyle[
	headfont=\bfseries\sffamily\color{ForestGreen!70!black},
	bodyfont=\normalfont,
	spaceabove=2pt,
	spacebelow=1pt,
	mdframed={style=mdgreenboxsq},
	headpunct={},
]{thmgreenboxsq}
\declaretheoremstyle[
	headfont=\bfseries\sffamily\color{ForestGreen!70!black},
	bodyfont=\normalfont,
	spaceabove=2pt,
	spacebelow=1pt,
	mdframed={style=mdgreenboxsq},
	headpunct={},
]{thmgreenboxsq*}

\mdfdefinestyle{mdblackbox}{%
	skipabove=8pt,
	linewidth=3pt,
	rightline=false,
	leftline=true,
	topline=false,
	bottomline=false,
	linecolor=black,
	backgroundcolor=RedViolet!5!gray!5,
}
\declaretheoremstyle[
	headfont=\bfseries,
	bodyfont=\normalfont\small,
	spaceabove=0pt,
	spacebelow=0pt,
	mdframed={style=mdblackbox}
]{thmblackbox}

\theoremstyle{plain}
\declaretheorem[name=Question,sibling=thm,style=thmblackbox]{ques}
\declaretheorem[name=Remark,sibling=thm,style=thmgreenboxsq]{remark}
\declaretheorem[name=Remark,sibling=thm,style=thmgreenboxsq*]{remark*}
\newtheorem{ass}[thm]{Assumptions}

\theoremstyle{definition}
\newtheorem*{problem}{Problem}
\newtheorem{claim}[thm]{Claim}
\theoremstyle{remark}
\newtheorem*{case}{Case}
\newtheorem*{notation}{Notation}
\newtheorem*{note}{Note}
\newtheorem*{motivation}{Motivation}
\newtheorem*{intuition}{Intuition}
\newtheorem*{conjecture}{Conjecture}

% Make section starts with 1 for report type
%\renewcommand\thesection{\arabic{section}}

% End example and intermezzo environments with a small diamond (just like proof
% environments end with a small square)
\usepackage{etoolbox}
\AtEndEnvironment{vb}{\null\hfill$\diamond$}%
\AtEndEnvironment{intermezzo}{\null\hfill$\diamond$}%
% \AtEndEnvironment{opmerking}{\null\hfill$\diamond$}%

% Fix some spacing
% http://tex.stackexchange.com/questions/22119/how-can-i-change-the-spacing-before-theorems-with-amsthm
\makeatletter
\def\thm@space@setup{%
  \thm@preskip=\parskip \thm@postskip=0pt
}

% Fix some stuff
% %http://tex.stackexchange.com/questions/76273/multiple-pdfs-with-page-group-included-in-a-single-page-warning
\pdfsuppresswarningpagegroup=1


% My name
\author{Jaden Wang}



\begin{document}
\section{Lie Theory}

For any point $ x$ in a $ n$-dimensional smooth manifold  $ M^{n}$, the tangent space  $ T_xM$ is the set of velocity vectors at $ x$ of all curves on $ M$ passing through $ x$. Intuitively it is a hyperplane $ \rr^{n}$ tangent to $ M$ at  $ x$. The tangent bundle $ TM$ of a smooth manifold $ M$ is the union of all its tangent spaces: $ TM := \bigcup_{ x \in M} T_x M = \{(x,v): x \in M, v \in T_xM\}  $. It is also a smooth manifold.

Every smooth map $ f:M \to N$ between smooth manifolds $ M,N$ induces a differential  $ Tf: TM \to TN$ between the tangle bundles. At each base point $ x \in M$, $ T_x f : T_xM \to T_{f(x)} N$ is a linear map from the tangent space $ T_xM$ based at $ x$: it is the best linear approximation (or derivative) of $ f$ (or in matrix form, the Jacobian of $ f$) at  $ x$.
 

A \allbold{Lie group} is a smooth manifold that is also a group. Examples include $ \text{SO}( n), \text{SE}( n)$. Using $ e$ to denote the identity element of the group, the \allbold{Lie algebra} $ \mathfrak{ g}$ of a Lie group $ G$ is the vector space $ T_e G$ endowed with a Lie bracket $ [ \cdot , \cdot ] : \mathfrak{ g} \times \mathfrak{ g} \to \mathfrak{ g}$ that makes $ T_eG$ an algebra.  For example, cross product in $ \rr^3$ is a Lie bracket of the Lie algebra of $ \rr^3$. This Lie algebra is isomorphic to the Lie algebra $ \mathfrak{ so}(3)$ of $ \text{SO}(3) $.

Left and right multiplication in a Lie group is a smooth map between itself, so they each has a differential. Fix $ g \in G$, then $ L_g : G \to G, h \mapsto gh$ and  $ R_g : G \to G, h \mapsto hg$ denote left and right multiplication by $ g$. Then $ TL_g: TG \to TG, (h, \xi) \mapsto (gh, T_hL_g (\xi))$ and $ TR_g: TG \to TG, (h, \xi) \mapsto (hg, T_h R_g(\xi))$ are the induced differentials.

The tangent bundle $ TG$ of any Lie group $ G$ is also a Lie group. The group operation comes from the differential of multiplication $ \mu$ in  $ G$. If $ \mu: G \times G \to G, (g,h) \mapsto gh$, then product rule gives
\begin{align*}
	d\mu : TG \times TG \to TG,\ ((g, \xi) , (h, \eta)) \mapsto (gh, T_hL_g \eta + T_g R_h \xi).
\end{align*}


\begin{defn}
The \allbold{adjoint action} of a Lie group $ G$ on its Lie algebra  $ \mathfrak{ g}$ is a Lie group homomorphism $ \Ad: G \to \aut( \mathfrak{ g})$, where $ g.( \xi) = g \xi g^{-1} : = T_{g^{-1}} L_g \circ T_e R_{g^{-1}} (\xi)$.


\end{defn}

\begin{prop}
Let $ G$ be a Lie group and  $ \mathfrak{ g }$ be its Lie algebra. Then $ TG \cong \mathfrak{ g} \rtimes_{ \Ad} G$ as a Lie group.
\end{prop}

\begin{proof}
Recall that group operation on $ TG$ is the differential of the group operation on  $ G$. If $ \mu: G \times G \to G, (g,h) \mapsto gh$, then product rule gives
\begin{align*}
	d\mu : TG \times TG \to TG,\ ((g, \xi) , (h, \eta)) \mapsto (gh, T_hL_g \eta + T_g R_h \xi).
\end{align*}

The map $ \phi: \mathfrak{ g} \rtimes_{ \Ad} G \to TG,\ (\xi,g) \mapsto (g,T_eR_g \xi)$ is a bijection since $ T_e R_g$ is invertible. We can show that it is also a diffeomorphism and a Lie group homomorphism, yielding the Lie group isomorphism.
\end{proof}

\begin{coro}
The special Euclidean group $ \text{SE}( n) := \mathfrak{ so }(n) \rtimes_{\Ad} \text{SO}( n) \cong T \text{SO}( n) $. 
\end{coro}

Since $ S^{n}$ is a double-cover of $ \text{SO}( n) $, we have that $ TS^{n} \cong \rr^{n} \times S^{n}$ is a double-cover of $ T \text{SO}( n) \cong \text{SE}( n) $.
\section{The Quaternions and its dual}
The quaternions $ \mathbb{H}$ as a vector space is isomorphic to $ \rr^{4}$. It is additionally a division algebra. We can view it as either isomorphic to $ \Cl_{0,2} ( \rr) $ or $ \Cl_{ 0 ,3}^{+} ( \rr)$. TODO the scalar part of multiplication follows quadratic form of signature 1,3.

Conjugation in $ \mathbb{H}$ is $ q^* = q_s - q_v$. 

The unit quaternions $ \mathbb{H}_u$ is diffeomorphic to $ S^3$ and isomorphic to $ \text{SU}(2) $ as a Lie group. Conjugation $ q^* $ in $ \mathbb{H}_u$ is the same as inverse $ q^{-1}$. Geometrically, it keeps the real axis fixed but finds the antipodal point with respect to the remaining dimensions. If we regard $ S^3 \cong S^{1} \vee S^{2}$, the join TODO then 
We denote the Lie algebra $ T_e \mathbb{H}_u$ of $ \mathbb{H}_u$ as $ \mathfrak{ h}$.

\begin{defn}
The \allbold{dual quaternions} $ \widetilde{\mathbb{H}} = \{q_1 + \epsilon q_2: q_1, q_2 \in \mathbb{H}, \epsilon^2 = 0\}$.  
\end{defn}
This is simply dual numbers for quaternions. Note that $ \epsilon$ can be understood as an infinitesimal, and the part with $ \epsilon$ can be understood as derivatives, which obey product rule during multiplication. This is no different than writing
\begin{align*}
	\widetilde{\mathbb{H}} = \{(q, v): q \in \mathbb{H}, v \in T_q \mathbb{H}\} = T \mathbb{H}, 
\end{align*}
since tangent vectors in $ T \mathbb{H}$ and derivatives are the same thing and obey the same multiplication. The algebraic definition is more suitable for computation, but the tangent bundle definition invites additional geometric insights.

Sometimes we restrict to unit quaternions $ \mathbb{H}_u$ and its dual.
\begin{defn}
	The \allbold{dual quaternions} $ \widetilde{\mathbb{H}_u}$ is defined to be the tangent bundle $ T \mathbb{H}_u$ of the unit quaternions. Thus $ \widetilde{\mathbb{H}_u} \cong \mathfrak{ h} \times \mathbb{H}_u \cong TS^{3}$ and is a double-cover of $ \text{SE}(3) $.
\end{defn}
\section{Dynamics on Dual Quaternions}
It is important to emphasize the difference between a vector and its coordinates. A vector $ v$ is an element in the vector space ($ \rr^3$ ) and does not depend on the coordinate system (basis) we choose for the vector space. On the other hand, once we choose a basis $ B$, then the coordinates $ v_B$ tells us the coefficients of the basis elements which linearly combines into $ v$. Whenever we use coordinates, we 

We shall use the following identifications throughout the paper: the Euclidean space $ \rr^{3}$ in the inertial frame is identified by the Lie algebra $ \mathfrak{ h} := T_e \mathbb{H}_u$. If the body frame is obtained from the inertial frame by multiplying $ q \in \mathbb{H}_u$ (representing a rotation), then the body frame is identified by the tangent space $ T_q \mathbb{H}_u$. Since rotation is a linear operation, we know that the body coordinates is related to the inertial coordinates via a change of basis. That is, given a vector $ r$ with body coordinates $ r_B$, its inertial coordinates $ r_I = \Ad_q(r_B)= q r_B q^{-1} = q r_B q^*:= T_e L_q \circ T_q R_{q^{-1}} (r_B) $. Thus we obtain the same vector $T_e R_q(r_I) =: r_I q = q r_B:= T_e L_q(r_B) \in T_q \mathbb{H}_u$, which matches the physical intuition that rotation and translation commute.

We fix an (inertial) basis in the Lie algebra $ \mathfrak{ h}$. 

Since any configuration of a rigid-body can be represented by a translation $ r$ of CM from a fixed point (\emph{e.g.} initial position) and a rotation $ q$ from a fixed inertial frame (\emph{e.g.} initial frame), we can represent this motion using the dual quaternion $ (q, r_I q) = (q,q r_B) \in T_q \mathbb{H}_u$.

Since $ q q^{-1} = 1$, differentiating both sides with respect to time yields
\begin{align}
	\dot{q}q^{-1} + q \dot{q}^{-1} &= 0 \\
	p:= \dot{q}q^{-1} &=- q \dot{q}^{-1}\label{eq:qq}
\end{align}
where $ \dot{q}^{-1} := \frac{d q^{-1}}{d t} $. Taking the time derivative of \cref{eq:}, since $ r_B$ is fixed in time, we have
\begin{align*}
	r_I &= q r_B q^{-1} \\
	\dot{r}_I &= \dot{q} r_B q^{-1} + q r_B \dot{q}^{-1} \\
	&= \dot{q} q^{-1} r_I + r_I q \dot{q}^{-1} \\
	&= \dot{q}q^{-1} r_I - r_I \dot{q} q^{-1} && \cref{eq:qq}\\
	&= p \times r_I - r_I \times p \\
	&= 2 p \times r_I
\end{align*}
Moreover, by the transport theorem, $ \dot{r}_I = \frac{d}{d t}\left( r_B \right)_B   + \omega \times r_I = \omega \times r_I$. This yields that
\begin{align} \label{eq:omega}
	\omega = 2p = 2 \dot{q} q^{-1} .
\end{align}
From this, we obtain the quaternion dynamics:
\begin{align}
	\dot{q} &=  \dot{q} q^{-1} q \\
	\dot{q}&= \frac{1}{2} \omega q
\end{align}

We treat $ r_I$ and $ \omega$ as pure quaternions. Then $\frac{1}{2}\dot{r}_I = \frac{1}{2} r_I q $ and $\dot{q} = \frac{1}{2}wq$ are both tangent vectors. Then
\begin{align*}
	\begin{pmatrix} q\\ \dot{q} = \frac{1}{2} \omega q \end{pmatrix}  \to \begin{pmatrix} q\\ \frac{1}{2} r_I q \end{pmatrix} =: \widetilde{ q} \in T_q \mathbb{H}\\
	\begin{pmatrix} \omega \\ \dot{\omega} \end{pmatrix} \to \begin{pmatrix} \omega \\ \dot{r}_I \end{pmatrix} =: \widetilde{ \omega} \in T_{ \omega} \mathbb{H} 
\end{align*}
Notice that we swap out the tangent vectors in order to incorporate translation into the dynamics. Since
\begin{align*}
	\begin{pmatrix} \dot{q}\\ \ddot{q} \end{pmatrix} = \begin{pmatrix} \frac{1}{2} wq \\ \frac{1}{2} \left( \dot{w}q + w\dot{q} \right)  \end{pmatrix}  = \frac{1}{2} \begin{pmatrix} w\\ \dot{w} \end{pmatrix} \begin{pmatrix} q\\ \dot{q} \end{pmatrix} \in T_{\frac{1}{2}wq} \mathbb{H}
\end{align*}
the substitution yields the dual quaternion dynamics
\begin{align*}
	\dot{\widetilde{ q}} = \frac{1}{2} \begin{pmatrix} \omega\\ \dot{r}_I \end{pmatrix} \begin{pmatrix} q\\ \frac{1}{2} r_I q \end{pmatrix} = \frac{1}{2} \widetilde{ w} \widetilde{ q} \in T_{\frac{1}{2}\omega q} \mathbb{H}
\end{align*}
mirroring the quaternion dynamics.
\end{document}
