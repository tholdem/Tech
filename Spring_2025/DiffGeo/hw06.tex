\documentclass[12pt]{article}
%Fall 2022
% Some basic packages
\usepackage{standalone}[subpreambles=true]
\usepackage[utf8]{inputenc}
\usepackage[T1]{fontenc}
\usepackage{textcomp}
\usepackage[english]{babel}
\usepackage{url}
\usepackage{graphicx}
%\usepackage{quiver}
\usepackage{float}
\usepackage{enumitem}
\usepackage{lmodern}
\usepackage{comment}
\usepackage{hyperref}
\usepackage[usenames,svgnames,dvipsnames]{xcolor}
\usepackage[margin=1in]{geometry}
\usepackage{pdfpages}

\pdfminorversion=7

% Don't indent paragraphs, leave some space between them
\usepackage{parskip}

% Hide page number when page is empty
\usepackage{emptypage}
\usepackage{subcaption}
\usepackage{multicol}
\usepackage[b]{esvect}

% Math stuff
\usepackage{amsmath, amsfonts, mathtools, amsthm, amssymb}
\usepackage{bbm}
\usepackage{stmaryrd}
\allowdisplaybreaks

% Fancy script capitals
\usepackage{mathrsfs}
\usepackage{cancel}
% Bold math
\usepackage{bm}
% Some shortcuts
\newcommand{\rr}{\ensuremath{\mathbb{R}}}
\newcommand{\zz}{\ensuremath{\mathbb{Z}}}
\newcommand{\qq}{\ensuremath{\mathbb{Q}}}
\newcommand{\nn}{\ensuremath{\mathbb{N}}}
\newcommand{\ff}{\ensuremath{\mathbb{F}}}
\newcommand{\cc}{\ensuremath{\mathbb{C}}}
\newcommand{\ee}{\ensuremath{\mathbb{E}}}
\newcommand{\hh}{\ensuremath{\mathbb{H}}}
\renewcommand\O{\ensuremath{\emptyset}}
\newcommand{\norm}[1]{{\left\lVert{#1}\right\rVert}}
\newcommand{\dbracket}[1]{{\left\llbracket{#1}\right\rrbracket}}
\newcommand{\ve}[1]{{\bm{#1}}}
\newcommand\allbold[1]{{\boldmath\textbf{#1}}}
\DeclareMathOperator{\lcm}{lcm}
\DeclareMathOperator{\im}{im}
\DeclareMathOperator{\coim}{coim}
\DeclareMathOperator{\dom}{dom}
\DeclareMathOperator{\tr}{tr}
\DeclareMathOperator{\rank}{rank}
\DeclareMathOperator*{\var}{Var}
\DeclareMathOperator*{\ev}{E}
\DeclareMathOperator{\dg}{deg}
\DeclareMathOperator{\aff}{aff}
\DeclareMathOperator{\conv}{conv}
\DeclareMathOperator{\inte}{int}
\DeclareMathOperator*{\argmin}{argmin}
\DeclareMathOperator*{\argmax}{argmax}
\DeclareMathOperator{\graph}{graph}
\DeclareMathOperator{\sgn}{sgn}
\DeclareMathOperator*{\Rep}{Rep}
\DeclareMathOperator{\Proj}{Proj}
\DeclareMathOperator{\mat}{mat}
\DeclareMathOperator{\diag}{diag}
\DeclareMathOperator{\aut}{Aut}
\DeclareMathOperator{\gal}{Gal}
\DeclareMathOperator{\inn}{Inn}
\DeclareMathOperator{\edm}{End}
\DeclareMathOperator{\Hom}{Hom}
\DeclareMathOperator{\ext}{Ext}
\DeclareMathOperator{\tor}{Tor}
\DeclareMathOperator{\Span}{Span}
\DeclareMathOperator{\Stab}{Stab}
\DeclareMathOperator{\cont}{cont}
\DeclareMathOperator{\Ann}{Ann}
\DeclareMathOperator{\Div}{div}
\DeclareMathOperator{\curl}{curl}
\DeclareMathOperator{\nat}{Nat}
\DeclareMathOperator{\gr}{Gr}
\DeclareMathOperator{\vect}{Vect}
\DeclareMathOperator{\id}{id}
\DeclareMathOperator{\Mod}{Mod}
\DeclareMathOperator{\sign}{sign}
\DeclareMathOperator{\Surf}{Surf}
\DeclareMathOperator{\fcone}{fcone}
\DeclareMathOperator{\Rot}{Rot}
\DeclareMathOperator{\grad}{grad}
\DeclareMathOperator{\atan2}{atan2}
\DeclareMathOperator{\Ric}{Ric}
\let\vec\relax
\DeclareMathOperator{\vec}{vec}
\let\Re\relax
\DeclareMathOperator{\Re}{Re}
\let\Im\relax
\DeclareMathOperator{\Im}{Im}
% Put x \to \infty below \lim
\let\svlim\lim\def\lim{\svlim\limits}

%wide hat
\usepackage{scalerel,stackengine}
\stackMath
\newcommand*\wh[1]{%
\savestack{\tmpbox}{\stretchto{%
  \scaleto{%
    \scalerel*[\widthof{\ensuremath{#1}}]{\kern-.6pt\bigwedge\kern-.6pt}%
    {\rule[-\textheight/2]{1ex}{\textheight}}%WIDTH-LIMITED BIG WEDGE
  }{\textheight}% 
}{0.5ex}}%
\stackon[1pt]{#1}{\tmpbox}%
}
\parskip 1ex

%Make implies and impliedby shorter
\let\implies\Rightarrow
\let\impliedby\Leftarrow
\let\iff\Leftrightarrow
\let\epsilon\varepsilon

% Add \contra symbol to denote contradiction
\usepackage{stmaryrd} % for \lightning
\newcommand\contra{\scalebox{1.5}{$\lightning$}}

% \let\phi\varphi

% Command for short corrections
% Usage: 1+1=\correct{3}{2}

\definecolor{correct}{HTML}{009900}
\newcommand\correct[2]{\ensuremath{\:}{\color{red}{#1}}\ensuremath{\to }{\color{correct}{#2}}\ensuremath{\:}}
\newcommand\green[1]{{\color{correct}{#1}}}

% horizontal rule
\newcommand\hr{
    \noindent\rule[0.5ex]{\linewidth}{0.5pt}
}

% hide parts
\newcommand\hide[1]{}

% si unitx
\usepackage{siunitx}
\sisetup{locale = FR}

%allows pmatrix to stretch
\makeatletter
\renewcommand*\env@matrix[1][\arraystretch]{%
  \edef\arraystretch{#1}%
  \hskip -\arraycolsep
  \let\@ifnextchar\new@ifnextchar
  \array{*\c@MaxMatrixCols c}}
\makeatother

\renewcommand{\arraystretch}{0.8}

\renewcommand{\baselinestretch}{1.5}

\usepackage{graphics}
\usepackage{epstopdf}

\RequirePackage{hyperref}
%%
%% Add support for color in order to color the hyperlinks.
%% 
\hypersetup{
  colorlinks = true,
  urlcolor = blue,
  citecolor = blue
}
%%fakesection Links
\hypersetup{
    colorlinks,
    linkcolor={red!50!black},
    citecolor={green!50!black},
    urlcolor={blue!80!black}
}
%customization of cleveref
\RequirePackage[capitalize,nameinlink]{cleveref}[0.19]

% Per SIAM Style Manual, "section" should be lowercase
\crefname{section}{section}{sections}
\crefname{subsection}{subsection}{subsections}
\Crefname{section}{Section}{Sections}
\Crefname{subsection}{Subsection}{Subsections}

% Per SIAM Style Manual, "Figure" should be spelled out in references
\Crefname{figure}{Figure}{Figures}

% Per SIAM Style Manual, don't say equation in front on an equation.
\crefformat{equation}{\textup{#2(#1)#3}}
\crefrangeformat{equation}{\textup{#3(#1)#4--#5(#2)#6}}
\crefmultiformat{equation}{\textup{#2(#1)#3}}{ and \textup{#2(#1)#3}}
{, \textup{#2(#1)#3}}{, and \textup{#2(#1)#3}}
\crefrangemultiformat{equation}{\textup{#3(#1)#4--#5(#2)#6}}%
{ and \textup{#3(#1)#4--#5(#2)#6}}{, \textup{#3(#1)#4--#5(#2)#6}}{, and \textup{#3(#1)#4--#5(#2)#6}}

% But spell it out at the beginning of a sentence.
\Crefformat{equation}{#2Equation~\textup{(#1)}#3}
\Crefrangeformat{equation}{Equations~\textup{#3(#1)#4--#5(#2)#6}}
\Crefmultiformat{equation}{Equations~\textup{#2(#1)#3}}{ and \textup{#2(#1)#3}}
{, \textup{#2(#1)#3}}{, and \textup{#2(#1)#3}}
\Crefrangemultiformat{equation}{Equations~\textup{#3(#1)#4--#5(#2)#6}}%
{ and \textup{#3(#1)#4--#5(#2)#6}}{, \textup{#3(#1)#4--#5(#2)#6}}{, and \textup{#3(#1)#4--#5(#2)#6}}

% Make number non-italic in any environment.
\crefdefaultlabelformat{#2\textup{#1}#3}

% Environments
\makeatother
% For box around Definition, Theorem, \ldots
%%fakesection Theorems
\usepackage{thmtools}
\usepackage[framemethod=TikZ]{mdframed}

\theoremstyle{definition}
\mdfdefinestyle{mdbluebox}{%
	roundcorner = 10pt,
	linewidth=1pt,
	skipabove=12pt,
	innerbottommargin=9pt,
	skipbelow=2pt,
	nobreak=true,
	linecolor=blue,
	backgroundcolor=TealBlue!5,
}
\declaretheoremstyle[
	headfont=\sffamily\bfseries\color{MidnightBlue},
	mdframed={style=mdbluebox},
	headpunct={\\[3pt]},
	postheadspace={0pt}
]{thmbluebox}

\mdfdefinestyle{mdredbox}{%
	linewidth=0.5pt,
	skipabove=12pt,
	frametitleaboveskip=5pt,
	frametitlebelowskip=0pt,
	skipbelow=2pt,
	frametitlefont=\bfseries,
	innertopmargin=4pt,
	innerbottommargin=8pt,
	nobreak=false,
	linecolor=RawSienna,
	backgroundcolor=Salmon!5,
}
\declaretheoremstyle[
	headfont=\bfseries\color{RawSienna},
	mdframed={style=mdredbox},
	headpunct={\\[3pt]},
	postheadspace={0pt},
]{thmredbox}

\declaretheorem[%
style=thmbluebox,name=Theorem,numberwithin=section]{thm}
\declaretheorem[style=thmbluebox,name=Lemma,sibling=thm]{lem}
\declaretheorem[style=thmbluebox,name=Proposition,sibling=thm]{prop}
\declaretheorem[style=thmbluebox,name=Corollary,sibling=thm]{coro}
\declaretheorem[style=thmredbox,name=Example,sibling=thm]{eg}

\mdfdefinestyle{mdgreenbox}{%
	roundcorner = 10pt,
	linewidth=1pt,
	skipabove=12pt,
	innerbottommargin=9pt,
	skipbelow=2pt,
	nobreak=true,
	linecolor=ForestGreen,
	backgroundcolor=ForestGreen!5,
}

\declaretheoremstyle[
	headfont=\bfseries\sffamily\color{ForestGreen!70!black},
	bodyfont=\normalfont,
	spaceabove=2pt,
	spacebelow=1pt,
	mdframed={style=mdgreenbox},
	headpunct={ --- },
]{thmgreenbox}

\declaretheorem[style=thmgreenbox,name=Definition,sibling=thm]{defn}

\mdfdefinestyle{mdgreenboxsq}{%
	linewidth=1pt,
	skipabove=12pt,
	innerbottommargin=9pt,
	skipbelow=2pt,
	nobreak=true,
	linecolor=ForestGreen,
	backgroundcolor=ForestGreen!5,
}
\declaretheoremstyle[
	headfont=\bfseries\sffamily\color{ForestGreen!70!black},
	bodyfont=\normalfont,
	spaceabove=2pt,
	spacebelow=1pt,
	mdframed={style=mdgreenboxsq},
	headpunct={},
]{thmgreenboxsq}
\declaretheoremstyle[
	headfont=\bfseries\sffamily\color{ForestGreen!70!black},
	bodyfont=\normalfont,
	spaceabove=2pt,
	spacebelow=1pt,
	mdframed={style=mdgreenboxsq},
	headpunct={},
]{thmgreenboxsq*}

\mdfdefinestyle{mdblackbox}{%
	skipabove=8pt,
	linewidth=3pt,
	rightline=false,
	leftline=true,
	topline=false,
	bottomline=false,
	linecolor=black,
	backgroundcolor=RedViolet!5!gray!5,
}
\declaretheoremstyle[
	headfont=\bfseries,
	bodyfont=\normalfont\small,
	spaceabove=0pt,
	spacebelow=0pt,
	mdframed={style=mdblackbox}
]{thmblackbox}

\theoremstyle{plain}
\declaretheorem[name=Question,sibling=thm,style=thmblackbox]{ques}
\declaretheorem[name=Remark,sibling=thm,style=thmgreenboxsq]{remark}
\declaretheorem[name=Remark,sibling=thm,style=thmgreenboxsq*]{remark*}
\newtheorem{ass}[thm]{Assumptions}

\theoremstyle{definition}
\newtheorem*{problem}{Problem}
\newtheorem{claim}[thm]{Claim}
\theoremstyle{remark}
\newtheorem*{case}{Case}
\newtheorem*{notation}{Notation}
\newtheorem*{note}{Note}
\newtheorem*{motivation}{Motivation}
\newtheorem*{intuition}{Intuition}
\newtheorem*{conjecture}{Conjecture}

% Make section starts with 1 for report type
%\renewcommand\thesection{\arabic{section}}

% End example and intermezzo environments with a small diamond (just like proof
% environments end with a small square)
\usepackage{etoolbox}
\AtEndEnvironment{vb}{\null\hfill$\diamond$}%
\AtEndEnvironment{intermezzo}{\null\hfill$\diamond$}%
% \AtEndEnvironment{opmerking}{\null\hfill$\diamond$}%

% Fix some spacing
% http://tex.stackexchange.com/questions/22119/how-can-i-change-the-spacing-before-theorems-with-amsthm
\makeatletter
\def\thm@space@setup{%
  \thm@preskip=\parskip \thm@postskip=0pt
}

% Fix some stuff
% %http://tex.stackexchange.com/questions/76273/multiple-pdfs-with-page-group-included-in-a-single-page-warning
\pdfsuppresswarningpagegroup=1


% My name
\author{Jaden Wang}



\begin{document}
\centerline {\textsf{\textbf{\LARGE{Homework 6}}}}
\centerline {Jaden Wang}
\vspace{.15in}
\begin{problem}[LN15 0.4]
	Show that the bracket satisfies the following properties: $ [X,Y] = - [Y,X]$ and  $ [X,[Y,Z]] + [Y,[Z,X]] + [Z,[X,Y]] =0$.
\end{problem}
\begin{proof}
	The pointwise definition of Lie bracket can be expressed succinctly as $ [X,Y] = XY-YX$. Then we have
\begin{align*}
	[X,Y] = XY - YX = -(YX-XY) = - [Y,X],
\end{align*}
and
\begin{align*}
	[X,[Y,Z]] + [Y,[Z,X]] + [Z,[X,Y]] =& XYZ-XZY-YZX+ZYX \\
	& YZX- ZYX-ZXY+YXZ\\
	& ZXY - YXZ-XYZ + XZY\\
	=& 0 .
\end{align*}
\end{proof}
\begin{problem}[LN15 0.5]
Show that a connection is symmetric iff the corresponding Christoffel symbol satisfy $ \Gamma_{ii}^{k} = \Gamma_{ji}^{k}$.
\end{problem}
\begin{proof}
Recall that in local coordinates, using Einstein notation we have
\begin{align*}
	\nabla _XY = \left( X(Y^{k}) + X^{j} Y^{i} \Gamma_{ij}^{k} \right) E_k . 
\end{align*}
Then by unifying $ i,j$ indices of $ \nabla _X Y$ and $ \nabla _YX$, we obtain
\begin{align*}
	\nabla _XY - \nabla _YX &= \left( X(Y^{k}) - Y(X^{k}) + X^{i}Y^{j} \left( \Gamma_{ji}^{k} - \Gamma_{ij}^{k} \right)  \right) E_k \\
				&=X(Y^{k})E_k - Y(X^{k})E_k + X^{i}Y^{j} \left( \Gamma_{ji}^{k} - \Gamma_{ij}^{k} \right) E_k \\
				&=XY-YX + X^{i}Y^{j} \left( \Gamma_{ji}^{k} - \Gamma_{ij}^{k} \right) E_k \\
				&=[X,Y] + X^{i}Y^{j} \left( \Gamma_{ji}^{k} - \Gamma_{ij}^{k} \right) E_k .
\end{align*}
Since $ X,Y$ are arbitrary, we see that  $ \nabla _X Y- \nabla _YX = [X,Y]$ iff $ \Gamma_{ji}^{k} - \Gamma_{ij}^{k} = 0$.
\end{proof}
\begin{problem}[do Carmo 3.1]

\end{problem}
\begin{proof}
First, we compute the Jacobian of $ \phi$:
\begin{align*}
	D \phi(u,v) &= \begin{pmatrix} \frac{\partial \phi}{\partial u}(u,v) & \frac{\partial \phi}{\partial v}(u,v)  \end{pmatrix} \\
		    &= \begin{pmatrix} -f(v) \sin u & f'(v) \cos u\\ f(v) \cos u & f'(v) \sin u\\ 0 & g'(v) \end{pmatrix} .
\end{align*}
Since $ f'(v)^2 + g'(v)^2 \neq 0$ and $ f(v) \neq 0$, we can check
\begin{align*}
	\langle \frac{\partial \phi}{\partial u}, \frac{\partial \phi}{\partial u}  \rangle &= f(v)^2 \neq 0 \\
	\langle \frac{\partial \phi}{\partial u}, \frac{\partial \phi}{\partial v}  \rangle &= 0 \\
	\langle \frac{\partial \phi}{\partial v} , \frac{\partial \phi}{\partial v}   \rangle &= f'(v)^2 + g'(v)^2 \neq 0
\end{align*}
That is, they are nonzero and their dot product is 0, so they are orthogonal. Thus $ D \phi$ is injective everywhere, \emph{i.e.} $ \phi$ is an immersion.

\begin{enumerate}[label=(\alph*)]
	\item Recall that the induced metric is just the pairwise dot products of the pushforward basis under $ \phi$. Thus by the above computation, we have $ g_{11} = f^2$, $ g_{12} = 0$, and $ g_{22} = (f')^2+(g')^2$, which are all functions of $ v$.
	\item Then $ g^{11} = \frac{1}{f^2}$, $ g^{12} = 0$, $ g^{22} = \frac{1}{(f')^2+(g')^2}$. Based on the equation $ \Gamma_{ij}^{m} = \frac{1}{2} g^{  km} \left(- \frac{\partial g_{ij}}{\partial x_k} + \frac{\partial g_{jk}}{\partial x_i}  + \frac{\partial g_{ki}}{\partial x_j}  \right) $, we obtain  
\begin{align*}
	\Gamma_{11}^{1} &= \frac{1}{2f^2} \left( -0+0+0 \right) + 0 = 0\\ 
	\Gamma_{11}^{2} &= 0 + \frac{1}{2((f')^2+(g')^2)} \left( -2ff' + 0 +0  \right) =-\frac{ff'}{(f')^2+(g')^2}    \\
	\Gamma_{12}^{1}&= \frac{1}{2f^2} (-0+0+2ff') + 0 =  \frac{ff'}{f^2 } \\
	\Gamma_{12}^{2} &= 0 + \frac{1}{2((f')^2+(g')^2)}(-0+0+0) = 0   \\
	\Gamma_{21}^{1} &= \Gamma_{12}^{1} = \frac{ff'}{f^2 } \\
	\Gamma_{21}^{2} &= \Gamma_{12}^{2}=0 \\
\Gamma_{22}^{1} &= \frac{1}{2f^2} ( -0+ 0+0)+ 0 = 0 \\
\Gamma_{22}^{2} &= 0 + \frac{1}{2((f')^2+(g')^2)}(-1+1+1) (2f'f''+2g'g'')  \\&= \frac{f'f''+g'g''}{(f')^2+(g')^2}   .
\end{align*}
The equation of geodesic $ \gamma(t) = (u(t),v(t)) $ is
\begin{align*}
	\ddot{ \gamma}^{k} + \dot{ \gamma}^{i} \dot{ \gamma}^{j} \Gamma_{ij}^{k}( \gamma) &= 0 \\ 
	\begin{cases}
	\ddot{u} + \dot{u} \dot{v} \frac{f f'}{ f^2} + \dot{v} \dot{u} \frac{ f f'}{f^2 } &= 0 \\
	\ddot{v} - \dot{u}^2 \frac{ff'}{ (f')^2+(g')^2}+ \dot{v}^2 \frac{f'f''+g'g''}{ (f')^2+(g')^2} &= 0 \\
	\end{cases}\\
	\begin{cases}
	\ddot{u} +  \frac{2f f'}{ f^2} \dot{u} \dot{v} &= 0 \\
	\ddot{v} -  \frac{ff'}{ (f')^2+(g')^2} \dot{u}^2+ \frac{f'f''+g'g''}{ (f')^2+(g')^2} \dot{v}^2 &= 0 \\
	\end{cases}
\end{align*}
\item We compute
\begin{align*}
	| \dot{ \gamma}|^2 &= \begin{pmatrix} \dot{u} & \dot{v} \end{pmatrix} \begin{pmatrix} f^2 & 0\\ 0& (f')^2+(g')^2 \end{pmatrix} \begin{pmatrix} \dot{u}\\\dot{v} \end{pmatrix}   \\
	&= f^2 \dot{u}^2 + ((f')^2+(g')^2) \dot{v}^2 .
\end{align*}
Taking the time derivative of this equation, using $ \dot{f} = f'\dot{v}$, $ \dot{f'} = f'' \dot{v}$, and the first equation $ \ddot{u} = - \frac{2ff'}{ f^2} \dot{u}\dot{v}$ we obtain
\begin{align*}
	&2ff' \dot{v} \dot{u}^2 + 2f^2 \dot{u} \ddot{u} + 2(f'f'' + g'g'') \dot{v}^3 + 2 ((f')^2+(g')^2) \dot{v} \ddot{v} \\
	=& 2\dot{v}(-ff'\dot{u}^2 + (f'f''+g'g'')\dot{v}^2 + ((f')^2+(g')^2)\ddot{v}).
\end{align*}
Since we exclude parallels, $ \dot{v} \neq 0$. The energy is constant iff time derivative is 0, where we can simply divide the equation by $2 \dot{v} ( (f')^2+(g')^2) \neq 0$ to obtain the second equation. 

Let $ P(t) = (a(t), b)$ where $ b$ is constant. Then $ \dot{P}(t) = (\dot{a}(t),0)$. Recall that
\begin{align*}
	\cos \beta &= \frac{ \langle \dot{ \gamma}, \dot{P} \rangle_{(u,v)}}{ | \dot{ \gamma}||\dot{P}|} \\
	&= \frac{ f^2 \dot{u} \dot{a}}{ | \dot{ \gamma}| |f\dot{a}| } \\
	&= \frac{ \sgn(f) \sgn(\dot{a}) f \dot{u}}{ | \dot{ \gamma}|} .
\end{align*}
Using the fact that $ \frac{d}{dt}| \dot{ \gamma}| =0$, the time derivative of $ f(v) \cos \beta$ is
\begin{align*}
	f'(v) \dot{v} \cos \beta + f(v) \frac{d}{dt} (\cos \beta) &= \frac{ \sgn(f) \sgn(\dot{a}) ff' \dot{u}\dot{v}}{ | \dot{ \gamma}|} +  \frac{\sgn(f) \sgn(\dot{a})f(f'\dot{v}\dot{u}+f \ddot{u}) | \dot{ \gamma}| - 0}{ |\dot{ \gamma}|^2} \\
								  &= \frac{\sgn(f) \sgn(\dot{a})}{| \dot{ \gamma}|} (f \ddot{u} + 2 ff'\dot{u}\dot{v}) .
\end{align*}
Since $ r \cos \beta = |f(v)| \cos \beta$, it is constant iff this time derivative is 0 iff the first equation holds.
\item Since $ r = |v|$, and  $ |v| \cos \beta $ is constant, the constant $ c$ is either zero or nonzero. If $ c=0$, then since we can vary $ v$ it must be that  $ \cos \beta = 0$. Since $ \beta < \pi$, it must be that $ \beta \equiv \frac{\pi}{2}$, which means the geodesic must intersect parallels at right angle all the time, making it a meridian which we exclude. Thus $ c$ must be nonzero and WLOG let $ c>0$. This forces $ \cos \beta >0$. Since we can decrease the radius at will, $ \cos \beta$ is forced to increase. But $ \cos \beta$ max out at 1, so it must be that $ \beta = 0$ precisely when $ |v| = c$. Since we can no longer decrease $ |v|$ further, and  $ |v|$ is not allowed to be constant,  $ |v|$ must increase instead. Therefore, $ \gamma$ is going up again. If we can argue that $ \gamma$ must always rotate around and is trapped between a minimum $ r$ and a maximum  $ r$, we would complete the proof. But this seems tedious.
\end{enumerate}
\end{proof}

\begin{problem}[do Carmo 3.7]
Let $ M$ be a Riemannian manifold of dimension $ n$ and let  $ p \in M$. Show that there exists a neighborhood $ U \subseteq M$ of $ p$ and  $ n$ vector fields  $ E_1,\ldots,E_n \in \mathfrak{ X} (U) $, orthonormal at each point of $ U$,  s.t.\ at $ p$,  $ \nabla _{E_i} E_j(p) =0$. This is called a geodesic frame at $ p$.
\end{problem}
\begin{proof}
	At point $ p$, since  $ \exp_p$ is a local diffeomorphism, there exists open sets $ V \subseteq T_pM$ around origin and $ U \subseteq M$ around $ p$ s.t.\ $ V \cong U$ under $ \exp_p$. Since for every point $ v$ in $ V$, we have a canonical orthonormal frame $ F_1(v), \ldots , F_n(v) \in T_v(T_pM)$, under the pushforward by diffeomorphism we obtain vector fields $ E_i(u) := d (\exp_p)_v(F_i)$ in $ \mathfrak{X}(U)$. By Gauss's Lemma, we obtain
\begin{align*}
	\langle E_i, E_j \rangle = \langle d(\exp_p)_v(F_i), d(\exp_p)_v(F_j)  \rangle = \langle F_i , F_j \rangle = 0 . 
\end{align*}
Since the exponential map preserves length, we conclude that $ E_i$ is an orthonormal frame of $ U$. Moreover, we know that  $ d(\exp_p)_0 $ is the identity, so $ E_i(p) = F_i(0)$. In fact, they exactly coincide. Thus, $ \nabla _{E_i(p)} E_j (p)  = \nabla _{F_i(0)} F_j(0) =0$, and $ U$ is the neighborhood we seek.
\end{proof}
\end{document}
