\documentclass[12pt]{article}
\newcommand{\alert}[1]{{\bf \color{red} [Alert:] #1}}
\newcommand{\todo}[1]{{\bf \color{orange} [TODO:] #1}}
\newcommand{\real}[1][]{\mathbb{R}^{#1}}
\newcommand{\myeqn}[1]{(\ref{#1})}
\newcommand{\myex}[1]{Example \ref{#1}}
\newcommand{\defeq}{\stackrel{\mathrm{def}}{=}}
\newcommand{\parder}[2]{\frac{\partial #1}{\partial #2}}
\newcommand{\Lie}[3][]{\mathsf{L}_{#3}^{#1} #2}
\newcommand{\LieA}[1]{\mathsf{Lie}(#1)}
\newcommand{\lieder}[2]{\mathcal{L}_{#2} #1}
\renewcommand{\t}{^{\mbox{\tiny\sf T}}}
\newcommand{\trans}{^{\mbox{\tiny\sf T}}}
\newcommand{\markup}[1]{\{\textbf{#1}\}}
\newcommand{\msub}[1]{_\mathrm{#1}}
\newcommand{\msup}[1]{^\mathrm{#1}}
\newcommand{\inv}[1]{#1^{-1}}
\newcommand{\pinv}[1]{{#1}^{+}}
\newcommand{\myfracA}[2]{\displaystyle{\frac{#1}{#2}}}
\newcommand{\myfracB}[2]{{#1}/{#2}}
\newcommand{\mydiffA}[1]{\dot{#1}}
\newcommand{\mydiffB}[2]{\myfracA{\mathrm{d}{#1}}{\mathrm{d}{#2}}}
\newcommand{\ball}[2]{\mathcal{B}_{#1}\left(#2\right)}
\newcommand{\acos}[1]{\cos^{-1}\left(#1\right)}
\newcommand{\asin}[1]{\sin^{-1}\left(#1\right)}
\newcommand{\mani}{\mathcal{M}}
\newcommand{\tang}[2]{\mathsf{T}_{#1} #2}
\newcommand{\LieB}[2]{[ #1, #2 ]}
\newcommand{\LieBad}[3][]{\mathsf{ad}_{#2}^{#1} #3}
\newcommand{\ReachVT}{\mathcal{R}^V_T}
\newcommand{\ReachVt}{\mathcal{R}^V_t}
\newcommand{\ReachVTe}{\mathcal{R}^V_{\le T}}
\newcommand{\ReachT}{\mathcal{R}_T}
\newcommand{\Reacht}{\mathcal{R}_t}
\newcommand{\ReachTe}{\mathcal{R}_{\le T}}
\newcommand{\accLA}[1]{\mathsf{Lie}(#1)}
\newcommand{\accD}{\Delta_{\mathcal{F}}}
\newcommand{\accSA}{\mathsf{Lie}(\mathcal{G},f)}
\newcommand{\accDS}{\Delta_{\mathcal{G}}}
\newcommand{\eval}[3]{\mathsf{Ev}^{#2}_{#1}\left( #3 \right)}
\newcommand{\stlc}{\textsc{stlc}}
\newcommand{\clf}{\textsc{clf}}
\newcommand{\jqlf}{\textsc{jqlf}}
\newcommand{\dlas}{\textsc{dlas}}
\newcommand{\Ad}[2]{\mathsf{Ad}_{#1} #2}
\newcommand{\xe}{\ensuremath{x_e}}
\newcommand{\lebg}[1]{\mathcal{L}_{#1}}
\newcommand{\lebgx}[1]{\mathcal{L}_{#1 \mathrm{e}}}
\newcommand{\dom}{D}
\newcommand{\domT}{[t_0,\infty) \times D}
\newcommand{\rarrow}{\rightarrow}
\renewcommand{\d}{\mathrm{d}}
\renewcommand{\Re}{\mathbb{R}}
\newcommand{\C}{\mathrm{C}}

\newcommand{\QED}{{\unskip\nobreak\hfil\penalty50\hskip2em\vadjust{}
		\nobreak\hfil$\Box$\parfillskip=0pt\finalhyphendemerits=0\par}\vspace{0.1cm}}
\newcommand{\eoEx}{{\unskip\nobreak\hfil\penalty50\hskip0em\vadjust{}
		\nobreak\hfil$\Large\Diamond$\parfillskip=0pt\finalhyphendemerits=0\par}\vspace{0.1cm}}

\newcommand{\sgn}{\ensuremath{\operatorname{sgn}}}
\newcommand{\sat}{\ensuremath{\operatorname{sat}}}

\newcommand{\half}{\frac{1}{2}}
\newcommand{\shalf}{\mbox{$\frac{1}{2}$}}
\newcommand{\marcom}[1]{\marginpar{\footnotesize #1}}
\newcommand{\der}{\mathrm{D}}
\newcommand{\e}{\mathrm{e}}
\newcommand{\dt}{\mathrm{d}t}

\newcommand{\cA}{\ensuremath{\mathcal{A}}}
\newcommand{\cB}{\ensuremath{\mathcal{B}}}
\newcommand{\cG}{\ensuremath{\mathcal{G}}}
\newcommand{\cK}{\ensuremath{\mathcal{K}}}
\newcommand{\cW}{\ensuremath{\mathcal{W}}}
\newcommand{\cZ}{\ensuremath{\mathcal{Z}}}
\newcommand{\cS}{\ensuremath{\mathcal{S}}}
\newcommand{\cD}{\ensuremath{\mathcal{D}}}
\newcommand{\cP}{\ensuremath{\mathcal{P}}}
\newcommand{\cV}{\ensuremath{\mathcal{V}}}
\newcommand{\cL}{\ensuremath{\mathcal{L}}}
\newcommand{\cN}{\ensuremath{\mathcal{N}}}
\newcommand{\cI}{\ensuremath{\mathcal{I}}}
\newcommand{\cR}{\ensuremath{\mathcal{R}}}
\newcommand{\cM}{\ensuremath{\mathcal{M}}}
\newcommand{\cC}{\ensuremath{\mathcal{C}}}
\newcommand{\cF}{\ensuremath{\mathcal{F}}}
\newcommand{\cH}{\ensuremath{\mathcal{H}}}
\newcommand{\cO}{\ensuremath{\mathcal{O}}}
\newcommand{\cX}{\ensuremath{\mathcal{X}}}
\newcommand{\cY}{\ensuremath{\mathcal{Y}}}
\newcommand{\Ci}{\ensuremath{\mathcal{C}^\infty}}
\newcommand{\ISS}{\textsc{iss}}
\newcommand{\LISS}{\textsc{liss}}
\newcommand{\GAS}{\textsc{gas}}
\newcommand{\GS}{\textsc{gs}}
\newcommand{\LES}{\textsc{les}}
\newcommand{\GUAS}{\textsc{guas}}
\newcommand{\BIBO}{\textsc{bibo}}
\newcommand{\spec}{\ensuremath{\operatorname{spec}}}
\newcommand{\spn}{\ensuremath{\operatorname{span}}}
\renewcommand{\i}{\mathrm{i\,}}

\renewcommand{\implies}{\Rightarrow}

\renewcommand{\theenumi}{$\roman{enumi})$}
\renewcommand{\labelenumi}{\theenumi}

\font\ptmten=zptmcmrm scaled 1200
\newcommand{\w}{\mbox{{\ptmten w}}}
\newcommand{\z}{\mbox{{\ptmten z}}}
\renewcommand{\Re}{\mathbb{R}}

\newcommand{\cl}{\operatorname{cl}}
\newcommand{\intr}{\operatorname{int}}
\newcommand{\rank}{\operatorname{rank}}
\newcommand{\co}{\operatorname{co}}
\newcommand{\aff}{\operatorname{aff}}

\theoremstyle{plain}
\newtheorem{theorem}{Theorem}[chapter]
\newtheorem{claim}[theorem]{Claim}
\newtheorem{corollary}[theorem]{Corollary}
\newtheorem{prop}[theorem]{Proposition}
\newtheorem{fact}[theorem]{Fact}
\newtheorem{lemma}[theorem]{Lemma}

\newtheorem{remark}{Remark}[chapter]

\theoremstyle{definition}
\newtheorem{assume}[theorem]{Assumption}
\newtheorem{defn}[theorem]{Definition}
\newtheorem{problem}[theorem]{Problem}
\newtheorem{exercise}{Exercise}
\newtheorem{example}[theorem]{Example}


\begin{document}
\centerline {\textsf{\textbf{\LARGE{Homework 6}}}}
\centerline {Jaden Wang}
\vspace{.15in}
\begin{problem}[LN15 0.4]
	Show that the bracket satisfies the following properties: $ [X,Y] = - [Y,X]$ and  $ [X,[Y,Z]] + [Y,[Z,X]] + [Z,[X,Y]] =0$.
\end{problem}
\begin{proof}
	The pointwise definition of Lie bracket can be expressed succinctly as $ [X,Y] = XY-YX$. Then we have
\begin{align*}
	[X,Y] = XY - YX = -(YX-XY) = - [Y,X],
\end{align*}
and
\begin{align*}
	[X,[Y,Z]] + [Y,[Z,X]] + [Z,[X,Y]] =& XYZ-XZY-YZX+ZYX \\
	& YZX- ZYX-ZXY+YXZ\\
	& ZXY - YXZ-XYZ + XZY\\
	=& 0 .
\end{align*}
\end{proof}
\begin{problem}[LN15 0.5]
Show that a connection is symmetric iff the corresponding Christoffel symbol satisfy $ \Gamma_{ii}^{k} = \Gamma_{ji}^{k}$.
\end{problem}
\begin{proof}
Recall that in local coordinates, using Einstein notation we have
\begin{align*}
	\nabla _XY = \left( X(Y^{k}) + X^{j} Y^{i} \Gamma_{ij}^{k} \right) E_k . 
\end{align*}
Then by unifying $ i,j$ indices of $ \nabla _X Y$ and $ \nabla _YX$, we obtain
\begin{align*}
	\nabla _XY - \nabla _YX &= \left( X(Y^{k}) - Y(X^{k}) + X^{i}Y^{j} \left( \Gamma_{ji}^{k} - \Gamma_{ij}^{k} \right)  \right) E_k \\
				&=X(Y^{k})E_k - Y(X^{k})E_k + X^{i}Y^{j} \left( \Gamma_{ji}^{k} - \Gamma_{ij}^{k} \right) E_k \\
				&=XY-YX + X^{i}Y^{j} \left( \Gamma_{ji}^{k} - \Gamma_{ij}^{k} \right) E_k \\
				&=[X,Y] + X^{i}Y^{j} \left( \Gamma_{ji}^{k} - \Gamma_{ij}^{k} \right) E_k .
\end{align*}
Since $ X,Y$ are arbitrary, we see that  $ \nabla _X Y- \nabla _YX = [X,Y]$ iff $ \Gamma_{ji}^{k} - \Gamma_{ij}^{k} = 0$.
\end{proof}
\begin{problem}[do Carmo 3.1]

\end{problem}
\begin{proof}
First, we compute the Jacobian of $ \phi$:
\begin{align*}
	D \phi(u,v) &= \begin{pmatrix} \frac{\partial \phi}{\partial u}(u,v) & \frac{\partial \phi}{\partial v}(u,v)  \end{pmatrix} \\
		    &= \begin{pmatrix} -f(v) \sin u & f'(v) \cos u\\ f(v) \cos u & f'(v) \sin u\\ 0 & g'(v) \end{pmatrix} .
\end{align*}
Since $ f'(v)^2 + g'(v)^2 \neq 0$ and $ f(v) \neq 0$, we can check
\begin{align*}
	\langle \frac{\partial \phi}{\partial u}, \frac{\partial \phi}{\partial u}  \rangle &= f(v)^2 \neq 0 \\
	\langle \frac{\partial \phi}{\partial u}, \frac{\partial \phi}{\partial v}  \rangle &= 0 \\
	\langle \frac{\partial \phi}{\partial v} , \frac{\partial \phi}{\partial v}   \rangle &= f'(v)^2 + g'(v)^2 \neq 0
\end{align*}
That is, they are nonzero and their dot product is 0, so they are orthogonal. Thus $ D \phi$ is injective everywhere, \emph{i.e.} $ \phi$ is an immersion.

\begin{enumerate}[label=(\alph*)]
	\item Recall that the induced metric is just the pairwise dot products of the pushforward basis under $ \phi$. Thus by the above computation, we have $ g_{11} = f^2$, $ g_{12} = 0$, and $ g_{22} = (f')^2+(g')^2$, which are all functions of $ v$.
	\item Then $ g^{11} = \frac{1}{f^2}$, $ g^{12} = 0$, $ g^{22} = \frac{1}{(f')^2+(g')^2}$. Based on the equation $ \Gamma_{ij}^{m} = \frac{1}{2} g^{  km} \left(- \frac{\partial g_{ij}}{\partial x_k} + \frac{\partial g_{jk}}{\partial x_i}  + \frac{\partial g_{ki}}{\partial x_j}  \right) $, we obtain  
\begin{align*}
	\Gamma_{11}^{1} &= \frac{1}{2f^2} \left( -0+0+0 \right) + 0 = 0\\ 
	\Gamma_{11}^{2} &= 0 + \frac{1}{2((f')^2+(g')^2)} \left( -2ff' + 0 +0  \right) =-\frac{ff'}{(f')^2+(g')^2}    \\
	\Gamma_{12}^{1}&= \frac{1}{2f^2} (-0+0+2ff') + 0 =  \frac{ff'}{f^2 } \\
	\Gamma_{12}^{2} &= 0 + \frac{1}{2((f')^2+(g')^2)}(-0+0+0) = 0   \\
	\Gamma_{21}^{1} &= \Gamma_{12}^{1} = \frac{ff'}{f^2 } \\
	\Gamma_{21}^{2} &= \Gamma_{12}^{2}=0 \\
\Gamma_{22}^{1} &= \frac{1}{2f^2} ( -0+ 0+0)+ 0 = 0 \\
\Gamma_{22}^{2} &= 0 + \frac{1}{2((f')^2+(g')^2)}(-1+1+1) (2f'f''+2g'g'')  \\&= \frac{f'f''+g'g''}{(f')^2+(g')^2}   .
\end{align*}
The equation of geodesic $ \gamma(t) = (u(t),v(t)) $ is
\begin{align*}
	\ddot{ \gamma}^{k} + \dot{ \gamma}^{i} \dot{ \gamma}^{j} \Gamma_{ij}^{k}( \gamma) &= 0 \\ 
	\begin{cases}
	\ddot{u} + \dot{u} \dot{v} \frac{f f'}{ f^2} + \dot{v} \dot{u} \frac{ f f'}{f^2 } &= 0 \\
	\ddot{v} - \dot{u}^2 \frac{ff'}{ (f')^2+(g')^2}+ \dot{v}^2 \frac{f'f''+g'g''}{ (f')^2+(g')^2} &= 0 \\
	\end{cases}\\
	\begin{cases}
	\ddot{u} +  \frac{2f f'}{ f^2} \dot{u} \dot{v} &= 0 \\
	\ddot{v} -  \frac{ff'}{ (f')^2+(g')^2} \dot{u}^2+ \frac{f'f''+g'g''}{ (f')^2+(g')^2} \dot{v}^2 &= 0 \\
	\end{cases}
\end{align*}
\item We compute
\begin{align*}
	| \dot{ \gamma}|^2 &= \begin{pmatrix} \dot{u} & \dot{v} \end{pmatrix} \begin{pmatrix} f^2 & 0\\ 0& (f')^2+(g')^2 \end{pmatrix} \begin{pmatrix} \dot{u}\\\dot{v} \end{pmatrix}   \\
	&= f^2 \dot{u}^2 + ((f')^2+(g')^2) \dot{v}^2 .
\end{align*}
Taking the time derivative of this equation, using $ \dot{f} = f'\dot{v}$, $ \dot{f'} = f'' \dot{v}$, and the first equation $ \ddot{u} = - \frac{2ff'}{ f^2} \dot{u}\dot{v}$ we obtain
\begin{align*}
	&2ff' \dot{v} \dot{u}^2 + 2f^2 \dot{u} \ddot{u} + 2(f'f'' + g'g'') \dot{v}^3 + 2 ((f')^2+(g')^2) \dot{v} \ddot{v} \\
	=& 2\dot{v}(-ff'\dot{u}^2 + (f'f''+g'g'')\dot{v}^2 + ((f')^2+(g')^2)\ddot{v}).
\end{align*}
Since we exclude parallels, $ \dot{v} \neq 0$. The energy is constant iff time derivative is 0, where we can simply divide the equation by $2 \dot{v} ( (f')^2+(g')^2) \neq 0$ to obtain the second equation. 

Let $ P(t) = (a(t), b)$ where $ b$ is constant. Then $ \dot{P}(t) = (\dot{a}(t),0)$. Recall that
\begin{align*}
	\cos \beta &= \frac{ \langle \dot{ \gamma}, \dot{P} \rangle_{(u,v)}}{ | \dot{ \gamma}||\dot{P}|} \\
	&= \frac{ f^2 \dot{u} \dot{a}}{ | \dot{ \gamma}| |f\dot{a}| } \\
	&= \frac{ \sgn(f) \sgn(\dot{a}) f \dot{u}}{ | \dot{ \gamma}|} .
\end{align*}
Using the fact that $ \frac{d}{dt}| \dot{ \gamma}| =0$, the time derivative of $ f(v) \cos \beta$ is
\begin{align*}
	f'(v) \dot{v} \cos \beta + f(v) \frac{d}{dt} (\cos \beta) &= \frac{ \sgn(f) \sgn(\dot{a}) ff' \dot{u}\dot{v}}{ | \dot{ \gamma}|} +  \frac{\sgn(f) \sgn(\dot{a})f(f'\dot{v}\dot{u}+f \ddot{u}) | \dot{ \gamma}| - 0}{ |\dot{ \gamma}|^2} \\
								  &= \frac{\sgn(f) \sgn(\dot{a})}{| \dot{ \gamma}|} (f \ddot{u} + 2 ff'\dot{u}\dot{v}) .
\end{align*}
Since $ r \cos \beta = |f(v)| \cos \beta$, it is constant iff this time derivative is 0 iff the first equation holds.
\item Since $ r = |v|$, and  $ |v| \cos \beta $ is constant, the constant $ c$ is either zero or nonzero. If $ c=0$, then since we can vary $ v$ it must be that  $ \cos \beta = 0$. Since $ \beta < \pi$, it must be that $ \beta \equiv \frac{\pi}{2}$, which means the geodesic must intersect parallels at right angle all the time, making it a meridian which we exclude. Thus $ c$ must be nonzero and WLOG let $ c>0$. This forces $ \cos \beta >0$. Since we can decrease the radius at will, $ \cos \beta$ is forced to increase. But $ \cos \beta$ max out at 1, so it must be that $ \beta = 0$ precisely when $ |v| = c$. Since we can no longer decrease $ |v|$ further, and  $ |v|$ is not allowed to be constant,  $ |v|$ must increase instead. Therefore, $ \gamma$ is going up again. If we can argue that $ \gamma$ must always rotate around and is trapped between a minimum $ r$ and a maximum  $ r$, we would complete the proof. But this seems tedious.
\end{enumerate}
\end{proof}

\begin{problem}[do Carmo 3.7]
Let $ M$ be a Riemannian manifold of dimension $ n$ and let  $ p \in M$. Show that there exists a neighborhood $ U \subseteq M$ of $ p$ and  $ n$ vector fields  $ E_1,\ldots,E_n \in \mathfrak{ X} (U) $, orthonormal at each point of $ U$,  s.t.\ at $ p$,  $ \nabla _{E_i} E_j(p) =0$. This is called a geodesic frame at $ p$.
\end{problem}
\begin{proof}
	At point $ p$, since  $ \exp_p$ is a local diffeomorphism, there exists open sets $ V \subseteq T_pM$ around origin and $ U \subseteq M$ around $ p$ s.t.\ $ V \cong U$ under $ \exp_p$. Since for every point $ v$ in $ V$, we have a canonical orthonormal frame $ F_1(v), \ldots , F_n(v) \in T_v(T_pM)$, under the pushforward by diffeomorphism we obtain vector fields $ E_i(u) := d (\exp_p)_v(F_i)$ in $ \mathfrak{X}(U)$. By Gauss's Lemma, we obtain
\begin{align*}
	\langle E_i, E_j \rangle = \langle d(\exp_p)_v(F_i), d(\exp_p)_v(F_j)  \rangle = \langle F_i , F_j \rangle = 0 . 
\end{align*}
Since the exponential map preserves length, we conclude that $ E_i$ is an orthonormal frame of $ U$. Moreover, we know that  $ d(\exp_p)_0 $ is the identity, so $ E_i(p) = F_i(0)$. In fact, they exactly coincide. Thus, $ \nabla _{E_i(p)} E_j (p)  = \nabla _{F_i(0)} F_j(0) =0$, and $ U$ is the neighborhood we seek.
\end{proof}
\end{document}
