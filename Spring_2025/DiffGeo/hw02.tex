\documentclass[12pt]{article}
\newcommand{\alert}[1]{{\bf \color{red} [Alert:] #1}}
\newcommand{\todo}[1]{{\bf \color{orange} [TODO:] #1}}
\newcommand{\real}[1][]{\mathbb{R}^{#1}}
\newcommand{\myeqn}[1]{(\ref{#1})}
\newcommand{\myex}[1]{Example \ref{#1}}
\newcommand{\defeq}{\stackrel{\mathrm{def}}{=}}
\newcommand{\parder}[2]{\frac{\partial #1}{\partial #2}}
\newcommand{\Lie}[3][]{\mathsf{L}_{#3}^{#1} #2}
\newcommand{\LieA}[1]{\mathsf{Lie}(#1)}
\newcommand{\lieder}[2]{\mathcal{L}_{#2} #1}
\renewcommand{\t}{^{\mbox{\tiny\sf T}}}
\newcommand{\trans}{^{\mbox{\tiny\sf T}}}
\newcommand{\markup}[1]{\{\textbf{#1}\}}
\newcommand{\msub}[1]{_\mathrm{#1}}
\newcommand{\msup}[1]{^\mathrm{#1}}
\newcommand{\inv}[1]{#1^{-1}}
\newcommand{\pinv}[1]{{#1}^{+}}
\newcommand{\myfracA}[2]{\displaystyle{\frac{#1}{#2}}}
\newcommand{\myfracB}[2]{{#1}/{#2}}
\newcommand{\mydiffA}[1]{\dot{#1}}
\newcommand{\mydiffB}[2]{\myfracA{\mathrm{d}{#1}}{\mathrm{d}{#2}}}
\newcommand{\ball}[2]{\mathcal{B}_{#1}\left(#2\right)}
\newcommand{\acos}[1]{\cos^{-1}\left(#1\right)}
\newcommand{\asin}[1]{\sin^{-1}\left(#1\right)}
\newcommand{\mani}{\mathcal{M}}
\newcommand{\tang}[2]{\mathsf{T}_{#1} #2}
\newcommand{\LieB}[2]{[ #1, #2 ]}
\newcommand{\LieBad}[3][]{\mathsf{ad}_{#2}^{#1} #3}
\newcommand{\ReachVT}{\mathcal{R}^V_T}
\newcommand{\ReachVt}{\mathcal{R}^V_t}
\newcommand{\ReachVTe}{\mathcal{R}^V_{\le T}}
\newcommand{\ReachT}{\mathcal{R}_T}
\newcommand{\Reacht}{\mathcal{R}_t}
\newcommand{\ReachTe}{\mathcal{R}_{\le T}}
\newcommand{\accLA}[1]{\mathsf{Lie}(#1)}
\newcommand{\accD}{\Delta_{\mathcal{F}}}
\newcommand{\accSA}{\mathsf{Lie}(\mathcal{G},f)}
\newcommand{\accDS}{\Delta_{\mathcal{G}}}
\newcommand{\eval}[3]{\mathsf{Ev}^{#2}_{#1}\left( #3 \right)}
\newcommand{\stlc}{\textsc{stlc}}
\newcommand{\clf}{\textsc{clf}}
\newcommand{\jqlf}{\textsc{jqlf}}
\newcommand{\dlas}{\textsc{dlas}}
\newcommand{\Ad}[2]{\mathsf{Ad}_{#1} #2}
\newcommand{\xe}{\ensuremath{x_e}}
\newcommand{\lebg}[1]{\mathcal{L}_{#1}}
\newcommand{\lebgx}[1]{\mathcal{L}_{#1 \mathrm{e}}}
\newcommand{\dom}{D}
\newcommand{\domT}{[t_0,\infty) \times D}
\newcommand{\rarrow}{\rightarrow}
\renewcommand{\d}{\mathrm{d}}
\renewcommand{\Re}{\mathbb{R}}
\newcommand{\C}{\mathrm{C}}

\newcommand{\QED}{{\unskip\nobreak\hfil\penalty50\hskip2em\vadjust{}
		\nobreak\hfil$\Box$\parfillskip=0pt\finalhyphendemerits=0\par}\vspace{0.1cm}}
\newcommand{\eoEx}{{\unskip\nobreak\hfil\penalty50\hskip0em\vadjust{}
		\nobreak\hfil$\Large\Diamond$\parfillskip=0pt\finalhyphendemerits=0\par}\vspace{0.1cm}}

\newcommand{\sgn}{\ensuremath{\operatorname{sgn}}}
\newcommand{\sat}{\ensuremath{\operatorname{sat}}}

\newcommand{\half}{\frac{1}{2}}
\newcommand{\shalf}{\mbox{$\frac{1}{2}$}}
\newcommand{\marcom}[1]{\marginpar{\footnotesize #1}}
\newcommand{\der}{\mathrm{D}}
\newcommand{\e}{\mathrm{e}}
\newcommand{\dt}{\mathrm{d}t}

\newcommand{\cA}{\ensuremath{\mathcal{A}}}
\newcommand{\cB}{\ensuremath{\mathcal{B}}}
\newcommand{\cG}{\ensuremath{\mathcal{G}}}
\newcommand{\cK}{\ensuremath{\mathcal{K}}}
\newcommand{\cW}{\ensuremath{\mathcal{W}}}
\newcommand{\cZ}{\ensuremath{\mathcal{Z}}}
\newcommand{\cS}{\ensuremath{\mathcal{S}}}
\newcommand{\cD}{\ensuremath{\mathcal{D}}}
\newcommand{\cP}{\ensuremath{\mathcal{P}}}
\newcommand{\cV}{\ensuremath{\mathcal{V}}}
\newcommand{\cL}{\ensuremath{\mathcal{L}}}
\newcommand{\cN}{\ensuremath{\mathcal{N}}}
\newcommand{\cI}{\ensuremath{\mathcal{I}}}
\newcommand{\cR}{\ensuremath{\mathcal{R}}}
\newcommand{\cM}{\ensuremath{\mathcal{M}}}
\newcommand{\cC}{\ensuremath{\mathcal{C}}}
\newcommand{\cF}{\ensuremath{\mathcal{F}}}
\newcommand{\cH}{\ensuremath{\mathcal{H}}}
\newcommand{\cO}{\ensuremath{\mathcal{O}}}
\newcommand{\cX}{\ensuremath{\mathcal{X}}}
\newcommand{\cY}{\ensuremath{\mathcal{Y}}}
\newcommand{\Ci}{\ensuremath{\mathcal{C}^\infty}}
\newcommand{\ISS}{\textsc{iss}}
\newcommand{\LISS}{\textsc{liss}}
\newcommand{\GAS}{\textsc{gas}}
\newcommand{\GS}{\textsc{gs}}
\newcommand{\LES}{\textsc{les}}
\newcommand{\GUAS}{\textsc{guas}}
\newcommand{\BIBO}{\textsc{bibo}}
\newcommand{\spec}{\ensuremath{\operatorname{spec}}}
\newcommand{\spn}{\ensuremath{\operatorname{span}}}
\renewcommand{\i}{\mathrm{i\,}}

\renewcommand{\implies}{\Rightarrow}

\renewcommand{\theenumi}{$\roman{enumi})$}
\renewcommand{\labelenumi}{\theenumi}

\font\ptmten=zptmcmrm scaled 1200
\newcommand{\w}{\mbox{{\ptmten w}}}
\newcommand{\z}{\mbox{{\ptmten z}}}
\renewcommand{\Re}{\mathbb{R}}

\newcommand{\cl}{\operatorname{cl}}
\newcommand{\intr}{\operatorname{int}}
\newcommand{\rank}{\operatorname{rank}}
\newcommand{\co}{\operatorname{co}}
\newcommand{\aff}{\operatorname{aff}}

\theoremstyle{plain}
\newtheorem{theorem}{Theorem}[chapter]
\newtheorem{claim}[theorem]{Claim}
\newtheorem{corollary}[theorem]{Corollary}
\newtheorem{prop}[theorem]{Proposition}
\newtheorem{fact}[theorem]{Fact}
\newtheorem{lemma}[theorem]{Lemma}

\newtheorem{remark}{Remark}[chapter]

\theoremstyle{definition}
\newtheorem{assume}[theorem]{Assumption}
\newtheorem{defn}[theorem]{Definition}
\newtheorem{problem}[theorem]{Problem}
\newtheorem{exercise}{Exercise}
\newtheorem{example}[theorem]{Example}


\begin{document}
\centerline {\textsf{\textbf{\LARGE{Homework 2}}}}
\centerline {Jaden Wang}
\vspace{.15in}

\begin{problem}[LN 5.8]
	Show that if $ f: \rr^{n} \to \rr^{m}$ and we identify $ T_p \rr^{n}$ and $ T_{f(p)}\rr^{m}$ with $ \rr^{n}$ and $ \rr^{m}$ the standard way ($ [ \alpha] \mapsto \alpha'(0)$), then $ df_p$ may be identified with the linear transformation determined by the Jacobian matrix $ (\partial f^{i}/ \partial x_j)$.
\end{problem}
\begin{proof}
	Define $ \alpha_i: (- \epsilon, \epsilon) \to \rr^{n}, t\mapsto (0,\ldots,t,\ldots,0)$ where $ t$ is at the  $ i$th entry. Then  $ [ \alpha_i] \mapsto  \alpha_i'(0) = e_i$ under the identification. The identification yields
\begin{align*}
	df_p: T_p \rr^{n} \to T_{f(p)} \rr^{m}, [ \alpha] \mapsto [f \circ \alpha] \implies\\
	df_p : \rr^{n} \to \rr^{m}, \alpha'(0) \mapsto (f \circ \alpha)'(0).
\end{align*}
Then,
\begin{align*}
	df_p(e_i) &= df_p( \alpha'(0)) \\
	&= (f \circ \alpha_i)'(0) \\
	&= Df(0) \circ \alpha_i'(0) && \text{ Euclidean chain rule} \\
	&= Df(0)(e_i)
\end{align*}
Thus, we see that $ df_p$ (after identification) and the Jacobian matrix $ Df(0)$ agree on the standard basis. Therefore they represent the same linear map.
\end{proof}

The following two exercises show the functoriality of the differential operator.

\begin{problem}[LN 5.9]
Show that if $ f: M \to N$ and $ g:N \to L$ are smooth maps, then, for any $ p \in M$, we have the chain rule
\begin{align*}
	d(g \circ f)_p = d g_{f(p)} \circ df(p).
\end{align*}
\end{problem}
\begin{proof}
	Let $ [\gamma]$ be a tangent vector in $ T_pM$. Thus $ \gamma(0)=p$. Recall the definition (with identification):
\begin{align*}
	df_p( [ \gamma]) = (f \circ \gamma)'(0).
\end{align*}
By repeatedly applying this definition, we have
 \begin{align*}
	 d(g \circ f)_p ([ \gamma]) &= ((g \circ f) \circ \gamma)'(0)\\
				    &= ((g \circ (f \circ \gamma))'(0) \\
				    &= d g_{f \circ \alpha(0)} (f \circ \gamma)'(0) \\
				    &= d g_{f(p)} df_p([ \gamma]). 
\end{align*}
\end{proof}

\begin{problem}[LN 5.10]
Show that if $ f: M \to N$ is a diffeomorphism, then $ df_p$ is a linear isomorphism for all $ p \in M$. In particular, conclude that if $ M$ and  $ N$ are diffeomorphic, then  $ \dim (M) = \dim (N)$.

\end{problem}
\begin{proof}
Since $ f$ is a diffeomorphism, it admits a smooth inverse  $ f^{-1}$. Then given $ p \in M$, we have
\begin{align*}
	f^{-1} \circ f &= \text{id}_{M} \\
	d(f^{-1} \circ f)_p &= \text{id}_{T_pM} && \text{differential of identity is identity}  \\
	d(f^{-1})_{f(p)} \circ df_p &= \text{id}_{T_pM}. && \text{chain rule} 
\end{align*}
The other direction follows similarly. Thus $ df_p$ has a two-sided linear inverse --- it is a linear isomorphism. By linear algebra, the tangent spaces of $ M$ and  $ N$ must have the same dimension for all points. Since the tangent space and the manifold must have the same dimension,  $ \dim M = \dim N$.
\end{proof}

\begin{problem}[do Carmo 0.2]
Prove that the tangent bundle of a smooth manifold $ M$ is orientable.
\end{problem}
\begin{proof}
Let $ \{U_{ \alpha}, \phi_{ \alpha}\}_{ \alpha \in J} $ be an atlas of $ M$. Since $ \phi_{ \alpha}$ is a diffeomorphism, the differential $ d\phi_{ \alpha}: TU_{ \alpha} \to \rr^{n} \times  \rr^{n}, (p,v) \mapsto (\phi_{ \alpha}(p), d\phi_p(v)) $ is a diffeomorphism. Thus $ \{TU_{ \alpha}, (\phi_{ \alpha}, d\phi_{ \alpha})\}_{ \alpha \in J} $ is an atlas of $ TM$. For any $ p \in U_{ \alpha} \cap U_{ \beta}$, let $ q = \phi_{ \alpha}(p)$. The transition map is
\begin{align*}
	d\phi_{ \beta} \circ d\phi_{ \alpha}^{-1}(q,v) &= (\phi_{ \beta} \circ \phi_{ \alpha}^{-1}(q), d\phi_{ \beta}_{p} \circ d\phi^{-1}_{ \alpha}_q(v)) \\
	d(d\phi_{ \beta} \circ d\phi_{ \alpha}^{-1})_{(q,v)} &= (d(\phi_{ \beta} \circ \phi_{ \alpha}^{-1})_q, d(d\phi_{ \beta}_{p} \circ d\phi_{ \alpha}^{-1}_q)_v) \\
	&= (d(\phi_{ \beta} \circ \phi_{ \alpha}^{-1})_q,d(d(\phi_{ \beta} \circ \phi_{ \alpha}^{-1})_q)_v).
\end{align*}
Notice that $\phi_{ \alpha \beta }_q := d(\phi_{ \beta} \circ \phi_{ \alpha}^{-1})_q$ is a linear operator, so its derivative is itself for any $ v$. It is also the derivative of a diffeomorphism so it has nonzero determinant. Thus we obtain
\begin{align*}
	\det d(d\phi_{ \beta} \circ d\phi_{ \alpha}^{-1})_{(q,v)}&= \det ( \phi_{ \alpha, \beta}_q, \phi_{ \alpha, \beta}_q) \\
								 &= \det(\phi_{ \alpha, \beta}_q)^2 >0 ,
\end{align*}
where the last equality comes from determinant of the product linear operator. This proves that the transition maps are orientation-preserving and thus $ TM$ is orientable.
\end{proof}

\begin{problem}[do Carmo 0.8]
Let $ M_1,M_2$ be smooth manifolds. Let $ f: M_1 \to M_2$ be a local diffeomorphism. Prove that if $ M_2$ is orientable, then so is $ M_1$.
\end{problem}
\begin{proof}
Since $ M_2$ is orientable, it admits an atlas $ \{(V_{ \alpha},\psi_{ \alpha})\}_{ \alpha \in J} $ where all transition maps are orientation-preserving, \emph{i.e.} the determinant is positive. For any point $x \in M_1$, there exists an open set $ O_x$  such that  $ f|_{O_x}$ is a diffeomorphism onto its image. For any $ V_{ \alpha}$ that contain $ f(x)$, we set $ U_{ \alpha,x} = f ^{-1}(V_{ \alpha} \cap f(O_x))$, $ \phi_{ \alpha,x} = \psi_{ \alpha} \circ f|_{U_{ \alpha,x}}$. Then if there is another $ U_{ \beta,x}$ that contains $ x$, for any $ p \in U_{ \alpha,x} \cap U_{ \beta,x}$, let $ q = \phi_{ \alpha,x}(p)$. Notice $ f(p) = \psi_{ \alpha}^{-1}(q)$. We abuse the notation $ f$ for  $ f|_{U_{ \alpha,x} \cap U_{ \beta,x}}$ below, the transition map is
\begin{align*}
	\phi_{ \beta,x} \circ \phi_{ \alpha,x}^{-1}(q) &= \psi_{ \beta} \circ f \circ f^{-1} \circ \psi_{ \alpha}^{-1}(q) \\
	d(\phi_{ \beta,x} \circ \phi_{ \alpha,x}^{-1})_q &= d \psi_{ \beta}_{ f(p)} \circ df_p \circ df^{-1}_{f(p)} \circ d (\psi_{ \alpha}^{-1})_q \\
\det d(\phi_{ \beta,x} \circ \phi_{ \alpha,x}^{-1})_q &= \left( \det  d \psi_{ \beta}_{f(p)} \det  d (\psi_{ \alpha}^{-1})_q \right) \left(\det  df_p \det  df^{-1}_{f(p)} \right) \\
\det d(\phi_{ \beta,x} \circ \phi_{ \alpha,x}^{-1})_q &= \det \left( d \psi_{ \beta}_{f(p)} d (\psi_{ \alpha}^{-1})_q \right) > 0.
\end{align*}
Therefore, $ \{(U_{ \alpha, x}, \phi_{ \alpha,x})\}_{ \alpha \in J, x \in M_1} $ is an atlas of $ M_1$ with orientation-preserving maps.
\end{proof}
\end{document}
