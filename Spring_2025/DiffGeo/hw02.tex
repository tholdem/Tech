\documentclass[12pt]{article}
%Fall 2022
% Some basic packages
\usepackage{standalone}[subpreambles=true]
\usepackage[utf8]{inputenc}
\usepackage[T1]{fontenc}
\usepackage{textcomp}
\usepackage[english]{babel}
\usepackage{url}
\usepackage{graphicx}
%\usepackage{quiver}
\usepackage{float}
\usepackage{enumitem}
\usepackage{lmodern}
\usepackage{comment}
\usepackage{hyperref}
\usepackage[usenames,svgnames,dvipsnames]{xcolor}
\usepackage[margin=1in]{geometry}
\usepackage{pdfpages}

\pdfminorversion=7

% Don't indent paragraphs, leave some space between them
\usepackage{parskip}

% Hide page number when page is empty
\usepackage{emptypage}
\usepackage{subcaption}
\usepackage{multicol}
\usepackage[b]{esvect}

% Math stuff
\usepackage{amsmath, amsfonts, mathtools, amsthm, amssymb}
\usepackage{bbm}
\usepackage{stmaryrd}
\allowdisplaybreaks

% Fancy script capitals
\usepackage{mathrsfs}
\usepackage{cancel}
% Bold math
\usepackage{bm}
% Some shortcuts
\newcommand{\rr}{\ensuremath{\mathbb{R}}}
\newcommand{\zz}{\ensuremath{\mathbb{Z}}}
\newcommand{\qq}{\ensuremath{\mathbb{Q}}}
\newcommand{\nn}{\ensuremath{\mathbb{N}}}
\newcommand{\ff}{\ensuremath{\mathbb{F}}}
\newcommand{\cc}{\ensuremath{\mathbb{C}}}
\newcommand{\ee}{\ensuremath{\mathbb{E}}}
\newcommand{\hh}{\ensuremath{\mathbb{H}}}
\renewcommand\O{\ensuremath{\emptyset}}
\newcommand{\norm}[1]{{\left\lVert{#1}\right\rVert}}
\newcommand{\dbracket}[1]{{\left\llbracket{#1}\right\rrbracket}}
\newcommand{\ve}[1]{{\bm{#1}}}
\newcommand\allbold[1]{{\boldmath\textbf{#1}}}
\DeclareMathOperator{\lcm}{lcm}
\DeclareMathOperator{\im}{im}
\DeclareMathOperator{\coim}{coim}
\DeclareMathOperator{\dom}{dom}
\DeclareMathOperator{\tr}{tr}
\DeclareMathOperator{\rank}{rank}
\DeclareMathOperator*{\var}{Var}
\DeclareMathOperator*{\ev}{E}
\DeclareMathOperator{\dg}{deg}
\DeclareMathOperator{\aff}{aff}
\DeclareMathOperator{\conv}{conv}
\DeclareMathOperator{\inte}{int}
\DeclareMathOperator*{\argmin}{argmin}
\DeclareMathOperator*{\argmax}{argmax}
\DeclareMathOperator{\graph}{graph}
\DeclareMathOperator{\sgn}{sgn}
\DeclareMathOperator*{\Rep}{Rep}
\DeclareMathOperator{\Proj}{Proj}
\DeclareMathOperator{\mat}{mat}
\DeclareMathOperator{\diag}{diag}
\DeclareMathOperator{\aut}{Aut}
\DeclareMathOperator{\gal}{Gal}
\DeclareMathOperator{\inn}{Inn}
\DeclareMathOperator{\edm}{End}
\DeclareMathOperator{\Hom}{Hom}
\DeclareMathOperator{\ext}{Ext}
\DeclareMathOperator{\tor}{Tor}
\DeclareMathOperator{\Span}{Span}
\DeclareMathOperator{\Stab}{Stab}
\DeclareMathOperator{\cont}{cont}
\DeclareMathOperator{\Ann}{Ann}
\DeclareMathOperator{\Div}{div}
\DeclareMathOperator{\curl}{curl}
\DeclareMathOperator{\nat}{Nat}
\DeclareMathOperator{\gr}{Gr}
\DeclareMathOperator{\vect}{Vect}
\DeclareMathOperator{\id}{id}
\DeclareMathOperator{\Mod}{Mod}
\DeclareMathOperator{\sign}{sign}
\DeclareMathOperator{\Surf}{Surf}
\DeclareMathOperator{\fcone}{fcone}
\DeclareMathOperator{\Rot}{Rot}
\DeclareMathOperator{\grad}{grad}
\DeclareMathOperator{\atan2}{atan2}
\DeclareMathOperator{\Ric}{Ric}
\let\vec\relax
\DeclareMathOperator{\vec}{vec}
\let\Re\relax
\DeclareMathOperator{\Re}{Re}
\let\Im\relax
\DeclareMathOperator{\Im}{Im}
% Put x \to \infty below \lim
\let\svlim\lim\def\lim{\svlim\limits}

%wide hat
\usepackage{scalerel,stackengine}
\stackMath
\newcommand*\wh[1]{%
\savestack{\tmpbox}{\stretchto{%
  \scaleto{%
    \scalerel*[\widthof{\ensuremath{#1}}]{\kern-.6pt\bigwedge\kern-.6pt}%
    {\rule[-\textheight/2]{1ex}{\textheight}}%WIDTH-LIMITED BIG WEDGE
  }{\textheight}% 
}{0.5ex}}%
\stackon[1pt]{#1}{\tmpbox}%
}
\parskip 1ex

%Make implies and impliedby shorter
\let\implies\Rightarrow
\let\impliedby\Leftarrow
\let\iff\Leftrightarrow
\let\epsilon\varepsilon

% Add \contra symbol to denote contradiction
\usepackage{stmaryrd} % for \lightning
\newcommand\contra{\scalebox{1.5}{$\lightning$}}

% \let\phi\varphi

% Command for short corrections
% Usage: 1+1=\correct{3}{2}

\definecolor{correct}{HTML}{009900}
\newcommand\correct[2]{\ensuremath{\:}{\color{red}{#1}}\ensuremath{\to }{\color{correct}{#2}}\ensuremath{\:}}
\newcommand\green[1]{{\color{correct}{#1}}}

% horizontal rule
\newcommand\hr{
    \noindent\rule[0.5ex]{\linewidth}{0.5pt}
}

% hide parts
\newcommand\hide[1]{}

% si unitx
\usepackage{siunitx}
\sisetup{locale = FR}

%allows pmatrix to stretch
\makeatletter
\renewcommand*\env@matrix[1][\arraystretch]{%
  \edef\arraystretch{#1}%
  \hskip -\arraycolsep
  \let\@ifnextchar\new@ifnextchar
  \array{*\c@MaxMatrixCols c}}
\makeatother

\renewcommand{\arraystretch}{0.8}

\renewcommand{\baselinestretch}{1.5}

\usepackage{graphics}
\usepackage{epstopdf}

\RequirePackage{hyperref}
%%
%% Add support for color in order to color the hyperlinks.
%% 
\hypersetup{
  colorlinks = true,
  urlcolor = blue,
  citecolor = blue
}
%%fakesection Links
\hypersetup{
    colorlinks,
    linkcolor={red!50!black},
    citecolor={green!50!black},
    urlcolor={blue!80!black}
}
%customization of cleveref
\RequirePackage[capitalize,nameinlink]{cleveref}[0.19]

% Per SIAM Style Manual, "section" should be lowercase
\crefname{section}{section}{sections}
\crefname{subsection}{subsection}{subsections}
\Crefname{section}{Section}{Sections}
\Crefname{subsection}{Subsection}{Subsections}

% Per SIAM Style Manual, "Figure" should be spelled out in references
\Crefname{figure}{Figure}{Figures}

% Per SIAM Style Manual, don't say equation in front on an equation.
\crefformat{equation}{\textup{#2(#1)#3}}
\crefrangeformat{equation}{\textup{#3(#1)#4--#5(#2)#6}}
\crefmultiformat{equation}{\textup{#2(#1)#3}}{ and \textup{#2(#1)#3}}
{, \textup{#2(#1)#3}}{, and \textup{#2(#1)#3}}
\crefrangemultiformat{equation}{\textup{#3(#1)#4--#5(#2)#6}}%
{ and \textup{#3(#1)#4--#5(#2)#6}}{, \textup{#3(#1)#4--#5(#2)#6}}{, and \textup{#3(#1)#4--#5(#2)#6}}

% But spell it out at the beginning of a sentence.
\Crefformat{equation}{#2Equation~\textup{(#1)}#3}
\Crefrangeformat{equation}{Equations~\textup{#3(#1)#4--#5(#2)#6}}
\Crefmultiformat{equation}{Equations~\textup{#2(#1)#3}}{ and \textup{#2(#1)#3}}
{, \textup{#2(#1)#3}}{, and \textup{#2(#1)#3}}
\Crefrangemultiformat{equation}{Equations~\textup{#3(#1)#4--#5(#2)#6}}%
{ and \textup{#3(#1)#4--#5(#2)#6}}{, \textup{#3(#1)#4--#5(#2)#6}}{, and \textup{#3(#1)#4--#5(#2)#6}}

% Make number non-italic in any environment.
\crefdefaultlabelformat{#2\textup{#1}#3}

% Environments
\makeatother
% For box around Definition, Theorem, \ldots
%%fakesection Theorems
\usepackage{thmtools}
\usepackage[framemethod=TikZ]{mdframed}

\theoremstyle{definition}
\mdfdefinestyle{mdbluebox}{%
	roundcorner = 10pt,
	linewidth=1pt,
	skipabove=12pt,
	innerbottommargin=9pt,
	skipbelow=2pt,
	nobreak=true,
	linecolor=blue,
	backgroundcolor=TealBlue!5,
}
\declaretheoremstyle[
	headfont=\sffamily\bfseries\color{MidnightBlue},
	mdframed={style=mdbluebox},
	headpunct={\\[3pt]},
	postheadspace={0pt}
]{thmbluebox}

\mdfdefinestyle{mdredbox}{%
	linewidth=0.5pt,
	skipabove=12pt,
	frametitleaboveskip=5pt,
	frametitlebelowskip=0pt,
	skipbelow=2pt,
	frametitlefont=\bfseries,
	innertopmargin=4pt,
	innerbottommargin=8pt,
	nobreak=false,
	linecolor=RawSienna,
	backgroundcolor=Salmon!5,
}
\declaretheoremstyle[
	headfont=\bfseries\color{RawSienna},
	mdframed={style=mdredbox},
	headpunct={\\[3pt]},
	postheadspace={0pt},
]{thmredbox}

\declaretheorem[%
style=thmbluebox,name=Theorem,numberwithin=section]{thm}
\declaretheorem[style=thmbluebox,name=Lemma,sibling=thm]{lem}
\declaretheorem[style=thmbluebox,name=Proposition,sibling=thm]{prop}
\declaretheorem[style=thmbluebox,name=Corollary,sibling=thm]{coro}
\declaretheorem[style=thmredbox,name=Example,sibling=thm]{eg}

\mdfdefinestyle{mdgreenbox}{%
	roundcorner = 10pt,
	linewidth=1pt,
	skipabove=12pt,
	innerbottommargin=9pt,
	skipbelow=2pt,
	nobreak=true,
	linecolor=ForestGreen,
	backgroundcolor=ForestGreen!5,
}

\declaretheoremstyle[
	headfont=\bfseries\sffamily\color{ForestGreen!70!black},
	bodyfont=\normalfont,
	spaceabove=2pt,
	spacebelow=1pt,
	mdframed={style=mdgreenbox},
	headpunct={ --- },
]{thmgreenbox}

\declaretheorem[style=thmgreenbox,name=Definition,sibling=thm]{defn}

\mdfdefinestyle{mdgreenboxsq}{%
	linewidth=1pt,
	skipabove=12pt,
	innerbottommargin=9pt,
	skipbelow=2pt,
	nobreak=true,
	linecolor=ForestGreen,
	backgroundcolor=ForestGreen!5,
}
\declaretheoremstyle[
	headfont=\bfseries\sffamily\color{ForestGreen!70!black},
	bodyfont=\normalfont,
	spaceabove=2pt,
	spacebelow=1pt,
	mdframed={style=mdgreenboxsq},
	headpunct={},
]{thmgreenboxsq}
\declaretheoremstyle[
	headfont=\bfseries\sffamily\color{ForestGreen!70!black},
	bodyfont=\normalfont,
	spaceabove=2pt,
	spacebelow=1pt,
	mdframed={style=mdgreenboxsq},
	headpunct={},
]{thmgreenboxsq*}

\mdfdefinestyle{mdblackbox}{%
	skipabove=8pt,
	linewidth=3pt,
	rightline=false,
	leftline=true,
	topline=false,
	bottomline=false,
	linecolor=black,
	backgroundcolor=RedViolet!5!gray!5,
}
\declaretheoremstyle[
	headfont=\bfseries,
	bodyfont=\normalfont\small,
	spaceabove=0pt,
	spacebelow=0pt,
	mdframed={style=mdblackbox}
]{thmblackbox}

\theoremstyle{plain}
\declaretheorem[name=Question,sibling=thm,style=thmblackbox]{ques}
\declaretheorem[name=Remark,sibling=thm,style=thmgreenboxsq]{remark}
\declaretheorem[name=Remark,sibling=thm,style=thmgreenboxsq*]{remark*}
\newtheorem{ass}[thm]{Assumptions}

\theoremstyle{definition}
\newtheorem*{problem}{Problem}
\newtheorem{claim}[thm]{Claim}
\theoremstyle{remark}
\newtheorem*{case}{Case}
\newtheorem*{notation}{Notation}
\newtheorem*{note}{Note}
\newtheorem*{motivation}{Motivation}
\newtheorem*{intuition}{Intuition}
\newtheorem*{conjecture}{Conjecture}

% Make section starts with 1 for report type
%\renewcommand\thesection{\arabic{section}}

% End example and intermezzo environments with a small diamond (just like proof
% environments end with a small square)
\usepackage{etoolbox}
\AtEndEnvironment{vb}{\null\hfill$\diamond$}%
\AtEndEnvironment{intermezzo}{\null\hfill$\diamond$}%
% \AtEndEnvironment{opmerking}{\null\hfill$\diamond$}%

% Fix some spacing
% http://tex.stackexchange.com/questions/22119/how-can-i-change-the-spacing-before-theorems-with-amsthm
\makeatletter
\def\thm@space@setup{%
  \thm@preskip=\parskip \thm@postskip=0pt
}

% Fix some stuff
% %http://tex.stackexchange.com/questions/76273/multiple-pdfs-with-page-group-included-in-a-single-page-warning
\pdfsuppresswarningpagegroup=1


% My name
\author{Jaden Wang}



\begin{document}
\centerline {\textsf{\textbf{\LARGE{Homework 2}}}}
\centerline {Jaden Wang}
\vspace{.15in}

\begin{problem}[LN 5.8]
	Show that if $ f: \rr^{n} \to \rr^{m}$ and we identify $ T_p \rr^{n}$ and $ T_{f(p)}\rr^{m}$ with $ \rr^{n}$ and $ \rr^{m}$ the standard way ($ [ \alpha] \mapsto \alpha'(0)$), then $ df_p$ may be identified with the linear transformation determined by the Jacobian matrix $ (\partial f^{i}/ \partial x_j)$.
\end{problem}
\begin{proof}
	Define $ \alpha_i: (- \epsilon, \epsilon) \to \rr^{n}, t\mapsto (0,\ldots,t,\ldots,0)$ where $ t$ is at the  $ i$th entry. Then  $ [ \alpha_i] \mapsto  \alpha_i'(0) = e_i$ under the identification. The identification yields
\begin{align*}
	df_p: T_p \rr^{n} \to T_{f(p)} \rr^{m}, [ \alpha] \mapsto [f \circ \alpha] \implies\\
	df_p : \rr^{n} \to \rr^{m}, \alpha'(0) \mapsto (f \circ \alpha)'(0).
\end{align*}
Then,
\begin{align*}
	df_p(e_i) &= df_p( \alpha'(0)) \\
	&= (f \circ \alpha_i)'(0) \\
	&= Df(0) \circ \alpha_i'(0) && \text{ Euclidean chain rule} \\
	&= Df(0)(e_i)
\end{align*}
Thus, we see that $ df_p$ (after identification) and the Jacobian matrix $ Df(0)$ agree on the standard basis. Therefore they represent the same linear map.
\end{proof}

The following two exercises show the functoriality of the differential operator.

\begin{problem}[LN 5.9]
Show that if $ f: M \to N$ and $ g:N \to L$ are smooth maps, then, for any $ p \in M$, we have the chain rule
\begin{align*}
	d(g \circ f)_p = d g_{f(p)} \circ df(p).
\end{align*}
\end{problem}
\begin{proof}
	Let $ [\gamma]$ be a tangent vector in $ T_pM$. Thus $ \gamma(0)=p$. Recall the definition (with identification):
\begin{align*}
	df_p( [ \gamma]) = (f \circ \gamma)'(0).
\end{align*}
By repeatedly applying this definition, we have
 \begin{align*}
	 d(g \circ f)_p ([ \gamma]) &= ((g \circ f) \circ \gamma)'(0)\\
				    &= ((g \circ (f \circ \gamma))'(0) \\
				    &= d g_{f \circ \alpha(0)} (f \circ \gamma)'(0) \\
				    &= d g_{f(p)} df_p([ \gamma]). 
\end{align*}
\end{proof}

\begin{problem}[LN 5.10]
Show that if $ f: M \to N$ is a diffeomorphism, then $ df_p$ is a linear isomorphism for all $ p \in M$. In particular, conclude that if $ M$ and  $ N$ are diffeomorphic, then  $ \dim (M) = \dim (N)$.

\end{problem}
\begin{proof}
Since $ f$ is a diffeomorphism, it admits a smooth inverse  $ f^{-1}$. Then given $ p \in M$, we have
\begin{align*}
	f^{-1} \circ f &= \text{id}_{M} \\
	d(f^{-1} \circ f)_p &= \text{id}_{T_pM} && \text{differential of identity is identity}  \\
	d(f^{-1})_{f(p)} \circ df_p &= \text{id}_{T_pM}. && \text{chain rule} 
\end{align*}
The other direction follows similarly. Thus $ df_p$ has a two-sided linear inverse --- it is a linear isomorphism. By linear algebra, the tangent spaces of $ M$ and  $ N$ must have the same dimension for all points. Since the tangent space and the manifold must have the same dimension,  $ \dim M = \dim N$.
\end{proof}

\begin{problem}[do Carmo 0.2]
Prove that the tangent bundle of a smooth manifold $ M$ is orientable.
\end{problem}
\begin{proof}
Let $ \{U_{ \alpha}, \phi_{ \alpha}\}_{ \alpha \in J} $ be an atlas of $ M$. Since $ \phi_{ \alpha}$ is a diffeomorphism, the differential $ d\phi_{ \alpha}: TU_{ \alpha} \to \rr^{n} \times  \rr^{n}, (p,v) \mapsto (\phi_{ \alpha}(p), d\phi_p(v)) $ is a diffeomorphism. Thus $ \{TU_{ \alpha}, (\phi_{ \alpha}, d\phi_{ \alpha})\}_{ \alpha \in J} $ is an atlas of $ TM$. For any $ p \in U_{ \alpha} \cap U_{ \beta}$, let $ q = \phi_{ \alpha}(p)$. The transition map is
\begin{align*}
	d\phi_{ \beta} \circ d\phi_{ \alpha}^{-1}(q,v) &= (\phi_{ \beta} \circ \phi_{ \alpha}^{-1}(q), d\phi_{ \beta}_{p} \circ d\phi^{-1}_{ \alpha}_q(v)) \\
	d(d\phi_{ \beta} \circ d\phi_{ \alpha}^{-1})_{(q,v)} &= (d(\phi_{ \beta} \circ \phi_{ \alpha}^{-1})_q, d(d\phi_{ \beta}_{p} \circ d\phi_{ \alpha}^{-1}_q)_v) \\
	&= (d(\phi_{ \beta} \circ \phi_{ \alpha}^{-1})_q,d(d(\phi_{ \beta} \circ \phi_{ \alpha}^{-1})_q)_v).
\end{align*}
Notice that $\phi_{ \alpha \beta }_q := d(\phi_{ \beta} \circ \phi_{ \alpha}^{-1})_q$ is a linear operator, so its derivative is itself for any $ v$. It is also the derivative of a diffeomorphism so it has nonzero determinant. Thus we obtain
\begin{align*}
	\det d(d\phi_{ \beta} \circ d\phi_{ \alpha}^{-1})_{(q,v)}&= \det ( \phi_{ \alpha, \beta}_q, \phi_{ \alpha, \beta}_q) \\
								 &= \det(\phi_{ \alpha, \beta}_q)^2 >0 ,
\end{align*}
where the last equality comes from determinant of the product linear operator. This proves that the transition maps are orientation-preserving and thus $ TM$ is orientable.
\end{proof}

\begin{problem}[do Carmo 0.8]
Let $ M_1,M_2$ be smooth manifolds. Let $ f: M_1 \to M_2$ be a local diffeomorphism. Prove that if $ M_2$ is orientable, then so is $ M_1$.
\end{problem}
\begin{proof}
Since $ M_2$ is orientable, it admits an atlas $ \{(V_{ \alpha},\psi_{ \alpha})\}_{ \alpha \in J} $ where all transition maps are orientation-preserving, \emph{i.e.} the determinant is positive. For any point $x \in M_1$, there exists an open set $ O_x$  such that  $ f|_{O_x}$ is a diffeomorphism onto its image. For any $ V_{ \alpha}$ that contain $ f(x)$, we set $ U_{ \alpha,x} = f ^{-1}(V_{ \alpha} \cap f(O_x))$, $ \phi_{ \alpha,x} = \psi_{ \alpha} \circ f|_{U_{ \alpha,x}}$. Then if there is another $ U_{ \beta,x}$ that contains $ x$, for any $ p \in U_{ \alpha,x} \cap U_{ \beta,x}$, let $ q = \phi_{ \alpha,x}(p)$. Notice $ f(p) = \psi_{ \alpha}^{-1}(q)$. We abuse the notation $ f$ for  $ f|_{U_{ \alpha,x} \cap U_{ \beta,x}}$ below, the transition map is
\begin{align*}
	\phi_{ \beta,x} \circ \phi_{ \alpha,x}^{-1}(q) &= \psi_{ \beta} \circ f \circ f^{-1} \circ \psi_{ \alpha}^{-1}(q) \\
	d(\phi_{ \beta,x} \circ \phi_{ \alpha,x}^{-1})_q &= d \psi_{ \beta}_{ f(p)} \circ df_p \circ df^{-1}_{f(p)} \circ d (\psi_{ \alpha}^{-1})_q \\
\det d(\phi_{ \beta,x} \circ \phi_{ \alpha,x}^{-1})_q &= \left( \det  d \psi_{ \beta}_{f(p)} \det  d (\psi_{ \alpha}^{-1})_q \right) \left(\det  df_p \det  df^{-1}_{f(p)} \right) \\
\det d(\phi_{ \beta,x} \circ \phi_{ \alpha,x}^{-1})_q &= \det \left( d \psi_{ \beta}_{f(p)} d (\psi_{ \alpha}^{-1})_q \right) > 0.
\end{align*}
Therefore, $ \{(U_{ \alpha, x}, \phi_{ \alpha,x})\}_{ \alpha \in J, x \in M_1} $ is an atlas of $ M_1$ with orientation-preserving maps.
\end{proof}
\end{document}
