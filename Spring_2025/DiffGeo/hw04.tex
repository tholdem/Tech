\documentclass[12pt]{article}
%Fall 2022
% Some basic packages
\usepackage{standalone}[subpreambles=true]
\usepackage[utf8]{inputenc}
\usepackage[T1]{fontenc}
\usepackage{textcomp}
\usepackage[english]{babel}
\usepackage{url}
\usepackage{graphicx}
%\usepackage{quiver}
\usepackage{float}
\usepackage{enumitem}
\usepackage{lmodern}
\usepackage{comment}
\usepackage{hyperref}
\usepackage[usenames,svgnames,dvipsnames]{xcolor}
\usepackage[margin=1in]{geometry}
\usepackage{pdfpages}

\pdfminorversion=7

% Don't indent paragraphs, leave some space between them
\usepackage{parskip}

% Hide page number when page is empty
\usepackage{emptypage}
\usepackage{subcaption}
\usepackage{multicol}
\usepackage[b]{esvect}

% Math stuff
\usepackage{amsmath, amsfonts, mathtools, amsthm, amssymb}
\usepackage{bbm}
\usepackage{stmaryrd}
\allowdisplaybreaks

% Fancy script capitals
\usepackage{mathrsfs}
\usepackage{cancel}
% Bold math
\usepackage{bm}
% Some shortcuts
\newcommand{\rr}{\ensuremath{\mathbb{R}}}
\newcommand{\zz}{\ensuremath{\mathbb{Z}}}
\newcommand{\qq}{\ensuremath{\mathbb{Q}}}
\newcommand{\nn}{\ensuremath{\mathbb{N}}}
\newcommand{\ff}{\ensuremath{\mathbb{F}}}
\newcommand{\cc}{\ensuremath{\mathbb{C}}}
\newcommand{\ee}{\ensuremath{\mathbb{E}}}
\newcommand{\hh}{\ensuremath{\mathbb{H}}}
\renewcommand\O{\ensuremath{\emptyset}}
\newcommand{\norm}[1]{{\left\lVert{#1}\right\rVert}}
\newcommand{\dbracket}[1]{{\left\llbracket{#1}\right\rrbracket}}
\newcommand{\ve}[1]{{\bm{#1}}}
\newcommand\allbold[1]{{\boldmath\textbf{#1}}}
\DeclareMathOperator{\lcm}{lcm}
\DeclareMathOperator{\im}{im}
\DeclareMathOperator{\coim}{coim}
\DeclareMathOperator{\dom}{dom}
\DeclareMathOperator{\tr}{tr}
\DeclareMathOperator{\rank}{rank}
\DeclareMathOperator*{\var}{Var}
\DeclareMathOperator*{\ev}{E}
\DeclareMathOperator{\dg}{deg}
\DeclareMathOperator{\aff}{aff}
\DeclareMathOperator{\conv}{conv}
\DeclareMathOperator{\inte}{int}
\DeclareMathOperator*{\argmin}{argmin}
\DeclareMathOperator*{\argmax}{argmax}
\DeclareMathOperator{\graph}{graph}
\DeclareMathOperator{\sgn}{sgn}
\DeclareMathOperator*{\Rep}{Rep}
\DeclareMathOperator{\Proj}{Proj}
\DeclareMathOperator{\mat}{mat}
\DeclareMathOperator{\diag}{diag}
\DeclareMathOperator{\aut}{Aut}
\DeclareMathOperator{\gal}{Gal}
\DeclareMathOperator{\inn}{Inn}
\DeclareMathOperator{\edm}{End}
\DeclareMathOperator{\Hom}{Hom}
\DeclareMathOperator{\ext}{Ext}
\DeclareMathOperator{\tor}{Tor}
\DeclareMathOperator{\Span}{Span}
\DeclareMathOperator{\Stab}{Stab}
\DeclareMathOperator{\cont}{cont}
\DeclareMathOperator{\Ann}{Ann}
\DeclareMathOperator{\Div}{div}
\DeclareMathOperator{\curl}{curl}
\DeclareMathOperator{\nat}{Nat}
\DeclareMathOperator{\gr}{Gr}
\DeclareMathOperator{\vect}{Vect}
\DeclareMathOperator{\id}{id}
\DeclareMathOperator{\Mod}{Mod}
\DeclareMathOperator{\sign}{sign}
\DeclareMathOperator{\Surf}{Surf}
\DeclareMathOperator{\fcone}{fcone}
\DeclareMathOperator{\Rot}{Rot}
\DeclareMathOperator{\grad}{grad}
\DeclareMathOperator{\atan2}{atan2}
\DeclareMathOperator{\Ric}{Ric}
\let\vec\relax
\DeclareMathOperator{\vec}{vec}
\let\Re\relax
\DeclareMathOperator{\Re}{Re}
\let\Im\relax
\DeclareMathOperator{\Im}{Im}
% Put x \to \infty below \lim
\let\svlim\lim\def\lim{\svlim\limits}

%wide hat
\usepackage{scalerel,stackengine}
\stackMath
\newcommand*\wh[1]{%
\savestack{\tmpbox}{\stretchto{%
  \scaleto{%
    \scalerel*[\widthof{\ensuremath{#1}}]{\kern-.6pt\bigwedge\kern-.6pt}%
    {\rule[-\textheight/2]{1ex}{\textheight}}%WIDTH-LIMITED BIG WEDGE
  }{\textheight}% 
}{0.5ex}}%
\stackon[1pt]{#1}{\tmpbox}%
}
\parskip 1ex

%Make implies and impliedby shorter
\let\implies\Rightarrow
\let\impliedby\Leftarrow
\let\iff\Leftrightarrow
\let\epsilon\varepsilon

% Add \contra symbol to denote contradiction
\usepackage{stmaryrd} % for \lightning
\newcommand\contra{\scalebox{1.5}{$\lightning$}}

% \let\phi\varphi

% Command for short corrections
% Usage: 1+1=\correct{3}{2}

\definecolor{correct}{HTML}{009900}
\newcommand\correct[2]{\ensuremath{\:}{\color{red}{#1}}\ensuremath{\to }{\color{correct}{#2}}\ensuremath{\:}}
\newcommand\green[1]{{\color{correct}{#1}}}

% horizontal rule
\newcommand\hr{
    \noindent\rule[0.5ex]{\linewidth}{0.5pt}
}

% hide parts
\newcommand\hide[1]{}

% si unitx
\usepackage{siunitx}
\sisetup{locale = FR}

%allows pmatrix to stretch
\makeatletter
\renewcommand*\env@matrix[1][\arraystretch]{%
  \edef\arraystretch{#1}%
  \hskip -\arraycolsep
  \let\@ifnextchar\new@ifnextchar
  \array{*\c@MaxMatrixCols c}}
\makeatother

\renewcommand{\arraystretch}{0.8}

\renewcommand{\baselinestretch}{1.5}

\usepackage{graphics}
\usepackage{epstopdf}

\RequirePackage{hyperref}
%%
%% Add support for color in order to color the hyperlinks.
%% 
\hypersetup{
  colorlinks = true,
  urlcolor = blue,
  citecolor = blue
}
%%fakesection Links
\hypersetup{
    colorlinks,
    linkcolor={red!50!black},
    citecolor={green!50!black},
    urlcolor={blue!80!black}
}
%customization of cleveref
\RequirePackage[capitalize,nameinlink]{cleveref}[0.19]

% Per SIAM Style Manual, "section" should be lowercase
\crefname{section}{section}{sections}
\crefname{subsection}{subsection}{subsections}
\Crefname{section}{Section}{Sections}
\Crefname{subsection}{Subsection}{Subsections}

% Per SIAM Style Manual, "Figure" should be spelled out in references
\Crefname{figure}{Figure}{Figures}

% Per SIAM Style Manual, don't say equation in front on an equation.
\crefformat{equation}{\textup{#2(#1)#3}}
\crefrangeformat{equation}{\textup{#3(#1)#4--#5(#2)#6}}
\crefmultiformat{equation}{\textup{#2(#1)#3}}{ and \textup{#2(#1)#3}}
{, \textup{#2(#1)#3}}{, and \textup{#2(#1)#3}}
\crefrangemultiformat{equation}{\textup{#3(#1)#4--#5(#2)#6}}%
{ and \textup{#3(#1)#4--#5(#2)#6}}{, \textup{#3(#1)#4--#5(#2)#6}}{, and \textup{#3(#1)#4--#5(#2)#6}}

% But spell it out at the beginning of a sentence.
\Crefformat{equation}{#2Equation~\textup{(#1)}#3}
\Crefrangeformat{equation}{Equations~\textup{#3(#1)#4--#5(#2)#6}}
\Crefmultiformat{equation}{Equations~\textup{#2(#1)#3}}{ and \textup{#2(#1)#3}}
{, \textup{#2(#1)#3}}{, and \textup{#2(#1)#3}}
\Crefrangemultiformat{equation}{Equations~\textup{#3(#1)#4--#5(#2)#6}}%
{ and \textup{#3(#1)#4--#5(#2)#6}}{, \textup{#3(#1)#4--#5(#2)#6}}{, and \textup{#3(#1)#4--#5(#2)#6}}

% Make number non-italic in any environment.
\crefdefaultlabelformat{#2\textup{#1}#3}

% Environments
\makeatother
% For box around Definition, Theorem, \ldots
%%fakesection Theorems
\usepackage{thmtools}
\usepackage[framemethod=TikZ]{mdframed}

\theoremstyle{definition}
\mdfdefinestyle{mdbluebox}{%
	roundcorner = 10pt,
	linewidth=1pt,
	skipabove=12pt,
	innerbottommargin=9pt,
	skipbelow=2pt,
	nobreak=true,
	linecolor=blue,
	backgroundcolor=TealBlue!5,
}
\declaretheoremstyle[
	headfont=\sffamily\bfseries\color{MidnightBlue},
	mdframed={style=mdbluebox},
	headpunct={\\[3pt]},
	postheadspace={0pt}
]{thmbluebox}

\mdfdefinestyle{mdredbox}{%
	linewidth=0.5pt,
	skipabove=12pt,
	frametitleaboveskip=5pt,
	frametitlebelowskip=0pt,
	skipbelow=2pt,
	frametitlefont=\bfseries,
	innertopmargin=4pt,
	innerbottommargin=8pt,
	nobreak=false,
	linecolor=RawSienna,
	backgroundcolor=Salmon!5,
}
\declaretheoremstyle[
	headfont=\bfseries\color{RawSienna},
	mdframed={style=mdredbox},
	headpunct={\\[3pt]},
	postheadspace={0pt},
]{thmredbox}

\declaretheorem[%
style=thmbluebox,name=Theorem,numberwithin=section]{thm}
\declaretheorem[style=thmbluebox,name=Lemma,sibling=thm]{lem}
\declaretheorem[style=thmbluebox,name=Proposition,sibling=thm]{prop}
\declaretheorem[style=thmbluebox,name=Corollary,sibling=thm]{coro}
\declaretheorem[style=thmredbox,name=Example,sibling=thm]{eg}

\mdfdefinestyle{mdgreenbox}{%
	roundcorner = 10pt,
	linewidth=1pt,
	skipabove=12pt,
	innerbottommargin=9pt,
	skipbelow=2pt,
	nobreak=true,
	linecolor=ForestGreen,
	backgroundcolor=ForestGreen!5,
}

\declaretheoremstyle[
	headfont=\bfseries\sffamily\color{ForestGreen!70!black},
	bodyfont=\normalfont,
	spaceabove=2pt,
	spacebelow=1pt,
	mdframed={style=mdgreenbox},
	headpunct={ --- },
]{thmgreenbox}

\declaretheorem[style=thmgreenbox,name=Definition,sibling=thm]{defn}

\mdfdefinestyle{mdgreenboxsq}{%
	linewidth=1pt,
	skipabove=12pt,
	innerbottommargin=9pt,
	skipbelow=2pt,
	nobreak=true,
	linecolor=ForestGreen,
	backgroundcolor=ForestGreen!5,
}
\declaretheoremstyle[
	headfont=\bfseries\sffamily\color{ForestGreen!70!black},
	bodyfont=\normalfont,
	spaceabove=2pt,
	spacebelow=1pt,
	mdframed={style=mdgreenboxsq},
	headpunct={},
]{thmgreenboxsq}
\declaretheoremstyle[
	headfont=\bfseries\sffamily\color{ForestGreen!70!black},
	bodyfont=\normalfont,
	spaceabove=2pt,
	spacebelow=1pt,
	mdframed={style=mdgreenboxsq},
	headpunct={},
]{thmgreenboxsq*}

\mdfdefinestyle{mdblackbox}{%
	skipabove=8pt,
	linewidth=3pt,
	rightline=false,
	leftline=true,
	topline=false,
	bottomline=false,
	linecolor=black,
	backgroundcolor=RedViolet!5!gray!5,
}
\declaretheoremstyle[
	headfont=\bfseries,
	bodyfont=\normalfont\small,
	spaceabove=0pt,
	spacebelow=0pt,
	mdframed={style=mdblackbox}
]{thmblackbox}

\theoremstyle{plain}
\declaretheorem[name=Question,sibling=thm,style=thmblackbox]{ques}
\declaretheorem[name=Remark,sibling=thm,style=thmgreenboxsq]{remark}
\declaretheorem[name=Remark,sibling=thm,style=thmgreenboxsq*]{remark*}
\newtheorem{ass}[thm]{Assumptions}

\theoremstyle{definition}
\newtheorem*{problem}{Problem}
\newtheorem{claim}[thm]{Claim}
\theoremstyle{remark}
\newtheorem*{case}{Case}
\newtheorem*{notation}{Notation}
\newtheorem*{note}{Note}
\newtheorem*{motivation}{Motivation}
\newtheorem*{intuition}{Intuition}
\newtheorem*{conjecture}{Conjecture}

% Make section starts with 1 for report type
%\renewcommand\thesection{\arabic{section}}

% End example and intermezzo environments with a small diamond (just like proof
% environments end with a small square)
\usepackage{etoolbox}
\AtEndEnvironment{vb}{\null\hfill$\diamond$}%
\AtEndEnvironment{intermezzo}{\null\hfill$\diamond$}%
% \AtEndEnvironment{opmerking}{\null\hfill$\diamond$}%

% Fix some spacing
% http://tex.stackexchange.com/questions/22119/how-can-i-change-the-spacing-before-theorems-with-amsthm
\makeatletter
\def\thm@space@setup{%
  \thm@preskip=\parskip \thm@postskip=0pt
}

% Fix some stuff
% %http://tex.stackexchange.com/questions/76273/multiple-pdfs-with-page-group-included-in-a-single-page-warning
\pdfsuppresswarningpagegroup=1


% My name
\author{Jaden Wang}



\begin{document}
\centerline {\textsf{\textbf{\LARGE{Homework 4}}}}
\centerline {Jaden Wang}
\vspace{.15in}
\begin{problem}[LN12 0.2.2]
Show that the stereograpphic projection $ \pi: S^2 \setminus \{(0,0,1)\} \to \rr^2$ is a conformal transformation.
\end{problem}
To show that $ \pi$ preserves angle, we use the Euclidean metric. Recall that by treating $ S^2$ as the unit sphere with north pole $ N=(0,0,1)$, the stereographic projection is $ \pi: (\overline{p},p_3) \mapsto \frac{\overline{p}}{ 1- p_3}$, where $ \overline{p} = (p_1,p_2)$. Notice for any $ p \in S^2$, we have $ \norm{ \overline{p}} ^2 = p_1^2 + p_2^2 = 1- p_3^2 $; for any $ v \in T_p S^2$, we have $ \langle v ,p \rangle = 0$ so $ \langle \overline{v} , \overline{p} \rangle = - v_3p_3$. Let $ \gamma$ be a curve s.t.\ $ \gamma(0) = p$ and $ \gamma'(0) = v$. Then we have
\begin{align*}
	d\pi_p (v) &= (\pi \circ \gamma)'(0)\\
	&= \left( \frac{ \overline{ \gamma}}{ 1- \gamma_3} \right)'(0)  \\
	&= \left( \frac{ \overline{ \gamma}' (1- \gamma_3) - \overline{ \gamma} (- \gamma_3')}{(1- \gamma_3)^2 } \right)(0)  \\
	&= \frac{\overline{v}(1-p_3) + v_3 \overline{p}}{ (1-p_3)^2} .
\end{align*}
Then we compute
\begin{align*}
	g_p(d\pi_p(v),d\pi_p(w)) &= \langle d\pi_p(v),d\pi_p(w) \rangle\\
	&=\frac{\langle \overline{v},\overline{w} \rangle}{ (1-p_3)^2} + \frac{v_3 \langle \overline{w},\overline{p} \rangle + w_3 \langle \overline{v},\overline{p} \rangle}{ (1-p_3)^3} + \frac{\norm{ \overline{p}}^2 v_3 w_3 }{ (1-p_3)^{4}}\\
	&= \frac{\langle \overline{v},\overline{w} \rangle}{ (1-p_3)^2} - \frac{2v_3 w_3 p_3 }{ (1-p_3)^3} + \frac{\norm{ \overline{p}}^2 v_3 w_3 }{ (1-p_3)^{4}}\\
	&= \frac{\langle \overline{v},\overline{w} \rangle}{ (1-p_3)^2} + \frac{v_3 w_3( 2p_3^2 - 2 p_3 + \norm{ \overline{p}}^2) }{ (1-p_3)^{4}}\\
	&= \frac{\langle \overline{v},\overline{w} \rangle}{ (1-p_3)^2} + \frac{v_3 w_3( p_3^2 - 2 p_3 + 1 ) }{ (1-p_3)^{4}}\\
	&= \frac{\langle \overline{v},\overline{w} \rangle}{ (1-p_3)^2} + \frac{v_3 w_3( p_3 - 1 )^2 }{ (1-p_3)^{4}}\\
	&= \frac{ v_1 w_1 + v_2 w_2}{ (1-p_3)^2} + \frac{v_3 w_3}{ (1-p_3)^{2}}\\
	&= \frac{ \langle v,w \rangle}{ (1-p_3)^2} \\
	&= \frac{g_p(v,w)}{ (1- p_3)^2} .
\end{align*}
It follows that
\begin{align*}
	\theta (d\pi_p(v),d\pi_p(w) ) &= \arccos \left( \frac{g_p(d\pi_p(v),d\pi_p(w))}{ g_p(d\pi_p(v),d\pi_p(v))^{\frac{1}{2}} g_p(d\pi_p(w),d\pi_p(w))^{\frac{1}{2}} } \right)  \\
	 &= \arccos \left( \frac{g_p(v,w)}{ g_p(v,v)^{\frac{1}{2}} g_p(w,w)^{\frac{1}{2}} } \right)  \\
	 &= \theta(v,w) .
\end{align*}

\begin{problem}[LN12 0.3.2]
Compute the metric of the surface given by the graph of a function $ f: \Omega \subseteq \rr^2 \to \rr$.
\end{problem}
Let $ F(x,y) = (x,y,f(x,y))$ be the graph of  $ f$. We compute
\begin{align*}
	\frac{\partial F}{\partial x} &= \begin{pmatrix} 1\\0\\ \frac{\partial f}{\partial x}  \end{pmatrix} \\
	\frac{\partial F}{\partial y} &= \begin{pmatrix} 0\\1\\ \frac{\partial f}{\partial y}  \end{pmatrix}  .
\end{align*}

Then then pullback metric is
\begin{align*}
	G = \begin{pmatrix} 1+ \left( \frac{\partial f}{\partial x}  \right) ^2 & \frac{\partial f}{\partial x} \frac{\partial f}{\partial y} \\ \frac{\partial f}{\partial x} \frac{\partial f}{\partial y}&  1+ \left( \frac{\partial f}{\partial y}  \right) ^2   \end{pmatrix} .
\end{align*}
\begin{problem}[LN13 0.2]
Compute the area of a torus of revolution in $ \rr^3$.
\end{problem}
Let $ R$ be the radius of the longitudinal circle and  $ r$ be that of the meridian circle. Any point on the initial meridian can be parameterized by $ (x(\phi),z(\phi)) := (R+r \cos \phi, r \sin\phi)$, where $ \phi \in (0,2\pi)$. Then we can parameterize the torus using 
\begin{align*}
	f(\theta,\phi) = \begin{pmatrix}  \cos \theta x(\phi)\\ \sin \theta x(\phi) \\z(\phi)  \end{pmatrix}  ,
\end{align*}
where $ \theta \in (0,2\pi)$. We compute
\begin{align*}
	\frac{\partial f}{\partial \theta} &= \begin{pmatrix} - \sin \theta x \\ \cos \theta x \\ 0 \end{pmatrix}  \\
	\frac{\partial f}{\partial \phi} &= \begin{pmatrix} -r\cos\theta \sin \phi \\ -r \sin\theta \sin\phi \\ r \cos\phi  \end{pmatrix}  .
\end{align*}
Thus the pullback metric is
\begin{align*}
	G(\theta,\phi) = \begin{pmatrix} x(\phi)^2 & 0\\0& r^2 \end{pmatrix} 
\end{align*}
Then
\begin{align*}
	A &= \int_U \sqrt{g} \\
	&= \int_{ 0}^{ 2\pi} \int_{ 0}^{ 2\pi} \sqrt{ \det G} \  d\phi d\theta    \\
	&= \int_{ 0}^{ 2\pi} \int_{ 0}^{ 2\pi} xr\  d\phi d\theta    \\
	&= \int_{ 0}^{ 2\pi} \int_{ 0}^{ 2\pi} (R+r\cos\phi)r\  d\phi d\theta    \\
	&= \int_{ 0}^{ 2\pi} (Rr \phi - r^2 \sin\phi)|_0^{2\pi} d\theta    \\
	&= \int_{ 0}^{ 2\pi} 2\pi Rr\ d\theta    \\
	&= 4\pi^2 Rr .
\end{align*}
\begin{problem}[LN13 0.9]
Show that every manifold is normal, \emph{i.e.} for every disjoint closed sets $ A_1, A_2$ in $ M$, there exists a pair of disjoint open subsets  $ U_1,U_2$ of $ M$  s.t.\ $ A_1 \subset U_1, A_2 \subset U_2$.
\end{problem}
\begin{proof}
	WLOG suppose $ M$ is path-connected since we can just take the union of disjoint open sets from all path components. Since every manifold admits a metric, let $ g$ be a Riemannian metric of  $ M$.  Therefore, for every point $ x \in A_1, y \in A_2$, let $ \Gamma$ be the set of all possible paths between $ x$ and  $ y$, we can define the distance between them $ d(x,y) = \inf_{ \gamma \in \Gamma}L[ \gamma]  $, where $ L[\gamma]$ is computed the usual way using $ g$. This distance can be checked to be a metric and therefore endows $ M$ with a metric space structure. Every metric space is normal. We prove this below.

	Define the distance between a point $ x$ and a set $ A$ to be  $ d(x,A) = \inf\left\{ d(x,y): y \in A \right\} $. Since $ A_1, A_2$ are closed so they contain all their limit points, for any $ y \in M\setminus A_1$ and $ x \in M \setminus A_2$, we must have $ d(y,A_1) = \epsilon_y > 0 $ and $ d(x, A_2) = \epsilon_x > 0$. Let $ r_x = \frac{ \epsilon_x}{ 3}$ and $ r_y = \frac{ \epsilon_y}{ 3}$. Then take
	\begin{align*}
		U_1 &:= \bigcup_{ x \in A_1} B_{ r_x}(x) \\
		U_2 &:= \bigcup_{ y \in A_2} B_{ r_y}(y) ,
	\end{align*}
which are unions of open balls and thus open. Now suppose there exists a $ p \in U_1 \cap U_2$. Then by definition $ p \in B_{r_x}(x) \cap B_{r_y}(y)$ for some $ x \in A_1$ and $ y \in A_2$. It follows that
\begin{align*}
	\frac{ \epsilon_x + \epsilon_y}{ 2} &\leq d(x,y) && \text{definition of } \epsilon_x, \epsilon_y  \\
					    &\leq d(x,p) + d(p,y) && \Delta \text{ inequality}  \\
	&\leq r_x + r_y  \\
	&= \frac{ \epsilon_x + \epsilon_y}{ 3} ,
\end{align*}
a contradiction since $ \epsilon_x + \epsilon_y >0$. Hence $ U_1,U_2$ are disjoint and are the open sets we seek.
\end{proof}

\begin{problem}[LN13 0.12]
Let $ M \subseteq \rr^{n}$ be an embedded submanifold which may be parameteried by $ f: U \to \rr^{n}$, for some open set $ U \subseteq \rr^{m}$, \emph{i.e.} , $ f$ is a 1-to-1 smooth immersion and  $ f(U) =M$. Show that then  $ vol(M) = \int_U \sqrt{ \det (J_x(f)^{T} J_x(f))} dx $, where $ J_x(f)$ is the Jacobian matrix of  $ f $ at  $ x$. (Note that since we define Jacobian differently than the lecture notes, the order is flipped.)
\end{problem}
\begin{proof}
	Since $ f$ is a 1-to-1 immersion onto $ M$,  $ f$ is a diffeomorphism onto its image and thus a parameterization of  $ M$. Since $ M$ is an embedded submanifold of  $ \rr^{n}$, it is endowed with the ambient Euclidean metric. So we can endow the pullback metric on $ U$ which is just  $ {g_{ij}}_x = \langle \frac{\partial f}{\partial x_i}(x) , \frac{\partial f}{\partial x_j}(x)  \rangle$. Recall that the $ i$th column of the Jacobian $ J_x(f)_{i} = \frac{\partial f}{\partial x_i}$. Therefore, the $ ij$th entry of  $ J_x(f)^{T} J_x(f)$ is precisely $\langle \frac{\partial f}{\partial x_i}(x) , \frac{\partial f}{\partial x_j}(x)  \rangle = {g_{ij}}_x$. Then we have
\begin{align*}
	vol(M) &= \int_U \sqrt{ \det G_x} dx \\
	&= \int_U \sqrt{ \det (J_x(f)^{T} J_x(f))} dx .
\end{align*}
\end{proof}
\end{document}
