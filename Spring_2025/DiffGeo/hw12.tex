\documentclass[12pt]{article}
\newcommand{\alert}[1]{{\bf \color{red} [Alert:] #1}}
\newcommand{\todo}[1]{{\bf \color{orange} [TODO:] #1}}
\newcommand{\real}[1][]{\mathbb{R}^{#1}}
\newcommand{\myeqn}[1]{(\ref{#1})}
\newcommand{\myex}[1]{Example \ref{#1}}
\newcommand{\defeq}{\stackrel{\mathrm{def}}{=}}
\newcommand{\parder}[2]{\frac{\partial #1}{\partial #2}}
\newcommand{\Lie}[3][]{\mathsf{L}_{#3}^{#1} #2}
\newcommand{\LieA}[1]{\mathsf{Lie}(#1)}
\newcommand{\lieder}[2]{\mathcal{L}_{#2} #1}
\renewcommand{\t}{^{\mbox{\tiny\sf T}}}
\newcommand{\trans}{^{\mbox{\tiny\sf T}}}
\newcommand{\markup}[1]{\{\textbf{#1}\}}
\newcommand{\msub}[1]{_\mathrm{#1}}
\newcommand{\msup}[1]{^\mathrm{#1}}
\newcommand{\inv}[1]{#1^{-1}}
\newcommand{\pinv}[1]{{#1}^{+}}
\newcommand{\myfracA}[2]{\displaystyle{\frac{#1}{#2}}}
\newcommand{\myfracB}[2]{{#1}/{#2}}
\newcommand{\mydiffA}[1]{\dot{#1}}
\newcommand{\mydiffB}[2]{\myfracA{\mathrm{d}{#1}}{\mathrm{d}{#2}}}
\newcommand{\ball}[2]{\mathcal{B}_{#1}\left(#2\right)}
\newcommand{\acos}[1]{\cos^{-1}\left(#1\right)}
\newcommand{\asin}[1]{\sin^{-1}\left(#1\right)}
\newcommand{\mani}{\mathcal{M}}
\newcommand{\tang}[2]{\mathsf{T}_{#1} #2}
\newcommand{\LieB}[2]{[ #1, #2 ]}
\newcommand{\LieBad}[3][]{\mathsf{ad}_{#2}^{#1} #3}
\newcommand{\ReachVT}{\mathcal{R}^V_T}
\newcommand{\ReachVt}{\mathcal{R}^V_t}
\newcommand{\ReachVTe}{\mathcal{R}^V_{\le T}}
\newcommand{\ReachT}{\mathcal{R}_T}
\newcommand{\Reacht}{\mathcal{R}_t}
\newcommand{\ReachTe}{\mathcal{R}_{\le T}}
\newcommand{\accLA}[1]{\mathsf{Lie}(#1)}
\newcommand{\accD}{\Delta_{\mathcal{F}}}
\newcommand{\accSA}{\mathsf{Lie}(\mathcal{G},f)}
\newcommand{\accDS}{\Delta_{\mathcal{G}}}
\newcommand{\eval}[3]{\mathsf{Ev}^{#2}_{#1}\left( #3 \right)}
\newcommand{\stlc}{\textsc{stlc}}
\newcommand{\clf}{\textsc{clf}}
\newcommand{\jqlf}{\textsc{jqlf}}
\newcommand{\dlas}{\textsc{dlas}}
\newcommand{\Ad}[2]{\mathsf{Ad}_{#1} #2}
\newcommand{\xe}{\ensuremath{x_e}}
\newcommand{\lebg}[1]{\mathcal{L}_{#1}}
\newcommand{\lebgx}[1]{\mathcal{L}_{#1 \mathrm{e}}}
\newcommand{\dom}{D}
\newcommand{\domT}{[t_0,\infty) \times D}
\newcommand{\rarrow}{\rightarrow}
\renewcommand{\d}{\mathrm{d}}
\renewcommand{\Re}{\mathbb{R}}
\newcommand{\C}{\mathrm{C}}

\newcommand{\QED}{{\unskip\nobreak\hfil\penalty50\hskip2em\vadjust{}
		\nobreak\hfil$\Box$\parfillskip=0pt\finalhyphendemerits=0\par}\vspace{0.1cm}}
\newcommand{\eoEx}{{\unskip\nobreak\hfil\penalty50\hskip0em\vadjust{}
		\nobreak\hfil$\Large\Diamond$\parfillskip=0pt\finalhyphendemerits=0\par}\vspace{0.1cm}}

\newcommand{\sgn}{\ensuremath{\operatorname{sgn}}}
\newcommand{\sat}{\ensuremath{\operatorname{sat}}}

\newcommand{\half}{\frac{1}{2}}
\newcommand{\shalf}{\mbox{$\frac{1}{2}$}}
\newcommand{\marcom}[1]{\marginpar{\footnotesize #1}}
\newcommand{\der}{\mathrm{D}}
\newcommand{\e}{\mathrm{e}}
\newcommand{\dt}{\mathrm{d}t}

\newcommand{\cA}{\ensuremath{\mathcal{A}}}
\newcommand{\cB}{\ensuremath{\mathcal{B}}}
\newcommand{\cG}{\ensuremath{\mathcal{G}}}
\newcommand{\cK}{\ensuremath{\mathcal{K}}}
\newcommand{\cW}{\ensuremath{\mathcal{W}}}
\newcommand{\cZ}{\ensuremath{\mathcal{Z}}}
\newcommand{\cS}{\ensuremath{\mathcal{S}}}
\newcommand{\cD}{\ensuremath{\mathcal{D}}}
\newcommand{\cP}{\ensuremath{\mathcal{P}}}
\newcommand{\cV}{\ensuremath{\mathcal{V}}}
\newcommand{\cL}{\ensuremath{\mathcal{L}}}
\newcommand{\cN}{\ensuremath{\mathcal{N}}}
\newcommand{\cI}{\ensuremath{\mathcal{I}}}
\newcommand{\cR}{\ensuremath{\mathcal{R}}}
\newcommand{\cM}{\ensuremath{\mathcal{M}}}
\newcommand{\cC}{\ensuremath{\mathcal{C}}}
\newcommand{\cF}{\ensuremath{\mathcal{F}}}
\newcommand{\cH}{\ensuremath{\mathcal{H}}}
\newcommand{\cO}{\ensuremath{\mathcal{O}}}
\newcommand{\cX}{\ensuremath{\mathcal{X}}}
\newcommand{\cY}{\ensuremath{\mathcal{Y}}}
\newcommand{\Ci}{\ensuremath{\mathcal{C}^\infty}}
\newcommand{\ISS}{\textsc{iss}}
\newcommand{\LISS}{\textsc{liss}}
\newcommand{\GAS}{\textsc{gas}}
\newcommand{\GS}{\textsc{gs}}
\newcommand{\LES}{\textsc{les}}
\newcommand{\GUAS}{\textsc{guas}}
\newcommand{\BIBO}{\textsc{bibo}}
\newcommand{\spec}{\ensuremath{\operatorname{spec}}}
\newcommand{\spn}{\ensuremath{\operatorname{span}}}
\renewcommand{\i}{\mathrm{i\,}}

\renewcommand{\implies}{\Rightarrow}

\renewcommand{\theenumi}{$\roman{enumi})$}
\renewcommand{\labelenumi}{\theenumi}

\font\ptmten=zptmcmrm scaled 1200
\newcommand{\w}{\mbox{{\ptmten w}}}
\newcommand{\z}{\mbox{{\ptmten z}}}
\renewcommand{\Re}{\mathbb{R}}

\newcommand{\cl}{\operatorname{cl}}
\newcommand{\intr}{\operatorname{int}}
\newcommand{\rank}{\operatorname{rank}}
\newcommand{\co}{\operatorname{co}}
\newcommand{\aff}{\operatorname{aff}}

\theoremstyle{plain}
\newtheorem{theorem}{Theorem}[chapter]
\newtheorem{claim}[theorem]{Claim}
\newtheorem{corollary}[theorem]{Corollary}
\newtheorem{prop}[theorem]{Proposition}
\newtheorem{fact}[theorem]{Fact}
\newtheorem{lemma}[theorem]{Lemma}

\newtheorem{remark}{Remark}[chapter]

\theoremstyle{definition}
\newtheorem{assume}[theorem]{Assumption}
\newtheorem{defn}[theorem]{Definition}
\newtheorem{problem}[theorem]{Problem}
\newtheorem{exercise}{Exercise}
\newtheorem{example}[theorem]{Example}


\begin{document}
\centerline {\textsf{\textbf{\LARGE{Homework 12}}}}
\centerline {Jaden Wang}
\vspace{.15in}

\begin{problem}[7.1]
If $ M,N$ are Riemannian manifolds such that the inclusion  $ i: M \subset N$ is an isometric immersion, show by an example that the strict inequality $ d_M(x,y) > d_N(x,y)$ can occur.
\end{problem}
\begin{proof}
Let $ N= \rr^2$ with the Euclidean metric and $ M= \rr^2 \setminus D^2$ where $ D^2$ is the closed unit disk centered at the origin. Then the inclusion is clearly an isometric immersion. Notice  $ d_M((-1.1,0),(1.1,0)) > \pi$ since any path has to go around half of the unit disk. But $ d_N((-1.1,0),(1.1,0)) = 2.2 < \pi$. 
\end{proof}
\begin{problem}[7.2]
Let $ \widetilde{ M}$ be a covering space of a Riemannian manifold $ M$. Show that it is possible to give  $ \widetilde{ M}$ a Riemannian structure such that the covering map $ \pi: \widetilde{ M} \to M$ is a local isometry. Show that $ \widetilde{ M}$ is complete in the covering metric if and only if $ M$ is complete.
\end{problem}
\begin{proof}
	Let $ g$ be the metric on  $ M$. Define the pullback metric $ \pi^{*}g_{\widetilde{ p}} = g_{\pi(p)}$ for any $ \widetilde{ p} \in \widetilde{ M}$. Since $ \pi$ is already a local diffeomorphism, by definition of pullback metric it is a local isometry. Recall that in a covering space, any path downstairs can be lifted to a unique path upstairs once we choose an initial point, and any path upstairs can be projected to a unique path downstairs. Since geodesics are defined using local notions (covariant derivative), and since the covering map is a local isometry, any geodesic downstairs can be lifted to a unique geodesic upstairs with chosen initial point and vice versa. Thus if $ M$ is complete, for any point $ \widetilde{ p} \in \widetilde{ M}$ we can take any geodesic $ \widetilde{ \gamma}$ emanating from $ \widetilde{ p}$, project it to a geodesic in $ M$ and extend it for all time and then lift it back to the same starting point $ \widetilde{ p}$, which by uniqueness guarantees to coincide with the original geodesic $ \widetilde{ \gamma}$ at original time interval. The reverse direction is pretty much identical. By Hopf--Rinow, geodesic completeness and completeness are equivalent so we are done with the proof.
\end{proof}
\begin{problem}[7.9]
Consider the upper half-plane with the metric  $ g_{11} = 1, g_{12}=0, g_{22} = \frac{1}{y}$. Show that the length of the vertical segment $ x=0$,  $ \epsilon\leq y \leq 1$ with $ \epsilon>0$ tends to 2 as $ \epsilon \to 0$. Conclude from this that such a metric is not complete.
\end{problem}
\begin{proof}
Let $ \gamma(t) = (0,t)$ so $ \gamma'(t) = (0,1)$. Then the desired length is
\begin{align*}
	\int_{ \epsilon}^{ 1} \sqrt{g_{(0,t)} ((0,1),(0,1))}\ dt  &= \int_{ \epsilon}^{ 1} t^{-\frac{1}{2}} \ dt  \\
	&= 2 t^{\frac{1}{2}} \big|_{ \epsilon}^{1} \\
	&= 2- 2 \sqrt{ \epsilon}  .
\end{align*}
Clearly as $ \epsilon \to 0$, the length tends to 2. Now we wish to prove the negation of statement (e) in Hopf--Rinow: for all sequences of compact subsets $ K_n \subset M, K_n \subset K_{n+1} $ and $ \bigcup_{ n} K_n =M $, there exists a $ q_n \not\in K_n$ such that $ \lim_{ n \to \infty}  d(p,q_n) < \infty $.

Given any such sequence $ (K_n) $, and WLOG assume the intersection between $ K_n$ and the $ y$-axis is nonempty since eventually  $ K_n$ has to tend to the entire $ \rr_+^2$ and we only care about tail behavior.  Since $ K_n$ is compact and the projection function is continuous, there exists a point $ (0,y_n) \in K_n $ with a minimum  $ y$-value  $ y_n > 0$ in $ K_n$. Define  $ q_n = \left(0, \frac{y_n}{ 2}\right)$. By minimality of $ y_n$, we have $ q_n \not\in K_n$ and $ q_n \to (0,0)$. However, $ d((0,1),q_n) \to 2 < \infty$, proving the negation. 
\end{proof}
\begin{problem}[7.10]
Prove that the upper half-pane $ \rr_+^2$ with the Lobatchevski metric $ g_{11}= g_{22} = \frac{1}{y^2}$, $ g_{12} = 0 $ is complete.
\end{problem}
\begin{proof}
From Example 3.10 of Chapter 3, we know that its geodesics are vertical rays from the $ x$-axis and semicircles with center on the $ x$-axis. Given any $ p = (x,y) \in \rr_+^2$ and a semicircle or ray emanating from $ p$, clearly it can always be extended upward as there is no boundary or holes when  $ y$-value increases. But as  $ y$-value decreases, we just need to check that it doesn't reach the boundary $ x$-axis in finite time. In the case of a ray, as before we can integrate its length which is $ \ln y -\ln \epsilon$. As $ \epsilon \to 0$, the length goes to infinity. In the case of a semicircle, we can parameterize it as $ \gamma(t) = (y\cos t + x, y\sin t)$ with radius $ y$ and center at $ (x,0)$. 
\begin{align*}
	L( \gamma) &= \int_{ \epsilon}^{ \frac{\pi}{2}} \frac{1}{y \sin t}  \norm{ (-y \sin t, y \cos t)} dt\\  
	&= \int_{ \epsilon}^{ \frac{\pi}{2}} \frac{dt}{\sin t}  \\
	&= \ln \tan \frac{t}{2} \big|_{ \epsilon}^{\frac{\pi}{2}} \\
	&= \ln \tan \frac{\pi}{4} - \ln \epsilon .
\end{align*}
Again the arc length approaches $ \infty$ as $ \epsilon \to 0$. Since the geodesic moves at constant speed, it cannot cover arbitrarily large arc length in finite time. That it, it can be extended to all time. By Hopf--Rinow, the Lobatchevsky plane is complete.
\end{proof}
\begin{problem}[7.12]
A Riemannian manifold is \emph{homogeneous} if given $ p,q \in M$ there exists an isometry of $ M$ which takes  $ p$ into  $ q$. Prove that any homogeneous manifold is complete. 
\end{problem}
\begin{proof}
Let $ B \subset T_pM$ be the largest closed ball with radius $ r< \infty$ centered at 0 where $ \exp_p$ is still defined. Take any $ v \in \partial B$ so it has length $ r$. Define  $ q= \exp_p( v)$ which is also $ \gamma(1)$ where $ \gamma$ is the geodesic emanating from $ p$ with  $ \dot{ \gamma} (0) = v$. Let $ w = \dot{ \gamma}(1)$ which also has length $ r$ since geodesic has constant speed (parallel translation is an isometry). Let  $ f$ be the isometry that takes  $ p$ to  $ q$ and define  $ u = df^{-1}_q(w)$ which again has length  $ r$ and therefore  $ u \in B$. Let $ c(t)$ be the geodesic emanating from  $ p$ with  $ \dot{c}(0) = u$ and we have $ c(1) = \exp_p( u)$. Since isometry maps geodesics to geodesics, we see that $ f \circ c$ becomes a geodesic emanating from  $ q$ with initial velocity $ w$. By local uniqueness of geodesic, we can concatenate $ \gamma(t)$ and $ f \circ c(t)$ at $ q$ into a single geodesic. But notice that $ \exp_p( 2v) = f \circ c(1) $. Since $ v$ is  on the boundary of $ B$ and is arbitrary, it is a contradiction that $ 2v \in B$. Thus, we must have $ r = \infty $. Thus $ \exp_p$ is defined for all of $ T_pM$ and therefore by Hopf-Rinow, the manifold is complete.
\end{proof}
\end{document}
