\documentclass[12pt]{article}
%Fall 2022
% Some basic packages
\usepackage{standalone}[subpreambles=true]
\usepackage[utf8]{inputenc}
\usepackage[T1]{fontenc}
\usepackage{textcomp}
\usepackage[english]{babel}
\usepackage{url}
\usepackage{graphicx}
%\usepackage{quiver}
\usepackage{float}
\usepackage{enumitem}
\usepackage{lmodern}
\usepackage{comment}
\usepackage{hyperref}
\usepackage[usenames,svgnames,dvipsnames]{xcolor}
\usepackage[margin=1in]{geometry}
\usepackage{pdfpages}

\pdfminorversion=7

% Don't indent paragraphs, leave some space between them
\usepackage{parskip}

% Hide page number when page is empty
\usepackage{emptypage}
\usepackage{subcaption}
\usepackage{multicol}
\usepackage[b]{esvect}

% Math stuff
\usepackage{amsmath, amsfonts, mathtools, amsthm, amssymb}
\usepackage{bbm}
\usepackage{stmaryrd}
\allowdisplaybreaks

% Fancy script capitals
\usepackage{mathrsfs}
\usepackage{cancel}
% Bold math
\usepackage{bm}
% Some shortcuts
\newcommand{\rr}{\ensuremath{\mathbb{R}}}
\newcommand{\zz}{\ensuremath{\mathbb{Z}}}
\newcommand{\qq}{\ensuremath{\mathbb{Q}}}
\newcommand{\nn}{\ensuremath{\mathbb{N}}}
\newcommand{\ff}{\ensuremath{\mathbb{F}}}
\newcommand{\cc}{\ensuremath{\mathbb{C}}}
\newcommand{\ee}{\ensuremath{\mathbb{E}}}
\newcommand{\hh}{\ensuremath{\mathbb{H}}}
\renewcommand\O{\ensuremath{\emptyset}}
\newcommand{\norm}[1]{{\left\lVert{#1}\right\rVert}}
\newcommand{\dbracket}[1]{{\left\llbracket{#1}\right\rrbracket}}
\newcommand{\ve}[1]{{\bm{#1}}}
\newcommand\allbold[1]{{\boldmath\textbf{#1}}}
\DeclareMathOperator{\lcm}{lcm}
\DeclareMathOperator{\im}{im}
\DeclareMathOperator{\coim}{coim}
\DeclareMathOperator{\dom}{dom}
\DeclareMathOperator{\tr}{tr}
\DeclareMathOperator{\rank}{rank}
\DeclareMathOperator*{\var}{Var}
\DeclareMathOperator*{\ev}{E}
\DeclareMathOperator{\dg}{deg}
\DeclareMathOperator{\aff}{aff}
\DeclareMathOperator{\conv}{conv}
\DeclareMathOperator{\inte}{int}
\DeclareMathOperator*{\argmin}{argmin}
\DeclareMathOperator*{\argmax}{argmax}
\DeclareMathOperator{\graph}{graph}
\DeclareMathOperator{\sgn}{sgn}
\DeclareMathOperator*{\Rep}{Rep}
\DeclareMathOperator{\Proj}{Proj}
\DeclareMathOperator{\mat}{mat}
\DeclareMathOperator{\diag}{diag}
\DeclareMathOperator{\aut}{Aut}
\DeclareMathOperator{\gal}{Gal}
\DeclareMathOperator{\inn}{Inn}
\DeclareMathOperator{\edm}{End}
\DeclareMathOperator{\Hom}{Hom}
\DeclareMathOperator{\ext}{Ext}
\DeclareMathOperator{\tor}{Tor}
\DeclareMathOperator{\Span}{Span}
\DeclareMathOperator{\Stab}{Stab}
\DeclareMathOperator{\cont}{cont}
\DeclareMathOperator{\Ann}{Ann}
\DeclareMathOperator{\Div}{div}
\DeclareMathOperator{\curl}{curl}
\DeclareMathOperator{\nat}{Nat}
\DeclareMathOperator{\gr}{Gr}
\DeclareMathOperator{\vect}{Vect}
\DeclareMathOperator{\id}{id}
\DeclareMathOperator{\Mod}{Mod}
\DeclareMathOperator{\sign}{sign}
\DeclareMathOperator{\Surf}{Surf}
\DeclareMathOperator{\fcone}{fcone}
\DeclareMathOperator{\Rot}{Rot}
\DeclareMathOperator{\grad}{grad}
\DeclareMathOperator{\atan2}{atan2}
\DeclareMathOperator{\Ric}{Ric}
\let\vec\relax
\DeclareMathOperator{\vec}{vec}
\let\Re\relax
\DeclareMathOperator{\Re}{Re}
\let\Im\relax
\DeclareMathOperator{\Im}{Im}
% Put x \to \infty below \lim
\let\svlim\lim\def\lim{\svlim\limits}

%wide hat
\usepackage{scalerel,stackengine}
\stackMath
\newcommand*\wh[1]{%
\savestack{\tmpbox}{\stretchto{%
  \scaleto{%
    \scalerel*[\widthof{\ensuremath{#1}}]{\kern-.6pt\bigwedge\kern-.6pt}%
    {\rule[-\textheight/2]{1ex}{\textheight}}%WIDTH-LIMITED BIG WEDGE
  }{\textheight}% 
}{0.5ex}}%
\stackon[1pt]{#1}{\tmpbox}%
}
\parskip 1ex

%Make implies and impliedby shorter
\let\implies\Rightarrow
\let\impliedby\Leftarrow
\let\iff\Leftrightarrow
\let\epsilon\varepsilon

% Add \contra symbol to denote contradiction
\usepackage{stmaryrd} % for \lightning
\newcommand\contra{\scalebox{1.5}{$\lightning$}}

% \let\phi\varphi

% Command for short corrections
% Usage: 1+1=\correct{3}{2}

\definecolor{correct}{HTML}{009900}
\newcommand\correct[2]{\ensuremath{\:}{\color{red}{#1}}\ensuremath{\to }{\color{correct}{#2}}\ensuremath{\:}}
\newcommand\green[1]{{\color{correct}{#1}}}

% horizontal rule
\newcommand\hr{
    \noindent\rule[0.5ex]{\linewidth}{0.5pt}
}

% hide parts
\newcommand\hide[1]{}

% si unitx
\usepackage{siunitx}
\sisetup{locale = FR}

%allows pmatrix to stretch
\makeatletter
\renewcommand*\env@matrix[1][\arraystretch]{%
  \edef\arraystretch{#1}%
  \hskip -\arraycolsep
  \let\@ifnextchar\new@ifnextchar
  \array{*\c@MaxMatrixCols c}}
\makeatother

\renewcommand{\arraystretch}{0.8}

\renewcommand{\baselinestretch}{1.5}

\usepackage{graphics}
\usepackage{epstopdf}

\RequirePackage{hyperref}
%%
%% Add support for color in order to color the hyperlinks.
%% 
\hypersetup{
  colorlinks = true,
  urlcolor = blue,
  citecolor = blue
}
%%fakesection Links
\hypersetup{
    colorlinks,
    linkcolor={red!50!black},
    citecolor={green!50!black},
    urlcolor={blue!80!black}
}
%customization of cleveref
\RequirePackage[capitalize,nameinlink]{cleveref}[0.19]

% Per SIAM Style Manual, "section" should be lowercase
\crefname{section}{section}{sections}
\crefname{subsection}{subsection}{subsections}
\Crefname{section}{Section}{Sections}
\Crefname{subsection}{Subsection}{Subsections}

% Per SIAM Style Manual, "Figure" should be spelled out in references
\Crefname{figure}{Figure}{Figures}

% Per SIAM Style Manual, don't say equation in front on an equation.
\crefformat{equation}{\textup{#2(#1)#3}}
\crefrangeformat{equation}{\textup{#3(#1)#4--#5(#2)#6}}
\crefmultiformat{equation}{\textup{#2(#1)#3}}{ and \textup{#2(#1)#3}}
{, \textup{#2(#1)#3}}{, and \textup{#2(#1)#3}}
\crefrangemultiformat{equation}{\textup{#3(#1)#4--#5(#2)#6}}%
{ and \textup{#3(#1)#4--#5(#2)#6}}{, \textup{#3(#1)#4--#5(#2)#6}}{, and \textup{#3(#1)#4--#5(#2)#6}}

% But spell it out at the beginning of a sentence.
\Crefformat{equation}{#2Equation~\textup{(#1)}#3}
\Crefrangeformat{equation}{Equations~\textup{#3(#1)#4--#5(#2)#6}}
\Crefmultiformat{equation}{Equations~\textup{#2(#1)#3}}{ and \textup{#2(#1)#3}}
{, \textup{#2(#1)#3}}{, and \textup{#2(#1)#3}}
\Crefrangemultiformat{equation}{Equations~\textup{#3(#1)#4--#5(#2)#6}}%
{ and \textup{#3(#1)#4--#5(#2)#6}}{, \textup{#3(#1)#4--#5(#2)#6}}{, and \textup{#3(#1)#4--#5(#2)#6}}

% Make number non-italic in any environment.
\crefdefaultlabelformat{#2\textup{#1}#3}

% Environments
\makeatother
% For box around Definition, Theorem, \ldots
%%fakesection Theorems
\usepackage{thmtools}
\usepackage[framemethod=TikZ]{mdframed}

\theoremstyle{definition}
\mdfdefinestyle{mdbluebox}{%
	roundcorner = 10pt,
	linewidth=1pt,
	skipabove=12pt,
	innerbottommargin=9pt,
	skipbelow=2pt,
	nobreak=true,
	linecolor=blue,
	backgroundcolor=TealBlue!5,
}
\declaretheoremstyle[
	headfont=\sffamily\bfseries\color{MidnightBlue},
	mdframed={style=mdbluebox},
	headpunct={\\[3pt]},
	postheadspace={0pt}
]{thmbluebox}

\mdfdefinestyle{mdredbox}{%
	linewidth=0.5pt,
	skipabove=12pt,
	frametitleaboveskip=5pt,
	frametitlebelowskip=0pt,
	skipbelow=2pt,
	frametitlefont=\bfseries,
	innertopmargin=4pt,
	innerbottommargin=8pt,
	nobreak=false,
	linecolor=RawSienna,
	backgroundcolor=Salmon!5,
}
\declaretheoremstyle[
	headfont=\bfseries\color{RawSienna},
	mdframed={style=mdredbox},
	headpunct={\\[3pt]},
	postheadspace={0pt},
]{thmredbox}

\declaretheorem[%
style=thmbluebox,name=Theorem,numberwithin=section]{thm}
\declaretheorem[style=thmbluebox,name=Lemma,sibling=thm]{lem}
\declaretheorem[style=thmbluebox,name=Proposition,sibling=thm]{prop}
\declaretheorem[style=thmbluebox,name=Corollary,sibling=thm]{coro}
\declaretheorem[style=thmredbox,name=Example,sibling=thm]{eg}

\mdfdefinestyle{mdgreenbox}{%
	roundcorner = 10pt,
	linewidth=1pt,
	skipabove=12pt,
	innerbottommargin=9pt,
	skipbelow=2pt,
	nobreak=true,
	linecolor=ForestGreen,
	backgroundcolor=ForestGreen!5,
}

\declaretheoremstyle[
	headfont=\bfseries\sffamily\color{ForestGreen!70!black},
	bodyfont=\normalfont,
	spaceabove=2pt,
	spacebelow=1pt,
	mdframed={style=mdgreenbox},
	headpunct={ --- },
]{thmgreenbox}

\declaretheorem[style=thmgreenbox,name=Definition,sibling=thm]{defn}

\mdfdefinestyle{mdgreenboxsq}{%
	linewidth=1pt,
	skipabove=12pt,
	innerbottommargin=9pt,
	skipbelow=2pt,
	nobreak=true,
	linecolor=ForestGreen,
	backgroundcolor=ForestGreen!5,
}
\declaretheoremstyle[
	headfont=\bfseries\sffamily\color{ForestGreen!70!black},
	bodyfont=\normalfont,
	spaceabove=2pt,
	spacebelow=1pt,
	mdframed={style=mdgreenboxsq},
	headpunct={},
]{thmgreenboxsq}
\declaretheoremstyle[
	headfont=\bfseries\sffamily\color{ForestGreen!70!black},
	bodyfont=\normalfont,
	spaceabove=2pt,
	spacebelow=1pt,
	mdframed={style=mdgreenboxsq},
	headpunct={},
]{thmgreenboxsq*}

\mdfdefinestyle{mdblackbox}{%
	skipabove=8pt,
	linewidth=3pt,
	rightline=false,
	leftline=true,
	topline=false,
	bottomline=false,
	linecolor=black,
	backgroundcolor=RedViolet!5!gray!5,
}
\declaretheoremstyle[
	headfont=\bfseries,
	bodyfont=\normalfont\small,
	spaceabove=0pt,
	spacebelow=0pt,
	mdframed={style=mdblackbox}
]{thmblackbox}

\theoremstyle{plain}
\declaretheorem[name=Question,sibling=thm,style=thmblackbox]{ques}
\declaretheorem[name=Remark,sibling=thm,style=thmgreenboxsq]{remark}
\declaretheorem[name=Remark,sibling=thm,style=thmgreenboxsq*]{remark*}
\newtheorem{ass}[thm]{Assumptions}

\theoremstyle{definition}
\newtheorem*{problem}{Problem}
\newtheorem{claim}[thm]{Claim}
\theoremstyle{remark}
\newtheorem*{case}{Case}
\newtheorem*{notation}{Notation}
\newtheorem*{note}{Note}
\newtheorem*{motivation}{Motivation}
\newtheorem*{intuition}{Intuition}
\newtheorem*{conjecture}{Conjecture}

% Make section starts with 1 for report type
%\renewcommand\thesection{\arabic{section}}

% End example and intermezzo environments with a small diamond (just like proof
% environments end with a small square)
\usepackage{etoolbox}
\AtEndEnvironment{vb}{\null\hfill$\diamond$}%
\AtEndEnvironment{intermezzo}{\null\hfill$\diamond$}%
% \AtEndEnvironment{opmerking}{\null\hfill$\diamond$}%

% Fix some spacing
% http://tex.stackexchange.com/questions/22119/how-can-i-change-the-spacing-before-theorems-with-amsthm
\makeatletter
\def\thm@space@setup{%
  \thm@preskip=\parskip \thm@postskip=0pt
}

% Fix some stuff
% %http://tex.stackexchange.com/questions/76273/multiple-pdfs-with-page-group-included-in-a-single-page-warning
\pdfsuppresswarningpagegroup=1


% My name
\author{Jaden Wang}



\begin{document}
\centerline {\textsf{\textbf{\LARGE{Homework 12}}}}
\centerline {Jaden Wang}
\vspace{.15in}

\begin{problem}[7.1]
If $ M,N$ are Riemannian manifolds such that the inclusion  $ i: M \subset N$ is an isometric immersion, show by an example that the strict inequality $ d_M(x,y) > d_N(x,y)$ can occur.
\end{problem}
\begin{proof}
Let $ N= \rr^2$ with the Euclidean metric and $ M= \rr^2 \setminus D^2$ where $ D^2$ is the closed unit disk centered at the origin. Then the inclusion is clearly an isometric immersion. Notice  $ d_M((-1.1,0),(1.1,0)) > \pi$ since any path has to go around half of the unit disk. But $ d_N((-1.1,0),(1.1,0)) = 2.2 < \pi$. 
\end{proof}
\begin{problem}[7.2]
Let $ \widetilde{ M}$ be a covering space of a Riemannian manifold $ M$. Show that it is possible to give  $ \widetilde{ M}$ a Riemannian structure such that the covering map $ \pi: \widetilde{ M} \to M$ is a local isometry. Show that $ \widetilde{ M}$ is complete in the covering metric if and only if $ M$ is complete.
\end{problem}
\begin{proof}
	Let $ g$ be the metric on  $ M$. Define the pullback metric $ \pi^{*}g_{\widetilde{ p}} = g_{\pi(p)}$ for any $ \widetilde{ p} \in \widetilde{ M}$. Since $ \pi$ is already a local diffeomorphism, by definition of pullback metric it is a local isometry. Recall that in a covering space, any path downstairs can be lifted to a unique path upstairs once we choose an initial point, and any path upstairs can be projected to a unique path downstairs. Since geodesics are defined using local notions (covariant derivative), and since the covering map is a local isometry, any geodesic downstairs can be lifted to a unique geodesic upstairs with chosen initial point and vice versa. Thus if $ M$ is complete, for any point $ \widetilde{ p} \in \widetilde{ M}$ we can take any geodesic $ \widetilde{ \gamma}$ emanating from $ \widetilde{ p}$, project it to a geodesic in $ M$ and extend it for all time and then lift it back to the same starting point $ \widetilde{ p}$, which by uniqueness guarantees to coincide with the original geodesic $ \widetilde{ \gamma}$ at original time interval. The reverse direction is pretty much identical. By Hopf--Rinow, geodesic completeness and completeness are equivalent so we are done with the proof.
\end{proof}
\begin{problem}[7.9]
Consider the upper half-plane with the metric  $ g_{11} = 1, g_{12}=0, g_{22} = \frac{1}{y}$. Show that the length of the vertical segment $ x=0$,  $ \epsilon\leq y \leq 1$ with $ \epsilon>0$ tends to 2 as $ \epsilon \to 0$. Conclude from this that such a metric is not complete.
\end{problem}
\begin{proof}
Let $ \gamma(t) = (0,t)$ so $ \gamma'(t) = (0,1)$. Then the desired length is
\begin{align*}
	\int_{ \epsilon}^{ 1} \sqrt{g_{(0,t)} ((0,1),(0,1))}\ dt  &= \int_{ \epsilon}^{ 1} t^{-\frac{1}{2}} \ dt  \\
	&= 2 t^{\frac{1}{2}} \big|_{ \epsilon}^{1} \\
	&= 2- 2 \sqrt{ \epsilon}  .
\end{align*}
Clearly as $ \epsilon \to 0$, the length tends to 2. Now we wish to prove the negation of statement (e) in Hopf--Rinow: for all sequences of compact subsets $ K_n \subset M, K_n \subset K_{n+1} $ and $ \bigcup_{ n} K_n =M $, there exists a $ q_n \not\in K_n$ such that $ \lim_{ n \to \infty}  d(p,q_n) < \infty $.

Given any such sequence $ (K_n) $, and WLOG assume the intersection between $ K_n$ and the $ y$-axis is nonempty since eventually  $ K_n$ has to tend to the entire $ \rr_+^2$ and we only care about tail behavior.  Since $ K_n$ is compact and the projection function is continuous, there exists a point $ (0,y_n) \in K_n $ with a minimum  $ y$-value  $ y_n > 0$ in $ K_n$. Define  $ q_n = \left(0, \frac{y_n}{ 2}\right)$. By minimality of $ y_n$, we have $ q_n \not\in K_n$ and $ q_n \to (0,0)$. However, $ d((0,1),q_n) \to 2 < \infty$, proving the negation. 
\end{proof}
\begin{problem}[7.10]
Prove that the upper half-pane $ \rr_+^2$ with the Lobatchevski metric $ g_{11}= g_{22} = \frac{1}{y^2}$, $ g_{12} = 0 $ is complete.
\end{problem}
\begin{proof}
From Example 3.10 of Chapter 3, we know that its geodesics are vertical rays from the $ x$-axis and semicircles with center on the $ x$-axis. Given any $ p = (x,y) \in \rr_+^2$ and a semicircle or ray emanating from $ p$, clearly it can always be extended upward as there is no boundary or holes when  $ y$-value increases. But as  $ y$-value decreases, we just need to check that it doesn't reach the boundary $ x$-axis in finite time. In the case of a ray, as before we can integrate its length which is $ \ln y -\ln \epsilon$. As $ \epsilon \to 0$, the length goes to infinity. In the case of a semicircle, we can parameterize it as $ \gamma(t) = (y\cos t + x, y\sin t)$ with radius $ y$ and center at $ (x,0)$. 
\begin{align*}
	L( \gamma) &= \int_{ \epsilon}^{ \frac{\pi}{2}} \frac{1}{y \sin t}  \norm{ (-y \sin t, y \cos t)} dt\\  
	&= \int_{ \epsilon}^{ \frac{\pi}{2}} \frac{dt}{\sin t}  \\
	&= \ln \tan \frac{t}{2} \big|_{ \epsilon}^{\frac{\pi}{2}} \\
	&= \ln \tan \frac{\pi}{4} - \ln \epsilon .
\end{align*}
Again the arc length approaches $ \infty$ as $ \epsilon \to 0$. Since the geodesic moves at constant speed, it cannot cover arbitrarily large arc length in finite time. That it, it can be extended to all time. By Hopf--Rinow, the Lobatchevsky plane is complete.
\end{proof}
\begin{problem}[7.12]
A Riemannian manifold is \emph{homogeneous} if given $ p,q \in M$ there exists an isometry of $ M$ which takes  $ p$ into  $ q$. Prove that any homogeneous manifold is complete. 
\end{problem}
\begin{proof}
Let $ B \subset T_pM$ be the largest closed ball with radius $ r< \infty$ centered at 0 where $ \exp_p$ is still defined. Take any $ v \in \partial B$ so it has length $ r$. Define  $ q= \exp_p( v)$ which is also $ \gamma(1)$ where $ \gamma$ is the geodesic emanating from $ p$ with  $ \dot{ \gamma} (0) = v$. Let $ w = \dot{ \gamma}(1)$ which also has length $ r$ since geodesic has constant speed (parallel translation is an isometry). Let  $ f$ be the isometry that takes  $ p$ to  $ q$ and define  $ u = df^{-1}_q(w)$ which again has length  $ r$ and therefore  $ u \in B$. Let $ c(t)$ be the geodesic emanating from  $ p$ with  $ \dot{c}(0) = u$ and we have $ c(1) = \exp_p( u)$. Since isometry maps geodesics to geodesics, we see that $ f \circ c$ becomes a geodesic emanating from  $ q$ with initial velocity $ w$. By local uniqueness of geodesic, we can concatenate $ \gamma(t)$ and $ f \circ c(t)$ at $ q$ into a single geodesic. But notice that $ \exp_p( 2v) = f \circ c(1) $. Since $ v$ is  on the boundary of $ B$ and is arbitrary, it is a contradiction that $ 2v \in B$. Thus, we must have $ r = \infty $. Thus $ \exp_p$ is defined for all of $ T_pM$ and therefore by Hopf-Rinow, the manifold is complete.
\end{proof}
\end{document}
