\documentclass[12pt]{article}
\newcommand{\alert}[1]{{\bf \color{red} [Alert:] #1}}
\newcommand{\todo}[1]{{\bf \color{orange} [TODO:] #1}}
\newcommand{\real}[1][]{\mathbb{R}^{#1}}
\newcommand{\myeqn}[1]{(\ref{#1})}
\newcommand{\myex}[1]{Example \ref{#1}}
\newcommand{\defeq}{\stackrel{\mathrm{def}}{=}}
\newcommand{\parder}[2]{\frac{\partial #1}{\partial #2}}
\newcommand{\Lie}[3][]{\mathsf{L}_{#3}^{#1} #2}
\newcommand{\LieA}[1]{\mathsf{Lie}(#1)}
\newcommand{\lieder}[2]{\mathcal{L}_{#2} #1}
\renewcommand{\t}{^{\mbox{\tiny\sf T}}}
\newcommand{\trans}{^{\mbox{\tiny\sf T}}}
\newcommand{\markup}[1]{\{\textbf{#1}\}}
\newcommand{\msub}[1]{_\mathrm{#1}}
\newcommand{\msup}[1]{^\mathrm{#1}}
\newcommand{\inv}[1]{#1^{-1}}
\newcommand{\pinv}[1]{{#1}^{+}}
\newcommand{\myfracA}[2]{\displaystyle{\frac{#1}{#2}}}
\newcommand{\myfracB}[2]{{#1}/{#2}}
\newcommand{\mydiffA}[1]{\dot{#1}}
\newcommand{\mydiffB}[2]{\myfracA{\mathrm{d}{#1}}{\mathrm{d}{#2}}}
\newcommand{\ball}[2]{\mathcal{B}_{#1}\left(#2\right)}
\newcommand{\acos}[1]{\cos^{-1}\left(#1\right)}
\newcommand{\asin}[1]{\sin^{-1}\left(#1\right)}
\newcommand{\mani}{\mathcal{M}}
\newcommand{\tang}[2]{\mathsf{T}_{#1} #2}
\newcommand{\LieB}[2]{[ #1, #2 ]}
\newcommand{\LieBad}[3][]{\mathsf{ad}_{#2}^{#1} #3}
\newcommand{\ReachVT}{\mathcal{R}^V_T}
\newcommand{\ReachVt}{\mathcal{R}^V_t}
\newcommand{\ReachVTe}{\mathcal{R}^V_{\le T}}
\newcommand{\ReachT}{\mathcal{R}_T}
\newcommand{\Reacht}{\mathcal{R}_t}
\newcommand{\ReachTe}{\mathcal{R}_{\le T}}
\newcommand{\accLA}[1]{\mathsf{Lie}(#1)}
\newcommand{\accD}{\Delta_{\mathcal{F}}}
\newcommand{\accSA}{\mathsf{Lie}(\mathcal{G},f)}
\newcommand{\accDS}{\Delta_{\mathcal{G}}}
\newcommand{\eval}[3]{\mathsf{Ev}^{#2}_{#1}\left( #3 \right)}
\newcommand{\stlc}{\textsc{stlc}}
\newcommand{\clf}{\textsc{clf}}
\newcommand{\jqlf}{\textsc{jqlf}}
\newcommand{\dlas}{\textsc{dlas}}
\newcommand{\Ad}[2]{\mathsf{Ad}_{#1} #2}
\newcommand{\xe}{\ensuremath{x_e}}
\newcommand{\lebg}[1]{\mathcal{L}_{#1}}
\newcommand{\lebgx}[1]{\mathcal{L}_{#1 \mathrm{e}}}
\newcommand{\dom}{D}
\newcommand{\domT}{[t_0,\infty) \times D}
\newcommand{\rarrow}{\rightarrow}
\renewcommand{\d}{\mathrm{d}}
\renewcommand{\Re}{\mathbb{R}}
\newcommand{\C}{\mathrm{C}}

\newcommand{\QED}{{\unskip\nobreak\hfil\penalty50\hskip2em\vadjust{}
		\nobreak\hfil$\Box$\parfillskip=0pt\finalhyphendemerits=0\par}\vspace{0.1cm}}
\newcommand{\eoEx}{{\unskip\nobreak\hfil\penalty50\hskip0em\vadjust{}
		\nobreak\hfil$\Large\Diamond$\parfillskip=0pt\finalhyphendemerits=0\par}\vspace{0.1cm}}

\newcommand{\sgn}{\ensuremath{\operatorname{sgn}}}
\newcommand{\sat}{\ensuremath{\operatorname{sat}}}

\newcommand{\half}{\frac{1}{2}}
\newcommand{\shalf}{\mbox{$\frac{1}{2}$}}
\newcommand{\marcom}[1]{\marginpar{\footnotesize #1}}
\newcommand{\der}{\mathrm{D}}
\newcommand{\e}{\mathrm{e}}
\newcommand{\dt}{\mathrm{d}t}

\newcommand{\cA}{\ensuremath{\mathcal{A}}}
\newcommand{\cB}{\ensuremath{\mathcal{B}}}
\newcommand{\cG}{\ensuremath{\mathcal{G}}}
\newcommand{\cK}{\ensuremath{\mathcal{K}}}
\newcommand{\cW}{\ensuremath{\mathcal{W}}}
\newcommand{\cZ}{\ensuremath{\mathcal{Z}}}
\newcommand{\cS}{\ensuremath{\mathcal{S}}}
\newcommand{\cD}{\ensuremath{\mathcal{D}}}
\newcommand{\cP}{\ensuremath{\mathcal{P}}}
\newcommand{\cV}{\ensuremath{\mathcal{V}}}
\newcommand{\cL}{\ensuremath{\mathcal{L}}}
\newcommand{\cN}{\ensuremath{\mathcal{N}}}
\newcommand{\cI}{\ensuremath{\mathcal{I}}}
\newcommand{\cR}{\ensuremath{\mathcal{R}}}
\newcommand{\cM}{\ensuremath{\mathcal{M}}}
\newcommand{\cC}{\ensuremath{\mathcal{C}}}
\newcommand{\cF}{\ensuremath{\mathcal{F}}}
\newcommand{\cH}{\ensuremath{\mathcal{H}}}
\newcommand{\cO}{\ensuremath{\mathcal{O}}}
\newcommand{\cX}{\ensuremath{\mathcal{X}}}
\newcommand{\cY}{\ensuremath{\mathcal{Y}}}
\newcommand{\Ci}{\ensuremath{\mathcal{C}^\infty}}
\newcommand{\ISS}{\textsc{iss}}
\newcommand{\LISS}{\textsc{liss}}
\newcommand{\GAS}{\textsc{gas}}
\newcommand{\GS}{\textsc{gs}}
\newcommand{\LES}{\textsc{les}}
\newcommand{\GUAS}{\textsc{guas}}
\newcommand{\BIBO}{\textsc{bibo}}
\newcommand{\spec}{\ensuremath{\operatorname{spec}}}
\newcommand{\spn}{\ensuremath{\operatorname{span}}}
\renewcommand{\i}{\mathrm{i\,}}

\renewcommand{\implies}{\Rightarrow}

\renewcommand{\theenumi}{$\roman{enumi})$}
\renewcommand{\labelenumi}{\theenumi}

\font\ptmten=zptmcmrm scaled 1200
\newcommand{\w}{\mbox{{\ptmten w}}}
\newcommand{\z}{\mbox{{\ptmten z}}}
\renewcommand{\Re}{\mathbb{R}}

\newcommand{\cl}{\operatorname{cl}}
\newcommand{\intr}{\operatorname{int}}
\newcommand{\rank}{\operatorname{rank}}
\newcommand{\co}{\operatorname{co}}
\newcommand{\aff}{\operatorname{aff}}

\theoremstyle{plain}
\newtheorem{theorem}{Theorem}[chapter]
\newtheorem{claim}[theorem]{Claim}
\newtheorem{corollary}[theorem]{Corollary}
\newtheorem{prop}[theorem]{Proposition}
\newtheorem{fact}[theorem]{Fact}
\newtheorem{lemma}[theorem]{Lemma}

\newtheorem{remark}{Remark}[chapter]

\theoremstyle{definition}
\newtheorem{assume}[theorem]{Assumption}
\newtheorem{defn}[theorem]{Definition}
\newtheorem{problem}[theorem]{Problem}
\newtheorem{exercise}{Exercise}
\newtheorem{example}[theorem]{Example}


\begin{document}
\centerline {\textsf{\textbf{\LARGE{Homework 7}}}}
\centerline {Jaden Wang}
\vspace{.15in}

\begin{problem}[do Carmo 3.8]
Let $ M$ be a Riemannian manifold. Let  $ X \in \mathfrak{X}(M)$ and $ f \in \mathcal{D}(M)$. Define the divergence of $ X$ as a function  $ \Div X(p) =$ trace of the linear map $ Y(p) \to \nabla _Y X(p), p \in M$, and the gradient of $ f$ as a vector field  $ \grad f$  on $ M$ defined by
 \begin{align*}
	\langle \grad f(p), v \rangle = df_p(v), \qquad p \in M, \quad v \in T_p\dot{M}
\end{align*}
\begin{enumerate}[label=(\alph*)]
	\item Let $ E_i$ be a geodesic frame at $ p \in M$. Show that
		\begin{align*}
			\grad f(p) = \sum_{ i= 1}^{ n} (E_i(f)) E_i(p),
		\end{align*}
		and
		\begin{align*}
			\Div X(p) = \sum_{ i= 1}^{ n} E_i(f_i)(p),
		\end{align*}
		where $ X = f^{i} E_i$.
	\item Let $ M = \rr^{n}$, with coordinates $ (x_1,\ldots,x_n)$ and $ \frac{\partial }{\partial x_i} = (0,\ldots,1,\ldots,0) = e_i$. Show that
		\begin{align*}
			\grad f = \sum_{ i= 1}^{ n} \frac{\partial f}{\partial x_i} e_i, 
		\end{align*}
		$ \Div X = \sum_{ i= 1}^{ n} \frac{\partial f_i}{\partial x_i} $, where $ X = f^{i} e_i$.
\end{enumerate}
\end{problem}
\begin{proof}
\begin{enumerate}[label=(\alph*)]
	\item Recall that a geodesic frame  Write $ v = v^{i} E_i(p)$. Then
	\begin{align*}
		df_p(v) &= df_p\left( v^{i} E_i(p) \right)  \\
			&= v^{i} df_p(E_i(p))  && \text{linear map}  \\
		&= v^{i} \frac{\partial f}{\partial x_i}(p) \\
		&= v^{i} E_i(f) (p) \\
		&= \langle E_i(f) E_i(p), v^{i} E_i(p) \rangle \\
		&= \langle \grad f, v \rangle .
	\end{align*}
	Thus $ \grad f = E^{i}(f) E_i(p)$.

	Recall that trace of a linear map $ L$ is defined as $ \tr L := \sum_{ i= 1}^{ n}  \langle L(E_i), E_i \rangle$.
	\begin{align*}
		\Div X(p) &= \Div \left( f^{i} E_i \right)(p)  \\
		&= \sum_{ j= 1}^{ n}  \langle \nabla _{E_j} \left( f^{i} E_i \right)(p)  , E_j(p)  \rangle \\
		&= \sum_{j= 1}^{ n}  \langle E_j(f^{i})E_i(p) + f^{i}\nabla_{E_j} E_i(p), E_j(p) \rangle\\
		&= \sum_{ j=1}^{ n}  \langle E_j(f^{i})E_i(p) + 0, E_j(p) \rangle && \text{geodesic frame} \\
		&=  E_i(f^{i})(p) .
	\end{align*}
\item This is immediate.
\begin{align*}
	\grad f(p) &= e^{i}(f) e_i(p) \\
	&= \frac{\partial f}{\partial x_i} e_i  .
\end{align*}
\begin{align*}
	\Div X (p) &= E_i(f^{i}) (p) \\
	&= \frac{\partial f^{i}}{\partial x_i}   .
\end{align*}
\end{enumerate}
\end{proof}

\begin{problem}[do Carmo 3.9]
Let $ M$ be a Riemannian manifold. Define the Laplacian $ \Delta: \mathcal{ D}(M) \to \mathcal{ D}(M)$ of $ M$ by
 \begin{align*}
	\Delta f = \Div \grad f, \qquad f \in \mathcal{ D}(M).
\end{align*}
\begin{enumerate}[label=(\alph*)]
	\item Let $ E_i$ be a geodesic frame at $ p \in M$. Prove that
	\begin{align*}
		\Delta f(p) = E_i (E^{i}(f))(p) .
	\end{align*}
	Conclude that if $ M = \rr^{n}$, $ \Delta$ coincides with the usual Laplacian, namely $ \Delta f = \sum_i \frac{\partial^2 f}{\partial { x_i}^2} $. 
\item Show that
	\begin{align*}
		\Delta (f \cdot g) = f \Delta g + g \Delta f + 2\langle \grad f, \grad g \rangle .
	\end{align*}
\end{enumerate}
\end{problem}
\begin{proof}
\begin{enumerate}[label=(\alph*)]
	\item This is immediate from previous problem. Since $ \grad f = E^i(f) E_i$,
	\begin{align*}
		\Delta f(p) &= \Div (E^i(f) E_i)(p) \\
		&= E_i(E^{i} (f)) (p) .
	\end{align*}
	If $ M = \rr^{n}$, then 
	\begin{align*}
		\Delta f = E_{i} (E^{i}(f)) = E_{i} \left( \frac{\partial f}{\partial x^i}  \right) = \sum_i \frac{\partial^2 f}{\partial { x_{i}}^2} .
	\end{align*}
\item First we compute
	\begin{align*}
		\langle \grad f, \grad g \rangle = \langle E^{i}(f) E_i, E^{i}(g) E_i \rangle = E^{i}(f) E_{i}(g) .
	\end{align*}
It follows that
\begin{align*}
	\Delta (f \cdot g) &= E_i(E^{i}(f \cdot g)) \\
			   &= E_i(E^{i}(f) g + f E^{i}(g)) && \text{Leibniz rule of derivation}  \\
			   &= E_i(E^{i}(f)) g + E^i(f)E_i(g) + E_i (f) E^{i}(g) + f E_i(E^{i}(g)) \\
			   &= g\Delta f  + 2 \langle \grad f, \grad g \rangle + f \Delta g .
\end{align*}
\end{enumerate}
\end{proof}

\begin{problem}[do Carmo 3.10]
	Let $ f: I \times [0,a] \to M$ be a parametrized surface such that for all $ t_0 \in [0,a]$, the curve $ s \to f(s,t_0), s \in [0,1]$ is a geodesic parametrized by arc length, which is orthogonal to the curve $ t \to f(0,t), t \in [0,a]$, at the point $ f(0,t_0)$. Prove that, for all $ (s_0,t_0) \in I \times [0,a]$, the curves $ s \to f(s,t_0), t \to f(s_0,t)$ are orthogonal.
\end{problem}

\begin{proof}
	We want to show that $ \langle \frac{\partial f}{\partial s} , \frac{\partial f}{\partial t}  \rangle \equiv 0$. Since we already know that $ \langle \frac{\partial f}{\partial s} \big|_{(0,t_0)} , \frac{\partial f}{\partial t} \big|_{(0,t_0)}  \rangle = 0$ for any $ t_0 \in [0,a]$, it suffices to show that the inner product is constant as we vary $ s$,  \emph{i.e.} it has 0 derivative wrt $ s$. We compute
\begin{align*}
	\frac{\partial }{\partial s} \left\langle \frac{\partial f}{\partial s} , \frac{\partial f}{\partial t} \right \rangle &= \left\langle \frac{D}{\partial s} \frac{\partial f}{\partial s}, \frac{\partial f}{\partial t}   \right\rangle + \left\langle \frac{\partial f}{\partial s} , \frac{D}{\partial s} \frac{\partial f}{\partial t}  \right\rangle  && \text{Leibniz rule}  \\
	&=  \left\langle \frac{D}{\partial s} \frac{\partial f}{\partial s}, \frac{\partial f}{\partial t}   \right\rangle + \left\langle \frac{\partial f}{\partial s} , \frac{D}{\partial t} \frac{\partial f}{\partial s}  \right\rangle  && \text{symmetry lemma} \\
	&=  \left\langle 0, \frac{\partial f}{\partial t}   \right\rangle + \left\langle \frac{\partial f}{\partial s} , \frac{D}{\partial t} \frac{\partial f}{\partial s}  \right\rangle  && \text{geodesic along }s \\
	&= \frac{1}{2} \frac{\partial }{\partial t} \left\langle \frac{\partial f}{\partial s}, \frac{\partial f}{\partial s}  \right\rangle \\
	&= 0  .
\end{align*}
\end{proof}
\end{document}
