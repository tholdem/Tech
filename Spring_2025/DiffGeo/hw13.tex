\documentclass[12pt]{article}
\newcommand{\alert}[1]{{\bf \color{red} [Alert:] #1}}
\newcommand{\todo}[1]{{\bf \color{orange} [TODO:] #1}}
\newcommand{\real}[1][]{\mathbb{R}^{#1}}
\newcommand{\myeqn}[1]{(\ref{#1})}
\newcommand{\myex}[1]{Example \ref{#1}}
\newcommand{\defeq}{\stackrel{\mathrm{def}}{=}}
\newcommand{\parder}[2]{\frac{\partial #1}{\partial #2}}
\newcommand{\Lie}[3][]{\mathsf{L}_{#3}^{#1} #2}
\newcommand{\LieA}[1]{\mathsf{Lie}(#1)}
\newcommand{\lieder}[2]{\mathcal{L}_{#2} #1}
\renewcommand{\t}{^{\mbox{\tiny\sf T}}}
\newcommand{\trans}{^{\mbox{\tiny\sf T}}}
\newcommand{\markup}[1]{\{\textbf{#1}\}}
\newcommand{\msub}[1]{_\mathrm{#1}}
\newcommand{\msup}[1]{^\mathrm{#1}}
\newcommand{\inv}[1]{#1^{-1}}
\newcommand{\pinv}[1]{{#1}^{+}}
\newcommand{\myfracA}[2]{\displaystyle{\frac{#1}{#2}}}
\newcommand{\myfracB}[2]{{#1}/{#2}}
\newcommand{\mydiffA}[1]{\dot{#1}}
\newcommand{\mydiffB}[2]{\myfracA{\mathrm{d}{#1}}{\mathrm{d}{#2}}}
\newcommand{\ball}[2]{\mathcal{B}_{#1}\left(#2\right)}
\newcommand{\acos}[1]{\cos^{-1}\left(#1\right)}
\newcommand{\asin}[1]{\sin^{-1}\left(#1\right)}
\newcommand{\mani}{\mathcal{M}}
\newcommand{\tang}[2]{\mathsf{T}_{#1} #2}
\newcommand{\LieB}[2]{[ #1, #2 ]}
\newcommand{\LieBad}[3][]{\mathsf{ad}_{#2}^{#1} #3}
\newcommand{\ReachVT}{\mathcal{R}^V_T}
\newcommand{\ReachVt}{\mathcal{R}^V_t}
\newcommand{\ReachVTe}{\mathcal{R}^V_{\le T}}
\newcommand{\ReachT}{\mathcal{R}_T}
\newcommand{\Reacht}{\mathcal{R}_t}
\newcommand{\ReachTe}{\mathcal{R}_{\le T}}
\newcommand{\accLA}[1]{\mathsf{Lie}(#1)}
\newcommand{\accD}{\Delta_{\mathcal{F}}}
\newcommand{\accSA}{\mathsf{Lie}(\mathcal{G},f)}
\newcommand{\accDS}{\Delta_{\mathcal{G}}}
\newcommand{\eval}[3]{\mathsf{Ev}^{#2}_{#1}\left( #3 \right)}
\newcommand{\stlc}{\textsc{stlc}}
\newcommand{\clf}{\textsc{clf}}
\newcommand{\jqlf}{\textsc{jqlf}}
\newcommand{\dlas}{\textsc{dlas}}
\newcommand{\Ad}[2]{\mathsf{Ad}_{#1} #2}
\newcommand{\xe}{\ensuremath{x_e}}
\newcommand{\lebg}[1]{\mathcal{L}_{#1}}
\newcommand{\lebgx}[1]{\mathcal{L}_{#1 \mathrm{e}}}
\newcommand{\dom}{D}
\newcommand{\domT}{[t_0,\infty) \times D}
\newcommand{\rarrow}{\rightarrow}
\renewcommand{\d}{\mathrm{d}}
\renewcommand{\Re}{\mathbb{R}}
\newcommand{\C}{\mathrm{C}}

\newcommand{\QED}{{\unskip\nobreak\hfil\penalty50\hskip2em\vadjust{}
		\nobreak\hfil$\Box$\parfillskip=0pt\finalhyphendemerits=0\par}\vspace{0.1cm}}
\newcommand{\eoEx}{{\unskip\nobreak\hfil\penalty50\hskip0em\vadjust{}
		\nobreak\hfil$\Large\Diamond$\parfillskip=0pt\finalhyphendemerits=0\par}\vspace{0.1cm}}

\newcommand{\sgn}{\ensuremath{\operatorname{sgn}}}
\newcommand{\sat}{\ensuremath{\operatorname{sat}}}

\newcommand{\half}{\frac{1}{2}}
\newcommand{\shalf}{\mbox{$\frac{1}{2}$}}
\newcommand{\marcom}[1]{\marginpar{\footnotesize #1}}
\newcommand{\der}{\mathrm{D}}
\newcommand{\e}{\mathrm{e}}
\newcommand{\dt}{\mathrm{d}t}

\newcommand{\cA}{\ensuremath{\mathcal{A}}}
\newcommand{\cB}{\ensuremath{\mathcal{B}}}
\newcommand{\cG}{\ensuremath{\mathcal{G}}}
\newcommand{\cK}{\ensuremath{\mathcal{K}}}
\newcommand{\cW}{\ensuremath{\mathcal{W}}}
\newcommand{\cZ}{\ensuremath{\mathcal{Z}}}
\newcommand{\cS}{\ensuremath{\mathcal{S}}}
\newcommand{\cD}{\ensuremath{\mathcal{D}}}
\newcommand{\cP}{\ensuremath{\mathcal{P}}}
\newcommand{\cV}{\ensuremath{\mathcal{V}}}
\newcommand{\cL}{\ensuremath{\mathcal{L}}}
\newcommand{\cN}{\ensuremath{\mathcal{N}}}
\newcommand{\cI}{\ensuremath{\mathcal{I}}}
\newcommand{\cR}{\ensuremath{\mathcal{R}}}
\newcommand{\cM}{\ensuremath{\mathcal{M}}}
\newcommand{\cC}{\ensuremath{\mathcal{C}}}
\newcommand{\cF}{\ensuremath{\mathcal{F}}}
\newcommand{\cH}{\ensuremath{\mathcal{H}}}
\newcommand{\cO}{\ensuremath{\mathcal{O}}}
\newcommand{\cX}{\ensuremath{\mathcal{X}}}
\newcommand{\cY}{\ensuremath{\mathcal{Y}}}
\newcommand{\Ci}{\ensuremath{\mathcal{C}^\infty}}
\newcommand{\ISS}{\textsc{iss}}
\newcommand{\LISS}{\textsc{liss}}
\newcommand{\GAS}{\textsc{gas}}
\newcommand{\GS}{\textsc{gs}}
\newcommand{\LES}{\textsc{les}}
\newcommand{\GUAS}{\textsc{guas}}
\newcommand{\BIBO}{\textsc{bibo}}
\newcommand{\spec}{\ensuremath{\operatorname{spec}}}
\newcommand{\spn}{\ensuremath{\operatorname{span}}}
\renewcommand{\i}{\mathrm{i\,}}

\renewcommand{\implies}{\Rightarrow}

\renewcommand{\theenumi}{$\roman{enumi})$}
\renewcommand{\labelenumi}{\theenumi}

\font\ptmten=zptmcmrm scaled 1200
\newcommand{\w}{\mbox{{\ptmten w}}}
\newcommand{\z}{\mbox{{\ptmten z}}}
\renewcommand{\Re}{\mathbb{R}}

\newcommand{\cl}{\operatorname{cl}}
\newcommand{\intr}{\operatorname{int}}
\newcommand{\rank}{\operatorname{rank}}
\newcommand{\co}{\operatorname{co}}
\newcommand{\aff}{\operatorname{aff}}

\theoremstyle{plain}
\newtheorem{theorem}{Theorem}[chapter]
\newtheorem{claim}[theorem]{Claim}
\newtheorem{corollary}[theorem]{Corollary}
\newtheorem{prop}[theorem]{Proposition}
\newtheorem{fact}[theorem]{Fact}
\newtheorem{lemma}[theorem]{Lemma}

\newtheorem{remark}{Remark}[chapter]

\theoremstyle{definition}
\newtheorem{assume}[theorem]{Assumption}
\newtheorem{defn}[theorem]{Definition}
\newtheorem{problem}[theorem]{Problem}
\newtheorem{exercise}{Exercise}
\newtheorem{example}[theorem]{Example}


\begin{document}
\centerline {\textsf{\textbf{\LARGE{Homework 13}}}}
\centerline {Jaden Wang}
\vspace{.15in}
\begin{problem}[Do Carmo 8.1]
Consider on a neighborhood of $ \rr^{n}$, $ n>2$, the metric
 \begin{align*}
	g_{ij} = \frac{\delta_{ij}}{ F^2},
\end{align*}
where $ F \neq 0$ is a function on  $ \rr^{n}$. Denote $ F_i = \frac{\partial F}{\partial x_i}, F_{ij} = \frac{\partial^2 F}{\partial { x_i} \partial x_j}  $.
\begin{enumerate}[label=(\alph*)]
	\item Show that a necessary and sufficient condition for the metric to have constant curvature $ K$ is
	 \begin{align*}
		\begin{cases}
			F_{ij} =0, \quad i\neq j\\
			F(F_{jj}+F_{ii}) = K + \sum_{ i= 1}^{ n} (F_i)^2 .\\
		\end{cases}
	\end{align*}
\item Use above to prove that the metric $ g_{ij}$ has constant curvature $ K$ iff
	 \begin{align*}
		F= G_1(x_1) + G_2(x_2) + \cdots + G_n(x_n),
	\end{align*}
	where $ G_i(x_i) = ax_i^2+b_i x_i + c_i$ and $ K = \sum_{ i= 1}^{ n} (4c_ia -b_i^2)$.
\item Put $ a = \frac{K}{4}, b_i =0, c_i =\frac{1}{n}$ and obtain the formula of Riemann
	\begin{align*}
		g_{ij} = \frac{\delta_{ij}}{ \left( 1+\frac{K}{4} \norm{ x}^2  \right) ^2}
	\end{align*}
	for a metric $ g_{ij}$ of constant curvature $ K$ ( $ \norm{ \cdot } $ here denotes Euclidean norm). If $ K<0$ then metric  $ g_{ij}$ is defined in a ball of radius $ \sqrt{ \frac{4}{-K}} $.
\item If $ K>0$, the metric is defined on all of  $ \rr^{n}$. Show that such a metric on $ \rr^{n}$ is not complete.
\end{enumerate}
\end{problem}
\begin{proof}
\begin{enumerate}[label=(\alph*)]
	\item We compute
	\begin{align*}
		f_i = \frac{F_i}{ F}, \quad f_{ij} = -\frac{F_iF_j}{ F^2} + \frac{F_{ij} }{ F}.
	\end{align*}
	Based on the formula from book, if any three indices are distinct, then by Chapter 4 Corollary 3.5, constant sectional curvature is equivalent to
	\begin{align*}
		0= R_{ijk\ell} &= R_{ijk}^{s \ell} g_{s \ell} \\
		&= -\delta_{is}(-f_k f_j - f_{kj}) g_{i\ell} + \delta_{js}(f_if_k+f_{ki})g_{j\ell}  \\
		&= -\frac{1}{F^2}(-F_k F_j + F_kF_j - F F_{kj}) \frac{\delta_{i \ell}}{ F^2} + \frac{1}{F^2}(F_iF_k- F_i F_k+F F_{ki}) \frac{\delta_{j\ell}}{ F^2}  \\
		&= \frac{ \delta_{i \ell}F_{kj}} {F^3} - \frac{ \delta_{j\ell}F_{ki}}{F^3}  .
	\end{align*}
	Since $ F \neq 0$, it follows that  $ F_{ij} =0$ as long as $ i \neq j$. Thus we establish equivalence for the first equation.

	The second equivalence is a straightforward computation using formula from book:
	\begin{align*}
		K &= \left( - \sum_{\ell} \frac{F_\ell^2}{ F^2} + \frac{F_i^2}{ F^2} + \frac{F_j}{ F^2} - \frac{F_i^2}{ F^2} + \frac{F_{ii} }{ F} - \frac{F_j^2}{ F^2} + \frac{F_{jj}}{ F}  \right)F^2\\
	F(F_{jj} + F_{ii} )	&= K + \sum_{\ell} F_{\ell}^2 .
	\end{align*}
\item The second partial is zero for $ i \neq j$ iff there are no cross terms in  $ F$ by basic Calculus.  Due to this fact and the fact that partials commute in $ \rr^{n}$, $ (F_{ii})_j = (F_{ij})_i =0$ whenever $ i \neq j$. This is equivalent to $ F_{ii} = F_{jj}$ being constants which we call $ 2a$. Then calculus gives that $ G_i(x_i) = a x_i^2 + b_i_ x_i + c_i$. And the second equation therefore is equivalent to $ K = \sum_{ i= 1}^{ n} (4 c_i a - b_i^2)$.   
\item This is obvious.
\item Given any point $ x \in \rr^{n}$, let $ \gamma(t) = tx$ so $ \gamma'(t) = x$. Then the distance between the origin $ 0$ and  $ x$ is upper-bounded by the length of this path:
\begin{align*}
	\int_{ 0}^{ 1} \sqrt{g_{ \gamma(t)}( \gamma'(t), \gamma'(t))} dt &=  \int_0^1 \frac{ \norm{  x}  }{ 1+ \frac{K}{4} \norm{ t x}^2   } dt \\
	&= \frac{4}{K \norm{ x} } \int_{ 0}^{ 1} \frac{1}{ \left( \frac{2}{\sqrt{K} \norm{ x} } \right)^2  +  t^2} dt  \\
	&= \frac{4}{K\norm{ x} } \frac{\sqrt{K} \norm{ x}  }{2 } \arctan \left( \frac{\sqrt{K} \norm{ x} t}{2} \right)\bigg|_{0}^1   \\
	&= \frac{2}{\sqrt{K} } \arctan\left( \frac{\sqrt{K} \norm{ x}  }{ 2}  \right)  \\
	&< \frac{2}{\sqrt{K} } \frac{\pi}{ 2}  \\
	&= \frac{\pi}{ \sqrt{K} }   .
\end{align*}
	Now consider the harmonic series $ (x_n) = \left(\sum_{ i= 1}^{ n} \frac{1}{n},0,\ldots,0\right)$. The series is Cauchy because the distance between any element and origin has the same upper bound, and the sequence is monotone increasing in the first entry so as a consequence of Monotone Convergence Theorem, the distance between any two elements $ x_n, x_m$ must be less than any given $ \epsilon$ when $ n,m$ is large enough. However, the series diverges to $ \infty$ so Cauchy sequence does not converge in $ \rr^{n}$, and thus the metric is not complete.
\end{enumerate}
\end{proof}
\end{document}
