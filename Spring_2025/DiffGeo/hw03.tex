\documentclass[12pt]{article}
%Fall 2022
% Some basic packages
\usepackage{standalone}[subpreambles=true]
\usepackage[utf8]{inputenc}
\usepackage[T1]{fontenc}
\usepackage{textcomp}
\usepackage[english]{babel}
\usepackage{url}
\usepackage{graphicx}
%\usepackage{quiver}
\usepackage{float}
\usepackage{enumitem}
\usepackage{lmodern}
\usepackage{comment}
\usepackage{hyperref}
\usepackage[usenames,svgnames,dvipsnames]{xcolor}
\usepackage[margin=1in]{geometry}
\usepackage{pdfpages}

\pdfminorversion=7

% Don't indent paragraphs, leave some space between them
\usepackage{parskip}

% Hide page number when page is empty
\usepackage{emptypage}
\usepackage{subcaption}
\usepackage{multicol}
\usepackage[b]{esvect}

% Math stuff
\usepackage{amsmath, amsfonts, mathtools, amsthm, amssymb}
\usepackage{bbm}
\usepackage{stmaryrd}
\allowdisplaybreaks

% Fancy script capitals
\usepackage{mathrsfs}
\usepackage{cancel}
% Bold math
\usepackage{bm}
% Some shortcuts
\newcommand{\rr}{\ensuremath{\mathbb{R}}}
\newcommand{\zz}{\ensuremath{\mathbb{Z}}}
\newcommand{\qq}{\ensuremath{\mathbb{Q}}}
\newcommand{\nn}{\ensuremath{\mathbb{N}}}
\newcommand{\ff}{\ensuremath{\mathbb{F}}}
\newcommand{\cc}{\ensuremath{\mathbb{C}}}
\newcommand{\ee}{\ensuremath{\mathbb{E}}}
\newcommand{\hh}{\ensuremath{\mathbb{H}}}
\renewcommand\O{\ensuremath{\emptyset}}
\newcommand{\norm}[1]{{\left\lVert{#1}\right\rVert}}
\newcommand{\dbracket}[1]{{\left\llbracket{#1}\right\rrbracket}}
\newcommand{\ve}[1]{{\bm{#1}}}
\newcommand\allbold[1]{{\boldmath\textbf{#1}}}
\DeclareMathOperator{\lcm}{lcm}
\DeclareMathOperator{\im}{im}
\DeclareMathOperator{\coim}{coim}
\DeclareMathOperator{\dom}{dom}
\DeclareMathOperator{\tr}{tr}
\DeclareMathOperator{\rank}{rank}
\DeclareMathOperator*{\var}{Var}
\DeclareMathOperator*{\ev}{E}
\DeclareMathOperator{\dg}{deg}
\DeclareMathOperator{\aff}{aff}
\DeclareMathOperator{\conv}{conv}
\DeclareMathOperator{\inte}{int}
\DeclareMathOperator*{\argmin}{argmin}
\DeclareMathOperator*{\argmax}{argmax}
\DeclareMathOperator{\graph}{graph}
\DeclareMathOperator{\sgn}{sgn}
\DeclareMathOperator*{\Rep}{Rep}
\DeclareMathOperator{\Proj}{Proj}
\DeclareMathOperator{\mat}{mat}
\DeclareMathOperator{\diag}{diag}
\DeclareMathOperator{\aut}{Aut}
\DeclareMathOperator{\gal}{Gal}
\DeclareMathOperator{\inn}{Inn}
\DeclareMathOperator{\edm}{End}
\DeclareMathOperator{\Hom}{Hom}
\DeclareMathOperator{\ext}{Ext}
\DeclareMathOperator{\tor}{Tor}
\DeclareMathOperator{\Span}{Span}
\DeclareMathOperator{\Stab}{Stab}
\DeclareMathOperator{\cont}{cont}
\DeclareMathOperator{\Ann}{Ann}
\DeclareMathOperator{\Div}{div}
\DeclareMathOperator{\curl}{curl}
\DeclareMathOperator{\nat}{Nat}
\DeclareMathOperator{\gr}{Gr}
\DeclareMathOperator{\vect}{Vect}
\DeclareMathOperator{\id}{id}
\DeclareMathOperator{\Mod}{Mod}
\DeclareMathOperator{\sign}{sign}
\DeclareMathOperator{\Surf}{Surf}
\DeclareMathOperator{\fcone}{fcone}
\DeclareMathOperator{\Rot}{Rot}
\DeclareMathOperator{\grad}{grad}
\DeclareMathOperator{\atan2}{atan2}
\DeclareMathOperator{\Ric}{Ric}
\let\vec\relax
\DeclareMathOperator{\vec}{vec}
\let\Re\relax
\DeclareMathOperator{\Re}{Re}
\let\Im\relax
\DeclareMathOperator{\Im}{Im}
% Put x \to \infty below \lim
\let\svlim\lim\def\lim{\svlim\limits}

%wide hat
\usepackage{scalerel,stackengine}
\stackMath
\newcommand*\wh[1]{%
\savestack{\tmpbox}{\stretchto{%
  \scaleto{%
    \scalerel*[\widthof{\ensuremath{#1}}]{\kern-.6pt\bigwedge\kern-.6pt}%
    {\rule[-\textheight/2]{1ex}{\textheight}}%WIDTH-LIMITED BIG WEDGE
  }{\textheight}% 
}{0.5ex}}%
\stackon[1pt]{#1}{\tmpbox}%
}
\parskip 1ex

%Make implies and impliedby shorter
\let\implies\Rightarrow
\let\impliedby\Leftarrow
\let\iff\Leftrightarrow
\let\epsilon\varepsilon

% Add \contra symbol to denote contradiction
\usepackage{stmaryrd} % for \lightning
\newcommand\contra{\scalebox{1.5}{$\lightning$}}

% \let\phi\varphi

% Command for short corrections
% Usage: 1+1=\correct{3}{2}

\definecolor{correct}{HTML}{009900}
\newcommand\correct[2]{\ensuremath{\:}{\color{red}{#1}}\ensuremath{\to }{\color{correct}{#2}}\ensuremath{\:}}
\newcommand\green[1]{{\color{correct}{#1}}}

% horizontal rule
\newcommand\hr{
    \noindent\rule[0.5ex]{\linewidth}{0.5pt}
}

% hide parts
\newcommand\hide[1]{}

% si unitx
\usepackage{siunitx}
\sisetup{locale = FR}

%allows pmatrix to stretch
\makeatletter
\renewcommand*\env@matrix[1][\arraystretch]{%
  \edef\arraystretch{#1}%
  \hskip -\arraycolsep
  \let\@ifnextchar\new@ifnextchar
  \array{*\c@MaxMatrixCols c}}
\makeatother

\renewcommand{\arraystretch}{0.8}

\renewcommand{\baselinestretch}{1.5}

\usepackage{graphics}
\usepackage{epstopdf}

\RequirePackage{hyperref}
%%
%% Add support for color in order to color the hyperlinks.
%% 
\hypersetup{
  colorlinks = true,
  urlcolor = blue,
  citecolor = blue
}
%%fakesection Links
\hypersetup{
    colorlinks,
    linkcolor={red!50!black},
    citecolor={green!50!black},
    urlcolor={blue!80!black}
}
%customization of cleveref
\RequirePackage[capitalize,nameinlink]{cleveref}[0.19]

% Per SIAM Style Manual, "section" should be lowercase
\crefname{section}{section}{sections}
\crefname{subsection}{subsection}{subsections}
\Crefname{section}{Section}{Sections}
\Crefname{subsection}{Subsection}{Subsections}

% Per SIAM Style Manual, "Figure" should be spelled out in references
\Crefname{figure}{Figure}{Figures}

% Per SIAM Style Manual, don't say equation in front on an equation.
\crefformat{equation}{\textup{#2(#1)#3}}
\crefrangeformat{equation}{\textup{#3(#1)#4--#5(#2)#6}}
\crefmultiformat{equation}{\textup{#2(#1)#3}}{ and \textup{#2(#1)#3}}
{, \textup{#2(#1)#3}}{, and \textup{#2(#1)#3}}
\crefrangemultiformat{equation}{\textup{#3(#1)#4--#5(#2)#6}}%
{ and \textup{#3(#1)#4--#5(#2)#6}}{, \textup{#3(#1)#4--#5(#2)#6}}{, and \textup{#3(#1)#4--#5(#2)#6}}

% But spell it out at the beginning of a sentence.
\Crefformat{equation}{#2Equation~\textup{(#1)}#3}
\Crefrangeformat{equation}{Equations~\textup{#3(#1)#4--#5(#2)#6}}
\Crefmultiformat{equation}{Equations~\textup{#2(#1)#3}}{ and \textup{#2(#1)#3}}
{, \textup{#2(#1)#3}}{, and \textup{#2(#1)#3}}
\Crefrangemultiformat{equation}{Equations~\textup{#3(#1)#4--#5(#2)#6}}%
{ and \textup{#3(#1)#4--#5(#2)#6}}{, \textup{#3(#1)#4--#5(#2)#6}}{, and \textup{#3(#1)#4--#5(#2)#6}}

% Make number non-italic in any environment.
\crefdefaultlabelformat{#2\textup{#1}#3}

% Environments
\makeatother
% For box around Definition, Theorem, \ldots
%%fakesection Theorems
\usepackage{thmtools}
\usepackage[framemethod=TikZ]{mdframed}

\theoremstyle{definition}
\mdfdefinestyle{mdbluebox}{%
	roundcorner = 10pt,
	linewidth=1pt,
	skipabove=12pt,
	innerbottommargin=9pt,
	skipbelow=2pt,
	nobreak=true,
	linecolor=blue,
	backgroundcolor=TealBlue!5,
}
\declaretheoremstyle[
	headfont=\sffamily\bfseries\color{MidnightBlue},
	mdframed={style=mdbluebox},
	headpunct={\\[3pt]},
	postheadspace={0pt}
]{thmbluebox}

\mdfdefinestyle{mdredbox}{%
	linewidth=0.5pt,
	skipabove=12pt,
	frametitleaboveskip=5pt,
	frametitlebelowskip=0pt,
	skipbelow=2pt,
	frametitlefont=\bfseries,
	innertopmargin=4pt,
	innerbottommargin=8pt,
	nobreak=false,
	linecolor=RawSienna,
	backgroundcolor=Salmon!5,
}
\declaretheoremstyle[
	headfont=\bfseries\color{RawSienna},
	mdframed={style=mdredbox},
	headpunct={\\[3pt]},
	postheadspace={0pt},
]{thmredbox}

\declaretheorem[%
style=thmbluebox,name=Theorem,numberwithin=section]{thm}
\declaretheorem[style=thmbluebox,name=Lemma,sibling=thm]{lem}
\declaretheorem[style=thmbluebox,name=Proposition,sibling=thm]{prop}
\declaretheorem[style=thmbluebox,name=Corollary,sibling=thm]{coro}
\declaretheorem[style=thmredbox,name=Example,sibling=thm]{eg}

\mdfdefinestyle{mdgreenbox}{%
	roundcorner = 10pt,
	linewidth=1pt,
	skipabove=12pt,
	innerbottommargin=9pt,
	skipbelow=2pt,
	nobreak=true,
	linecolor=ForestGreen,
	backgroundcolor=ForestGreen!5,
}

\declaretheoremstyle[
	headfont=\bfseries\sffamily\color{ForestGreen!70!black},
	bodyfont=\normalfont,
	spaceabove=2pt,
	spacebelow=1pt,
	mdframed={style=mdgreenbox},
	headpunct={ --- },
]{thmgreenbox}

\declaretheorem[style=thmgreenbox,name=Definition,sibling=thm]{defn}

\mdfdefinestyle{mdgreenboxsq}{%
	linewidth=1pt,
	skipabove=12pt,
	innerbottommargin=9pt,
	skipbelow=2pt,
	nobreak=true,
	linecolor=ForestGreen,
	backgroundcolor=ForestGreen!5,
}
\declaretheoremstyle[
	headfont=\bfseries\sffamily\color{ForestGreen!70!black},
	bodyfont=\normalfont,
	spaceabove=2pt,
	spacebelow=1pt,
	mdframed={style=mdgreenboxsq},
	headpunct={},
]{thmgreenboxsq}
\declaretheoremstyle[
	headfont=\bfseries\sffamily\color{ForestGreen!70!black},
	bodyfont=\normalfont,
	spaceabove=2pt,
	spacebelow=1pt,
	mdframed={style=mdgreenboxsq},
	headpunct={},
]{thmgreenboxsq*}

\mdfdefinestyle{mdblackbox}{%
	skipabove=8pt,
	linewidth=3pt,
	rightline=false,
	leftline=true,
	topline=false,
	bottomline=false,
	linecolor=black,
	backgroundcolor=RedViolet!5!gray!5,
}
\declaretheoremstyle[
	headfont=\bfseries,
	bodyfont=\normalfont\small,
	spaceabove=0pt,
	spacebelow=0pt,
	mdframed={style=mdblackbox}
]{thmblackbox}

\theoremstyle{plain}
\declaretheorem[name=Question,sibling=thm,style=thmblackbox]{ques}
\declaretheorem[name=Remark,sibling=thm,style=thmgreenboxsq]{remark}
\declaretheorem[name=Remark,sibling=thm,style=thmgreenboxsq*]{remark*}
\newtheorem{ass}[thm]{Assumptions}

\theoremstyle{definition}
\newtheorem*{problem}{Problem}
\newtheorem{claim}[thm]{Claim}
\theoremstyle{remark}
\newtheorem*{case}{Case}
\newtheorem*{notation}{Notation}
\newtheorem*{note}{Note}
\newtheorem*{motivation}{Motivation}
\newtheorem*{intuition}{Intuition}
\newtheorem*{conjecture}{Conjecture}

% Make section starts with 1 for report type
%\renewcommand\thesection{\arabic{section}}

% End example and intermezzo environments with a small diamond (just like proof
% environments end with a small square)
\usepackage{etoolbox}
\AtEndEnvironment{vb}{\null\hfill$\diamond$}%
\AtEndEnvironment{intermezzo}{\null\hfill$\diamond$}%
% \AtEndEnvironment{opmerking}{\null\hfill$\diamond$}%

% Fix some spacing
% http://tex.stackexchange.com/questions/22119/how-can-i-change-the-spacing-before-theorems-with-amsthm
\makeatletter
\def\thm@space@setup{%
  \thm@preskip=\parskip \thm@postskip=0pt
}

% Fix some stuff
% %http://tex.stackexchange.com/questions/76273/multiple-pdfs-with-page-group-included-in-a-single-page-warning
\pdfsuppresswarningpagegroup=1


% My name
\author{Jaden Wang}



\begin{document}
\centerline {\textsf{\textbf{\LARGE{Homework 3}}}}
\centerline {Jaden Wang}
\vspace{.15in}

\begin{problem}[LN12 0.1.1]
Show that the antipodal reflection $ a: S^{n} \to S^{n}$, $ a(x) = -x$ is an isometry. 
\end{problem}
\begin{proof}
The antipodal reflection is clearly smooth and has itself as the smooth inverse, and therefore is a diffeomorphism. For any $ p \in S^{n}, v \in T_pS^{n}$, and any smooth curve $ \gamma: (- \epsilon, \epsilon) \to S^{n}$ s.t.\ $ \gamma(0) =p, \gamma'(0) = v$, its derivative is
\begin{align*}
	da_p(v) &= \left( a \circ \gamma \right) '(0)\\
	&= (-\gamma)'(0) \\
	&= - \gamma'(0) \\
	&= - v .
\end{align*}
For any Riemannian metric $ g$ on  $ S^{n}$ and any $ p \in S^{n}$, we have
\begin{align*}
	g_p(da_p(v),da_p(w)) &= g_p(-v,-w) \\
			     &= -g_p(v,-w) && \text{bilinearity}  \\
	&= g_p(v,w) ,
\end{align*}
which proves that it is an isometry.
\end{proof}

\begin{problem}[LN12 0.2.1]
Show that inversion $ i: \rr^{n}\setminus  \{0\} \to \rr^{n} $ given by $ i(x) = \frac{x}{\norm{ x} ^2}$ is a conformal transformation.
\end{problem}
\begin{proof}
Let $ M = \rr^{n}\setminus \{0\} $. Given $ p \in M, v \in T_pM $, and $ \gamma: (- \epsilon, \epsilon) \to M $ s.t.\ $ \gamma(0) = p$ and $ \gamma'(0) = v$, we compute the derivative at $ p$:
\begin{align*}
	di_p(v) &= (i \circ \gamma)'(0)\\
		&= \left( \frac{\gamma}{ \norm{ \gamma}^2 } \right)'(0)\\
		&= \left( \frac{ \gamma' \norm{ \gamma}^2 - 2 \gamma \langle \gamma, \gamma' \rangle}{ \norm{ \gamma}^{4}  } \right)(0) && \text{quotient rule}  \\
		&= \frac{\norm{ p}^2 v - 2 p \langle p,v \rangle}{ \norm{ p}^{4}  } .
\end{align*}
 Given $ v,w \in T_pM$, we have
 \begin{align*}
	 g_p ( di_p(v),di_p(w) )&= \langle \frac{\norm{ p}^2 v - 2p \langle p,v \rangle}{ \norm{ p}^{4}  }, \frac{\norm{ p}^2 w - 2 p \langle p,w \rangle }{ \norm{ p}^{4}  }\rangle \\
	 &=  \frac{ \langle v,w \rangle}{\norm{ p}^4 } - \frac{4 v^{T}pp ^{T} w}{ \norm{ p}^{6} } + \frac{4 v^{T}p(p ^{T}p)p ^{T}w}{ \norm{ p}^{8} }\\
	 &=  \frac{ \langle v,w \rangle}{\norm{ p}^4 } - \frac{4 v^{T}pp ^{T} w}{ \norm{ p}^{6} } + \frac{4 v^{T}pp ^{T}w}{ \norm{ p}^{6} }\\
	 &= \frac{\langle v,w \rangle}{ \norm{ p}^{4} } .
 \end{align*}

Then the angle $ \theta(di_p(v),di_p(w))$ between $ di_p(v),di_p(w)$ is
\begin{align*}
	\theta(di_p(v),di_p(w)) &= \arccos \left( \frac{g_p ( di_p(v),di_p(w) ) }{ g_p(di_p(v),di_p(v))^{\frac{1}{2}} g_p(di_p(w),di_p(w))^{\frac{1}{2}}} \right) \\
	&= \arccos \left( \frac{ \frac{\langle v,w \rangle}{ \norm{ p} ^{4}}}{ \frac{\norm{ v} \norm{ w}  }{ \norm{ p}^{4} }  } \right)  \\
	&= \arccos \left( \frac{\langle v,w \rangle}{ \norm{ v} \norm{ w} } \right)  \\
	&= \arccos \left( \frac{g_p(v,w)}{ g_p(v,v)^{\frac{1}{2}} g_p(w,w)^{\frac{1}{2}}} \right)  \\
	&= \theta(v,w) .
\end{align*}
That is, $ i$ is conformal.
\end{proof}

\begin{problem}[LN12 0.2.3]
Show that the Poincar\'e half-plane and disk are isometric.
\end{problem}
\begin{proof}
Translating the half-plane up by 1 to get $ H = \{(x,y) \in \rr^2:y >1\} $ is clearly an isometry. Let $ D = \{(x,y): x^2+(y-0.5)^2 < 0.25 \} $ be the open disk centered at $ (0,0.5)$ with radius  $ 0.5$. Inversion is a composition of smooth maps and thus smooth. First we show that the inversion map is a dffeomorphism between $ H$ and $ D$. Given  $ p=(x,y) \in H$, observe
\begin{align*}
	\norm{ \frac{p}{\norm{ p}^2 } - \begin{pmatrix} 0\\ \frac{1}{2} \end{pmatrix} }^2 &= \norm{ \begin{pmatrix} \frac{x}{x^2+y^2} \\ \frac{2y - (x^2+y^2)}{ 2(x^2+y^2)} \end{pmatrix} } ^2 \\ 
	&= \frac{4x^2 + 4y^2 -4(x^2+y^2)y +(x^2+y^2)^2}{ 4(x^2+y^2)^2 } \\
	&= \frac{1}{4} + 4(x^2+y^2)(1-y) \\
	&< \frac{1}{4}  
\end{align*}
since $ y>1$. Thus the restricted inversion $ i: H \to D$ is well-defined and remains smooth. I claim that its inverse is itself, which we name $ j:D \to H$ due to differences in domain and codomain. We check $ j$ is well-defined: if we have $ p=(x,y) \in D$, then
\begin{align*}
	x^2 + (y-0.5)^2 = x^2+y^2-y + \frac{1}{4} &< \frac{1}{4}\\
	y &> x^2 + y^2\\
	\frac{y}{x^2+y^2} &> 1 && p \neq 0\\
	y_{j(p)} & >1.
\end{align*}
Since it is clear that $ i \circ j(p) = j \circ i (p) = p $, inversion is a diffeomorphism. Now endow $ H$ and $ D$ with the modified metrics $ g$ and $ h$ respectively, where $ g_p(v,w) = \frac{\langle v,w \rangle}{ (y-1)^2}$ and $ h_p(v,w) = \frac{\langle v,w \rangle}{ (0.25 - \norm{ p-(0,0.5)}^2 )^2}$ due to the translation and scaling of the plane and disk. We observe
\begin{align*}
	h_{i(p)}(di_p(v),di_p(w)) &= \frac{\langle di_p(v),di_p(w) \rangle}{ (0.25-\norm{i(p) - (0,0.5)}^2)^2} \\
	&= \frac{\langle v,w \rangle}{ \norm{ p}^{4} (0.25-\norm{ \frac{p}{\norm{ p}^2 } - (0,0.5)}^2 )^2 } \\
&= \frac{\langle v,w \rangle}{ \norm{ p}^{4} \left( 0.25- \frac{1-y}{\norm{ p}^2} -0.25 \right) ^2 } \\
	&= \frac{\langle v,w \rangle}{ (y-1)^2} \\
	&= g_p(v,w) .
\end{align*}
Hence $ (H,g)$ and  $ (D,h)$ are isometric.
\end{proof}

\begin{problem}[LN12 0.3.1]
Compute the metric of $ S^2$ in terms of spherical coordinates $ \theta$ and $ \phi$.
\end{problem}
The parametric equation is $ f(\theta,\phi) = (\cos \theta \sin \phi, \sin \theta \sin \phi, \cos \phi)$. Then
\begin{align*}
	\frac{\partial f}{\partial \theta} &= (-\sin \theta \sin \phi, \cos \theta \sin \phi, 0) \\
	\frac{\partial f}{\partial \phi} &= (\cos \theta \cos \phi, \sin \theta \cos \phi, -\sin \phi) . 
\end{align*}
Since $ S^2$ is endowed with the ambience Euclidean metric $ \langle \cdot , \cdot  \rangle$, the pullback metric on the parameter space is $ g_{ij} = \langle \frac{\partial f}{\partial x_i}, \frac{\partial f}{\partial x_j}  \rangle$, which is
\begin{align*}
	G(\theta,\phi) &= \begin{pmatrix} \sin^2 \theta \sin^2\phi + \cos^2\theta \sin^2\phi & 0\\ 0& \cos^2\theta\cos^2\theta+\sin^2\theta\cos^2\phi+\sin^2\phi \end{pmatrix} \\
	  &= \begin{pmatrix} \sin^2\phi&0\\0& 1 \end{pmatrix}   .
\end{align*}

\begin{problem}[LN12 0.4.1]
Compute the length of the radius of the Poincar\'e disk (with respect to the Poincar\'e metric).
\end{problem}
Consider the unit open disk $ D$ with the metric $ g_p(v,w) = \frac{\langle v,w \rangle}{ (1-\norm{ p}^2 )^2}$ and the curve $ \gamma: [0,1) \to D, t\mapsto (0,t)$ which traces out the radius. Then
\begin{align*}
	L[ \gamma] &= \int_{ 0}^{ 1} g_{ \gamma(t)}( \gamma'(t), \gamma'(t))^{\frac{1}{2}} dt\\ 
	&= \int_{ 0}^{ 1} g_{ \gamma(t)}((0,1),(0,1))^{\frac{1}{2}} dt\\
	&= \int_{ 0}^{ 1} \frac{1}{ 1- \norm{ (0,t)}^2 } dt \\
	&= \int_{ 0}^{ 1} \frac{1}{1-t^2 } dt \\
	&= \int_{ 0}^{ 1} \frac{dt}{1-t} + \int_{ 0}^{ 1} \frac{dt}{ 1+t}   \\
	&= (\ln |1-t| + \ln |1+t|)\bigg|_0^1 ,
\end{align*}
which diverges. Thus loosely speaking, the radius is infinite. 
\end{document}
