\documentclass[12pt]{article}
\newcommand{\alert}[1]{{\bf \color{red} [Alert:] #1}}
\newcommand{\todo}[1]{{\bf \color{orange} [TODO:] #1}}
\newcommand{\real}[1][]{\mathbb{R}^{#1}}
\newcommand{\myeqn}[1]{(\ref{#1})}
\newcommand{\myex}[1]{Example \ref{#1}}
\newcommand{\defeq}{\stackrel{\mathrm{def}}{=}}
\newcommand{\parder}[2]{\frac{\partial #1}{\partial #2}}
\newcommand{\Lie}[3][]{\mathsf{L}_{#3}^{#1} #2}
\newcommand{\LieA}[1]{\mathsf{Lie}(#1)}
\newcommand{\lieder}[2]{\mathcal{L}_{#2} #1}
\renewcommand{\t}{^{\mbox{\tiny\sf T}}}
\newcommand{\trans}{^{\mbox{\tiny\sf T}}}
\newcommand{\markup}[1]{\{\textbf{#1}\}}
\newcommand{\msub}[1]{_\mathrm{#1}}
\newcommand{\msup}[1]{^\mathrm{#1}}
\newcommand{\inv}[1]{#1^{-1}}
\newcommand{\pinv}[1]{{#1}^{+}}
\newcommand{\myfracA}[2]{\displaystyle{\frac{#1}{#2}}}
\newcommand{\myfracB}[2]{{#1}/{#2}}
\newcommand{\mydiffA}[1]{\dot{#1}}
\newcommand{\mydiffB}[2]{\myfracA{\mathrm{d}{#1}}{\mathrm{d}{#2}}}
\newcommand{\ball}[2]{\mathcal{B}_{#1}\left(#2\right)}
\newcommand{\acos}[1]{\cos^{-1}\left(#1\right)}
\newcommand{\asin}[1]{\sin^{-1}\left(#1\right)}
\newcommand{\mani}{\mathcal{M}}
\newcommand{\tang}[2]{\mathsf{T}_{#1} #2}
\newcommand{\LieB}[2]{[ #1, #2 ]}
\newcommand{\LieBad}[3][]{\mathsf{ad}_{#2}^{#1} #3}
\newcommand{\ReachVT}{\mathcal{R}^V_T}
\newcommand{\ReachVt}{\mathcal{R}^V_t}
\newcommand{\ReachVTe}{\mathcal{R}^V_{\le T}}
\newcommand{\ReachT}{\mathcal{R}_T}
\newcommand{\Reacht}{\mathcal{R}_t}
\newcommand{\ReachTe}{\mathcal{R}_{\le T}}
\newcommand{\accLA}[1]{\mathsf{Lie}(#1)}
\newcommand{\accD}{\Delta_{\mathcal{F}}}
\newcommand{\accSA}{\mathsf{Lie}(\mathcal{G},f)}
\newcommand{\accDS}{\Delta_{\mathcal{G}}}
\newcommand{\eval}[3]{\mathsf{Ev}^{#2}_{#1}\left( #3 \right)}
\newcommand{\stlc}{\textsc{stlc}}
\newcommand{\clf}{\textsc{clf}}
\newcommand{\jqlf}{\textsc{jqlf}}
\newcommand{\dlas}{\textsc{dlas}}
\newcommand{\Ad}[2]{\mathsf{Ad}_{#1} #2}
\newcommand{\xe}{\ensuremath{x_e}}
\newcommand{\lebg}[1]{\mathcal{L}_{#1}}
\newcommand{\lebgx}[1]{\mathcal{L}_{#1 \mathrm{e}}}
\newcommand{\dom}{D}
\newcommand{\domT}{[t_0,\infty) \times D}
\newcommand{\rarrow}{\rightarrow}
\renewcommand{\d}{\mathrm{d}}
\renewcommand{\Re}{\mathbb{R}}
\newcommand{\C}{\mathrm{C}}

\newcommand{\QED}{{\unskip\nobreak\hfil\penalty50\hskip2em\vadjust{}
		\nobreak\hfil$\Box$\parfillskip=0pt\finalhyphendemerits=0\par}\vspace{0.1cm}}
\newcommand{\eoEx}{{\unskip\nobreak\hfil\penalty50\hskip0em\vadjust{}
		\nobreak\hfil$\Large\Diamond$\parfillskip=0pt\finalhyphendemerits=0\par}\vspace{0.1cm}}

\newcommand{\sgn}{\ensuremath{\operatorname{sgn}}}
\newcommand{\sat}{\ensuremath{\operatorname{sat}}}

\newcommand{\half}{\frac{1}{2}}
\newcommand{\shalf}{\mbox{$\frac{1}{2}$}}
\newcommand{\marcom}[1]{\marginpar{\footnotesize #1}}
\newcommand{\der}{\mathrm{D}}
\newcommand{\e}{\mathrm{e}}
\newcommand{\dt}{\mathrm{d}t}

\newcommand{\cA}{\ensuremath{\mathcal{A}}}
\newcommand{\cB}{\ensuremath{\mathcal{B}}}
\newcommand{\cG}{\ensuremath{\mathcal{G}}}
\newcommand{\cK}{\ensuremath{\mathcal{K}}}
\newcommand{\cW}{\ensuremath{\mathcal{W}}}
\newcommand{\cZ}{\ensuremath{\mathcal{Z}}}
\newcommand{\cS}{\ensuremath{\mathcal{S}}}
\newcommand{\cD}{\ensuremath{\mathcal{D}}}
\newcommand{\cP}{\ensuremath{\mathcal{P}}}
\newcommand{\cV}{\ensuremath{\mathcal{V}}}
\newcommand{\cL}{\ensuremath{\mathcal{L}}}
\newcommand{\cN}{\ensuremath{\mathcal{N}}}
\newcommand{\cI}{\ensuremath{\mathcal{I}}}
\newcommand{\cR}{\ensuremath{\mathcal{R}}}
\newcommand{\cM}{\ensuremath{\mathcal{M}}}
\newcommand{\cC}{\ensuremath{\mathcal{C}}}
\newcommand{\cF}{\ensuremath{\mathcal{F}}}
\newcommand{\cH}{\ensuremath{\mathcal{H}}}
\newcommand{\cO}{\ensuremath{\mathcal{O}}}
\newcommand{\cX}{\ensuremath{\mathcal{X}}}
\newcommand{\cY}{\ensuremath{\mathcal{Y}}}
\newcommand{\Ci}{\ensuremath{\mathcal{C}^\infty}}
\newcommand{\ISS}{\textsc{iss}}
\newcommand{\LISS}{\textsc{liss}}
\newcommand{\GAS}{\textsc{gas}}
\newcommand{\GS}{\textsc{gs}}
\newcommand{\LES}{\textsc{les}}
\newcommand{\GUAS}{\textsc{guas}}
\newcommand{\BIBO}{\textsc{bibo}}
\newcommand{\spec}{\ensuremath{\operatorname{spec}}}
\newcommand{\spn}{\ensuremath{\operatorname{span}}}
\renewcommand{\i}{\mathrm{i\,}}

\renewcommand{\implies}{\Rightarrow}

\renewcommand{\theenumi}{$\roman{enumi})$}
\renewcommand{\labelenumi}{\theenumi}

\font\ptmten=zptmcmrm scaled 1200
\newcommand{\w}{\mbox{{\ptmten w}}}
\newcommand{\z}{\mbox{{\ptmten z}}}
\renewcommand{\Re}{\mathbb{R}}

\newcommand{\cl}{\operatorname{cl}}
\newcommand{\intr}{\operatorname{int}}
\newcommand{\rank}{\operatorname{rank}}
\newcommand{\co}{\operatorname{co}}
\newcommand{\aff}{\operatorname{aff}}

\theoremstyle{plain}
\newtheorem{theorem}{Theorem}[chapter]
\newtheorem{claim}[theorem]{Claim}
\newtheorem{corollary}[theorem]{Corollary}
\newtheorem{prop}[theorem]{Proposition}
\newtheorem{fact}[theorem]{Fact}
\newtheorem{lemma}[theorem]{Lemma}

\newtheorem{remark}{Remark}[chapter]

\theoremstyle{definition}
\newtheorem{assume}[theorem]{Assumption}
\newtheorem{defn}[theorem]{Definition}
\newtheorem{problem}[theorem]{Problem}
\newtheorem{exercise}{Exercise}
\newtheorem{example}[theorem]{Example}


\begin{document}
\centerline {\textsf{\textbf{\LARGE{Homework 3}}}}
\centerline {Jaden Wang}
\vspace{.15in}

\begin{problem}[LN12 0.1.1]
Show that the antipodal reflection $ a: S^{n} \to S^{n}$, $ a(x) = -x$ is an isometry. 
\end{problem}
\begin{proof}
The antipodal reflection is clearly smooth and has itself as the smooth inverse, and therefore is a diffeomorphism. For any $ p \in S^{n}, v \in T_pS^{n}$, and any smooth curve $ \gamma: (- \epsilon, \epsilon) \to S^{n}$ s.t.\ $ \gamma(0) =p, \gamma'(0) = v$, its derivative is
\begin{align*}
	da_p(v) &= \left( a \circ \gamma \right) '(0)\\
	&= (-\gamma)'(0) \\
	&= - \gamma'(0) \\
	&= - v .
\end{align*}
For any Riemannian metric $ g$ on  $ S^{n}$ and any $ p \in S^{n}$, we have
\begin{align*}
	g_p(da_p(v),da_p(w)) &= g_p(-v,-w) \\
			     &= -g_p(v,-w) && \text{bilinearity}  \\
	&= g_p(v,w) ,
\end{align*}
which proves that it is an isometry.
\end{proof}

\begin{problem}[LN12 0.2.1]
Show that inversion $ i: \rr^{n}\setminus  \{0\} \to \rr^{n} $ given by $ i(x) = \frac{x}{\norm{ x} ^2}$ is a conformal transformation.
\end{problem}
\begin{proof}
Let $ M = \rr^{n}\setminus \{0\} $. Given $ p \in M, v \in T_pM $, and $ \gamma: (- \epsilon, \epsilon) \to M $ s.t.\ $ \gamma(0) = p$ and $ \gamma'(0) = v$, we compute the derivative at $ p$:
\begin{align*}
	di_p(v) &= (i \circ \gamma)'(0)\\
		&= \left( \frac{\gamma}{ \norm{ \gamma}^2 } \right)'(0)\\
		&= \left( \frac{ \gamma' \norm{ \gamma}^2 - 2 \gamma \langle \gamma, \gamma' \rangle}{ \norm{ \gamma}^{4}  } \right)(0) && \text{quotient rule}  \\
		&= \frac{\norm{ p}^2 v - 2 p \langle p,v \rangle}{ \norm{ p}^{4}  } .
\end{align*}
 Given $ v,w \in T_pM$, we have
 \begin{align*}
	 g_p ( di_p(v),di_p(w) )&= \langle \frac{\norm{ p}^2 v - 2p \langle p,v \rangle}{ \norm{ p}^{4}  }, \frac{\norm{ p}^2 w - 2 p \langle p,w \rangle }{ \norm{ p}^{4}  }\rangle \\
	 &=  \frac{ \langle v,w \rangle}{\norm{ p}^4 } - \frac{4 v^{T}pp ^{T} w}{ \norm{ p}^{6} } + \frac{4 v^{T}p(p ^{T}p)p ^{T}w}{ \norm{ p}^{8} }\\
	 &=  \frac{ \langle v,w \rangle}{\norm{ p}^4 } - \frac{4 v^{T}pp ^{T} w}{ \norm{ p}^{6} } + \frac{4 v^{T}pp ^{T}w}{ \norm{ p}^{6} }\\
	 &= \frac{\langle v,w \rangle}{ \norm{ p}^{4} } .
 \end{align*}

Then the angle $ \theta(di_p(v),di_p(w))$ between $ di_p(v),di_p(w)$ is
\begin{align*}
	\theta(di_p(v),di_p(w)) &= \arccos \left( \frac{g_p ( di_p(v),di_p(w) ) }{ g_p(di_p(v),di_p(v))^{\frac{1}{2}} g_p(di_p(w),di_p(w))^{\frac{1}{2}}} \right) \\
	&= \arccos \left( \frac{ \frac{\langle v,w \rangle}{ \norm{ p} ^{4}}}{ \frac{\norm{ v} \norm{ w}  }{ \norm{ p}^{4} }  } \right)  \\
	&= \arccos \left( \frac{\langle v,w \rangle}{ \norm{ v} \norm{ w} } \right)  \\
	&= \arccos \left( \frac{g_p(v,w)}{ g_p(v,v)^{\frac{1}{2}} g_p(w,w)^{\frac{1}{2}}} \right)  \\
	&= \theta(v,w) .
\end{align*}
That is, $ i$ is conformal.
\end{proof}

\begin{problem}[LN12 0.2.3]
Show that the Poincar\'e half-plane and disk are isometric.
\end{problem}
\begin{proof}
Translating the half-plane up by 1 to get $ H = \{(x,y) \in \rr^2:y >1\} $ is clearly an isometry. Let $ D = \{(x,y): x^2+(y-0.5)^2 < 0.25 \} $ be the open disk centered at $ (0,0.5)$ with radius  $ 0.5$. Inversion is a composition of smooth maps and thus smooth. First we show that the inversion map is a dffeomorphism between $ H$ and $ D$. Given  $ p=(x,y) \in H$, observe
\begin{align*}
	\norm{ \frac{p}{\norm{ p}^2 } - \begin{pmatrix} 0\\ \frac{1}{2} \end{pmatrix} }^2 &= \norm{ \begin{pmatrix} \frac{x}{x^2+y^2} \\ \frac{2y - (x^2+y^2)}{ 2(x^2+y^2)} \end{pmatrix} } ^2 \\ 
	&= \frac{4x^2 + 4y^2 -4(x^2+y^2)y +(x^2+y^2)^2}{ 4(x^2+y^2)^2 } \\
	&= \frac{1}{4} + 4(x^2+y^2)(1-y) \\
	&< \frac{1}{4}  
\end{align*}
since $ y>1$. Thus the restricted inversion $ i: H \to D$ is well-defined and remains smooth. I claim that its inverse is itself, which we name $ j:D \to H$ due to differences in domain and codomain. We check $ j$ is well-defined: if we have $ p=(x,y) \in D$, then
\begin{align*}
	x^2 + (y-0.5)^2 = x^2+y^2-y + \frac{1}{4} &< \frac{1}{4}\\
	y &> x^2 + y^2\\
	\frac{y}{x^2+y^2} &> 1 && p \neq 0\\
	y_{j(p)} & >1.
\end{align*}
Since it is clear that $ i \circ j(p) = j \circ i (p) = p $, inversion is a diffeomorphism. Now endow $ H$ and $ D$ with the modified metrics $ g$ and $ h$ respectively, where $ g_p(v,w) = \frac{\langle v,w \rangle}{ (y-1)^2}$ and $ h_p(v,w) = \frac{\langle v,w \rangle}{ (0.25 - \norm{ p-(0,0.5)}^2 )^2}$ due to the translation and scaling of the plane and disk. We observe
\begin{align*}
	h_{i(p)}(di_p(v),di_p(w)) &= \frac{\langle di_p(v),di_p(w) \rangle}{ (0.25-\norm{i(p) - (0,0.5)}^2)^2} \\
	&= \frac{\langle v,w \rangle}{ \norm{ p}^{4} (0.25-\norm{ \frac{p}{\norm{ p}^2 } - (0,0.5)}^2 )^2 } \\
&= \frac{\langle v,w \rangle}{ \norm{ p}^{4} \left( 0.25- \frac{1-y}{\norm{ p}^2} -0.25 \right) ^2 } \\
	&= \frac{\langle v,w \rangle}{ (y-1)^2} \\
	&= g_p(v,w) .
\end{align*}
Hence $ (H,g)$ and  $ (D,h)$ are isometric.
\end{proof}

\begin{problem}[LN12 0.3.1]
Compute the metric of $ S^2$ in terms of spherical coordinates $ \theta$ and $ \phi$.
\end{problem}
The parametric equation is $ f(\theta,\phi) = (\cos \theta \sin \phi, \sin \theta \sin \phi, \cos \phi)$. Then
\begin{align*}
	\frac{\partial f}{\partial \theta} &= (-\sin \theta \sin \phi, \cos \theta \sin \phi, 0) \\
	\frac{\partial f}{\partial \phi} &= (\cos \theta \cos \phi, \sin \theta \cos \phi, -\sin \phi) . 
\end{align*}
Since $ S^2$ is endowed with the ambience Euclidean metric $ \langle \cdot , \cdot  \rangle$, the pullback metric on the parameter space is $ g_{ij} = \langle \frac{\partial f}{\partial x_i}, \frac{\partial f}{\partial x_j}  \rangle$, which is
\begin{align*}
	G(\theta,\phi) &= \begin{pmatrix} \sin^2 \theta \sin^2\phi + \cos^2\theta \sin^2\phi & 0\\ 0& \cos^2\theta\cos^2\theta+\sin^2\theta\cos^2\phi+\sin^2\phi \end{pmatrix} \\
	  &= \begin{pmatrix} \sin^2\phi&0\\0& 1 \end{pmatrix}   .
\end{align*}

\begin{problem}[LN12 0.4.1]
Compute the length of the radius of the Poincar\'e disk (with respect to the Poincar\'e metric).
\end{problem}
Consider the unit open disk $ D$ with the metric $ g_p(v,w) = \frac{\langle v,w \rangle}{ (1-\norm{ p}^2 )^2}$ and the curve $ \gamma: [0,1) \to D, t\mapsto (0,t)$ which traces out the radius. Then
\begin{align*}
	L[ \gamma] &= \int_{ 0}^{ 1} g_{ \gamma(t)}( \gamma'(t), \gamma'(t))^{\frac{1}{2}} dt\\ 
	&= \int_{ 0}^{ 1} g_{ \gamma(t)}((0,1),(0,1))^{\frac{1}{2}} dt\\
	&= \int_{ 0}^{ 1} \frac{1}{ 1- \norm{ (0,t)}^2 } dt \\
	&= \int_{ 0}^{ 1} \frac{1}{1-t^2 } dt \\
	&= \int_{ 0}^{ 1} \frac{dt}{1-t} + \int_{ 0}^{ 1} \frac{dt}{ 1+t}   \\
	&= (\ln |1-t| + \ln |1+t|)\bigg|_0^1 ,
\end{align*}
which diverges. Thus loosely speaking, the radius is infinite. 
\end{document}
