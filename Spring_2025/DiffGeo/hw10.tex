\documentclass[12pt]{article}
%Fall 2022
% Some basic packages
\usepackage{standalone}[subpreambles=true]
\usepackage[utf8]{inputenc}
\usepackage[T1]{fontenc}
\usepackage{textcomp}
\usepackage[english]{babel}
\usepackage{url}
\usepackage{graphicx}
%\usepackage{quiver}
\usepackage{float}
\usepackage{enumitem}
\usepackage{lmodern}
\usepackage{comment}
\usepackage{hyperref}
\usepackage[usenames,svgnames,dvipsnames]{xcolor}
\usepackage[margin=1in]{geometry}
\usepackage{pdfpages}

\pdfminorversion=7

% Don't indent paragraphs, leave some space between them
\usepackage{parskip}

% Hide page number when page is empty
\usepackage{emptypage}
\usepackage{subcaption}
\usepackage{multicol}
\usepackage[b]{esvect}

% Math stuff
\usepackage{amsmath, amsfonts, mathtools, amsthm, amssymb}
\usepackage{bbm}
\usepackage{stmaryrd}
\allowdisplaybreaks

% Fancy script capitals
\usepackage{mathrsfs}
\usepackage{cancel}
% Bold math
\usepackage{bm}
% Some shortcuts
\newcommand{\rr}{\ensuremath{\mathbb{R}}}
\newcommand{\zz}{\ensuremath{\mathbb{Z}}}
\newcommand{\qq}{\ensuremath{\mathbb{Q}}}
\newcommand{\nn}{\ensuremath{\mathbb{N}}}
\newcommand{\ff}{\ensuremath{\mathbb{F}}}
\newcommand{\cc}{\ensuremath{\mathbb{C}}}
\newcommand{\ee}{\ensuremath{\mathbb{E}}}
\newcommand{\hh}{\ensuremath{\mathbb{H}}}
\renewcommand\O{\ensuremath{\emptyset}}
\newcommand{\norm}[1]{{\left\lVert{#1}\right\rVert}}
\newcommand{\dbracket}[1]{{\left\llbracket{#1}\right\rrbracket}}
\newcommand{\ve}[1]{{\bm{#1}}}
\newcommand\allbold[1]{{\boldmath\textbf{#1}}}
\DeclareMathOperator{\lcm}{lcm}
\DeclareMathOperator{\im}{im}
\DeclareMathOperator{\coim}{coim}
\DeclareMathOperator{\dom}{dom}
\DeclareMathOperator{\tr}{tr}
\DeclareMathOperator{\rank}{rank}
\DeclareMathOperator*{\var}{Var}
\DeclareMathOperator*{\ev}{E}
\DeclareMathOperator{\dg}{deg}
\DeclareMathOperator{\aff}{aff}
\DeclareMathOperator{\conv}{conv}
\DeclareMathOperator{\inte}{int}
\DeclareMathOperator*{\argmin}{argmin}
\DeclareMathOperator*{\argmax}{argmax}
\DeclareMathOperator{\graph}{graph}
\DeclareMathOperator{\sgn}{sgn}
\DeclareMathOperator*{\Rep}{Rep}
\DeclareMathOperator{\Proj}{Proj}
\DeclareMathOperator{\mat}{mat}
\DeclareMathOperator{\diag}{diag}
\DeclareMathOperator{\aut}{Aut}
\DeclareMathOperator{\gal}{Gal}
\DeclareMathOperator{\inn}{Inn}
\DeclareMathOperator{\edm}{End}
\DeclareMathOperator{\Hom}{Hom}
\DeclareMathOperator{\ext}{Ext}
\DeclareMathOperator{\tor}{Tor}
\DeclareMathOperator{\Span}{Span}
\DeclareMathOperator{\Stab}{Stab}
\DeclareMathOperator{\cont}{cont}
\DeclareMathOperator{\Ann}{Ann}
\DeclareMathOperator{\Div}{div}
\DeclareMathOperator{\curl}{curl}
\DeclareMathOperator{\nat}{Nat}
\DeclareMathOperator{\gr}{Gr}
\DeclareMathOperator{\vect}{Vect}
\DeclareMathOperator{\id}{id}
\DeclareMathOperator{\Mod}{Mod}
\DeclareMathOperator{\sign}{sign}
\DeclareMathOperator{\Surf}{Surf}
\DeclareMathOperator{\fcone}{fcone}
\DeclareMathOperator{\Rot}{Rot}
\DeclareMathOperator{\grad}{grad}
\DeclareMathOperator{\atan2}{atan2}
\DeclareMathOperator{\Ric}{Ric}
\let\vec\relax
\DeclareMathOperator{\vec}{vec}
\let\Re\relax
\DeclareMathOperator{\Re}{Re}
\let\Im\relax
\DeclareMathOperator{\Im}{Im}
% Put x \to \infty below \lim
\let\svlim\lim\def\lim{\svlim\limits}

%wide hat
\usepackage{scalerel,stackengine}
\stackMath
\newcommand*\wh[1]{%
\savestack{\tmpbox}{\stretchto{%
  \scaleto{%
    \scalerel*[\widthof{\ensuremath{#1}}]{\kern-.6pt\bigwedge\kern-.6pt}%
    {\rule[-\textheight/2]{1ex}{\textheight}}%WIDTH-LIMITED BIG WEDGE
  }{\textheight}% 
}{0.5ex}}%
\stackon[1pt]{#1}{\tmpbox}%
}
\parskip 1ex

%Make implies and impliedby shorter
\let\implies\Rightarrow
\let\impliedby\Leftarrow
\let\iff\Leftrightarrow
\let\epsilon\varepsilon

% Add \contra symbol to denote contradiction
\usepackage{stmaryrd} % for \lightning
\newcommand\contra{\scalebox{1.5}{$\lightning$}}

% \let\phi\varphi

% Command for short corrections
% Usage: 1+1=\correct{3}{2}

\definecolor{correct}{HTML}{009900}
\newcommand\correct[2]{\ensuremath{\:}{\color{red}{#1}}\ensuremath{\to }{\color{correct}{#2}}\ensuremath{\:}}
\newcommand\green[1]{{\color{correct}{#1}}}

% horizontal rule
\newcommand\hr{
    \noindent\rule[0.5ex]{\linewidth}{0.5pt}
}

% hide parts
\newcommand\hide[1]{}

% si unitx
\usepackage{siunitx}
\sisetup{locale = FR}

%allows pmatrix to stretch
\makeatletter
\renewcommand*\env@matrix[1][\arraystretch]{%
  \edef\arraystretch{#1}%
  \hskip -\arraycolsep
  \let\@ifnextchar\new@ifnextchar
  \array{*\c@MaxMatrixCols c}}
\makeatother

\renewcommand{\arraystretch}{0.8}

\renewcommand{\baselinestretch}{1.5}

\usepackage{graphics}
\usepackage{epstopdf}

\RequirePackage{hyperref}
%%
%% Add support for color in order to color the hyperlinks.
%% 
\hypersetup{
  colorlinks = true,
  urlcolor = blue,
  citecolor = blue
}
%%fakesection Links
\hypersetup{
    colorlinks,
    linkcolor={red!50!black},
    citecolor={green!50!black},
    urlcolor={blue!80!black}
}
%customization of cleveref
\RequirePackage[capitalize,nameinlink]{cleveref}[0.19]

% Per SIAM Style Manual, "section" should be lowercase
\crefname{section}{section}{sections}
\crefname{subsection}{subsection}{subsections}
\Crefname{section}{Section}{Sections}
\Crefname{subsection}{Subsection}{Subsections}

% Per SIAM Style Manual, "Figure" should be spelled out in references
\Crefname{figure}{Figure}{Figures}

% Per SIAM Style Manual, don't say equation in front on an equation.
\crefformat{equation}{\textup{#2(#1)#3}}
\crefrangeformat{equation}{\textup{#3(#1)#4--#5(#2)#6}}
\crefmultiformat{equation}{\textup{#2(#1)#3}}{ and \textup{#2(#1)#3}}
{, \textup{#2(#1)#3}}{, and \textup{#2(#1)#3}}
\crefrangemultiformat{equation}{\textup{#3(#1)#4--#5(#2)#6}}%
{ and \textup{#3(#1)#4--#5(#2)#6}}{, \textup{#3(#1)#4--#5(#2)#6}}{, and \textup{#3(#1)#4--#5(#2)#6}}

% But spell it out at the beginning of a sentence.
\Crefformat{equation}{#2Equation~\textup{(#1)}#3}
\Crefrangeformat{equation}{Equations~\textup{#3(#1)#4--#5(#2)#6}}
\Crefmultiformat{equation}{Equations~\textup{#2(#1)#3}}{ and \textup{#2(#1)#3}}
{, \textup{#2(#1)#3}}{, and \textup{#2(#1)#3}}
\Crefrangemultiformat{equation}{Equations~\textup{#3(#1)#4--#5(#2)#6}}%
{ and \textup{#3(#1)#4--#5(#2)#6}}{, \textup{#3(#1)#4--#5(#2)#6}}{, and \textup{#3(#1)#4--#5(#2)#6}}

% Make number non-italic in any environment.
\crefdefaultlabelformat{#2\textup{#1}#3}

% Environments
\makeatother
% For box around Definition, Theorem, \ldots
%%fakesection Theorems
\usepackage{thmtools}
\usepackage[framemethod=TikZ]{mdframed}

\theoremstyle{definition}
\mdfdefinestyle{mdbluebox}{%
	roundcorner = 10pt,
	linewidth=1pt,
	skipabove=12pt,
	innerbottommargin=9pt,
	skipbelow=2pt,
	nobreak=true,
	linecolor=blue,
	backgroundcolor=TealBlue!5,
}
\declaretheoremstyle[
	headfont=\sffamily\bfseries\color{MidnightBlue},
	mdframed={style=mdbluebox},
	headpunct={\\[3pt]},
	postheadspace={0pt}
]{thmbluebox}

\mdfdefinestyle{mdredbox}{%
	linewidth=0.5pt,
	skipabove=12pt,
	frametitleaboveskip=5pt,
	frametitlebelowskip=0pt,
	skipbelow=2pt,
	frametitlefont=\bfseries,
	innertopmargin=4pt,
	innerbottommargin=8pt,
	nobreak=false,
	linecolor=RawSienna,
	backgroundcolor=Salmon!5,
}
\declaretheoremstyle[
	headfont=\bfseries\color{RawSienna},
	mdframed={style=mdredbox},
	headpunct={\\[3pt]},
	postheadspace={0pt},
]{thmredbox}

\declaretheorem[%
style=thmbluebox,name=Theorem,numberwithin=section]{thm}
\declaretheorem[style=thmbluebox,name=Lemma,sibling=thm]{lem}
\declaretheorem[style=thmbluebox,name=Proposition,sibling=thm]{prop}
\declaretheorem[style=thmbluebox,name=Corollary,sibling=thm]{coro}
\declaretheorem[style=thmredbox,name=Example,sibling=thm]{eg}

\mdfdefinestyle{mdgreenbox}{%
	roundcorner = 10pt,
	linewidth=1pt,
	skipabove=12pt,
	innerbottommargin=9pt,
	skipbelow=2pt,
	nobreak=true,
	linecolor=ForestGreen,
	backgroundcolor=ForestGreen!5,
}

\declaretheoremstyle[
	headfont=\bfseries\sffamily\color{ForestGreen!70!black},
	bodyfont=\normalfont,
	spaceabove=2pt,
	spacebelow=1pt,
	mdframed={style=mdgreenbox},
	headpunct={ --- },
]{thmgreenbox}

\declaretheorem[style=thmgreenbox,name=Definition,sibling=thm]{defn}

\mdfdefinestyle{mdgreenboxsq}{%
	linewidth=1pt,
	skipabove=12pt,
	innerbottommargin=9pt,
	skipbelow=2pt,
	nobreak=true,
	linecolor=ForestGreen,
	backgroundcolor=ForestGreen!5,
}
\declaretheoremstyle[
	headfont=\bfseries\sffamily\color{ForestGreen!70!black},
	bodyfont=\normalfont,
	spaceabove=2pt,
	spacebelow=1pt,
	mdframed={style=mdgreenboxsq},
	headpunct={},
]{thmgreenboxsq}
\declaretheoremstyle[
	headfont=\bfseries\sffamily\color{ForestGreen!70!black},
	bodyfont=\normalfont,
	spaceabove=2pt,
	spacebelow=1pt,
	mdframed={style=mdgreenboxsq},
	headpunct={},
]{thmgreenboxsq*}

\mdfdefinestyle{mdblackbox}{%
	skipabove=8pt,
	linewidth=3pt,
	rightline=false,
	leftline=true,
	topline=false,
	bottomline=false,
	linecolor=black,
	backgroundcolor=RedViolet!5!gray!5,
}
\declaretheoremstyle[
	headfont=\bfseries,
	bodyfont=\normalfont\small,
	spaceabove=0pt,
	spacebelow=0pt,
	mdframed={style=mdblackbox}
]{thmblackbox}

\theoremstyle{plain}
\declaretheorem[name=Question,sibling=thm,style=thmblackbox]{ques}
\declaretheorem[name=Remark,sibling=thm,style=thmgreenboxsq]{remark}
\declaretheorem[name=Remark,sibling=thm,style=thmgreenboxsq*]{remark*}
\newtheorem{ass}[thm]{Assumptions}

\theoremstyle{definition}
\newtheorem*{problem}{Problem}
\newtheorem{claim}[thm]{Claim}
\theoremstyle{remark}
\newtheorem*{case}{Case}
\newtheorem*{notation}{Notation}
\newtheorem*{note}{Note}
\newtheorem*{motivation}{Motivation}
\newtheorem*{intuition}{Intuition}
\newtheorem*{conjecture}{Conjecture}

% Make section starts with 1 for report type
%\renewcommand\thesection{\arabic{section}}

% End example and intermezzo environments with a small diamond (just like proof
% environments end with a small square)
\usepackage{etoolbox}
\AtEndEnvironment{vb}{\null\hfill$\diamond$}%
\AtEndEnvironment{intermezzo}{\null\hfill$\diamond$}%
% \AtEndEnvironment{opmerking}{\null\hfill$\diamond$}%

% Fix some spacing
% http://tex.stackexchange.com/questions/22119/how-can-i-change-the-spacing-before-theorems-with-amsthm
\makeatletter
\def\thm@space@setup{%
  \thm@preskip=\parskip \thm@postskip=0pt
}

% Fix some stuff
% %http://tex.stackexchange.com/questions/76273/multiple-pdfs-with-page-group-included-in-a-single-page-warning
\pdfsuppresswarningpagegroup=1


% My name
\author{Jaden Wang}



\begin{document}
\centerline {\textsf{\textbf{\LARGE{Homework 10}}}}
\centerline {Jaden Wang}
\vspace{.15in}
\begin{problem}[Do Carmo 4.4]
Let $ M$ be a Riemannian manifold with the following property: given any  $ p,q \in M$, the parallel transport from $ p$ to  $ q$ does not depend on the curve. Prove that the curvature of  $ M$ is identically zero,  \emph{i.e.} for all $ X,Y,Z \in \mathfrak{X}(M)$, $ R(X,Y)Z = 0$.
\end{problem}
\begin{proof}
Consider a parametrized surface $ f: U \subset \rr^2 \to M$, where $ U$ is the  $ \epsilon$-neighorhood of the unit square in $ \rr^2$ and $ f(s,0) = f(0,0)$ for all $ s$. Given $ V_0 \in T_{(0,0)}M$, define a vector field  $ V$ along  $ f$ by setting  $ V(s,0) = V_0$ and if $ t \neq 0$,  $ V(s,t)$ is the parallel transport of  $ V_0$ along the curve $t \mapsto  f(s,t)$. By Lemma 4.1, any vector field along a parametrized surface $ f$ satisfies
 \begin{align*}
	\frac{D}{\partial t} \frac{D}{ \partial s} V - \frac{D}{\partial s} \frac{D}{\partial t} V = R \left( \frac{\partial f}{\partial s} , \frac{\partial f}{\partial t}  \right) .
\end{align*}
Since $ V(s,t)$ is parallel transported along  $ t$-curves,  $ \frac{D}{\partial t} V \equiv 0$. Since parallel transport in $ M$ does not depend on the curve, $ V(s,1)$ is also the parallel transport of  $ V(0,1)$ along an $ s$-curve.  Therefore, $ \frac{D}{\partial s} V(s,1) \equiv 0$. It follows that
\begin{align*}
	R_{f(s,1)}\left( \frac{\partial f}{\partial s} (s,1), \frac{\partial f}{\partial t} (s,1) \right) V(s,1) =0 .
\end{align*}
In particular this is true for $ s=0$. Since $ f$ and  $ V_0$ are arbitrary, given $ p \in M$, $ X,Y,Z \in \mathfrak{X}(M)$, we can let $ f(0,1) = p$, $ \frac{\partial f}{\partial s}(0,1) = X(p) , \frac{\partial f}{\partial t}(0,1) = Y(p) $, and $ V_0$ equal the parallel transport from $ V(0,1):= Z(p)$. The result follows pointwise.

\end{proof}
\begin{problem}[5.1]
Let $ M$ be a Riemannian manifold with sectional curvature identically zero. Show that, for every  $ p \in M$, the mapping $ \exp_p: B_{ \epsilon}(0) \subset T_pM \to B_{ \epsilon}(p)$ is an isometry, where $ B_{ \epsilon}(p)$ is a normal ball at $ p$.
\end{problem}
\begin{proof}
By the polarization trick, it suffices to show that for any $ v \in B_{ \epsilon}(0)$ and $ w \in T_v(T_pM)$, we have
\begin{align*}
	\norm{ d(\exp_p)_v(w)} &= \norm{ w} .
\end{align*}
It in turn suffices to show that
\begin{align*}
	\norm{ d(\exp_p)_{tv}(tw)} &= \norm{ tw} .
\end{align*}
Suppose $V(0) =v$,  $ V'(0) = w$, then we have the Jacobi field along the geodesic $ \gamma(t) = \exp_p(tv)$:
 \begin{align*}
	 J(t) &= \frac{\partial }{\partial s} \left( \exp_p (tV(s)) \right)(t,0) \\ 
	 &= d\left( \exp_p \right)_{tV(0)} (tV'(0))  \\
	 &= d\left( \exp_p \right)_{tv} (tw)  .
\end{align*}
Let $ E_i$ be an orthonormal parallel frame along $ \gamma$ and let $ J(t) = f_{i}(t) E_i(t)$. Since the frame is parallel along the geodesic, we have $ \frac{D}{ \partial t} E_i(t) \equiv 0$ so $ J''(t) = f_i''(t) E_i(t)$. Since the sectional curvature $ K \equiv 0$, the only potentially nonzero curvatures of the form $ R_{ijij}= K$ must be zero as well. Thus the Jacobi equation reduces to $ f_i''(t) = 0$, yielding $ f_i(t) = a_i t + b_i$. Since $ J(0) = 0$, we have  $ f_i(t) = a_i t$. Also recall $ J'(0) = V'(0) = w = f_i'(0) E_i(0) = a_i E_i(0)$. Therefore, we obtain
\begin{align*}
	\norm{ d(\exp_p)_{tv} (tw) }  &= \norm{ J(t)} \\
	&= \norm{t a_i E_i(t)}  \\
	&= t \sum_{ i= 1}^{ n} a_i^2 && E_i \text{ is orthonormal} \\
	&= \norm{ tw}  .
\end{align*}
\end{proof}

\begin{problem}[5.6]
Let $ M$ be a Riemannian manifold of dimension 2. Let  $ B_{ \delta}(p)$ be a normal ball around the point $ p \in M$ and consider the parametrized surface
\begin{align*}
	f(\rho, \theta) = \exp_p(\rho v(\theta)), \qquad 0<\rho< \delta, -\pi < \theta < \theta,
\end{align*}
where $ v(\theta)$ is a circle of radius $ 1$ in  $ T_pM$ parametrized by the angle.
 \begin{enumerate}[label=(\alph*)]
	\item Show that $ (\rho, \theta)$ are coordinates in an open set $ U \subset M$ formed by the open ball $ B_{ \delta}(p) $ minus the ray $ \exp_p(-\rho v(0))$. This is the polar coordinates at $ p$. 
	\item Show that $ g_{ij}$ of the Riemannian metric in these coordinates are $ g_{12} =0, g_{11} = \norm{ \frac{\partial f}{\partial \rho} }^2 = \norm{ v(\theta)}^2=  1, g_{22} = \norm{ \frac{\partial f}{\partial \theta} }^2$.
	\item Show that, along the geodesic $ f(\rho,0)$, we have
		\begin{align*}
			(\sqrt{g_{22}} )_{\rho \rho} = -K(p) \rho+R(p),
		\end{align*}
		where $ \lim_{ \rho \to 0} \frac{R(p)}{ \rho} = 0$.
	\item Prove that
		\begin{align*}
			\lim_{ \rho \to 0} \frac{(\sqrt{g_{22}}_{\rho \rho} )}{\sqrt{g_{ 22}}  } =-K(p) .
		\end{align*}
\end{enumerate}
\end{problem}
\begin{proof}
\begin{enumerate}[label=(\alph*)]
	\item Since the image of $ f$ is contained in a normal ball,  $ \exp_p$ has a smooth inverse. Using the norm and inverse tangent (which is only bijective if we remove the ray) we can compose a smooth inverse of $ f$.
	\item Notice $ v(\theta)$ is tangent to the curve $ \rho v(\theta)$. To compute the pullback metric, we have:
		\begin{align*}
			g_{11}(\rho,\theta) &= \left\langle \frac{\partial }{\partial \rho}(\rho,\theta), \frac{\partial }{\partial \rho}(\rho, \theta)  \right\rangle\\
					    &= \left\langle \frac{\partial f}{\partial \rho}(\rho,\theta), \frac{\partial f}{\partial \rho}(\rho, \theta)   \right\rangle \\
		       &= \left| \frac{\partial f}{\partial \rho}(\rho,\theta)  \right|^2  && \text{proof of Gauss's lemma}  \\
			&= \norm{ \frac{\partial }{\partial \rho} \exp_p (\rho v(\theta)) }^2  \\
			&= \norm{ d(\exp_p)_{\rho v(\theta)} [v(\theta)] }^2  \\
			&= \norm{ v(\theta)}^2 && \text{geodesic preserves tangent vector length}    \\
			&= 1 \\
			g_{12} &=  \left\langle \frac{\partial }{\partial \rho}, \frac{\partial }{\partial \theta}   \right\rangle \\
					    &=\left\langle \frac{\partial f}{\partial \rho}, \frac{\partial f}{\partial \theta}  \right\rangle  \\
					    &= 0&& \text{proof of Gauss's lemma}\\
			g_{22} &= \left\langle  \frac{\partial }{\partial \theta}, \frac{\partial }{\partial \theta}\right\rangle \\
			       &= \left\langle \frac{\partial f}{\partial \theta}, \frac{\partial f}{\partial \theta}   \right\rangle \\
			       &=\left| \frac{\partial f}{\partial \theta}  \right|^2  && \text{proof of Gauss's lemma} .
		\end{align*}
	\item Since $ |J(\rho)| = \left| \frac{\partial f}{\partial \theta}(\rho,0)  \right| = \sqrt{g_{22}}(\rho)  $ by part (b), by Corollary 5.2.10 we obtain
\begin{align*}
\sqrt{g_{22}}(\rho)  &= \rho - \frac{1}{6} K(p) \rho^3 + \widetilde{ R}(\rho) ,
\end{align*}
where $ \lim_{ \rho \to 0} \frac{\widetilde{ R}(\rho)}{ \rho^3} =0$. Differentiating twice with respect to $ \rho$, we obtain
\begin{align*}
	\sqrt{g_{22}}_{\rho \rho} = - K(p) \rho + R(\rho), 
\end{align*}
where $ \lim_{ \rho \to 0} \frac{R(\rho)}{ \rho} = 0$.
\item
\begin{align*}
\lim_{ \rho \to 0} \frac{(\sqrt{g_{22}}_{\rho \rho} )}{\sqrt{g_{ 22}}  } &= \lim_{ \rho \to 0} \frac{-K(p)\rho + R(\rho)}{ \rho - \frac{1}{6} K(p) \rho^3 + \widetilde{ R}(\rho)} \\
&= \lim_{ \rho \to 0} \frac{-K(p)\rho + O(\rho^2)}{ \rho + O(\rho^3)} \\
&= -K(p) .
\end{align*}
\end{enumerate}
\end{proof}

\begin{problem}[5.7]
Let $ M$ be Riemannian manifold with dimension 2. Let  $ p \in M$ and $ V \subset T_pM$ be a normal neighborhood. Let $ S_r(0) \subset V$ be a circle of radius $ r$ centered at the origin, and let  $ L_r$ be the length of the curve  $ \exp_p(S_r)$ in $ M$. Prove that the sectional curvature at $ p$ is
 \begin{align*}
	K(p) = \lim_{ r \to 0} \frac{3}{\pi} \frac{2\pi r - L_r}{ r^3}.
\end{align*}
\end{problem}
\begin{proof}
First, using result of Exercise 5.6, we obtain
\begin{align*}
	L_r &= \int_{ -\pi}^{ \pi} \sqrt{ \left\langle \frac{\partial }{\partial \theta} , \frac{\partial }{\partial \theta}   \right\rangle} d \theta \\  
	&= \int_{ -\pi}^{ \pi} \sqrt{g_{22}(r,\theta)} d \theta   .
\end{align*}
Thus by Corollary 5.2.10 we have
\begin{align*}
	2 \pi r - L_r &= \int_{ -\pi}^{ \pi} \left( r- \left( r - \frac{1}{6} K(p) r^3 + \widetilde{ R}(\rho) + r \right)  \right)  d \theta \\
	&= \int_{ -\pi}^{ \pi} \left( \frac{1}{6} K(p) - \frac{\widetilde{ R}(r)}{ r^3} \right)  d \theta \\
	&= 2\pi \left( \frac{1}{6} K(p) - \frac{\widetilde{ R}(r)}{ r^3} \right)  .
\end{align*}
Therefore we obtain
\begin{align*}
	\lim_{ r \to 0} \frac{3}{\pi} \frac{2\pi r - L_r}{ r^3}&= \lim_{ r \to 0} \frac{3}{\pi} 2\pi\left( \frac{1}{6} K(p) - \frac{\widetilde{ R}(r)}{ r^3} \right)     \\
	&= K(p).
\end{align*}
\end{proof}
\end{document}
