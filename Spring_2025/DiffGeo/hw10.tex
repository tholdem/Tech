\documentclass[12pt]{article}
\newcommand{\alert}[1]{{\bf \color{red} [Alert:] #1}}
\newcommand{\todo}[1]{{\bf \color{orange} [TODO:] #1}}
\newcommand{\real}[1][]{\mathbb{R}^{#1}}
\newcommand{\myeqn}[1]{(\ref{#1})}
\newcommand{\myex}[1]{Example \ref{#1}}
\newcommand{\defeq}{\stackrel{\mathrm{def}}{=}}
\newcommand{\parder}[2]{\frac{\partial #1}{\partial #2}}
\newcommand{\Lie}[3][]{\mathsf{L}_{#3}^{#1} #2}
\newcommand{\LieA}[1]{\mathsf{Lie}(#1)}
\newcommand{\lieder}[2]{\mathcal{L}_{#2} #1}
\renewcommand{\t}{^{\mbox{\tiny\sf T}}}
\newcommand{\trans}{^{\mbox{\tiny\sf T}}}
\newcommand{\markup}[1]{\{\textbf{#1}\}}
\newcommand{\msub}[1]{_\mathrm{#1}}
\newcommand{\msup}[1]{^\mathrm{#1}}
\newcommand{\inv}[1]{#1^{-1}}
\newcommand{\pinv}[1]{{#1}^{+}}
\newcommand{\myfracA}[2]{\displaystyle{\frac{#1}{#2}}}
\newcommand{\myfracB}[2]{{#1}/{#2}}
\newcommand{\mydiffA}[1]{\dot{#1}}
\newcommand{\mydiffB}[2]{\myfracA{\mathrm{d}{#1}}{\mathrm{d}{#2}}}
\newcommand{\ball}[2]{\mathcal{B}_{#1}\left(#2\right)}
\newcommand{\acos}[1]{\cos^{-1}\left(#1\right)}
\newcommand{\asin}[1]{\sin^{-1}\left(#1\right)}
\newcommand{\mani}{\mathcal{M}}
\newcommand{\tang}[2]{\mathsf{T}_{#1} #2}
\newcommand{\LieB}[2]{[ #1, #2 ]}
\newcommand{\LieBad}[3][]{\mathsf{ad}_{#2}^{#1} #3}
\newcommand{\ReachVT}{\mathcal{R}^V_T}
\newcommand{\ReachVt}{\mathcal{R}^V_t}
\newcommand{\ReachVTe}{\mathcal{R}^V_{\le T}}
\newcommand{\ReachT}{\mathcal{R}_T}
\newcommand{\Reacht}{\mathcal{R}_t}
\newcommand{\ReachTe}{\mathcal{R}_{\le T}}
\newcommand{\accLA}[1]{\mathsf{Lie}(#1)}
\newcommand{\accD}{\Delta_{\mathcal{F}}}
\newcommand{\accSA}{\mathsf{Lie}(\mathcal{G},f)}
\newcommand{\accDS}{\Delta_{\mathcal{G}}}
\newcommand{\eval}[3]{\mathsf{Ev}^{#2}_{#1}\left( #3 \right)}
\newcommand{\stlc}{\textsc{stlc}}
\newcommand{\clf}{\textsc{clf}}
\newcommand{\jqlf}{\textsc{jqlf}}
\newcommand{\dlas}{\textsc{dlas}}
\newcommand{\Ad}[2]{\mathsf{Ad}_{#1} #2}
\newcommand{\xe}{\ensuremath{x_e}}
\newcommand{\lebg}[1]{\mathcal{L}_{#1}}
\newcommand{\lebgx}[1]{\mathcal{L}_{#1 \mathrm{e}}}
\newcommand{\dom}{D}
\newcommand{\domT}{[t_0,\infty) \times D}
\newcommand{\rarrow}{\rightarrow}
\renewcommand{\d}{\mathrm{d}}
\renewcommand{\Re}{\mathbb{R}}
\newcommand{\C}{\mathrm{C}}

\newcommand{\QED}{{\unskip\nobreak\hfil\penalty50\hskip2em\vadjust{}
		\nobreak\hfil$\Box$\parfillskip=0pt\finalhyphendemerits=0\par}\vspace{0.1cm}}
\newcommand{\eoEx}{{\unskip\nobreak\hfil\penalty50\hskip0em\vadjust{}
		\nobreak\hfil$\Large\Diamond$\parfillskip=0pt\finalhyphendemerits=0\par}\vspace{0.1cm}}

\newcommand{\sgn}{\ensuremath{\operatorname{sgn}}}
\newcommand{\sat}{\ensuremath{\operatorname{sat}}}

\newcommand{\half}{\frac{1}{2}}
\newcommand{\shalf}{\mbox{$\frac{1}{2}$}}
\newcommand{\marcom}[1]{\marginpar{\footnotesize #1}}
\newcommand{\der}{\mathrm{D}}
\newcommand{\e}{\mathrm{e}}
\newcommand{\dt}{\mathrm{d}t}

\newcommand{\cA}{\ensuremath{\mathcal{A}}}
\newcommand{\cB}{\ensuremath{\mathcal{B}}}
\newcommand{\cG}{\ensuremath{\mathcal{G}}}
\newcommand{\cK}{\ensuremath{\mathcal{K}}}
\newcommand{\cW}{\ensuremath{\mathcal{W}}}
\newcommand{\cZ}{\ensuremath{\mathcal{Z}}}
\newcommand{\cS}{\ensuremath{\mathcal{S}}}
\newcommand{\cD}{\ensuremath{\mathcal{D}}}
\newcommand{\cP}{\ensuremath{\mathcal{P}}}
\newcommand{\cV}{\ensuremath{\mathcal{V}}}
\newcommand{\cL}{\ensuremath{\mathcal{L}}}
\newcommand{\cN}{\ensuremath{\mathcal{N}}}
\newcommand{\cI}{\ensuremath{\mathcal{I}}}
\newcommand{\cR}{\ensuremath{\mathcal{R}}}
\newcommand{\cM}{\ensuremath{\mathcal{M}}}
\newcommand{\cC}{\ensuremath{\mathcal{C}}}
\newcommand{\cF}{\ensuremath{\mathcal{F}}}
\newcommand{\cH}{\ensuremath{\mathcal{H}}}
\newcommand{\cO}{\ensuremath{\mathcal{O}}}
\newcommand{\cX}{\ensuremath{\mathcal{X}}}
\newcommand{\cY}{\ensuremath{\mathcal{Y}}}
\newcommand{\Ci}{\ensuremath{\mathcal{C}^\infty}}
\newcommand{\ISS}{\textsc{iss}}
\newcommand{\LISS}{\textsc{liss}}
\newcommand{\GAS}{\textsc{gas}}
\newcommand{\GS}{\textsc{gs}}
\newcommand{\LES}{\textsc{les}}
\newcommand{\GUAS}{\textsc{guas}}
\newcommand{\BIBO}{\textsc{bibo}}
\newcommand{\spec}{\ensuremath{\operatorname{spec}}}
\newcommand{\spn}{\ensuremath{\operatorname{span}}}
\renewcommand{\i}{\mathrm{i\,}}

\renewcommand{\implies}{\Rightarrow}

\renewcommand{\theenumi}{$\roman{enumi})$}
\renewcommand{\labelenumi}{\theenumi}

\font\ptmten=zptmcmrm scaled 1200
\newcommand{\w}{\mbox{{\ptmten w}}}
\newcommand{\z}{\mbox{{\ptmten z}}}
\renewcommand{\Re}{\mathbb{R}}

\newcommand{\cl}{\operatorname{cl}}
\newcommand{\intr}{\operatorname{int}}
\newcommand{\rank}{\operatorname{rank}}
\newcommand{\co}{\operatorname{co}}
\newcommand{\aff}{\operatorname{aff}}

\theoremstyle{plain}
\newtheorem{theorem}{Theorem}[chapter]
\newtheorem{claim}[theorem]{Claim}
\newtheorem{corollary}[theorem]{Corollary}
\newtheorem{prop}[theorem]{Proposition}
\newtheorem{fact}[theorem]{Fact}
\newtheorem{lemma}[theorem]{Lemma}

\newtheorem{remark}{Remark}[chapter]

\theoremstyle{definition}
\newtheorem{assume}[theorem]{Assumption}
\newtheorem{defn}[theorem]{Definition}
\newtheorem{problem}[theorem]{Problem}
\newtheorem{exercise}{Exercise}
\newtheorem{example}[theorem]{Example}


\begin{document}
\centerline {\textsf{\textbf{\LARGE{Homework 10}}}}
\centerline {Jaden Wang}
\vspace{.15in}
\begin{problem}[Do Carmo 4.4]
Let $ M$ be a Riemannian manifold with the following property: given any  $ p,q \in M$, the parallel transport from $ p$ to  $ q$ does not depend on the curve. Prove that the curvature of  $ M$ is identically zero,  \emph{i.e.} for all $ X,Y,Z \in \mathfrak{X}(M)$, $ R(X,Y)Z = 0$.
\end{problem}
\begin{proof}
Consider a parametrized surface $ f: U \subset \rr^2 \to M$, where $ U$ is the  $ \epsilon$-neighorhood of the unit square in $ \rr^2$ and $ f(s,0) = f(0,0)$ for all $ s$. Given $ V_0 \in T_{(0,0)}M$, define a vector field  $ V$ along  $ f$ by setting  $ V(s,0) = V_0$ and if $ t \neq 0$,  $ V(s,t)$ is the parallel transport of  $ V_0$ along the curve $t \mapsto  f(s,t)$. By Lemma 4.1, any vector field along a parametrized surface $ f$ satisfies
 \begin{align*}
	\frac{D}{\partial t} \frac{D}{ \partial s} V - \frac{D}{\partial s} \frac{D}{\partial t} V = R \left( \frac{\partial f}{\partial s} , \frac{\partial f}{\partial t}  \right) .
\end{align*}
Since $ V(s,t)$ is parallel transported along  $ t$-curves,  $ \frac{D}{\partial t} V \equiv 0$. Since parallel transport in $ M$ does not depend on the curve, $ V(s,1)$ is also the parallel transport of  $ V(0,1)$ along an $ s$-curve.  Therefore, $ \frac{D}{\partial s} V(s,1) \equiv 0$. It follows that
\begin{align*}
	R_{f(s,1)}\left( \frac{\partial f}{\partial s} (s,1), \frac{\partial f}{\partial t} (s,1) \right) V(s,1) =0 .
\end{align*}
In particular this is true for $ s=0$. Since $ f$ and  $ V_0$ are arbitrary, given $ p \in M$, $ X,Y,Z \in \mathfrak{X}(M)$, we can let $ f(0,1) = p$, $ \frac{\partial f}{\partial s}(0,1) = X(p) , \frac{\partial f}{\partial t}(0,1) = Y(p) $, and $ V_0$ equal the parallel transport from $ V(0,1):= Z(p)$. The result follows pointwise.

\end{proof}
\begin{problem}[5.1]
Let $ M$ be a Riemannian manifold with sectional curvature identically zero. Show that, for every  $ p \in M$, the mapping $ \exp_p: B_{ \epsilon}(0) \subset T_pM \to B_{ \epsilon}(p)$ is an isometry, where $ B_{ \epsilon}(p)$ is a normal ball at $ p$.
\end{problem}
\begin{proof}
By the polarization trick, it suffices to show that for any $ v \in B_{ \epsilon}(0)$ and $ w \in T_v(T_pM)$, we have
\begin{align*}
	\norm{ d(\exp_p)_v(w)} &= \norm{ w} .
\end{align*}
It in turn suffices to show that
\begin{align*}
	\norm{ d(\exp_p)_{tv}(tw)} &= \norm{ tw} .
\end{align*}
Suppose $V(0) =v$,  $ V'(0) = w$, then we have the Jacobi field along the geodesic $ \gamma(t) = \exp_p(tv)$:
 \begin{align*}
	 J(t) &= \frac{\partial }{\partial s} \left( \exp_p (tV(s)) \right)(t,0) \\ 
	 &= d\left( \exp_p \right)_{tV(0)} (tV'(0))  \\
	 &= d\left( \exp_p \right)_{tv} (tw)  .
\end{align*}
Let $ E_i$ be an orthonormal parallel frame along $ \gamma$ and let $ J(t) = f_{i}(t) E_i(t)$. Since the frame is parallel along the geodesic, we have $ \frac{D}{ \partial t} E_i(t) \equiv 0$ so $ J''(t) = f_i''(t) E_i(t)$. Since the sectional curvature $ K \equiv 0$, the only potentially nonzero curvatures of the form $ R_{ijij}= K$ must be zero as well. Thus the Jacobi equation reduces to $ f_i''(t) = 0$, yielding $ f_i(t) = a_i t + b_i$. Since $ J(0) = 0$, we have  $ f_i(t) = a_i t$. Also recall $ J'(0) = V'(0) = w = f_i'(0) E_i(0) = a_i E_i(0)$. Therefore, we obtain
\begin{align*}
	\norm{ d(\exp_p)_{tv} (tw) }  &= \norm{ J(t)} \\
	&= \norm{t a_i E_i(t)}  \\
	&= t \sum_{ i= 1}^{ n} a_i^2 && E_i \text{ is orthonormal} \\
	&= \norm{ tw}  .
\end{align*}
\end{proof}

\begin{problem}[5.6]
Let $ M$ be a Riemannian manifold of dimension 2. Let  $ B_{ \delta}(p)$ be a normal ball around the point $ p \in M$ and consider the parametrized surface
\begin{align*}
	f(\rho, \theta) = \exp_p(\rho v(\theta)), \qquad 0<\rho< \delta, -\pi < \theta < \theta,
\end{align*}
where $ v(\theta)$ is a circle of radius $ 1$ in  $ T_pM$ parametrized by the angle.
 \begin{enumerate}[label=(\alph*)]
	\item Show that $ (\rho, \theta)$ are coordinates in an open set $ U \subset M$ formed by the open ball $ B_{ \delta}(p) $ minus the ray $ \exp_p(-\rho v(0))$. This is the polar coordinates at $ p$. 
	\item Show that $ g_{ij}$ of the Riemannian metric in these coordinates are $ g_{12} =0, g_{11} = \norm{ \frac{\partial f}{\partial \rho} }^2 = \norm{ v(\theta)}^2=  1, g_{22} = \norm{ \frac{\partial f}{\partial \theta} }^2$.
	\item Show that, along the geodesic $ f(\rho,0)$, we have
		\begin{align*}
			(\sqrt{g_{22}} )_{\rho \rho} = -K(p) \rho+R(p),
		\end{align*}
		where $ \lim_{ \rho \to 0} \frac{R(p)}{ \rho} = 0$.
	\item Prove that
		\begin{align*}
			\lim_{ \rho \to 0} \frac{(\sqrt{g_{22}}_{\rho \rho} )}{\sqrt{g_{ 22}}  } =-K(p) .
		\end{align*}
\end{enumerate}
\end{problem}
\begin{proof}
\begin{enumerate}[label=(\alph*)]
	\item Since the image of $ f$ is contained in a normal ball,  $ \exp_p$ has a smooth inverse. Using the norm and inverse tangent (which is only bijective if we remove the ray) we can compose a smooth inverse of $ f$.
	\item Notice $ v(\theta)$ is tangent to the curve $ \rho v(\theta)$. To compute the pullback metric, we have:
		\begin{align*}
			g_{11}(\rho,\theta) &= \left\langle \frac{\partial }{\partial \rho}(\rho,\theta), \frac{\partial }{\partial \rho}(\rho, \theta)  \right\rangle\\
					    &= \left\langle \frac{\partial f}{\partial \rho}(\rho,\theta), \frac{\partial f}{\partial \rho}(\rho, \theta)   \right\rangle \\
		       &= \left| \frac{\partial f}{\partial \rho}(\rho,\theta)  \right|^2  && \text{proof of Gauss's lemma}  \\
			&= \norm{ \frac{\partial }{\partial \rho} \exp_p (\rho v(\theta)) }^2  \\
			&= \norm{ d(\exp_p)_{\rho v(\theta)} [v(\theta)] }^2  \\
			&= \norm{ v(\theta)}^2 && \text{geodesic preserves tangent vector length}    \\
			&= 1 \\
			g_{12} &=  \left\langle \frac{\partial }{\partial \rho}, \frac{\partial }{\partial \theta}   \right\rangle \\
					    &=\left\langle \frac{\partial f}{\partial \rho}, \frac{\partial f}{\partial \theta}  \right\rangle  \\
					    &= 0&& \text{proof of Gauss's lemma}\\
			g_{22} &= \left\langle  \frac{\partial }{\partial \theta}, \frac{\partial }{\partial \theta}\right\rangle \\
			       &= \left\langle \frac{\partial f}{\partial \theta}, \frac{\partial f}{\partial \theta}   \right\rangle \\
			       &=\left| \frac{\partial f}{\partial \theta}  \right|^2  && \text{proof of Gauss's lemma} .
		\end{align*}
	\item Since $ |J(\rho)| = \left| \frac{\partial f}{\partial \theta}(\rho,0)  \right| = \sqrt{g_{22}}(\rho)  $ by part (b), by Corollary 5.2.10 we obtain
\begin{align*}
\sqrt{g_{22}}(\rho)  &= \rho - \frac{1}{6} K(p) \rho^3 + \widetilde{ R}(\rho) ,
\end{align*}
where $ \lim_{ \rho \to 0} \frac{\widetilde{ R}(\rho)}{ \rho^3} =0$. Differentiating twice with respect to $ \rho$, we obtain
\begin{align*}
	\sqrt{g_{22}}_{\rho \rho} = - K(p) \rho + R(\rho), 
\end{align*}
where $ \lim_{ \rho \to 0} \frac{R(\rho)}{ \rho} = 0$.
\item
\begin{align*}
\lim_{ \rho \to 0} \frac{(\sqrt{g_{22}}_{\rho \rho} )}{\sqrt{g_{ 22}}  } &= \lim_{ \rho \to 0} \frac{-K(p)\rho + R(\rho)}{ \rho - \frac{1}{6} K(p) \rho^3 + \widetilde{ R}(\rho)} \\
&= \lim_{ \rho \to 0} \frac{-K(p)\rho + O(\rho^2)}{ \rho + O(\rho^3)} \\
&= -K(p) .
\end{align*}
\end{enumerate}
\end{proof}

\begin{problem}[5.7]
Let $ M$ be Riemannian manifold with dimension 2. Let  $ p \in M$ and $ V \subset T_pM$ be a normal neighborhood. Let $ S_r(0) \subset V$ be a circle of radius $ r$ centered at the origin, and let  $ L_r$ be the length of the curve  $ \exp_p(S_r)$ in $ M$. Prove that the sectional curvature at $ p$ is
 \begin{align*}
	K(p) = \lim_{ r \to 0} \frac{3}{\pi} \frac{2\pi r - L_r}{ r^3}.
\end{align*}
\end{problem}
\begin{proof}
First, using result of Exercise 5.6, we obtain
\begin{align*}
	L_r &= \int_{ -\pi}^{ \pi} \sqrt{ \left\langle \frac{\partial }{\partial \theta} , \frac{\partial }{\partial \theta}   \right\rangle} d \theta \\  
	&= \int_{ -\pi}^{ \pi} \sqrt{g_{22}(r,\theta)} d \theta   .
\end{align*}
Thus by Corollary 5.2.10 we have
\begin{align*}
	2 \pi r - L_r &= \int_{ -\pi}^{ \pi} \left( r- \left( r - \frac{1}{6} K(p) r^3 + \widetilde{ R}(\rho) + r \right)  \right)  d \theta \\
	&= \int_{ -\pi}^{ \pi} \left( \frac{1}{6} K(p) - \frac{\widetilde{ R}(r)}{ r^3} \right)  d \theta \\
	&= 2\pi \left( \frac{1}{6} K(p) - \frac{\widetilde{ R}(r)}{ r^3} \right)  .
\end{align*}
Therefore we obtain
\begin{align*}
	\lim_{ r \to 0} \frac{3}{\pi} \frac{2\pi r - L_r}{ r^3}&= \lim_{ r \to 0} \frac{3}{\pi} 2\pi\left( \frac{1}{6} K(p) - \frac{\widetilde{ R}(r)}{ r^3} \right)     \\
	&= K(p).
\end{align*}
\end{proof}
\end{document}
