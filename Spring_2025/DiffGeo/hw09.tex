\documentclass[12pt]{article}
%Fall 2022
% Some basic packages
\usepackage{standalone}[subpreambles=true]
\usepackage[utf8]{inputenc}
\usepackage[T1]{fontenc}
\usepackage{textcomp}
\usepackage[english]{babel}
\usepackage{url}
\usepackage{graphicx}
%\usepackage{quiver}
\usepackage{float}
\usepackage{enumitem}
\usepackage{lmodern}
\usepackage{comment}
\usepackage{hyperref}
\usepackage[usenames,svgnames,dvipsnames]{xcolor}
\usepackage[margin=1in]{geometry}
\usepackage{pdfpages}

\pdfminorversion=7

% Don't indent paragraphs, leave some space between them
\usepackage{parskip}

% Hide page number when page is empty
\usepackage{emptypage}
\usepackage{subcaption}
\usepackage{multicol}
\usepackage[b]{esvect}

% Math stuff
\usepackage{amsmath, amsfonts, mathtools, amsthm, amssymb}
\usepackage{bbm}
\usepackage{stmaryrd}
\allowdisplaybreaks

% Fancy script capitals
\usepackage{mathrsfs}
\usepackage{cancel}
% Bold math
\usepackage{bm}
% Some shortcuts
\newcommand{\rr}{\ensuremath{\mathbb{R}}}
\newcommand{\zz}{\ensuremath{\mathbb{Z}}}
\newcommand{\qq}{\ensuremath{\mathbb{Q}}}
\newcommand{\nn}{\ensuremath{\mathbb{N}}}
\newcommand{\ff}{\ensuremath{\mathbb{F}}}
\newcommand{\cc}{\ensuremath{\mathbb{C}}}
\newcommand{\ee}{\ensuremath{\mathbb{E}}}
\newcommand{\hh}{\ensuremath{\mathbb{H}}}
\renewcommand\O{\ensuremath{\emptyset}}
\newcommand{\norm}[1]{{\left\lVert{#1}\right\rVert}}
\newcommand{\dbracket}[1]{{\left\llbracket{#1}\right\rrbracket}}
\newcommand{\ve}[1]{{\bm{#1}}}
\newcommand\allbold[1]{{\boldmath\textbf{#1}}}
\DeclareMathOperator{\lcm}{lcm}
\DeclareMathOperator{\im}{im}
\DeclareMathOperator{\coim}{coim}
\DeclareMathOperator{\dom}{dom}
\DeclareMathOperator{\tr}{tr}
\DeclareMathOperator{\rank}{rank}
\DeclareMathOperator*{\var}{Var}
\DeclareMathOperator*{\ev}{E}
\DeclareMathOperator{\dg}{deg}
\DeclareMathOperator{\aff}{aff}
\DeclareMathOperator{\conv}{conv}
\DeclareMathOperator{\inte}{int}
\DeclareMathOperator*{\argmin}{argmin}
\DeclareMathOperator*{\argmax}{argmax}
\DeclareMathOperator{\graph}{graph}
\DeclareMathOperator{\sgn}{sgn}
\DeclareMathOperator*{\Rep}{Rep}
\DeclareMathOperator{\Proj}{Proj}
\DeclareMathOperator{\mat}{mat}
\DeclareMathOperator{\diag}{diag}
\DeclareMathOperator{\aut}{Aut}
\DeclareMathOperator{\gal}{Gal}
\DeclareMathOperator{\inn}{Inn}
\DeclareMathOperator{\edm}{End}
\DeclareMathOperator{\Hom}{Hom}
\DeclareMathOperator{\ext}{Ext}
\DeclareMathOperator{\tor}{Tor}
\DeclareMathOperator{\Span}{Span}
\DeclareMathOperator{\Stab}{Stab}
\DeclareMathOperator{\cont}{cont}
\DeclareMathOperator{\Ann}{Ann}
\DeclareMathOperator{\Div}{div}
\DeclareMathOperator{\curl}{curl}
\DeclareMathOperator{\nat}{Nat}
\DeclareMathOperator{\gr}{Gr}
\DeclareMathOperator{\vect}{Vect}
\DeclareMathOperator{\id}{id}
\DeclareMathOperator{\Mod}{Mod}
\DeclareMathOperator{\sign}{sign}
\DeclareMathOperator{\Surf}{Surf}
\DeclareMathOperator{\fcone}{fcone}
\DeclareMathOperator{\Rot}{Rot}
\DeclareMathOperator{\grad}{grad}
\DeclareMathOperator{\atan2}{atan2}
\DeclareMathOperator{\Ric}{Ric}
\let\vec\relax
\DeclareMathOperator{\vec}{vec}
\let\Re\relax
\DeclareMathOperator{\Re}{Re}
\let\Im\relax
\DeclareMathOperator{\Im}{Im}
% Put x \to \infty below \lim
\let\svlim\lim\def\lim{\svlim\limits}

%wide hat
\usepackage{scalerel,stackengine}
\stackMath
\newcommand*\wh[1]{%
\savestack{\tmpbox}{\stretchto{%
  \scaleto{%
    \scalerel*[\widthof{\ensuremath{#1}}]{\kern-.6pt\bigwedge\kern-.6pt}%
    {\rule[-\textheight/2]{1ex}{\textheight}}%WIDTH-LIMITED BIG WEDGE
  }{\textheight}% 
}{0.5ex}}%
\stackon[1pt]{#1}{\tmpbox}%
}
\parskip 1ex

%Make implies and impliedby shorter
\let\implies\Rightarrow
\let\impliedby\Leftarrow
\let\iff\Leftrightarrow
\let\epsilon\varepsilon

% Add \contra symbol to denote contradiction
\usepackage{stmaryrd} % for \lightning
\newcommand\contra{\scalebox{1.5}{$\lightning$}}

% \let\phi\varphi

% Command for short corrections
% Usage: 1+1=\correct{3}{2}

\definecolor{correct}{HTML}{009900}
\newcommand\correct[2]{\ensuremath{\:}{\color{red}{#1}}\ensuremath{\to }{\color{correct}{#2}}\ensuremath{\:}}
\newcommand\green[1]{{\color{correct}{#1}}}

% horizontal rule
\newcommand\hr{
    \noindent\rule[0.5ex]{\linewidth}{0.5pt}
}

% hide parts
\newcommand\hide[1]{}

% si unitx
\usepackage{siunitx}
\sisetup{locale = FR}

%allows pmatrix to stretch
\makeatletter
\renewcommand*\env@matrix[1][\arraystretch]{%
  \edef\arraystretch{#1}%
  \hskip -\arraycolsep
  \let\@ifnextchar\new@ifnextchar
  \array{*\c@MaxMatrixCols c}}
\makeatother

\renewcommand{\arraystretch}{0.8}

\renewcommand{\baselinestretch}{1.5}

\usepackage{graphics}
\usepackage{epstopdf}

\RequirePackage{hyperref}
%%
%% Add support for color in order to color the hyperlinks.
%% 
\hypersetup{
  colorlinks = true,
  urlcolor = blue,
  citecolor = blue
}
%%fakesection Links
\hypersetup{
    colorlinks,
    linkcolor={red!50!black},
    citecolor={green!50!black},
    urlcolor={blue!80!black}
}
%customization of cleveref
\RequirePackage[capitalize,nameinlink]{cleveref}[0.19]

% Per SIAM Style Manual, "section" should be lowercase
\crefname{section}{section}{sections}
\crefname{subsection}{subsection}{subsections}
\Crefname{section}{Section}{Sections}
\Crefname{subsection}{Subsection}{Subsections}

% Per SIAM Style Manual, "Figure" should be spelled out in references
\Crefname{figure}{Figure}{Figures}

% Per SIAM Style Manual, don't say equation in front on an equation.
\crefformat{equation}{\textup{#2(#1)#3}}
\crefrangeformat{equation}{\textup{#3(#1)#4--#5(#2)#6}}
\crefmultiformat{equation}{\textup{#2(#1)#3}}{ and \textup{#2(#1)#3}}
{, \textup{#2(#1)#3}}{, and \textup{#2(#1)#3}}
\crefrangemultiformat{equation}{\textup{#3(#1)#4--#5(#2)#6}}%
{ and \textup{#3(#1)#4--#5(#2)#6}}{, \textup{#3(#1)#4--#5(#2)#6}}{, and \textup{#3(#1)#4--#5(#2)#6}}

% But spell it out at the beginning of a sentence.
\Crefformat{equation}{#2Equation~\textup{(#1)}#3}
\Crefrangeformat{equation}{Equations~\textup{#3(#1)#4--#5(#2)#6}}
\Crefmultiformat{equation}{Equations~\textup{#2(#1)#3}}{ and \textup{#2(#1)#3}}
{, \textup{#2(#1)#3}}{, and \textup{#2(#1)#3}}
\Crefrangemultiformat{equation}{Equations~\textup{#3(#1)#4--#5(#2)#6}}%
{ and \textup{#3(#1)#4--#5(#2)#6}}{, \textup{#3(#1)#4--#5(#2)#6}}{, and \textup{#3(#1)#4--#5(#2)#6}}

% Make number non-italic in any environment.
\crefdefaultlabelformat{#2\textup{#1}#3}

% Environments
\makeatother
% For box around Definition, Theorem, \ldots
%%fakesection Theorems
\usepackage{thmtools}
\usepackage[framemethod=TikZ]{mdframed}

\theoremstyle{definition}
\mdfdefinestyle{mdbluebox}{%
	roundcorner = 10pt,
	linewidth=1pt,
	skipabove=12pt,
	innerbottommargin=9pt,
	skipbelow=2pt,
	nobreak=true,
	linecolor=blue,
	backgroundcolor=TealBlue!5,
}
\declaretheoremstyle[
	headfont=\sffamily\bfseries\color{MidnightBlue},
	mdframed={style=mdbluebox},
	headpunct={\\[3pt]},
	postheadspace={0pt}
]{thmbluebox}

\mdfdefinestyle{mdredbox}{%
	linewidth=0.5pt,
	skipabove=12pt,
	frametitleaboveskip=5pt,
	frametitlebelowskip=0pt,
	skipbelow=2pt,
	frametitlefont=\bfseries,
	innertopmargin=4pt,
	innerbottommargin=8pt,
	nobreak=false,
	linecolor=RawSienna,
	backgroundcolor=Salmon!5,
}
\declaretheoremstyle[
	headfont=\bfseries\color{RawSienna},
	mdframed={style=mdredbox},
	headpunct={\\[3pt]},
	postheadspace={0pt},
]{thmredbox}

\declaretheorem[%
style=thmbluebox,name=Theorem,numberwithin=section]{thm}
\declaretheorem[style=thmbluebox,name=Lemma,sibling=thm]{lem}
\declaretheorem[style=thmbluebox,name=Proposition,sibling=thm]{prop}
\declaretheorem[style=thmbluebox,name=Corollary,sibling=thm]{coro}
\declaretheorem[style=thmredbox,name=Example,sibling=thm]{eg}

\mdfdefinestyle{mdgreenbox}{%
	roundcorner = 10pt,
	linewidth=1pt,
	skipabove=12pt,
	innerbottommargin=9pt,
	skipbelow=2pt,
	nobreak=true,
	linecolor=ForestGreen,
	backgroundcolor=ForestGreen!5,
}

\declaretheoremstyle[
	headfont=\bfseries\sffamily\color{ForestGreen!70!black},
	bodyfont=\normalfont,
	spaceabove=2pt,
	spacebelow=1pt,
	mdframed={style=mdgreenbox},
	headpunct={ --- },
]{thmgreenbox}

\declaretheorem[style=thmgreenbox,name=Definition,sibling=thm]{defn}

\mdfdefinestyle{mdgreenboxsq}{%
	linewidth=1pt,
	skipabove=12pt,
	innerbottommargin=9pt,
	skipbelow=2pt,
	nobreak=true,
	linecolor=ForestGreen,
	backgroundcolor=ForestGreen!5,
}
\declaretheoremstyle[
	headfont=\bfseries\sffamily\color{ForestGreen!70!black},
	bodyfont=\normalfont,
	spaceabove=2pt,
	spacebelow=1pt,
	mdframed={style=mdgreenboxsq},
	headpunct={},
]{thmgreenboxsq}
\declaretheoremstyle[
	headfont=\bfseries\sffamily\color{ForestGreen!70!black},
	bodyfont=\normalfont,
	spaceabove=2pt,
	spacebelow=1pt,
	mdframed={style=mdgreenboxsq},
	headpunct={},
]{thmgreenboxsq*}

\mdfdefinestyle{mdblackbox}{%
	skipabove=8pt,
	linewidth=3pt,
	rightline=false,
	leftline=true,
	topline=false,
	bottomline=false,
	linecolor=black,
	backgroundcolor=RedViolet!5!gray!5,
}
\declaretheoremstyle[
	headfont=\bfseries,
	bodyfont=\normalfont\small,
	spaceabove=0pt,
	spacebelow=0pt,
	mdframed={style=mdblackbox}
]{thmblackbox}

\theoremstyle{plain}
\declaretheorem[name=Question,sibling=thm,style=thmblackbox]{ques}
\declaretheorem[name=Remark,sibling=thm,style=thmgreenboxsq]{remark}
\declaretheorem[name=Remark,sibling=thm,style=thmgreenboxsq*]{remark*}
\newtheorem{ass}[thm]{Assumptions}

\theoremstyle{definition}
\newtheorem*{problem}{Problem}
\newtheorem{claim}[thm]{Claim}
\theoremstyle{remark}
\newtheorem*{case}{Case}
\newtheorem*{notation}{Notation}
\newtheorem*{note}{Note}
\newtheorem*{motivation}{Motivation}
\newtheorem*{intuition}{Intuition}
\newtheorem*{conjecture}{Conjecture}

% Make section starts with 1 for report type
%\renewcommand\thesection{\arabic{section}}

% End example and intermezzo environments with a small diamond (just like proof
% environments end with a small square)
\usepackage{etoolbox}
\AtEndEnvironment{vb}{\null\hfill$\diamond$}%
\AtEndEnvironment{intermezzo}{\null\hfill$\diamond$}%
% \AtEndEnvironment{opmerking}{\null\hfill$\diamond$}%

% Fix some spacing
% http://tex.stackexchange.com/questions/22119/how-can-i-change-the-spacing-before-theorems-with-amsthm
\makeatletter
\def\thm@space@setup{%
  \thm@preskip=\parskip \thm@postskip=0pt
}

% Fix some stuff
% %http://tex.stackexchange.com/questions/76273/multiple-pdfs-with-page-group-included-in-a-single-page-warning
\pdfsuppresswarningpagegroup=1


% My name
\author{Jaden Wang}



\begin{document}
\centerline {\textsf{\textbf{\LARGE{Homework 9}}}}
\centerline {Jaden Wang}
\vspace{.15in}
\begin{problem}[4.6]
Let $ M$ be a Riemannian manifold.  $ M$ is a locally symmetric space if  $ \nabla R =0$, where $ R$ is the curvature tensor of  $ M$.
 \begin{enumerate}[label=(\alph*)]
	 \item Let $ \gamma:[0,\ell) \to M$ be a geodesic of $ M$. Let  $ X,Y,Z$ be parallel vector fields along  $ \gamma$. Prove that $ R(X,Y)Z$ is a parallel field along  $ \gamma$.
	 \item Prove that if $ M$ is locally symmetric, connected, and has dimension two, then  $ M$ has constant sectional curvature.
	 \item Prove that if $ M$ has constant (sectional) curvature, then  $ M$ is a locally symmetric space.
\end{enumerate}
\end{problem}
\begin{proof}
\begin{enumerate}[label=(\alph*)]
	\item Let $ V:= \dot{\gamma}$ be the velocity field along $ \gamma$. Since $ X,Y,Z$ are parallel along $ \gamma$, we have $ \nabla _V X = \nabla _V Y = \nabla _V Z = 0$ along $ \gamma$. Let $ W$ be any vector field, then tensor derivative yields
\begin{align*}
\nabla R(X,Y,Z,W,V) &= V(R(X,Y,Z,W)) - R(\nabla_V X,Y,Z,W) - R(X,\nabla_V Y, Z,W) \\
		     & \quad - R(X,Y,\nabla_VZ , W) - R(X,Y,Z,\nabla_VW) \\	
 &= V(R(X,Y,Z,W)) - R(0,Y,Z,W) - R(X,0, Z,W)\\
 & \quad - R(X,Y,0 , W) - R(X,Y,Z,\nabla_VW) \\	
 &= V(R(X,Y,Z,W)) - R(X,Y,Z,\nabla_VW) =0 .
\end{align*}
This yields $ V(R(X,Y,Z,W)) = R(X,Y,Z,\nabla_VW)$. Recall the Leibniz rule for the metric:
\begin{align*}
	V\left( \left\langle R(X,Y)Z,W \right\rangle \right) &= \left\langle \nabla_V (R(X,Y)Z), W \right\rangle + \left\langle R(X,Y)Z,\nabla_VW \right\rangle .
\end{align*}
It follows that $ \left\langle \nabla_V (R(X,Y)Z), W \right\rangle =0$. Since $ W$ is arbitrary, it must be that  $ \nabla_V (R(X,Y)Z) = 0$, proving that $ R(X,Y)Z$ is parallel along $ \gamma$.
\item First, let $ c = K(p)$ for any  $ p \in M$. Now define $ A = \{p \in M: K(p) = c\}$. Thus we know that $ A$ is not empty. Since  $ K$ is a smooth function on  $ M$,  $ A$ is the preimage of a single point $ c$ and therefore is closed in $ M$. To prove that  $ M$ has constant sectional curvature, it suffices to show that  $ A$ is open as well so we must have $ A = M$. That is, we wish to show that any $ p \in A$ has an open neighborhood  $ U$ such that  $ U \subseteq A$.

	Consider the normal neighborhood $ U$ of $p$ with geodesic frame  $ \{E_1,E_2\} $. Since $ M$ has dimension two, this frame spans the entire  $ TU$ so we only need to show $ K$ is constant in  $ U$ under this frame. Let $ q \in U$ and let $ \gamma$ be a geodesic connecting $ p$ and $ q$. By definition of geodesic frame, $ E_1$ and $ E_2$ are parallel fields along $ \gamma$. Since $ M$ is locally symmetric, it follows from part (a) that $ R(E_1,E_2)E_1$ is a parallel field along $ \gamma $ as well. Since $ K(p) = c$, we have
\begin{align*}
	K(p) &=	\frac{\left\langle R(E_1,E_2)E_1(p), E_2(p) \right\rangle}{ \norm{ E_1(p)}^2 \norm{ E_2(p)}^2 - \left\langle E_1(p),E_2(p) \right\rangle^2   } \\
	     &=	\frac{\left\langle R(E_1,E_2)E_1(q), E_2(q) \right\rangle}{ \norm{ E_1(q)}^2 \norm{ E_2(q)}^2 - \left\langle E_1(q),E_2(q) \right\rangle^2 } && \text{parallel transport is isometry}  \\
	 &= K(q) = c .
\end{align*}
We conclude that $ U \subseteq A$ and thus $ A$ is open as desired.
\item Since $ M$ has constant sectional curvature $ K_0$, by Lemma 3.4, $ R = K_0 R'$, where
	\begin{align*}
		\left\langle R'(X,Y,Z),W \right\rangle = \left\langle X,Z \right\rangle \left\langle Y,W \right\rangle - \left\langle Y,Z \right\rangle \left\langle X, W\right\rangle.
	\end{align*}
\begin{claim}
For any $ V \in \mathfrak{X}(M)$, $ \nabla_V R' =0$.
\end{claim}
The proof is a straightforward computation:
\begin{align*}
	V(R'(X,Y,Z,W)) &= V[\left\langle X,Z \right\rangle \left\langle Y,W \right\rangle - \left\langle Y,Z \right\rangle \left\langle X,W \right\rangle ] \\
		      &=  \left[ \left\langle \nabla_VX,Z \right\rangle+ \left\langle X, \nabla_VZ \right\rangle \right] \left\langle Y,W \right\rangle + \left\langle X,Z \right\rangle \left[ \left\langle \nabla_VY,W \right\rangle + \left\langle Y,\nabla_VW \right\rangle \right] \\
		      &\quad  - [\left\langle \nabla_VY,Z \right\rangle + \left\langle Y,\nabla_VZ \right\rangle] \left\langle X,W \right\rangle - \left\langle Y,Z \right\rangle [\left\langle \nabla_VX,W \right\rangle + \left\langle X,\nabla_VW \right\rangle]  \\
		      &= \left\langle \nabla_VX,Z \right\rangle \left\langle Y,W \right\rangle+ \left\langle X, \nabla_VZ \right\rangle  \left\langle Y,W \right\rangle \\
		      & \quad + \left\langle X,Z \right\rangle  \left\langle \nabla_VY,W \right\rangle + \left\langle X,Z \right\rangle\left\langle Y,\nabla_VW \right\rangle  \\
		      & \quad - \left\langle \nabla_VY,Z \right\rangle \left\langle X,W \right\rangle - \left\langle Y,\nabla_VZ \right\rangle\left\langle X,W \right\rangle \\
		      &\quad - \left\langle Y,Z \right\rangle\left\langle \nabla_VX,W \right\rangle - \left\langle Y,Z \right\rangle \left\langle X,\nabla_VW \right\rangle  \\
	&= R'(\nabla_V X,Y,Z,W) + R'(X,\nabla_VY,Z,W) \\
	&\quad + R'(X,Y,\nabla_VZ,W) + R'(X,Y,Z,\nabla_VW)   .
\end{align*}
The claim follows immediately. Using this claim, we obtain
\begin{align*}
	V(R(X,Y,Z,W)) &= K_0 V(R'(X,Y,Z,W)) \\
	&= K_0[ R'(\nabla_V X,Y,Z,W) + R'(X,\nabla_VY,Z,W) \\
	&\quad + R'(X,Y,\nabla_VZ,W) + R'(X,Y,Z,\nabla_VW) ] \\
	&= R(\nabla_V X,Y,Z,W) + R(X,\nabla_VY,Z,W) \\
	&\quad + R(X,Y,\nabla_VZ,W) + R(X,Y,Z,\nabla_VW)  .
\end{align*}
It follows that $ \nabla R(X,Y,Z,W,V) = V(R(X,Y,Z,W)) - (R(\nabla_V X,Y,Z,W) + R(X,\nabla_VY,Z,W)+ R(X,Y,\nabla_VZ,W) + R(X,Y,Z,\nabla_VW)) =0  $. That is, $ M$ is locally symmetric.
\end{enumerate}
\end{proof}

\begin{problem}[4.8 (Schur's Theorem)]
Let $ M^{n}$ be a connected Riemmannian manifold with $ n \geq 3$. Suppose that  $ M$ is isotropic, that is, for each $ p \in M$, the sectional curvature $ K(p, \sigma)$ does not depend on $ \sigma \subseteq T_pM$. Prove that $ M$ has constant sectional curvature, that is,  $ K(p, \sigma)$ also does not depend on $ p$. 
\end{problem}
\begin{proof}
By the proof of Lemma 3.4, since $ K(p)$ is independent of  $ \sigma \subset T_pM$, we have $ R(p) = K(p) R'(p)$. By the claim from previous problem, we know that for any $ V \in \mathfrak{X}(M)$, $ \nabla_V R' = 0$. It follows that $ \nabla_V (R) = \nabla_V(KR') = V(K)R' + K \nabla_VR' = V(K)R'$. Now consider the 2nd Bianchi identity:
\begin{align*}
	0&=\nabla R(X,Y,Z,W,V) + \nabla R(X,Y,W,V,Z) + \nabla R(X,Y,V,Z,W) \\
	0&=V(K)R'(X,Y,Z,W) + Z(K)R'(X,Y,W,V) + W(K) R'(X,Y,V,Z) \\
	0&=V(K)[\left\langle X,Z \right\rangle \left\langle Y,W \right\rangle - \left\langle Y,Z \right\rangle \left\langle X,W \right\rangle]\\
	 & \quad +Z(K) [\left\langle X,W \right\rangle \left\langle Y,V \right\rangle - \left\langle Y,W \right\rangle \left\langle X,V \right\rangle]  \\
	 &\quad + W(K) [\left\langle X,V \right\rangle \left\langle Y,Z \right\rangle - \left\langle Y,V \right\rangle \left\langle X,Z \right\rangle]  .
\end{align*}
Since $ n \geq 3$, for any  $Z$ we can find  $ W,Y$ s.t.\ $ \left\langle Z,W \right\rangle = \left\langle Y,W \right\rangle = \left\langle Y,Z \right\rangle = 0$ ( \emph{i.e.} $ Z,W,Y$ are linearly independent) and $ \left\langle Y,Y \right\rangle \equiv 1$. The equation becomes
\begin{align*}
	0 &= Z(K) \left\langle X,W \right\rangle - W(K) \left\langle X,Z \right\rangle \\
	0&= \left\langle Z(K)W- W(K)Z, X \right\rangle \\
	0 &= Z(K)W-W(K)Z && X \text{ is arbitrary} \\
	0 &= Z(K) := dK(Z) && Z,W \text{ linearly independent}  .
\end{align*}
Since $ Z$ is arbitrary, we conclude that  $ dK \equiv 0$. That is,  $ K$ is constant everywhere.
\end{proof}
\begin{problem}[4.9]
Prove that the scalar curvature $ K(p)$ at  $  p \in M$ is given by
\begin{align*}
	K(p) = \frac{1}{\omega_{n-1}} \int_{S^{n-1}} \Ric_p(x) dS^{n-1},
\end{align*}
where $ \omega_{n-1}$ is the area of the sphere $ S^{n-1}$ in $ T_pM$ and  $ dS^{n-1}$ is the area elements on $ S^{n-1}$.
\end{problem}

\begin{proof}
Fix a $ p \in M$. Recall the symmetric bilinear form $ Q(x,y)$ that is the trace of the linear map $ z \mapsto R_p(x,z)y$. We know that there is a real symmetric matrix $ A$  s.t.\ $ \frac{1}{n-1}Q(x,y) = \left\langle Ax, y  \right\rangle$. Spectral Theorem yields an orthonormal eigenbasis $ \{e_i\} $ of $ A$ with corresponding real eigenvalues $ \lambda_i$.  Thus, if a unit vector $ x = x^{i} e_i$, we have $ \Ric_p(x) =  \frac{1}{n-1}Q(x,x) = \lambda^i x_{i}^2$. Moreover, $ x$ is a unit normal vector to  $ S^{n-1}$. Let $ V = \lambda^{i}x^{i} e_i $, with $ \Div V = \sum_{ i= 1}^{ n} \lambda_i $, we obtain
\begin{align*}
	\frac{1}{\omega_{n-1}} \int_{S^{n-1}} \Ric_p(x) dS^{n-1} &= \frac{1}{\omega_{n-1}} \int_{S^{n-1}} \lambda^{i} x_i^2 dS^{n-1}\\
	&= \frac{1}{\omega_{n-1}} \int_{S^{n-1}} \left\langle V, x \right\rangle dS^{n-1} \\
	&= \frac{1}{\omega_{n-1}} \int_{B^{n}} \Div V \ dB^{n}  && \text{Stokes Theorem} \\
	&= \frac{ \Div V }{ \omega_{n-1}} \int_{B^{n}} dB^{n}  \\
	&= \frac{ \sum_{ i= 1}^{ n}  \lambda_i}{n}  && area(S^{n-1}) = \frac{d (R^{n}vol(B^{n}))}{d R}\bigg|_{R=1} = n \cdot  vol(B^{n}) \\
	&= \frac{ \sum_{ i= 1}^{ n} \Ric_p(e_i)}{n}  \\
	&= K(p) .
\end{align*}
\end{proof}
\end{document}
