\documentclass[12pt]{article}
\newcommand{\alert}[1]{{\bf \color{red} [Alert:] #1}}
\newcommand{\todo}[1]{{\bf \color{orange} [TODO:] #1}}
\newcommand{\real}[1][]{\mathbb{R}^{#1}}
\newcommand{\myeqn}[1]{(\ref{#1})}
\newcommand{\myex}[1]{Example \ref{#1}}
\newcommand{\defeq}{\stackrel{\mathrm{def}}{=}}
\newcommand{\parder}[2]{\frac{\partial #1}{\partial #2}}
\newcommand{\Lie}[3][]{\mathsf{L}_{#3}^{#1} #2}
\newcommand{\LieA}[1]{\mathsf{Lie}(#1)}
\newcommand{\lieder}[2]{\mathcal{L}_{#2} #1}
\renewcommand{\t}{^{\mbox{\tiny\sf T}}}
\newcommand{\trans}{^{\mbox{\tiny\sf T}}}
\newcommand{\markup}[1]{\{\textbf{#1}\}}
\newcommand{\msub}[1]{_\mathrm{#1}}
\newcommand{\msup}[1]{^\mathrm{#1}}
\newcommand{\inv}[1]{#1^{-1}}
\newcommand{\pinv}[1]{{#1}^{+}}
\newcommand{\myfracA}[2]{\displaystyle{\frac{#1}{#2}}}
\newcommand{\myfracB}[2]{{#1}/{#2}}
\newcommand{\mydiffA}[1]{\dot{#1}}
\newcommand{\mydiffB}[2]{\myfracA{\mathrm{d}{#1}}{\mathrm{d}{#2}}}
\newcommand{\ball}[2]{\mathcal{B}_{#1}\left(#2\right)}
\newcommand{\acos}[1]{\cos^{-1}\left(#1\right)}
\newcommand{\asin}[1]{\sin^{-1}\left(#1\right)}
\newcommand{\mani}{\mathcal{M}}
\newcommand{\tang}[2]{\mathsf{T}_{#1} #2}
\newcommand{\LieB}[2]{[ #1, #2 ]}
\newcommand{\LieBad}[3][]{\mathsf{ad}_{#2}^{#1} #3}
\newcommand{\ReachVT}{\mathcal{R}^V_T}
\newcommand{\ReachVt}{\mathcal{R}^V_t}
\newcommand{\ReachVTe}{\mathcal{R}^V_{\le T}}
\newcommand{\ReachT}{\mathcal{R}_T}
\newcommand{\Reacht}{\mathcal{R}_t}
\newcommand{\ReachTe}{\mathcal{R}_{\le T}}
\newcommand{\accLA}[1]{\mathsf{Lie}(#1)}
\newcommand{\accD}{\Delta_{\mathcal{F}}}
\newcommand{\accSA}{\mathsf{Lie}(\mathcal{G},f)}
\newcommand{\accDS}{\Delta_{\mathcal{G}}}
\newcommand{\eval}[3]{\mathsf{Ev}^{#2}_{#1}\left( #3 \right)}
\newcommand{\stlc}{\textsc{stlc}}
\newcommand{\clf}{\textsc{clf}}
\newcommand{\jqlf}{\textsc{jqlf}}
\newcommand{\dlas}{\textsc{dlas}}
\newcommand{\Ad}[2]{\mathsf{Ad}_{#1} #2}
\newcommand{\xe}{\ensuremath{x_e}}
\newcommand{\lebg}[1]{\mathcal{L}_{#1}}
\newcommand{\lebgx}[1]{\mathcal{L}_{#1 \mathrm{e}}}
\newcommand{\dom}{D}
\newcommand{\domT}{[t_0,\infty) \times D}
\newcommand{\rarrow}{\rightarrow}
\renewcommand{\d}{\mathrm{d}}
\renewcommand{\Re}{\mathbb{R}}
\newcommand{\C}{\mathrm{C}}

\newcommand{\QED}{{\unskip\nobreak\hfil\penalty50\hskip2em\vadjust{}
		\nobreak\hfil$\Box$\parfillskip=0pt\finalhyphendemerits=0\par}\vspace{0.1cm}}
\newcommand{\eoEx}{{\unskip\nobreak\hfil\penalty50\hskip0em\vadjust{}
		\nobreak\hfil$\Large\Diamond$\parfillskip=0pt\finalhyphendemerits=0\par}\vspace{0.1cm}}

\newcommand{\sgn}{\ensuremath{\operatorname{sgn}}}
\newcommand{\sat}{\ensuremath{\operatorname{sat}}}

\newcommand{\half}{\frac{1}{2}}
\newcommand{\shalf}{\mbox{$\frac{1}{2}$}}
\newcommand{\marcom}[1]{\marginpar{\footnotesize #1}}
\newcommand{\der}{\mathrm{D}}
\newcommand{\e}{\mathrm{e}}
\newcommand{\dt}{\mathrm{d}t}

\newcommand{\cA}{\ensuremath{\mathcal{A}}}
\newcommand{\cB}{\ensuremath{\mathcal{B}}}
\newcommand{\cG}{\ensuremath{\mathcal{G}}}
\newcommand{\cK}{\ensuremath{\mathcal{K}}}
\newcommand{\cW}{\ensuremath{\mathcal{W}}}
\newcommand{\cZ}{\ensuremath{\mathcal{Z}}}
\newcommand{\cS}{\ensuremath{\mathcal{S}}}
\newcommand{\cD}{\ensuremath{\mathcal{D}}}
\newcommand{\cP}{\ensuremath{\mathcal{P}}}
\newcommand{\cV}{\ensuremath{\mathcal{V}}}
\newcommand{\cL}{\ensuremath{\mathcal{L}}}
\newcommand{\cN}{\ensuremath{\mathcal{N}}}
\newcommand{\cI}{\ensuremath{\mathcal{I}}}
\newcommand{\cR}{\ensuremath{\mathcal{R}}}
\newcommand{\cM}{\ensuremath{\mathcal{M}}}
\newcommand{\cC}{\ensuremath{\mathcal{C}}}
\newcommand{\cF}{\ensuremath{\mathcal{F}}}
\newcommand{\cH}{\ensuremath{\mathcal{H}}}
\newcommand{\cO}{\ensuremath{\mathcal{O}}}
\newcommand{\cX}{\ensuremath{\mathcal{X}}}
\newcommand{\cY}{\ensuremath{\mathcal{Y}}}
\newcommand{\Ci}{\ensuremath{\mathcal{C}^\infty}}
\newcommand{\ISS}{\textsc{iss}}
\newcommand{\LISS}{\textsc{liss}}
\newcommand{\GAS}{\textsc{gas}}
\newcommand{\GS}{\textsc{gs}}
\newcommand{\LES}{\textsc{les}}
\newcommand{\GUAS}{\textsc{guas}}
\newcommand{\BIBO}{\textsc{bibo}}
\newcommand{\spec}{\ensuremath{\operatorname{spec}}}
\newcommand{\spn}{\ensuremath{\operatorname{span}}}
\renewcommand{\i}{\mathrm{i\,}}

\renewcommand{\implies}{\Rightarrow}

\renewcommand{\theenumi}{$\roman{enumi})$}
\renewcommand{\labelenumi}{\theenumi}

\font\ptmten=zptmcmrm scaled 1200
\newcommand{\w}{\mbox{{\ptmten w}}}
\newcommand{\z}{\mbox{{\ptmten z}}}
\renewcommand{\Re}{\mathbb{R}}

\newcommand{\cl}{\operatorname{cl}}
\newcommand{\intr}{\operatorname{int}}
\newcommand{\rank}{\operatorname{rank}}
\newcommand{\co}{\operatorname{co}}
\newcommand{\aff}{\operatorname{aff}}

\theoremstyle{plain}
\newtheorem{theorem}{Theorem}[chapter]
\newtheorem{claim}[theorem]{Claim}
\newtheorem{corollary}[theorem]{Corollary}
\newtheorem{prop}[theorem]{Proposition}
\newtheorem{fact}[theorem]{Fact}
\newtheorem{lemma}[theorem]{Lemma}

\newtheorem{remark}{Remark}[chapter]

\theoremstyle{definition}
\newtheorem{assume}[theorem]{Assumption}
\newtheorem{defn}[theorem]{Definition}
\newtheorem{problem}[theorem]{Problem}
\newtheorem{exercise}{Exercise}
\newtheorem{example}[theorem]{Example}


\begin{document}
\centerline {\textsf{\textbf{\LARGE{Homework 9}}}}
\centerline {Jaden Wang}
\vspace{.15in}
\begin{problem}[4.6]
Let $ M$ be a Riemannian manifold.  $ M$ is a locally symmetric space if  $ \nabla R =0$, where $ R$ is the curvature tensor of  $ M$.
 \begin{enumerate}[label=(\alph*)]
	 \item Let $ \gamma:[0,\ell) \to M$ be a geodesic of $ M$. Let  $ X,Y,Z$ be parallel vector fields along  $ \gamma$. Prove that $ R(X,Y)Z$ is a parallel field along  $ \gamma$.
	 \item Prove that if $ M$ is locally symmetric, connected, and has dimension two, then  $ M$ has constant sectional curvature.
	 \item Prove that if $ M$ has constant (sectional) curvature, then  $ M$ is a locally symmetric space.
\end{enumerate}
\end{problem}
\begin{proof}
\begin{enumerate}[label=(\alph*)]
	\item Let $ V:= \dot{\gamma}$ be the velocity field along $ \gamma$. Since $ X,Y,Z$ are parallel along $ \gamma$, we have $ \nabla _V X = \nabla _V Y = \nabla _V Z = 0$ along $ \gamma$. Let $ W$ be any vector field, then tensor derivative yields
\begin{align*}
\nabla R(X,Y,Z,W,V) &= V(R(X,Y,Z,W)) - R(\nabla_V X,Y,Z,W) - R(X,\nabla_V Y, Z,W) \\
		     & \quad - R(X,Y,\nabla_VZ , W) - R(X,Y,Z,\nabla_VW) \\	
 &= V(R(X,Y,Z,W)) - R(0,Y,Z,W) - R(X,0, Z,W)\\
 & \quad - R(X,Y,0 , W) - R(X,Y,Z,\nabla_VW) \\	
 &= V(R(X,Y,Z,W)) - R(X,Y,Z,\nabla_VW) =0 .
\end{align*}
This yields $ V(R(X,Y,Z,W)) = R(X,Y,Z,\nabla_VW)$. Recall the Leibniz rule for the metric:
\begin{align*}
	V\left( \left\langle R(X,Y)Z,W \right\rangle \right) &= \left\langle \nabla_V (R(X,Y)Z), W \right\rangle + \left\langle R(X,Y)Z,\nabla_VW \right\rangle .
\end{align*}
It follows that $ \left\langle \nabla_V (R(X,Y)Z), W \right\rangle =0$. Since $ W$ is arbitrary, it must be that  $ \nabla_V (R(X,Y)Z) = 0$, proving that $ R(X,Y)Z$ is parallel along $ \gamma$.
\item First, let $ c = K(p)$ for any  $ p \in M$. Now define $ A = \{p \in M: K(p) = c\}$. Thus we know that $ A$ is not empty. Since  $ K$ is a smooth function on  $ M$,  $ A$ is the preimage of a single point $ c$ and therefore is closed in $ M$. To prove that  $ M$ has constant sectional curvature, it suffices to show that  $ A$ is open as well so we must have $ A = M$. That is, we wish to show that any $ p \in A$ has an open neighborhood  $ U$ such that  $ U \subseteq A$.

	Consider the normal neighborhood $ U$ of $p$ with geodesic frame  $ \{E_1,E_2\} $. Since $ M$ has dimension two, this frame spans the entire  $ TU$ so we only need to show $ K$ is constant in  $ U$ under this frame. Let $ q \in U$ and let $ \gamma$ be a geodesic connecting $ p$ and $ q$. By definition of geodesic frame, $ E_1$ and $ E_2$ are parallel fields along $ \gamma$. Since $ M$ is locally symmetric, it follows from part (a) that $ R(E_1,E_2)E_1$ is a parallel field along $ \gamma $ as well. Since $ K(p) = c$, we have
\begin{align*}
	K(p) &=	\frac{\left\langle R(E_1,E_2)E_1(p), E_2(p) \right\rangle}{ \norm{ E_1(p)}^2 \norm{ E_2(p)}^2 - \left\langle E_1(p),E_2(p) \right\rangle^2   } \\
	     &=	\frac{\left\langle R(E_1,E_2)E_1(q), E_2(q) \right\rangle}{ \norm{ E_1(q)}^2 \norm{ E_2(q)}^2 - \left\langle E_1(q),E_2(q) \right\rangle^2 } && \text{parallel transport is isometry}  \\
	 &= K(q) = c .
\end{align*}
We conclude that $ U \subseteq A$ and thus $ A$ is open as desired.
\item Since $ M$ has constant sectional curvature $ K_0$, by Lemma 3.4, $ R = K_0 R'$, where
	\begin{align*}
		\left\langle R'(X,Y,Z),W \right\rangle = \left\langle X,Z \right\rangle \left\langle Y,W \right\rangle - \left\langle Y,Z \right\rangle \left\langle X, W\right\rangle.
	\end{align*}
\begin{claim}
For any $ V \in \mathfrak{X}(M)$, $ \nabla_V R' =0$.
\end{claim}
The proof is a straightforward computation:
\begin{align*}
	V(R'(X,Y,Z,W)) &= V[\left\langle X,Z \right\rangle \left\langle Y,W \right\rangle - \left\langle Y,Z \right\rangle \left\langle X,W \right\rangle ] \\
		      &=  \left[ \left\langle \nabla_VX,Z \right\rangle+ \left\langle X, \nabla_VZ \right\rangle \right] \left\langle Y,W \right\rangle + \left\langle X,Z \right\rangle \left[ \left\langle \nabla_VY,W \right\rangle + \left\langle Y,\nabla_VW \right\rangle \right] \\
		      &\quad  - [\left\langle \nabla_VY,Z \right\rangle + \left\langle Y,\nabla_VZ \right\rangle] \left\langle X,W \right\rangle - \left\langle Y,Z \right\rangle [\left\langle \nabla_VX,W \right\rangle + \left\langle X,\nabla_VW \right\rangle]  \\
		      &= \left\langle \nabla_VX,Z \right\rangle \left\langle Y,W \right\rangle+ \left\langle X, \nabla_VZ \right\rangle  \left\langle Y,W \right\rangle \\
		      & \quad + \left\langle X,Z \right\rangle  \left\langle \nabla_VY,W \right\rangle + \left\langle X,Z \right\rangle\left\langle Y,\nabla_VW \right\rangle  \\
		      & \quad - \left\langle \nabla_VY,Z \right\rangle \left\langle X,W \right\rangle - \left\langle Y,\nabla_VZ \right\rangle\left\langle X,W \right\rangle \\
		      &\quad - \left\langle Y,Z \right\rangle\left\langle \nabla_VX,W \right\rangle - \left\langle Y,Z \right\rangle \left\langle X,\nabla_VW \right\rangle  \\
	&= R'(\nabla_V X,Y,Z,W) + R'(X,\nabla_VY,Z,W) \\
	&\quad + R'(X,Y,\nabla_VZ,W) + R'(X,Y,Z,\nabla_VW)   .
\end{align*}
The claim follows immediately. Using this claim, we obtain
\begin{align*}
	V(R(X,Y,Z,W)) &= K_0 V(R'(X,Y,Z,W)) \\
	&= K_0[ R'(\nabla_V X,Y,Z,W) + R'(X,\nabla_VY,Z,W) \\
	&\quad + R'(X,Y,\nabla_VZ,W) + R'(X,Y,Z,\nabla_VW) ] \\
	&= R(\nabla_V X,Y,Z,W) + R(X,\nabla_VY,Z,W) \\
	&\quad + R(X,Y,\nabla_VZ,W) + R(X,Y,Z,\nabla_VW)  .
\end{align*}
It follows that $ \nabla R(X,Y,Z,W,V) = V(R(X,Y,Z,W)) - (R(\nabla_V X,Y,Z,W) + R(X,\nabla_VY,Z,W)+ R(X,Y,\nabla_VZ,W) + R(X,Y,Z,\nabla_VW)) =0  $. That is, $ M$ is locally symmetric.
\end{enumerate}
\end{proof}

\begin{problem}[4.8 (Schur's Theorem)]
Let $ M^{n}$ be a connected Riemmannian manifold with $ n \geq 3$. Suppose that  $ M$ is isotropic, that is, for each $ p \in M$, the sectional curvature $ K(p, \sigma)$ does not depend on $ \sigma \subseteq T_pM$. Prove that $ M$ has constant sectional curvature, that is,  $ K(p, \sigma)$ also does not depend on $ p$. 
\end{problem}
\begin{proof}
By the proof of Lemma 3.4, since $ K(p)$ is independent of  $ \sigma \subset T_pM$, we have $ R(p) = K(p) R'(p)$. By the claim from previous problem, we know that for any $ V \in \mathfrak{X}(M)$, $ \nabla_V R' = 0$. It follows that $ \nabla_V (R) = \nabla_V(KR') = V(K)R' + K \nabla_VR' = V(K)R'$. Now consider the 2nd Bianchi identity:
\begin{align*}
	0&=\nabla R(X,Y,Z,W,V) + \nabla R(X,Y,W,V,Z) + \nabla R(X,Y,V,Z,W) \\
	0&=V(K)R'(X,Y,Z,W) + Z(K)R'(X,Y,W,V) + W(K) R'(X,Y,V,Z) \\
	0&=V(K)[\left\langle X,Z \right\rangle \left\langle Y,W \right\rangle - \left\langle Y,Z \right\rangle \left\langle X,W \right\rangle]\\
	 & \quad +Z(K) [\left\langle X,W \right\rangle \left\langle Y,V \right\rangle - \left\langle Y,W \right\rangle \left\langle X,V \right\rangle]  \\
	 &\quad + W(K) [\left\langle X,V \right\rangle \left\langle Y,Z \right\rangle - \left\langle Y,V \right\rangle \left\langle X,Z \right\rangle]  .
\end{align*}
Since $ n \geq 3$, for any  $Z$ we can find  $ W,Y$ s.t.\ $ \left\langle Z,W \right\rangle = \left\langle Y,W \right\rangle = \left\langle Y,Z \right\rangle = 0$ ( \emph{i.e.} $ Z,W,Y$ are linearly independent) and $ \left\langle Y,Y \right\rangle \equiv 1$. The equation becomes
\begin{align*}
	0 &= Z(K) \left\langle X,W \right\rangle - W(K) \left\langle X,Z \right\rangle \\
	0&= \left\langle Z(K)W- W(K)Z, X \right\rangle \\
	0 &= Z(K)W-W(K)Z && X \text{ is arbitrary} \\
	0 &= Z(K) := dK(Z) && Z,W \text{ linearly independent}  .
\end{align*}
Since $ Z$ is arbitrary, we conclude that  $ dK \equiv 0$. That is,  $ K$ is constant everywhere.
\end{proof}
\begin{problem}[4.9]
Prove that the scalar curvature $ K(p)$ at  $  p \in M$ is given by
\begin{align*}
	K(p) = \frac{1}{\omega_{n-1}} \int_{S^{n-1}} \Ric_p(x) dS^{n-1},
\end{align*}
where $ \omega_{n-1}$ is the area of the sphere $ S^{n-1}$ in $ T_pM$ and  $ dS^{n-1}$ is the area elements on $ S^{n-1}$.
\end{problem}

\begin{proof}
Fix a $ p \in M$. Recall the symmetric bilinear form $ Q(x,y)$ that is the trace of the linear map $ z \mapsto R_p(x,z)y$. We know that there is a real symmetric matrix $ A$  s.t.\ $ \frac{1}{n-1}Q(x,y) = \left\langle Ax, y  \right\rangle$. Spectral Theorem yields an orthonormal eigenbasis $ \{e_i\} $ of $ A$ with corresponding real eigenvalues $ \lambda_i$.  Thus, if a unit vector $ x = x^{i} e_i$, we have $ \Ric_p(x) =  \frac{1}{n-1}Q(x,x) = \lambda^i x_{i}^2$. Moreover, $ x$ is a unit normal vector to  $ S^{n-1}$. Let $ V = \lambda^{i}x^{i} e_i $, with $ \Div V = \sum_{ i= 1}^{ n} \lambda_i $, we obtain
\begin{align*}
	\frac{1}{\omega_{n-1}} \int_{S^{n-1}} \Ric_p(x) dS^{n-1} &= \frac{1}{\omega_{n-1}} \int_{S^{n-1}} \lambda^{i} x_i^2 dS^{n-1}\\
	&= \frac{1}{\omega_{n-1}} \int_{S^{n-1}} \left\langle V, x \right\rangle dS^{n-1} \\
	&= \frac{1}{\omega_{n-1}} \int_{B^{n}} \Div V \ dB^{n}  && \text{Stokes Theorem} \\
	&= \frac{ \Div V }{ \omega_{n-1}} \int_{B^{n}} dB^{n}  \\
	&= \frac{ \sum_{ i= 1}^{ n}  \lambda_i}{n}  && area(S^{n-1}) = \frac{d (R^{n}vol(B^{n}))}{d R}\bigg|_{R=1} = n \cdot  vol(B^{n}) \\
	&= \frac{ \sum_{ i= 1}^{ n} \Ric_p(e_i)}{n}  \\
	&= K(p) .
\end{align*}
\end{proof}
\end{document}
