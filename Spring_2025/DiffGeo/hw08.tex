\documentclass[12pt]{article}
\newcommand{\alert}[1]{{\bf \color{red} [Alert:] #1}}
\newcommand{\todo}[1]{{\bf \color{orange} [TODO:] #1}}
\newcommand{\real}[1][]{\mathbb{R}^{#1}}
\newcommand{\myeqn}[1]{(\ref{#1})}
\newcommand{\myex}[1]{Example \ref{#1}}
\newcommand{\defeq}{\stackrel{\mathrm{def}}{=}}
\newcommand{\parder}[2]{\frac{\partial #1}{\partial #2}}
\newcommand{\Lie}[3][]{\mathsf{L}_{#3}^{#1} #2}
\newcommand{\LieA}[1]{\mathsf{Lie}(#1)}
\newcommand{\lieder}[2]{\mathcal{L}_{#2} #1}
\renewcommand{\t}{^{\mbox{\tiny\sf T}}}
\newcommand{\trans}{^{\mbox{\tiny\sf T}}}
\newcommand{\markup}[1]{\{\textbf{#1}\}}
\newcommand{\msub}[1]{_\mathrm{#1}}
\newcommand{\msup}[1]{^\mathrm{#1}}
\newcommand{\inv}[1]{#1^{-1}}
\newcommand{\pinv}[1]{{#1}^{+}}
\newcommand{\myfracA}[2]{\displaystyle{\frac{#1}{#2}}}
\newcommand{\myfracB}[2]{{#1}/{#2}}
\newcommand{\mydiffA}[1]{\dot{#1}}
\newcommand{\mydiffB}[2]{\myfracA{\mathrm{d}{#1}}{\mathrm{d}{#2}}}
\newcommand{\ball}[2]{\mathcal{B}_{#1}\left(#2\right)}
\newcommand{\acos}[1]{\cos^{-1}\left(#1\right)}
\newcommand{\asin}[1]{\sin^{-1}\left(#1\right)}
\newcommand{\mani}{\mathcal{M}}
\newcommand{\tang}[2]{\mathsf{T}_{#1} #2}
\newcommand{\LieB}[2]{[ #1, #2 ]}
\newcommand{\LieBad}[3][]{\mathsf{ad}_{#2}^{#1} #3}
\newcommand{\ReachVT}{\mathcal{R}^V_T}
\newcommand{\ReachVt}{\mathcal{R}^V_t}
\newcommand{\ReachVTe}{\mathcal{R}^V_{\le T}}
\newcommand{\ReachT}{\mathcal{R}_T}
\newcommand{\Reacht}{\mathcal{R}_t}
\newcommand{\ReachTe}{\mathcal{R}_{\le T}}
\newcommand{\accLA}[1]{\mathsf{Lie}(#1)}
\newcommand{\accD}{\Delta_{\mathcal{F}}}
\newcommand{\accSA}{\mathsf{Lie}(\mathcal{G},f)}
\newcommand{\accDS}{\Delta_{\mathcal{G}}}
\newcommand{\eval}[3]{\mathsf{Ev}^{#2}_{#1}\left( #3 \right)}
\newcommand{\stlc}{\textsc{stlc}}
\newcommand{\clf}{\textsc{clf}}
\newcommand{\jqlf}{\textsc{jqlf}}
\newcommand{\dlas}{\textsc{dlas}}
\newcommand{\Ad}[2]{\mathsf{Ad}_{#1} #2}
\newcommand{\xe}{\ensuremath{x_e}}
\newcommand{\lebg}[1]{\mathcal{L}_{#1}}
\newcommand{\lebgx}[1]{\mathcal{L}_{#1 \mathrm{e}}}
\newcommand{\dom}{D}
\newcommand{\domT}{[t_0,\infty) \times D}
\newcommand{\rarrow}{\rightarrow}
\renewcommand{\d}{\mathrm{d}}
\renewcommand{\Re}{\mathbb{R}}
\newcommand{\C}{\mathrm{C}}

\newcommand{\QED}{{\unskip\nobreak\hfil\penalty50\hskip2em\vadjust{}
		\nobreak\hfil$\Box$\parfillskip=0pt\finalhyphendemerits=0\par}\vspace{0.1cm}}
\newcommand{\eoEx}{{\unskip\nobreak\hfil\penalty50\hskip0em\vadjust{}
		\nobreak\hfil$\Large\Diamond$\parfillskip=0pt\finalhyphendemerits=0\par}\vspace{0.1cm}}

\newcommand{\sgn}{\ensuremath{\operatorname{sgn}}}
\newcommand{\sat}{\ensuremath{\operatorname{sat}}}

\newcommand{\half}{\frac{1}{2}}
\newcommand{\shalf}{\mbox{$\frac{1}{2}$}}
\newcommand{\marcom}[1]{\marginpar{\footnotesize #1}}
\newcommand{\der}{\mathrm{D}}
\newcommand{\e}{\mathrm{e}}
\newcommand{\dt}{\mathrm{d}t}

\newcommand{\cA}{\ensuremath{\mathcal{A}}}
\newcommand{\cB}{\ensuremath{\mathcal{B}}}
\newcommand{\cG}{\ensuremath{\mathcal{G}}}
\newcommand{\cK}{\ensuremath{\mathcal{K}}}
\newcommand{\cW}{\ensuremath{\mathcal{W}}}
\newcommand{\cZ}{\ensuremath{\mathcal{Z}}}
\newcommand{\cS}{\ensuremath{\mathcal{S}}}
\newcommand{\cD}{\ensuremath{\mathcal{D}}}
\newcommand{\cP}{\ensuremath{\mathcal{P}}}
\newcommand{\cV}{\ensuremath{\mathcal{V}}}
\newcommand{\cL}{\ensuremath{\mathcal{L}}}
\newcommand{\cN}{\ensuremath{\mathcal{N}}}
\newcommand{\cI}{\ensuremath{\mathcal{I}}}
\newcommand{\cR}{\ensuremath{\mathcal{R}}}
\newcommand{\cM}{\ensuremath{\mathcal{M}}}
\newcommand{\cC}{\ensuremath{\mathcal{C}}}
\newcommand{\cF}{\ensuremath{\mathcal{F}}}
\newcommand{\cH}{\ensuremath{\mathcal{H}}}
\newcommand{\cO}{\ensuremath{\mathcal{O}}}
\newcommand{\cX}{\ensuremath{\mathcal{X}}}
\newcommand{\cY}{\ensuremath{\mathcal{Y}}}
\newcommand{\Ci}{\ensuremath{\mathcal{C}^\infty}}
\newcommand{\ISS}{\textsc{iss}}
\newcommand{\LISS}{\textsc{liss}}
\newcommand{\GAS}{\textsc{gas}}
\newcommand{\GS}{\textsc{gs}}
\newcommand{\LES}{\textsc{les}}
\newcommand{\GUAS}{\textsc{guas}}
\newcommand{\BIBO}{\textsc{bibo}}
\newcommand{\spec}{\ensuremath{\operatorname{spec}}}
\newcommand{\spn}{\ensuremath{\operatorname{span}}}
\renewcommand{\i}{\mathrm{i\,}}

\renewcommand{\implies}{\Rightarrow}

\renewcommand{\theenumi}{$\roman{enumi})$}
\renewcommand{\labelenumi}{\theenumi}

\font\ptmten=zptmcmrm scaled 1200
\newcommand{\w}{\mbox{{\ptmten w}}}
\newcommand{\z}{\mbox{{\ptmten z}}}
\renewcommand{\Re}{\mathbb{R}}

\newcommand{\cl}{\operatorname{cl}}
\newcommand{\intr}{\operatorname{int}}
\newcommand{\rank}{\operatorname{rank}}
\newcommand{\co}{\operatorname{co}}
\newcommand{\aff}{\operatorname{aff}}

\theoremstyle{plain}
\newtheorem{theorem}{Theorem}[chapter]
\newtheorem{claim}[theorem]{Claim}
\newtheorem{corollary}[theorem]{Corollary}
\newtheorem{prop}[theorem]{Proposition}
\newtheorem{fact}[theorem]{Fact}
\newtheorem{lemma}[theorem]{Lemma}

\newtheorem{remark}{Remark}[chapter]

\theoremstyle{definition}
\newtheorem{assume}[theorem]{Assumption}
\newtheorem{defn}[theorem]{Definition}
\newtheorem{problem}[theorem]{Problem}
\newtheorem{exercise}{Exercise}
\newtheorem{example}[theorem]{Example}


\begin{document}
\centerline {\textsf{\textbf{\LARGE{Homework 8}}}}
\centerline {Jaden Wang}
\vspace{.15in}
\begin{problem}[1]
Compute the coefficients of the Riemann curvature tensor $ R$ in terms of the Christoffel symbols  $ \Gamma_{ij}^{k}$.
\end{problem}
\begin{proof}
	We use Einstein notation throughout.
\begin{align*}
	R_{ijk}^{ \ell} \frac{\partial }{\partial x_\ell} &=  R\left( \frac{\partial }{\partial x_i} , \frac{\partial }{\partial x_j}  \right) \frac{\partial }{\partial x_k}   \\
							  &= \left( \nabla _{\frac{\partial }{\partial x_j} } \nabla_{ \frac{\partial }{\partial x_i}}- \nabla_{ \frac{\partial }{\partial x_i}} \nabla_{ \frac{\partial }{\partial x_j} } + \nabla _{ \left[ \frac{\partial }{\partial x_i} ,\frac{\partial }{\partial x_j}  \right]}  \right) \frac{\partial }{\partial x_k} \\
							  &= \nabla _{ \frac{\partial }{\partial x_j} } \nabla_{ \frac{\partial }{\partial x_i}} \frac{\partial }{\partial x_k} - \nabla_{ \frac{\partial }{\partial x_i}} \nabla_{ \frac{\partial }{\partial x_j} }\frac{\partial }{\partial x_k}   \\
							  &= \nabla _{ \frac{\partial }{\partial x_j} } \left( \Gamma_{ k i}^{ s} \frac{\partial }{\partial x_s}  \right) - \nabla _{ \frac{\partial }{\partial x_i} } \left( \Gamma_{ k j}^{ s} \frac{\partial }{\partial x_s}  \right)   \\
							  &= \frac{\partial \Gamma_{ k i}^{ s} }{\partial x_j} \frac{\partial }{\partial x_s} + \Gamma_{ k i}^{ s} \Gamma_{ sj}^{ \ell} \frac{\partial }{\partial x_\ell} -  \frac{\partial \Gamma_{ k j}^{ s} }{\partial x_i} \frac{\partial }{\partial x_s} - \Gamma_{ k j}^{ s} \Gamma_{ si}^{ \ell} \frac{\partial }{\partial x_\ell}  && \text{Leibniz rule}  \\
							  &=  \frac{\partial \Gamma_{ k i}^{ \ell} }{\partial x_j} \frac{\partial }{\partial x_\ell} + \Gamma_{ k i}^{ s} \Gamma_{ sj}^{ \ell} \frac{\partial }{\partial x_\ell} -  \frac{\partial \Gamma_{ k j}^{ \ell} }{\partial x_i} \frac{\partial }{\partial x_\ell} - \Gamma_{ k j}^{ s} \Gamma_{ si}^{ \ell} \frac{\partial }{\partial x_\ell} && \text{reindexing}  \\
							  &= \left(  \frac{\partial \Gamma_{ ik}^{ \ell} }{\partial x_j} -  \frac{\partial \Gamma_{ jk}^{ \ell} }{\partial x_i} + \Gamma_{ ik}^{ s} \Gamma_{ js}^{ \ell} - \Gamma_{ jk}^{ s} \Gamma_{ is}^{ \ell} \right)  \frac{\partial }{\partial x_\ell}. && \nabla \text{ symmetric} 
\end{align*}
Hence, the coefficients of the Riemann curvature tensor is
\begin{align*}
	R_{ijk}^{ \ell} =   \frac{\partial \Gamma_{ ik}^{ \ell} }{\partial x_j} -  \frac{\partial \Gamma_{ jk}^{ \ell} }{\partial x_i}  + \Gamma_{ ik}^{ s} \Gamma_{ js}^{ \ell} - \Gamma_{ jk}^{ s} \Gamma_{ is}^{ \ell}. 
\end{align*}
\end{proof}
\begin{problem}[2]
Show that the curvature of $ \rr^{n}$ is zero by (i) using the formula from the last exercise, and (ii) using the abstract definition of  $ R$ in terms of the covariant derivative  $ \nabla $.
\end{problem}
\begin{proof}
\begin{enumerate}[label=(\roman*)]
	\item 	Recall that
\begin{align*}
	\Gamma_{ i j}^{ k} = \frac{1}{2} g^{sk} \left( - \frac{\partial g_{ij}}{\partial x_s} + \frac{\partial g_{js}}{\partial x_i} + \frac{\partial g_{si}}{\partial x_j}    \right) .
\end{align*}
In $ \rr^{n}$, $ g^{sk} = g_{sk}^{-1} = \delta_{sk}$. Thus
\begin{align*}
	\Gamma_{ i j}^{ k} = \frac{1}{2} \left( - \frac{\partial g_{ii}}{\partial x_i} + \frac{\partial g_{ii}}{\partial x_i} + \frac{\partial g_{ii}}{\partial x_i}    \right) = 0 . 
\end{align*}
We immediately have
\begin{align*}
	R_{ijk}^{\ell} = 0 .
\end{align*}
Therefore, the curvature is zero.
\item In $ \rr^{n}$, a global canonical basis enables $ \nabla_X Z = (X(Z^{1},\ldots,X(Z^{n})) =: X(Z)$. Thus we have
\begin{align*}
	R(X,Y)Z&=(\nabla_Y \nabla_X - \nabla _X \nabla _Y  + \nabla _{[X,Y]})Z \\
	&= YX(Z) - XY(Z) + XY(Z) - YX(Z) = 0  . 
\end{align*}
\end{enumerate}
\end{proof}
\begin{problem}[3]
Compute the curvature of the hyperbolic plane $ H^2$ using the formula from the first exercise.
\end{problem}
\begin{proof}
Recall that the metric tensor of $ H^2$ is $ g_{ij} = \delta_{ij} /y^2$. Thus $ g^{ij} = \delta_{ij} y^2$, and 
\begin{align*}
	\Gamma_{ 1 1}^{ 1} &= 0 \\
	\Gamma_{ 1 2}^{ 1} &= \Gamma_{ 2 1}^{ 1}  = \frac{1}{2} y^2\left(-0 - \frac{2}{y^3} + 0  \right) +0  = -\frac{1}{y}\\
	\Gamma_{ 2 2}^{ 1} &= \frac{1}{2} y^2 \left( -0 + 0 + 0 \right) = 0  \\
	\Gamma_{ 1 1}^{ 2} &= 0+\frac{1}{2} y^2 \left( \frac{2}{y^3} +0 + 0 \right) = \frac{1}{y}  \\
	\Gamma_{ 1 2}^{ 2} &= \Gamma_{ 2 1}^{ 2} = 0+ \frac{1}{2} y^2 \left( -0 +0 +0 \right) = 0\\
	\Gamma_{ 2 2}^{ 2} &= 0+ \frac{1}{2} y^2 \left( \frac{2}{y^3} - \frac{2}{y^3} - \frac{2}{y^3} \right)  = -\frac{1}{y} .
\end{align*}
We define $ R_{ijk\ell} : = R_{ijk}^{s} g_{s\ell}$. Using identities, we obtain
\begin{align*}
	R_{1111} &= - R_{1111} \iff R_{1111} = 0 \\
	R_{1112} &= R_{1211} =-R_{1121} = -R_{2111} \\
	&= -R_{1112} \iff R_{1112}= 0  \\
	R_{1122} &= R_{2211}  \\
	&= -R_{1122} \iff R_{1122} =0 \\
	R_{1221} &= R_{2112} = -R_{2121} = -R_{1212} \\
	&= R_{122}^{1} g_{11} + R_{122}^{2} g_{21} \\
		    &= \left( \frac{1}{y^2} -0+ \frac{1}{y^2} +0-0- \frac{1}{y^2} \right) \frac{1}{y^2} +0  = \frac{1}{y^{4}}\\
	R_{1222} &= R_{2212} = -R_{2122} = -R_{2221}  \\
	&= -R_{1222} \iff R_{1222}=0 \\
	R_{2222} &= -R_{2222} \iff R_{2222} = 0 .
\end{align*}
There are a total of $ 2^{4} = 16$ terms so we have computed all of them. Therefore, we have
\begin{align*}
	R_{1212} &= \left\langle R(E_1,E_2)E_1 , E_2 \right\rangle = -\frac{1}{y^{4}}\\
	K(E_1,E_2) &= \frac{R_{1212}}{ \norm{ E_1 \wedge E_2}^2 }\\ 
		 &= \frac{R_{1212}}{ \norm{ E_1}^2 \norm{ E_2}^2 -  \left\langle E_1, E_2 \right\rangle^2 }  \\
		 &= \frac{-\frac{1}{y^{4}}}{ \frac{1}{y^2} \frac{1}{y^2} -0} \\
		 &=- 1 .
\end{align*}
\end{proof}
\begin{problem}[4]
Compute the curvature of the unit sphere $ S^2$ using the formula from the first exercise.
\end{problem}
\begin{proof}
	Recall that the unit sphere under spherical basis $ \{ \frac{\partial }{\partial \phi} , \frac{\partial }{\partial \theta}  \} $ (longitude, latitude) has metric tensor $ G = \begin{pmatrix} 1&0\\0& \sin^2 \phi \end{pmatrix} $.
\begin{align*}
	\Gamma_{ 1 1}^{ 1} &= 0 \\
	\Gamma_{ 1 2}^{ 1} &= \Gamma_{ 2 1}^{ 1} = 0 \\
	\Gamma_{ 2 2}^{ 1} &= \frac{1}{2} \left( - 2 \sin \phi \cos \phi +0+0 \right) = - \sin \phi \cos \phi  \\
	\Gamma_{ 1 1}^{ 2} &= 0 \\
	\Gamma_{ 1 2}^{ 2} &= \Gamma_{ 2 1}^{ 2} = \frac{1}{2} \frac{1}{ \sin^2\phi} \left( -0+0+ 2 \sin \phi \cos \phi \right) = \frac{\cos \phi}{ \sin \phi } =\cot \phi \\
	\Gamma_{ 2 2}^{ 2} &= \frac{1}{2} \sin^2 \phi \left( -0+0+0 \right) =0  .
\end{align*}
As before, curvature is 0 except for the following terms:
\begin{align*}
	R_{\phi \theta \theta \phi} = R_{1221} &= R_{2112} = -R_{2121} = -R_{1212} \\
	&= R_{122}^{1} g_{11} + R_{122}^{2} g_{21} \\
	&= \left( 0+ \cos^2 \phi - \sin^2 \phi +0 - \frac{\cos \phi}{ \sin \phi} \sin \phi \cos \phi - 0 -0  \right) \cdot 1 +0\\
	&= -\sin^2 \phi .
\end{align*}
Thus $ R_{1212} = \sin^2 \phi$ and the sectional curvature is
\begin{align*}
	K(E_1,E_2) &= \frac{R_{1212}}{ \norm{ E_1}^2 \norm{ E_2}^2   } \\
	&= \frac{ \sin^2 \phi}{ \sin^2 \phi } \\
	&= 1 .
\end{align*}

\end{proof}
\end{document}
