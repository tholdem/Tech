\documentclass[12pt]{article}
%Fall 2022
% Some basic packages
\usepackage{standalone}[subpreambles=true]
\usepackage[utf8]{inputenc}
\usepackage[T1]{fontenc}
\usepackage{textcomp}
\usepackage[english]{babel}
\usepackage{url}
\usepackage{graphicx}
%\usepackage{quiver}
\usepackage{float}
\usepackage{enumitem}
\usepackage{lmodern}
\usepackage{comment}
\usepackage{hyperref}
\usepackage[usenames,svgnames,dvipsnames]{xcolor}
\usepackage[margin=1in]{geometry}
\usepackage{pdfpages}

\pdfminorversion=7

% Don't indent paragraphs, leave some space between them
\usepackage{parskip}

% Hide page number when page is empty
\usepackage{emptypage}
\usepackage{subcaption}
\usepackage{multicol}
\usepackage[b]{esvect}

% Math stuff
\usepackage{amsmath, amsfonts, mathtools, amsthm, amssymb}
\usepackage{bbm}
\usepackage{stmaryrd}
\allowdisplaybreaks

% Fancy script capitals
\usepackage{mathrsfs}
\usepackage{cancel}
% Bold math
\usepackage{bm}
% Some shortcuts
\newcommand{\rr}{\ensuremath{\mathbb{R}}}
\newcommand{\zz}{\ensuremath{\mathbb{Z}}}
\newcommand{\qq}{\ensuremath{\mathbb{Q}}}
\newcommand{\nn}{\ensuremath{\mathbb{N}}}
\newcommand{\ff}{\ensuremath{\mathbb{F}}}
\newcommand{\cc}{\ensuremath{\mathbb{C}}}
\newcommand{\ee}{\ensuremath{\mathbb{E}}}
\newcommand{\hh}{\ensuremath{\mathbb{H}}}
\renewcommand\O{\ensuremath{\emptyset}}
\newcommand{\norm}[1]{{\left\lVert{#1}\right\rVert}}
\newcommand{\dbracket}[1]{{\left\llbracket{#1}\right\rrbracket}}
\newcommand{\ve}[1]{{\bm{#1}}}
\newcommand\allbold[1]{{\boldmath\textbf{#1}}}
\DeclareMathOperator{\lcm}{lcm}
\DeclareMathOperator{\im}{im}
\DeclareMathOperator{\coim}{coim}
\DeclareMathOperator{\dom}{dom}
\DeclareMathOperator{\tr}{tr}
\DeclareMathOperator{\rank}{rank}
\DeclareMathOperator*{\var}{Var}
\DeclareMathOperator*{\ev}{E}
\DeclareMathOperator{\dg}{deg}
\DeclareMathOperator{\aff}{aff}
\DeclareMathOperator{\conv}{conv}
\DeclareMathOperator{\inte}{int}
\DeclareMathOperator*{\argmin}{argmin}
\DeclareMathOperator*{\argmax}{argmax}
\DeclareMathOperator{\graph}{graph}
\DeclareMathOperator{\sgn}{sgn}
\DeclareMathOperator*{\Rep}{Rep}
\DeclareMathOperator{\Proj}{Proj}
\DeclareMathOperator{\mat}{mat}
\DeclareMathOperator{\diag}{diag}
\DeclareMathOperator{\aut}{Aut}
\DeclareMathOperator{\gal}{Gal}
\DeclareMathOperator{\inn}{Inn}
\DeclareMathOperator{\edm}{End}
\DeclareMathOperator{\Hom}{Hom}
\DeclareMathOperator{\ext}{Ext}
\DeclareMathOperator{\tor}{Tor}
\DeclareMathOperator{\Span}{Span}
\DeclareMathOperator{\Stab}{Stab}
\DeclareMathOperator{\cont}{cont}
\DeclareMathOperator{\Ann}{Ann}
\DeclareMathOperator{\Div}{div}
\DeclareMathOperator{\curl}{curl}
\DeclareMathOperator{\nat}{Nat}
\DeclareMathOperator{\gr}{Gr}
\DeclareMathOperator{\vect}{Vect}
\DeclareMathOperator{\id}{id}
\DeclareMathOperator{\Mod}{Mod}
\DeclareMathOperator{\sign}{sign}
\DeclareMathOperator{\Surf}{Surf}
\DeclareMathOperator{\fcone}{fcone}
\DeclareMathOperator{\Rot}{Rot}
\DeclareMathOperator{\grad}{grad}
\DeclareMathOperator{\atan2}{atan2}
\DeclareMathOperator{\Ric}{Ric}
\let\vec\relax
\DeclareMathOperator{\vec}{vec}
\let\Re\relax
\DeclareMathOperator{\Re}{Re}
\let\Im\relax
\DeclareMathOperator{\Im}{Im}
% Put x \to \infty below \lim
\let\svlim\lim\def\lim{\svlim\limits}

%wide hat
\usepackage{scalerel,stackengine}
\stackMath
\newcommand*\wh[1]{%
\savestack{\tmpbox}{\stretchto{%
  \scaleto{%
    \scalerel*[\widthof{\ensuremath{#1}}]{\kern-.6pt\bigwedge\kern-.6pt}%
    {\rule[-\textheight/2]{1ex}{\textheight}}%WIDTH-LIMITED BIG WEDGE
  }{\textheight}% 
}{0.5ex}}%
\stackon[1pt]{#1}{\tmpbox}%
}
\parskip 1ex

%Make implies and impliedby shorter
\let\implies\Rightarrow
\let\impliedby\Leftarrow
\let\iff\Leftrightarrow
\let\epsilon\varepsilon

% Add \contra symbol to denote contradiction
\usepackage{stmaryrd} % for \lightning
\newcommand\contra{\scalebox{1.5}{$\lightning$}}

% \let\phi\varphi

% Command for short corrections
% Usage: 1+1=\correct{3}{2}

\definecolor{correct}{HTML}{009900}
\newcommand\correct[2]{\ensuremath{\:}{\color{red}{#1}}\ensuremath{\to }{\color{correct}{#2}}\ensuremath{\:}}
\newcommand\green[1]{{\color{correct}{#1}}}

% horizontal rule
\newcommand\hr{
    \noindent\rule[0.5ex]{\linewidth}{0.5pt}
}

% hide parts
\newcommand\hide[1]{}

% si unitx
\usepackage{siunitx}
\sisetup{locale = FR}

%allows pmatrix to stretch
\makeatletter
\renewcommand*\env@matrix[1][\arraystretch]{%
  \edef\arraystretch{#1}%
  \hskip -\arraycolsep
  \let\@ifnextchar\new@ifnextchar
  \array{*\c@MaxMatrixCols c}}
\makeatother

\renewcommand{\arraystretch}{0.8}

\renewcommand{\baselinestretch}{1.5}

\usepackage{graphics}
\usepackage{epstopdf}

\RequirePackage{hyperref}
%%
%% Add support for color in order to color the hyperlinks.
%% 
\hypersetup{
  colorlinks = true,
  urlcolor = blue,
  citecolor = blue
}
%%fakesection Links
\hypersetup{
    colorlinks,
    linkcolor={red!50!black},
    citecolor={green!50!black},
    urlcolor={blue!80!black}
}
%customization of cleveref
\RequirePackage[capitalize,nameinlink]{cleveref}[0.19]

% Per SIAM Style Manual, "section" should be lowercase
\crefname{section}{section}{sections}
\crefname{subsection}{subsection}{subsections}
\Crefname{section}{Section}{Sections}
\Crefname{subsection}{Subsection}{Subsections}

% Per SIAM Style Manual, "Figure" should be spelled out in references
\Crefname{figure}{Figure}{Figures}

% Per SIAM Style Manual, don't say equation in front on an equation.
\crefformat{equation}{\textup{#2(#1)#3}}
\crefrangeformat{equation}{\textup{#3(#1)#4--#5(#2)#6}}
\crefmultiformat{equation}{\textup{#2(#1)#3}}{ and \textup{#2(#1)#3}}
{, \textup{#2(#1)#3}}{, and \textup{#2(#1)#3}}
\crefrangemultiformat{equation}{\textup{#3(#1)#4--#5(#2)#6}}%
{ and \textup{#3(#1)#4--#5(#2)#6}}{, \textup{#3(#1)#4--#5(#2)#6}}{, and \textup{#3(#1)#4--#5(#2)#6}}

% But spell it out at the beginning of a sentence.
\Crefformat{equation}{#2Equation~\textup{(#1)}#3}
\Crefrangeformat{equation}{Equations~\textup{#3(#1)#4--#5(#2)#6}}
\Crefmultiformat{equation}{Equations~\textup{#2(#1)#3}}{ and \textup{#2(#1)#3}}
{, \textup{#2(#1)#3}}{, and \textup{#2(#1)#3}}
\Crefrangemultiformat{equation}{Equations~\textup{#3(#1)#4--#5(#2)#6}}%
{ and \textup{#3(#1)#4--#5(#2)#6}}{, \textup{#3(#1)#4--#5(#2)#6}}{, and \textup{#3(#1)#4--#5(#2)#6}}

% Make number non-italic in any environment.
\crefdefaultlabelformat{#2\textup{#1}#3}

% Environments
\makeatother
% For box around Definition, Theorem, \ldots
%%fakesection Theorems
\usepackage{thmtools}
\usepackage[framemethod=TikZ]{mdframed}

\theoremstyle{definition}
\mdfdefinestyle{mdbluebox}{%
	roundcorner = 10pt,
	linewidth=1pt,
	skipabove=12pt,
	innerbottommargin=9pt,
	skipbelow=2pt,
	nobreak=true,
	linecolor=blue,
	backgroundcolor=TealBlue!5,
}
\declaretheoremstyle[
	headfont=\sffamily\bfseries\color{MidnightBlue},
	mdframed={style=mdbluebox},
	headpunct={\\[3pt]},
	postheadspace={0pt}
]{thmbluebox}

\mdfdefinestyle{mdredbox}{%
	linewidth=0.5pt,
	skipabove=12pt,
	frametitleaboveskip=5pt,
	frametitlebelowskip=0pt,
	skipbelow=2pt,
	frametitlefont=\bfseries,
	innertopmargin=4pt,
	innerbottommargin=8pt,
	nobreak=false,
	linecolor=RawSienna,
	backgroundcolor=Salmon!5,
}
\declaretheoremstyle[
	headfont=\bfseries\color{RawSienna},
	mdframed={style=mdredbox},
	headpunct={\\[3pt]},
	postheadspace={0pt},
]{thmredbox}

\declaretheorem[%
style=thmbluebox,name=Theorem,numberwithin=section]{thm}
\declaretheorem[style=thmbluebox,name=Lemma,sibling=thm]{lem}
\declaretheorem[style=thmbluebox,name=Proposition,sibling=thm]{prop}
\declaretheorem[style=thmbluebox,name=Corollary,sibling=thm]{coro}
\declaretheorem[style=thmredbox,name=Example,sibling=thm]{eg}

\mdfdefinestyle{mdgreenbox}{%
	roundcorner = 10pt,
	linewidth=1pt,
	skipabove=12pt,
	innerbottommargin=9pt,
	skipbelow=2pt,
	nobreak=true,
	linecolor=ForestGreen,
	backgroundcolor=ForestGreen!5,
}

\declaretheoremstyle[
	headfont=\bfseries\sffamily\color{ForestGreen!70!black},
	bodyfont=\normalfont,
	spaceabove=2pt,
	spacebelow=1pt,
	mdframed={style=mdgreenbox},
	headpunct={ --- },
]{thmgreenbox}

\declaretheorem[style=thmgreenbox,name=Definition,sibling=thm]{defn}

\mdfdefinestyle{mdgreenboxsq}{%
	linewidth=1pt,
	skipabove=12pt,
	innerbottommargin=9pt,
	skipbelow=2pt,
	nobreak=true,
	linecolor=ForestGreen,
	backgroundcolor=ForestGreen!5,
}
\declaretheoremstyle[
	headfont=\bfseries\sffamily\color{ForestGreen!70!black},
	bodyfont=\normalfont,
	spaceabove=2pt,
	spacebelow=1pt,
	mdframed={style=mdgreenboxsq},
	headpunct={},
]{thmgreenboxsq}
\declaretheoremstyle[
	headfont=\bfseries\sffamily\color{ForestGreen!70!black},
	bodyfont=\normalfont,
	spaceabove=2pt,
	spacebelow=1pt,
	mdframed={style=mdgreenboxsq},
	headpunct={},
]{thmgreenboxsq*}

\mdfdefinestyle{mdblackbox}{%
	skipabove=8pt,
	linewidth=3pt,
	rightline=false,
	leftline=true,
	topline=false,
	bottomline=false,
	linecolor=black,
	backgroundcolor=RedViolet!5!gray!5,
}
\declaretheoremstyle[
	headfont=\bfseries,
	bodyfont=\normalfont\small,
	spaceabove=0pt,
	spacebelow=0pt,
	mdframed={style=mdblackbox}
]{thmblackbox}

\theoremstyle{plain}
\declaretheorem[name=Question,sibling=thm,style=thmblackbox]{ques}
\declaretheorem[name=Remark,sibling=thm,style=thmgreenboxsq]{remark}
\declaretheorem[name=Remark,sibling=thm,style=thmgreenboxsq*]{remark*}
\newtheorem{ass}[thm]{Assumptions}

\theoremstyle{definition}
\newtheorem*{problem}{Problem}
\newtheorem{claim}[thm]{Claim}
\theoremstyle{remark}
\newtheorem*{case}{Case}
\newtheorem*{notation}{Notation}
\newtheorem*{note}{Note}
\newtheorem*{motivation}{Motivation}
\newtheorem*{intuition}{Intuition}
\newtheorem*{conjecture}{Conjecture}

% Make section starts with 1 for report type
%\renewcommand\thesection{\arabic{section}}

% End example and intermezzo environments with a small diamond (just like proof
% environments end with a small square)
\usepackage{etoolbox}
\AtEndEnvironment{vb}{\null\hfill$\diamond$}%
\AtEndEnvironment{intermezzo}{\null\hfill$\diamond$}%
% \AtEndEnvironment{opmerking}{\null\hfill$\diamond$}%

% Fix some spacing
% http://tex.stackexchange.com/questions/22119/how-can-i-change-the-spacing-before-theorems-with-amsthm
\makeatletter
\def\thm@space@setup{%
  \thm@preskip=\parskip \thm@postskip=0pt
}

% Fix some stuff
% %http://tex.stackexchange.com/questions/76273/multiple-pdfs-with-page-group-included-in-a-single-page-warning
\pdfsuppresswarningpagegroup=1


% My name
\author{Jaden Wang}



\begin{document}
\centerline {\textsf{\textbf{\LARGE{Homework 8}}}}
\centerline {Jaden Wang}
\vspace{.15in}
\begin{problem}[1]
Compute the coefficients of the Riemann curvature tensor $ R$ in terms of the Christoffel symbols  $ \Gamma_{ij}^{k}$.
\end{problem}
\begin{proof}
	We use Einstein notation throughout.
\begin{align*}
	R_{ijk}^{ \ell} \frac{\partial }{\partial x_\ell} &=  R\left( \frac{\partial }{\partial x_i} , \frac{\partial }{\partial x_j}  \right) \frac{\partial }{\partial x_k}   \\
							  &= \left( \nabla _{\frac{\partial }{\partial x_j} } \nabla_{ \frac{\partial }{\partial x_i}}- \nabla_{ \frac{\partial }{\partial x_i}} \nabla_{ \frac{\partial }{\partial x_j} } + \nabla _{ \left[ \frac{\partial }{\partial x_i} ,\frac{\partial }{\partial x_j}  \right]}  \right) \frac{\partial }{\partial x_k} \\
							  &= \nabla _{ \frac{\partial }{\partial x_j} } \nabla_{ \frac{\partial }{\partial x_i}} \frac{\partial }{\partial x_k} - \nabla_{ \frac{\partial }{\partial x_i}} \nabla_{ \frac{\partial }{\partial x_j} }\frac{\partial }{\partial x_k}   \\
							  &= \nabla _{ \frac{\partial }{\partial x_j} } \left( \Gamma_{ k i}^{ s} \frac{\partial }{\partial x_s}  \right) - \nabla _{ \frac{\partial }{\partial x_i} } \left( \Gamma_{ k j}^{ s} \frac{\partial }{\partial x_s}  \right)   \\
							  &= \frac{\partial \Gamma_{ k i}^{ s} }{\partial x_j} \frac{\partial }{\partial x_s} + \Gamma_{ k i}^{ s} \Gamma_{ sj}^{ \ell} \frac{\partial }{\partial x_\ell} -  \frac{\partial \Gamma_{ k j}^{ s} }{\partial x_i} \frac{\partial }{\partial x_s} - \Gamma_{ k j}^{ s} \Gamma_{ si}^{ \ell} \frac{\partial }{\partial x_\ell}  && \text{Leibniz rule}  \\
							  &=  \frac{\partial \Gamma_{ k i}^{ \ell} }{\partial x_j} \frac{\partial }{\partial x_\ell} + \Gamma_{ k i}^{ s} \Gamma_{ sj}^{ \ell} \frac{\partial }{\partial x_\ell} -  \frac{\partial \Gamma_{ k j}^{ \ell} }{\partial x_i} \frac{\partial }{\partial x_\ell} - \Gamma_{ k j}^{ s} \Gamma_{ si}^{ \ell} \frac{\partial }{\partial x_\ell} && \text{reindexing}  \\
							  &= \left(  \frac{\partial \Gamma_{ ik}^{ \ell} }{\partial x_j} -  \frac{\partial \Gamma_{ jk}^{ \ell} }{\partial x_i} + \Gamma_{ ik}^{ s} \Gamma_{ js}^{ \ell} - \Gamma_{ jk}^{ s} \Gamma_{ is}^{ \ell} \right)  \frac{\partial }{\partial x_\ell}. && \nabla \text{ symmetric} 
\end{align*}
Hence, the coefficients of the Riemann curvature tensor is
\begin{align*}
	R_{ijk}^{ \ell} =   \frac{\partial \Gamma_{ ik}^{ \ell} }{\partial x_j} -  \frac{\partial \Gamma_{ jk}^{ \ell} }{\partial x_i}  + \Gamma_{ ik}^{ s} \Gamma_{ js}^{ \ell} - \Gamma_{ jk}^{ s} \Gamma_{ is}^{ \ell}. 
\end{align*}
\end{proof}
\begin{problem}[2]
Show that the curvature of $ \rr^{n}$ is zero by (i) using the formula from the last exercise, and (ii) using the abstract definition of  $ R$ in terms of the covariant derivative  $ \nabla $.
\end{problem}
\begin{proof}
\begin{enumerate}[label=(\roman*)]
	\item 	Recall that
\begin{align*}
	\Gamma_{ i j}^{ k} = \frac{1}{2} g^{sk} \left( - \frac{\partial g_{ij}}{\partial x_s} + \frac{\partial g_{js}}{\partial x_i} + \frac{\partial g_{si}}{\partial x_j}    \right) .
\end{align*}
In $ \rr^{n}$, $ g^{sk} = g_{sk}^{-1} = \delta_{sk}$. Thus
\begin{align*}
	\Gamma_{ i j}^{ k} = \frac{1}{2} \left( - \frac{\partial g_{ii}}{\partial x_i} + \frac{\partial g_{ii}}{\partial x_i} + \frac{\partial g_{ii}}{\partial x_i}    \right) = 0 . 
\end{align*}
We immediately have
\begin{align*}
	R_{ijk}^{\ell} = 0 .
\end{align*}
Therefore, the curvature is zero.
\item In $ \rr^{n}$, a global canonical basis enables $ \nabla_X Z = (X(Z^{1},\ldots,X(Z^{n})) =: X(Z)$. Thus we have
\begin{align*}
	R(X,Y)Z&=(\nabla_Y \nabla_X - \nabla _X \nabla _Y  + \nabla _{[X,Y]})Z \\
	&= YX(Z) - XY(Z) + XY(Z) - YX(Z) = 0  . 
\end{align*}
\end{enumerate}
\end{proof}
\begin{problem}[3]
Compute the curvature of the hyperbolic plane $ H^2$ using the formula from the first exercise.
\end{problem}
\begin{proof}
Recall that the metric tensor of $ H^2$ is $ g_{ij} = \delta_{ij} /y^2$. Thus $ g^{ij} = \delta_{ij} y^2$, and 
\begin{align*}
	\Gamma_{ 1 1}^{ 1} &= 0 \\
	\Gamma_{ 1 2}^{ 1} &= \Gamma_{ 2 1}^{ 1}  = \frac{1}{2} y^2\left(-0 - \frac{2}{y^3} + 0  \right) +0  = -\frac{1}{y}\\
	\Gamma_{ 2 2}^{ 1} &= \frac{1}{2} y^2 \left( -0 + 0 + 0 \right) = 0  \\
	\Gamma_{ 1 1}^{ 2} &= 0+\frac{1}{2} y^2 \left( \frac{2}{y^3} +0 + 0 \right) = \frac{1}{y}  \\
	\Gamma_{ 1 2}^{ 2} &= \Gamma_{ 2 1}^{ 2} = 0+ \frac{1}{2} y^2 \left( -0 +0 +0 \right) = 0\\
	\Gamma_{ 2 2}^{ 2} &= 0+ \frac{1}{2} y^2 \left( \frac{2}{y^3} - \frac{2}{y^3} - \frac{2}{y^3} \right)  = -\frac{1}{y} .
\end{align*}
We define $ R_{ijk\ell} : = R_{ijk}^{s} g_{s\ell}$. Using identities, we obtain
\begin{align*}
	R_{1111} &= - R_{1111} \iff R_{1111} = 0 \\
	R_{1112} &= R_{1211} =-R_{1121} = -R_{2111} \\
	&= -R_{1112} \iff R_{1112}= 0  \\
	R_{1122} &= R_{2211}  \\
	&= -R_{1122} \iff R_{1122} =0 \\
	R_{1221} &= R_{2112} = -R_{2121} = -R_{1212} \\
	&= R_{122}^{1} g_{11} + R_{122}^{2} g_{21} \\
		    &= \left( \frac{1}{y^2} -0+ \frac{1}{y^2} +0-0- \frac{1}{y^2} \right) \frac{1}{y^2} +0  = \frac{1}{y^{4}}\\
	R_{1222} &= R_{2212} = -R_{2122} = -R_{2221}  \\
	&= -R_{1222} \iff R_{1222}=0 \\
	R_{2222} &= -R_{2222} \iff R_{2222} = 0 .
\end{align*}
There are a total of $ 2^{4} = 16$ terms so we have computed all of them. Therefore, we have
\begin{align*}
	R_{1212} &= \left\langle R(E_1,E_2)E_1 , E_2 \right\rangle = -\frac{1}{y^{4}}\\
	K(E_1,E_2) &= \frac{R_{1212}}{ \norm{ E_1 \wedge E_2}^2 }\\ 
		 &= \frac{R_{1212}}{ \norm{ E_1}^2 \norm{ E_2}^2 -  \left\langle E_1, E_2 \right\rangle^2 }  \\
		 &= \frac{-\frac{1}{y^{4}}}{ \frac{1}{y^2} \frac{1}{y^2} -0} \\
		 &=- 1 .
\end{align*}
\end{proof}
\begin{problem}[4]
Compute the curvature of the unit sphere $ S^2$ using the formula from the first exercise.
\end{problem}
\begin{proof}
	Recall that the unit sphere under spherical basis $ \{ \frac{\partial }{\partial \phi} , \frac{\partial }{\partial \theta}  \} $ (longitude, latitude) has metric tensor $ G = \begin{pmatrix} 1&0\\0& \sin^2 \phi \end{pmatrix} $.
\begin{align*}
	\Gamma_{ 1 1}^{ 1} &= 0 \\
	\Gamma_{ 1 2}^{ 1} &= \Gamma_{ 2 1}^{ 1} = 0 \\
	\Gamma_{ 2 2}^{ 1} &= \frac{1}{2} \left( - 2 \sin \phi \cos \phi +0+0 \right) = - \sin \phi \cos \phi  \\
	\Gamma_{ 1 1}^{ 2} &= 0 \\
	\Gamma_{ 1 2}^{ 2} &= \Gamma_{ 2 1}^{ 2} = \frac{1}{2} \frac{1}{ \sin^2\phi} \left( -0+0+ 2 \sin \phi \cos \phi \right) = \frac{\cos \phi}{ \sin \phi } =\cot \phi \\
	\Gamma_{ 2 2}^{ 2} &= \frac{1}{2} \sin^2 \phi \left( -0+0+0 \right) =0  .
\end{align*}
As before, curvature is 0 except for the following terms:
\begin{align*}
	R_{\phi \theta \theta \phi} = R_{1221} &= R_{2112} = -R_{2121} = -R_{1212} \\
	&= R_{122}^{1} g_{11} + R_{122}^{2} g_{21} \\
	&= \left( 0+ \cos^2 \phi - \sin^2 \phi +0 - \frac{\cos \phi}{ \sin \phi} \sin \phi \cos \phi - 0 -0  \right) \cdot 1 +0\\
	&= -\sin^2 \phi .
\end{align*}
Thus $ R_{1212} = \sin^2 \phi$ and the sectional curvature is
\begin{align*}
	K(E_1,E_2) &= \frac{R_{1212}}{ \norm{ E_1}^2 \norm{ E_2}^2   } \\
	&= \frac{ \sin^2 \phi}{ \sin^2 \phi } \\
	&= 1 .
\end{align*}

\end{proof}
\end{document}
