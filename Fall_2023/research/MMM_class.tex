\documentclass[12pt,class=article,crop=false]{standalone} 
\newcommand{\alert}[1]{{\bf \color{red} [Alert:] #1}}
\newcommand{\todo}[1]{{\bf \color{orange} [TODO:] #1}}
\newcommand{\real}[1][]{\mathbb{R}^{#1}}
\newcommand{\myeqn}[1]{(\ref{#1})}
\newcommand{\myex}[1]{Example \ref{#1}}
\newcommand{\defeq}{\stackrel{\mathrm{def}}{=}}
\newcommand{\parder}[2]{\frac{\partial #1}{\partial #2}}
\newcommand{\Lie}[3][]{\mathsf{L}_{#3}^{#1} #2}
\newcommand{\LieA}[1]{\mathsf{Lie}(#1)}
\newcommand{\lieder}[2]{\mathcal{L}_{#2} #1}
\renewcommand{\t}{^{\mbox{\tiny\sf T}}}
\newcommand{\trans}{^{\mbox{\tiny\sf T}}}
\newcommand{\markup}[1]{\{\textbf{#1}\}}
\newcommand{\msub}[1]{_\mathrm{#1}}
\newcommand{\msup}[1]{^\mathrm{#1}}
\newcommand{\inv}[1]{#1^{-1}}
\newcommand{\pinv}[1]{{#1}^{+}}
\newcommand{\myfracA}[2]{\displaystyle{\frac{#1}{#2}}}
\newcommand{\myfracB}[2]{{#1}/{#2}}
\newcommand{\mydiffA}[1]{\dot{#1}}
\newcommand{\mydiffB}[2]{\myfracA{\mathrm{d}{#1}}{\mathrm{d}{#2}}}
\newcommand{\ball}[2]{\mathcal{B}_{#1}\left(#2\right)}
\newcommand{\acos}[1]{\cos^{-1}\left(#1\right)}
\newcommand{\asin}[1]{\sin^{-1}\left(#1\right)}
\newcommand{\mani}{\mathcal{M}}
\newcommand{\tang}[2]{\mathsf{T}_{#1} #2}
\newcommand{\LieB}[2]{[ #1, #2 ]}
\newcommand{\LieBad}[3][]{\mathsf{ad}_{#2}^{#1} #3}
\newcommand{\ReachVT}{\mathcal{R}^V_T}
\newcommand{\ReachVt}{\mathcal{R}^V_t}
\newcommand{\ReachVTe}{\mathcal{R}^V_{\le T}}
\newcommand{\ReachT}{\mathcal{R}_T}
\newcommand{\Reacht}{\mathcal{R}_t}
\newcommand{\ReachTe}{\mathcal{R}_{\le T}}
\newcommand{\accLA}[1]{\mathsf{Lie}(#1)}
\newcommand{\accD}{\Delta_{\mathcal{F}}}
\newcommand{\accSA}{\mathsf{Lie}(\mathcal{G},f)}
\newcommand{\accDS}{\Delta_{\mathcal{G}}}
\newcommand{\eval}[3]{\mathsf{Ev}^{#2}_{#1}\left( #3 \right)}
\newcommand{\stlc}{\textsc{stlc}}
\newcommand{\clf}{\textsc{clf}}
\newcommand{\jqlf}{\textsc{jqlf}}
\newcommand{\dlas}{\textsc{dlas}}
\newcommand{\Ad}[2]{\mathsf{Ad}_{#1} #2}
\newcommand{\xe}{\ensuremath{x_e}}
\newcommand{\lebg}[1]{\mathcal{L}_{#1}}
\newcommand{\lebgx}[1]{\mathcal{L}_{#1 \mathrm{e}}}
\newcommand{\dom}{D}
\newcommand{\domT}{[t_0,\infty) \times D}
\newcommand{\rarrow}{\rightarrow}
\renewcommand{\d}{\mathrm{d}}
\renewcommand{\Re}{\mathbb{R}}
\newcommand{\C}{\mathrm{C}}

\newcommand{\QED}{{\unskip\nobreak\hfil\penalty50\hskip2em\vadjust{}
		\nobreak\hfil$\Box$\parfillskip=0pt\finalhyphendemerits=0\par}\vspace{0.1cm}}
\newcommand{\eoEx}{{\unskip\nobreak\hfil\penalty50\hskip0em\vadjust{}
		\nobreak\hfil$\Large\Diamond$\parfillskip=0pt\finalhyphendemerits=0\par}\vspace{0.1cm}}

\newcommand{\sgn}{\ensuremath{\operatorname{sgn}}}
\newcommand{\sat}{\ensuremath{\operatorname{sat}}}

\newcommand{\half}{\frac{1}{2}}
\newcommand{\shalf}{\mbox{$\frac{1}{2}$}}
\newcommand{\marcom}[1]{\marginpar{\footnotesize #1}}
\newcommand{\der}{\mathrm{D}}
\newcommand{\e}{\mathrm{e}}
\newcommand{\dt}{\mathrm{d}t}

\newcommand{\cA}{\ensuremath{\mathcal{A}}}
\newcommand{\cB}{\ensuremath{\mathcal{B}}}
\newcommand{\cG}{\ensuremath{\mathcal{G}}}
\newcommand{\cK}{\ensuremath{\mathcal{K}}}
\newcommand{\cW}{\ensuremath{\mathcal{W}}}
\newcommand{\cZ}{\ensuremath{\mathcal{Z}}}
\newcommand{\cS}{\ensuremath{\mathcal{S}}}
\newcommand{\cD}{\ensuremath{\mathcal{D}}}
\newcommand{\cP}{\ensuremath{\mathcal{P}}}
\newcommand{\cV}{\ensuremath{\mathcal{V}}}
\newcommand{\cL}{\ensuremath{\mathcal{L}}}
\newcommand{\cN}{\ensuremath{\mathcal{N}}}
\newcommand{\cI}{\ensuremath{\mathcal{I}}}
\newcommand{\cR}{\ensuremath{\mathcal{R}}}
\newcommand{\cM}{\ensuremath{\mathcal{M}}}
\newcommand{\cC}{\ensuremath{\mathcal{C}}}
\newcommand{\cF}{\ensuremath{\mathcal{F}}}
\newcommand{\cH}{\ensuremath{\mathcal{H}}}
\newcommand{\cO}{\ensuremath{\mathcal{O}}}
\newcommand{\cX}{\ensuremath{\mathcal{X}}}
\newcommand{\cY}{\ensuremath{\mathcal{Y}}}
\newcommand{\Ci}{\ensuremath{\mathcal{C}^\infty}}
\newcommand{\ISS}{\textsc{iss}}
\newcommand{\LISS}{\textsc{liss}}
\newcommand{\GAS}{\textsc{gas}}
\newcommand{\GS}{\textsc{gs}}
\newcommand{\LES}{\textsc{les}}
\newcommand{\GUAS}{\textsc{guas}}
\newcommand{\BIBO}{\textsc{bibo}}
\newcommand{\spec}{\ensuremath{\operatorname{spec}}}
\newcommand{\spn}{\ensuremath{\operatorname{span}}}
\renewcommand{\i}{\mathrm{i\,}}

\renewcommand{\implies}{\Rightarrow}

\renewcommand{\theenumi}{$\roman{enumi})$}
\renewcommand{\labelenumi}{\theenumi}

\font\ptmten=zptmcmrm scaled 1200
\newcommand{\w}{\mbox{{\ptmten w}}}
\newcommand{\z}{\mbox{{\ptmten z}}}
\renewcommand{\Re}{\mathbb{R}}

\newcommand{\cl}{\operatorname{cl}}
\newcommand{\intr}{\operatorname{int}}
\newcommand{\rank}{\operatorname{rank}}
\newcommand{\co}{\operatorname{co}}
\newcommand{\aff}{\operatorname{aff}}

\theoremstyle{plain}
\newtheorem{theorem}{Theorem}[chapter]
\newtheorem{claim}[theorem]{Claim}
\newtheorem{corollary}[theorem]{Corollary}
\newtheorem{prop}[theorem]{Proposition}
\newtheorem{fact}[theorem]{Fact}
\newtheorem{lemma}[theorem]{Lemma}

\newtheorem{remark}{Remark}[chapter]

\theoremstyle{definition}
\newtheorem{assume}[theorem]{Assumption}
\newtheorem{defn}[theorem]{Definition}
\newtheorem{problem}[theorem]{Problem}
\newtheorem{exercise}{Exercise}
\newtheorem{example}[theorem]{Example}


\begin{document}
\section{Motivation}
Let $ S_g$ denote the closed oriented (smooth) surface of genus $ g \geq 2$ and let $ \Mod(S_g)$ denote its mapping class group. To understand $ \Mod(S_g)$, it is natural to ask what its cohomology is. This is especially an important question because cohomology classes of the mapping class group of $ S_g$ exactly corresponds to characteristic classes of  $ S_g$-bundles. However, we know relatively little about $ H^{k}(\Mod(S_g))$ for all possible $ k$ and  $ g$. One family that is completely known to us is within the \emph{stable range} when $ k \leq 2 \left\lfloor \frac{g}{3} \right \rfloor$. Madsen and Weiss \cite{TODO} proved that the \emph{tautological classes} $ \kappa_i$ which coincides with the Miller-Morita-Mumford (MMM) characteristic classes $ e_i$ generate $ H^{k}(\Mod(S_g))$ as a polynomial algebra in this range. The purpose of this paper is to understand these $ e_i$ and show that in particular $ e_1$ is non-trivial.
\section{Background}
\begin{defn}
Let $ A$ be an abelian group and  $ k$ be a natural number. A \allbold{characteristic class of $ S_g$-bundles of degree  $ k$ with coefficients in  $ A$} is a natural transformation from the functor $ \Surf_g(-): \textsf{pchTop}^{op} \to \textsf{Set}$ to the cohomology functor $ H^{k}(-;A): \textsf{pchTop}^{op} \to \textsf{Set}  $, where $ \Surf_g(M)$ outputs the set of isomorphism classes of  $ S_g$-bundles with base $ M$ and $ \textsf{pchTop} $ denotes the full subcategory of the homotopy category of topological spaces $ \textsf{hTop}$ with objects restricted to (homotopy classes of) paracompact spaces.
\end{defn}
Since any manifold is paracompact, we can ignore this technicality for the purpose of this paper. Moreover, if the categorical language is not familiar to the reader, it suffices to understand a characteristic class as a cohomology class of the base that is natural with respect to bundle pullbacks. That is, if $f$ is a bundle map between two bundles $ F \to E_1 \to M_1$ and $ F \to E_2 \to M_2$, then $ f^* ( \alpha(E_2)) = \alpha(f^* E_2) = \alpha(E_1) $. Note that the last equality is a property of bundle maps.
\begin{prop} \label{prop:charclass}
The set of characteristic classes of $ S_g$-bundles of degree $ k$ with coefficients in $ \zz$ is in bijection with  $ H^{k}(\Mod(S_g);\zz)$.
\end{prop}

The following proposition enables us to switch back and forth between algebraic formalism and geometric intuition. See the notes of Mike Hutchings \cite{TODO} for a proof. 
\begin{prop} \label{prop:cupintersection} 
	Let $ X^{n}$ be a closed oriented smooth manifold and suppose $ A^{n-i},B^{n-j}$ are oriented smooth submanifolds of $ X$. Let  $ [A]^* \in H^{i}(X;\zz), [B]^* \in H^{j}(X:\zz)$ respectively denote the Poincar\'{e} duals to the fundamental classes of $ A$ and  $ B$ included into $ X$. Then we have
	\begin{align*}
		[A]^* \smile [B]^* = [A \cap B]^* \in H^{i+j}(X;\zz).
	\end{align*}
That is, the cup product is the Poincar\'{e} dual to intersection.
\end{prop}

In particular, when $ i+j =n$, then the intersection class is just an integer, which we call the intersection number. Also, we will use the notation $ [A]^* $ to denote the Poincar\'{e} dual to a fundamental class $ [A]$ throughout the paper.

\begin{defn}
The \allbold{vertical tangle bundle} $ \rr^{2} \to V \to E$ of a surface bundle $ S_g \to E \xrightarrow{ \pi}  M$ is defined to be $ V= \{(v,e) \in TE: d\pi_e(v) = 0 \in T_e M\} $.
\end{defn}
We can think of $ V$ as the contribution of tangents of fibers to the tangent bundle  $ TE$. Since the rest of the tangent bundle comes from the tangents of base space, we have $ TE \cong \pi^* TM \oplus V$, where $ \oplus $ is the Whitney sum of vector bundles.

When $ V$ is orientable, there exists an Euler class  $ e(V) \in H^{2}(E;\zz)$. For each $ i \in \nn$, we can take cup products of $ e(V)$ to obtain  $ e^{i+1}(V) \in H^{2i+2}(E;\zz)$. Therefore, when $ \dim  M = 2i$, $ e^{i+1}(V)$ lives in the top cohomology group of $ E$ and thus is an integer.  From the intersection perspective, recall that for a vector bundle, the zero locus of a section (which is a vector field) is the transversal intersection of the vector field with the zero vector field. Since the Euler class here represents the zero locus of a vertical vector field $ E \to V$, $ e^{i+1}$ can be understood as the intersection number of the zero loci of $ i+1$ vertical vector fields. Since the zero vertical vector field of $ V$ can be identified with $ E$, the intersection lives in $ E$ as well. By a dimension count using transversality, the intersection is represented by a class in $ H^{2i+2}(E;\zz)$.

Next, we introduce the Gysin homomorphism.
\begin{defn}
Let $ N,M$ be two oriented closed manifolds and  $ f: N \to M$ be a map. The Gysin homomorphism $ f_!: H^{p}(N;\zz) \to H^{p-d}(M;\zz)$ is defined to be the composition
\begin{align*}
	H^{p}(N;\zz) \cong H_{n-p}(N;\zz) \xrightarrow{ f_*} H_{n-p}(M;\zz) \cong H^{p-d}(M;\zz), 
\end{align*}
where $ n = \dim N$, $ d = \dim N- \dim M$, and the isomorphisms represent Poincar\'{e} duality. The dual Gysin homomorphism $ f^!:H_p(M;\zz) \to H_{p+d}(N;\zz)$ is defined similarly.
\end{defn}
We see from the definition that the Gysin homomorphism is nothing but applying an induced map on the homology. Since homology classes are submanifolds, we can understand Gysin homomorphism geometrically by tracking what the map does to the submanifolds representing the homology classes.

We finally introduce the main object of interest in this paper.
\begin{defn}
	Define \allbold{the $ i$th Miller-Morita-Mumford class (MMM class)} to be $ e_i(E) = \pi_!\left( e^{i+1} \right) \in H^{2i}(M;\zz)$.
\end{defn}
\begin{prop}
The cohomology class $ e_i(E)$ is natural with respect to pullbacks. That is, if $ f$ is a bundle map shown in the diagram,
% https://q.uiver.app/#q=WzAsOCxbMSwxLCJmXipFXzIiXSxbMSwyLCJNXzEiXSxbMSwwLCJcXHRpbGRle2Z9XipWXzIiXSxbMiwxLCJFXzIiXSxbMiwyLCJNXzIiXSxbMiwwLCJWXzIiXSxbMCwwLCJWXzEiXSxbMCwxLCJFXzEiXSxbMCwxXSxbMiwwXSxbMCwzLCJcXHRpbGRle2Z9Il0sWzMsNCwiXFxwaSJdLFsxLDQsImYiXSxbNSwzXSxbMiw1XSxbNiwyLCJcXGNvbmciLDIseyJzdHlsZSI6eyJib2R5Ijp7Im5hbWUiOiJkYXNoZWQifX19XSxbNiw3XSxbNywwLCJcXGNvbmciLDAseyJzdHlsZSI6eyJib2R5Ijp7Im5hbWUiOiJkYXNoZWQifX19XSxbNywxLCJwIiwyXV0=
\[\begin{tikzcd}
	{V_1} & {\tilde{f}^*V_2} & {V_2} \\
	{E_1} & {f^*E_2} & {E_2} \\
	& {M_1} & {M_2}
	\arrow[from=2-2, to=3-2]
	\arrow[from=1-2, to=2-2]
	\arrow["{\tilde{f}}", from=2-2, to=2-3]
	\arrow["\pi", from=2-3, to=3-3]
	\arrow["f", from=3-2, to=3-3]
	\arrow[from=1-3, to=2-3]
	\arrow[from=1-2, to=1-3]
	\arrow["\cong"', dashed, from=1-1, to=1-2]
	\arrow[from=1-1, to=2-1]
	\arrow["\cong", dashed, from=2-1, to=2-2]
	\arrow["p"', from=2-1, to=3-2]
\end{tikzcd}\]
then
\begin{align*}
	e_i(E_1) = f^* e_i(E_2).
\end{align*}
\end{prop}
\begin{proof}
First, recall that the Euler class $ e(V_1)$ is natural with respect to pullbacks. That is,
\begin{align*}
	e(V_1) = e(\widetilde{ f}^* V_2) = \widetilde{ f}^* e(V_2).
\end{align*}
\begin{align*}
	e_i(E_1) &= \pi_!\left( e^{i+1} (\widetilde{ f}^* V_2) \right)\\ 
		 &= \pi_! \left( (\widetilde{ f}^* e)^{i+1}(V_2) \right) && e \text{ is natural}  \\
		 &= \pi_!\left( \widetilde{ f}^* \left( e^{i+1}(V_2) \right)  \right) && \smile \text{ is natural}  \\
		 &= f^* p_! \left( e^{i+1}(V_2) \right) && \text{Gysin is natural}  \\
		 &= f^* e_i(E_2) 
\end{align*}
\end{proof}
Therefore, the MMM class is indeed a characteristic class of surface bundles. By \cref{prop:charclass}, $ e_i \in H^{2i}(\Mod(S_g);\zz)$.

\section{Construction of the AK manifold}
First, we introduce a crucial object for the construction in a somewhat colloquial fashion.
\begin{defn}
	Let $ M$ be an oriented smooth manifold and  $ B$ be an oriented submanifold. \allbold{An $ m$-fold cyclic branched covering} $ \pi: \widetilde{ M} \to M$ \allbold{ramified along $ B$} is a covering map over $M\setminus B$ and a diffeomorphism between $ \pi^{-1}(B)$ and $ B$. Alternatively, it is the quotient map that quotients out the action of $ \zz /m$ on $ \widetilde{ M}$ which is free except on a set of fixed points $ B$ with codimesnion 2.
\end{defn}
The mental picture we want to have is the following:
TODO

Although for each $ i$, we have found a characteristic class $ e_i$ of surface bundles, this class is only interesting if it is not trivial, \emph{i.e.} corresponding to the zero class in $ H^{2i}(\Mod(S_g);\zz)$. Historically, the first MMM class $ e_1$ is found to be non-trivial by a fact of Hirzebruch that says: for an oriented surface bundle $ E$ over an oriented closed surface $ M$, $ \langle e_1,[M] \rangle = 3 \sign(E)$, the signature of $ E$; luckily, surface bundles with non-zero signatures have already been constructed by Atiya and Kodaira independently. We now present this construction. We start with some important facts.

\begin{lem} \label{lem:preimage} 
	Let $ M^{n}$  be an closed oriented manifold and let $ \pi: \widetilde{ M} \to M$ be a finite covering, where the orientation of $ \widetilde{ M}$ is induced by $ M$. Suppose that $ B$ is a  $ (n-k)$-dimensional oriented submanifold of  $ M$, then  $ \pi^* [B]^* $ can be represented by the oriented submanifold $ \widetilde{ B} = \pi ^{-1}(B)$ of $ \widetilde{ M}$.
\end{lem}

The following proposition is the main tool we need to construct an interesting manifold.
\begin{prop} \label{prop:divisible} 
	Let $ M^{n}$ be a closed oriented smooth manifold and let  $ B \subset M$ be an oriented submanifold of codimension 2. Suppose that, for some positive integer $ m$,  $ [B] \in H_{n-2}(M;\zz)$ is divisible by $ m$. Then there exists an  $ m$-fold cyclic ramified covering  $ \widetilde{ M} \to M$ ramified along $ B$. 
\end{prop}


Let us begin the construction. We start with $ M_1 = S_{g_1}$. Let $ (M_2, \rho_1)$ be a $ m$-fold cyclic covering of  $ M_1$, \emph{i.e.} its deck transformation group is a cyclic group of order $ m$, generated by  $ \sigma$. By a fact of Euler characteristic of covering spaces, $ 2- 2g_2 =\chi(M_2) = m \chi(M_1) = m (2-2g_1)$ so $ g_2 = 1+m(g_1-1)$.

Since any subgroup of $ \pi_1(M_2)$ gives a covering of $ M_2$, let $ (M_3, \rho_2)$ be such covering induced by the kernel $ K$ of the surjective map
\begin{align*}
	\pi_1(M_2) \twoheadrightarrow H_1(M_2; \zz) \twoheadrightarrow H_1(M_2;\zz /m).
\end{align*}
Since $ H_1(M_2 ; \zz) = \zz^{2g_2}$, $ H_1(M_2; \zz /m) \cong H_1(M_2;\zz) \otimes \zz /m \cong \left( \zz /m \right)^{2g_2} $ by the universal coefficient theorem. Therefore, the first isomorphism theorem yields that $ [\pi_1(M_2):K] = |H_1(M_2;\zz /m)|= m^{2g_2}$. That is, $ M_3$ is a $ m^{2g_2}$-fold covering of $ M_2$. Then $ 2-2g_3 = m^{2g_2}(2-2g_2)$ which gives $ g_3 = 1+m^{2g_2}(g_2-1)$.

Now consider the trivial bundle $ M_3 \times M_2$ where $ M_2$ is viewed as the fiber. For $ i=0,\ldots,m-1$, the map $ \sigma^{i} \rho_2 : M_3 \to M_2$ projects a point down to $ M_2$ and then permutes it to a point in the fiber of $ \rho_1$. Therefore, the graph $ \Gamma_i$ of $ \sigma^{i} \circ \rho_2$ lives in $ M_3 \times M_2$ and is disjoint from all other $ i$'s. Let $ D$ be the (disjoint) union of all  $ \Gamma_i$ so by the properties of graph, it is a submanifold of codimension 2. We now aim to construct a branched cover over $ M_3 \times M_2$ with $ D$ being the ramification locus. By \cref{prop:divisible}, it suffices to show that  $ [D]$ is divisible by  $ m$ in  $ H_2(M_3 \times M_2; \zz)$ or equivalently show that $ [D]_m^* $ vanishes in  $ H^2(M_3 \times M_2; \zz /m)$.

Consider the diagram below. 
% https://q.uiver.app/#q=WzAsMTEsWzMsMiwiTV8xIl0sWzMsMSwiTV8xXFx0aW1lcyBNXzEiXSxbMiwyLCJNXzIiXSxbMiwxLCJNXzIgXFx0aW1lcyBNXzIiXSxbMSwyLCJNXzMiXSxbMSwxLCJNXzMgXFx0aW1lcyBNXzIiXSxbMCwxLCJFIl0sWzMsMCwiRF8xIl0sWzIsMCwiRF8yPWZfMV57LTF9KERfMSkiXSxbMSwwLCJEPWZfMl57LTF9KERfMikiXSxbMCwwLCJcXHRpbGRle0R9Oj1mXnstMX0oRCkiXSxbMSwwLCJwIl0sWzIsMCwiXFxyaG9fMSJdLFszLDEsImZfMSJdLFs0LDIsIlxccmhvXzIiXSxbNSw0LCJwIiwyXSxbNSwzLCJmXzIiXSxbNiw1LCJmIl0sWzYsNF0sWzcsMSwiIiwxLHsic3R5bGUiOnsidGFpbCI6eyJuYW1lIjoiaG9vayIsInNpZGUiOiJib3R0b20ifX19XSxbOCwzLCIiLDEseyJzdHlsZSI6eyJ0YWlsIjp7Im5hbWUiOiJob29rIiwic2lkZSI6ImJvdHRvbSJ9fX1dLFs5LDUsIiIsMSx7InN0eWxlIjp7InRhaWwiOnsibmFtZSI6Imhvb2siLCJzaWRlIjoiYm90dG9tIn19fV0sWzMsMiwicCIsMl0sWzEwLDYsIiIsMCx7InN0eWxlIjp7InRhaWwiOnsibmFtZSI6Imhvb2siLCJzaWRlIjoiYm90dG9tIn19fV1d
\[\begin{tikzcd}
	{\tilde{D}:=f^{-1}(D)} & {D=f_2^{-1}(D_2)} & {D_2:=f_1^{-1}(D_1)} & {D_1:=\Delta} \\
	E & {M_3 \times M_2} & {M_2 \times M_2} & {M_1\times M_1} \\
	& {M_3} & {M_2} & {M_1}
	\arrow["p", from=2-4, to=3-4]
	\arrow["{\rho_1}", from=3-3, to=3-4]
	\arrow["{f_1}", from=2-3, to=2-4]
	\arrow["{\rho_2}", from=3-2, to=3-3]
	\arrow["p"', from=2-2, to=3-2]
	\arrow["{f_2}", from=2-2, to=2-3]
	\arrow["f", from=2-1, to=2-2]
	\arrow[from=2-1, to=3-2]
	\arrow[hook', from=1-4, to=2-4]
	\arrow[hook', from=1-3, to=2-3]
	\arrow[hook', from=1-2, to=2-2]
	\arrow["p"', from=2-3, to=3-3]
	\arrow[hook', from=1-1, to=2-1]
\end{tikzcd}\]
Note that $ f_1 = (\rho_1,\rho_1)$, $ f_2=(\rho_2, \text{id}_{ M_2})$, and $ p$ is projection onto the first factor. Let $ D_1 := \Delta(M_1 \times M_1)$ the diagonal and $ D_2:= f_1^{-1}(D_1)$. In fact, it is true that $ D=f_2^{-1}(D_2)$. To see this, note that by the definition of $ D$, any  $ (a,b) \in D$ satisfies that both $ \rho_2(a)$ and $ b$ are in the same fiber of $ \rho_1$. Thus $ \rho_1\rho_2(a) = \rho_1(b)$, which precisely says that $ f_1 f_2(a,b) \in \Delta(M_1\times M_1)$. That is, $ D \subseteq f_2^{-1}f_1^{-1}(D_1)$. Moreover, $ D_2$ are all pairs of elements in $ M_2$ where both live in the same fiber of $ \rho_1$. Any $ (x,y) \in f_2^{-1}(D_2)$ satisfies that $ \rho_2(x)$ and $ y$ live in the same fiber of $ \rho_1$. Since $ \rho_2$ is surjective and iterating through all $ \sigma^{i}$ gives access to all such pairs, we conclude that $ (x,y) \in D$ and therefore $ D = f_2^{-1}(D_2)$.

By \cref{lem:preimage} saying that preimage of a finite covering map of a submanifold corresponds to the induced map on cohomology class represented by the submanifold,
\begin{align*}
	[D]_m^* = f_2^* ([D_2]_m^* ) = f_2^* f_1^* ([D_1]_m^* ).
\end{align*}
Let us investigate $ H^2(M_1 \times M_1; \zz)$ where $ [D_1]^* $ resides. By the K\"{u}nneth formula, we have
\begin{align*}
	H^2(M_1 \times M_1; \zz) = H^2(M_1 \times \{*\} ;\zz) \oplus \left( H^{1}(M_1;\zz)  \otimes H^1(M_1;\zz) \right) \oplus H^2(\{*\} \times M_1 ;\zz).
\end{align*}
Hence, $ [D_1]^* $ is a linear combination of elements in these components. Notice that $ \rho_1^* : H^2(M_1; \zz /m) \cong \zz /m \to \zz /m \cong H^2(M_2; \zz /m)$ has degree $ m$ and thus a trivial homomorphism. Moreover, $ \rho_2^* : H^1(M_2; \zz /m) \to H^1(M_3; \zz /m)$ is also trivial by the definition of $ \rho_2$. Therefore, $ f_1^* $ kills the degree 2 components in $ [D_1]_m^* $ and $ f_2^* $ kills the remaining middle term in $ [D_2]_m^* $. Thus $ f_2^* f_1^* $ kills $ [D_1]_m^* $ which means $ [D]_m^*  = 0$.

This way, \cref{prop:divisible} gives us a branched covering $ E$ of  $ M_3$ ramified along $ D$. This  $ E$ is the AK manifold we seek.

\section{Nontriviality of $ e_i$}
\begin{lem} \label{lem:compute} 
Let $ \pi:E \to M$ and $ \widetilde{ \pi}: E \to M$ be two surface bundles over the same base space $ M$. Suppose that  $ f: \widetilde{ E} \to E$ is an $ m$-fold cyclic branched covering map ramified along an oriented submanifold  $ D \subset E$ of codimension 2. Suppose further that $ D$ intersects each fiber of  $ \pi$ transversely at exactly  $ m$ points and the following diagram commutes.
% https://q.uiver.app/#q=WzAsNixbMSwyLCJNXzIiXSxbMCwxLCJcXHRpbGRle0V9Il0sWzEsMSwiRSJdLFswLDJdLFswLDAsIlxcdGlsZGV7RH0iXSxbMSwwLCJEIl0sWzEsMiwiZiJdLFsyLDAsIlxccGkiXSxbMSwwLCJcXHRpbGRle1xccGl9IiwyXSxbNSwyLCIiLDAseyJzdHlsZSI6eyJ0YWlsIjp7Im5hbWUiOiJob29rIiwic2lkZSI6ImJvdHRvbSJ9fX1dLFs0LDEsIiIsMCx7InN0eWxlIjp7InRhaWwiOnsibmFtZSI6Imhvb2siLCJzaWRlIjoiYm90dG9tIn19fV0sWzQsNSwiXFxjb25nIiwyXV0=
\[\begin{tikzcd}
	{\tilde{D}:=f^{-1}(D)} & D \\
	{\tilde{E}} & E \\
	{} & {M_2}
	\arrow["f", from=2-1, to=2-2]
	\arrow["\pi", from=2-2, to=3-2]
	\arrow["{\tilde{\pi}}"', from=2-1, to=3-2]
	\arrow[hook', from=1-2, to=2-2]
	\arrow[hook', from=1-1, to=2-1]
	\arrow["\cong"', from=1-1, to=1-2]
\end{tikzcd}\]

Then the following equalities hold:
 \begin{enumerate}[label=(\roman*)]
	 \item $ f^* ([D]^* ) = m [\widetilde{ D}]^* $.
	 \item $ e(\widetilde{ V}) = f^* (e(V)) - (m-1)[\widetilde{ D}]^* = f^* \left( e(V) - \left( 1-\frac{1}{m} \right) [\widetilde{ D}]^*  \right) $, where $ e(V)$ and  $ e(\widetilde{ V})$ are the Euler classes of the vertical tangent bundle of $ E$ and  $ \widetilde{ E}$ respectively.
\end{enumerate}
\end{lem}
We remark that the commutative diagram simply requires that the branched covering is compatible with the surface bundle structure: $ f$ restricted to each fiber of $ \widetilde{ E}$ is a branched covering map over the fiber of $ E$. The top part simply restates that the ramification locus remains unchanged by the branched covering up to isomorphisms. Moreover, the second statement can be intuitively understood as an inclusion-exclusion principal. When we pullback to the branched cover, everything is multiplied by $ m$ except for the ramification locus, so we need to subtract the  $ (m-1)$ copies of the ramificiation locus that we overcount.
\begin{proof}
\begin{enumerate}[label=(\roman*)]
	\item This follows from \cref{lem:preimage}? we shall only prove it for the case when $ \dim D =0$. TODO 
	\item Let $ N(D), N(\widetilde{ D})$ denote the closed tubular neighborhoods of $ D$ and  $ \widetilde{ D}$ respectively. Let $ e:= e(V)$ and $ \widetilde{ e} := e(\widetilde{ V})$. Since $ f^* V \cong \widetilde{ V}$, by naturality we have $ f^* e(V) = e(f^* V) =e(\widetilde{ V})$. It is thus not hard to see that the images of $ e$ and  $ \widetilde{ e}$ coincide in the commutative diagram.

From the exact sequence of relative cohomology, we obtain
\begin{align*}
	\cdots \to H^2\left( \widetilde{ E}, \widetilde{ E}\setminus \inte N\left( \widetilde{ D} \right)  \right) \xrightarrow{ p^* }  H^2 \left( \widetilde{ E} \right) \xrightarrow{ \widetilde{ i}^* }  H^2 \left( \widetilde{ E} \setminus \inte N \left( \widetilde{ D} \right)  \right) \to \cdots 
\end{align*}
We also have
\begin{align*}
	H^2\left( \widetilde{ E}, \widetilde{ E} \setminus \inte N\left( \widetilde{ D} \right)  \right) &\cong H^2 \left( N \left( \widetilde{ D} \right) , \partial N \left( \widetilde{ D} \right)  \right) && \text{ excision} \\
													 & \cong H_{n-2}\left( \widetilde{ D} \right) = \zz \left[ \widetilde{ D} \right]  && \text{ Poincar\'{e} duality} 
\end{align*}
where we excise out $ \widetilde{ E} \setminus N'(\widetilde{ D})$; we make $ N'(\widetilde{ D}) \supsetneq N(\widetilde{ D})$ to satisfy the assumption of excision.

Since $( \widetilde{ E}, \widetilde{ E} \setminus \inte N( \widetilde{ D})) $ is clearly a good pair, we know that
\begin{align*}
	H^2(\widetilde{ E}, \widetilde{ E} \setminus \inte N\widetilde{ D}) \cong H_{n-2}(\widetilde{ E}, \widetilde{ E} \setminus \inte N\widetilde{ D}) \cong \widetilde{ H}_{n-2}\left( \widetilde{ E}/ (\widetilde{ E} \setminus \inte N\widetilde{ D}) \right) \cong H_{n-2}(\widetilde{ D}) \cong H^2(\widetilde{ D}).
\end{align*}
Thus, we can think of $ p^* $ being induced by the quotient map $ p: \widetilde{ E} \to \widetilde{ E} / (\widetilde{ E} \setminus N\widetilde{ D})$, which is identity on $ \widetilde{ D}$ and collapses everything else (up to homotopy) to a point. Thus $ p^*$ should be "identity'' on $H^2(\widetilde{ D})$ and the image of $ p^* $ should just be multiples of the  generator $ [\widetilde{ D}]^* $.

From the exact sequence, we obtain $ \ker \widetilde{ i}^* = \im p^* $. Since $ \widetilde{ e}, f^*(e)$ has the same image under $ \widetilde{ i}^* $, we have $ \widetilde{ e} \in f^* (e) + \ker \widetilde{ i}^*  = f^* (e) + \im p^* $. That is,
\begin{align} \label{eq:euler} 
\widetilde{ e}	=f^* (e) + a [\widetilde{ D}]^*, \text{ for some }  a \in \zz.
\end{align}

Now let us compute $ a$ using Euler characteristics of the fibers. Let $ g,\widetilde{ g}$ be the genus of fibers $ F,\widetilde{ F}$ of $ E, \widetilde{ E}$ respectively. We have
\begin{align*}
	\chi \left( \widetilde{ F} \right) = 2-2\widetilde{ g} = m(2-2g-m)+m.
\end{align*}

By restricting \cref{eq:euler} to $ \widetilde{ F}$, we have the equality:
 \begin{align*}
	 \chi \left( \widetilde{ F} \right)=\langle \widetilde{ e}|_{\widetilde{ F}}, [ \widetilde{ F}]  \rangle &= \langle f^* e, [ \widetilde{ F}]  \rangle + a \langle [ \widetilde{ D} ] ^* , [ \widetilde{ F} ]  \rangle\\
															   &= m \langle e|_F, [F] \rangle + a m && \text{ kronecker product is intersection} \\
															   &= m \chi(F) + am \\
															   m(2-2g-m)+m&= m(2-2g) + am
\end{align*}
which yields $ a=1-m$.
\end{enumerate}
\end{proof}

From now on, with an abuse of notation, we shall interchangeably write the (co)homology class, its Poincar\'{e} dual, and the integer it represents (if applicable).

Geometrically, $ e_i(E)$ projects the intersection of zero loci of $ i+1$ vertical vector fields, which lives in $ E$, down to the base $ M$. Since intersection number is invariant under small perturbation, we can always perturb the vertical vector fields so that intersection points always lie in distinct fibers. This way, an intersection point downstairs corresponds to exactly one intersection point upstairs, and we have $ e_i(E) = \pi_!\left( e^{i+1}(V) \right) = e^{i+1}$ as a number. Hence, it suffices to show that $ e^{i+1}(E)$ is non-trivial.

\begin{thm}
The first MMM class $ e_1$ is a nontrivial characteristic class.
\end{thm}
\begin{proof}
It suffices to demonstrate that $ e_1$ is nonzero for the AK manifold.

By \cref{lem:compute} (ii),
\begin{align*}
	e\left( \widetilde{ V} \right) &= \pi^* (e(V)) - [\widetilde{ D}]^*  \\
\end{align*}
Squaring both sides yields
\begin{align*}
	e(\widetilde{ V})^2 &= \pi^* (e(V)^2) -2 \pi^* (e(V))[ \widetilde{ D}]^* + ([\widetilde{ D}]^*)^2 &&\text{ naturality}  \\
	e_1(\widetilde{ E}) &= m e_1(E) - 2\pi^*(e(V))[\widetilde{ D}]^* +([\widetilde{ D}]^* ) && \text{ \cref{lem:preimage}}  \\
			    &= m e_1(E) - 2\left(e \left( \widetilde{ V} \right) + \left[ \widetilde{ D} \right]^*  \right) \left[ \widetilde{ D} \right]^* + \left( \left[ \widetilde{ D} \right]^*   \right)^2 &&\text{ substitution}   \\
			    &= m e_1(E) - 2 e \left( \widetilde{ V} \right) \left[ \widetilde{ D} \right] ^* - I\left( \widetilde{ D}, \widetilde{ D} \right) && \smile \text{ is } \cap 
\end{align*}
Since each fiber of $\widetilde{ E}$ is transversal to $ \widetilde{ D}$, we see that $ \widetilde{ V}|_{\widetilde{ D}} \oplus T\widetilde{ D} \cong T\widetilde{ E}|_{\widetilde{ D}}$ which implies that $\widetilde{ V}|_{\widetilde{ D}} \cong N\widetilde{ D}$ the normal bundle. Thus $ e(\widetilde{ V})[\widetilde{ D}]^* = e(\widetilde{ V}|_{\widetilde{ D}})= e(N\widetilde{ D}) = I(\widetilde{ D},\widetilde{ D})$. Thus we conclude that
\begin{align*}
	e_1\left( \widetilde{ E} \right) = m e_1(E) - 3 I\left( \widetilde{ D}, \widetilde{ D} \right) .
\end{align*}
Now we compute
\begin{align*}
	I \left( \widetilde{ D}, \widetilde{ D} \right) &= \left[ \widetilde{ D} \right] ^* \smile  \left[ \widetilde{ D} \right] ^*\\
							&= \frac{1}{m} f^* [D]^* \smile \frac{1}{m} f^* [D]^*  && \text{ \cref{lem:compute} (i)}  \\
							&= \frac{1}{m^2} f^* \left( [D]^*  \smile [D]^* \right)  && \text{ naturality} \\
							&= \frac{1}{m^2} f^* I(D,D) && \text{\cref{prop:cupintersection}} \\
							&= \frac{1}{m}I(D,D) && \text{ \cref{lem:compute}(i)} 
\end{align*}

Since $ D$ is the disjoint union of $ m$ graphs, by symmetry we have $ I(D,D) = m I\left( \Gamma_{\rho_2}, \Gamma_{\rho_2} \right)$. As before, $ V|_D \cong ND$. Moreover, we see that
% https://q.uiver.app/#q=WzAsNSxbMiwwLCJUTV8yIl0sWzIsMSwiTV8yIl0sWzEsMSwiRCBcXHN1YnNldCBNXzMgXFx0aW1lcyBNXzIiXSxbMSwwLCJURCJdLFswLDAsIlZ8X0QiXSxbMCwxXSxbMiwxLCJwIiwyXSxbMywyXSxbMywwXSxbNCwzLCJcXGNvbmciLDAseyJzdHlsZSI6eyJib2R5Ijp7Im5hbWUiOiJkYXNoZWQifX19XSxbNCwyXV0=
\[\begin{tikzcd}
	{V|_D} & TD & {TM_2} \\
	& {D \subset M_3 \times M_2} & {M_2}
	\arrow[from=1-3, to=2-3]
	\arrow["p"', from=2-2, to=2-3]
	\arrow[from=1-2, to=2-2]
	\arrow[from=1-2, to=1-3]
	\arrow["\cong", dashed, from=1-1, to=1-2]
	\arrow[from=1-1, to=2-2]
\end{tikzcd}\]
Thus, $ TD \cong V|_D \cong ND$, which implies that $ T \Gamma_{\rho_2} \cong N \Gamma_{\rho_2}$. It follows that
\begin{align*}
	e_1(E) &= m \cdot  e_1(M_3 \times M_2) - \frac{3}{m} I(D,D) \\
	&= 0 - 3 I\left( \Gamma_{\rho_2}, \Gamma_{\rho_2} \right)  \\
	&= -3 e\left( N \Gamma_{\rho_2} \right)  \\
	&= -3 e \left( T \Gamma_{\rho_2} \right)  \\
	&= -3 \chi\left( \Gamma_{\rho_2} \right)  \\
	&= -3 \chi(M_3) && \text{graph is diffeo to domain} \\
	&= -3 (2-2g_3) \neq 0 .
\end{align*}
\end{proof}
\end{document}
