\documentclass[12pt]{article}
\newcommand{\alert}[1]{{\bf \color{red} [Alert:] #1}}
\newcommand{\todo}[1]{{\bf \color{orange} [TODO:] #1}}
\newcommand{\real}[1][]{\mathbb{R}^{#1}}
\newcommand{\myeqn}[1]{(\ref{#1})}
\newcommand{\myex}[1]{Example \ref{#1}}
\newcommand{\defeq}{\stackrel{\mathrm{def}}{=}}
\newcommand{\parder}[2]{\frac{\partial #1}{\partial #2}}
\newcommand{\Lie}[3][]{\mathsf{L}_{#3}^{#1} #2}
\newcommand{\LieA}[1]{\mathsf{Lie}(#1)}
\newcommand{\lieder}[2]{\mathcal{L}_{#2} #1}
\renewcommand{\t}{^{\mbox{\tiny\sf T}}}
\newcommand{\trans}{^{\mbox{\tiny\sf T}}}
\newcommand{\markup}[1]{\{\textbf{#1}\}}
\newcommand{\msub}[1]{_\mathrm{#1}}
\newcommand{\msup}[1]{^\mathrm{#1}}
\newcommand{\inv}[1]{#1^{-1}}
\newcommand{\pinv}[1]{{#1}^{+}}
\newcommand{\myfracA}[2]{\displaystyle{\frac{#1}{#2}}}
\newcommand{\myfracB}[2]{{#1}/{#2}}
\newcommand{\mydiffA}[1]{\dot{#1}}
\newcommand{\mydiffB}[2]{\myfracA{\mathrm{d}{#1}}{\mathrm{d}{#2}}}
\newcommand{\ball}[2]{\mathcal{B}_{#1}\left(#2\right)}
\newcommand{\acos}[1]{\cos^{-1}\left(#1\right)}
\newcommand{\asin}[1]{\sin^{-1}\left(#1\right)}
\newcommand{\mani}{\mathcal{M}}
\newcommand{\tang}[2]{\mathsf{T}_{#1} #2}
\newcommand{\LieB}[2]{[ #1, #2 ]}
\newcommand{\LieBad}[3][]{\mathsf{ad}_{#2}^{#1} #3}
\newcommand{\ReachVT}{\mathcal{R}^V_T}
\newcommand{\ReachVt}{\mathcal{R}^V_t}
\newcommand{\ReachVTe}{\mathcal{R}^V_{\le T}}
\newcommand{\ReachT}{\mathcal{R}_T}
\newcommand{\Reacht}{\mathcal{R}_t}
\newcommand{\ReachTe}{\mathcal{R}_{\le T}}
\newcommand{\accLA}[1]{\mathsf{Lie}(#1)}
\newcommand{\accD}{\Delta_{\mathcal{F}}}
\newcommand{\accSA}{\mathsf{Lie}(\mathcal{G},f)}
\newcommand{\accDS}{\Delta_{\mathcal{G}}}
\newcommand{\eval}[3]{\mathsf{Ev}^{#2}_{#1}\left( #3 \right)}
\newcommand{\stlc}{\textsc{stlc}}
\newcommand{\clf}{\textsc{clf}}
\newcommand{\jqlf}{\textsc{jqlf}}
\newcommand{\dlas}{\textsc{dlas}}
\newcommand{\Ad}[2]{\mathsf{Ad}_{#1} #2}
\newcommand{\xe}{\ensuremath{x_e}}
\newcommand{\lebg}[1]{\mathcal{L}_{#1}}
\newcommand{\lebgx}[1]{\mathcal{L}_{#1 \mathrm{e}}}
\newcommand{\dom}{D}
\newcommand{\domT}{[t_0,\infty) \times D}
\newcommand{\rarrow}{\rightarrow}
\renewcommand{\d}{\mathrm{d}}
\renewcommand{\Re}{\mathbb{R}}
\newcommand{\C}{\mathrm{C}}

\newcommand{\QED}{{\unskip\nobreak\hfil\penalty50\hskip2em\vadjust{}
		\nobreak\hfil$\Box$\parfillskip=0pt\finalhyphendemerits=0\par}\vspace{0.1cm}}
\newcommand{\eoEx}{{\unskip\nobreak\hfil\penalty50\hskip0em\vadjust{}
		\nobreak\hfil$\Large\Diamond$\parfillskip=0pt\finalhyphendemerits=0\par}\vspace{0.1cm}}

\newcommand{\sgn}{\ensuremath{\operatorname{sgn}}}
\newcommand{\sat}{\ensuremath{\operatorname{sat}}}

\newcommand{\half}{\frac{1}{2}}
\newcommand{\shalf}{\mbox{$\frac{1}{2}$}}
\newcommand{\marcom}[1]{\marginpar{\footnotesize #1}}
\newcommand{\der}{\mathrm{D}}
\newcommand{\e}{\mathrm{e}}
\newcommand{\dt}{\mathrm{d}t}

\newcommand{\cA}{\ensuremath{\mathcal{A}}}
\newcommand{\cB}{\ensuremath{\mathcal{B}}}
\newcommand{\cG}{\ensuremath{\mathcal{G}}}
\newcommand{\cK}{\ensuremath{\mathcal{K}}}
\newcommand{\cW}{\ensuremath{\mathcal{W}}}
\newcommand{\cZ}{\ensuremath{\mathcal{Z}}}
\newcommand{\cS}{\ensuremath{\mathcal{S}}}
\newcommand{\cD}{\ensuremath{\mathcal{D}}}
\newcommand{\cP}{\ensuremath{\mathcal{P}}}
\newcommand{\cV}{\ensuremath{\mathcal{V}}}
\newcommand{\cL}{\ensuremath{\mathcal{L}}}
\newcommand{\cN}{\ensuremath{\mathcal{N}}}
\newcommand{\cI}{\ensuremath{\mathcal{I}}}
\newcommand{\cR}{\ensuremath{\mathcal{R}}}
\newcommand{\cM}{\ensuremath{\mathcal{M}}}
\newcommand{\cC}{\ensuremath{\mathcal{C}}}
\newcommand{\cF}{\ensuremath{\mathcal{F}}}
\newcommand{\cH}{\ensuremath{\mathcal{H}}}
\newcommand{\cO}{\ensuremath{\mathcal{O}}}
\newcommand{\cX}{\ensuremath{\mathcal{X}}}
\newcommand{\cY}{\ensuremath{\mathcal{Y}}}
\newcommand{\Ci}{\ensuremath{\mathcal{C}^\infty}}
\newcommand{\ISS}{\textsc{iss}}
\newcommand{\LISS}{\textsc{liss}}
\newcommand{\GAS}{\textsc{gas}}
\newcommand{\GS}{\textsc{gs}}
\newcommand{\LES}{\textsc{les}}
\newcommand{\GUAS}{\textsc{guas}}
\newcommand{\BIBO}{\textsc{bibo}}
\newcommand{\spec}{\ensuremath{\operatorname{spec}}}
\newcommand{\spn}{\ensuremath{\operatorname{span}}}
\renewcommand{\i}{\mathrm{i\,}}

\renewcommand{\implies}{\Rightarrow}

\renewcommand{\theenumi}{$\roman{enumi})$}
\renewcommand{\labelenumi}{\theenumi}

\font\ptmten=zptmcmrm scaled 1200
\newcommand{\w}{\mbox{{\ptmten w}}}
\newcommand{\z}{\mbox{{\ptmten z}}}
\renewcommand{\Re}{\mathbb{R}}

\newcommand{\cl}{\operatorname{cl}}
\newcommand{\intr}{\operatorname{int}}
\newcommand{\rank}{\operatorname{rank}}
\newcommand{\co}{\operatorname{co}}
\newcommand{\aff}{\operatorname{aff}}

\theoremstyle{plain}
\newtheorem{theorem}{Theorem}[chapter]
\newtheorem{claim}[theorem]{Claim}
\newtheorem{corollary}[theorem]{Corollary}
\newtheorem{prop}[theorem]{Proposition}
\newtheorem{fact}[theorem]{Fact}
\newtheorem{lemma}[theorem]{Lemma}

\newtheorem{remark}{Remark}[chapter]

\theoremstyle{definition}
\newtheorem{assume}[theorem]{Assumption}
\newtheorem{defn}[theorem]{Definition}
\newtheorem{problem}[theorem]{Problem}
\newtheorem{exercise}{Exercise}
\newtheorem{example}[theorem]{Example}


\begin{document}
\centerline {\textsf{\textbf{\LARGE{Homework 1}}}}
\centerline {Jaden Wang}
\vspace{.15in}

\begin{problem}[1.1.1]
\begin{enumerate}[label=(\alph*)]
	\item 
	\begin{align*}
	u^* &= -Q^{-1}S \\
	    &= - \begin{pmatrix} -2&-1\\-1&-1 \end{pmatrix} \begin{pmatrix} 0\\1 \end{pmatrix}  \\
	    &= \begin{pmatrix} 1\\1 \end{pmatrix} 
\end{align*}
Since $ Q < 0$,  $ u^* $ is a global maximum. Then
\begin{align*}
	L^* &= -\frac{1}{2} S^{T}Q^{-1}S= \frac{1}{2} S^{T}u^* \\
	    &= \frac{1}{2} \begin{pmatrix} 0&1 \end{pmatrix} \begin{pmatrix} 1\\1 \end{pmatrix}  \\
	&= \frac{1}{2} 
\end{align*}
\begin{align*}
	L_u &= Qu + S \\
	&= \begin{pmatrix} -u_1+u_2\\u_1-2u_2 + 1 \end{pmatrix}  
\end{align*}
\item
\begin{align*}
	u^* &= - \frac{1}{3} \begin{pmatrix} 2&-1\\-1&-1 \end{pmatrix} \begin{pmatrix} 0\\1 \end{pmatrix} \\
	&= \begin{pmatrix} \frac{1}{3}\\\frac{1}{3} \end{pmatrix}  \\
\end{align*}
Since $ Q$ is indefinite,  $ u^* $ is a saddle point.
\begin{align*}
	L^* &= \frac{1}{2} S^{T} u^*  \\
	    &= \frac{1}{2} \begin{pmatrix} 0&1 \end{pmatrix} \begin{pmatrix} \frac{1}{3} \\ \frac{1}{3} \end{pmatrix}  \\
	    &= \frac{1}{6} 
\end{align*}

\begin{align*}
	L_u &= Qu + S \\
	&= \begin{pmatrix} -u_1+u_2\\u_1+2u_2+1 \end{pmatrix} 
\end{align*}
\end{enumerate}

TODO
\end{problem}
\begin{problem}[1.1.2]
We see that
\begin{align*}
Q= \begin{pmatrix} 2&-1\\-1&2 \end{pmatrix}, S= \begin{pmatrix} 3\\0 \end{pmatrix} 
\end{align*}
So
\begin{align*}
	x^* &= -Q^{-1}S = - \frac{1}{3} \begin{pmatrix} 2&1\\1&2 \end{pmatrix} \begin{pmatrix} 3\\0 \end{pmatrix}  \\
	&= \begin{pmatrix} -2\\-1 \end{pmatrix}   
\end{align*}
So 


\end{problem}

\begin{problem}[1.1.3]
\begin{align*}
	\nabla f(x,y) = \begin{pmatrix} 2x\\4y^3 \end{pmatrix} 
\end{align*}
So $ \nabla f(0,0) = 0$, indeed a critical point. The Hessian is
\begin{align*}
	\nabla ^2 f(x,y) = \begin{pmatrix} 2&0\\0&12y^2 \end{pmatrix} 
\end{align*}
which is positive semidefinite and thus singular at the origin. 

Since $ x^2 \geq 0$ and $ y^{4} \geq 0$, it is clear that $ f(x,y) \geq 0$. So  $ f(0,0)=0$ is indeed a minimum of  $ f$.
\end{problem}

\begin{problem}[1.2.1]
The cost function is $ L(x,y) = \frac{1}{2} (x-20)^2 + \frac{1}{2}(y-30)^2$ and the constraint is $ F(x,y)= y-\sqrt{3}x = 0$. The Hamiltonian is
\begin{align*}
	H(x,y) = L(x,y)+ \lambda F(x,y).
\end{align*}
Then
\begin{align*}
	H_x &= x-20 - \sqrt{3} \lambda =0\\
	H_y &= y-30 + \lambda = 0 \\
	H_\lambda &= y- \sqrt{3} x=0 
\end{align*}
Solving this yields $ x = \frac{1}{2}(15\sqrt{3}+10) $ and $ y = \frac{1}{2}(45+10\sqrt{3}) $. The distance is then $ d^* = \sqrt{x^2+y^2} =36 $ miles. Thus the time is about $ 36 / 10 = 3.6$ hours.
\end{problem}

\begin{problem}[1.2.2]
The cost function is $ L(x_3,y_3) = \frac{1}{2}(x_3-x_1)^2 + \frac{1}{2} (y_3-y_1)^2 + \frac{1}{2}(x_3-x_2)^2 + \frac{1}{2} (y_3-y_2)^2$. The constraint is $ F(x_3,y_3) = \frac{1}{2}(x_3-x_1)^2 + \frac{1}{2} (y_3-y_1)^2 - \frac{1}{2}(x_3-x_2)^2 - \frac{1}{2} (y_3-y_2)^2 =0$. Therefore,
\begin{align*}
	H_{x_3} &= x_3-x_1 + x_3-x_2 + \lambda(x_3-x_1) - \lambda (x_3-x_2)=0\\
	H_{y_3} &= y_3-y_1 + y_3-y_2 + \lambda(y_3-y_1) - \lambda (y_3-y_2) \\
	H_\lambda &= 0 
\end{align*}
Solving this yields $ x_3 = $
\end{problem}
\end{document}
