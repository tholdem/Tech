\documentclass[12pt]{article}
\newcommand{\alert}[1]{{\bf \color{red} [Alert:] #1}}
\newcommand{\todo}[1]{{\bf \color{orange} [TODO:] #1}}
\newcommand{\real}[1][]{\mathbb{R}^{#1}}
\newcommand{\myeqn}[1]{(\ref{#1})}
\newcommand{\myex}[1]{Example \ref{#1}}
\newcommand{\defeq}{\stackrel{\mathrm{def}}{=}}
\newcommand{\parder}[2]{\frac{\partial #1}{\partial #2}}
\newcommand{\Lie}[3][]{\mathsf{L}_{#3}^{#1} #2}
\newcommand{\LieA}[1]{\mathsf{Lie}(#1)}
\newcommand{\lieder}[2]{\mathcal{L}_{#2} #1}
\renewcommand{\t}{^{\mbox{\tiny\sf T}}}
\newcommand{\trans}{^{\mbox{\tiny\sf T}}}
\newcommand{\markup}[1]{\{\textbf{#1}\}}
\newcommand{\msub}[1]{_\mathrm{#1}}
\newcommand{\msup}[1]{^\mathrm{#1}}
\newcommand{\inv}[1]{#1^{-1}}
\newcommand{\pinv}[1]{{#1}^{+}}
\newcommand{\myfracA}[2]{\displaystyle{\frac{#1}{#2}}}
\newcommand{\myfracB}[2]{{#1}/{#2}}
\newcommand{\mydiffA}[1]{\dot{#1}}
\newcommand{\mydiffB}[2]{\myfracA{\mathrm{d}{#1}}{\mathrm{d}{#2}}}
\newcommand{\ball}[2]{\mathcal{B}_{#1}\left(#2\right)}
\newcommand{\acos}[1]{\cos^{-1}\left(#1\right)}
\newcommand{\asin}[1]{\sin^{-1}\left(#1\right)}
\newcommand{\mani}{\mathcal{M}}
\newcommand{\tang}[2]{\mathsf{T}_{#1} #2}
\newcommand{\LieB}[2]{[ #1, #2 ]}
\newcommand{\LieBad}[3][]{\mathsf{ad}_{#2}^{#1} #3}
\newcommand{\ReachVT}{\mathcal{R}^V_T}
\newcommand{\ReachVt}{\mathcal{R}^V_t}
\newcommand{\ReachVTe}{\mathcal{R}^V_{\le T}}
\newcommand{\ReachT}{\mathcal{R}_T}
\newcommand{\Reacht}{\mathcal{R}_t}
\newcommand{\ReachTe}{\mathcal{R}_{\le T}}
\newcommand{\accLA}[1]{\mathsf{Lie}(#1)}
\newcommand{\accD}{\Delta_{\mathcal{F}}}
\newcommand{\accSA}{\mathsf{Lie}(\mathcal{G},f)}
\newcommand{\accDS}{\Delta_{\mathcal{G}}}
\newcommand{\eval}[3]{\mathsf{Ev}^{#2}_{#1}\left( #3 \right)}
\newcommand{\stlc}{\textsc{stlc}}
\newcommand{\clf}{\textsc{clf}}
\newcommand{\jqlf}{\textsc{jqlf}}
\newcommand{\dlas}{\textsc{dlas}}
\newcommand{\Ad}[2]{\mathsf{Ad}_{#1} #2}
\newcommand{\xe}{\ensuremath{x_e}}
\newcommand{\lebg}[1]{\mathcal{L}_{#1}}
\newcommand{\lebgx}[1]{\mathcal{L}_{#1 \mathrm{e}}}
\newcommand{\dom}{D}
\newcommand{\domT}{[t_0,\infty) \times D}
\newcommand{\rarrow}{\rightarrow}
\renewcommand{\d}{\mathrm{d}}
\renewcommand{\Re}{\mathbb{R}}
\newcommand{\C}{\mathrm{C}}

\newcommand{\QED}{{\unskip\nobreak\hfil\penalty50\hskip2em\vadjust{}
		\nobreak\hfil$\Box$\parfillskip=0pt\finalhyphendemerits=0\par}\vspace{0.1cm}}
\newcommand{\eoEx}{{\unskip\nobreak\hfil\penalty50\hskip0em\vadjust{}
		\nobreak\hfil$\Large\Diamond$\parfillskip=0pt\finalhyphendemerits=0\par}\vspace{0.1cm}}

\newcommand{\sgn}{\ensuremath{\operatorname{sgn}}}
\newcommand{\sat}{\ensuremath{\operatorname{sat}}}

\newcommand{\half}{\frac{1}{2}}
\newcommand{\shalf}{\mbox{$\frac{1}{2}$}}
\newcommand{\marcom}[1]{\marginpar{\footnotesize #1}}
\newcommand{\der}{\mathrm{D}}
\newcommand{\e}{\mathrm{e}}
\newcommand{\dt}{\mathrm{d}t}

\newcommand{\cA}{\ensuremath{\mathcal{A}}}
\newcommand{\cB}{\ensuremath{\mathcal{B}}}
\newcommand{\cG}{\ensuremath{\mathcal{G}}}
\newcommand{\cK}{\ensuremath{\mathcal{K}}}
\newcommand{\cW}{\ensuremath{\mathcal{W}}}
\newcommand{\cZ}{\ensuremath{\mathcal{Z}}}
\newcommand{\cS}{\ensuremath{\mathcal{S}}}
\newcommand{\cD}{\ensuremath{\mathcal{D}}}
\newcommand{\cP}{\ensuremath{\mathcal{P}}}
\newcommand{\cV}{\ensuremath{\mathcal{V}}}
\newcommand{\cL}{\ensuremath{\mathcal{L}}}
\newcommand{\cN}{\ensuremath{\mathcal{N}}}
\newcommand{\cI}{\ensuremath{\mathcal{I}}}
\newcommand{\cR}{\ensuremath{\mathcal{R}}}
\newcommand{\cM}{\ensuremath{\mathcal{M}}}
\newcommand{\cC}{\ensuremath{\mathcal{C}}}
\newcommand{\cF}{\ensuremath{\mathcal{F}}}
\newcommand{\cH}{\ensuremath{\mathcal{H}}}
\newcommand{\cO}{\ensuremath{\mathcal{O}}}
\newcommand{\cX}{\ensuremath{\mathcal{X}}}
\newcommand{\cY}{\ensuremath{\mathcal{Y}}}
\newcommand{\Ci}{\ensuremath{\mathcal{C}^\infty}}
\newcommand{\ISS}{\textsc{iss}}
\newcommand{\LISS}{\textsc{liss}}
\newcommand{\GAS}{\textsc{gas}}
\newcommand{\GS}{\textsc{gs}}
\newcommand{\LES}{\textsc{les}}
\newcommand{\GUAS}{\textsc{guas}}
\newcommand{\BIBO}{\textsc{bibo}}
\newcommand{\spec}{\ensuremath{\operatorname{spec}}}
\newcommand{\spn}{\ensuremath{\operatorname{span}}}
\renewcommand{\i}{\mathrm{i\,}}

\renewcommand{\implies}{\Rightarrow}

\renewcommand{\theenumi}{$\roman{enumi})$}
\renewcommand{\labelenumi}{\theenumi}

\font\ptmten=zptmcmrm scaled 1200
\newcommand{\w}{\mbox{{\ptmten w}}}
\newcommand{\z}{\mbox{{\ptmten z}}}
\renewcommand{\Re}{\mathbb{R}}

\newcommand{\cl}{\operatorname{cl}}
\newcommand{\intr}{\operatorname{int}}
\newcommand{\rank}{\operatorname{rank}}
\newcommand{\co}{\operatorname{co}}
\newcommand{\aff}{\operatorname{aff}}

\theoremstyle{plain}
\newtheorem{theorem}{Theorem}[chapter]
\newtheorem{claim}[theorem]{Claim}
\newtheorem{corollary}[theorem]{Corollary}
\newtheorem{prop}[theorem]{Proposition}
\newtheorem{fact}[theorem]{Fact}
\newtheorem{lemma}[theorem]{Lemma}

\newtheorem{remark}{Remark}[chapter]

\theoremstyle{definition}
\newtheorem{assume}[theorem]{Assumption}
\newtheorem{defn}[theorem]{Definition}
\newtheorem{problem}[theorem]{Problem}
\newtheorem{exercise}{Exercise}
\newtheorem{example}[theorem]{Example}


\begin{document}
\centerline {\textsf{\textbf{\LARGE{Homework 4}}}}
\centerline {Jaden Wang}
\vspace{.15in}
\begin{problem}[1]
\begin{enumerate}[label=(\alph*)]
The Hamiltonian of the problem is given by
\begin{align*}
	H(x_1,x_2,u,p_1,p_2) = \frac{1}{2} u ^2+ p_1x_2+p_2(u-x_2). 
\end{align*}
The adjoint equations are given by
\begin{align*}
	\dot{p_1} &= -H_{x_1} = 0 \\
	\dot{p_2} &= -H_{x_2} = p_2-p_1
\end{align*}
The first-order condition demands
\begin{align*}
	H_u = u+p_2 &= 0\\
	u&= -p_2 
\end{align*}
Plugging this into the differential equations yield
\begin{align*}
	\dot{x_1} &= x_2 \\
	\dot{x_2} &= -x_2-p_2 \\
\end{align*}
Together with the adjoint equations, we have 4 first-order equations and require 4 boundary conditions.
\item Since all initial and final times and states are fixed, we have 4 boundary conditions $ x_1(0)=x_2(0)=0$, $ x_1(3)=1$, and $ x_2(3)=2$. Mathematica yields
\begin{align*}
	u(t) &= \frac{6e^{3+t}+6e^{t}-e^{6}+4e^{3}-3}{e^{6}+4e^{3}-5}\\
	&=-0.6811 + 0.2642 e^{t}.
\end{align*}
\item When $ x_2(3)$ is free, in its place we instead have the transversality condition $ p_2(3) = 0$. This yields the solution
\begin{align*}
	u(t) &= -\frac{2e^3(e^{t}-e^{3})}{3e^{6}+4e^3-1 }\\
	&= 0.6256 - 0.0311e^{t} \\
\end{align*}
\item I would add a final penalty term $ \Phi$:
\begin{align*}
	\mathcal{ J} = \underbrace{ \frac{1}{2} \left( (x_1(3)-1)^2+(x_2(3)-2)^2 \right)  }_{\Phi(3) } + \frac{1}{2} \int_0^3 u^2 dt .
\end{align*}
Then instead of $ p_1(3)=p_2(3)=0$, we have
\begin{align*}
	p_1(3) &= \Phi_{x_1}(3) = x_1(3)-1 \\
	p_2(3) &= \Phi_{x_2}(3) = x_2(3)-2 \\
\end{align*}
Then new control is
\begin{align*}
	u(t) &= \frac{8e^{3+t}+6e^{t}+e^{6}+4e^3-3}{7e^{6}+8e^3-7}\\
	&= 0.1615+0.056 e^{t} .
\end{align*}
We see that $ x_1(3) = 0.8385$ and $ x_2(3) = 0.7142$, which are not very close to $ (1,2)$. To improve accuracy, I would increase the weight of the penalty term. We see that the cost in part (a) is  $ 4.2859$. The cost as a function of the weight coefficient  $ c$ is shown in the figure below:


\item It is clear that $ \Psi = \begin{pmatrix} 2&5 \end{pmatrix} $. By transversality condition from Equation 5.234, we have the boundary conditions
\begin{align*}
	\begin{pmatrix} -p_1(3)\\-p_2(3) \end{pmatrix} &= \begin{pmatrix} 2\\5 \end{pmatrix} \lambda 
\end{align*}
Together with the initial conditions $ x_1(0)=x_2(0)=0$ and the terminal condition $ 2 x_1(3) + 5x_2(3) = 20$, we have 5 boundary conditions for 4 differential equations and an unknown $ \lambda$. This allows us to solve by Mathematica and obtain the optimal control:
\begin{align*}
	u(t)=-p_2(t)= 1.4341+0.1071 e^{t}
\end{align*}
\item We have the same $\Psi $ but $ h(t) = 20 +\frac{t^2}{ 2}$ so $ \dot{h}(t_f) = t_f$. When $ t_f$ is also free, based on Equation 5.234 we have the transversality conditions
\begin{align*}
	\begin{pmatrix} H(t_f)\\ -p_1(t_f)\\-p_2(t_f) \end{pmatrix} &= \begin{pmatrix} -t_f\\ 2\\5 \end{pmatrix} \lambda 
\end{align*}
Note that by plugging in $ u=-p_2$, we have
\begin{align*}
	H &= -\frac{1}{2} p_2^2+(p_1-p_2)x_2\\
	H(t_f)&= -\frac{25}{2} \lambda + 3x_2(t_f) = -t_f 
\end{align*}
Together with two initial conditions and the terminal condition
\begin{align*}
	2x_1(t_f) +5x_2(t_f) = 20 + \frac{t_f^2}{ 2},
\end{align*}
we have a total of 6 boundary conditions to match the 4 differential equations and two unknowns $ \lambda$ and $ t_f$. Mathematica yields
\end{enumerate}
\end{problem}
\begin{problem}[2]
The Hamiltonian is
\begin{align*}
	H = \frac{1}{2} u^2 + p(ax+bu)
\end{align*}
The adjoint equation is
\begin{align*}
	\dot{p} = -H_x = ap \implies p(t) = C e^{at}.
\end{align*}
And the first-order condition is
\begin{align*}
	H_u = u+bp = 0 \implies u=-bp
\end{align*}
Thus
\begin{align*}
	\dot{x} = ax-b^2p = ax - b^2 C e^{at}, \quad x(0)=x_0,x(tf) = 0
\end{align*}
We have $ x(t) = b^2 C t e^{at} + A e^{at}$, $ x(0) = A = x_0$, and
\begin{align*}
	x(t_f) = b^2 C t_f e^{at_f} + x_0 e^{at_f} &= 0 \\
	C &= - \frac{x_0}{ b^2 t_f}
\end{align*}
Thus,
\begin{align*}
	u(t) = -bp = -b \cdot \left( - \frac{x_0}{b^2 t_f} \right) e^{at}  = \frac{x_0}{ b t_f} e^{at}
\end{align*}
\end{problem}
\begin{problem}[3]
We see that $ \Phi(t) = \frac{1}{2}(x(t)-1)^2$ and the Hamiltonian is
\begin{align*}
	H = \frac{1}{2} (x^2+u^2)+pu .
\end{align*}
The adjoint equation is
\begin{align*}
	\dot{p}= -H_x = x
\end{align*}
The first-order condition says
\begin{align*}
	H_u = u +p=0 \implies u = -p .
\end{align*}
Since $ x(2)$ is free, we have the transversality condition
 \begin{align*}
	p(2) = \Phi_x(2) = x(2)-1 .
\end{align*}
Thus we have two boundary conditions for two differential equations. Mathematica yields
\begin{align*}
	u(t) = -p(t) = \frac{\sin(1-t)}{\cos 1 - \sin 1 }.
\end{align*}
The associated cost is $ 14.9283$.
\end{problem}
\begin{problem}[5]
The cost has the Meyer form $ J = t_f$ so  $ \Phi(t) = t$. The Hamiltonian is
\begin{align*}
	H(x,y,p_1,p_2) &= p_1 r \cos \beta + p_2 r \sin \beta .
\end{align*}
The adjoint equations are
\begin{align*}
	\dot{p_1} &= -H_x = - p_1 \frac{x}{r} \cos \beta - p_2 \frac{x}{r} \sin \beta \\
	\dot{p_2} &= -H_y = p_1 \frac{y}{r} \cos \beta - p_2 \frac{x}{r} \sin \beta 
\end{align*}
First-order condition yields
\begin{align*}
	H_{\beta} = -p_1 r \sin \beta + p_2 r \cos \beta &= 0 \\
	\tan \beta &= \frac{p_2}{p_1} 
\end{align*}
Since $ t_f$ is free, transversality demands $ H(t_f) = - \Phi_t(t_f) = -1$ and 
\end{problem}
\end{document}
