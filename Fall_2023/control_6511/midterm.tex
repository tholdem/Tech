\documentclass[12pt]{article}
\newcommand{\alert}[1]{{\bf \color{red} [Alert:] #1}}
\newcommand{\todo}[1]{{\bf \color{orange} [TODO:] #1}}
\newcommand{\real}[1][]{\mathbb{R}^{#1}}
\newcommand{\myeqn}[1]{(\ref{#1})}
\newcommand{\myex}[1]{Example \ref{#1}}
\newcommand{\defeq}{\stackrel{\mathrm{def}}{=}}
\newcommand{\parder}[2]{\frac{\partial #1}{\partial #2}}
\newcommand{\Lie}[3][]{\mathsf{L}_{#3}^{#1} #2}
\newcommand{\LieA}[1]{\mathsf{Lie}(#1)}
\newcommand{\lieder}[2]{\mathcal{L}_{#2} #1}
\renewcommand{\t}{^{\mbox{\tiny\sf T}}}
\newcommand{\trans}{^{\mbox{\tiny\sf T}}}
\newcommand{\markup}[1]{\{\textbf{#1}\}}
\newcommand{\msub}[1]{_\mathrm{#1}}
\newcommand{\msup}[1]{^\mathrm{#1}}
\newcommand{\inv}[1]{#1^{-1}}
\newcommand{\pinv}[1]{{#1}^{+}}
\newcommand{\myfracA}[2]{\displaystyle{\frac{#1}{#2}}}
\newcommand{\myfracB}[2]{{#1}/{#2}}
\newcommand{\mydiffA}[1]{\dot{#1}}
\newcommand{\mydiffB}[2]{\myfracA{\mathrm{d}{#1}}{\mathrm{d}{#2}}}
\newcommand{\ball}[2]{\mathcal{B}_{#1}\left(#2\right)}
\newcommand{\acos}[1]{\cos^{-1}\left(#1\right)}
\newcommand{\asin}[1]{\sin^{-1}\left(#1\right)}
\newcommand{\mani}{\mathcal{M}}
\newcommand{\tang}[2]{\mathsf{T}_{#1} #2}
\newcommand{\LieB}[2]{[ #1, #2 ]}
\newcommand{\LieBad}[3][]{\mathsf{ad}_{#2}^{#1} #3}
\newcommand{\ReachVT}{\mathcal{R}^V_T}
\newcommand{\ReachVt}{\mathcal{R}^V_t}
\newcommand{\ReachVTe}{\mathcal{R}^V_{\le T}}
\newcommand{\ReachT}{\mathcal{R}_T}
\newcommand{\Reacht}{\mathcal{R}_t}
\newcommand{\ReachTe}{\mathcal{R}_{\le T}}
\newcommand{\accLA}[1]{\mathsf{Lie}(#1)}
\newcommand{\accD}{\Delta_{\mathcal{F}}}
\newcommand{\accSA}{\mathsf{Lie}(\mathcal{G},f)}
\newcommand{\accDS}{\Delta_{\mathcal{G}}}
\newcommand{\eval}[3]{\mathsf{Ev}^{#2}_{#1}\left( #3 \right)}
\newcommand{\stlc}{\textsc{stlc}}
\newcommand{\clf}{\textsc{clf}}
\newcommand{\jqlf}{\textsc{jqlf}}
\newcommand{\dlas}{\textsc{dlas}}
\newcommand{\Ad}[2]{\mathsf{Ad}_{#1} #2}
\newcommand{\xe}{\ensuremath{x_e}}
\newcommand{\lebg}[1]{\mathcal{L}_{#1}}
\newcommand{\lebgx}[1]{\mathcal{L}_{#1 \mathrm{e}}}
\newcommand{\dom}{D}
\newcommand{\domT}{[t_0,\infty) \times D}
\newcommand{\rarrow}{\rightarrow}
\renewcommand{\d}{\mathrm{d}}
\renewcommand{\Re}{\mathbb{R}}
\newcommand{\C}{\mathrm{C}}

\newcommand{\QED}{{\unskip\nobreak\hfil\penalty50\hskip2em\vadjust{}
		\nobreak\hfil$\Box$\parfillskip=0pt\finalhyphendemerits=0\par}\vspace{0.1cm}}
\newcommand{\eoEx}{{\unskip\nobreak\hfil\penalty50\hskip0em\vadjust{}
		\nobreak\hfil$\Large\Diamond$\parfillskip=0pt\finalhyphendemerits=0\par}\vspace{0.1cm}}

\newcommand{\sgn}{\ensuremath{\operatorname{sgn}}}
\newcommand{\sat}{\ensuremath{\operatorname{sat}}}

\newcommand{\half}{\frac{1}{2}}
\newcommand{\shalf}{\mbox{$\frac{1}{2}$}}
\newcommand{\marcom}[1]{\marginpar{\footnotesize #1}}
\newcommand{\der}{\mathrm{D}}
\newcommand{\e}{\mathrm{e}}
\newcommand{\dt}{\mathrm{d}t}

\newcommand{\cA}{\ensuremath{\mathcal{A}}}
\newcommand{\cB}{\ensuremath{\mathcal{B}}}
\newcommand{\cG}{\ensuremath{\mathcal{G}}}
\newcommand{\cK}{\ensuremath{\mathcal{K}}}
\newcommand{\cW}{\ensuremath{\mathcal{W}}}
\newcommand{\cZ}{\ensuremath{\mathcal{Z}}}
\newcommand{\cS}{\ensuremath{\mathcal{S}}}
\newcommand{\cD}{\ensuremath{\mathcal{D}}}
\newcommand{\cP}{\ensuremath{\mathcal{P}}}
\newcommand{\cV}{\ensuremath{\mathcal{V}}}
\newcommand{\cL}{\ensuremath{\mathcal{L}}}
\newcommand{\cN}{\ensuremath{\mathcal{N}}}
\newcommand{\cI}{\ensuremath{\mathcal{I}}}
\newcommand{\cR}{\ensuremath{\mathcal{R}}}
\newcommand{\cM}{\ensuremath{\mathcal{M}}}
\newcommand{\cC}{\ensuremath{\mathcal{C}}}
\newcommand{\cF}{\ensuremath{\mathcal{F}}}
\newcommand{\cH}{\ensuremath{\mathcal{H}}}
\newcommand{\cO}{\ensuremath{\mathcal{O}}}
\newcommand{\cX}{\ensuremath{\mathcal{X}}}
\newcommand{\cY}{\ensuremath{\mathcal{Y}}}
\newcommand{\Ci}{\ensuremath{\mathcal{C}^\infty}}
\newcommand{\ISS}{\textsc{iss}}
\newcommand{\LISS}{\textsc{liss}}
\newcommand{\GAS}{\textsc{gas}}
\newcommand{\GS}{\textsc{gs}}
\newcommand{\LES}{\textsc{les}}
\newcommand{\GUAS}{\textsc{guas}}
\newcommand{\BIBO}{\textsc{bibo}}
\newcommand{\spec}{\ensuremath{\operatorname{spec}}}
\newcommand{\spn}{\ensuremath{\operatorname{span}}}
\renewcommand{\i}{\mathrm{i\,}}

\renewcommand{\implies}{\Rightarrow}

\renewcommand{\theenumi}{$\roman{enumi})$}
\renewcommand{\labelenumi}{\theenumi}

\font\ptmten=zptmcmrm scaled 1200
\newcommand{\w}{\mbox{{\ptmten w}}}
\newcommand{\z}{\mbox{{\ptmten z}}}
\renewcommand{\Re}{\mathbb{R}}

\newcommand{\cl}{\operatorname{cl}}
\newcommand{\intr}{\operatorname{int}}
\newcommand{\rank}{\operatorname{rank}}
\newcommand{\co}{\operatorname{co}}
\newcommand{\aff}{\operatorname{aff}}

\theoremstyle{plain}
\newtheorem{theorem}{Theorem}[chapter]
\newtheorem{claim}[theorem]{Claim}
\newtheorem{corollary}[theorem]{Corollary}
\newtheorem{prop}[theorem]{Proposition}
\newtheorem{fact}[theorem]{Fact}
\newtheorem{lemma}[theorem]{Lemma}

\newtheorem{remark}{Remark}[chapter]

\theoremstyle{definition}
\newtheorem{assume}[theorem]{Assumption}
\newtheorem{defn}[theorem]{Definition}
\newtheorem{problem}[theorem]{Problem}
\newtheorem{exercise}{Exercise}
\newtheorem{example}[theorem]{Example}


\begin{document}
\centerline {\textsf{\textbf{\LARGE{Midterm}}}}
\centerline {Jaden Wang}
\vspace{.15in}

\begin{problem}[1]
\begin{enumerate}[label=(\alph*)]
	\item 
First we augment the domain $ D = \rr^{n+m}$ and let $ \overline{f}(\overline{x}) := f(x)$ and $ \overline{g}_i(\overline{x}) := g_i(x)+ \epsilon_i$ where $ \overline{x} = (x, \epsilon)$. Let $ \overline{F} = \begin{pmatrix} \overline{f}\\\overline{g} \end{pmatrix} $. Then
\begin{align*}
	\nabla \overline{F}(\overline{x}) = \begin{pmatrix} f'(x)& 0\\g'(x) & I_m \end{pmatrix} 
\end{align*}
where $ I_m$ is the $ m\times m$ identity matrix. Let the constrained domain $ D_0:= \{(x, \epsilon) \in \rr^{n} \times \rr^{m}: \epsilon \succeq 0\} = \rr^{n} \times \rr_+^{m}$. Suppose $ \overline{x}_0$ is a local minimizer. We see that $ \fcone(D_0,\overline{x}_0) = D_0 = \fcone(\rr^{n},x) \times \fcone(\rr_+^{m}, \epsilon)$ (which is already convex). Thus for any $ \overline{\xi} \in \fcone(D_0, \overline{x}_0)$, we have $ \overline{\xi} = (\eta, \xi) \in \fcone(\rr^{n},x) \times \fcone(\rr_+^{m}, \epsilon)$. Notice that $ \fcone( \rr_+^{m}, \epsilon) $ depends on whether $ \epsilon_i=0$ \emph{i.e.} at the boundary. If $ \epsilon_i$ is in the interior \emph{i.e.} $ \epsilon_i > 0$, then $ \fcone( \rr_+, \epsilon_i)  = \rr$. Otherwise, whenever $ \epsilon_i=0$, the feasible cone of $ i$th factor of $ \rr$ becomes $ \rr_+$.

First-order necessarily condition demands that
\begin{align*}
	\begin{pmatrix} \mu& \lambda^{T} \end{pmatrix} \nabla \overline{F}(\overline{x}_0) \overline{ \xi} &\geq 0\\
	\begin{pmatrix} \mu f'(x_0) + \lambda^{T}g'(x_0) & \lambda^{T}\end{pmatrix} \begin{pmatrix} \eta\\ \xi \end{pmatrix}  &\geq 0\\
	[\mu f'(x_0) + \lambda^{T}g'(x_0)] \eta + \lambda^{T} \xi &\geq 0.
\end{align*}
By setting $ \eta = 0$, we have $ \lambda^{T} \xi \geq 0$. Notice that whenever $ \epsilon_i > 0$, $ \xi_i \in \rr$ so by choosing other $ \xi_j = 0$, it forces $ \lambda_i = 0$. Whenever $ \epsilon_i = 0$ \emph{i.e.} $ g_i(x_0)=0$, $ \xi_i \geq 0$ so by choosing other $ \xi_j =0$, it forces $ \lambda_i \geq 0$. In either case, we have $ \lambda_i g_i(x_0) = 0 $ and $ \lambda \succeq 0$. Since $ \eta \in \rr^{n}$, by setting $ \xi = 0$ and plugging in $ \eta \neq 0$ and $ -\eta$, we have $ \mu f'(x_0) + \lambda^{T} g'(x_0) = 0$. This leads to the first-order necessary condition for the inequality constraint optimization problem: if $ x_0 \in \rr^{n}$ is a local minimizer of the problem, there exists $ \mu \in \{0,1\} $ and $ \lambda \in \rr^{m}$ such that they are not both zero and such that
\begin{align*}
	\mu f'(x_0) + \lambda^{T}g'(x_0) &= 0 \\
	\lambda_i g_i(x_0) &= 0 \\
	\lambda& \succeq 0.
\end{align*}
\item The Lagrangian for the augmented problem is
\begin{align*}
	\mathscr{L}(\overline{x},\mu,\lambda) = \mu \overline{f}(\overline{x}) + \lambda^{T} \overline{g}(\overline{x})
\end{align*}
According to Theorem 3.39, for the augmented problem we define
\begin{align*}
	J(\overline{x}_0):= \{\zeta=(\eta,\xi) \in \fcone( D_0, \overline{x}_0): f'(x_0) \eta \leq 0 \text{ and } g'(x_0) \eta + \xi = 0 \}. 
\end{align*}
We see that when $ \epsilon_i>0$, $ \xi_i \in \rr$, so we can always achieve $ g_i'(x)\eta + \xi_i =0$, so we can remove this inactive constraint $ g_i(x)$ in our consideration of $ J$ (and thus second-order conditions). 

When  $ \epsilon_i=0$ (active constraint, assume $ \lambda_i>0$), $ \xi_i \geq 0$, thus we have $ g_i'(x) \eta \leq 0$. Moreover, from previous part we know that by choosing $ \xi = 0$, we have
 \begin{align*}
	 [\mu f'(x_0)+\lambda^{T}g'(x_0)] \eta \geq 0
\end{align*}
Thus if $ f'(x_0) \eta \leq 0$, $ \mu f'(x_0) \eta \leq 0$ as well, which forces $ \lambda^{T} g'(x_0) \eta \geq 0$. Since $ \eta$ is any vector in the half-hyperplane that forms obtuse angle from $ f'(x_0)$, we must have $ \lambda_i g_i'(x_0) \eta \geq 0$ (otherwise we can just tweak to values of $ \eta$ to make the negative entry dominates). Since $ \lambda_i>0$, we have $ g_i'(x_0) \eta \geq 0$ and thus $ g_i'(x_0) \eta = 0$. Thus we eliminated $ \xi$ from the condition so $ J$ reduces to
\begin{align*}
	J(\overline{x}_0)=J(x_0)= \{\eta \in \fcone( \rr^{n}, x_0)=\rr^{n}: f'(x_0) \eta \leq 0 \text{ and } g_i'(x_0) \eta = 0 \ \forall \ g_i(x_0)=0, \lambda_i>0 \}. 
\end{align*}
Moreover, we see that
\begin{align*}
	\mathscr{L}_{\overline{x}} &= \mu \overline{f}_{\overline{x}}(\overline{x}) + \lambda^{T} \overline{g}_{\overline{x}}(\overline{x}) \\
				   &= \mu \begin{pmatrix} f_x(x)&0 \end{pmatrix} + \lambda^{T} \begin{pmatrix} g_x(x) & I_m \end{pmatrix}  \\
	\mathscr{L}_{\overline{x}\overline{x}} &= \mu \begin{pmatrix} f_{x x}(x)&0\\0&0 \end{pmatrix} + \lambda^{T} \begin{pmatrix} g_{x x}(x)&0\\0&0 \end{pmatrix}  \\
					       &= \begin{pmatrix}  \mathscr{L}_{x x}&0\\0&0 \end{pmatrix} 
\end{align*}
so $ \xi$ again becomes irrelevant. Thus the second-order necessary condition from Theorem 3.39 reduces to: let $ f:\rr^{n} \to \rr$ and $ g:\rr^{n} \to \rr^{m}$ be $ C^2$. If $ x_0$ is a local minimizer of the inequality constrained minimization problem, then for all $ \eta \in J(x_0)$, there exists nonzero $ (\mu,\lambda) \in \{0,1\} \times \rr^{m} $ such that it satisfies the first-order conditions in (a), and
\begin{align*}
	\eta^{T} \mathscr{L}_{x x}(x_0,\mu,\lambda)\eta \geq 0.
\end{align*}
\end{enumerate}
\end{problem}

\begin{problem}[2]
From homework and $ D=Av^2+BL^2 = Av^2+ B(mg)^2$, we have
\begin{align*}
	F(m,v(m),v'(m))&= \frac{c v}{ Av^2+B(mg)^2} \left( 1+\frac{m}{c} v' \right) \\
	\frac{\partial F}{\partial v} &= \frac{-Acv^2+Bc(mg)^2}{(Av^2+B(mg)^2)^2 } = \frac{-c(Av^2-B(mg)^2)}{(Av^2+B(mg)^2)^2 }\\
	\frac{\partial F}{\partial v'} &= \frac{mv}{Av^2+B(mg)^2} \\
	\frac{d}{dm} \frac{\partial F}{\partial v'} &= \frac{v+mv'}{ Av^2+ B(mg)^2} -\frac{mv(2Avv'+2Bmg^2)}{ (Av^2+B(mg)^2)^2} \\
	&= \frac{v+mv'}{ Av^2+ B(mg)^2} - \frac{2Av^2v'+2B(mg)^2v}{(A v^2+B(mg)^2)^2 } \\
	&= \frac{(v-mv')(Av^2-B(mg)^2)}{(Av^2+B(mg)^2)^2 } 
\end{align*}
Euler-Lagrange demands that
\begin{align*}
	F_v &= \frac{d}{dm} F_{v'}\\
	(v-mv'+c)(Av^2-B(mg)^2) &= 0 && \text{ denominator}>0 \\
	\frac{dv}{v+c} = \frac{dm}{m} &\text{ or } Av^2=B(mg)^2\\ 
	v(m)=C_1m-c &\text{ or } v(m) = mg \sqrt{\frac{B}{A}}  
\end{align*}
Since $ c$ is a positive constant, the first equation would imply that the velocity is negative when mass is zero, which makes no physical sense. It follows that the extremal is $ v(m) = mg \sqrt{\frac{B}{A}} $. Since the problem intuitively should have a maximum, and this is the only candidate, this must be the maximizer.

\end{problem}

\begin{problem}[3]
For simplicity, write $ T(v) = Av^2+\frac{B}{Cv^2}$ where $ A,B,C>0$ are corresponding constants. We see that the domain of  $ T$ is implicitly  $ \rr_{++}$ which is a convex set. Since $ Av^2$ and $ \frac{B}{Cv^2}$ are clearly strictly convex in this domain based on their epigraphs, their sum which is $ T$ is also strictly convex (clear from definition of convex functions): if $ f=g+h$ where $ g,h$ are strictly convex functions, then
 \begin{align*}
	 f(tx+(1-t)y) &= g(tx+(1-t)y)+ h(tx+(1-t)y)\\ 
		      &< tg(x)+(1-t)g(y)+ th(x)+(1-t)h(y)\\ 
		      &= t(g+h)(x) + (1-t)(g+h)(y) = tf(x)+(1-t)f(y)
\end{align*}
First order condition is
\begin{align*}
	T'(v) = 2Av-\frac{2B}{ Cv^3} &= 0 \\
	2ACv^{4} -2B &= 0 && C>0,v>0\\
	v^*  &= \sqrt[4]{\frac{B}{AC}}  
\end{align*}
Since $ T''(v) = 2A + \frac{6B}{Cv^{4} } >0$ for all $ v$, we see that  $ v^* $ is a strict local minimizer. Since $ T$ is strictly convex, Theorem 1.30 gives that  $ v^* $ is a global minimizer and Proposition 1.31 states that this is the unique global minimizer. Let $ C_p := C_{D_{par}}$, we have
\begin{align*}
	T(v^* ) &= A \sqrt{\frac{B}{AC}} + \frac{B}{C\sqrt{\frac{B}{AC}}  } = \sqrt{\frac{AB}{ C}} + \sqrt{\frac{AB}{ C}} = 2\sqrt{C_p K W^2}  \\
	C_L &= \frac{W}{Cv^2} = \sqrt{\frac{C_p}{ K}}  \\
	C_D &= C_p + KC_L^2 = 2 C_p \\
	C_L / C_D &= \frac{1}{2 \sqrt{K C_p} } 
\end{align*}
\end{problem}

\begin{problem}[4]
First, if $ a_1=a_2=0$, the function is constantly zero so every $ (x_1,x_2)$ that satisfies the constraint is a global minimizer. So we assume at least one $ a_i \neq 0$.

Suppose $ \mu=1$. If $ \lambda =0$, the constraint vanishes so the nonzero linear function goes to $ -\infty$ and has no minimum, so we need $ \lambda\neq 0$. When $ \mu=0$, $ \lambda \neq 0$ by assumption. Thus in either case, $ \lambda \neq 0$.
\begin{align*}
	\mathscr{L}(x,\mu,\lambda) &= \mu(a_1x_1+a_2x_2)+ \lambda(b_1x_1^2+b_2x_2^2) =0\\
	\mathscr{L}_x&= \begin{pmatrix} \mu a_1+2\lambda b_1 x_1\\ \mu a_2+2 \lambda b_2 x_2\end{pmatrix} = \begin{pmatrix} 0\\0 \end{pmatrix}  
\end{align*}

Assume that $ b_1 \neq 0$. Then we can use the constraint to solve $ x_1^2 =- \frac{b_2}{ b_1} x_2^2 $. If $ -\frac{b_2}{ b_1} \geq 0$, then $ x_1 = \pm\sqrt{-\frac{b_2}{ b_1}} x_2$ and $ f$ becomes a linear equation in  $ x_2$ with no constraint. Since at least one $ a_i \neq 0$, this unconstrained linear function goes to $ -\infty$, there is no solution. Now suppose $ -\frac{b_2}{b_1}<0$, then $ x_1^2 = -\frac{b_2}{b_1} x_2^2 \leq 0$ so $ x_1^2 =0$. This forces $ x_1=x_2=0$. Since $ g'(0,0) = \begin{pmatrix} 0\\0 \end{pmatrix} $, the solution is abnormal. This candidate satisfies the first-order conditions for $ \mu=0$. Thus $ \lambda$ can be any nonzero number. The second order condition requires that $ \mathscr{L}_{x x}(x,0,\lambda) \succ 0$ since null space of  $ g'$ is  $ \rr^2$. We see that
\begin{align*}
	\mathscr{L}_{x x} = \begin{pmatrix} 2b_1 \lambda&0\\0& 2b_2 \lambda \end{pmatrix} 
\end{align*}
which can be made positive definite since $ \frac{b_2}{b_1} >0$ and we can choose $ \lambda \neq 0$ so that $ \lambda b_1 >0$. It is thus a strict local minimizer. Then the minimum of $ f$ is 0 in this case.

By symmetry of the problem, the case when $ b_2\neq 0$ is the same. It remains to check when $ b_1 =b_2=0$. But this means that the nonzero linear function is unconstrained and goes to $ -\infty$.

Hence, the only solution to the nontrivial minimization problem is when $ b_1 \neq 0$ or $ b_2 \neq 0$ with $ f(0,0)=0$.
\end{problem}

\begin{problem}[5]
\begin{enumerate}[label=(\alph*)]
	\item 
We have
\begin{align*}
	F(t,x,\dot{x}) &= \frac{1}{2}((\dot{x}-x)^2- \alpha x^2) \\
	F_x &= -\dot{x} + (1- \alpha)x\\
	F_{\dot{x}} &= \dot{x}-x
\end{align*}
Euler-Lagrange demands
\begin{align*}
	\frac{d}{dt} F_{\dot{x}} &= \ddot{x} - \dot{x} =-\dot{x}+(1- \alpha) x\\
	\ddot{x} &= -( \alpha-1) x\\
	x(t) &= A \cos \sqrt{ \alpha-1}t + B \sin \sqrt{ \alpha-1}t &&  \alpha>1  \\
	x(0) &= A+0 = 0
\end{align*}
Thus the extremal trajectory is $ x(t) = B \sin \sqrt{ \alpha-1}t, B \in \rr $. It can be verified by direct integration that $ J$ is 0 for any  $ B$.
\item We see that $ F_{\dot{x}\dot{x}} = 1 >0$ so the accessory minimization problem is regular. For  $ \alpha=2$ and  $ T=\pi$, we have $ F_x = -x-r$, $F_{x x}=-1 $, $ F_r = r-x$,  $ F_{r r}= 1$, and $ F_{xr}=-1$. Let the perturbation be $ f$. Then
\begin{align*}
	\omega(t,x,r) = -\frac{1}{2} f^2-f \dot{f}+\frac{1}{2} \dot{f}^2
\end{align*}
The Legendre condition requires
\begin{align*}
	\omega_f &= \omega_{rt} + \omega(rf) \dot{f} + \omega_{r r} \ddot{f}\\
	-f -\dot{f} &= -\dot{f} + \ddot{f}-\dot{f} + \ddot{f}\\
	2 \ddot{f} -\dot{f} + f &= 0 \\
	f(t) &= e^{\frac{t}{4}} \left( C \cos \frac{\sqrt{7} }{ 4}t + D \sin \frac{\sqrt{7} }{ 4}t \right)  \\
	f(0) &= C = 0 \\
	f(\pi) &= D e^{\frac{\pi}{4}} \sin \frac{\sqrt{7} \pi}{ 4} =0  \implies D = 0
\end{align*}
Thus $ f(t) \equiv 0$ and there is no conjugate point between  $ [0,\pi]$ as everything vanishes in that interval. 
\end{enumerate}
\end{problem}


\end{document}
