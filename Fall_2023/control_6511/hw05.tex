\documentclass[12pt]{article}
\newcommand{\alert}[1]{{\bf \color{red} [Alert:] #1}}
\newcommand{\todo}[1]{{\bf \color{orange} [TODO:] #1}}
\newcommand{\real}[1][]{\mathbb{R}^{#1}}
\newcommand{\myeqn}[1]{(\ref{#1})}
\newcommand{\myex}[1]{Example \ref{#1}}
\newcommand{\defeq}{\stackrel{\mathrm{def}}{=}}
\newcommand{\parder}[2]{\frac{\partial #1}{\partial #2}}
\newcommand{\Lie}[3][]{\mathsf{L}_{#3}^{#1} #2}
\newcommand{\LieA}[1]{\mathsf{Lie}(#1)}
\newcommand{\lieder}[2]{\mathcal{L}_{#2} #1}
\renewcommand{\t}{^{\mbox{\tiny\sf T}}}
\newcommand{\trans}{^{\mbox{\tiny\sf T}}}
\newcommand{\markup}[1]{\{\textbf{#1}\}}
\newcommand{\msub}[1]{_\mathrm{#1}}
\newcommand{\msup}[1]{^\mathrm{#1}}
\newcommand{\inv}[1]{#1^{-1}}
\newcommand{\pinv}[1]{{#1}^{+}}
\newcommand{\myfracA}[2]{\displaystyle{\frac{#1}{#2}}}
\newcommand{\myfracB}[2]{{#1}/{#2}}
\newcommand{\mydiffA}[1]{\dot{#1}}
\newcommand{\mydiffB}[2]{\myfracA{\mathrm{d}{#1}}{\mathrm{d}{#2}}}
\newcommand{\ball}[2]{\mathcal{B}_{#1}\left(#2\right)}
\newcommand{\acos}[1]{\cos^{-1}\left(#1\right)}
\newcommand{\asin}[1]{\sin^{-1}\left(#1\right)}
\newcommand{\mani}{\mathcal{M}}
\newcommand{\tang}[2]{\mathsf{T}_{#1} #2}
\newcommand{\LieB}[2]{[ #1, #2 ]}
\newcommand{\LieBad}[3][]{\mathsf{ad}_{#2}^{#1} #3}
\newcommand{\ReachVT}{\mathcal{R}^V_T}
\newcommand{\ReachVt}{\mathcal{R}^V_t}
\newcommand{\ReachVTe}{\mathcal{R}^V_{\le T}}
\newcommand{\ReachT}{\mathcal{R}_T}
\newcommand{\Reacht}{\mathcal{R}_t}
\newcommand{\ReachTe}{\mathcal{R}_{\le T}}
\newcommand{\accLA}[1]{\mathsf{Lie}(#1)}
\newcommand{\accD}{\Delta_{\mathcal{F}}}
\newcommand{\accSA}{\mathsf{Lie}(\mathcal{G},f)}
\newcommand{\accDS}{\Delta_{\mathcal{G}}}
\newcommand{\eval}[3]{\mathsf{Ev}^{#2}_{#1}\left( #3 \right)}
\newcommand{\stlc}{\textsc{stlc}}
\newcommand{\clf}{\textsc{clf}}
\newcommand{\jqlf}{\textsc{jqlf}}
\newcommand{\dlas}{\textsc{dlas}}
\newcommand{\Ad}[2]{\mathsf{Ad}_{#1} #2}
\newcommand{\xe}{\ensuremath{x_e}}
\newcommand{\lebg}[1]{\mathcal{L}_{#1}}
\newcommand{\lebgx}[1]{\mathcal{L}_{#1 \mathrm{e}}}
\newcommand{\dom}{D}
\newcommand{\domT}{[t_0,\infty) \times D}
\newcommand{\rarrow}{\rightarrow}
\renewcommand{\d}{\mathrm{d}}
\renewcommand{\Re}{\mathbb{R}}
\newcommand{\C}{\mathrm{C}}

\newcommand{\QED}{{\unskip\nobreak\hfil\penalty50\hskip2em\vadjust{}
		\nobreak\hfil$\Box$\parfillskip=0pt\finalhyphendemerits=0\par}\vspace{0.1cm}}
\newcommand{\eoEx}{{\unskip\nobreak\hfil\penalty50\hskip0em\vadjust{}
		\nobreak\hfil$\Large\Diamond$\parfillskip=0pt\finalhyphendemerits=0\par}\vspace{0.1cm}}

\newcommand{\sgn}{\ensuremath{\operatorname{sgn}}}
\newcommand{\sat}{\ensuremath{\operatorname{sat}}}

\newcommand{\half}{\frac{1}{2}}
\newcommand{\shalf}{\mbox{$\frac{1}{2}$}}
\newcommand{\marcom}[1]{\marginpar{\footnotesize #1}}
\newcommand{\der}{\mathrm{D}}
\newcommand{\e}{\mathrm{e}}
\newcommand{\dt}{\mathrm{d}t}

\newcommand{\cA}{\ensuremath{\mathcal{A}}}
\newcommand{\cB}{\ensuremath{\mathcal{B}}}
\newcommand{\cG}{\ensuremath{\mathcal{G}}}
\newcommand{\cK}{\ensuremath{\mathcal{K}}}
\newcommand{\cW}{\ensuremath{\mathcal{W}}}
\newcommand{\cZ}{\ensuremath{\mathcal{Z}}}
\newcommand{\cS}{\ensuremath{\mathcal{S}}}
\newcommand{\cD}{\ensuremath{\mathcal{D}}}
\newcommand{\cP}{\ensuremath{\mathcal{P}}}
\newcommand{\cV}{\ensuremath{\mathcal{V}}}
\newcommand{\cL}{\ensuremath{\mathcal{L}}}
\newcommand{\cN}{\ensuremath{\mathcal{N}}}
\newcommand{\cI}{\ensuremath{\mathcal{I}}}
\newcommand{\cR}{\ensuremath{\mathcal{R}}}
\newcommand{\cM}{\ensuremath{\mathcal{M}}}
\newcommand{\cC}{\ensuremath{\mathcal{C}}}
\newcommand{\cF}{\ensuremath{\mathcal{F}}}
\newcommand{\cH}{\ensuremath{\mathcal{H}}}
\newcommand{\cO}{\ensuremath{\mathcal{O}}}
\newcommand{\cX}{\ensuremath{\mathcal{X}}}
\newcommand{\cY}{\ensuremath{\mathcal{Y}}}
\newcommand{\Ci}{\ensuremath{\mathcal{C}^\infty}}
\newcommand{\ISS}{\textsc{iss}}
\newcommand{\LISS}{\textsc{liss}}
\newcommand{\GAS}{\textsc{gas}}
\newcommand{\GS}{\textsc{gs}}
\newcommand{\LES}{\textsc{les}}
\newcommand{\GUAS}{\textsc{guas}}
\newcommand{\BIBO}{\textsc{bibo}}
\newcommand{\spec}{\ensuremath{\operatorname{spec}}}
\newcommand{\spn}{\ensuremath{\operatorname{span}}}
\renewcommand{\i}{\mathrm{i\,}}

\renewcommand{\implies}{\Rightarrow}

\renewcommand{\theenumi}{$\roman{enumi})$}
\renewcommand{\labelenumi}{\theenumi}

\font\ptmten=zptmcmrm scaled 1200
\newcommand{\w}{\mbox{{\ptmten w}}}
\newcommand{\z}{\mbox{{\ptmten z}}}
\renewcommand{\Re}{\mathbb{R}}

\newcommand{\cl}{\operatorname{cl}}
\newcommand{\intr}{\operatorname{int}}
\newcommand{\rank}{\operatorname{rank}}
\newcommand{\co}{\operatorname{co}}
\newcommand{\aff}{\operatorname{aff}}

\theoremstyle{plain}
\newtheorem{theorem}{Theorem}[chapter]
\newtheorem{claim}[theorem]{Claim}
\newtheorem{corollary}[theorem]{Corollary}
\newtheorem{prop}[theorem]{Proposition}
\newtheorem{fact}[theorem]{Fact}
\newtheorem{lemma}[theorem]{Lemma}

\newtheorem{remark}{Remark}[chapter]

\theoremstyle{definition}
\newtheorem{assume}[theorem]{Assumption}
\newtheorem{defn}[theorem]{Definition}
\newtheorem{problem}[theorem]{Problem}
\newtheorem{exercise}{Exercise}
\newtheorem{example}[theorem]{Example}


\begin{document}
\centerline {\textsf{\textbf{\LARGE{Homework 5}}}}
\centerline {Jaden Wang}
\vspace{.15in}
\begin{problem}[1]
\begin{align*}
	J(u) = \frac{1}{2} cx^2(t_f) + \frac{1}{2} \int_0^{t_f} u^2(t) dt
\end{align*}
Let $ \phi(x_f) = \frac{1}{2}c x_f^2$. The Hamiltonian is
 \begin{align*}
	H =\frac{1}{2} u^2 + p u
\end{align*}
\end{problem}
\begin{problem}[3]
The Hamiltonian is
 \begin{align*}
	H = u +p(- \alpha x+u) = (p+1)u - \alpha p x
\end{align*}
So the switching function is $ p+1$. Since the Hamiltonian is linear in $ u$, we want to use the Pontryagin maximum principle. To minimize $ H(x^* ,u,p^* ,t)$, we want to make  $ (p+1)u$ as negative as possible. Thus
 \begin{align*}
	u^* = \begin{cases}
		m & p^* <-1\\
		0& p^* >-1
	\end{cases}
\end{align*}
I claim that $ p^*  \neq -1$. Suppose $ p^* =1$. Since $ T$ is free, transversality yields  $ H(T) = 0$. But since  $ H$ is time-independent,  $ H$ is constant so  $ H \equiv 0$. However, we see that
 \begin{align*}
	H(0) = 0 \cdot u - \alpha (-1) x(0) = \alpha a >0,
\end{align*}
a contradiciton.

Now solving $ \dot{p} = -H_x = \alpha p$, we get $ p(t) = K e^{ \al\wh{ p}}$. Notice that this is a monotone function and therefore can at most cross the line $ p=-1$ once. That is, the control will switch at most once.
\begin{case}[1]
If $ K \geq 0$,  $ p(t) > -1$ for all  $ t$ so we have no switching and  $ u^* =0$. Then solving $ \dot{x} = - \alpha x $ yields $ x(t) = K_1 e^{- \alpha t}$ and $ x(0) = K_1 = a$. Moreover, we have
\begin{align*}
	x(T) = a e^{- \alpha T} &= c \\
	e^{- \alpha T} &= \frac{c}{a} >0 \\
	- \alpha T &= \ln \frac{c}{a} \\
	T &=  \frac{1}{ \alpha} \ln \frac{a}{c} 
\end{align*}
If $ a>c$, then  $ \frac{a}{c}>1$ and $ T >0$. Thus, $J^* = 0 $. But if $ a\leq c$, then  $ T \leq 0$ which is unphysical. Thus in the case when $ K \geq 0$, $ a>c$, we have $ u^* \equiv 0$.
\end{case}
\begin{case}[2]
If $ K < 0$,  $ p$ might cross the line  $ p=-1$ once from above. Since $ p(0) = K$ starts in the $ p>-1$ regime, we have $ u^*(0) = 0$. But
\begin{align*}
	H(0) = (p(0)+1) u(0) - \alpha p(0) x(0) &= 0 \\
	- \alpha a p(0) &= 0 \\
	K=p(0) &= 0 && \alpha, a>0,
\end{align*}
a contradiction. Thus in this case we also have no switching and $ u^*  \equiv m$. Then we solve
\begin{align*}
	\dot{x} &= - \alpha x+m \\
\frac{dx}{m- \alpha x} &=  dt \\
x(t) &= \frac{m}{ \alpha} - K_2 e^{- \alpha t} \\
x(0) &= \frac{m}{ \alpha} - K_2 = a \\
x(t) &=  \frac{m}{ \alpha} - \left( \frac{m}{ \alpha}-a \right) e^{- \alpha t}  
\end{align*}
We have
\begin{align*}
	x(T) = \frac{m}{ \alpha} - \left(  \frac{m}{ \alpha} - a \right) e^{- \alpha T} &= c \\
	T &=  \frac{1}{ \alpha} \ln \left( \frac{m- \alpha a}{m- \alpha c } \right)  
\end{align*}
If $ a<c$, we see that  $ T>0$, and  $ J^*  = m T = \frac{m}{ \alpha} \ln \left( \frac{m- \alpha a}{ m- \alpha c} \right) $. If $ a \geq c$, we get  $ T \leq 0$ which is unphysical. Thus in the case when  $ K<0$,  $ a<c$, we have  $ u^*  \equiv m$.
\end{case}
\end{problem}
\end{document}
