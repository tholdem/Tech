\documentclass[12pt]{article}
\newcommand{\alert}[1]{{\bf \color{red} [Alert:] #1}}
\newcommand{\todo}[1]{{\bf \color{orange} [TODO:] #1}}
\newcommand{\real}[1][]{\mathbb{R}^{#1}}
\newcommand{\myeqn}[1]{(\ref{#1})}
\newcommand{\myex}[1]{Example \ref{#1}}
\newcommand{\defeq}{\stackrel{\mathrm{def}}{=}}
\newcommand{\parder}[2]{\frac{\partial #1}{\partial #2}}
\newcommand{\Lie}[3][]{\mathsf{L}_{#3}^{#1} #2}
\newcommand{\LieA}[1]{\mathsf{Lie}(#1)}
\newcommand{\lieder}[2]{\mathcal{L}_{#2} #1}
\renewcommand{\t}{^{\mbox{\tiny\sf T}}}
\newcommand{\trans}{^{\mbox{\tiny\sf T}}}
\newcommand{\markup}[1]{\{\textbf{#1}\}}
\newcommand{\msub}[1]{_\mathrm{#1}}
\newcommand{\msup}[1]{^\mathrm{#1}}
\newcommand{\inv}[1]{#1^{-1}}
\newcommand{\pinv}[1]{{#1}^{+}}
\newcommand{\myfracA}[2]{\displaystyle{\frac{#1}{#2}}}
\newcommand{\myfracB}[2]{{#1}/{#2}}
\newcommand{\mydiffA}[1]{\dot{#1}}
\newcommand{\mydiffB}[2]{\myfracA{\mathrm{d}{#1}}{\mathrm{d}{#2}}}
\newcommand{\ball}[2]{\mathcal{B}_{#1}\left(#2\right)}
\newcommand{\acos}[1]{\cos^{-1}\left(#1\right)}
\newcommand{\asin}[1]{\sin^{-1}\left(#1\right)}
\newcommand{\mani}{\mathcal{M}}
\newcommand{\tang}[2]{\mathsf{T}_{#1} #2}
\newcommand{\LieB}[2]{[ #1, #2 ]}
\newcommand{\LieBad}[3][]{\mathsf{ad}_{#2}^{#1} #3}
\newcommand{\ReachVT}{\mathcal{R}^V_T}
\newcommand{\ReachVt}{\mathcal{R}^V_t}
\newcommand{\ReachVTe}{\mathcal{R}^V_{\le T}}
\newcommand{\ReachT}{\mathcal{R}_T}
\newcommand{\Reacht}{\mathcal{R}_t}
\newcommand{\ReachTe}{\mathcal{R}_{\le T}}
\newcommand{\accLA}[1]{\mathsf{Lie}(#1)}
\newcommand{\accD}{\Delta_{\mathcal{F}}}
\newcommand{\accSA}{\mathsf{Lie}(\mathcal{G},f)}
\newcommand{\accDS}{\Delta_{\mathcal{G}}}
\newcommand{\eval}[3]{\mathsf{Ev}^{#2}_{#1}\left( #3 \right)}
\newcommand{\stlc}{\textsc{stlc}}
\newcommand{\clf}{\textsc{clf}}
\newcommand{\jqlf}{\textsc{jqlf}}
\newcommand{\dlas}{\textsc{dlas}}
\newcommand{\Ad}[2]{\mathsf{Ad}_{#1} #2}
\newcommand{\xe}{\ensuremath{x_e}}
\newcommand{\lebg}[1]{\mathcal{L}_{#1}}
\newcommand{\lebgx}[1]{\mathcal{L}_{#1 \mathrm{e}}}
\newcommand{\dom}{D}
\newcommand{\domT}{[t_0,\infty) \times D}
\newcommand{\rarrow}{\rightarrow}
\renewcommand{\d}{\mathrm{d}}
\renewcommand{\Re}{\mathbb{R}}
\newcommand{\C}{\mathrm{C}}

\newcommand{\QED}{{\unskip\nobreak\hfil\penalty50\hskip2em\vadjust{}
		\nobreak\hfil$\Box$\parfillskip=0pt\finalhyphendemerits=0\par}\vspace{0.1cm}}
\newcommand{\eoEx}{{\unskip\nobreak\hfil\penalty50\hskip0em\vadjust{}
		\nobreak\hfil$\Large\Diamond$\parfillskip=0pt\finalhyphendemerits=0\par}\vspace{0.1cm}}

\newcommand{\sgn}{\ensuremath{\operatorname{sgn}}}
\newcommand{\sat}{\ensuremath{\operatorname{sat}}}

\newcommand{\half}{\frac{1}{2}}
\newcommand{\shalf}{\mbox{$\frac{1}{2}$}}
\newcommand{\marcom}[1]{\marginpar{\footnotesize #1}}
\newcommand{\der}{\mathrm{D}}
\newcommand{\e}{\mathrm{e}}
\newcommand{\dt}{\mathrm{d}t}

\newcommand{\cA}{\ensuremath{\mathcal{A}}}
\newcommand{\cB}{\ensuremath{\mathcal{B}}}
\newcommand{\cG}{\ensuremath{\mathcal{G}}}
\newcommand{\cK}{\ensuremath{\mathcal{K}}}
\newcommand{\cW}{\ensuremath{\mathcal{W}}}
\newcommand{\cZ}{\ensuremath{\mathcal{Z}}}
\newcommand{\cS}{\ensuremath{\mathcal{S}}}
\newcommand{\cD}{\ensuremath{\mathcal{D}}}
\newcommand{\cP}{\ensuremath{\mathcal{P}}}
\newcommand{\cV}{\ensuremath{\mathcal{V}}}
\newcommand{\cL}{\ensuremath{\mathcal{L}}}
\newcommand{\cN}{\ensuremath{\mathcal{N}}}
\newcommand{\cI}{\ensuremath{\mathcal{I}}}
\newcommand{\cR}{\ensuremath{\mathcal{R}}}
\newcommand{\cM}{\ensuremath{\mathcal{M}}}
\newcommand{\cC}{\ensuremath{\mathcal{C}}}
\newcommand{\cF}{\ensuremath{\mathcal{F}}}
\newcommand{\cH}{\ensuremath{\mathcal{H}}}
\newcommand{\cO}{\ensuremath{\mathcal{O}}}
\newcommand{\cX}{\ensuremath{\mathcal{X}}}
\newcommand{\cY}{\ensuremath{\mathcal{Y}}}
\newcommand{\Ci}{\ensuremath{\mathcal{C}^\infty}}
\newcommand{\ISS}{\textsc{iss}}
\newcommand{\LISS}{\textsc{liss}}
\newcommand{\GAS}{\textsc{gas}}
\newcommand{\GS}{\textsc{gs}}
\newcommand{\LES}{\textsc{les}}
\newcommand{\GUAS}{\textsc{guas}}
\newcommand{\BIBO}{\textsc{bibo}}
\newcommand{\spec}{\ensuremath{\operatorname{spec}}}
\newcommand{\spn}{\ensuremath{\operatorname{span}}}
\renewcommand{\i}{\mathrm{i\,}}

\renewcommand{\implies}{\Rightarrow}

\renewcommand{\theenumi}{$\roman{enumi})$}
\renewcommand{\labelenumi}{\theenumi}

\font\ptmten=zptmcmrm scaled 1200
\newcommand{\w}{\mbox{{\ptmten w}}}
\newcommand{\z}{\mbox{{\ptmten z}}}
\renewcommand{\Re}{\mathbb{R}}

\newcommand{\cl}{\operatorname{cl}}
\newcommand{\intr}{\operatorname{int}}
\newcommand{\rank}{\operatorname{rank}}
\newcommand{\co}{\operatorname{co}}
\newcommand{\aff}{\operatorname{aff}}

\theoremstyle{plain}
\newtheorem{theorem}{Theorem}[chapter]
\newtheorem{claim}[theorem]{Claim}
\newtheorem{corollary}[theorem]{Corollary}
\newtheorem{prop}[theorem]{Proposition}
\newtheorem{fact}[theorem]{Fact}
\newtheorem{lemma}[theorem]{Lemma}

\newtheorem{remark}{Remark}[chapter]

\theoremstyle{definition}
\newtheorem{assume}[theorem]{Assumption}
\newtheorem{defn}[theorem]{Definition}
\newtheorem{problem}[theorem]{Problem}
\newtheorem{exercise}{Exercise}
\newtheorem{example}[theorem]{Example}

%include Matlab code
\usepackage{sectsty}
\usepackage{listings}
\usepackage{color}
\usepackage{alltt}
\definecolor{dkgreen}{rgb}{0,0.6,0}
\definecolor{gray}{rgb}{0.5,0.5,0.5}
\definecolor{mauve}{rgb}{0.58,0,0.82}

\lstset{frame=tb,
  language=Matlab,
  aboveskip=3mm,
  belowskip=3mm,
  showstringspaces=false,
  columns=flexible,
  basicstyle={\small\ttfamily},
  numbers=none,
  numberstyle=\tiny\color{gray},
  keywordstyle=\color{blue},
  commentstyle=\color{dkgreen},
  stringstyle=\color{mauve},
  breaklines=true,
  breakatwhitespace=true,
  tabsize=3
}
\begin{document}
\centerline {\textsf{\textbf{\LARGE{Homework 2}}}}
\centerline {Jaden Wang}
\vspace{.15in}

\begin{problem}[1]
Since $ Dg(x,y) = \begin{bmatrix} 4x^3-3x^2\\2y \end{bmatrix}$, it equals zero exactly when $ y=0$ and  $ x=0$ or  $ \frac{4}{3}$. Strong normality is satisfied for all other points. This this case, the Lagrangian is
\begin{align*}
	\mathscr{L}((x,y),1,\lambda) = x + \lambda(y^2+x^{4} - x^3) .
\end{align*}
The first-order conditions require that
\begin{align*}
	\mathscr{L}_x &= 1+4\lambda x^3-3\lambda x^2 =0\\
	\mathscr{L}_y &= 2\lambda y =0\\
	\mathscr{L}_\lambda &= y^2+x^{4}-x^3=0 
\end{align*}
which forces $ y=0$,  $ x=1$, and thus  $ \lambda = -1$. Thus $ (1,0) \in \inte \rr^2$ is a candidate local minimizer.  Since $ g'(1,0) = \begin{bmatrix} 1\\0 \end{bmatrix} $, the null space is $ N = t \begin{bmatrix} 0\\1 \end{bmatrix} $. Since 
\begin{align*}
	\mathscr{L}_{(x,y) (x,y)} = \begin{pmatrix} -12x^2+6x & 0\\0& -2 \end{pmatrix} 
\end{align*}
we see that for the basis vector of  $ N$,
\begin{align*}
	\begin{pmatrix} 0&1 \end{pmatrix} \begin{pmatrix} -6&0\\0&-2 \end{pmatrix} \begin{pmatrix} 0\\1 \end{pmatrix} = -2 <0 
\end{align*}
Thus $ (1,0)$ is not a local minimum. The only remaining possibilities are the abnormal cases $ (0,0)$ or $ (\frac{4}{3},0)$. The Lagrangian is
\begin{align*}
	\mathscr{L}((x,y),0,\lambda) &= \lambda(y^2+x^{4}-x^3) \\
	\mathscr{L}_x &= 4 \lambda x^3-3 \lambda x^2 =0 \\
	\mathscr{L}_y &= 0 \\
	\mathscr{L}_\lambda &= 0 
\end{align*}
which yields $ x=0$,  $ y=0$, and  $ \lambda$ can be anything nonzero. I claim that $ (0,0)$ is the global minimum. Suppose $ x<0$, then $ x^{4} >0$ and $ -x^3 >0$, thus $ y^2 + x^{4} - x^3 >0$, not satisfying the constraint. Thus $ f(x)$ is lower bounded by  $ x=0$, which is achieved by  $ (0,0)$.
\end{problem}

\begin{problem}[2]
For $ \mu=1$:
\begin{align*}
	\mathscr{L}(x,1,\lambda,\nu) = x_1^2 - x_2 + \lambda_1(x_1^2+x_2^2-1) &+ \lambda_2(x_2-2)+ \nu(x_1^3+x_2-1) \\
	\mathscr{L}_x= \begin{pmatrix} x_1(2+2 \lambda_1 + 3 \nu x_1)\\ 2\lambda_1 x_2+ \lambda_2 + \nu-1 \end{pmatrix} &= \begin{pmatrix} 0\\0 \end{pmatrix}  \\
	\lambda_1 (x_1^2+x_2^2 -1 ) &= 0 \\
	\lambda_2 (x_2 -2) &= 0 \\
	\lambda_1 , \lambda_2  &\geq 0\\
	x_1^3+x_2-1 &= 0 \\
	x_1^2+x_2^2 -1 &\leq 0\\
	x_2 -2	& \leq 0
\end{align*}
Since if $ x_2^*=2$, $ x_1^2+(x_2^*)^2-1 = x_1^2 + 3 >0$ not primal feasible, so it must be that $ \lambda_2^* = 0$. The solutions that satisfy the system of equations and inequality constraints above are $ x_1=0$, $ x_2 =1$, $\lambda_2=0$, and $ \nu=1-2 \lambda_1$; $ x_1= 0.544, x_2= 0.839, \lambda_1 = 4.92, \lambda_2 = 0,$ and $ \nu= -7.26$.

We see that only $ \lambda_1$ is active, so the Jacobian to test normality is
\begin{align*}
	\nabla \begin{pmatrix} x_1^2+x_2^2\\x_1^3+x_2 \end{pmatrix}  &= \begin{pmatrix} 2x_1&2x_2\\3x_1^2&1 \end{pmatrix}
\end{align*}
Since $ \ell=2$, the only time this matrix is not rank 2 is when $ x_1 =0$. This means that our solution $ (0,1)$ is abnormal, but $ (0.544,0.839)$ is strongly normal.

\begin{align*}
	\mathscr{L}_{x x} &= \begin{pmatrix} 6\nu x_1+2\lambda_1+2&0\\0&2\lambda_1 \end{pmatrix} = \begin{pmatrix} -11.86&0\\0&9.84 \end{pmatrix}  \\
\end{align*}
is a local saddle point.

When $ \mu=0$,
\begin{align*}
	\mathscr{L}(x,0,\lambda,\nu) = \lambda_1(x_1^2+x_2^2-1) &+ \lambda_2(x_2-2)+ \nu(x_1^3+x_2-1) \\
	\mathscr{L}_x= \begin{pmatrix} x_1(2 \lambda_1 + 3 \nu x_1)\\ 2\lambda_1 x_2+ \lambda_2 + \nu \end{pmatrix} &= \begin{pmatrix} 0\\0 \end{pmatrix}  \\
	\lambda_1 (x_1^2+x_2^2 -1 ) &= 0 \\
	\lambda_2 (x_2 -2) &= 0 \\
	\lambda_1 , \lambda_2  &\geq 0\\
	x_1^3+x_2-1 &= 0 \\
	x_1^2+x_2^2 -1 &\leq 0\\
	x_2 -2	& \leq 0
\end{align*}
and we get the same solution as the case when $ \mu=1$.

Since this is the only candidate, $ (0,1)$ is the minimizer we seek, the minimum is $ f(0,1)=-1$.
\end{problem}
\begin{problem}[3]
Since maximizing $ x_1$ is the same as minimizing $ -x_1$, we have $ f(x_1,x_2) = -x_1$  For all other points, assume $ \mu=1$:
\begin{align*}
	\mathscr{L}(x_1,x_2,1,\lambda_1,\lambda_2) &= -x_1 + \lambda_1(x_2-(1-x_1)^3)-\lambda_2 x_2\\
	\mathscr{L}_x = \begin{pmatrix} -1+3\lambda_1(1-x_1)^2\\ \lambda_1-\lambda_2 \end{pmatrix} &=0 \\
	\lambda_1(x_2-(1-x_1)^3) &= 0 \\
	-\lambda_2 x_2 &= 0 \\
	\lambda_1,\lambda_2 &\geq 0\\
	x_2-(1-x_1)^3 &\leq 0\\
	-x_2 &\leq 0 
\end{align*}
We see that $ \lambda_2 \neq 0$ since otherwise $ \lambda_1 =0$ and we have $ -1 =0$. It must be that  $ x_2 = 0$ and thus $ -(1-x_1)^3=0$ which forces $ x_1 = 1$, but this violates the first equation, so there is no solution.

Now for $ \mu=0$,
\begin{align*}
	\mathscr{L}(x_1,x_2,0,\lambda_1,\lambda_2) &= \lambda_1(x_2-(1-x_1)^3)-\lambda_2 x_2\\
	\mathscr{L}_x = \begin{pmatrix} 3\lambda_1(1-x_1)^2\\ \lambda_1-\lambda_2 \end{pmatrix} &=0 \\
	\lambda_1(x_2-(1-x_1)^3) &= 0 \\
	-\lambda_2 x_2 &= 0 \\
	\lambda_1,\lambda_2 &\geq 0\\
	x_2-(1-x_1)^3 &\leq 0\\
	-x_2 &\leq 0 
\end{align*}
The solution is $ (1,0)$ with both constraints active. We have $ g(x_1,x_2) = \begin{pmatrix} x_2 - (1-x_1)^3\\-x_2 \end{pmatrix} $.
\begin{align*}
	g'(x_1,x_2) = \begin{pmatrix} 3(1-x_1)^2&1\\0&-1 \end{pmatrix} 
\end{align*}
Thus the point is abnormal iff $ x_1=1$. So $ (1,0)$ is abnormal! Thus the maximum of  $ x_1$ is achieved at 1.
~\begin{figure}[H]
	\centering
	\includegraphics[width=0.7\textwidth]{./figures/gradient.png}
	\caption{We see that at $ (1,0)$ the gradients of two active constraints are parallel.}
\end{figure}
\end{problem}
\begin{problem}[4]
For $ \mu=1$, we have
\begin{align*}
	\mathscr{L}(x,1,\lambda) = -5x_1-x_2 +\lambda_1(-x_1) + \lambda_2 & (3x_1+x_2-11)+ \lambda_3(x_1-2x_2-2) \\
	\mathscr{L}_x = \begin{pmatrix}  -5-\lambda_1+3\lambda_2+\lambda_3\\ -1+\lambda_2-2\lambda_3 \end{pmatrix}  &= \begin{pmatrix} 0\\0 \end{pmatrix}  \\
	\lambda_1(-x_1) &= 0 \\
	\lambda_2(3x_1+x_2-11) &= 0 \\
	\lambda_3(x_1-2x_2-2) &= 0 \\
	\lambda_1,\lambda_2,\lambda_3 &\geq 0\\
	-x_1 &\leq 0\\
	3x_1+x_2-11 &\leq 0\\
	x_1-2x_2-2 & \leq 0
\end{align*}
The only solution is $ x_1= \frac{24}{7}$, $ x_2 = \frac{5}{7}$, $ \lambda_1 = 0$, $ \lambda_2=\frac{11}{7}$, and $ \lambda_3 = \frac{2}{7}$.

Since $ \lambda_1=0$, the Jacobian to test normality is
We compute
\begin{align*}
	\nabla \begin{pmatrix} 3x_1+x_2-11\\x_1-2x_2-2 \end{pmatrix}  &= \begin{pmatrix} 3&1\\1&-2 \end{pmatrix}  
\end{align*}
which has rank $ 2$, same as the dimension of its image. Thus the solution is strongly normal.

\begin{align*}
	\mathscr{L}_{x x} = \begin{pmatrix} 0&0\\0&0 \end{pmatrix} 
\end{align*}

For $ \mu=0$, we have
\begin{align*}
	\mathscr{L}(x,0,\lambda) = \lambda_1(-x_1) + \lambda_2 & (3x_1+x_2-11)+ \lambda_3(x_1-2x_2-2) \\
	\mathscr{L}_x = \begin{pmatrix}  -\lambda_1+3\lambda_2+\lambda_3\\ \lambda_2-2\lambda_3 \end{pmatrix}  &= \begin{pmatrix} 0\\0 \end{pmatrix}  \\
	\lambda_1(-x_1) &= 0 \\
	\lambda_2(3x_1+x_2-11) &= 0 \\
	\lambda_3(x_1-2x_2-2) &= 0 \\
	\lambda_1,\lambda_2,\lambda_3 &\geq 0\\
	-x_1 &\leq 0\\
	3x_1+x_2-11 &\leq 0\\
	x_1-2x_2-2 & \leq 0
\end{align*}
which has no solution with one of multipliers nonzero.
~\begin{figure}[H]
	\centering
	\includegraphics[width=0.6\textwidth]{./figures/feasible_region}
	\caption{The feasible region is the triangle formed by the three points. We see that the minimizer is at a vertex of the triangle.}
	\label{fig:feasible_region}
\end{figure}
The minimum is therefore $ -5 \cdot \frac{24}{7}-\frac{5}{7} = -\frac{125}{ 7}$.

\begin{lstlisting}
%%%%%%%%%%%%%%%%%%%%%%%%%%%%%%%%%%%%%%%%%%%%%%%%%%%%%%%%%%%%%%%%%%%%%%%%%%
%Problem 4
function [c,ceq] = constraints4(x)
x1=x(1);
x2=x(2);
c=[-x1;3*x1+x2-11;x1-2*x2-2];
ceq = [];
%%%%%%%%%%%%%%%%%%%%%%%%%%%%%%%%%%%%%%%%%%%%%%%%%%%%%%%%%%%%%%%%%%%%%%%%%%
f=@(x) -5*x(1)-x(2);
A = [];
b = [];
Aeq = [];
beq = [];
lb = [];
ub = [];
nonlcon=@constraints4;
x0=[1;0];
x=fmincon(f,x0,A,b,Aeq,beq,lb,ub,nonlcon)
fx = f(x)
%%%%%%%%%%%%%%%%%%%%%%%%%%%%%%%%%%%%%%%%%%%%%%%%%%%%%%%%%%%%%%%%%%%%%%%%%%
\end{lstlisting}
The output is a match!
\end{problem}

\begin{problem}[5]
\begin{enumerate}[label=(\arabic*)]
	\item \begin{align*}
	\min\quad & \begin{pmatrix} 400&360&550&470&600&500 \end{pmatrix} \begin{pmatrix} x_1&x_2&x_3&x_4&x_5&x_6 \end{pmatrix}^{T}  \\
	\text{subject to } \quad & \begin{pmatrix} 1&1&0&0&0&0\\0&0&1&1&0&0\\0&0&0&0&1&1\\1&0&1&0&1&0\\0&1&0&1&0&1 \end{pmatrix} \begin{pmatrix} x_1&x_2&x_3&x_4&x_5&x_6 \end{pmatrix}^{T}  = \begin{pmatrix} 200\\360\\340\\500\\400 \end{pmatrix} 
\end{align*}
\item 
\begin{lstlisting}
%%%%%%%%%%%%%%%%%%%%%%%%%%%%%%%%%%%%%%%%%%%%%%%%%%%%%%%%%%%%%%%%%%%%%%%%%%
%Problem 5
f=[400;360;550;470;600;500];
A = [];
b = [];
Aeq = [1 1 0 0 0 0;0 0 1 1 0 0; 0 0 0 0 1 1; 1 0 1 0 1 0; 0 1 0 1 0 1];
beq = [200;360;340;500;400];
lb = [0;0;0;0;0;0];
ub = [500;400;500;400;500;400];
x=linprog(f,A,b,Aeq,beq,lb,ub);
fx=f'*x
%%%%%%%%%%%%%%%%%%%%%%%%%%%%%%%%%%%%%%%%%%%%%%%%%%%%%%%%%%%%%%%%%%%%%%%%%%
\end{lstlisting}

The minimum cost schedule is $ \begin{pmatrix} 200&0&300&60&0&340 \end{pmatrix}^T $ and the minimum cost is $ 443200$ dollars.

\end{enumerate}
\end{problem}

\begin{problem}[6]
\begin{enumerate}[label=(\arabic*)]
	\item 
\begin{align*}
	\mathscr{L}= V \sin \gamma + \lambda_1 (T(V)\cos( \alpha+ \epsilon)- D(V, \alpha) - mg \sin \gamma) + \lambda_2 (T(V) \sin ( \alpha + \epsilon)+ L(V, \alpha)- mg \cos \gamma)
\end{align*}
So the first-order necessary conditions are
\begin{align*}
	\mathscr{L}_V &= \sin \gamma + \lambda_1 (T'(V) \cos ( \alpha+ \epsilon)-D_V(V, \alpha)) + \lambda_2(T'(V) \sin ( \alpha + \epsilon)+L_V(V, \alpha)) =0 \\
	&= \sin \gamma + (\lambda_1\cos ( \alpha + 0.0349)+ \lambda_2 \sin ( \alpha+ 0.0349)) ( -0.04312+2 \cdot 0.008392V)\\ 
	& \quad -2\lambda_1V(0.07351-0.08617 \alpha+1.996 \alpha^2) +  2\lambda_2 V (0.1667+6.231 \alpha - 21.65 [ \max (0, \alpha - 0.2094)]^2)\\
	\mathscr{L}_{ \alpha} &= -\lambda_1 (T(V) \sin ( \alpha+ \epsilon) - V^2(-0.08617+2 \cdot 1.996 \alpha +6.231-2 \cdot  21.65[(\alpha - 0.2094) \text{ or } 0]) )\\
	\mathscr{L}_{ \gamma} &= (V-mg) \cos \gamma + mg \sin \gamma \\
	0&=T(V) \cos( \alpha + \epsilon)-D(V, \alpha) - mg \sin \gamma \\
0&=T(V) \sin ( \alpha+ \epsilon) + L(V, \alpha)-mg \cos \gamma 
\end{align*}

\item
\begin{lstlisting}
%%%%%%%%%%%%%%%%%%%%%%%%%%%%%%%%%%%%%%%%%%%%%%%%%%%%%%%%%%%%%%%%%%%%%%%%%%
%Problem 6b
function [c,ceq] = constraints6(x)
T=@(V) 0.2476-0.04312*V+0.008392*V^2;
D=@(V,a) V^2*(0.07351-0.08617*a+1.996*a^2);
L=@(V,a) V^2*(0.1667+6.231*a-21.65*(max([0 a-0.2094])^2));
e=0.0349;
w=180000;
c=[];
V=x(1);
a=x(2);
gamma=x(3);
ceq = [T(V)*cos(a+e)-D(V,a)-w*sin(gamma);T(V)*sin(a+e)+L(V,a)-w*cos(gamma)];
%%%%%%%%%%%%%%%%%%%%%%%%%%%%%%%%%%%%%%%%%%%%%%%%%%%%%%%%%%%%%%%%%%%%%%%%%%
g=32.17;
W=180000;
l=2*W/(0.002203*1560*g);
f=@(x) -x(1)*sin(x(3)); %maximize is the same as minimize the negative
A = [];
b = [];
Aeq = [];
beq = [];
lb = [1,0.1,0.1];
ub = [2,0.2,0.2];
nonlcon=@constraints6;
x0=[342/sqrt(g*l);6.39/180*pi;6.31/180*pi];
[x,fval,exitflag,output,lambda,grad,hessian]=fmincon(f,x0,A,b,Aeq,beq,lb,ub,nonlcon)
fval
%%%%%%%%%%%%%%%%%%%%%%%%%%%%%%%%%%%%%%%%%%%%%%%%%%%%%%%%%%%%%%%%%%%%%%%%%%
\end{lstlisting}
I cannot get the correct values after hours of debugging.

\item I do not have the lambdas from b so I cannot check if it is zero.

\end{enumerate}
\end{problem}
\end{document}
