\documentclass[12pt]{article}
\newcommand{\alert}[1]{{\bf \color{red} [Alert:] #1}}
\newcommand{\todo}[1]{{\bf \color{orange} [TODO:] #1}}
\newcommand{\real}[1][]{\mathbb{R}^{#1}}
\newcommand{\myeqn}[1]{(\ref{#1})}
\newcommand{\myex}[1]{Example \ref{#1}}
\newcommand{\defeq}{\stackrel{\mathrm{def}}{=}}
\newcommand{\parder}[2]{\frac{\partial #1}{\partial #2}}
\newcommand{\Lie}[3][]{\mathsf{L}_{#3}^{#1} #2}
\newcommand{\LieA}[1]{\mathsf{Lie}(#1)}
\newcommand{\lieder}[2]{\mathcal{L}_{#2} #1}
\renewcommand{\t}{^{\mbox{\tiny\sf T}}}
\newcommand{\trans}{^{\mbox{\tiny\sf T}}}
\newcommand{\markup}[1]{\{\textbf{#1}\}}
\newcommand{\msub}[1]{_\mathrm{#1}}
\newcommand{\msup}[1]{^\mathrm{#1}}
\newcommand{\inv}[1]{#1^{-1}}
\newcommand{\pinv}[1]{{#1}^{+}}
\newcommand{\myfracA}[2]{\displaystyle{\frac{#1}{#2}}}
\newcommand{\myfracB}[2]{{#1}/{#2}}
\newcommand{\mydiffA}[1]{\dot{#1}}
\newcommand{\mydiffB}[2]{\myfracA{\mathrm{d}{#1}}{\mathrm{d}{#2}}}
\newcommand{\ball}[2]{\mathcal{B}_{#1}\left(#2\right)}
\newcommand{\acos}[1]{\cos^{-1}\left(#1\right)}
\newcommand{\asin}[1]{\sin^{-1}\left(#1\right)}
\newcommand{\mani}{\mathcal{M}}
\newcommand{\tang}[2]{\mathsf{T}_{#1} #2}
\newcommand{\LieB}[2]{[ #1, #2 ]}
\newcommand{\LieBad}[3][]{\mathsf{ad}_{#2}^{#1} #3}
\newcommand{\ReachVT}{\mathcal{R}^V_T}
\newcommand{\ReachVt}{\mathcal{R}^V_t}
\newcommand{\ReachVTe}{\mathcal{R}^V_{\le T}}
\newcommand{\ReachT}{\mathcal{R}_T}
\newcommand{\Reacht}{\mathcal{R}_t}
\newcommand{\ReachTe}{\mathcal{R}_{\le T}}
\newcommand{\accLA}[1]{\mathsf{Lie}(#1)}
\newcommand{\accD}{\Delta_{\mathcal{F}}}
\newcommand{\accSA}{\mathsf{Lie}(\mathcal{G},f)}
\newcommand{\accDS}{\Delta_{\mathcal{G}}}
\newcommand{\eval}[3]{\mathsf{Ev}^{#2}_{#1}\left( #3 \right)}
\newcommand{\stlc}{\textsc{stlc}}
\newcommand{\clf}{\textsc{clf}}
\newcommand{\jqlf}{\textsc{jqlf}}
\newcommand{\dlas}{\textsc{dlas}}
\newcommand{\Ad}[2]{\mathsf{Ad}_{#1} #2}
\newcommand{\xe}{\ensuremath{x_e}}
\newcommand{\lebg}[1]{\mathcal{L}_{#1}}
\newcommand{\lebgx}[1]{\mathcal{L}_{#1 \mathrm{e}}}
\newcommand{\dom}{D}
\newcommand{\domT}{[t_0,\infty) \times D}
\newcommand{\rarrow}{\rightarrow}
\renewcommand{\d}{\mathrm{d}}
\renewcommand{\Re}{\mathbb{R}}
\newcommand{\C}{\mathrm{C}}

\newcommand{\QED}{{\unskip\nobreak\hfil\penalty50\hskip2em\vadjust{}
		\nobreak\hfil$\Box$\parfillskip=0pt\finalhyphendemerits=0\par}\vspace{0.1cm}}
\newcommand{\eoEx}{{\unskip\nobreak\hfil\penalty50\hskip0em\vadjust{}
		\nobreak\hfil$\Large\Diamond$\parfillskip=0pt\finalhyphendemerits=0\par}\vspace{0.1cm}}

\newcommand{\sgn}{\ensuremath{\operatorname{sgn}}}
\newcommand{\sat}{\ensuremath{\operatorname{sat}}}

\newcommand{\half}{\frac{1}{2}}
\newcommand{\shalf}{\mbox{$\frac{1}{2}$}}
\newcommand{\marcom}[1]{\marginpar{\footnotesize #1}}
\newcommand{\der}{\mathrm{D}}
\newcommand{\e}{\mathrm{e}}
\newcommand{\dt}{\mathrm{d}t}

\newcommand{\cA}{\ensuremath{\mathcal{A}}}
\newcommand{\cB}{\ensuremath{\mathcal{B}}}
\newcommand{\cG}{\ensuremath{\mathcal{G}}}
\newcommand{\cK}{\ensuremath{\mathcal{K}}}
\newcommand{\cW}{\ensuremath{\mathcal{W}}}
\newcommand{\cZ}{\ensuremath{\mathcal{Z}}}
\newcommand{\cS}{\ensuremath{\mathcal{S}}}
\newcommand{\cD}{\ensuremath{\mathcal{D}}}
\newcommand{\cP}{\ensuremath{\mathcal{P}}}
\newcommand{\cV}{\ensuremath{\mathcal{V}}}
\newcommand{\cL}{\ensuremath{\mathcal{L}}}
\newcommand{\cN}{\ensuremath{\mathcal{N}}}
\newcommand{\cI}{\ensuremath{\mathcal{I}}}
\newcommand{\cR}{\ensuremath{\mathcal{R}}}
\newcommand{\cM}{\ensuremath{\mathcal{M}}}
\newcommand{\cC}{\ensuremath{\mathcal{C}}}
\newcommand{\cF}{\ensuremath{\mathcal{F}}}
\newcommand{\cH}{\ensuremath{\mathcal{H}}}
\newcommand{\cO}{\ensuremath{\mathcal{O}}}
\newcommand{\cX}{\ensuremath{\mathcal{X}}}
\newcommand{\cY}{\ensuremath{\mathcal{Y}}}
\newcommand{\Ci}{\ensuremath{\mathcal{C}^\infty}}
\newcommand{\ISS}{\textsc{iss}}
\newcommand{\LISS}{\textsc{liss}}
\newcommand{\GAS}{\textsc{gas}}
\newcommand{\GS}{\textsc{gs}}
\newcommand{\LES}{\textsc{les}}
\newcommand{\GUAS}{\textsc{guas}}
\newcommand{\BIBO}{\textsc{bibo}}
\newcommand{\spec}{\ensuremath{\operatorname{spec}}}
\newcommand{\spn}{\ensuremath{\operatorname{span}}}
\renewcommand{\i}{\mathrm{i\,}}

\renewcommand{\implies}{\Rightarrow}

\renewcommand{\theenumi}{$\roman{enumi})$}
\renewcommand{\labelenumi}{\theenumi}

\font\ptmten=zptmcmrm scaled 1200
\newcommand{\w}{\mbox{{\ptmten w}}}
\newcommand{\z}{\mbox{{\ptmten z}}}
\renewcommand{\Re}{\mathbb{R}}

\newcommand{\cl}{\operatorname{cl}}
\newcommand{\intr}{\operatorname{int}}
\newcommand{\rank}{\operatorname{rank}}
\newcommand{\co}{\operatorname{co}}
\newcommand{\aff}{\operatorname{aff}}

\theoremstyle{plain}
\newtheorem{theorem}{Theorem}[chapter]
\newtheorem{claim}[theorem]{Claim}
\newtheorem{corollary}[theorem]{Corollary}
\newtheorem{prop}[theorem]{Proposition}
\newtheorem{fact}[theorem]{Fact}
\newtheorem{lemma}[theorem]{Lemma}

\newtheorem{remark}{Remark}[chapter]

\theoremstyle{definition}
\newtheorem{assume}[theorem]{Assumption}
\newtheorem{defn}[theorem]{Definition}
\newtheorem{problem}[theorem]{Problem}
\newtheorem{exercise}{Exercise}
\newtheorem{example}[theorem]{Example}


\begin{document}
\centerline {\textsf{\textbf{\LARGE{Homework 2}}}}
\centerline {Jaden Wang}
\vspace{.15in}
\begin{problem}[2.1(i)]
The initial curve can be parameterized: $x_0=s, y_0=s^2, z_0=u(x_0,y_0)= s^2$. Let us check it is not characteristic. The characteristic equation is $ a \sigma_0 + b \sigma_1 = 0$, where $ ( \sigma_0, \sigma_1)$ is the outward normal to the characteristic curve which has tangent $ (a,b)$. Thus we don't want the initial curve to have tangent parallel to  $ (a,b)$. And $ (a,b)$ for initial curve is parameterized by  $ (s,s ^2)$. We check
 \begin{align*}
	 \det \begin{pmatrix} s& s ^2\\1 & 2s \end{pmatrix} = 2s ^2-s ^2 = s ^2 \neq 0 \iff s \neq 0,
\end{align*}
which is always true or we wouldn't have an initial curve. We can safely proceed to solve
\begin{align*}
	\begin{cases}
		\frac{d x}{d t} = x, x(0) = s&\implies x= s e^{t}\\
		\frac{d y}{d t} =y, y(0) = s ^2& \implies y= s ^2 e^{t}\\
		\frac{d z}{d t} = c=z+1, z(0) = s ^2& \implies z= (s ^2+1)e^{t}\\
	\end{cases}
\end{align*}
We have $ s=\frac{y}{x}$ and $ e^{t} = \frac{x^2}{ y}$. So the solution is $u(x,y)= \left( \frac{y^2}{x^2}+1 \right) \frac{x^2}{y}-1 = y+\frac{x^2}{y}-1$.
\end{problem}
\begin{problem}[2]
We shall use the method of integrating factor. Multiplying the PDE by $ e^{ct}$ yields
\begin{align*}
	e^{ct} u_t + e^{ct} b \cdot D_x u +c e^{ct} u&= 0 \\
	\left( e^{ct}u \right)_t + e^{ct} b \cdot D_x u &= 0 && \text{ product rule}  \\
	\left( e^{ct}u \right)_t + b \cdot D_x \left( e^{ct}u \right)  &= 0 && e^{ct}\text{ can be treated as constant}  \\
	e^{ct}u&= g(x-tb) && \text{linear transport solution}\\
	u&= g(x-tb) e^{-ct} && e^{ct} \neq 0 
\end{align*}

\end{problem}

\begin{problem}[3]
This is Burger's equation. The initial curve is parameterized as $t_0=0, x_0=s, z_0 = \frac{1}{1+s ^2}$. We see that
\begin{align*}
	\det \begin{pmatrix}   1&u\\0&1 \end{pmatrix} \neq 0 
\end{align*}
Then ODEs are
\begin{align*}
	\frac{dt}{ d \tau} &= 1, t(0)=0 \implies t = \tau\\
	\frac{dx}{ d \tau} &= u, x(0) = s \implies x(t)= s+ \frac{1}{1+s^2}t =x_0+tu(x_0,0)\\
	\frac{dz}{ d \tau} &= 0, z(0) = \frac{1}{1+s ^2} \implies z(t) = \frac{1}{1+s ^2} = u(x_0,0)
\end{align*}
Therefore, $ u(x,t)$ is completely determined by  $ x_0$. The solution blows up when $ w:=u_x$ tends to infinity. We see
 \begin{align*}
	0=\left( u_t+uu_x \right) _x &= u_{xt} + u_x ^2 + uu_{xx} \\
	&= w_t+w^2+uw_x 
\end{align*}
Moreover, on the characteristic line $ x(t)= x_0+tu(x_0,0)$, we have
\begin{align*}
	\frac{d}{dt}(w) &= w_t + w_x \frac{d x}{d t}  \\
	&= w_t + w_x u(x_0,0)
\end{align*}
Combining the two equations, we have the following ODE on the characteristic line:
\begin{align*}
	\begin{cases}
		\dot{w} &= -w^2\\
		w(x_0,0) &= \left( \frac{1}{1+x_0^2} \right)' = -\frac{2x_0}{ (1+x_0^2)^2}
	\end{cases}
\end{align*}
Solving this yields
\begin{align*}
	w^{-1}&= t - \frac{(1+x_0^2)^2}{ 2x_0}
\end{align*}
so $ w$ blows up when  $ t= \frac{(1+x_0^2)^2}{  2x_0}$. To find the first time it blows up, \emph{i.e.} minimum time, we set
\begin{align*}
	t' = \frac{ 8x_0^2(1+x_0^2)-2(1+x_0^2)^2}{ 4x_0^{2}} &=0\\
	2(3x_0^{4}+2x_0^2-1) &= 0 \\
	x_0 &= \pm \frac{1}{\sqrt{3} } \\
	t &= \pm \frac{8}{ 3 \sqrt{3}}
\end{align*}
We can easily see from the graph that minimum is achieved at $ t= \frac{8}{ 3 \sqrt{3}}$.
\end{problem}
\begin{problem}[4]
	First, since $ (x,y)$ is the outward normal vector of the unit circle, the condition  $ \begin{pmatrix} a&b \end{pmatrix} \begin{pmatrix} x\\y \end{pmatrix} >0$ means that $ (a,b)$ also points outward from the tangent line at $ (x,y)$. This means that for a small $ t$, $ (x-ta,y-tb) \in \inte \Omega$.

Since $ \Omega$ is compact, $ u$ achieves maximum and minimum. Suppose the minimum is achieved at $ \inte \Omega$. Then necessary condition says $ \begin{pmatrix} u_x\\u_y \end{pmatrix}  = \begin{pmatrix} 0\\0 \end{pmatrix} $, which yields $ u(x,y) = au_x+bu_y = 0$. Thus by definition of minimum, $ u(x,y) \geq 0$ for all $ x,y$. Likewise, if maximum is achieved at the interior, then $ u(x,y) \leq 0$. We have four cases:
 \begin{case}[1] If both maximum and minimum are achieved in the interior, we have $ u \equiv 0$ immediately.
\end{case}
\begin{case}[2] If only minimum is in the interior, then $ u(x,y) \geq 0$. Let  $ (x^* ,y^* ) \in \partial \Omega$ denotes the maximizer. If $ u(x^* ,y^* ) = 0$, then $ u \equiv 0$. Suppose  $ u(x^* ,y^* )>0$, then by Taylor's theorem, since $ u \in C^{1}$,
\begin{align*}
	u(x^* -ta,y^* -tb) &= u(x^* ,y^* )-t \begin{pmatrix} a(x^* ,y^* ) & b(x^* ,y^* )\end{pmatrix} \begin{pmatrix} u_x(x^* ,y^* ) \\u_y(x^* ,y^* )\end{pmatrix} + r_1(t) \\
	&= u(x^* ,y^* ) + t \underbrace{ u(x^* ,y^* )}_{>0 }  + r_1(t)
\end{align*}
where $ \lim_{ t \to 0}  \frac{r_1(t)}{t } = 0$. Thus we can find a small enough $ t$  s.t.\ $ tu(x^* ,y^* )+r_1(t) >0$, contradicting that $ u(x^* ,y^* )$ is the maximum.
\end{case}

\begin{case}[3]
Similarly, if only maximum is in the interior, then $ u(x,y) \leq 0$. Let  $ (x_* ,y_* ) \in \partial \Omega$ denotes the minimizer. If $ u(x_* ,y_* ) = 0$, then $ u \equiv 0$. Suppose  $ u(x_* ,y_* )<0$, then by Taylor's theorem, since $ u \in C^{1}$,
\begin{align*}
	u(x_* -ta,y_* -tb) &= u(x_* ,y_* )-t \begin{pmatrix} a(x_* ,y_* ) & b(x_* ,y_* )\end{pmatrix} \begin{pmatrix} u_x(x_* ,y_* ) \\u_y(x_* ,y_* )\end{pmatrix} + r_1(t) \\
	&= u(x_* ,y_* ) + t \underbrace{ u(x_* ,y_* )}_{<0 }  + r_1(t)
\end{align*}
where $ \lim_{ t \to 0}  \frac{r_1(t)}{t } = 0$. Thus we can find a small enough $ t$  s.t.\ $ tu(x_* ,y_* )+r_1(t) <0$, contradicting that $ u(x_* ,y_* )$ is the minimum.
\end{case}
\begin{case}[4]
Suppose both maximum and minimum are achieved on $ \partial \Omega$. Let $ (x_* ,y_* )$ denotes the minimizer and $ (x^* ,y^* )$ denotes the maximizer. Then we have two subcases. If $ u(x_* ,y_* ) \geq 0$, this reduces to case 2. The remaining case is $u(x_*,y_*) <0$. But this is exactly the condition we need for the Taylor argument to work in case 3.
\end{case}
Hence, for all cases, we have $ u \equiv 0$.

I am aware of another proof where if a maximum is on the boundary, then the function is non-decreasing near the maximizer, so by continuity the derivative of $ u$ along outward tangent of the characteristic curve (which is exactly $ -u$) must be non-negative, so the maximum must be non-positive. Likewise for the minimum. But I already did it using Taylor's theorem so I omit it.
\end{problem}

\begin{problem}[5]
The initial curve $ \gamma_0$ is $ x_0^2+y_0^2=a^2$, and $ z_0=y$. The problem is not well-posed but we shall solve it anyway. The ODEs are
\begin{align*}
\begin{cases}
	\frac{d x}{d t} =y\\
	\frac{d y}{d t} =x\\
	\frac{d z}{d t} =0
\end{cases}
\end{align*}
The first two ODEs yield
\begin{align*}
	x \frac{d x}{d t} + y \frac{d y}{d t} = xy - xy &= 0 \\
	\frac{d}{dt} \left( \frac{1}{2} \left( x^2+y^2 \right)  \right) &= 0 \\
	x(t)^2+y(t)^2 &= C 
\end{align*}
Thus by the initial condition, $ C = a^2$. Moreover, the third ODE yields $ z = u = C_1$. By the initial condition, $ u \equiv y$. However, at point $ (a,0)$, we see that
 \begin{align*}
	y u_x - xu_y = y \cdot 0 - x \cdot 1 = -x = -a \neq 0,
\end{align*}
a contradiction. Therefore, such solution doesn't exist.
\end{problem}
\end{document}
