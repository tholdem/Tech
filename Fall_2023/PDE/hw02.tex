\documentclass[12pt]{article}
%Fall 2022
% Some basic packages
\usepackage{standalone}[subpreambles=true]
\usepackage[utf8]{inputenc}
\usepackage[T1]{fontenc}
\usepackage{textcomp}
\usepackage[english]{babel}
\usepackage{url}
\usepackage{graphicx}
%\usepackage{quiver}
\usepackage{float}
\usepackage{enumitem}
\usepackage{lmodern}
\usepackage{comment}
\usepackage{hyperref}
\usepackage[usenames,svgnames,dvipsnames]{xcolor}
\usepackage[margin=1in]{geometry}
\usepackage{pdfpages}

\pdfminorversion=7

% Don't indent paragraphs, leave some space between them
\usepackage{parskip}

% Hide page number when page is empty
\usepackage{emptypage}
\usepackage{subcaption}
\usepackage{multicol}
\usepackage[b]{esvect}

% Math stuff
\usepackage{amsmath, amsfonts, mathtools, amsthm, amssymb}
\usepackage{bbm}
\usepackage{stmaryrd}
\allowdisplaybreaks

% Fancy script capitals
\usepackage{mathrsfs}
\usepackage{cancel}
% Bold math
\usepackage{bm}
% Some shortcuts
\newcommand{\rr}{\ensuremath{\mathbb{R}}}
\newcommand{\zz}{\ensuremath{\mathbb{Z}}}
\newcommand{\qq}{\ensuremath{\mathbb{Q}}}
\newcommand{\nn}{\ensuremath{\mathbb{N}}}
\newcommand{\ff}{\ensuremath{\mathbb{F}}}
\newcommand{\cc}{\ensuremath{\mathbb{C}}}
\newcommand{\ee}{\ensuremath{\mathbb{E}}}
\newcommand{\hh}{\ensuremath{\mathbb{H}}}
\renewcommand\O{\ensuremath{\emptyset}}
\newcommand{\norm}[1]{{\left\lVert{#1}\right\rVert}}
\newcommand{\dbracket}[1]{{\left\llbracket{#1}\right\rrbracket}}
\newcommand{\ve}[1]{{\bm{#1}}}
\newcommand\allbold[1]{{\boldmath\textbf{#1}}}
\DeclareMathOperator{\lcm}{lcm}
\DeclareMathOperator{\im}{im}
\DeclareMathOperator{\coim}{coim}
\DeclareMathOperator{\dom}{dom}
\DeclareMathOperator{\tr}{tr}
\DeclareMathOperator{\rank}{rank}
\DeclareMathOperator*{\var}{Var}
\DeclareMathOperator*{\ev}{E}
\DeclareMathOperator{\dg}{deg}
\DeclareMathOperator{\aff}{aff}
\DeclareMathOperator{\conv}{conv}
\DeclareMathOperator{\inte}{int}
\DeclareMathOperator*{\argmin}{argmin}
\DeclareMathOperator*{\argmax}{argmax}
\DeclareMathOperator{\graph}{graph}
\DeclareMathOperator{\sgn}{sgn}
\DeclareMathOperator*{\Rep}{Rep}
\DeclareMathOperator{\Proj}{Proj}
\DeclareMathOperator{\mat}{mat}
\DeclareMathOperator{\diag}{diag}
\DeclareMathOperator{\aut}{Aut}
\DeclareMathOperator{\gal}{Gal}
\DeclareMathOperator{\inn}{Inn}
\DeclareMathOperator{\edm}{End}
\DeclareMathOperator{\Hom}{Hom}
\DeclareMathOperator{\ext}{Ext}
\DeclareMathOperator{\tor}{Tor}
\DeclareMathOperator{\Span}{Span}
\DeclareMathOperator{\Stab}{Stab}
\DeclareMathOperator{\cont}{cont}
\DeclareMathOperator{\Ann}{Ann}
\DeclareMathOperator{\Div}{div}
\DeclareMathOperator{\curl}{curl}
\DeclareMathOperator{\nat}{Nat}
\DeclareMathOperator{\gr}{Gr}
\DeclareMathOperator{\vect}{Vect}
\DeclareMathOperator{\id}{id}
\DeclareMathOperator{\Mod}{Mod}
\DeclareMathOperator{\sign}{sign}
\DeclareMathOperator{\Surf}{Surf}
\DeclareMathOperator{\fcone}{fcone}
\DeclareMathOperator{\Rot}{Rot}
\DeclareMathOperator{\grad}{grad}
\DeclareMathOperator{\atan2}{atan2}
\DeclareMathOperator{\Ric}{Ric}
\let\vec\relax
\DeclareMathOperator{\vec}{vec}
\let\Re\relax
\DeclareMathOperator{\Re}{Re}
\let\Im\relax
\DeclareMathOperator{\Im}{Im}
% Put x \to \infty below \lim
\let\svlim\lim\def\lim{\svlim\limits}

%wide hat
\usepackage{scalerel,stackengine}
\stackMath
\newcommand*\wh[1]{%
\savestack{\tmpbox}{\stretchto{%
  \scaleto{%
    \scalerel*[\widthof{\ensuremath{#1}}]{\kern-.6pt\bigwedge\kern-.6pt}%
    {\rule[-\textheight/2]{1ex}{\textheight}}%WIDTH-LIMITED BIG WEDGE
  }{\textheight}% 
}{0.5ex}}%
\stackon[1pt]{#1}{\tmpbox}%
}
\parskip 1ex

%Make implies and impliedby shorter
\let\implies\Rightarrow
\let\impliedby\Leftarrow
\let\iff\Leftrightarrow
\let\epsilon\varepsilon

% Add \contra symbol to denote contradiction
\usepackage{stmaryrd} % for \lightning
\newcommand\contra{\scalebox{1.5}{$\lightning$}}

% \let\phi\varphi

% Command for short corrections
% Usage: 1+1=\correct{3}{2}

\definecolor{correct}{HTML}{009900}
\newcommand\correct[2]{\ensuremath{\:}{\color{red}{#1}}\ensuremath{\to }{\color{correct}{#2}}\ensuremath{\:}}
\newcommand\green[1]{{\color{correct}{#1}}}

% horizontal rule
\newcommand\hr{
    \noindent\rule[0.5ex]{\linewidth}{0.5pt}
}

% hide parts
\newcommand\hide[1]{}

% si unitx
\usepackage{siunitx}
\sisetup{locale = FR}

%allows pmatrix to stretch
\makeatletter
\renewcommand*\env@matrix[1][\arraystretch]{%
  \edef\arraystretch{#1}%
  \hskip -\arraycolsep
  \let\@ifnextchar\new@ifnextchar
  \array{*\c@MaxMatrixCols c}}
\makeatother

\renewcommand{\arraystretch}{0.8}

\renewcommand{\baselinestretch}{1.5}

\usepackage{graphics}
\usepackage{epstopdf}

\RequirePackage{hyperref}
%%
%% Add support for color in order to color the hyperlinks.
%% 
\hypersetup{
  colorlinks = true,
  urlcolor = blue,
  citecolor = blue
}
%%fakesection Links
\hypersetup{
    colorlinks,
    linkcolor={red!50!black},
    citecolor={green!50!black},
    urlcolor={blue!80!black}
}
%customization of cleveref
\RequirePackage[capitalize,nameinlink]{cleveref}[0.19]

% Per SIAM Style Manual, "section" should be lowercase
\crefname{section}{section}{sections}
\crefname{subsection}{subsection}{subsections}
\Crefname{section}{Section}{Sections}
\Crefname{subsection}{Subsection}{Subsections}

% Per SIAM Style Manual, "Figure" should be spelled out in references
\Crefname{figure}{Figure}{Figures}

% Per SIAM Style Manual, don't say equation in front on an equation.
\crefformat{equation}{\textup{#2(#1)#3}}
\crefrangeformat{equation}{\textup{#3(#1)#4--#5(#2)#6}}
\crefmultiformat{equation}{\textup{#2(#1)#3}}{ and \textup{#2(#1)#3}}
{, \textup{#2(#1)#3}}{, and \textup{#2(#1)#3}}
\crefrangemultiformat{equation}{\textup{#3(#1)#4--#5(#2)#6}}%
{ and \textup{#3(#1)#4--#5(#2)#6}}{, \textup{#3(#1)#4--#5(#2)#6}}{, and \textup{#3(#1)#4--#5(#2)#6}}

% But spell it out at the beginning of a sentence.
\Crefformat{equation}{#2Equation~\textup{(#1)}#3}
\Crefrangeformat{equation}{Equations~\textup{#3(#1)#4--#5(#2)#6}}
\Crefmultiformat{equation}{Equations~\textup{#2(#1)#3}}{ and \textup{#2(#1)#3}}
{, \textup{#2(#1)#3}}{, and \textup{#2(#1)#3}}
\Crefrangemultiformat{equation}{Equations~\textup{#3(#1)#4--#5(#2)#6}}%
{ and \textup{#3(#1)#4--#5(#2)#6}}{, \textup{#3(#1)#4--#5(#2)#6}}{, and \textup{#3(#1)#4--#5(#2)#6}}

% Make number non-italic in any environment.
\crefdefaultlabelformat{#2\textup{#1}#3}

% Environments
\makeatother
% For box around Definition, Theorem, \ldots
%%fakesection Theorems
\usepackage{thmtools}
\usepackage[framemethod=TikZ]{mdframed}

\theoremstyle{definition}
\mdfdefinestyle{mdbluebox}{%
	roundcorner = 10pt,
	linewidth=1pt,
	skipabove=12pt,
	innerbottommargin=9pt,
	skipbelow=2pt,
	nobreak=true,
	linecolor=blue,
	backgroundcolor=TealBlue!5,
}
\declaretheoremstyle[
	headfont=\sffamily\bfseries\color{MidnightBlue},
	mdframed={style=mdbluebox},
	headpunct={\\[3pt]},
	postheadspace={0pt}
]{thmbluebox}

\mdfdefinestyle{mdredbox}{%
	linewidth=0.5pt,
	skipabove=12pt,
	frametitleaboveskip=5pt,
	frametitlebelowskip=0pt,
	skipbelow=2pt,
	frametitlefont=\bfseries,
	innertopmargin=4pt,
	innerbottommargin=8pt,
	nobreak=false,
	linecolor=RawSienna,
	backgroundcolor=Salmon!5,
}
\declaretheoremstyle[
	headfont=\bfseries\color{RawSienna},
	mdframed={style=mdredbox},
	headpunct={\\[3pt]},
	postheadspace={0pt},
]{thmredbox}

\declaretheorem[%
style=thmbluebox,name=Theorem,numberwithin=section]{thm}
\declaretheorem[style=thmbluebox,name=Lemma,sibling=thm]{lem}
\declaretheorem[style=thmbluebox,name=Proposition,sibling=thm]{prop}
\declaretheorem[style=thmbluebox,name=Corollary,sibling=thm]{coro}
\declaretheorem[style=thmredbox,name=Example,sibling=thm]{eg}

\mdfdefinestyle{mdgreenbox}{%
	roundcorner = 10pt,
	linewidth=1pt,
	skipabove=12pt,
	innerbottommargin=9pt,
	skipbelow=2pt,
	nobreak=true,
	linecolor=ForestGreen,
	backgroundcolor=ForestGreen!5,
}

\declaretheoremstyle[
	headfont=\bfseries\sffamily\color{ForestGreen!70!black},
	bodyfont=\normalfont,
	spaceabove=2pt,
	spacebelow=1pt,
	mdframed={style=mdgreenbox},
	headpunct={ --- },
]{thmgreenbox}

\declaretheorem[style=thmgreenbox,name=Definition,sibling=thm]{defn}

\mdfdefinestyle{mdgreenboxsq}{%
	linewidth=1pt,
	skipabove=12pt,
	innerbottommargin=9pt,
	skipbelow=2pt,
	nobreak=true,
	linecolor=ForestGreen,
	backgroundcolor=ForestGreen!5,
}
\declaretheoremstyle[
	headfont=\bfseries\sffamily\color{ForestGreen!70!black},
	bodyfont=\normalfont,
	spaceabove=2pt,
	spacebelow=1pt,
	mdframed={style=mdgreenboxsq},
	headpunct={},
]{thmgreenboxsq}
\declaretheoremstyle[
	headfont=\bfseries\sffamily\color{ForestGreen!70!black},
	bodyfont=\normalfont,
	spaceabove=2pt,
	spacebelow=1pt,
	mdframed={style=mdgreenboxsq},
	headpunct={},
]{thmgreenboxsq*}

\mdfdefinestyle{mdblackbox}{%
	skipabove=8pt,
	linewidth=3pt,
	rightline=false,
	leftline=true,
	topline=false,
	bottomline=false,
	linecolor=black,
	backgroundcolor=RedViolet!5!gray!5,
}
\declaretheoremstyle[
	headfont=\bfseries,
	bodyfont=\normalfont\small,
	spaceabove=0pt,
	spacebelow=0pt,
	mdframed={style=mdblackbox}
]{thmblackbox}

\theoremstyle{plain}
\declaretheorem[name=Question,sibling=thm,style=thmblackbox]{ques}
\declaretheorem[name=Remark,sibling=thm,style=thmgreenboxsq]{remark}
\declaretheorem[name=Remark,sibling=thm,style=thmgreenboxsq*]{remark*}
\newtheorem{ass}[thm]{Assumptions}

\theoremstyle{definition}
\newtheorem*{problem}{Problem}
\newtheorem{claim}[thm]{Claim}
\theoremstyle{remark}
\newtheorem*{case}{Case}
\newtheorem*{notation}{Notation}
\newtheorem*{note}{Note}
\newtheorem*{motivation}{Motivation}
\newtheorem*{intuition}{Intuition}
\newtheorem*{conjecture}{Conjecture}

% Make section starts with 1 for report type
%\renewcommand\thesection{\arabic{section}}

% End example and intermezzo environments with a small diamond (just like proof
% environments end with a small square)
\usepackage{etoolbox}
\AtEndEnvironment{vb}{\null\hfill$\diamond$}%
\AtEndEnvironment{intermezzo}{\null\hfill$\diamond$}%
% \AtEndEnvironment{opmerking}{\null\hfill$\diamond$}%

% Fix some spacing
% http://tex.stackexchange.com/questions/22119/how-can-i-change-the-spacing-before-theorems-with-amsthm
\makeatletter
\def\thm@space@setup{%
  \thm@preskip=\parskip \thm@postskip=0pt
}

% Fix some stuff
% %http://tex.stackexchange.com/questions/76273/multiple-pdfs-with-page-group-included-in-a-single-page-warning
\pdfsuppresswarningpagegroup=1


% My name
\author{Jaden Wang}



\begin{document}
\centerline {\textsf{\textbf{\LARGE{Homework 2}}}}
\centerline {Jaden Wang}
\vspace{.15in}
\begin{problem}[2.1(i)]
The initial curve can be parameterized: $x_0=s, y_0=s^2, z_0=u(x_0,y_0)= s^2$. Let us check it is not characteristic. The characteristic equation is $ a \sigma_0 + b \sigma_1 = 0$, where $ ( \sigma_0, \sigma_1)$ is the outward normal to the characteristic curve which has tangent $ (a,b)$. Thus we don't want the initial curve to have tangent parallel to  $ (a,b)$. And $ (a,b)$ for initial curve is parameterized by  $ (s,s ^2)$. We check
 \begin{align*}
	 \det \begin{pmatrix} s& s ^2\\1 & 2s \end{pmatrix} = 2s ^2-s ^2 = s ^2 \neq 0 \iff s \neq 0,
\end{align*}
which is always true or we wouldn't have an initial curve. We can safely proceed to solve
\begin{align*}
	\begin{cases}
		\frac{d x}{d t} = x, x(0) = s&\implies x= s e^{t}\\
		\frac{d y}{d t} =y, y(0) = s ^2& \implies y= s ^2 e^{t}\\
		\frac{d z}{d t} = c=z+1, z(0) = s ^2& \implies z= (s ^2+1)e^{t}\\
	\end{cases}
\end{align*}
We have $ s=\frac{y}{x}$ and $ e^{t} = \frac{x^2}{ y}$. So the solution is $u(x,y)= \left( \frac{y^2}{x^2}+1 \right) \frac{x^2}{y}-1 = y+\frac{x^2}{y}-1$.
\end{problem}
\begin{problem}[2]
We shall use the method of integrating factor. Multiplying the PDE by $ e^{ct}$ yields
\begin{align*}
	e^{ct} u_t + e^{ct} b \cdot D_x u +c e^{ct} u&= 0 \\
	\left( e^{ct}u \right)_t + e^{ct} b \cdot D_x u &= 0 && \text{ product rule}  \\
	\left( e^{ct}u \right)_t + b \cdot D_x \left( e^{ct}u \right)  &= 0 && e^{ct}\text{ can be treated as constant}  \\
	e^{ct}u&= g(x-tb) && \text{linear transport solution}\\
	u&= g(x-tb) e^{-ct} && e^{ct} \neq 0 
\end{align*}

\end{problem}

\begin{problem}[3]
This is Burger's equation. The initial curve is parameterized as $t_0=0, x_0=s, z_0 = \frac{1}{1+s ^2}$. We see that
\begin{align*}
	\det \begin{pmatrix}   1&u\\0&1 \end{pmatrix} \neq 0 
\end{align*}
Then ODEs are
\begin{align*}
	\frac{dt}{ d \tau} &= 1, t(0)=0 \implies t = \tau\\
	\frac{dx}{ d \tau} &= u, x(0) = s \implies x(t)= s+ \frac{1}{1+s^2}t =x_0+tu(x_0,0)\\
	\frac{dz}{ d \tau} &= 0, z(0) = \frac{1}{1+s ^2} \implies z(t) = \frac{1}{1+s ^2} = u(x_0,0)
\end{align*}
Therefore, $ u(x,t)$ is completely determined by  $ x_0$. The solution blows up when $ w:=u_x$ tends to infinity. We see
 \begin{align*}
	0=\left( u_t+uu_x \right) _x &= u_{xt} + u_x ^2 + uu_{xx} \\
	&= w_t+w^2+uw_x 
\end{align*}
Moreover, on the characteristic line $ x(t)= x_0+tu(x_0,0)$, we have
\begin{align*}
	\frac{d}{dt}(w) &= w_t + w_x \frac{d x}{d t}  \\
	&= w_t + w_x u(x_0,0)
\end{align*}
Combining the two equations, we have the following ODE on the characteristic line:
\begin{align*}
	\begin{cases}
		\dot{w} &= -w^2\\
		w(x_0,0) &= \left( \frac{1}{1+x_0^2} \right)' = -\frac{2x_0}{ (1+x_0^2)^2}
	\end{cases}
\end{align*}
Solving this yields
\begin{align*}
	w^{-1}&= t - \frac{(1+x_0^2)^2}{ 2x_0}
\end{align*}
so $ w$ blows up when  $ t= \frac{(1+x_0^2)^2}{  2x_0}$. To find the first time it blows up, \emph{i.e.} minimum time, we set
\begin{align*}
	t' = \frac{ 8x_0^2(1+x_0^2)-2(1+x_0^2)^2}{ 4x_0^{2}} &=0\\
	2(3x_0^{4}+2x_0^2-1) &= 0 \\
	x_0 &= \pm \frac{1}{\sqrt{3} } \\
	t &= \pm \frac{8}{ 3 \sqrt{3}}
\end{align*}
We can easily see from the graph that minimum is achieved at $ t= \frac{8}{ 3 \sqrt{3}}$.
\end{problem}
\begin{problem}[4]
	First, since $ (x,y)$ is the outward normal vector of the unit circle, the condition  $ \begin{pmatrix} a&b \end{pmatrix} \begin{pmatrix} x\\y \end{pmatrix} >0$ means that $ (a,b)$ also points outward from the tangent line at $ (x,y)$. This means that for a small $ t$, $ (x-ta,y-tb) \in \inte \Omega$.

Since $ \Omega$ is compact, $ u$ achieves maximum and minimum. Suppose the minimum is achieved at $ \inte \Omega$. Then necessary condition says $ \begin{pmatrix} u_x\\u_y \end{pmatrix}  = \begin{pmatrix} 0\\0 \end{pmatrix} $, which yields $ u(x,y) = au_x+bu_y = 0$. Thus by definition of minimum, $ u(x,y) \geq 0$ for all $ x,y$. Likewise, if maximum is achieved at the interior, then $ u(x,y) \leq 0$. We have four cases:
 \begin{case}[1] If both maximum and minimum are achieved in the interior, we have $ u \equiv 0$ immediately.
\end{case}
\begin{case}[2] If only minimum is in the interior, then $ u(x,y) \geq 0$. Let  $ (x^* ,y^* ) \in \partial \Omega$ denotes the maximizer. If $ u(x^* ,y^* ) = 0$, then $ u \equiv 0$. Suppose  $ u(x^* ,y^* )>0$, then by Taylor's theorem, since $ u \in C^{1}$,
\begin{align*}
	u(x^* -ta,y^* -tb) &= u(x^* ,y^* )-t \begin{pmatrix} a(x^* ,y^* ) & b(x^* ,y^* )\end{pmatrix} \begin{pmatrix} u_x(x^* ,y^* ) \\u_y(x^* ,y^* )\end{pmatrix} + r_1(t) \\
	&= u(x^* ,y^* ) + t \underbrace{ u(x^* ,y^* )}_{>0 }  + r_1(t)
\end{align*}
where $ \lim_{ t \to 0}  \frac{r_1(t)}{t } = 0$. Thus we can find a small enough $ t$  s.t.\ $ tu(x^* ,y^* )+r_1(t) >0$, contradicting that $ u(x^* ,y^* )$ is the maximum.
\end{case}

\begin{case}[3]
Similarly, if only maximum is in the interior, then $ u(x,y) \leq 0$. Let  $ (x_* ,y_* ) \in \partial \Omega$ denotes the minimizer. If $ u(x_* ,y_* ) = 0$, then $ u \equiv 0$. Suppose  $ u(x_* ,y_* )<0$, then by Taylor's theorem, since $ u \in C^{1}$,
\begin{align*}
	u(x_* -ta,y_* -tb) &= u(x_* ,y_* )-t \begin{pmatrix} a(x_* ,y_* ) & b(x_* ,y_* )\end{pmatrix} \begin{pmatrix} u_x(x_* ,y_* ) \\u_y(x_* ,y_* )\end{pmatrix} + r_1(t) \\
	&= u(x_* ,y_* ) + t \underbrace{ u(x_* ,y_* )}_{<0 }  + r_1(t)
\end{align*}
where $ \lim_{ t \to 0}  \frac{r_1(t)}{t } = 0$. Thus we can find a small enough $ t$  s.t.\ $ tu(x_* ,y_* )+r_1(t) <0$, contradicting that $ u(x_* ,y_* )$ is the minimum.
\end{case}
\begin{case}[4]
Suppose both maximum and minimum are achieved on $ \partial \Omega$. Let $ (x_* ,y_* )$ denotes the minimizer and $ (x^* ,y^* )$ denotes the maximizer. Then we have two subcases. If $ u(x_* ,y_* ) \geq 0$, this reduces to case 2. The remaining case is $u(x_*,y_*) <0$. But this is exactly the condition we need for the Taylor argument to work in case 3.
\end{case}
Hence, for all cases, we have $ u \equiv 0$.

I am aware of another proof where if a maximum is on the boundary, then the function is non-decreasing near the maximizer, so by continuity the derivative of $ u$ along outward tangent of the characteristic curve (which is exactly $ -u$) must be non-negative, so the maximum must be non-positive. Likewise for the minimum. But I already did it using Taylor's theorem so I omit it.
\end{problem}

\begin{problem}[5]
The initial curve $ \gamma_0$ is $ x_0^2+y_0^2=a^2$, and $ z_0=y$. The problem is not well-posed but we shall solve it anyway. The ODEs are
\begin{align*}
\begin{cases}
	\frac{d x}{d t} =y\\
	\frac{d y}{d t} =x\\
	\frac{d z}{d t} =0
\end{cases}
\end{align*}
The first two ODEs yield
\begin{align*}
	x \frac{d x}{d t} + y \frac{d y}{d t} = xy - xy &= 0 \\
	\frac{d}{dt} \left( \frac{1}{2} \left( x^2+y^2 \right)  \right) &= 0 \\
	x(t)^2+y(t)^2 &= C 
\end{align*}
Thus by the initial condition, $ C = a^2$. Moreover, the third ODE yields $ z = u = C_1$. By the initial condition, $ u \equiv y$. However, at point $ (a,0)$, we see that
 \begin{align*}
	y u_x - xu_y = y \cdot 0 - x \cdot 1 = -x = -a \neq 0,
\end{align*}
a contradiction. Therefore, such solution doesn't exist.
\end{problem}
\end{document}
