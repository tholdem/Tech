\documentclass[12pt]{article}
%Fall 2022
% Some basic packages
\usepackage{standalone}[subpreambles=true]
\usepackage[utf8]{inputenc}
\usepackage[T1]{fontenc}
\usepackage{textcomp}
\usepackage[english]{babel}
\usepackage{url}
\usepackage{graphicx}
%\usepackage{quiver}
\usepackage{float}
\usepackage{enumitem}
\usepackage{lmodern}
\usepackage{comment}
\usepackage{hyperref}
\usepackage[usenames,svgnames,dvipsnames]{xcolor}
\usepackage[margin=1in]{geometry}
\usepackage{pdfpages}

\pdfminorversion=7

% Don't indent paragraphs, leave some space between them
\usepackage{parskip}

% Hide page number when page is empty
\usepackage{emptypage}
\usepackage{subcaption}
\usepackage{multicol}
\usepackage[b]{esvect}

% Math stuff
\usepackage{amsmath, amsfonts, mathtools, amsthm, amssymb}
\usepackage{bbm}
\usepackage{stmaryrd}
\allowdisplaybreaks

% Fancy script capitals
\usepackage{mathrsfs}
\usepackage{cancel}
% Bold math
\usepackage{bm}
% Some shortcuts
\newcommand{\rr}{\ensuremath{\mathbb{R}}}
\newcommand{\zz}{\ensuremath{\mathbb{Z}}}
\newcommand{\qq}{\ensuremath{\mathbb{Q}}}
\newcommand{\nn}{\ensuremath{\mathbb{N}}}
\newcommand{\ff}{\ensuremath{\mathbb{F}}}
\newcommand{\cc}{\ensuremath{\mathbb{C}}}
\newcommand{\ee}{\ensuremath{\mathbb{E}}}
\newcommand{\hh}{\ensuremath{\mathbb{H}}}
\renewcommand\O{\ensuremath{\emptyset}}
\newcommand{\norm}[1]{{\left\lVert{#1}\right\rVert}}
\newcommand{\dbracket}[1]{{\left\llbracket{#1}\right\rrbracket}}
\newcommand{\ve}[1]{{\bm{#1}}}
\newcommand\allbold[1]{{\boldmath\textbf{#1}}}
\DeclareMathOperator{\lcm}{lcm}
\DeclareMathOperator{\im}{im}
\DeclareMathOperator{\coim}{coim}
\DeclareMathOperator{\dom}{dom}
\DeclareMathOperator{\tr}{tr}
\DeclareMathOperator{\rank}{rank}
\DeclareMathOperator*{\var}{Var}
\DeclareMathOperator*{\ev}{E}
\DeclareMathOperator{\dg}{deg}
\DeclareMathOperator{\aff}{aff}
\DeclareMathOperator{\conv}{conv}
\DeclareMathOperator{\inte}{int}
\DeclareMathOperator*{\argmin}{argmin}
\DeclareMathOperator*{\argmax}{argmax}
\DeclareMathOperator{\graph}{graph}
\DeclareMathOperator{\sgn}{sgn}
\DeclareMathOperator*{\Rep}{Rep}
\DeclareMathOperator{\Proj}{Proj}
\DeclareMathOperator{\mat}{mat}
\DeclareMathOperator{\diag}{diag}
\DeclareMathOperator{\aut}{Aut}
\DeclareMathOperator{\gal}{Gal}
\DeclareMathOperator{\inn}{Inn}
\DeclareMathOperator{\edm}{End}
\DeclareMathOperator{\Hom}{Hom}
\DeclareMathOperator{\ext}{Ext}
\DeclareMathOperator{\tor}{Tor}
\DeclareMathOperator{\Span}{Span}
\DeclareMathOperator{\Stab}{Stab}
\DeclareMathOperator{\cont}{cont}
\DeclareMathOperator{\Ann}{Ann}
\DeclareMathOperator{\Div}{div}
\DeclareMathOperator{\curl}{curl}
\DeclareMathOperator{\nat}{Nat}
\DeclareMathOperator{\gr}{Gr}
\DeclareMathOperator{\vect}{Vect}
\DeclareMathOperator{\id}{id}
\DeclareMathOperator{\Mod}{Mod}
\DeclareMathOperator{\sign}{sign}
\DeclareMathOperator{\Surf}{Surf}
\DeclareMathOperator{\fcone}{fcone}
\DeclareMathOperator{\Rot}{Rot}
\DeclareMathOperator{\grad}{grad}
\DeclareMathOperator{\atan2}{atan2}
\DeclareMathOperator{\Ric}{Ric}
\let\vec\relax
\DeclareMathOperator{\vec}{vec}
\let\Re\relax
\DeclareMathOperator{\Re}{Re}
\let\Im\relax
\DeclareMathOperator{\Im}{Im}
% Put x \to \infty below \lim
\let\svlim\lim\def\lim{\svlim\limits}

%wide hat
\usepackage{scalerel,stackengine}
\stackMath
\newcommand*\wh[1]{%
\savestack{\tmpbox}{\stretchto{%
  \scaleto{%
    \scalerel*[\widthof{\ensuremath{#1}}]{\kern-.6pt\bigwedge\kern-.6pt}%
    {\rule[-\textheight/2]{1ex}{\textheight}}%WIDTH-LIMITED BIG WEDGE
  }{\textheight}% 
}{0.5ex}}%
\stackon[1pt]{#1}{\tmpbox}%
}
\parskip 1ex

%Make implies and impliedby shorter
\let\implies\Rightarrow
\let\impliedby\Leftarrow
\let\iff\Leftrightarrow
\let\epsilon\varepsilon

% Add \contra symbol to denote contradiction
\usepackage{stmaryrd} % for \lightning
\newcommand\contra{\scalebox{1.5}{$\lightning$}}

% \let\phi\varphi

% Command for short corrections
% Usage: 1+1=\correct{3}{2}

\definecolor{correct}{HTML}{009900}
\newcommand\correct[2]{\ensuremath{\:}{\color{red}{#1}}\ensuremath{\to }{\color{correct}{#2}}\ensuremath{\:}}
\newcommand\green[1]{{\color{correct}{#1}}}

% horizontal rule
\newcommand\hr{
    \noindent\rule[0.5ex]{\linewidth}{0.5pt}
}

% hide parts
\newcommand\hide[1]{}

% si unitx
\usepackage{siunitx}
\sisetup{locale = FR}

%allows pmatrix to stretch
\makeatletter
\renewcommand*\env@matrix[1][\arraystretch]{%
  \edef\arraystretch{#1}%
  \hskip -\arraycolsep
  \let\@ifnextchar\new@ifnextchar
  \array{*\c@MaxMatrixCols c}}
\makeatother

\renewcommand{\arraystretch}{0.8}

\renewcommand{\baselinestretch}{1.5}

\usepackage{graphics}
\usepackage{epstopdf}

\RequirePackage{hyperref}
%%
%% Add support for color in order to color the hyperlinks.
%% 
\hypersetup{
  colorlinks = true,
  urlcolor = blue,
  citecolor = blue
}
%%fakesection Links
\hypersetup{
    colorlinks,
    linkcolor={red!50!black},
    citecolor={green!50!black},
    urlcolor={blue!80!black}
}
%customization of cleveref
\RequirePackage[capitalize,nameinlink]{cleveref}[0.19]

% Per SIAM Style Manual, "section" should be lowercase
\crefname{section}{section}{sections}
\crefname{subsection}{subsection}{subsections}
\Crefname{section}{Section}{Sections}
\Crefname{subsection}{Subsection}{Subsections}

% Per SIAM Style Manual, "Figure" should be spelled out in references
\Crefname{figure}{Figure}{Figures}

% Per SIAM Style Manual, don't say equation in front on an equation.
\crefformat{equation}{\textup{#2(#1)#3}}
\crefrangeformat{equation}{\textup{#3(#1)#4--#5(#2)#6}}
\crefmultiformat{equation}{\textup{#2(#1)#3}}{ and \textup{#2(#1)#3}}
{, \textup{#2(#1)#3}}{, and \textup{#2(#1)#3}}
\crefrangemultiformat{equation}{\textup{#3(#1)#4--#5(#2)#6}}%
{ and \textup{#3(#1)#4--#5(#2)#6}}{, \textup{#3(#1)#4--#5(#2)#6}}{, and \textup{#3(#1)#4--#5(#2)#6}}

% But spell it out at the beginning of a sentence.
\Crefformat{equation}{#2Equation~\textup{(#1)}#3}
\Crefrangeformat{equation}{Equations~\textup{#3(#1)#4--#5(#2)#6}}
\Crefmultiformat{equation}{Equations~\textup{#2(#1)#3}}{ and \textup{#2(#1)#3}}
{, \textup{#2(#1)#3}}{, and \textup{#2(#1)#3}}
\Crefrangemultiformat{equation}{Equations~\textup{#3(#1)#4--#5(#2)#6}}%
{ and \textup{#3(#1)#4--#5(#2)#6}}{, \textup{#3(#1)#4--#5(#2)#6}}{, and \textup{#3(#1)#4--#5(#2)#6}}

% Make number non-italic in any environment.
\crefdefaultlabelformat{#2\textup{#1}#3}

% Environments
\makeatother
% For box around Definition, Theorem, \ldots
%%fakesection Theorems
\usepackage{thmtools}
\usepackage[framemethod=TikZ]{mdframed}

\theoremstyle{definition}
\mdfdefinestyle{mdbluebox}{%
	roundcorner = 10pt,
	linewidth=1pt,
	skipabove=12pt,
	innerbottommargin=9pt,
	skipbelow=2pt,
	nobreak=true,
	linecolor=blue,
	backgroundcolor=TealBlue!5,
}
\declaretheoremstyle[
	headfont=\sffamily\bfseries\color{MidnightBlue},
	mdframed={style=mdbluebox},
	headpunct={\\[3pt]},
	postheadspace={0pt}
]{thmbluebox}

\mdfdefinestyle{mdredbox}{%
	linewidth=0.5pt,
	skipabove=12pt,
	frametitleaboveskip=5pt,
	frametitlebelowskip=0pt,
	skipbelow=2pt,
	frametitlefont=\bfseries,
	innertopmargin=4pt,
	innerbottommargin=8pt,
	nobreak=false,
	linecolor=RawSienna,
	backgroundcolor=Salmon!5,
}
\declaretheoremstyle[
	headfont=\bfseries\color{RawSienna},
	mdframed={style=mdredbox},
	headpunct={\\[3pt]},
	postheadspace={0pt},
]{thmredbox}

\declaretheorem[%
style=thmbluebox,name=Theorem,numberwithin=section]{thm}
\declaretheorem[style=thmbluebox,name=Lemma,sibling=thm]{lem}
\declaretheorem[style=thmbluebox,name=Proposition,sibling=thm]{prop}
\declaretheorem[style=thmbluebox,name=Corollary,sibling=thm]{coro}
\declaretheorem[style=thmredbox,name=Example,sibling=thm]{eg}

\mdfdefinestyle{mdgreenbox}{%
	roundcorner = 10pt,
	linewidth=1pt,
	skipabove=12pt,
	innerbottommargin=9pt,
	skipbelow=2pt,
	nobreak=true,
	linecolor=ForestGreen,
	backgroundcolor=ForestGreen!5,
}

\declaretheoremstyle[
	headfont=\bfseries\sffamily\color{ForestGreen!70!black},
	bodyfont=\normalfont,
	spaceabove=2pt,
	spacebelow=1pt,
	mdframed={style=mdgreenbox},
	headpunct={ --- },
]{thmgreenbox}

\declaretheorem[style=thmgreenbox,name=Definition,sibling=thm]{defn}

\mdfdefinestyle{mdgreenboxsq}{%
	linewidth=1pt,
	skipabove=12pt,
	innerbottommargin=9pt,
	skipbelow=2pt,
	nobreak=true,
	linecolor=ForestGreen,
	backgroundcolor=ForestGreen!5,
}
\declaretheoremstyle[
	headfont=\bfseries\sffamily\color{ForestGreen!70!black},
	bodyfont=\normalfont,
	spaceabove=2pt,
	spacebelow=1pt,
	mdframed={style=mdgreenboxsq},
	headpunct={},
]{thmgreenboxsq}
\declaretheoremstyle[
	headfont=\bfseries\sffamily\color{ForestGreen!70!black},
	bodyfont=\normalfont,
	spaceabove=2pt,
	spacebelow=1pt,
	mdframed={style=mdgreenboxsq},
	headpunct={},
]{thmgreenboxsq*}

\mdfdefinestyle{mdblackbox}{%
	skipabove=8pt,
	linewidth=3pt,
	rightline=false,
	leftline=true,
	topline=false,
	bottomline=false,
	linecolor=black,
	backgroundcolor=RedViolet!5!gray!5,
}
\declaretheoremstyle[
	headfont=\bfseries,
	bodyfont=\normalfont\small,
	spaceabove=0pt,
	spacebelow=0pt,
	mdframed={style=mdblackbox}
]{thmblackbox}

\theoremstyle{plain}
\declaretheorem[name=Question,sibling=thm,style=thmblackbox]{ques}
\declaretheorem[name=Remark,sibling=thm,style=thmgreenboxsq]{remark}
\declaretheorem[name=Remark,sibling=thm,style=thmgreenboxsq*]{remark*}
\newtheorem{ass}[thm]{Assumptions}

\theoremstyle{definition}
\newtheorem*{problem}{Problem}
\newtheorem{claim}[thm]{Claim}
\theoremstyle{remark}
\newtheorem*{case}{Case}
\newtheorem*{notation}{Notation}
\newtheorem*{note}{Note}
\newtheorem*{motivation}{Motivation}
\newtheorem*{intuition}{Intuition}
\newtheorem*{conjecture}{Conjecture}

% Make section starts with 1 for report type
%\renewcommand\thesection{\arabic{section}}

% End example and intermezzo environments with a small diamond (just like proof
% environments end with a small square)
\usepackage{etoolbox}
\AtEndEnvironment{vb}{\null\hfill$\diamond$}%
\AtEndEnvironment{intermezzo}{\null\hfill$\diamond$}%
% \AtEndEnvironment{opmerking}{\null\hfill$\diamond$}%

% Fix some spacing
% http://tex.stackexchange.com/questions/22119/how-can-i-change-the-spacing-before-theorems-with-amsthm
\makeatletter
\def\thm@space@setup{%
  \thm@preskip=\parskip \thm@postskip=0pt
}

% Fix some stuff
% %http://tex.stackexchange.com/questions/76273/multiple-pdfs-with-page-group-included-in-a-single-page-warning
\pdfsuppresswarningpagegroup=1


% My name
\author{Jaden Wang}



\begin{document}
\centerline {\textsf{\textbf{\LARGE{Homework 1}}}}
\centerline {Jaden Wang}
\vspace{.15in}

\begin{problem}[1.4.3]
The coefficient matrix of second order terms is
\begin{align*}
	A = \begin{pmatrix} 3&-1&0\\-1&2&-1\\0&-1&3 \end{pmatrix} 
\end{align*}
Spectral theorem yields
\begin{align*}
	O^{T}AO &= \diag \{4,3,1\}, \text{ where } O = \begin{pmatrix} \frac{1}{\sqrt{3} }& -\frac{1}{\sqrt{2} }& \frac{1}{\sqrt{6} }\\-\frac{1}{\sqrt{3} } & 0 & \frac{2}{\sqrt{6} }\\ \frac{1}{\sqrt{3} }& \frac{1}{\sqrt{2} } & \frac{1}{\sqrt{6} } \end{pmatrix}    \\
	U^{-1}AU &= \diag \{1,1,1\}, \text{ where } U = \begin{pmatrix} \frac{2}{\sqrt{3} }& -\frac{\sqrt{3} }{\sqrt{2} }& \frac{1}{\sqrt{6} }\\-\frac{2}{\sqrt{3} } & 0 & \frac{2}{\sqrt{6} }\\ \frac{2}{\sqrt{3} }& \frac{\sqrt{3} }{\sqrt{2} } & \frac{1}{\sqrt{6} } \end{pmatrix}  \\
\end{align*}
Thus by a change of basis via $ U$, under the change of variables  $ \begin{pmatrix} r\\s\\t \end{pmatrix}  =U^{-1} \begin{pmatrix} x\\y\\z \end{pmatrix}  $, where $ U^{-1} = \begin{pmatrix} \frac{\sqrt{3} }{6 }& - \frac{\sqrt{3} }{ 6}& \frac{\sqrt{3} }{ 6} \\ - \frac{\sqrt{6} }{ 6}&0& \frac{\sqrt{6} }{ 6}\\ \frac{\sqrt{6} }{ 6} & \frac{\sqrt{6} }{ 3} & \frac{\sqrt{6} }{ 6}  \end{pmatrix} $.

By the chain rule, the derivative of $ u$ under basis  $ u_x, u_y, u_z$ is $ Du$ under basis  $ u_r,u_s,u_t$ composed with Jacobian of this change of variable, which is evidently $ U^{-1}$. That is
\begin{align*}
	\begin{pmatrix} u_x &u_y&u_z \end{pmatrix} &= U^{-1} \begin{pmatrix} u_r&u_s&u_t \end{pmatrix}  \\
	u_x &= \frac{\sqrt{3} }{ 6} u_r  -\frac{\sqrt{3} }{ 6}  u_s  + \frac{\sqrt{3} }{ 6}  u_t \\
	u_y&= - \frac{\sqrt{6} }{ 6}u_r + \frac{\sqrt{6} }{ 6} u_t
\end{align*}
Thus, the standard form is
\begin{align*}
	u_{rr} + u_{ss} + u_{tt} - \frac{5 \sqrt{6}+ \sqrt{3}  }{6 }u_r + \frac{\sqrt{3} }{ 6}u_s + \frac{5\sqrt{6}-\sqrt{3}  } { 6}u_t +10 u\left( U (r,s,t) \right) =0
\end{align*}
\end{problem}

\begin{problem}[1.4.7]
We see that $ a=1,b=0,c=y$, so  $ d= ac - b^2 = y$. 
\begin{case}[1]
If $ y<0$, then we have the hyperbolic case.  Let $ s = s(x,y)$ and  $ t = t(x,y)$  s.t.\ the Jacobian of this change of variable is invertible. Then the linear principal part becomes $ L_0u = a^* u_{ss}+ 2b^* u_{st} +c^* u_{tt} $, where
\end{case}
\begin{align*}
	\begin{cases}
		a^* = s_x^2+ y s_y^2\\
		b^* = s_x t_x + y s_yt_y \\
		c^* = t_x^2+yt_y^2 \\
	\end{cases}
\end{align*}
To convert to standard form, we want $ a^* =c^* =0$ and $ b^* \neq 0$, which indeed makes the equation hyperbolic. Then $ s,t$ has to satisfy the following equation
 \begin{align*}
	\phi_x^2 + y \phi_y^2 =0
\end{align*}
Suppose $ \phi(x,y)=$constant, \emph{i.e.} a level set, then $ y$ can locally be expressed as a function of  $ x$ (or the other way around).  Thus if $ \phi_y \neq 0$, by implicit differentiation we have
\begin{align*}
	D\phi = \phi_x + \phi_y \frac{d y}{d x} &= 0 \\
	\frac{d y}{d x} &= - \frac{\phi_x}{ \phi_y} 
\end{align*}
Thus we have the following ODE:
\begin{align*}
	\left( \frac{d y}{d x}  \right) ^2+y &= 0 \\
\frac{dy}{ \sqrt{-y} }&= \pm dx \\
2\sqrt{-y}  &= \mp x +C 
\end{align*}
Thus we have
\begin{align*}
	\phi_1(x,y) = x + 2\sqrt{-y} = c_1 , \ \phi_2(x,y) = - x + 2\sqrt{-y} = c_2. 
\end{align*}
The Jacobian of the change of variable is invertible, so we let $ s=\phi_1$ and $ t = \phi_2$. Then $ s_x =1$, $ s_y = t_y = -(-y) ^{-\frac{1}{2}}$, $ t_x = -1, s_{yy} = t_{yy} = -\frac{1}{2} (-y)^{-\frac{3}{2}} $. Then $ u_x = u_s s_x + u_t t_x = u_s - u_t$ and $ u_y = u_s s_y + u_t t_y = - (-y)^{-\frac{1}{2}}(u_s + u_t)$. $ u_{x x} = u_{ss} s_x^2 +2u_{st} s_xt_x +u_s s_{xx} + u_{tt}t_x^2 +u_ts_{x x} =  u_{ss}+u_{tt}-2u_{st}$ and $ u_{yy} = u_{ss}s_y^2+2u_{st}s_yt_y + u_{tt}t_y^2 + u_s s_{yy} + u_t t_{yy} =(-y)^{-1} (u_{ss} +u_{tt}+2u_{st}) - \frac{1}{2} (-y)^{-\frac{3}{2}}(u_s + u_t)$.
Therefore, the equation becomes
\begin{align*}
u_{ss} + u_{tt} - 2u_{st} + y(-y)^{-1} (u_{ss}+u_{tt}+2u_{st})- \frac{1}{2} y (-y)^{-\frac{3}{2}} (u_s+u_t) -\frac{1}{2\sqrt{-y} }(u_s+u_t) &= 0 \\
-4u_{st} + \left(\frac{1}{ 2 \sqrt{-y}} - \frac{1}{2\sqrt{-y} } \right)(u_s+u_t)&= 0 \\
u_{st}&= 0 \\
u_t &= g(t) \\
u(s,t) &= f(s) + g(t) 
\end{align*}
for some function $ f(s)$ and  $ g(t)$.
\begin{case}[2]
If $ y>0$, then the equation is elliptic. Suppose  $ \phi(x,y) = \phi_1(x,y) + i\phi_2(x,y) =$constant, then we have
\begin{align*}
	\frac{d y}{d x} &= \pm i \sqrt{y}  \\
	\sqrt{y} &= \pm \frac{1}{2} i x +C  
\end{align*}
Thus $ \phi_1(x,y) = 2\sqrt{y} $ and $ \phi_2(x,y) = x$. The Jacobian is invertible so let $ s = 2\sqrt{y} $ and $ t = x$. Then we have  $ s_x = t_y = 0$,  $ s_y = y^{-\frac{1}{2}} = \frac{2}{s}$, and $ t_x = 1$. Thus
 \begin{align*}
	a^*  = 0 + y y^{-1} &= 1 \\
	c^* = 1^2+0 &= 1 
\end{align*}
Moreover, $ u_y = u_s s_y+u_t t_y = \frac{2}{s} u_s$. Thus, the standard form is
\begin{align*}
	u_{ss} + u_{tt} + \frac{2}{s}u_s = 0
\end{align*}
\end{case}
\begin{case}[3]
If $ y = 0$, the equation is parabolic. The equation becomes
\begin{align*}
	u_{x x} + \frac{1}{2}u_y =0
\end{align*}
which is already in standard form.
\end{case}
\end{problem}

\begin{problem}[1.4.9a]
We see that $ a=1$, $b=2$, and  $ c=3$. Then $ d= ac -b^2 = -1$ which means the equation is hyperbolic. 
\begin{align*}
	\begin{cases}
		a^* = s_x^2 + 4 s_xs_y+3s_y^2 =0\\
		b^* = s_x t_x + 2(s_x t_y + s_y t_x) + 3 s_yt_y\\
		c^* = t_x^2 + 4 t_xt_y + 3 t_y^2 =0
	\end{cases}
\end{align*}
As before, we solve the ODE
\begin{align*}
	\left( \frac{d y}{d x}  \right)^2 + 4 \frac{d y}{d x} +3 &= 0 \\
	\frac{d y}{d x} &= 3 \text{ or }1  \\
	y = 3x+C \text{ or } y &= x+C  
\end{align*}
Thus $ \phi_1(x,y)= -3x+y = c_1$ and $ \phi_2(x,y) = -x+y= c_2$. The Jacobian is invertible so we let $ s = -3x+y$ and  $ t = -x+y$. Then  $ s_x = -3$,  $ s_y=t_y=1$, and $ t_x = -1$. Thus  $ b^* = 3+2(-3-1)+3= -2$. Moreover, $ u_x = u_s s_x + u_t t_x = -3u_x -u_t$ and  $ u_y = u_s s_y + u_t t_y = u_s + u_t$. Thus the standard form is
 \begin{align*}
	-4 u_{st} + 3(-3u_s - u_t) - (u_s+u_t) +2u &= 0 \\
	u_{st} - \frac{5}{2} u_s + u_t - \frac{1}{2} u &= 0 
\end{align*}
Now suppose that $ u = v e^{\lambda s + \mu t} =: v w$. Then
\begin{align*}
	u_s &= v_s w + \lambda vw  \\
	u_t &= v_t w + \mu vw \\
	u_{st} &= v_{st}w+ \mu v_s w + \lambda v_t w + \lambda \mu vw 
\end{align*}
To eliminate first order terms, the sum of their coefficients must be zero. That is,
\begin{align*}
	\mu -\frac{5}{2} &= 0 \\
	\lambda + 1 &= 0 
\end{align*}
So $ \mu = \frac{5}{2}$ and $ \lambda=-1$. Thus the standard form becomes
\begin{align*}
	v_{st}w -\frac{5}{2} vw + \frac{5}{2}vw + \frac{5}{2}vw- \frac{1}{2} vw &= 0 \\
	(v_{st}-2vw) w &= 0 \\
	v_{st} - 2vw &= 0 && w > 0
\end{align*}
\end{problem}

\begin{problem}[1.4.11]
We see that
\begin{align*}
	u_t &= v_t w + vw_t \\
	u_x &= v_x w + vw_x \\
	u_{x x} &= v_{ xx} w + 2v_xw_x + vw_{x x}
\end{align*}
The equation becomes
\begin{align*}
	v_t w+vw_t - v_{x x}w-2v_xw_x - vw_{x x} + a v_xw + avw_x + bvw &= f(x,t) \\
	(v_t- v_{x x})w + v(w_t-w_{x x}+aw_x+bw) + v_x(aw-2w_x) &= f(x,t) 
\end{align*}
To obtain the desired form, the following must be true:
\begin{align*}
	w_t -w_{x x}+ aw_x + bw &= 0 \\
	aw-2w_x &= 0 \\
	w &\neq 0 
\end{align*}
Thus we have $ w_x = \frac{a}{2}w$ which yields $ w=g(t) e^{\frac{a}{2}x}$. Then $ w_{x x} = \frac{a^2}{ 4} g(t) e^{\frac{a}{2}x}$. Thus
\begin{align*}
	w_t &= \left(\frac{a^2}{ 4}- \frac{a^2}{ 2} - b \right) w  \\
	w(x,t) &= \exp \left( \left(-\frac{a^2}{ 4}-b \right)t+ \frac{a}{2}x  \right)  \neq 0
\end{align*}
\end{problem}

\begin{problem}[1.4.12]
Since $ u_t = \widetilde{ u}' \cdot \left( - \frac{x}{2} \right) t^{-\frac{3}{2}} $ and $ u_{x x} = \frac{1}{t} \widetilde{ u}''$. Let $ y:= \frac{x}{\sqrt{t} }$, then the ODE is
\begin{align*}
	-\frac{x}{t^{\frac{3}{2}}} \widetilde{ u}' - \frac{a^2}{ t}\widetilde{ u}'' &= 0 \\
	-\frac{1}{t} (y \widetilde{ u}' + a^2 \widetilde{ u}'') &= 0 \\
	a^2 \widetilde{ u}'' + y \widetilde{ u}' &= 0 \\
	u(y) &= c_1 \sqrt{ \frac{\pi}{2 }} a \text{ erf}\left( \frac{y}{ \sqrt{2} a} \right) +c_2    
\end{align*}
via substitution and integration by parts. Since $ u(0,t) = \widetilde{ u}(0) = 0$, we have that $ c_2 = 0$. Since $ u(x,0)=u_0$, and $ \lim_{ t \to 0} \widetilde{ u} = c_1 \sqrt{\frac{\pi}{ 2}}a \cdot 1 $, we have  $ c_1 = \frac{u_0}{ a} \sqrt{\frac{2}{\pi}} $. Thus
\begin{align*}
	u(y) = u_0 \text{ erf}\left( \frac{y}{\sqrt{2} a} \right).  
\end{align*}
\end{problem}
\end{document}
