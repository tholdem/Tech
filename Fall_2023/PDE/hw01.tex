\documentclass[12pt]{article}
\newcommand{\alert}[1]{{\bf \color{red} [Alert:] #1}}
\newcommand{\todo}[1]{{\bf \color{orange} [TODO:] #1}}
\newcommand{\real}[1][]{\mathbb{R}^{#1}}
\newcommand{\myeqn}[1]{(\ref{#1})}
\newcommand{\myex}[1]{Example \ref{#1}}
\newcommand{\defeq}{\stackrel{\mathrm{def}}{=}}
\newcommand{\parder}[2]{\frac{\partial #1}{\partial #2}}
\newcommand{\Lie}[3][]{\mathsf{L}_{#3}^{#1} #2}
\newcommand{\LieA}[1]{\mathsf{Lie}(#1)}
\newcommand{\lieder}[2]{\mathcal{L}_{#2} #1}
\renewcommand{\t}{^{\mbox{\tiny\sf T}}}
\newcommand{\trans}{^{\mbox{\tiny\sf T}}}
\newcommand{\markup}[1]{\{\textbf{#1}\}}
\newcommand{\msub}[1]{_\mathrm{#1}}
\newcommand{\msup}[1]{^\mathrm{#1}}
\newcommand{\inv}[1]{#1^{-1}}
\newcommand{\pinv}[1]{{#1}^{+}}
\newcommand{\myfracA}[2]{\displaystyle{\frac{#1}{#2}}}
\newcommand{\myfracB}[2]{{#1}/{#2}}
\newcommand{\mydiffA}[1]{\dot{#1}}
\newcommand{\mydiffB}[2]{\myfracA{\mathrm{d}{#1}}{\mathrm{d}{#2}}}
\newcommand{\ball}[2]{\mathcal{B}_{#1}\left(#2\right)}
\newcommand{\acos}[1]{\cos^{-1}\left(#1\right)}
\newcommand{\asin}[1]{\sin^{-1}\left(#1\right)}
\newcommand{\mani}{\mathcal{M}}
\newcommand{\tang}[2]{\mathsf{T}_{#1} #2}
\newcommand{\LieB}[2]{[ #1, #2 ]}
\newcommand{\LieBad}[3][]{\mathsf{ad}_{#2}^{#1} #3}
\newcommand{\ReachVT}{\mathcal{R}^V_T}
\newcommand{\ReachVt}{\mathcal{R}^V_t}
\newcommand{\ReachVTe}{\mathcal{R}^V_{\le T}}
\newcommand{\ReachT}{\mathcal{R}_T}
\newcommand{\Reacht}{\mathcal{R}_t}
\newcommand{\ReachTe}{\mathcal{R}_{\le T}}
\newcommand{\accLA}[1]{\mathsf{Lie}(#1)}
\newcommand{\accD}{\Delta_{\mathcal{F}}}
\newcommand{\accSA}{\mathsf{Lie}(\mathcal{G},f)}
\newcommand{\accDS}{\Delta_{\mathcal{G}}}
\newcommand{\eval}[3]{\mathsf{Ev}^{#2}_{#1}\left( #3 \right)}
\newcommand{\stlc}{\textsc{stlc}}
\newcommand{\clf}{\textsc{clf}}
\newcommand{\jqlf}{\textsc{jqlf}}
\newcommand{\dlas}{\textsc{dlas}}
\newcommand{\Ad}[2]{\mathsf{Ad}_{#1} #2}
\newcommand{\xe}{\ensuremath{x_e}}
\newcommand{\lebg}[1]{\mathcal{L}_{#1}}
\newcommand{\lebgx}[1]{\mathcal{L}_{#1 \mathrm{e}}}
\newcommand{\dom}{D}
\newcommand{\domT}{[t_0,\infty) \times D}
\newcommand{\rarrow}{\rightarrow}
\renewcommand{\d}{\mathrm{d}}
\renewcommand{\Re}{\mathbb{R}}
\newcommand{\C}{\mathrm{C}}

\newcommand{\QED}{{\unskip\nobreak\hfil\penalty50\hskip2em\vadjust{}
		\nobreak\hfil$\Box$\parfillskip=0pt\finalhyphendemerits=0\par}\vspace{0.1cm}}
\newcommand{\eoEx}{{\unskip\nobreak\hfil\penalty50\hskip0em\vadjust{}
		\nobreak\hfil$\Large\Diamond$\parfillskip=0pt\finalhyphendemerits=0\par}\vspace{0.1cm}}

\newcommand{\sgn}{\ensuremath{\operatorname{sgn}}}
\newcommand{\sat}{\ensuremath{\operatorname{sat}}}

\newcommand{\half}{\frac{1}{2}}
\newcommand{\shalf}{\mbox{$\frac{1}{2}$}}
\newcommand{\marcom}[1]{\marginpar{\footnotesize #1}}
\newcommand{\der}{\mathrm{D}}
\newcommand{\e}{\mathrm{e}}
\newcommand{\dt}{\mathrm{d}t}

\newcommand{\cA}{\ensuremath{\mathcal{A}}}
\newcommand{\cB}{\ensuremath{\mathcal{B}}}
\newcommand{\cG}{\ensuremath{\mathcal{G}}}
\newcommand{\cK}{\ensuremath{\mathcal{K}}}
\newcommand{\cW}{\ensuremath{\mathcal{W}}}
\newcommand{\cZ}{\ensuremath{\mathcal{Z}}}
\newcommand{\cS}{\ensuremath{\mathcal{S}}}
\newcommand{\cD}{\ensuremath{\mathcal{D}}}
\newcommand{\cP}{\ensuremath{\mathcal{P}}}
\newcommand{\cV}{\ensuremath{\mathcal{V}}}
\newcommand{\cL}{\ensuremath{\mathcal{L}}}
\newcommand{\cN}{\ensuremath{\mathcal{N}}}
\newcommand{\cI}{\ensuremath{\mathcal{I}}}
\newcommand{\cR}{\ensuremath{\mathcal{R}}}
\newcommand{\cM}{\ensuremath{\mathcal{M}}}
\newcommand{\cC}{\ensuremath{\mathcal{C}}}
\newcommand{\cF}{\ensuremath{\mathcal{F}}}
\newcommand{\cH}{\ensuremath{\mathcal{H}}}
\newcommand{\cO}{\ensuremath{\mathcal{O}}}
\newcommand{\cX}{\ensuremath{\mathcal{X}}}
\newcommand{\cY}{\ensuremath{\mathcal{Y}}}
\newcommand{\Ci}{\ensuremath{\mathcal{C}^\infty}}
\newcommand{\ISS}{\textsc{iss}}
\newcommand{\LISS}{\textsc{liss}}
\newcommand{\GAS}{\textsc{gas}}
\newcommand{\GS}{\textsc{gs}}
\newcommand{\LES}{\textsc{les}}
\newcommand{\GUAS}{\textsc{guas}}
\newcommand{\BIBO}{\textsc{bibo}}
\newcommand{\spec}{\ensuremath{\operatorname{spec}}}
\newcommand{\spn}{\ensuremath{\operatorname{span}}}
\renewcommand{\i}{\mathrm{i\,}}

\renewcommand{\implies}{\Rightarrow}

\renewcommand{\theenumi}{$\roman{enumi})$}
\renewcommand{\labelenumi}{\theenumi}

\font\ptmten=zptmcmrm scaled 1200
\newcommand{\w}{\mbox{{\ptmten w}}}
\newcommand{\z}{\mbox{{\ptmten z}}}
\renewcommand{\Re}{\mathbb{R}}

\newcommand{\cl}{\operatorname{cl}}
\newcommand{\intr}{\operatorname{int}}
\newcommand{\rank}{\operatorname{rank}}
\newcommand{\co}{\operatorname{co}}
\newcommand{\aff}{\operatorname{aff}}

\theoremstyle{plain}
\newtheorem{theorem}{Theorem}[chapter]
\newtheorem{claim}[theorem]{Claim}
\newtheorem{corollary}[theorem]{Corollary}
\newtheorem{prop}[theorem]{Proposition}
\newtheorem{fact}[theorem]{Fact}
\newtheorem{lemma}[theorem]{Lemma}

\newtheorem{remark}{Remark}[chapter]

\theoremstyle{definition}
\newtheorem{assume}[theorem]{Assumption}
\newtheorem{defn}[theorem]{Definition}
\newtheorem{problem}[theorem]{Problem}
\newtheorem{exercise}{Exercise}
\newtheorem{example}[theorem]{Example}


\begin{document}
\centerline {\textsf{\textbf{\LARGE{Homework 1}}}}
\centerline {Jaden Wang}
\vspace{.15in}

\begin{problem}[1.4.3]
The coefficient matrix of second order terms is
\begin{align*}
	A = \begin{pmatrix} 3&-1&0\\-1&2&-1\\0&-1&3 \end{pmatrix} 
\end{align*}
Spectral theorem yields
\begin{align*}
	O^{T}AO &= \diag \{4,3,1\}, \text{ where } O = \begin{pmatrix} \frac{1}{\sqrt{3} }& -\frac{1}{\sqrt{2} }& \frac{1}{\sqrt{6} }\\-\frac{1}{\sqrt{3} } & 0 & \frac{2}{\sqrt{6} }\\ \frac{1}{\sqrt{3} }& \frac{1}{\sqrt{2} } & \frac{1}{\sqrt{6} } \end{pmatrix}    \\
	U^{-1}AU &= \diag \{1,1,1\}, \text{ where } U = \begin{pmatrix} \frac{2}{\sqrt{3} }& -\frac{\sqrt{3} }{\sqrt{2} }& \frac{1}{\sqrt{6} }\\-\frac{2}{\sqrt{3} } & 0 & \frac{2}{\sqrt{6} }\\ \frac{2}{\sqrt{3} }& \frac{\sqrt{3} }{\sqrt{2} } & \frac{1}{\sqrt{6} } \end{pmatrix}  \\
\end{align*}
Thus by a change of basis via $ U$, under the change of variables  $ \begin{pmatrix} r\\s\\t \end{pmatrix}  =U^{-1} \begin{pmatrix} x\\y\\z \end{pmatrix}  $, where $ U^{-1} = \begin{pmatrix} \frac{\sqrt{3} }{6 }& - \frac{\sqrt{3} }{ 6}& \frac{\sqrt{3} }{ 6} \\ - \frac{\sqrt{6} }{ 6}&0& \frac{\sqrt{6} }{ 6}\\ \frac{\sqrt{6} }{ 6} & \frac{\sqrt{6} }{ 3} & \frac{\sqrt{6} }{ 6}  \end{pmatrix} $.

By the chain rule, the derivative of $ u$ under basis  $ u_x, u_y, u_z$ is $ Du$ under basis  $ u_r,u_s,u_t$ composed with Jacobian of this change of variable, which is evidently $ U^{-1}$. That is
\begin{align*}
	\begin{pmatrix} u_x &u_y&u_z \end{pmatrix} &= U^{-1} \begin{pmatrix} u_r&u_s&u_t \end{pmatrix}  \\
	u_x &= \frac{\sqrt{3} }{ 6} u_r  -\frac{\sqrt{3} }{ 6}  u_s  + \frac{\sqrt{3} }{ 6}  u_t \\
	u_y&= - \frac{\sqrt{6} }{ 6}u_r + \frac{\sqrt{6} }{ 6} u_t
\end{align*}
Thus, the standard form is
\begin{align*}
	u_{rr} + u_{ss} + u_{tt} - \frac{5 \sqrt{6}+ \sqrt{3}  }{6 }u_r + \frac{\sqrt{3} }{ 6}u_s + \frac{5\sqrt{6}-\sqrt{3}  } { 6}u_t +10 u\left( U (r,s,t) \right) =0
\end{align*}
\end{problem}

\begin{problem}[1.4.7]
We see that $ a=1,b=0,c=y$, so  $ d= ac - b^2 = y$. 
\begin{case}[1]
If $ y<0$, then we have the hyperbolic case.  Let $ s = s(x,y)$ and  $ t = t(x,y)$  s.t.\ the Jacobian of this change of variable is invertible. Then the linear principal part becomes $ L_0u = a^* u_{ss}+ 2b^* u_{st} +c^* u_{tt} $, where
\end{case}
\begin{align*}
	\begin{cases}
		a^* = s_x^2+ y s_y^2\\
		b^* = s_x t_x + y s_yt_y \\
		c^* = t_x^2+yt_y^2 \\
	\end{cases}
\end{align*}
To convert to standard form, we want $ a^* =c^* =0$ and $ b^* \neq 0$, which indeed makes the equation hyperbolic. Then $ s,t$ has to satisfy the following equation
 \begin{align*}
	\phi_x^2 + y \phi_y^2 =0
\end{align*}
Suppose $ \phi(x,y)=$constant, \emph{i.e.} a level set, then $ y$ can locally be expressed as a function of  $ x$ (or the other way around).  Thus if $ \phi_y \neq 0$, by implicit differentiation we have
\begin{align*}
	D\phi = \phi_x + \phi_y \frac{d y}{d x} &= 0 \\
	\frac{d y}{d x} &= - \frac{\phi_x}{ \phi_y} 
\end{align*}
Thus we have the following ODE:
\begin{align*}
	\left( \frac{d y}{d x}  \right) ^2+y &= 0 \\
\frac{dy}{ \sqrt{-y} }&= \pm dx \\
2\sqrt{-y}  &= \mp x +C 
\end{align*}
Thus we have
\begin{align*}
	\phi_1(x,y) = x + 2\sqrt{-y} = c_1 , \ \phi_2(x,y) = - x + 2\sqrt{-y} = c_2. 
\end{align*}
The Jacobian of the change of variable is invertible, so we let $ s=\phi_1$ and $ t = \phi_2$. Then $ s_x =1$, $ s_y = t_y = -(-y) ^{-\frac{1}{2}}$, $ t_x = -1, s_{yy} = t_{yy} = -\frac{1}{2} (-y)^{-\frac{3}{2}} $. Then $ u_x = u_s s_x + u_t t_x = u_s - u_t$ and $ u_y = u_s s_y + u_t t_y = - (-y)^{-\frac{1}{2}}(u_s + u_t)$. $ u_{x x} = u_{ss} s_x^2 +2u_{st} s_xt_x +u_s s_{xx} + u_{tt}t_x^2 +u_ts_{x x} =  u_{ss}+u_{tt}-2u_{st}$ and $ u_{yy} = u_{ss}s_y^2+2u_{st}s_yt_y + u_{tt}t_y^2 + u_s s_{yy} + u_t t_{yy} =(-y)^{-1} (u_{ss} +u_{tt}+2u_{st}) - \frac{1}{2} (-y)^{-\frac{3}{2}}(u_s + u_t)$.
Therefore, the equation becomes
\begin{align*}
u_{ss} + u_{tt} - 2u_{st} + y(-y)^{-1} (u_{ss}+u_{tt}+2u_{st})- \frac{1}{2} y (-y)^{-\frac{3}{2}} (u_s+u_t) -\frac{1}{2\sqrt{-y} }(u_s+u_t) &= 0 \\
-4u_{st} + \left(\frac{1}{ 2 \sqrt{-y}} - \frac{1}{2\sqrt{-y} } \right)(u_s+u_t)&= 0 \\
u_{st}&= 0 \\
u_t &= g(t) \\
u(s,t) &= f(s) + g(t) 
\end{align*}
for some function $ f(s)$ and  $ g(t)$.
\begin{case}[2]
If $ y>0$, then the equation is elliptic. Suppose  $ \phi(x,y) = \phi_1(x,y) + i\phi_2(x,y) =$constant, then we have
\begin{align*}
	\frac{d y}{d x} &= \pm i \sqrt{y}  \\
	\sqrt{y} &= \pm \frac{1}{2} i x +C  
\end{align*}
Thus $ \phi_1(x,y) = 2\sqrt{y} $ and $ \phi_2(x,y) = x$. The Jacobian is invertible so let $ s = 2\sqrt{y} $ and $ t = x$. Then we have  $ s_x = t_y = 0$,  $ s_y = y^{-\frac{1}{2}} = \frac{2}{s}$, and $ t_x = 1$. Thus
 \begin{align*}
	a^*  = 0 + y y^{-1} &= 1 \\
	c^* = 1^2+0 &= 1 
\end{align*}
Moreover, $ u_y = u_s s_y+u_t t_y = \frac{2}{s} u_s$. Thus, the standard form is
\begin{align*}
	u_{ss} + u_{tt} + \frac{2}{s}u_s = 0
\end{align*}
\end{case}
\begin{case}[3]
If $ y = 0$, the equation is parabolic. The equation becomes
\begin{align*}
	u_{x x} + \frac{1}{2}u_y =0
\end{align*}
which is already in standard form.
\end{case}
\end{problem}

\begin{problem}[1.4.9a]
We see that $ a=1$, $b=2$, and  $ c=3$. Then $ d= ac -b^2 = -1$ which means the equation is hyperbolic. 
\begin{align*}
	\begin{cases}
		a^* = s_x^2 + 4 s_xs_y+3s_y^2 =0\\
		b^* = s_x t_x + 2(s_x t_y + s_y t_x) + 3 s_yt_y\\
		c^* = t_x^2 + 4 t_xt_y + 3 t_y^2 =0
	\end{cases}
\end{align*}
As before, we solve the ODE
\begin{align*}
	\left( \frac{d y}{d x}  \right)^2 + 4 \frac{d y}{d x} +3 &= 0 \\
	\frac{d y}{d x} &= 3 \text{ or }1  \\
	y = 3x+C \text{ or } y &= x+C  
\end{align*}
Thus $ \phi_1(x,y)= -3x+y = c_1$ and $ \phi_2(x,y) = -x+y= c_2$. The Jacobian is invertible so we let $ s = -3x+y$ and  $ t = -x+y$. Then  $ s_x = -3$,  $ s_y=t_y=1$, and $ t_x = -1$. Thus  $ b^* = 3+2(-3-1)+3= -2$. Moreover, $ u_x = u_s s_x + u_t t_x = -3u_x -u_t$ and  $ u_y = u_s s_y + u_t t_y = u_s + u_t$. Thus the standard form is
 \begin{align*}
	-4 u_{st} + 3(-3u_s - u_t) - (u_s+u_t) +2u &= 0 \\
	u_{st} - \frac{5}{2} u_s + u_t - \frac{1}{2} u &= 0 
\end{align*}
Now suppose that $ u = v e^{\lambda s + \mu t} =: v w$. Then
\begin{align*}
	u_s &= v_s w + \lambda vw  \\
	u_t &= v_t w + \mu vw \\
	u_{st} &= v_{st}w+ \mu v_s w + \lambda v_t w + \lambda \mu vw 
\end{align*}
To eliminate first order terms, the sum of their coefficients must be zero. That is,
\begin{align*}
	\mu -\frac{5}{2} &= 0 \\
	\lambda + 1 &= 0 
\end{align*}
So $ \mu = \frac{5}{2}$ and $ \lambda=-1$. Thus the standard form becomes
\begin{align*}
	v_{st}w -\frac{5}{2} vw + \frac{5}{2}vw + \frac{5}{2}vw- \frac{1}{2} vw &= 0 \\
	(v_{st}-2vw) w &= 0 \\
	v_{st} - 2vw &= 0 && w > 0
\end{align*}
\end{problem}

\begin{problem}[1.4.11]
We see that
\begin{align*}
	u_t &= v_t w + vw_t \\
	u_x &= v_x w + vw_x \\
	u_{x x} &= v_{ xx} w + 2v_xw_x + vw_{x x}
\end{align*}
The equation becomes
\begin{align*}
	v_t w+vw_t - v_{x x}w-2v_xw_x - vw_{x x} + a v_xw + avw_x + bvw &= f(x,t) \\
	(v_t- v_{x x})w + v(w_t-w_{x x}+aw_x+bw) + v_x(aw-2w_x) &= f(x,t) 
\end{align*}
To obtain the desired form, the following must be true:
\begin{align*}
	w_t -w_{x x}+ aw_x + bw &= 0 \\
	aw-2w_x &= 0 \\
	w &\neq 0 
\end{align*}
Thus we have $ w_x = \frac{a}{2}w$ which yields $ w=g(t) e^{\frac{a}{2}x}$. Then $ w_{x x} = \frac{a^2}{ 4} g(t) e^{\frac{a}{2}x}$. Thus
\begin{align*}
	w_t &= \left(\frac{a^2}{ 4}- \frac{a^2}{ 2} - b \right) w  \\
	w(x,t) &= \exp \left( \left(-\frac{a^2}{ 4}-b \right)t+ \frac{a}{2}x  \right)  \neq 0
\end{align*}
\end{problem}

\begin{problem}[1.4.12]
Since $ u_t = \widetilde{ u}' \cdot \left( - \frac{x}{2} \right) t^{-\frac{3}{2}} $ and $ u_{x x} = \frac{1}{t} \widetilde{ u}''$. Let $ y:= \frac{x}{\sqrt{t} }$, then the ODE is
\begin{align*}
	-\frac{x}{t^{\frac{3}{2}}} \widetilde{ u}' - \frac{a^2}{ t}\widetilde{ u}'' &= 0 \\
	-\frac{1}{t} (y \widetilde{ u}' + a^2 \widetilde{ u}'') &= 0 \\
	a^2 \widetilde{ u}'' + y \widetilde{ u}' &= 0 \\
	u(y) &= c_1 \sqrt{ \frac{\pi}{2 }} a \text{ erf}\left( \frac{y}{ \sqrt{2} a} \right) +c_2    
\end{align*}
via substitution and integration by parts. Since $ u(0,t) = \widetilde{ u}(0) = 0$, we have that $ c_2 = 0$. Since $ u(x,0)=u_0$, and $ \lim_{ t \to 0} \widetilde{ u} = c_1 \sqrt{\frac{\pi}{ 2}}a \cdot 1 $, we have  $ c_1 = \frac{u_0}{ a} \sqrt{\frac{2}{\pi}} $. Thus
\begin{align*}
	u(y) = u_0 \text{ erf}\left( \frac{y}{\sqrt{2} a} \right).  
\end{align*}
\end{problem}
\end{document}
