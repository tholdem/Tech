\documentclass[12pt]{article}
\newcommand{\alert}[1]{{\bf \color{red} [Alert:] #1}}
\newcommand{\todo}[1]{{\bf \color{orange} [TODO:] #1}}
\newcommand{\real}[1][]{\mathbb{R}^{#1}}
\newcommand{\myeqn}[1]{(\ref{#1})}
\newcommand{\myex}[1]{Example \ref{#1}}
\newcommand{\defeq}{\stackrel{\mathrm{def}}{=}}
\newcommand{\parder}[2]{\frac{\partial #1}{\partial #2}}
\newcommand{\Lie}[3][]{\mathsf{L}_{#3}^{#1} #2}
\newcommand{\LieA}[1]{\mathsf{Lie}(#1)}
\newcommand{\lieder}[2]{\mathcal{L}_{#2} #1}
\renewcommand{\t}{^{\mbox{\tiny\sf T}}}
\newcommand{\trans}{^{\mbox{\tiny\sf T}}}
\newcommand{\markup}[1]{\{\textbf{#1}\}}
\newcommand{\msub}[1]{_\mathrm{#1}}
\newcommand{\msup}[1]{^\mathrm{#1}}
\newcommand{\inv}[1]{#1^{-1}}
\newcommand{\pinv}[1]{{#1}^{+}}
\newcommand{\myfracA}[2]{\displaystyle{\frac{#1}{#2}}}
\newcommand{\myfracB}[2]{{#1}/{#2}}
\newcommand{\mydiffA}[1]{\dot{#1}}
\newcommand{\mydiffB}[2]{\myfracA{\mathrm{d}{#1}}{\mathrm{d}{#2}}}
\newcommand{\ball}[2]{\mathcal{B}_{#1}\left(#2\right)}
\newcommand{\acos}[1]{\cos^{-1}\left(#1\right)}
\newcommand{\asin}[1]{\sin^{-1}\left(#1\right)}
\newcommand{\mani}{\mathcal{M}}
\newcommand{\tang}[2]{\mathsf{T}_{#1} #2}
\newcommand{\LieB}[2]{[ #1, #2 ]}
\newcommand{\LieBad}[3][]{\mathsf{ad}_{#2}^{#1} #3}
\newcommand{\ReachVT}{\mathcal{R}^V_T}
\newcommand{\ReachVt}{\mathcal{R}^V_t}
\newcommand{\ReachVTe}{\mathcal{R}^V_{\le T}}
\newcommand{\ReachT}{\mathcal{R}_T}
\newcommand{\Reacht}{\mathcal{R}_t}
\newcommand{\ReachTe}{\mathcal{R}_{\le T}}
\newcommand{\accLA}[1]{\mathsf{Lie}(#1)}
\newcommand{\accD}{\Delta_{\mathcal{F}}}
\newcommand{\accSA}{\mathsf{Lie}(\mathcal{G},f)}
\newcommand{\accDS}{\Delta_{\mathcal{G}}}
\newcommand{\eval}[3]{\mathsf{Ev}^{#2}_{#1}\left( #3 \right)}
\newcommand{\stlc}{\textsc{stlc}}
\newcommand{\clf}{\textsc{clf}}
\newcommand{\jqlf}{\textsc{jqlf}}
\newcommand{\dlas}{\textsc{dlas}}
\newcommand{\Ad}[2]{\mathsf{Ad}_{#1} #2}
\newcommand{\xe}{\ensuremath{x_e}}
\newcommand{\lebg}[1]{\mathcal{L}_{#1}}
\newcommand{\lebgx}[1]{\mathcal{L}_{#1 \mathrm{e}}}
\newcommand{\dom}{D}
\newcommand{\domT}{[t_0,\infty) \times D}
\newcommand{\rarrow}{\rightarrow}
\renewcommand{\d}{\mathrm{d}}
\renewcommand{\Re}{\mathbb{R}}
\newcommand{\C}{\mathrm{C}}

\newcommand{\QED}{{\unskip\nobreak\hfil\penalty50\hskip2em\vadjust{}
		\nobreak\hfil$\Box$\parfillskip=0pt\finalhyphendemerits=0\par}\vspace{0.1cm}}
\newcommand{\eoEx}{{\unskip\nobreak\hfil\penalty50\hskip0em\vadjust{}
		\nobreak\hfil$\Large\Diamond$\parfillskip=0pt\finalhyphendemerits=0\par}\vspace{0.1cm}}

\newcommand{\sgn}{\ensuremath{\operatorname{sgn}}}
\newcommand{\sat}{\ensuremath{\operatorname{sat}}}

\newcommand{\half}{\frac{1}{2}}
\newcommand{\shalf}{\mbox{$\frac{1}{2}$}}
\newcommand{\marcom}[1]{\marginpar{\footnotesize #1}}
\newcommand{\der}{\mathrm{D}}
\newcommand{\e}{\mathrm{e}}
\newcommand{\dt}{\mathrm{d}t}

\newcommand{\cA}{\ensuremath{\mathcal{A}}}
\newcommand{\cB}{\ensuremath{\mathcal{B}}}
\newcommand{\cG}{\ensuremath{\mathcal{G}}}
\newcommand{\cK}{\ensuremath{\mathcal{K}}}
\newcommand{\cW}{\ensuremath{\mathcal{W}}}
\newcommand{\cZ}{\ensuremath{\mathcal{Z}}}
\newcommand{\cS}{\ensuremath{\mathcal{S}}}
\newcommand{\cD}{\ensuremath{\mathcal{D}}}
\newcommand{\cP}{\ensuremath{\mathcal{P}}}
\newcommand{\cV}{\ensuremath{\mathcal{V}}}
\newcommand{\cL}{\ensuremath{\mathcal{L}}}
\newcommand{\cN}{\ensuremath{\mathcal{N}}}
\newcommand{\cI}{\ensuremath{\mathcal{I}}}
\newcommand{\cR}{\ensuremath{\mathcal{R}}}
\newcommand{\cM}{\ensuremath{\mathcal{M}}}
\newcommand{\cC}{\ensuremath{\mathcal{C}}}
\newcommand{\cF}{\ensuremath{\mathcal{F}}}
\newcommand{\cH}{\ensuremath{\mathcal{H}}}
\newcommand{\cO}{\ensuremath{\mathcal{O}}}
\newcommand{\cX}{\ensuremath{\mathcal{X}}}
\newcommand{\cY}{\ensuremath{\mathcal{Y}}}
\newcommand{\Ci}{\ensuremath{\mathcal{C}^\infty}}
\newcommand{\ISS}{\textsc{iss}}
\newcommand{\LISS}{\textsc{liss}}
\newcommand{\GAS}{\textsc{gas}}
\newcommand{\GS}{\textsc{gs}}
\newcommand{\LES}{\textsc{les}}
\newcommand{\GUAS}{\textsc{guas}}
\newcommand{\BIBO}{\textsc{bibo}}
\newcommand{\spec}{\ensuremath{\operatorname{spec}}}
\newcommand{\spn}{\ensuremath{\operatorname{span}}}
\renewcommand{\i}{\mathrm{i\,}}

\renewcommand{\implies}{\Rightarrow}

\renewcommand{\theenumi}{$\roman{enumi})$}
\renewcommand{\labelenumi}{\theenumi}

\font\ptmten=zptmcmrm scaled 1200
\newcommand{\w}{\mbox{{\ptmten w}}}
\newcommand{\z}{\mbox{{\ptmten z}}}
\renewcommand{\Re}{\mathbb{R}}

\newcommand{\cl}{\operatorname{cl}}
\newcommand{\intr}{\operatorname{int}}
\newcommand{\rank}{\operatorname{rank}}
\newcommand{\co}{\operatorname{co}}
\newcommand{\aff}{\operatorname{aff}}

\theoremstyle{plain}
\newtheorem{theorem}{Theorem}[chapter]
\newtheorem{claim}[theorem]{Claim}
\newtheorem{corollary}[theorem]{Corollary}
\newtheorem{prop}[theorem]{Proposition}
\newtheorem{fact}[theorem]{Fact}
\newtheorem{lemma}[theorem]{Lemma}

\newtheorem{remark}{Remark}[chapter]

\theoremstyle{definition}
\newtheorem{assume}[theorem]{Assumption}
\newtheorem{defn}[theorem]{Definition}
\newtheorem{problem}[theorem]{Problem}
\newtheorem{exercise}{Exercise}
\newtheorem{example}[theorem]{Example}


\begin{document}
\centerline {\textsf{\textbf{\LARGE{Homework }}}}
\centerline {Jaden Wang}
\vspace{.15in}

We first write down the Lagrangian of the system:
\begin{align*}
	T &= \frac{1}{2}m \left( \left( L \dot{\theta} \right)^2 + \left( L\sin \theta \dot{\phi} \right)^2   \right)  \\
	U &= mg L \cos \theta \\
	\mathcal{L} &= T - U \\
	&= \frac{mL^2}{2} \left( \dot{\theta}^2 + \left( \sin \theta \dot{\phi} \right)^2   \right)  - mg L \cos \theta
\end{align*}
Now we determine the generalized momenta and express $ \dot{ \ve{ q} }$ in terms of them:
\begin{align*}
	p_{\theta} &= \frac{\partial \mathcal{ L}}{\partial \dot{\theta}} = mL^2 \dot{\theta}\\
	\dot{\theta} &= \frac{p_{\theta}}{ mL^2}\\
	p_{\phi} &= \frac{\partial \mathcal{ L}}{\partial \dot{\phi}} = mL^2 (\sin \theta)^2 \dot{\phi} \\
	\dot{\phi} &= \frac{p_{\phi}}{ mL^2 (\sin \theta)^2} 
\end{align*}
The Hamiltonian is given by
\begin{align*}
	\mathcal{ H}(\theta,p_{\theta},\phi,p_{\phi}) &= p_{\theta} \dot{\theta} + p_{\phi} \dot{\phi} - \mathcal{ L} \\
	&= \frac{p_{\theta}^2}{ mL^2} + \frac{p_{\phi}^2}{ mL^2 (\sin \theta)^2} - \frac{mL^2}{ 2} \left( \frac{p_{\theta}^2}{ (mL^2)^2} + \frac{p_{\phi}^2}{ (mL^2 \sin \theta)^2 } \right) + mgL \cos \theta\\
	&= \frac{1}{2mL^2} \left( p_{\theta}^2 + \frac{p_{\phi}^2}{(\sin \theta)^2 } \right) + mgL \cos \theta \\
	&= T+U 
\end{align*}

Since $ \phi$ doesn't not explicitly show up in $ \mathcal{ H}$, it is an ignorable variable, \emph{i.e.}  $ \frac{\partial \mathcal{ H}}{\partial \phi} =0$. Hamilton's equations immediately tell us that $ p_{\phi}$ is a constant. Thus we reduce the degrees of freedom of the system by 1. This contrasts the Lagrangian approach where $ \dot{q}$ can remain a variable and prevent us from reducing the degrees of freedom.

Thus the Hamiltonian becomes
\begin{align*}
	\mathcal{ H}(\theta, p_{\theta}) = \frac{p_{\theta}^2}{ 2 \mu} + U_{eff}(\theta),
\end{align*}
where $ \mu = mL^2$ and $ U_{eff}(\theta) = \frac{p_{\phi}^2}{2\mu (\sin\theta)^2 } + mgL\cos \theta$ is the effective potential energy that solely depends on $ \theta$.

The equations of motions are
\begin{align*}
	\dot{\theta} &= \frac{p_{\theta}}{ mL^2}\\
	\dot{p_{\theta}} &= -\frac{\partial \mathcal{ H}}{\partial \theta} = \frac{\cos \theta p_{\phi}^2}{mL^2 (\sin \theta)^3} + mgL\sin \theta 
\end{align*}

The equilibrium points of the system locate at the minima (stable) or maxima (unstable) of the effective potential energy. 
~\begin{figure}[H]
	\centering
	\includegraphics[width=0.49\textwidth]{./figures/}
	\includegraphics[width=0.49\textwidth]{./figures/}
	\caption{We see that there is a unique minimum of $ U_{eff}$ at around $ \theta = 2.7$. In the plot of phase space orbits, we see that this is indeed the (stable) equilibrium point of the system.}
\end{figure}
\end{document}
