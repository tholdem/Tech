\documentclass[12pt]{article}
\newcommand{\alert}[1]{{\bf \color{red} [Alert:] #1}}
\newcommand{\todo}[1]{{\bf \color{orange} [TODO:] #1}}
\newcommand{\real}[1][]{\mathbb{R}^{#1}}
\newcommand{\myeqn}[1]{(\ref{#1})}
\newcommand{\myex}[1]{Example \ref{#1}}
\newcommand{\defeq}{\stackrel{\mathrm{def}}{=}}
\newcommand{\parder}[2]{\frac{\partial #1}{\partial #2}}
\newcommand{\Lie}[3][]{\mathsf{L}_{#3}^{#1} #2}
\newcommand{\LieA}[1]{\mathsf{Lie}(#1)}
\newcommand{\lieder}[2]{\mathcal{L}_{#2} #1}
\renewcommand{\t}{^{\mbox{\tiny\sf T}}}
\newcommand{\trans}{^{\mbox{\tiny\sf T}}}
\newcommand{\markup}[1]{\{\textbf{#1}\}}
\newcommand{\msub}[1]{_\mathrm{#1}}
\newcommand{\msup}[1]{^\mathrm{#1}}
\newcommand{\inv}[1]{#1^{-1}}
\newcommand{\pinv}[1]{{#1}^{+}}
\newcommand{\myfracA}[2]{\displaystyle{\frac{#1}{#2}}}
\newcommand{\myfracB}[2]{{#1}/{#2}}
\newcommand{\mydiffA}[1]{\dot{#1}}
\newcommand{\mydiffB}[2]{\myfracA{\mathrm{d}{#1}}{\mathrm{d}{#2}}}
\newcommand{\ball}[2]{\mathcal{B}_{#1}\left(#2\right)}
\newcommand{\acos}[1]{\cos^{-1}\left(#1\right)}
\newcommand{\asin}[1]{\sin^{-1}\left(#1\right)}
\newcommand{\mani}{\mathcal{M}}
\newcommand{\tang}[2]{\mathsf{T}_{#1} #2}
\newcommand{\LieB}[2]{[ #1, #2 ]}
\newcommand{\LieBad}[3][]{\mathsf{ad}_{#2}^{#1} #3}
\newcommand{\ReachVT}{\mathcal{R}^V_T}
\newcommand{\ReachVt}{\mathcal{R}^V_t}
\newcommand{\ReachVTe}{\mathcal{R}^V_{\le T}}
\newcommand{\ReachT}{\mathcal{R}_T}
\newcommand{\Reacht}{\mathcal{R}_t}
\newcommand{\ReachTe}{\mathcal{R}_{\le T}}
\newcommand{\accLA}[1]{\mathsf{Lie}(#1)}
\newcommand{\accD}{\Delta_{\mathcal{F}}}
\newcommand{\accSA}{\mathsf{Lie}(\mathcal{G},f)}
\newcommand{\accDS}{\Delta_{\mathcal{G}}}
\newcommand{\eval}[3]{\mathsf{Ev}^{#2}_{#1}\left( #3 \right)}
\newcommand{\stlc}{\textsc{stlc}}
\newcommand{\clf}{\textsc{clf}}
\newcommand{\jqlf}{\textsc{jqlf}}
\newcommand{\dlas}{\textsc{dlas}}
\newcommand{\Ad}[2]{\mathsf{Ad}_{#1} #2}
\newcommand{\xe}{\ensuremath{x_e}}
\newcommand{\lebg}[1]{\mathcal{L}_{#1}}
\newcommand{\lebgx}[1]{\mathcal{L}_{#1 \mathrm{e}}}
\newcommand{\dom}{D}
\newcommand{\domT}{[t_0,\infty) \times D}
\newcommand{\rarrow}{\rightarrow}
\renewcommand{\d}{\mathrm{d}}
\renewcommand{\Re}{\mathbb{R}}
\newcommand{\C}{\mathrm{C}}

\newcommand{\QED}{{\unskip\nobreak\hfil\penalty50\hskip2em\vadjust{}
		\nobreak\hfil$\Box$\parfillskip=0pt\finalhyphendemerits=0\par}\vspace{0.1cm}}
\newcommand{\eoEx}{{\unskip\nobreak\hfil\penalty50\hskip0em\vadjust{}
		\nobreak\hfil$\Large\Diamond$\parfillskip=0pt\finalhyphendemerits=0\par}\vspace{0.1cm}}

\newcommand{\sgn}{\ensuremath{\operatorname{sgn}}}
\newcommand{\sat}{\ensuremath{\operatorname{sat}}}

\newcommand{\half}{\frac{1}{2}}
\newcommand{\shalf}{\mbox{$\frac{1}{2}$}}
\newcommand{\marcom}[1]{\marginpar{\footnotesize #1}}
\newcommand{\der}{\mathrm{D}}
\newcommand{\e}{\mathrm{e}}
\newcommand{\dt}{\mathrm{d}t}

\newcommand{\cA}{\ensuremath{\mathcal{A}}}
\newcommand{\cB}{\ensuremath{\mathcal{B}}}
\newcommand{\cG}{\ensuremath{\mathcal{G}}}
\newcommand{\cK}{\ensuremath{\mathcal{K}}}
\newcommand{\cW}{\ensuremath{\mathcal{W}}}
\newcommand{\cZ}{\ensuremath{\mathcal{Z}}}
\newcommand{\cS}{\ensuremath{\mathcal{S}}}
\newcommand{\cD}{\ensuremath{\mathcal{D}}}
\newcommand{\cP}{\ensuremath{\mathcal{P}}}
\newcommand{\cV}{\ensuremath{\mathcal{V}}}
\newcommand{\cL}{\ensuremath{\mathcal{L}}}
\newcommand{\cN}{\ensuremath{\mathcal{N}}}
\newcommand{\cI}{\ensuremath{\mathcal{I}}}
\newcommand{\cR}{\ensuremath{\mathcal{R}}}
\newcommand{\cM}{\ensuremath{\mathcal{M}}}
\newcommand{\cC}{\ensuremath{\mathcal{C}}}
\newcommand{\cF}{\ensuremath{\mathcal{F}}}
\newcommand{\cH}{\ensuremath{\mathcal{H}}}
\newcommand{\cO}{\ensuremath{\mathcal{O}}}
\newcommand{\cX}{\ensuremath{\mathcal{X}}}
\newcommand{\cY}{\ensuremath{\mathcal{Y}}}
\newcommand{\Ci}{\ensuremath{\mathcal{C}^\infty}}
\newcommand{\ISS}{\textsc{iss}}
\newcommand{\LISS}{\textsc{liss}}
\newcommand{\GAS}{\textsc{gas}}
\newcommand{\GS}{\textsc{gs}}
\newcommand{\LES}{\textsc{les}}
\newcommand{\GUAS}{\textsc{guas}}
\newcommand{\BIBO}{\textsc{bibo}}
\newcommand{\spec}{\ensuremath{\operatorname{spec}}}
\newcommand{\spn}{\ensuremath{\operatorname{span}}}
\renewcommand{\i}{\mathrm{i\,}}

\renewcommand{\implies}{\Rightarrow}

\renewcommand{\theenumi}{$\roman{enumi})$}
\renewcommand{\labelenumi}{\theenumi}

\font\ptmten=zptmcmrm scaled 1200
\newcommand{\w}{\mbox{{\ptmten w}}}
\newcommand{\z}{\mbox{{\ptmten z}}}
\renewcommand{\Re}{\mathbb{R}}

\newcommand{\cl}{\operatorname{cl}}
\newcommand{\intr}{\operatorname{int}}
\newcommand{\rank}{\operatorname{rank}}
\newcommand{\co}{\operatorname{co}}
\newcommand{\aff}{\operatorname{aff}}

\theoremstyle{plain}
\newtheorem{theorem}{Theorem}[chapter]
\newtheorem{claim}[theorem]{Claim}
\newtheorem{corollary}[theorem]{Corollary}
\newtheorem{prop}[theorem]{Proposition}
\newtheorem{fact}[theorem]{Fact}
\newtheorem{lemma}[theorem]{Lemma}

\newtheorem{remark}{Remark}[chapter]

\theoremstyle{definition}
\newtheorem{assume}[theorem]{Assumption}
\newtheorem{defn}[theorem]{Definition}
\newtheorem{problem}[theorem]{Problem}
\newtheorem{exercise}{Exercise}
\newtheorem{example}[theorem]{Example}


\begin{document}
\centerline {\textsf{\textbf{\LARGE{Homework 5}}}}
\centerline {Jaden Wang}
\vspace{.15in}
In this homework, all derivatives are differentiated with respect to the inertial frame $ I$ unless specified otherwise. This is denoted as a subscript on the derivative. Subscripts can also indicate the reference points and should be clear based on context. Superscripts on vectors usually indicate the coordinate system, except that for position vectors, they denote the start and end of the vector.
\begin{problem}[1]
Let $ C$ be the center of mass. By Euler's first law, we have the equation of motion for translation:
\begin{align}
	\ve{ F}  = \frac{d \ve{ P} }{d t} = m \ve{ \dot{v}}_{C}.
\end{align}
Thus using inertial coordinates, at any given time we have
\begin{align}
	\ve{ \dot{v}}^{I} = \frac{C \ve{ F}^{B} }{m }  ,
\end{align}
where $ C$ is the DCM at that time.

By Euler's second law, in an inertia frame with a fixed reference point $ O$ we have
\begin{align}
	\ve{ M}_O = \ve{ \dot{L}}_O,
\end{align}
where $ \ve{ M}_O$ is the net moment about $ O$. By definition of moment and angular momentum, we can write
\begin{align}
	\ve{M}_O &= \ve{ M}_{C} + \ve{ r}^{OC} \times \ve{ F} \\
	\ve{ L}_O &= \ve{ L}_C + \ve{ r}^{OC} \times \ve{ P}  
\end{align}
Note that $ \ve{ F} $ and $ \ve{ P} $ do not depend on reference points. Equations 3-5 yield
\begin{align}
	\ve{ M}_C + \ve{ r}^{OC} \times \ve{ F}  &= \frac{d }{d t} \left( \ve{ L}_C + \ve{ r}^{OC} \times \ve{ P}  \right)   \\ 
	&= \ve{ \dot{L}}_C + \ve{ \dot{r}}^{OC} \times \ve{ P}  + \ve{ r}^{OC} \times \ve{ F}  \\
	&= \ve{ \dot{L}}_C + \ve{ v}_C \times m\ve{ v}_C  + \ve{ r}^{OC} \times \ve{ F}  \\
	\ve{ M}_C &= \ve{ \dot{L}}_C  ,
\end{align}
where we used Equation 1. Since $ \ve{ L}_C = \mathcal{ I}_C \ve{ \omega} $ (see Equation 30), it is much more convenient to use the body frame $ B$ so that the moment of inertia $ \mathcal{ I}_C$ remains constant. Since
\begin{align}
	\left(\frac{d \ve{ q} }{d t}\right)_I = \left( \frac{d \ve{ q} }{d t}  \right)_B + \ve{ \omega} \times \ve{ q}
\end{align}
for any vector $ \ve{ q} $, we have
\begin{align}
	\ve{ M}_C &= \left( \frac{d \ve{ L}_C }{d t}  \right)_B + \ve{ \omega} \times \ve{ L}_C  \\  
		&= \left( \frac{d (\mathcal{ I}_C \ve{\omega})}{d t}  \right)_B + \ve{ \omega} \times ( \mathcal{ I}_C \ve{ \omega} )   \\
	&= \mathcal{ I}_C \ve{ \dot{\omega}}_{B}+ \ve{ \omega} \times ( \mathcal{ I}_C \ve{ \omega} )\\
	&= \mathcal{ I}_C \ve{ \dot{\omega}}_{I} + \ve{ \omega} \times ( \mathcal{ I}_C \ve{ \omega} ) 
\end{align}
where we use the fact that $ \ve{ \dot{\omega}} _{I} = \ve{ \dot{\omega}}_{B}  + \ve{ \omega} \times \ve{ \omega} = \ve{ \dot{\omega}}_{B}   $ by Equation 10. For this problem, we obtain
\begin{align}
	\ve{ \dot{\omega}}^{B} = \mathcal{ I}^{-1} \left(\ve{ r}^{CP} \times \ve{ F}^{B} - \ve{ \omega}^{B} \times ( \mathcal{ I} \ve{ \omega}^{B} ) \right)
\end{align}
where $ \ve{ r}^{CP} $ and $ \mathcal{ I}$ are always in body coordinates.

\end{problem}

\begin{problem}[2]
We obtain
\begin{align*}
	\ve{ \dot{v}}^{I} = \begin{pmatrix} 0.3436\\1.8103\\0.4821 \end{pmatrix}  , \quad \ve{ \dot{\omega}}^{B} = \begin{pmatrix} -67.5008\\18.5284\\ -4.0689  \end{pmatrix}  
\end{align*}
\end{problem}

\begin{problem}[3]
Now we rederive the equations of motions using quantities associated with $ P$ instead of the center of mass. Since $ P$ and center of mass are both fixed in the body frame, by using the two-point formula for acceleration, we obtain
\begin{align}
	\ve{ F}^{B} = m \ve{ \dot{u}}^{B} - m \ve{ r}^{PC} \times \ve{ \dot{\omega}}^{B} + m \ve{ \omega}^{B}\times ( \ve{ \omega}^{B}  \times \ve{ r}^{PC} ) ,
\end{align}
where $ \ve{ u} $ is the velocity of $ P$. 

For rotational motion, again we have
\begin{align}
	\ve{M}_O &= \ve{ M}_{P} + \ve{ r}^{OP} \times \ve{ F} \\
	\ve{ L}_O &= \ve{ L}_P + \ve{ r}^{OP} \times \ve{ P} .
\end{align}
Together with Equation 3, we obtain
\begin{align}
	\ve{ M}_P + \ve{ r}^{OP} \times \ve{ F}   &= \ve{ \dot{L}}_P + \ve{ \dot{r}}^{OP} \times \ve{ P}   + \ve{ r}^{OP} \times \ve{ F}  \\
		  \ve{ M}_P  &= \ve{ \dot{L}}_P + \ve{ \dot{r}}^{OP} \times \ve{ P}  .
\end{align}

From the definition of angular momentum of a rigid body, we get
\begin{align}
	\ve{ L}_C &= \int \ve{ r}^{CQ} \times \ve{ v}_Q dm \\  
	\ve{ L}_P &= \int \ve{ r}^{PQ} \times \ve{ v}_Q dm .
\end{align}
By two-point velocity formula, we have
\begin{align}
	\ve{ v}_Q &= \ve{ v}_C + \ve{ \omega} \times \ve{ r}^{CQ}   \\ 
	\ve{ v}_Q &= \ve{ v}_P + \ve{ \omega} \times \ve{ r}^{PQ}   \\
	\ve{ v}_C &= \ve{ v}_P + \ve{ \omega} \times \ve{ r}^{PC}   
\end{align}
Thus Equation 21 becomes
\begin{align}
	\ve{ L}_C &= \int \ve{ r}^{CQ} \times ( \ve{ v}_C + \ve{ \omega} \times \ve{ r}^{CQ} )dm  \\
	&= \int \ve{ r}^{CQ} dm \times \ve{ v}_C + \int \ve{ r}^{CQ} \times \ve{ \omega} \times \ve{ r}^{CQ} dm      \\
	&= \ve{ 0} \times \ve{ v}_C + \int ( (\ve{ r}^{CQ} \cdot  \ve{ r}^{CQ}) \ve{ \omega} - \ve{ r}^{CQ}( \ve{ r}^{CQ} \cdot \ve{ \omega} ) )dm     && \text{center of mass}  \\
	&= \int ((\ve{ r}^{CQ} \cdot \ve{ r}^{CQ}) \mathcal{ U} - \ve{ r}^{CQ} (\ve{ r}^{CQ})^{T} )dm\ \ve{ \omega}    \\
	&=: \mathcal{ I}_C \ve{ \omega} 
\end{align}
Note that this justifies Equation 12. Moreover, Equation 22 becomes
\begin{align}
	\ve{ L}_P &= \int \ve{ r}^{PQ} \times ( \ve{ v}_P + \ve{ \omega} \times \ve{ r}^{PQ} )dm  \\
	&= \int (\ve{ r}^{PC}+ \ve{ r}^{CQ})  dm \times \ve{ v}_P + \int \ve{ r}^{PQ} \times \ve{ \omega} \times \ve{ r}^{PQ} dm      \\
	&= m \ve{ r}^{PC} \times \ve{ v}_P + \ve{ 0} \times \ve{ v}_C + \mathcal{ I}_P \ve{ \omega} \\
	&= m \ve{ r}^{PC} \times \ve{ v}_P + \mathcal{ I}_P \ve{ \omega} 
\end{align}
Now with Equation 33,10, and 25 in mind, Equation 20 becomes
\begin{align}
	\ve{ M}_P &= m \ve{ \dot{r}}^{PC} \times \ve{ v}_P + m \ve{ r}^{PC} \times \ve{ \dot{v}}_P + \mathcal{ I}_P \ve{ \dot{\omega}} + \ve{ \omega} \times (\mathcal{ I}_P \ve{ \omega})    + \ve{ \dot{r}}^{OP} \times (m\ve{ v}_C) \\
	&= m \ve{ \dot{r}}^{PC}  \times \ve{ v}_P + m\ve{ v}_P \times ( \ve{ v}_P+ \ve{ \omega} \times \ve{ r}^{PC} )  + m \ve{ r}^{PC} \times \ve{ \dot{v}}_P + \mathcal{ I}_P \ve{ \dot{\omega}} + \ve{ \omega} \times (\mathcal{ I}_P \ve{ \omega})     \\
	&= m \ve{ \dot{r}}^{PC}  \times \ve{ v}_P + m\ve{ v}_P \times \ve{ \dot{r}}^{PC} + m \ve{ r}^{PC} \times \ve{ \dot{v}}_P + \mathcal{ I}_P \ve{ \dot{\omega}} + \ve{ \omega} \times (\mathcal{ I}_P \ve{ \omega})     \\
	&= m \ve{ r}^{PC} \times \ve{ \dot{v}}_P + \mathcal{ I}_P \ve{ \dot{\omega}} + \ve{ \omega} \times (\mathcal{ I}_P \ve{ \omega})     .
\end{align}
Thus for this problem, since the force is applied to $ P$,  $ \ve{ M}_P = \ve{ 0}  $, so
\begin{align}
	\ve{ 0}  = m \ve{ r}^{PC} \times \ve{ \dot{u}}^{B} + \mathcal{ J} \ve{ \dot{\omega}}^{B} + \ve{ \omega}^{B} \times ( \mathcal{ J} \ve{ \omega}^{B} ).   
\end{align}
\end{problem}

\begin{problem}[4]
By parallel axis theorem, we have
\begin{align*}
	\mathcal{ J} &= \mathcal{ I} - m\ve{ \widetilde{ r}}^{PC} \ve{ \widetilde{ r}}^{PC} \\
		     &= \begin{pmatrix} 655.325&-38.22&-82.81\\-38.22&554.1050&-248.43\\-82.81&-248.43&132.1 \end{pmatrix}  
\end{align*}
where $ \widetilde{ \cdot  }$ denotes the dual/cross-product matrix of the vector.

By Equation 25, we have
\begin{align*}
	\ve{ u}^{I} &= \ve{ v}^{I} - C(\ve{ \omega}^{B} \times \ve{ r}^{PC} ) \\
	&= \begin{pmatrix} 32.89\\27.08\\-11.5 \end{pmatrix} 
\end{align*}
\end{problem}

\begin{problem}[5]
Using Equations 16 and 39, we can set up a matrix equation to solve for $ \ve{ \dot{u}} $ and $ \ve{ \dot{\omega}} $ :
\begin{align*}
	\begin{pmatrix} \ve{ \dot{u}}^{B}\\ \ve{ \dot{\omega}}^{B}   \end{pmatrix} &= \begin{pmatrix} m E_3& -m \ve{ \widetilde{ r}}^{PC} \\ m \ve{ \widetilde{ r}}^{PC} & \mathcal{ J}  \end{pmatrix}^{-1} \begin{pmatrix} \ve{ F}^{B} - m \ve{ \widetilde{ \omega}}^{B} \ve{ \widetilde{ \omega}}^{B} \ve{ r}^{PC} \\ - \ve{ \widetilde{ \omega}}^{B} \mathcal{ J} \ve{ \omega}^{B}     \end{pmatrix}   \\ 
	\ve{ \dot{\omega}}^{B} &= \begin{pmatrix} -67.5008\\18.5284\\-4.0689 \end{pmatrix}  \\ 
	\ve{ \dot{u}}^{I} &= C \ve{ \dot{u}}^{B}  \\
	&= \begin{pmatrix} 1042.4\\-75.149\\423.9981 \end{pmatrix}  
\end{align*}
\end{problem}

\begin{problem}[6]
Clearly $ \ve{ \dot{\omega}}^{B} $ match. Using the two-point formula for acceleration, we have
\begin{align*}
	\ve{ \dot{v}}^{I} &= \ve{ \dot{u}}^{I} + C( \ve{ \omega}^{B} \times \ve{ \omega}^{B} \times  \ve{ r}^{PC} + \ve{ \dot{\omega}}^{B} \times \ve{ r}^{PC}  )  \\ 
	&= \begin{pmatrix} 0.3436\\1.8103\\0.4821 \end{pmatrix} , 
\end{align*}
which again matches!
\end{problem}
\end{document}
