\documentclass[11pt]{article}
\newcommand{\alert}[1]{{\bf \color{red} [Alert:] #1}}
\newcommand{\todo}[1]{{\bf \color{orange} [TODO:] #1}}
\newcommand{\real}[1][]{\mathbb{R}^{#1}}
\newcommand{\myeqn}[1]{(\ref{#1})}
\newcommand{\myex}[1]{Example \ref{#1}}
\newcommand{\defeq}{\stackrel{\mathrm{def}}{=}}
\newcommand{\parder}[2]{\frac{\partial #1}{\partial #2}}
\newcommand{\Lie}[3][]{\mathsf{L}_{#3}^{#1} #2}
\newcommand{\LieA}[1]{\mathsf{Lie}(#1)}
\newcommand{\lieder}[2]{\mathcal{L}_{#2} #1}
\renewcommand{\t}{^{\mbox{\tiny\sf T}}}
\newcommand{\trans}{^{\mbox{\tiny\sf T}}}
\newcommand{\markup}[1]{\{\textbf{#1}\}}
\newcommand{\msub}[1]{_\mathrm{#1}}
\newcommand{\msup}[1]{^\mathrm{#1}}
\newcommand{\inv}[1]{#1^{-1}}
\newcommand{\pinv}[1]{{#1}^{+}}
\newcommand{\myfracA}[2]{\displaystyle{\frac{#1}{#2}}}
\newcommand{\myfracB}[2]{{#1}/{#2}}
\newcommand{\mydiffA}[1]{\dot{#1}}
\newcommand{\mydiffB}[2]{\myfracA{\mathrm{d}{#1}}{\mathrm{d}{#2}}}
\newcommand{\ball}[2]{\mathcal{B}_{#1}\left(#2\right)}
\newcommand{\acos}[1]{\cos^{-1}\left(#1\right)}
\newcommand{\asin}[1]{\sin^{-1}\left(#1\right)}
\newcommand{\mani}{\mathcal{M}}
\newcommand{\tang}[2]{\mathsf{T}_{#1} #2}
\newcommand{\LieB}[2]{[ #1, #2 ]}
\newcommand{\LieBad}[3][]{\mathsf{ad}_{#2}^{#1} #3}
\newcommand{\ReachVT}{\mathcal{R}^V_T}
\newcommand{\ReachVt}{\mathcal{R}^V_t}
\newcommand{\ReachVTe}{\mathcal{R}^V_{\le T}}
\newcommand{\ReachT}{\mathcal{R}_T}
\newcommand{\Reacht}{\mathcal{R}_t}
\newcommand{\ReachTe}{\mathcal{R}_{\le T}}
\newcommand{\accLA}[1]{\mathsf{Lie}(#1)}
\newcommand{\accD}{\Delta_{\mathcal{F}}}
\newcommand{\accSA}{\mathsf{Lie}(\mathcal{G},f)}
\newcommand{\accDS}{\Delta_{\mathcal{G}}}
\newcommand{\eval}[3]{\mathsf{Ev}^{#2}_{#1}\left( #3 \right)}
\newcommand{\stlc}{\textsc{stlc}}
\newcommand{\clf}{\textsc{clf}}
\newcommand{\jqlf}{\textsc{jqlf}}
\newcommand{\dlas}{\textsc{dlas}}
\newcommand{\Ad}[2]{\mathsf{Ad}_{#1} #2}
\newcommand{\xe}{\ensuremath{x_e}}
\newcommand{\lebg}[1]{\mathcal{L}_{#1}}
\newcommand{\lebgx}[1]{\mathcal{L}_{#1 \mathrm{e}}}
\newcommand{\dom}{D}
\newcommand{\domT}{[t_0,\infty) \times D}
\newcommand{\rarrow}{\rightarrow}
\renewcommand{\d}{\mathrm{d}}
\renewcommand{\Re}{\mathbb{R}}
\newcommand{\C}{\mathrm{C}}

\newcommand{\QED}{{\unskip\nobreak\hfil\penalty50\hskip2em\vadjust{}
		\nobreak\hfil$\Box$\parfillskip=0pt\finalhyphendemerits=0\par}\vspace{0.1cm}}
\newcommand{\eoEx}{{\unskip\nobreak\hfil\penalty50\hskip0em\vadjust{}
		\nobreak\hfil$\Large\Diamond$\parfillskip=0pt\finalhyphendemerits=0\par}\vspace{0.1cm}}

\newcommand{\sgn}{\ensuremath{\operatorname{sgn}}}
\newcommand{\sat}{\ensuremath{\operatorname{sat}}}

\newcommand{\half}{\frac{1}{2}}
\newcommand{\shalf}{\mbox{$\frac{1}{2}$}}
\newcommand{\marcom}[1]{\marginpar{\footnotesize #1}}
\newcommand{\der}{\mathrm{D}}
\newcommand{\e}{\mathrm{e}}
\newcommand{\dt}{\mathrm{d}t}

\newcommand{\cA}{\ensuremath{\mathcal{A}}}
\newcommand{\cB}{\ensuremath{\mathcal{B}}}
\newcommand{\cG}{\ensuremath{\mathcal{G}}}
\newcommand{\cK}{\ensuremath{\mathcal{K}}}
\newcommand{\cW}{\ensuremath{\mathcal{W}}}
\newcommand{\cZ}{\ensuremath{\mathcal{Z}}}
\newcommand{\cS}{\ensuremath{\mathcal{S}}}
\newcommand{\cD}{\ensuremath{\mathcal{D}}}
\newcommand{\cP}{\ensuremath{\mathcal{P}}}
\newcommand{\cV}{\ensuremath{\mathcal{V}}}
\newcommand{\cL}{\ensuremath{\mathcal{L}}}
\newcommand{\cN}{\ensuremath{\mathcal{N}}}
\newcommand{\cI}{\ensuremath{\mathcal{I}}}
\newcommand{\cR}{\ensuremath{\mathcal{R}}}
\newcommand{\cM}{\ensuremath{\mathcal{M}}}
\newcommand{\cC}{\ensuremath{\mathcal{C}}}
\newcommand{\cF}{\ensuremath{\mathcal{F}}}
\newcommand{\cH}{\ensuremath{\mathcal{H}}}
\newcommand{\cO}{\ensuremath{\mathcal{O}}}
\newcommand{\cX}{\ensuremath{\mathcal{X}}}
\newcommand{\cY}{\ensuremath{\mathcal{Y}}}
\newcommand{\Ci}{\ensuremath{\mathcal{C}^\infty}}
\newcommand{\ISS}{\textsc{iss}}
\newcommand{\LISS}{\textsc{liss}}
\newcommand{\GAS}{\textsc{gas}}
\newcommand{\GS}{\textsc{gs}}
\newcommand{\LES}{\textsc{les}}
\newcommand{\GUAS}{\textsc{guas}}
\newcommand{\BIBO}{\textsc{bibo}}
\newcommand{\spec}{\ensuremath{\operatorname{spec}}}
\newcommand{\spn}{\ensuremath{\operatorname{span}}}
\renewcommand{\i}{\mathrm{i\,}}

\renewcommand{\implies}{\Rightarrow}

\renewcommand{\theenumi}{$\roman{enumi})$}
\renewcommand{\labelenumi}{\theenumi}

\font\ptmten=zptmcmrm scaled 1200
\newcommand{\w}{\mbox{{\ptmten w}}}
\newcommand{\z}{\mbox{{\ptmten z}}}
\renewcommand{\Re}{\mathbb{R}}

\newcommand{\cl}{\operatorname{cl}}
\newcommand{\intr}{\operatorname{int}}
\newcommand{\rank}{\operatorname{rank}}
\newcommand{\co}{\operatorname{co}}
\newcommand{\aff}{\operatorname{aff}}

\theoremstyle{plain}
\newtheorem{theorem}{Theorem}[chapter]
\newtheorem{claim}[theorem]{Claim}
\newtheorem{corollary}[theorem]{Corollary}
\newtheorem{prop}[theorem]{Proposition}
\newtheorem{fact}[theorem]{Fact}
\newtheorem{lemma}[theorem]{Lemma}

\newtheorem{remark}{Remark}[chapter]

\theoremstyle{definition}
\newtheorem{assume}[theorem]{Assumption}
\newtheorem{defn}[theorem]{Definition}
\newtheorem{problem}[theorem]{Problem}
\newtheorem{exercise}{Exercise}
\newtheorem{example}[theorem]{Example}


\begin{document}
\centerline {\textsf{\textbf{\LARGE{Homework 6}}}}
\centerline {Jaden Wang}
\vspace{.15in}
\section{Lagrangian Mechanics}
Disclaimer: most derivatives and equations are computed by Mathematica. We have the following equantities where subscripts represent relating to (center of mass of) 1st or 2nd rod.
\begin{align*}
	\ve{ r}_1 = \frac{L}{2} \begin{pmatrix} \sin \theta_1 \\ -\cos \theta_1 \\0 \end{pmatrix} ,   \ \ve{ r}_2 = \begin{pmatrix} L \sin \theta_1 + \frac{L}{2} \sin \theta_2\\ -L \cos \theta_1 - \frac{L}{2} \cos \theta_2 \\0 \end{pmatrix}  \\
	\ve{ v}_1 = \frac{L}{2} \begin{pmatrix} \cos \theta_1 \dot{\theta}_1\\ \sin \theta_1 \dot{\theta}_1 \\0 \end{pmatrix}, \ \ve{ v}_2 = \begin{pmatrix} L \cos \theta_1 \dot{\theta}_1 + \frac{L}{2} \cos \theta_2 \dot{\theta}_2 \\L \sin \theta_1 \dot{\theta}_1 + \frac{L}{2} \sin \theta_2 \dot{\theta}_2\\0 \end{pmatrix} \\
	 \ve{ \omega}_1 = \begin{pmatrix} 0\\0\\ \dot{\theta}_1 \end{pmatrix} \ \ve{ \omega}_2 = \begin{pmatrix} 0\\0\\ \dot{\theta}_2 \end{pmatrix}, \  \mathcal{ I}_1 = \mathcal{ I}_2= \begin{pmatrix} \ve{ 0}_2 &0\\0& \frac{1}{12} m L^2 \end{pmatrix} 
\end{align*}
We can now compute the kinetic and potential energy of the system and therefore the Lagrangian and its derivatives:
\begin{align*}
	T &= \frac{1}{2} m (\ve{ v}_1 \cdot \ve{ v}_1+ \ve{ v}_2 \cdot \ve{ v}_2) + \frac{1}{2} \ve{ \omega}_1 ^{T} \mathcal{ I}_1 \ve{ \omega}_1 + \frac{1}{2} \ve{ \omega}_2^{T} \mathcal{ I}_2 \ve{ \omega}_2  \\
	&= \frac{1}{6} mL^2 \left( 4 \dot{\theta}_1^2 + 3 \cos ( \theta_1 - \theta_2) \dot{\theta}_1 \dot{\theta}_2 + \dot{\theta}_2^2 \right)  \\
	V &= -mg \frac{3L}{2}\cos \theta_1 -mg \frac{L}{2} \cos \theta_2 \\
	\mathscr{L} &:= T-V\\
	\frac{\partial \mathscr{L}}{\partial \theta_1} &= -\frac{1}{2} mL \left( L \sin(\theta_1 - \theta_2) \dot{\theta}_1 \dot{\theta}_2 + 3g \sin \theta_1  \right) \\
	\frac{\partial \mathscr{L}}{\partial \dot{\theta}_1} &= \frac{1}{6} mL^2 \left( 8 \dot{\theta}_1 + 3 \cos (\theta_1 - \theta_2) \dot{\theta}_2 \right)  \\
	\frac{d}{dt} \frac{\partial \mathscr{L}}{\partial \dot{\theta}_1} &= \frac{1}{6} mL^2 \left( 8 \ddot{\theta_1} -3 \sin(\theta_1 - \theta_2) (\dot{\theta}_1 -\dot{\theta}_2) \dot{\theta}_2 + 3 \cos(\theta_1 - \theta_2) \ddot{\theta}_2 \right)  \\
	\frac{\partial \mathscr{L}}{\partial \theta_2} &= \frac{1}{2} mL \left( L \sin(\theta_1 - \theta_2) \dot{\theta}_1 \dot{\theta}_2 - g \sin \theta_2  \right) \\
	\frac{\partial \mathscr{L}}{\partial \dot{\theta}_2} &= \frac{1}{6} mL^2 \left( 2 \dot{\theta}_2 + 3 \cos (\theta_1 - \theta_2) \dot{\theta}_1 \right)  \\
	\frac{d}{dt} \frac{\partial \mathscr{L}}{\partial \dot{\theta}_2} &= \frac{1}{6} mL^2 \left( 2 \ddot{\theta_2} -3 \sin(\theta_1 - \theta_2) (\dot{\theta}_1 -\dot{\theta}_2) \dot{\theta}_1 + 3 \cos(\theta_1 - \theta_2) \ddot{\theta}_1 \right)  
\end{align*}
Therefore, the Euler-Lagrange equations can be simplified as
\begin{align}
	mL \left( 9g \sin \theta_1 + 3L \sin (\theta_1 - \theta_2) \dot{\theta}_2^2 + 3L\cos (\theta_1 - \theta_2) \ddot{\theta}_2 + 8L \ddot{\theta}_1 \right) &= 0 \\
	mL \left( 3g \sin \theta_2 - 3L \sin (\theta_1 - \theta_2) \dot{\theta}_1^2 + 3L \cos (\theta_1 - \theta_2) \ddot{\theta}_1 + 2L \ddot{\theta}_2 \right) &= 0 .
\end{align}
\section{Newton-Euler}
Denote the gravity of two rods by $ \ve{ G}_1 = \ve{ G}_2 = \begin{pmatrix} 0\\-mg\\0 \end{pmatrix}$. Consider the 1st rod rotating about $ O$ which is fixed in the inertial frame. By parallel axis theorem, in the $ zz$ component we have $ \mathcal{ I}_1^{O} = \mathcal{ I}_1 + m\frac{L^2}{4} = \frac{1}{3} m L^2$ (and 0 elsewhere). We also have $ \ve{ r}_P = 2 \ve{ r}_1$. Let $ \ve{ F} = \begin{pmatrix} F_x \\ F_y \\0 \end{pmatrix}  $ be the tension force experienced by the 2nd rod at $ P$. Thus the tension force experienced by the 1st rod at $ P$ is $ -\ve{ F} $.  Then Euler's 2nd law about $ O$ states
\begin{align*}
	\frac{d \ve{ L}^{O} }{ dt} &= \ve{ M}_{net}^{O}  \\
	\mathcal{ I}_1^{O} \ddot{\theta_1} &= \ve{ r}_1 \times \ve{ G}_1 + \ve{ r}_P \times (-\ve{ F})  
\end{align*}
Only $ z$-components survive. Simplifying the equation yields
\begin{align}
	mL \left( 9g \sin \theta_1 + 3L \sin (\theta_1 - \theta_2) \dot{\theta}_2^2 + 3L\cos (\theta_1 - \theta_2) \ddot{\theta}_2 + 8L \ddot{\theta}_1 \right) = 0 .
\end{align}
We see that Equation 3 matches Equation 1!

Now let us consider the 2nd rod. By Euler's 1st law, we have
\begin{align*}
	m \ve{ \ddot{r}}_2 &= \ve{ F}_{net} = \ve{ G}_2  + \ve{ F}   \\
	\ve{ F} &= m \ve{ \ddot{r}}_2 - \ve{ G}_2   \\
	&= \begin{pmatrix} -\frac{1}{2}mL \left( 2 \sin \theta_1 \dot{\theta}_1^2 + \sin \theta_2 \dot{\theta}_2^2  - 2 \cos \theta_1 \ddot{\theta}_1 - \cos \theta_2 \ddot{\theta}_2 \right) \\ m \left( g + L \cos \theta_1 \dot{\theta}_1^2 + \frac{1}{2} L \cos\theta_2 \dot{\theta}_2^2  + L \sin \theta_1 \ddot{\theta}_1 + \frac{1}{2} L \sin\theta_2 \ddot{\theta}_2  \right) \\0 \end{pmatrix}  
\end{align*}
Euler's 2nd law on the 2nd rod about center of mass gives
\begin{align*}
	\frac{d \ve{ L}^{C_2} }{ dt} &= \ve{ M}_{net}^{C_2}  \\
	\mathcal{ I}_2 \ddot{\theta}_2 &= (\ve{ r}_p - \ve{ r}_2)  \times \ve{ F}  \\
	\begin{pmatrix} 0\\0\\ \frac{1}{12} mL^2 \ddot{\theta}_2 \end{pmatrix} &= \begin{pmatrix} -\frac{1}{2} L \sin \theta_2 \\ \frac{1}{2} L \cos \theta_2 \\0 \end{pmatrix} \times \ve{ F}  
\end{align*}
Only $ z$-components survive. Simplifying the equation yields
\begin{align}
	mL \left( 3g \sin \theta_2 - 3L \sin (\theta_1 - \theta_2) \dot{\theta}_1^2 + 3L \cos (\theta_1 - \theta_2) \ddot{\theta}_1 + 2L \ddot{\theta}_2 \right) = 0 .
\end{align}
We see that Equation 4 exactly matches Equation 2! Therefore, we have analytically shown that the two methods give equivalent equations of motion. Using $ m = 10 kg, L = 2m, g=9.8 m /s ^2, t_f = 20s$ and the initial conditions of $ \theta_1(0) = 1, \theta_2(0) = 0, \dot{\theta}(0) = 0, \dot{\theta}(0) =0$, we have the following plots:
~\begin{figure}[H]
	\centering
	\includegraphics[width=0.49\textwidth]{./figures/hw6.1.png}
	\includegraphics[width=0.49\textwidth]{./figures/hw6.2.png}
	\caption{Since the system start from being stationary and $ \theta_2$ being vertical, we expect $ \theta_1$ to drop and swing to the other side while $ \theta_2$ increasing initially. The plots match our expectations.}
\end{figure}
\end{document}
