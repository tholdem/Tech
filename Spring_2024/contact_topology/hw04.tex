\documentclass[12pt]{article}
\newcommand{\alert}[1]{{\bf \color{red} [Alert:] #1}}
\newcommand{\todo}[1]{{\bf \color{orange} [TODO:] #1}}
\newcommand{\real}[1][]{\mathbb{R}^{#1}}
\newcommand{\myeqn}[1]{(\ref{#1})}
\newcommand{\myex}[1]{Example \ref{#1}}
\newcommand{\defeq}{\stackrel{\mathrm{def}}{=}}
\newcommand{\parder}[2]{\frac{\partial #1}{\partial #2}}
\newcommand{\Lie}[3][]{\mathsf{L}_{#3}^{#1} #2}
\newcommand{\LieA}[1]{\mathsf{Lie}(#1)}
\newcommand{\lieder}[2]{\mathcal{L}_{#2} #1}
\renewcommand{\t}{^{\mbox{\tiny\sf T}}}
\newcommand{\trans}{^{\mbox{\tiny\sf T}}}
\newcommand{\markup}[1]{\{\textbf{#1}\}}
\newcommand{\msub}[1]{_\mathrm{#1}}
\newcommand{\msup}[1]{^\mathrm{#1}}
\newcommand{\inv}[1]{#1^{-1}}
\newcommand{\pinv}[1]{{#1}^{+}}
\newcommand{\myfracA}[2]{\displaystyle{\frac{#1}{#2}}}
\newcommand{\myfracB}[2]{{#1}/{#2}}
\newcommand{\mydiffA}[1]{\dot{#1}}
\newcommand{\mydiffB}[2]{\myfracA{\mathrm{d}{#1}}{\mathrm{d}{#2}}}
\newcommand{\ball}[2]{\mathcal{B}_{#1}\left(#2\right)}
\newcommand{\acos}[1]{\cos^{-1}\left(#1\right)}
\newcommand{\asin}[1]{\sin^{-1}\left(#1\right)}
\newcommand{\mani}{\mathcal{M}}
\newcommand{\tang}[2]{\mathsf{T}_{#1} #2}
\newcommand{\LieB}[2]{[ #1, #2 ]}
\newcommand{\LieBad}[3][]{\mathsf{ad}_{#2}^{#1} #3}
\newcommand{\ReachVT}{\mathcal{R}^V_T}
\newcommand{\ReachVt}{\mathcal{R}^V_t}
\newcommand{\ReachVTe}{\mathcal{R}^V_{\le T}}
\newcommand{\ReachT}{\mathcal{R}_T}
\newcommand{\Reacht}{\mathcal{R}_t}
\newcommand{\ReachTe}{\mathcal{R}_{\le T}}
\newcommand{\accLA}[1]{\mathsf{Lie}(#1)}
\newcommand{\accD}{\Delta_{\mathcal{F}}}
\newcommand{\accSA}{\mathsf{Lie}(\mathcal{G},f)}
\newcommand{\accDS}{\Delta_{\mathcal{G}}}
\newcommand{\eval}[3]{\mathsf{Ev}^{#2}_{#1}\left( #3 \right)}
\newcommand{\stlc}{\textsc{stlc}}
\newcommand{\clf}{\textsc{clf}}
\newcommand{\jqlf}{\textsc{jqlf}}
\newcommand{\dlas}{\textsc{dlas}}
\newcommand{\Ad}[2]{\mathsf{Ad}_{#1} #2}
\newcommand{\xe}{\ensuremath{x_e}}
\newcommand{\lebg}[1]{\mathcal{L}_{#1}}
\newcommand{\lebgx}[1]{\mathcal{L}_{#1 \mathrm{e}}}
\newcommand{\dom}{D}
\newcommand{\domT}{[t_0,\infty) \times D}
\newcommand{\rarrow}{\rightarrow}
\renewcommand{\d}{\mathrm{d}}
\renewcommand{\Re}{\mathbb{R}}
\newcommand{\C}{\mathrm{C}}

\newcommand{\QED}{{\unskip\nobreak\hfil\penalty50\hskip2em\vadjust{}
		\nobreak\hfil$\Box$\parfillskip=0pt\finalhyphendemerits=0\par}\vspace{0.1cm}}
\newcommand{\eoEx}{{\unskip\nobreak\hfil\penalty50\hskip0em\vadjust{}
		\nobreak\hfil$\Large\Diamond$\parfillskip=0pt\finalhyphendemerits=0\par}\vspace{0.1cm}}

\newcommand{\sgn}{\ensuremath{\operatorname{sgn}}}
\newcommand{\sat}{\ensuremath{\operatorname{sat}}}

\newcommand{\half}{\frac{1}{2}}
\newcommand{\shalf}{\mbox{$\frac{1}{2}$}}
\newcommand{\marcom}[1]{\marginpar{\footnotesize #1}}
\newcommand{\der}{\mathrm{D}}
\newcommand{\e}{\mathrm{e}}
\newcommand{\dt}{\mathrm{d}t}

\newcommand{\cA}{\ensuremath{\mathcal{A}}}
\newcommand{\cB}{\ensuremath{\mathcal{B}}}
\newcommand{\cG}{\ensuremath{\mathcal{G}}}
\newcommand{\cK}{\ensuremath{\mathcal{K}}}
\newcommand{\cW}{\ensuremath{\mathcal{W}}}
\newcommand{\cZ}{\ensuremath{\mathcal{Z}}}
\newcommand{\cS}{\ensuremath{\mathcal{S}}}
\newcommand{\cD}{\ensuremath{\mathcal{D}}}
\newcommand{\cP}{\ensuremath{\mathcal{P}}}
\newcommand{\cV}{\ensuremath{\mathcal{V}}}
\newcommand{\cL}{\ensuremath{\mathcal{L}}}
\newcommand{\cN}{\ensuremath{\mathcal{N}}}
\newcommand{\cI}{\ensuremath{\mathcal{I}}}
\newcommand{\cR}{\ensuremath{\mathcal{R}}}
\newcommand{\cM}{\ensuremath{\mathcal{M}}}
\newcommand{\cC}{\ensuremath{\mathcal{C}}}
\newcommand{\cF}{\ensuremath{\mathcal{F}}}
\newcommand{\cH}{\ensuremath{\mathcal{H}}}
\newcommand{\cO}{\ensuremath{\mathcal{O}}}
\newcommand{\cX}{\ensuremath{\mathcal{X}}}
\newcommand{\cY}{\ensuremath{\mathcal{Y}}}
\newcommand{\Ci}{\ensuremath{\mathcal{C}^\infty}}
\newcommand{\ISS}{\textsc{iss}}
\newcommand{\LISS}{\textsc{liss}}
\newcommand{\GAS}{\textsc{gas}}
\newcommand{\GS}{\textsc{gs}}
\newcommand{\LES}{\textsc{les}}
\newcommand{\GUAS}{\textsc{guas}}
\newcommand{\BIBO}{\textsc{bibo}}
\newcommand{\spec}{\ensuremath{\operatorname{spec}}}
\newcommand{\spn}{\ensuremath{\operatorname{span}}}
\renewcommand{\i}{\mathrm{i\,}}

\renewcommand{\implies}{\Rightarrow}

\renewcommand{\theenumi}{$\roman{enumi})$}
\renewcommand{\labelenumi}{\theenumi}

\font\ptmten=zptmcmrm scaled 1200
\newcommand{\w}{\mbox{{\ptmten w}}}
\newcommand{\z}{\mbox{{\ptmten z}}}
\renewcommand{\Re}{\mathbb{R}}

\newcommand{\cl}{\operatorname{cl}}
\newcommand{\intr}{\operatorname{int}}
\newcommand{\rank}{\operatorname{rank}}
\newcommand{\co}{\operatorname{co}}
\newcommand{\aff}{\operatorname{aff}}

\theoremstyle{plain}
\newtheorem{theorem}{Theorem}[chapter]
\newtheorem{claim}[theorem]{Claim}
\newtheorem{corollary}[theorem]{Corollary}
\newtheorem{prop}[theorem]{Proposition}
\newtheorem{fact}[theorem]{Fact}
\newtheorem{lemma}[theorem]{Lemma}

\newtheorem{remark}{Remark}[chapter]

\theoremstyle{definition}
\newtheorem{assume}[theorem]{Assumption}
\newtheorem{defn}[theorem]{Definition}
\newtheorem{problem}[theorem]{Problem}
\newtheorem{exercise}{Exercise}
\newtheorem{example}[theorem]{Example}


\begin{document}
\centerline {\textsf{\textbf{\LARGE{Homework 4}}}}
\centerline {Jaden Wang}
\vspace{.15in}

\begin{problem}[1: VIII Theorem 2]
If $ (M,\xi)$ is tight and $ L$ is a Legendrian knot with $ L = \partial \Sigma $, then prove $ tb(L)-r(L) \leq - \chi( \Sigma )$.

\begin{proof}
We can WLOG assume $ tb(L)<0$ since any negative stablization of  $ L$ decreases both $ tb(L)$ and $ r(L)$ by 1 so $ tb(L) - r(L)$ remains constant. Moreover, $ \chi( \Sigma _-) \leq -tb(L)$. Indeed, the only surface with boundary that gives positive Euler characteristic is a disk with $ \chi(D)=1$, and by Giroux Criterion, dividing curves cannot bound a disk, so the only possible disks appear on the boundary $ L$. The dividing curves must intersect $ L$ $ -2tb(L)$ times by Theorem VII.9. That means at most $ -tb(L)$ number of disks can be formed in  $ \Sigma _-$. Thus $ \chi( \Sigma _-)$ is upper bounded by $ -tb(L)$. Then
\begin{align*}
	tb(L) - r(L) &\leq tb(L) - r(L) - 2\chi( \Sigma _-) - 2 tb(L)\\
	&= -tb(L) - ( \chi( \Sigma _+) - \chi( \Sigma _-)) - 2 \chi( \Sigma_-) \\
	&= -tb(L) -   \chi( \Sigma _+) - \chi( \Sigma _-)\\
	&= - \chi( \Sigma ) 
\end{align*}
by Claim 1.
\end{proof}
\end{problem}
\begin{problem}[2: IV Lemma 4]
If $ \gamma_0, \gamma_1$ are cobordant via $ \Sigma \subset M \times I$, we project $ \Sigma $ to $ M$ and triangulate $ \pi( \Sigma )$ to obtain a 2-chain $ c$ in $ C_2(M)$. Show that $ \partial c  = \gamma_1 - \gamma_0$.
\begin{proof}
Let us triangulate $ \Sigma $ and obtain a 2-chain $ c'$ in $ C_2(M \times I)$. I claim that $ \partial( c') = \gamma_1 - \gamma_0$. This is because the boundary operator in a chain complex is exactly defined so that when a manifold with boundary is triangulated into a chain complex, its triangulated boundary is the image of the boundary operator. Let $ \pi_\# : C_2(M \times I) \to C_2(M)$ be the induced chain map, then we have
\begin{align*}
	\partial c = \partial \pi_\# (c') = \pi_\# \partial (c') = \pi_\# ( \gamma_1 - \gamma_0) = \gamma_1 - \gamma_0 , 
\end{align*}
where we use the commutativity of chain map and boundary map, and the fact that projection to $ M$ does not change  $ \gamma_i$.
\end{proof}
\end{problem}
\begin{problem}[3: IV Lemma 5]
Let $ y $ be a free generator of $ H_1(M)$ (since we assume $ d(x) \neq 0$ and $ x= d(x) y$ is thus non-torsion). Show that there exists a surface $ \alpha$ in $ M$  s.t.\ $ y \cdot \alpha = 1$. Note $ M$ is compact.

\begin{proof}
	By Poincare duality, $ PD(y)$ is a free generator in  $ H^2(M)$. By the universal coefficient theorem, $ H^2(M) \cong \Hom( H_2(M), \zz) \oplus \ext(H_1(M),\zz) $, where the $ \Hom$ term contains the free part and the $ \ext$ term contains the torsion part. Thus $ PD(y)$ can be viewed as a free generator in $\Hom(H_2(M), \zz)$ which is free. Choose a basis (containing $ PD(y)$) of $ \Hom(H_2(M), \zz)$, which is dual to $ \text{Free}(H_2(M))$. Then dualize $ \Hom(H_2(M), \zz)$ to get the double-dual under the dual basis containing $ PD(y)^*: \Hom( M, \zz) \to \zz$ that sends $ PD(y)$ to 1 and all other generators to 0. By the canonical isomorphism of a finite-rank free module and its double-dual, we know that the generator $ PD(y)^* $ is the evaluation map where $ PD(y)$ evaluated at some generator $ [ \alpha] \in \text{Free}(H_2(M))$ is 1. \emph{i.e.} $ \langle PD(y), [ \alpha] \rangle =1$. Let $ \alpha$ be a surface representing this corresponding generator. Then $ y \cdot \alpha = \langle PD(y), [\alpha] \rangle = 1$.
\end{proof}
\end{problem}

\begin{problem}[4: VIII Theorem 1]
Show that Theorem VIII.1 implies that there exists only finitely many classes in $ H^2(M)$ that can be the Euler class of a tight contact structure.
\begin{proof}
	By universal coefficient theorem, $ H^2(M) \cong \Hom( H_2(M), \zz) \oplus \ext(H_1(M), \zz)$, where $ \ext(H_1(M), \zz)$ is isomorphic to the torsion part of $ H_1(M)$ and $ \Hom( H_2(M), \zz)$ is isomorphic to the free part of $ H_2(M)$. Choose any basis of the free part of  $ H_2(M)$, which we can represent via surfaces in $ M$. For any such basis surface $ \Sigma $, if $ e(\xi)$ is the Euler class of  $ \xi$, then Theorem 1 says that $ |\langle e(\xi), [ \Sigma ] \rangle| \leq -\chi( \Sigma )$ or $ 0$, meaning that $ e(\xi)$ can only map basis to finitely many integers. This gives finitely many choices to define $ e(\xi)$ on $ \Hom( H_2(M), \zz)$. Now choose a generator set for the torsion part of $ H_1(M)$. Again we only have finitely many integers to map the generators by definition of torsion. Thus we completely define $ e(\xi)$ by where it maps the generators of $ H_1(M)$ and basis of $ H_2(M)$ but only have finitely many choices altogether. This proves the claim.
\end{proof}
\end{problem}
\end{document}
