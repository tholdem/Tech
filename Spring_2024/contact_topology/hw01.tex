\documentclass[12pt]{article}
\newcommand{\alert}[1]{{\bf \color{red} [Alert:] #1}}
\newcommand{\todo}[1]{{\bf \color{orange} [TODO:] #1}}
\newcommand{\real}[1][]{\mathbb{R}^{#1}}
\newcommand{\myeqn}[1]{(\ref{#1})}
\newcommand{\myex}[1]{Example \ref{#1}}
\newcommand{\defeq}{\stackrel{\mathrm{def}}{=}}
\newcommand{\parder}[2]{\frac{\partial #1}{\partial #2}}
\newcommand{\Lie}[3][]{\mathsf{L}_{#3}^{#1} #2}
\newcommand{\LieA}[1]{\mathsf{Lie}(#1)}
\newcommand{\lieder}[2]{\mathcal{L}_{#2} #1}
\renewcommand{\t}{^{\mbox{\tiny\sf T}}}
\newcommand{\trans}{^{\mbox{\tiny\sf T}}}
\newcommand{\markup}[1]{\{\textbf{#1}\}}
\newcommand{\msub}[1]{_\mathrm{#1}}
\newcommand{\msup}[1]{^\mathrm{#1}}
\newcommand{\inv}[1]{#1^{-1}}
\newcommand{\pinv}[1]{{#1}^{+}}
\newcommand{\myfracA}[2]{\displaystyle{\frac{#1}{#2}}}
\newcommand{\myfracB}[2]{{#1}/{#2}}
\newcommand{\mydiffA}[1]{\dot{#1}}
\newcommand{\mydiffB}[2]{\myfracA{\mathrm{d}{#1}}{\mathrm{d}{#2}}}
\newcommand{\ball}[2]{\mathcal{B}_{#1}\left(#2\right)}
\newcommand{\acos}[1]{\cos^{-1}\left(#1\right)}
\newcommand{\asin}[1]{\sin^{-1}\left(#1\right)}
\newcommand{\mani}{\mathcal{M}}
\newcommand{\tang}[2]{\mathsf{T}_{#1} #2}
\newcommand{\LieB}[2]{[ #1, #2 ]}
\newcommand{\LieBad}[3][]{\mathsf{ad}_{#2}^{#1} #3}
\newcommand{\ReachVT}{\mathcal{R}^V_T}
\newcommand{\ReachVt}{\mathcal{R}^V_t}
\newcommand{\ReachVTe}{\mathcal{R}^V_{\le T}}
\newcommand{\ReachT}{\mathcal{R}_T}
\newcommand{\Reacht}{\mathcal{R}_t}
\newcommand{\ReachTe}{\mathcal{R}_{\le T}}
\newcommand{\accLA}[1]{\mathsf{Lie}(#1)}
\newcommand{\accD}{\Delta_{\mathcal{F}}}
\newcommand{\accSA}{\mathsf{Lie}(\mathcal{G},f)}
\newcommand{\accDS}{\Delta_{\mathcal{G}}}
\newcommand{\eval}[3]{\mathsf{Ev}^{#2}_{#1}\left( #3 \right)}
\newcommand{\stlc}{\textsc{stlc}}
\newcommand{\clf}{\textsc{clf}}
\newcommand{\jqlf}{\textsc{jqlf}}
\newcommand{\dlas}{\textsc{dlas}}
\newcommand{\Ad}[2]{\mathsf{Ad}_{#1} #2}
\newcommand{\xe}{\ensuremath{x_e}}
\newcommand{\lebg}[1]{\mathcal{L}_{#1}}
\newcommand{\lebgx}[1]{\mathcal{L}_{#1 \mathrm{e}}}
\newcommand{\dom}{D}
\newcommand{\domT}{[t_0,\infty) \times D}
\newcommand{\rarrow}{\rightarrow}
\renewcommand{\d}{\mathrm{d}}
\renewcommand{\Re}{\mathbb{R}}
\newcommand{\C}{\mathrm{C}}

\newcommand{\QED}{{\unskip\nobreak\hfil\penalty50\hskip2em\vadjust{}
		\nobreak\hfil$\Box$\parfillskip=0pt\finalhyphendemerits=0\par}\vspace{0.1cm}}
\newcommand{\eoEx}{{\unskip\nobreak\hfil\penalty50\hskip0em\vadjust{}
		\nobreak\hfil$\Large\Diamond$\parfillskip=0pt\finalhyphendemerits=0\par}\vspace{0.1cm}}

\newcommand{\sgn}{\ensuremath{\operatorname{sgn}}}
\newcommand{\sat}{\ensuremath{\operatorname{sat}}}

\newcommand{\half}{\frac{1}{2}}
\newcommand{\shalf}{\mbox{$\frac{1}{2}$}}
\newcommand{\marcom}[1]{\marginpar{\footnotesize #1}}
\newcommand{\der}{\mathrm{D}}
\newcommand{\e}{\mathrm{e}}
\newcommand{\dt}{\mathrm{d}t}

\newcommand{\cA}{\ensuremath{\mathcal{A}}}
\newcommand{\cB}{\ensuremath{\mathcal{B}}}
\newcommand{\cG}{\ensuremath{\mathcal{G}}}
\newcommand{\cK}{\ensuremath{\mathcal{K}}}
\newcommand{\cW}{\ensuremath{\mathcal{W}}}
\newcommand{\cZ}{\ensuremath{\mathcal{Z}}}
\newcommand{\cS}{\ensuremath{\mathcal{S}}}
\newcommand{\cD}{\ensuremath{\mathcal{D}}}
\newcommand{\cP}{\ensuremath{\mathcal{P}}}
\newcommand{\cV}{\ensuremath{\mathcal{V}}}
\newcommand{\cL}{\ensuremath{\mathcal{L}}}
\newcommand{\cN}{\ensuremath{\mathcal{N}}}
\newcommand{\cI}{\ensuremath{\mathcal{I}}}
\newcommand{\cR}{\ensuremath{\mathcal{R}}}
\newcommand{\cM}{\ensuremath{\mathcal{M}}}
\newcommand{\cC}{\ensuremath{\mathcal{C}}}
\newcommand{\cF}{\ensuremath{\mathcal{F}}}
\newcommand{\cH}{\ensuremath{\mathcal{H}}}
\newcommand{\cO}{\ensuremath{\mathcal{O}}}
\newcommand{\cX}{\ensuremath{\mathcal{X}}}
\newcommand{\cY}{\ensuremath{\mathcal{Y}}}
\newcommand{\Ci}{\ensuremath{\mathcal{C}^\infty}}
\newcommand{\ISS}{\textsc{iss}}
\newcommand{\LISS}{\textsc{liss}}
\newcommand{\GAS}{\textsc{gas}}
\newcommand{\GS}{\textsc{gs}}
\newcommand{\LES}{\textsc{les}}
\newcommand{\GUAS}{\textsc{guas}}
\newcommand{\BIBO}{\textsc{bibo}}
\newcommand{\spec}{\ensuremath{\operatorname{spec}}}
\newcommand{\spn}{\ensuremath{\operatorname{span}}}
\renewcommand{\i}{\mathrm{i\,}}

\renewcommand{\implies}{\Rightarrow}

\renewcommand{\theenumi}{$\roman{enumi})$}
\renewcommand{\labelenumi}{\theenumi}

\font\ptmten=zptmcmrm scaled 1200
\newcommand{\w}{\mbox{{\ptmten w}}}
\newcommand{\z}{\mbox{{\ptmten z}}}
\renewcommand{\Re}{\mathbb{R}}

\newcommand{\cl}{\operatorname{cl}}
\newcommand{\intr}{\operatorname{int}}
\newcommand{\rank}{\operatorname{rank}}
\newcommand{\co}{\operatorname{co}}
\newcommand{\aff}{\operatorname{aff}}

\theoremstyle{plain}
\newtheorem{theorem}{Theorem}[chapter]
\newtheorem{claim}[theorem]{Claim}
\newtheorem{corollary}[theorem]{Corollary}
\newtheorem{prop}[theorem]{Proposition}
\newtheorem{fact}[theorem]{Fact}
\newtheorem{lemma}[theorem]{Lemma}

\newtheorem{remark}{Remark}[chapter]

\theoremstyle{definition}
\newtheorem{assume}[theorem]{Assumption}
\newtheorem{defn}[theorem]{Definition}
\newtheorem{problem}[theorem]{Problem}
\newtheorem{exercise}{Exercise}
\newtheorem{example}[theorem]{Example}


\begin{document}
\centerline {\textsf{\textbf{\LARGE{Homework 1}}}}
\centerline {Jaden Wang}
\vspace{.15in}
\begin{problem}[1]
Show that all plane fields can be locally written as $ \ker \alpha$ for some $ \alpha$.

\begin{proof}
Fix a Riemannian metric on $ M$ so we have the notion of orthogonality. Given $ p \in M$, take a contractible neighborhood $ U$ of  $ p$ and let  $ \ell := (\xi|_U)^{\perp}$ be the line bundle formed by orthogonal complements of $ \xi$ over $ U$. Since $ U$ is contractible, $ \ell \cong U \times \rr$. Take any nonzero smooth section $ s$ of $ \ell$, say $s: x \mapsto (x,1) $. Now define a 1-form $ \alpha$ of $ U$ by
\begin{align*}
	\alpha: U \to T^*U = (\xi|_U)^* \oplus \ell^*, x \mapsto (0,s),
\end{align*}
where $ 0: \xi|_U \to \rr$ is the zero function. From this definition, it is clear that $ \alpha$ is a smooth section of $ T^* U$ and thus a smooth 1-form and $ \xi|_U = \ker \alpha$ as desired.
\end{proof}
\end{problem}
\begin{problem}[2] Let $ M$ be an orientable manifold. Show that TFAE:
\begin{enumerate}[label=(\arabic*)]
	\item $ \xi$ can be written as $ \ker \alpha$ for some $ \alpha$.
	\item There exists a vector field $ v$ transverse to  $ \xi$ for all $ p \in M$.
	\item $ \xi$ is orientable.
\end{enumerate}
\begin{proof}
	$ (1) \implies (2):$ Let $ \ell := \xi^{\perp}$ be the global line bundle. Then $ \alpha|_{\ell^*}$ is a nonzero smooth section of $ \ell^*$. It follows that $ (\alpha|_{\ell^* })_p$ is a linear isomorphism from $ \ell_p \to \rr$. Therefore, there is a unique vector in $ \ell_p$ that is mapped to $ 1$. By thinking $ \alpha|_{\ell^* }$ as a smooth function from $ \ell \to \rr$ with trivial kernel, we have that $v:= (\alpha|_{\ell^* })^{-1}(1)$ is a smooth vector field in $ \ell$, which is transversal to $ \xi$.

	$ (2) \implies (1)$: Such vector field $ v$ gives a basis for $ \ell$. Construct $ \alpha$ by $ \alpha(\xi) = 0$ and $ \alpha(v) = 1$ (which determines where $ \ell$ is mapped). This is clearly a smooth section with $ \ker \alpha = \xi$ globally.

	$ (2) \implies (3):$ Such vector field $ v$ gives a smoothly varying basis for  $ \ell$, \emph{i.e.} an orientation. Since $ M$ is orientable, we have $ TM$ orientable. As $ \xi^{\perp}$ is also orientable, we can thus orient $ \xi$.

	$ (3) \implies (2):$ If $ \xi$ is orientable, and $ M$ is orientable by assumption, then  $ \ell$ is also orientable. Fix an atlas of $ M$. Given an orientation of $ \ell$ and $ p \in M$, we have a nonzero vector $ v_i$ for each chart $ U_i$ containing $ p$, \emph{i.e.} the basis vector for that chart. Since $ M$ is orientable, all the transition functions between charts are orientation preserving. That is, if $ p$ is in both  $ U_i,U_j$, then $ v_i$ and $ v_j$ differ by a positive scalar. Let $ \{\phi_i\} $ be a partition of unity for the atlas. Then for each point $ p$, we obtain a vector  $ v = \sum \phi_i v_i$ which is never zero since $ v_i$'s all positive scalar multiples of each other. Smoothness is provided by partition of unity, so we have the desired nonvanishing vector field.
\end{proof}

\end{problem}

\begin{problem}[3]
Let $ \alpha_3 = \cos r dz + r\sin r d \theta$. Show that $ \alpha_3 \wedge d \alpha_3 >0$.

\begin{proof}
\begin{align*}
	d \alpha_3 &= d(\cos r) \wedge dz + d(r\sin r) \wedge d\theta\\
	&=- \sin r dr \wedge dz + (\sin r +r \cos r) dr \wedge d\theta
\end{align*}
\begin{align*}
	\alpha_3 \wedge d \alpha_3 &= (\cos r dz + r \sin r d \theta) \wedge (- \sin r dr \wedge dz + (\sin r +r \cos r) dr \wedge d\theta)\\
	&= (\cos r \sin r + r \cos^2r )dz \wedge dr \wedge d\theta - r \sin^2 r d \theta \wedge dr \wedge dz  \\
	&= (\cos r \sin r + r) dz \wedge dr \wedge d\theta \\
	&= \left( \frac{\sin 2r}{ 2r} +1 \right) r dr \wedge d\theta \wedge dz  
\end{align*}
Note that this is already in volume form. We can check that $ \frac{\sin 2r}{2r }+1 >0$ for all $ r>0$. First, if  $ r> \frac{1}{2}$, then since $ |\sin 2r|\leq 1$, we have positivity. If $ 0<r\leq \frac{1}{2}$, then notice that let $ x:=2r$ and by Taylor expansion,
\begin{align*}
	\frac{\sin x}{ x} &= 1 - \frac{x^2}{ 3!} + \frac{x^{4}}{ 5!} -\frac{x^{6}}{ 7!} \ldots
\end{align*}
Clearly each negative term is dominated by the preceding positive term for $ 0<x\leq 1$. The expression thus remains positive.
\end{proof}
\end{problem}

\begin{problem}[4]
Prove Theorem I.4: Legendrian knots have contactomorphic neighborhoods. 
\end{problem}
~\begin{figure}[H]
	\centering
	\includegraphics[width=0.8\textwidth]{./figures/}
	\caption{}
	\label{fig:}
\end{figure}
\begin{proof}
Given two Legendrian knots $ K_1 \subset (M_1,\xi_1), K_2 \subset (M_2,\xi_2)$, we wish to find a diffeomorphism from a neighborhood $ U_1$ of $ K_1$ to a neighborhood $ U_2$ of $ K_2$ that maps $ \xi_1|_{K_1}$ to $ \xi_2|_{K_2}$. Then we finish the proof by wiggling $ U_2$ using the isotopy from Theorem II.1 so that after appropriate shrinking of neighborhoods, $ K_1,K_2$ have contactomorphic neighborhoods.

Take any diffeomorphism $ f: K_1 \to K_2$ (both are circles). Since $ K_i$ are Legendrian, $ T_{K_i}M_i \subset \xi_i$. Fix Riemannian metrics on $ M_i$, then the normal bundles $ \nu(K_i) = \ell_i \oplus \xi_i^{\perp}$, where $ \ell_i$ is the orthogonal complement of $ T_{K_i}M_i$ within $ \xi_i$.  Since $ T_{K_i}M_i$ is an orientable $ S^{1}$ vector bundle, it must be trivial so $ T_{K_i}M_i \cong S^{1} \times \rr^3$. In particular, the fiber can be canonically identified as $ TK_i  \oplus \ell_i\oplus \xi_i^{\perp}$. Choose $ L$ to be a fiberwise linear isomorphism that maps $ \ell_1$ to $ \ell_2$, $\xi_1^{\perp}$ to $\xi_2^{\perp}$. Define $ F: T_{K_1}M_1 \to T_{K_2}M_2, (x,(v,w,z)) \mapsto (f(x),(df_x(v), L(w,z)))$ which is a bundle map.

Here is a fact: the exponential map $ \exp$ yields a diffeomorphism from a neighborhood of the zero section of any submanifold $ N$ in $ \nu(N)$ to a neighborhood of $ N$. This way, we obtain a neighborhood $ U_i$ of $ K_i$ that is diffeomorphic to a neighborhood of the zero section of $ \nu(K_i)$. Then $ \exp|_{U_2} \circ F \circ \exp|_{U_1}^{-1}:U_1 \to U_2$ is a diffeomorphism that takes $ \xi_1|_{U_1}$ to $ \xi_2|_{U_2}$ (after shrinking neighborhoods appropriately).
\end{proof}
\end{document}
