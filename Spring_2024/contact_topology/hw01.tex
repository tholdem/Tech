\documentclass[12pt]{article}
%Fall 2022
% Some basic packages
\usepackage{standalone}[subpreambles=true]
\usepackage[utf8]{inputenc}
\usepackage[T1]{fontenc}
\usepackage{textcomp}
\usepackage[english]{babel}
\usepackage{url}
\usepackage{graphicx}
%\usepackage{quiver}
\usepackage{float}
\usepackage{enumitem}
\usepackage{lmodern}
\usepackage{comment}
\usepackage{hyperref}
\usepackage[usenames,svgnames,dvipsnames]{xcolor}
\usepackage[margin=1in]{geometry}
\usepackage{pdfpages}

\pdfminorversion=7

% Don't indent paragraphs, leave some space between them
\usepackage{parskip}

% Hide page number when page is empty
\usepackage{emptypage}
\usepackage{subcaption}
\usepackage{multicol}
\usepackage[b]{esvect}

% Math stuff
\usepackage{amsmath, amsfonts, mathtools, amsthm, amssymb}
\usepackage{bbm}
\usepackage{stmaryrd}
\allowdisplaybreaks

% Fancy script capitals
\usepackage{mathrsfs}
\usepackage{cancel}
% Bold math
\usepackage{bm}
% Some shortcuts
\newcommand{\rr}{\ensuremath{\mathbb{R}}}
\newcommand{\zz}{\ensuremath{\mathbb{Z}}}
\newcommand{\qq}{\ensuremath{\mathbb{Q}}}
\newcommand{\nn}{\ensuremath{\mathbb{N}}}
\newcommand{\ff}{\ensuremath{\mathbb{F}}}
\newcommand{\cc}{\ensuremath{\mathbb{C}}}
\newcommand{\ee}{\ensuremath{\mathbb{E}}}
\newcommand{\hh}{\ensuremath{\mathbb{H}}}
\renewcommand\O{\ensuremath{\emptyset}}
\newcommand{\norm}[1]{{\left\lVert{#1}\right\rVert}}
\newcommand{\dbracket}[1]{{\left\llbracket{#1}\right\rrbracket}}
\newcommand{\ve}[1]{{\bm{#1}}}
\newcommand\allbold[1]{{\boldmath\textbf{#1}}}
\DeclareMathOperator{\lcm}{lcm}
\DeclareMathOperator{\im}{im}
\DeclareMathOperator{\coim}{coim}
\DeclareMathOperator{\dom}{dom}
\DeclareMathOperator{\tr}{tr}
\DeclareMathOperator{\rank}{rank}
\DeclareMathOperator*{\var}{Var}
\DeclareMathOperator*{\ev}{E}
\DeclareMathOperator{\dg}{deg}
\DeclareMathOperator{\aff}{aff}
\DeclareMathOperator{\conv}{conv}
\DeclareMathOperator{\inte}{int}
\DeclareMathOperator*{\argmin}{argmin}
\DeclareMathOperator*{\argmax}{argmax}
\DeclareMathOperator{\graph}{graph}
\DeclareMathOperator{\sgn}{sgn}
\DeclareMathOperator*{\Rep}{Rep}
\DeclareMathOperator{\Proj}{Proj}
\DeclareMathOperator{\mat}{mat}
\DeclareMathOperator{\diag}{diag}
\DeclareMathOperator{\aut}{Aut}
\DeclareMathOperator{\gal}{Gal}
\DeclareMathOperator{\inn}{Inn}
\DeclareMathOperator{\edm}{End}
\DeclareMathOperator{\Hom}{Hom}
\DeclareMathOperator{\ext}{Ext}
\DeclareMathOperator{\tor}{Tor}
\DeclareMathOperator{\Span}{Span}
\DeclareMathOperator{\Stab}{Stab}
\DeclareMathOperator{\cont}{cont}
\DeclareMathOperator{\Ann}{Ann}
\DeclareMathOperator{\Div}{div}
\DeclareMathOperator{\curl}{curl}
\DeclareMathOperator{\nat}{Nat}
\DeclareMathOperator{\gr}{Gr}
\DeclareMathOperator{\vect}{Vect}
\DeclareMathOperator{\id}{id}
\DeclareMathOperator{\Mod}{Mod}
\DeclareMathOperator{\sign}{sign}
\DeclareMathOperator{\Surf}{Surf}
\DeclareMathOperator{\fcone}{fcone}
\DeclareMathOperator{\Rot}{Rot}
\DeclareMathOperator{\grad}{grad}
\DeclareMathOperator{\atan2}{atan2}
\DeclareMathOperator{\Ric}{Ric}
\let\vec\relax
\DeclareMathOperator{\vec}{vec}
\let\Re\relax
\DeclareMathOperator{\Re}{Re}
\let\Im\relax
\DeclareMathOperator{\Im}{Im}
% Put x \to \infty below \lim
\let\svlim\lim\def\lim{\svlim\limits}

%wide hat
\usepackage{scalerel,stackengine}
\stackMath
\newcommand*\wh[1]{%
\savestack{\tmpbox}{\stretchto{%
  \scaleto{%
    \scalerel*[\widthof{\ensuremath{#1}}]{\kern-.6pt\bigwedge\kern-.6pt}%
    {\rule[-\textheight/2]{1ex}{\textheight}}%WIDTH-LIMITED BIG WEDGE
  }{\textheight}% 
}{0.5ex}}%
\stackon[1pt]{#1}{\tmpbox}%
}
\parskip 1ex

%Make implies and impliedby shorter
\let\implies\Rightarrow
\let\impliedby\Leftarrow
\let\iff\Leftrightarrow
\let\epsilon\varepsilon

% Add \contra symbol to denote contradiction
\usepackage{stmaryrd} % for \lightning
\newcommand\contra{\scalebox{1.5}{$\lightning$}}

% \let\phi\varphi

% Command for short corrections
% Usage: 1+1=\correct{3}{2}

\definecolor{correct}{HTML}{009900}
\newcommand\correct[2]{\ensuremath{\:}{\color{red}{#1}}\ensuremath{\to }{\color{correct}{#2}}\ensuremath{\:}}
\newcommand\green[1]{{\color{correct}{#1}}}

% horizontal rule
\newcommand\hr{
    \noindent\rule[0.5ex]{\linewidth}{0.5pt}
}

% hide parts
\newcommand\hide[1]{}

% si unitx
\usepackage{siunitx}
\sisetup{locale = FR}

%allows pmatrix to stretch
\makeatletter
\renewcommand*\env@matrix[1][\arraystretch]{%
  \edef\arraystretch{#1}%
  \hskip -\arraycolsep
  \let\@ifnextchar\new@ifnextchar
  \array{*\c@MaxMatrixCols c}}
\makeatother

\renewcommand{\arraystretch}{0.8}

\renewcommand{\baselinestretch}{1.5}

\usepackage{graphics}
\usepackage{epstopdf}

\RequirePackage{hyperref}
%%
%% Add support for color in order to color the hyperlinks.
%% 
\hypersetup{
  colorlinks = true,
  urlcolor = blue,
  citecolor = blue
}
%%fakesection Links
\hypersetup{
    colorlinks,
    linkcolor={red!50!black},
    citecolor={green!50!black},
    urlcolor={blue!80!black}
}
%customization of cleveref
\RequirePackage[capitalize,nameinlink]{cleveref}[0.19]

% Per SIAM Style Manual, "section" should be lowercase
\crefname{section}{section}{sections}
\crefname{subsection}{subsection}{subsections}
\Crefname{section}{Section}{Sections}
\Crefname{subsection}{Subsection}{Subsections}

% Per SIAM Style Manual, "Figure" should be spelled out in references
\Crefname{figure}{Figure}{Figures}

% Per SIAM Style Manual, don't say equation in front on an equation.
\crefformat{equation}{\textup{#2(#1)#3}}
\crefrangeformat{equation}{\textup{#3(#1)#4--#5(#2)#6}}
\crefmultiformat{equation}{\textup{#2(#1)#3}}{ and \textup{#2(#1)#3}}
{, \textup{#2(#1)#3}}{, and \textup{#2(#1)#3}}
\crefrangemultiformat{equation}{\textup{#3(#1)#4--#5(#2)#6}}%
{ and \textup{#3(#1)#4--#5(#2)#6}}{, \textup{#3(#1)#4--#5(#2)#6}}{, and \textup{#3(#1)#4--#5(#2)#6}}

% But spell it out at the beginning of a sentence.
\Crefformat{equation}{#2Equation~\textup{(#1)}#3}
\Crefrangeformat{equation}{Equations~\textup{#3(#1)#4--#5(#2)#6}}
\Crefmultiformat{equation}{Equations~\textup{#2(#1)#3}}{ and \textup{#2(#1)#3}}
{, \textup{#2(#1)#3}}{, and \textup{#2(#1)#3}}
\Crefrangemultiformat{equation}{Equations~\textup{#3(#1)#4--#5(#2)#6}}%
{ and \textup{#3(#1)#4--#5(#2)#6}}{, \textup{#3(#1)#4--#5(#2)#6}}{, and \textup{#3(#1)#4--#5(#2)#6}}

% Make number non-italic in any environment.
\crefdefaultlabelformat{#2\textup{#1}#3}

% Environments
\makeatother
% For box around Definition, Theorem, \ldots
%%fakesection Theorems
\usepackage{thmtools}
\usepackage[framemethod=TikZ]{mdframed}

\theoremstyle{definition}
\mdfdefinestyle{mdbluebox}{%
	roundcorner = 10pt,
	linewidth=1pt,
	skipabove=12pt,
	innerbottommargin=9pt,
	skipbelow=2pt,
	nobreak=true,
	linecolor=blue,
	backgroundcolor=TealBlue!5,
}
\declaretheoremstyle[
	headfont=\sffamily\bfseries\color{MidnightBlue},
	mdframed={style=mdbluebox},
	headpunct={\\[3pt]},
	postheadspace={0pt}
]{thmbluebox}

\mdfdefinestyle{mdredbox}{%
	linewidth=0.5pt,
	skipabove=12pt,
	frametitleaboveskip=5pt,
	frametitlebelowskip=0pt,
	skipbelow=2pt,
	frametitlefont=\bfseries,
	innertopmargin=4pt,
	innerbottommargin=8pt,
	nobreak=false,
	linecolor=RawSienna,
	backgroundcolor=Salmon!5,
}
\declaretheoremstyle[
	headfont=\bfseries\color{RawSienna},
	mdframed={style=mdredbox},
	headpunct={\\[3pt]},
	postheadspace={0pt},
]{thmredbox}

\declaretheorem[%
style=thmbluebox,name=Theorem,numberwithin=section]{thm}
\declaretheorem[style=thmbluebox,name=Lemma,sibling=thm]{lem}
\declaretheorem[style=thmbluebox,name=Proposition,sibling=thm]{prop}
\declaretheorem[style=thmbluebox,name=Corollary,sibling=thm]{coro}
\declaretheorem[style=thmredbox,name=Example,sibling=thm]{eg}

\mdfdefinestyle{mdgreenbox}{%
	roundcorner = 10pt,
	linewidth=1pt,
	skipabove=12pt,
	innerbottommargin=9pt,
	skipbelow=2pt,
	nobreak=true,
	linecolor=ForestGreen,
	backgroundcolor=ForestGreen!5,
}

\declaretheoremstyle[
	headfont=\bfseries\sffamily\color{ForestGreen!70!black},
	bodyfont=\normalfont,
	spaceabove=2pt,
	spacebelow=1pt,
	mdframed={style=mdgreenbox},
	headpunct={ --- },
]{thmgreenbox}

\declaretheorem[style=thmgreenbox,name=Definition,sibling=thm]{defn}

\mdfdefinestyle{mdgreenboxsq}{%
	linewidth=1pt,
	skipabove=12pt,
	innerbottommargin=9pt,
	skipbelow=2pt,
	nobreak=true,
	linecolor=ForestGreen,
	backgroundcolor=ForestGreen!5,
}
\declaretheoremstyle[
	headfont=\bfseries\sffamily\color{ForestGreen!70!black},
	bodyfont=\normalfont,
	spaceabove=2pt,
	spacebelow=1pt,
	mdframed={style=mdgreenboxsq},
	headpunct={},
]{thmgreenboxsq}
\declaretheoremstyle[
	headfont=\bfseries\sffamily\color{ForestGreen!70!black},
	bodyfont=\normalfont,
	spaceabove=2pt,
	spacebelow=1pt,
	mdframed={style=mdgreenboxsq},
	headpunct={},
]{thmgreenboxsq*}

\mdfdefinestyle{mdblackbox}{%
	skipabove=8pt,
	linewidth=3pt,
	rightline=false,
	leftline=true,
	topline=false,
	bottomline=false,
	linecolor=black,
	backgroundcolor=RedViolet!5!gray!5,
}
\declaretheoremstyle[
	headfont=\bfseries,
	bodyfont=\normalfont\small,
	spaceabove=0pt,
	spacebelow=0pt,
	mdframed={style=mdblackbox}
]{thmblackbox}

\theoremstyle{plain}
\declaretheorem[name=Question,sibling=thm,style=thmblackbox]{ques}
\declaretheorem[name=Remark,sibling=thm,style=thmgreenboxsq]{remark}
\declaretheorem[name=Remark,sibling=thm,style=thmgreenboxsq*]{remark*}
\newtheorem{ass}[thm]{Assumptions}

\theoremstyle{definition}
\newtheorem*{problem}{Problem}
\newtheorem{claim}[thm]{Claim}
\theoremstyle{remark}
\newtheorem*{case}{Case}
\newtheorem*{notation}{Notation}
\newtheorem*{note}{Note}
\newtheorem*{motivation}{Motivation}
\newtheorem*{intuition}{Intuition}
\newtheorem*{conjecture}{Conjecture}

% Make section starts with 1 for report type
%\renewcommand\thesection{\arabic{section}}

% End example and intermezzo environments with a small diamond (just like proof
% environments end with a small square)
\usepackage{etoolbox}
\AtEndEnvironment{vb}{\null\hfill$\diamond$}%
\AtEndEnvironment{intermezzo}{\null\hfill$\diamond$}%
% \AtEndEnvironment{opmerking}{\null\hfill$\diamond$}%

% Fix some spacing
% http://tex.stackexchange.com/questions/22119/how-can-i-change-the-spacing-before-theorems-with-amsthm
\makeatletter
\def\thm@space@setup{%
  \thm@preskip=\parskip \thm@postskip=0pt
}

% Fix some stuff
% %http://tex.stackexchange.com/questions/76273/multiple-pdfs-with-page-group-included-in-a-single-page-warning
\pdfsuppresswarningpagegroup=1


% My name
\author{Jaden Wang}



\begin{document}
\centerline {\textsf{\textbf{\LARGE{Homework 1}}}}
\centerline {Jaden Wang}
\vspace{.15in}
\begin{problem}[1]
Show that all plane fields can be locally written as $ \ker \alpha$ for some $ \alpha$.

\begin{proof}
Fix a Riemannian metric on $ M$ so we have the notion of orthogonality. Given $ p \in M$, take a contractible neighborhood $ U$ of  $ p$ and let  $ \ell := (\xi|_U)^{\perp}$ be the line bundle formed by orthogonal complements of $ \xi$ over $ U$. Since $ U$ is contractible, $ \ell \cong U \times \rr$. Take any nonzero smooth section $ s$ of $ \ell$, say $s: x \mapsto (x,1) $. Now define a 1-form $ \alpha$ of $ U$ by
\begin{align*}
	\alpha: U \to T^*U = (\xi|_U)^* \oplus \ell^*, x \mapsto (0,s),
\end{align*}
where $ 0: \xi|_U \to \rr$ is the zero function. From this definition, it is clear that $ \alpha$ is a smooth section of $ T^* U$ and thus a smooth 1-form and $ \xi|_U = \ker \alpha$ as desired.
\end{proof}
\end{problem}
\begin{problem}[2] Let $ M$ be an orientable manifold. Show that TFAE:
\begin{enumerate}[label=(\arabic*)]
	\item $ \xi$ can be written as $ \ker \alpha$ for some $ \alpha$.
	\item There exists a vector field $ v$ transverse to  $ \xi$ for all $ p \in M$.
	\item $ \xi$ is orientable.
\end{enumerate}
\begin{proof}
	$ (1) \implies (2):$ Let $ \ell := \xi^{\perp}$ be the global line bundle. Then $ \alpha|_{\ell^*}$ is a nonzero smooth section of $ \ell^*$. It follows that $ (\alpha|_{\ell^* })_p$ is a linear isomorphism from $ \ell_p \to \rr$. Therefore, there is a unique vector in $ \ell_p$ that is mapped to $ 1$. By thinking $ \alpha|_{\ell^* }$ as a smooth function from $ \ell \to \rr$ with trivial kernel, we have that $v:= (\alpha|_{\ell^* })^{-1}(1)$ is a smooth vector field in $ \ell$, which is transversal to $ \xi$.

	$ (2) \implies (1)$: Such vector field $ v$ gives a basis for $ \ell$. Construct $ \alpha$ by $ \alpha(\xi) = 0$ and $ \alpha(v) = 1$ (which determines where $ \ell$ is mapped). This is clearly a smooth section with $ \ker \alpha = \xi$ globally.

	$ (2) \implies (3):$ Such vector field $ v$ gives a smoothly varying basis for  $ \ell$, \emph{i.e.} an orientation. Since $ M$ is orientable, we have $ TM$ orientable. As $ \xi^{\perp}$ is also orientable, we can thus orient $ \xi$.

	$ (3) \implies (2):$ If $ \xi$ is orientable, and $ M$ is orientable by assumption, then  $ \ell$ is also orientable. Fix an atlas of $ M$. Given an orientation of $ \ell$ and $ p \in M$, we have a nonzero vector $ v_i$ for each chart $ U_i$ containing $ p$, \emph{i.e.} the basis vector for that chart. Since $ M$ is orientable, all the transition functions between charts are orientation preserving. That is, if $ p$ is in both  $ U_i,U_j$, then $ v_i$ and $ v_j$ differ by a positive scalar. Let $ \{\phi_i\} $ be a partition of unity for the atlas. Then for each point $ p$, we obtain a vector  $ v = \sum \phi_i v_i$ which is never zero since $ v_i$'s all positive scalar multiples of each other. Smoothness is provided by partition of unity, so we have the desired nonvanishing vector field.
\end{proof}

\end{problem}

\begin{problem}[3]
Let $ \alpha_3 = \cos r dz + r\sin r d \theta$. Show that $ \alpha_3 \wedge d \alpha_3 >0$.

\begin{proof}
\begin{align*}
	d \alpha_3 &= d(\cos r) \wedge dz + d(r\sin r) \wedge d\theta\\
	&=- \sin r dr \wedge dz + (\sin r +r \cos r) dr \wedge d\theta
\end{align*}
\begin{align*}
	\alpha_3 \wedge d \alpha_3 &= (\cos r dz + r \sin r d \theta) \wedge (- \sin r dr \wedge dz + (\sin r +r \cos r) dr \wedge d\theta)\\
	&= (\cos r \sin r + r \cos^2r )dz \wedge dr \wedge d\theta - r \sin^2 r d \theta \wedge dr \wedge dz  \\
	&= (\cos r \sin r + r) dz \wedge dr \wedge d\theta \\
	&= \left( \frac{\sin 2r}{ 2r} +1 \right) r dr \wedge d\theta \wedge dz  
\end{align*}
Note that this is already in volume form. We can check that $ \frac{\sin 2r}{2r }+1 >0$ for all $ r>0$. First, if  $ r> \frac{1}{2}$, then since $ |\sin 2r|\leq 1$, we have positivity. If $ 0<r\leq \frac{1}{2}$, then notice that let $ x:=2r$ and by Taylor expansion,
\begin{align*}
	\frac{\sin x}{ x} &= 1 - \frac{x^2}{ 3!} + \frac{x^{4}}{ 5!} -\frac{x^{6}}{ 7!} \ldots
\end{align*}
Clearly each negative term is dominated by the preceding positive term for $ 0<x\leq 1$. The expression thus remains positive.
\end{proof}
\end{problem}

\begin{problem}[4]
Prove Theorem I.4: Legendrian knots have contactomorphic neighborhoods. 
\end{problem}
~\begin{figure}[H]
	\centering
	\includegraphics[width=0.8\textwidth]{./figures/}
	\caption{}
	\label{fig:}
\end{figure}
\begin{proof}
Given two Legendrian knots $ K_1 \subset (M_1,\xi_1), K_2 \subset (M_2,\xi_2)$, we wish to find a diffeomorphism from a neighborhood $ U_1$ of $ K_1$ to a neighborhood $ U_2$ of $ K_2$ that maps $ \xi_1|_{K_1}$ to $ \xi_2|_{K_2}$. Then we finish the proof by wiggling $ U_2$ using the isotopy from Theorem II.1 so that after appropriate shrinking of neighborhoods, $ K_1,K_2$ have contactomorphic neighborhoods.

Take any diffeomorphism $ f: K_1 \to K_2$ (both are circles). Since $ K_i$ are Legendrian, $ T_{K_i}M_i \subset \xi_i$. Fix Riemannian metrics on $ M_i$, then the normal bundles $ \nu(K_i) = \ell_i \oplus \xi_i^{\perp}$, where $ \ell_i$ is the orthogonal complement of $ T_{K_i}M_i$ within $ \xi_i$.  Since $ T_{K_i}M_i$ is an orientable $ S^{1}$ vector bundle, it must be trivial so $ T_{K_i}M_i \cong S^{1} \times \rr^3$. In particular, the fiber can be canonically identified as $ TK_i  \oplus \ell_i\oplus \xi_i^{\perp}$. Choose $ L$ to be a fiberwise linear isomorphism that maps $ \ell_1$ to $ \ell_2$, $\xi_1^{\perp}$ to $\xi_2^{\perp}$. Define $ F: T_{K_1}M_1 \to T_{K_2}M_2, (x,(v,w,z)) \mapsto (f(x),(df_x(v), L(w,z)))$ which is a bundle map.

Here is a fact: the exponential map $ \exp$ yields a diffeomorphism from a neighborhood of the zero section of any submanifold $ N$ in $ \nu(N)$ to a neighborhood of $ N$. This way, we obtain a neighborhood $ U_i$ of $ K_i$ that is diffeomorphic to a neighborhood of the zero section of $ \nu(K_i)$. Then $ \exp|_{U_2} \circ F \circ \exp|_{U_1}^{-1}:U_1 \to U_2$ is a diffeomorphism that takes $ \xi_1|_{U_1}$ to $ \xi_2|_{U_2}$ (after shrinking neighborhoods appropriately).
\end{proof}
\end{document}
