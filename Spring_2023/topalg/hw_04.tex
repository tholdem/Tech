\documentclass[12pt]{article}
\newcommand{\alert}[1]{{\bf \color{red} [Alert:] #1}}
\newcommand{\todo}[1]{{\bf \color{orange} [TODO:] #1}}
\newcommand{\real}[1][]{\mathbb{R}^{#1}}
\newcommand{\myeqn}[1]{(\ref{#1})}
\newcommand{\myex}[1]{Example \ref{#1}}
\newcommand{\defeq}{\stackrel{\mathrm{def}}{=}}
\newcommand{\parder}[2]{\frac{\partial #1}{\partial #2}}
\newcommand{\Lie}[3][]{\mathsf{L}_{#3}^{#1} #2}
\newcommand{\LieA}[1]{\mathsf{Lie}(#1)}
\newcommand{\lieder}[2]{\mathcal{L}_{#2} #1}
\renewcommand{\t}{^{\mbox{\tiny\sf T}}}
\newcommand{\trans}{^{\mbox{\tiny\sf T}}}
\newcommand{\markup}[1]{\{\textbf{#1}\}}
\newcommand{\msub}[1]{_\mathrm{#1}}
\newcommand{\msup}[1]{^\mathrm{#1}}
\newcommand{\inv}[1]{#1^{-1}}
\newcommand{\pinv}[1]{{#1}^{+}}
\newcommand{\myfracA}[2]{\displaystyle{\frac{#1}{#2}}}
\newcommand{\myfracB}[2]{{#1}/{#2}}
\newcommand{\mydiffA}[1]{\dot{#1}}
\newcommand{\mydiffB}[2]{\myfracA{\mathrm{d}{#1}}{\mathrm{d}{#2}}}
\newcommand{\ball}[2]{\mathcal{B}_{#1}\left(#2\right)}
\newcommand{\acos}[1]{\cos^{-1}\left(#1\right)}
\newcommand{\asin}[1]{\sin^{-1}\left(#1\right)}
\newcommand{\mani}{\mathcal{M}}
\newcommand{\tang}[2]{\mathsf{T}_{#1} #2}
\newcommand{\LieB}[2]{[ #1, #2 ]}
\newcommand{\LieBad}[3][]{\mathsf{ad}_{#2}^{#1} #3}
\newcommand{\ReachVT}{\mathcal{R}^V_T}
\newcommand{\ReachVt}{\mathcal{R}^V_t}
\newcommand{\ReachVTe}{\mathcal{R}^V_{\le T}}
\newcommand{\ReachT}{\mathcal{R}_T}
\newcommand{\Reacht}{\mathcal{R}_t}
\newcommand{\ReachTe}{\mathcal{R}_{\le T}}
\newcommand{\accLA}[1]{\mathsf{Lie}(#1)}
\newcommand{\accD}{\Delta_{\mathcal{F}}}
\newcommand{\accSA}{\mathsf{Lie}(\mathcal{G},f)}
\newcommand{\accDS}{\Delta_{\mathcal{G}}}
\newcommand{\eval}[3]{\mathsf{Ev}^{#2}_{#1}\left( #3 \right)}
\newcommand{\stlc}{\textsc{stlc}}
\newcommand{\clf}{\textsc{clf}}
\newcommand{\jqlf}{\textsc{jqlf}}
\newcommand{\dlas}{\textsc{dlas}}
\newcommand{\Ad}[2]{\mathsf{Ad}_{#1} #2}
\newcommand{\xe}{\ensuremath{x_e}}
\newcommand{\lebg}[1]{\mathcal{L}_{#1}}
\newcommand{\lebgx}[1]{\mathcal{L}_{#1 \mathrm{e}}}
\newcommand{\dom}{D}
\newcommand{\domT}{[t_0,\infty) \times D}
\newcommand{\rarrow}{\rightarrow}
\renewcommand{\d}{\mathrm{d}}
\renewcommand{\Re}{\mathbb{R}}
\newcommand{\C}{\mathrm{C}}

\newcommand{\QED}{{\unskip\nobreak\hfil\penalty50\hskip2em\vadjust{}
		\nobreak\hfil$\Box$\parfillskip=0pt\finalhyphendemerits=0\par}\vspace{0.1cm}}
\newcommand{\eoEx}{{\unskip\nobreak\hfil\penalty50\hskip0em\vadjust{}
		\nobreak\hfil$\Large\Diamond$\parfillskip=0pt\finalhyphendemerits=0\par}\vspace{0.1cm}}

\newcommand{\sgn}{\ensuremath{\operatorname{sgn}}}
\newcommand{\sat}{\ensuremath{\operatorname{sat}}}

\newcommand{\half}{\frac{1}{2}}
\newcommand{\shalf}{\mbox{$\frac{1}{2}$}}
\newcommand{\marcom}[1]{\marginpar{\footnotesize #1}}
\newcommand{\der}{\mathrm{D}}
\newcommand{\e}{\mathrm{e}}
\newcommand{\dt}{\mathrm{d}t}

\newcommand{\cA}{\ensuremath{\mathcal{A}}}
\newcommand{\cB}{\ensuremath{\mathcal{B}}}
\newcommand{\cG}{\ensuremath{\mathcal{G}}}
\newcommand{\cK}{\ensuremath{\mathcal{K}}}
\newcommand{\cW}{\ensuremath{\mathcal{W}}}
\newcommand{\cZ}{\ensuremath{\mathcal{Z}}}
\newcommand{\cS}{\ensuremath{\mathcal{S}}}
\newcommand{\cD}{\ensuremath{\mathcal{D}}}
\newcommand{\cP}{\ensuremath{\mathcal{P}}}
\newcommand{\cV}{\ensuremath{\mathcal{V}}}
\newcommand{\cL}{\ensuremath{\mathcal{L}}}
\newcommand{\cN}{\ensuremath{\mathcal{N}}}
\newcommand{\cI}{\ensuremath{\mathcal{I}}}
\newcommand{\cR}{\ensuremath{\mathcal{R}}}
\newcommand{\cM}{\ensuremath{\mathcal{M}}}
\newcommand{\cC}{\ensuremath{\mathcal{C}}}
\newcommand{\cF}{\ensuremath{\mathcal{F}}}
\newcommand{\cH}{\ensuremath{\mathcal{H}}}
\newcommand{\cO}{\ensuremath{\mathcal{O}}}
\newcommand{\cX}{\ensuremath{\mathcal{X}}}
\newcommand{\cY}{\ensuremath{\mathcal{Y}}}
\newcommand{\Ci}{\ensuremath{\mathcal{C}^\infty}}
\newcommand{\ISS}{\textsc{iss}}
\newcommand{\LISS}{\textsc{liss}}
\newcommand{\GAS}{\textsc{gas}}
\newcommand{\GS}{\textsc{gs}}
\newcommand{\LES}{\textsc{les}}
\newcommand{\GUAS}{\textsc{guas}}
\newcommand{\BIBO}{\textsc{bibo}}
\newcommand{\spec}{\ensuremath{\operatorname{spec}}}
\newcommand{\spn}{\ensuremath{\operatorname{span}}}
\renewcommand{\i}{\mathrm{i\,}}

\renewcommand{\implies}{\Rightarrow}

\renewcommand{\theenumi}{$\roman{enumi})$}
\renewcommand{\labelenumi}{\theenumi}

\font\ptmten=zptmcmrm scaled 1200
\newcommand{\w}{\mbox{{\ptmten w}}}
\newcommand{\z}{\mbox{{\ptmten z}}}
\renewcommand{\Re}{\mathbb{R}}

\newcommand{\cl}{\operatorname{cl}}
\newcommand{\intr}{\operatorname{int}}
\newcommand{\rank}{\operatorname{rank}}
\newcommand{\co}{\operatorname{co}}
\newcommand{\aff}{\operatorname{aff}}

\theoremstyle{plain}
\newtheorem{theorem}{Theorem}[chapter]
\newtheorem{claim}[theorem]{Claim}
\newtheorem{corollary}[theorem]{Corollary}
\newtheorem{prop}[theorem]{Proposition}
\newtheorem{fact}[theorem]{Fact}
\newtheorem{lemma}[theorem]{Lemma}

\newtheorem{remark}{Remark}[chapter]

\theoremstyle{definition}
\newtheorem{assume}[theorem]{Assumption}
\newtheorem{defn}[theorem]{Definition}
\newtheorem{problem}[theorem]{Problem}
\newtheorem{exercise}{Exercise}
\newtheorem{example}[theorem]{Example}


\begin{document}
\centerline {\textsf{\textbf{\LARGE{Homework 4}}}}
\centerline {Jaden Wang}
\vspace{.15in}
\begin{problem}[1]
To show that $ (M_*(f), \partial_f) $ is a complex, we need $ \partial _f^2 = 0$. Given $ a \in A_{n-1}, b \in B_n$,
\begin{align*}
	\partial _f(a,b) &= \left(  \partial _A^2(a), \partial _B(\partial _B(b)+f_{n-1}(a)) -f_{n-2} \partial _A(a)\right)     \\
			 &= (0, 0+ (\partial _B f_{n-1} - f_{n-2} \partial _A (a)) && A,B \text{ complex} \\
	&= (0,0) && f_n \text{ chain map} 
\end{align*}
Let $ \textsf{S} : \textsf{Chain} \to \textsf{Chain}$ be the shift functor that increases the chain index by 1 and negate the morphism. That is, $ A_*^{+} := \textsf{S}(A_*) = A_{*-1}$ with $\partial _{A^{+}} : =S(\partial_A) = - \partial_A$. Consider the short exact sequence
\begin{align*}
	0 \to B_* \xrightarrow{ i} M_*(f) \xrightarrow{ j} A_{*}^{+} \to 0 
\end{align*}
where $ i$ is the obvious inclusion map and  $ j: A_{*-1} \oplus B_* \to A_{*-1}$ is the obvious projection map whose kernel is exactly $ B_*$. Note that $ i,j$ are chain maps. Take $ b \in B_n$, then $ i \circ \partial _B (b) = (0, \partial _B(b)) = \partial _f(0,b) =\partial _f \circ i(b)$. Take $ (a,b)$ with $ a \in A_n^{+} = A_{n-1}$ and $ b \in B_n$, we see that $ j \circ \partial _f (a,b) = j( \partial_A^{+} (a),*) = \partial _A^{+} (a) = \partial _A^{+} \circ j(a,b)$. Then the snake lemma yields a long exact sequence as stated. It remains to check that the connecting homomorphism $ \partial_* = f_{*}$. Given $ [a] \in H_n(A_*^{+}) = H_{n-1}(A_*)$, we have $ j_*([(a,0)]) = a$ and that $ [(a,0)] $. By the definition of $ \partial _*: H_n(A_*^{+}) \to H_{n-1}(B_*)$, we have $ \partial _*([a]) = [i ^{-1} \partial _f(a,0)] = [i^{-1} (\partial _A(a),0+f_{n-1}(a))] = [f_{n-1}(a)] = f_*([a])$ since $ i^{-1}$ only picks out the second component. If $ H_n(M_*(f)) = 0$ for all $ n$, then we have $0 \to H_{n+1}(A_*^{+}) = H_n (A_*) \to H_n(B_*) \to 0$ for all $ n$, which is an isomorphism  on homology.
\end{problem}

\begin{problem}[2]
Let $ X = S^{1} \vee S^{1}$ and $ Y$ be the 2-fold cover as figure.
 ~\begin{figure}[H]
	\centering
	\includegraphics[width=0.4\textwidth]{./figures/2-cover-8.png}
	\caption{}
\end{figure}
Consider the 1-simplices $ a_1: \Delta_1 \to Y$ and $ a_2: \Delta_1 \to Y$. Since $ Y$ is 1-dimensional, the image of any 2-simplex must be at most 1-dimensional, and the boundary must be at most 0-dimensional. Thus $ a_1 - a_2$ is 1-dimensional and not a boundary of a 2-simplex, \emph{i.e.} they are distinct elements in the homology. However, since $ p \circ a_1 = a = p \circ a_2$, $ p_*([ a_1]) = [ a] = p_*([ a_2])$, showing that $ p_*$ is not injective.
\end{problem}
\begin{problem}[4]
Note that any possible pair is a good pair in this problem, since contractible implies that we can just treat the subspace as a point, and the subspace deformation retracts to the point via contractibility. Denote $ X_{12} := X_1 \cup X_2$. First I claim that $ H_n(X_{12}) = 0$ for $ n\geq 2$. If  $ X_2$ is empty then it is trivially true. If $ X_2$ is contractible, we have
\begin{align*}
	H_n(X_{12}) \cong \widetilde{ H}_n(X_{12}/X_2) \cong \widetilde{ H}_n(X_1 / (X_1 \cap X_2)) = \widetilde{ H}_n(*) = 0
\end{align*}
Next, I claim that $ H_n(X_{12} \cap X_3) = 0$ for $ n \geq 1$. Suppose $ (X_1 \cap X_3) \cap (X_2 \cap X_3) = X_1 \cap X_2 \cap X_3 = \O $. Then by additivity, $ H_n(X_{12} \cap X_3) = H_n(X_1 \cap X_3) \oplus  H_n(X_2 \cap X_3) = 0 \oplus 0 = 0$. Suppose the three-way intersection is not empty, WLOG we can assume both two-way intersections are contractible (if they are empty we can use additivity again to get trivial homology), by Mayer-Vietoris on the two-way intersections in the obvious way we obtain $ H_n(X_{12} \cap X_3) = 0 $.

Finally, by Mayer-Vietoris, for $ n \geq 2$ we have
\begin{align*}
	H_n(X_{12} \cap X_3)=0 \to H_n(X_{12}) \oplus H_n(X_3) = 0 \to H_n(X) \to H_{n-1}(X_{12} \cap X_3) = 0
\end{align*}
So $ H_n(X) =0$.
\end{problem}

\begin{problem}[8]
Recall that a cone is contractible via the straight-line homotopy to the cone tip. Denote the top cone $ C_+(X)$ and bottom cone  $ C_-(X)$. Let  $ U_+$ be the top cone with some ``open skirt" and $ U_-$ be the bottom cone with open skirt. They union to  $ SX$ and deformation retract to their respective cone.
~\begin{figure}[H]
	\centering
	\includegraphics[width=0.8\textwidth]{./figures/MV_suspension.png}
\end{figure}

Thus $ H_n(C_{\pm}X) \cong H_n(U_{\pm}) = 0$ for $ n \geq 1$ and  $ \zz$ for $ n=0$. Moreover,  $ U_+ \cap U_- \simeq X$ so $ H_n(U_+ \cap U_-) \cong H_n(X)$. Now apply Mayer-Vietoris for $ n \geq 1$:
\begin{align*}
	\to \underbrace{ H_{n+1}(U_+) \oplus H_{n+1}(U_-)}_{ 0} \to H_{n+1}(SX) \to \underbrace{ H_n(U_+ \cap U_-) }_{ \cong H_n(X)} \to \underbrace{ H_n(U_+) \oplus H_n(U_-)}_{ 0} \to
\end{align*}
So $ H_{n+1}(SX) \cong H_n(X)$ for $ n \geq 1$, and the reduced homology coincide with homology for this range. Outside this range, we have
\begin{align*}
	\to \underbrace{ H_{1}(U_+) \oplus H_{1}(U_-)}_{ 0} \to H_{1}(SX) \xrightarrow{ \partial }  \underbrace{ H_0(U_+ \cap U_-) }_{ \cong H_0(X)} \xrightarrow{ \phi}  \underbrace{ H_0(U_+) \oplus H_0(U_-)}_{ \zz \oplus \zz} \xrightarrow{ \psi}  \underbrace{ H_0(SX)}_{ \cong \zz}\to 0
\end{align*}
Since $ \psi$ is surjective, exactness and 1st isomorphism theorem yield $ \zz^2 / \im \phi \cong \zz$. Since $ \zz$ is free, we have a split short exact sequence $ 0 \to \im \phi \to \zz^2 \to \zz \to 0$ and hence $ \im \phi \oplus \zz \cong \zz^2$. Thus $ H_0(X) / \ker \phi \cong \im \phi \cong \zz$. By the same argument, $ H_0(X) \cong \ker \phi \oplus \zz$ so $ \widetilde{ H}_0(X) \cong \ker \phi = \im \partial \cong H_1(SX) = \widetilde{ H}_1(SX)$ since $ \partial $ is injective.

Finally, since $ SX$ is path-connected (can always connect two points via the cone tip),  $ \widetilde{ H}_0(SX) = 0 = \widetilde{ H}_{-1}(X)$.
\end{problem}

\begin{problem}[9]
We already know that $ T^2 := S^{1} \times S^{1}$ have
\begin{align*}
	H_n(T^2) &= \begin{cases}
		\zz^2 & n =2\\
		\zz & n=0,1\\
	\end{cases} \\
\end{align*}
Denote $ X = S^{1} \vee S^{1} \vee S^{2}$. Let $ A$ and  $ B$ be as shown in the figure. Clearly  $ A \simeq S^2$, $ B \simeq S^{1} \vee S^{1}$, $ A \cap B$ is contractible.
~\begin{figure}[H]
	\centering
	\includegraphics[width=0.8\textwidth]{./figures/MV_wedge.png}
\end{figure}
For $ i>2$, we have
\begin{align*}
	\cdots \to \underbrace{ H_i(A \cap B)}_{0 }  \to \underbrace{ H_i(A) \oplus H_i(B) }_{ 0 \oplus 0} \to H_i(X) \to \underbrace{ H_{i-1}(A \cap B)}_{ 0}  \to \cdots
\end{align*}
which implies $ H_i(X) = 0$. Else,
\begin{align*}
	\underbrace{ H_2(A \cap B)}_{ 0} \xrightarrow{ \phi_2}  \underbrace{ H_2(A) \oplus H_2(B)}_{ \cong \zz \oplus 0} \xrightarrow{ \psi_2}  H_2(X) \xrightarrow{ \partial_2}  \underbrace{ H_1(A \cap B)}_{ 0} \xrightarrow{ \phi_1}  \underbrace{ H_1(A) \oplus H_1(B)}_{ 0 \oplus \zz^2} \xrightarrow{ \psi_2}  H_1(X) \xrightarrow{ \partial _1}\\  
	\underbrace{ H_0(A \cap B)}_{ \cong \zz} \xrightarrow{ \phi_0} \underbrace{ H_0(A) \oplus H_0(B)}_{ \zz \oplus \zz} \to \cdots   
\end{align*}
We immediately have $ H_2(X) \cong H_2(A)\oplus H_2(B) \cong \zz$. Moreover, $ \psi_1$ is injective so $ \zz^2 \cong \im \psi_1 = \ker \partial _1 $. We see that $ \phi_0:1 \mapsto (1,1)$ is also injective, so $ \im \partial _1 = \ker \phi_0 = 0$, therefore by first isomorphism theorem, $ 0 = \im \partial _1 \cong H_1(X) / \ker \partial _1 =  H_1(X) / \zz^2$ which implies that $ H_1(X) = \zz^2$. Finally, $ X$ is clearly path-connected so  $ H_0(X) \cong \zz$. Therefore, the homology of $ X$ coincides with  $ T^2$. However, the universal cover of $ T^2$ is $ \rr^2$ which is contractible, yet the universal cover of $ X$ is the Caley tree of  $ F_2$ where each vertex wedges a $ S^2$, so it is homotopy equivalent to an infinity wedge of circles $ \bigvee^{ \infty} S^2$ by quotienting out the contractible Caley tree. Notice that we can apply Mayer-Vietoris on $ \bigvee^ \infty S^2 $ by letting $ A $ bean open set containing exactly one sphere, and  $ B$ be an open set containing the rest. Clearly  $ A \cap B$ is contractible, so it yields
\begin{align*}
	0 \to H_2(A) \oplus H_2(B) \to H_2(\bigvee^ \infty S^2) \to 0  
\end{align*}
That is, $H_2(\bigvee^ \infty S^2 ) \cong \zz \oplus H_2(B)$ which is not trivial. Since $ H_2(\rr^2) = 0$ yet $ H_2(\bigvee^{ \infty}S^2) \neq 0$, we prove the statement.
\end{problem}

\begin{problem}[10]
We think of $ \rr P^2$ as a disk with antipodal points in the boundary identified. Let $ A,B$ be as shown in the figure below.
~\begin{figure}[H]
	\centering
	\includegraphics[width=0.8\textwidth]{./figures/MV_RP2.png}
	\caption{}
\end{figure}
We see that $ B$ is an open disk which is contractible, and $ A \cap B$ is an open annulus which is homotopy equivalent to $ S^{1}$. Finally, we see that by deformation retracting to the boundary, $ A$ is homotopy equivalent to a circle with antipodal points identified, which is exactly $ \rr P^{1}$. It is easy to see that for $i>2 $, all parts involved have zero homology so $ H_i( \rr P^2) = 0$ for this range. Consider,
\begin{align*}
	\cdots \to \underbrace{ H_2(A) \oplus  H_2(B)}_{ 0 \oplus 0}  \xrightarrow{ \psi_2} H_2 (\rr P^2) \xrightarrow{ \partial_2} \underbrace{ H_1(A \cap B)}_{ \langle f_1-f_2 \rangle \cong \zz} \xrightarrow{ \phi_1} \underbrace{ H_1(A) \oplus H_1(B)}_{\langle g_1- g_2 \rangle \cong\zz \oplus 0} \xrightarrow{ \psi_1} H_1(\rr P^2) \xrightarrow{ \partial _1}\\
	\underbrace{ H_0(A \cap B)}_{ \cong \zz } \xrightarrow{ \phi_0} \underbrace{ H_0(A) \oplus H_0(B)}_{ \cong \zz\oplus \zz } \to \cdots    
\end{align*}
By figure, we see that inlusion of $ f_1-f_2$ into $ A$ wraps around the generator  $ g_1-g_2$ twice due to the identification. Hence $ \phi_1$ indices multiplication by 2 on the homology. Hence $ 0=\ker \phi_1 = \im \partial _2  = H_2(\rr P^2)$ and $ \im \phi_1 = 2 \zz$. Since $ \phi_0: 1 \mapsto (1,1)$ is injective, $ \im \partial _1 = \ker \phi_0 =0$. Thus $ \psi_1 $ is surjective and $ H_1(\rr P^2) = \im \psi_1 = \zz / \ker \psi_1 \cong \zz / \im  \phi_1 = \zz / 2 \zz =  \zz /2$.

In summary, we have
\begin{align*}
	H_i (\rr P^2) = \begin{cases}
		0 & i > 1\\
		\zz /2 & i =1\\
		\zz & i = 0\\
	\end{cases}.
\end{align*}

For $ \rr P^3$, let $ A,B$ be analogous open set on $ D^3$ with boundary antipodal points identified. Then $ A \simeq \rr P^2$, $ B \simeq \text{*}$, and $ A \cap B \simeq S^2$. Again for $ i>3$,  $ H_i(\rr P^3) =0$. Consider
\begin{align*}
	\cdots \to \underbrace{ H_3(A) \oplus  H_3(B)}_{ 0 \oplus 0}  \xrightarrow{ \psi_3} H_3 (\rr P^3) \xrightarrow{ \partial_3} \underbrace{ H_1(A \cap B)}_{ \cong \zz} \xrightarrow{ \phi_2} \underbrace{ H_2(A) \oplus H_2(B)}_{0\oplus 0} \xrightarrow{ \psi_2} H_2(\rr P^3) \xrightarrow{ \partial _2}\\
	\underbrace{ H_1(A \cap B)}_{0} \xrightarrow{ \phi_1} \underbrace{ H_1(A) \oplus H_1(B)}_{\zz /2 \oplus 0 } \xrightarrow{ \psi_1} H_1(\rr P^3) \xrightarrow{ \partial_1}  \underbrace{ H_0(A \cap B)}_{ \cong \zz } \xrightarrow{ \phi_0} \underbrace{ H_0(A) \oplus H_0(B)}_{ \cong \zz\oplus \zz } \to \cdots    
\end{align*}

We immediately have $ H_3(\rr P^3) \cong \zz$. By the same argument as in $ \rr P^2$, $ \partial _1$ is surjective, so $ H_1(\rr P^3) \cong  \zz /2 / \ker \psi_1 \cong \zz /2$.

In summary, we have
\begin{align*}
	H_i (\rr P^3) = \begin{cases}
		0 & i > 3\\
		\zz /2 & i =1\\
		\zz & i = 0,3\\
	\end{cases}.
\end{align*}


\end{problem}
\end{document}
