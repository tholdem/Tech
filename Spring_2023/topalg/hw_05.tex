\documentclass[12pt]{article}
\newcommand{\alert}[1]{{\bf \color{red} [Alert:] #1}}
\newcommand{\todo}[1]{{\bf \color{orange} [TODO:] #1}}
\newcommand{\real}[1][]{\mathbb{R}^{#1}}
\newcommand{\myeqn}[1]{(\ref{#1})}
\newcommand{\myex}[1]{Example \ref{#1}}
\newcommand{\defeq}{\stackrel{\mathrm{def}}{=}}
\newcommand{\parder}[2]{\frac{\partial #1}{\partial #2}}
\newcommand{\Lie}[3][]{\mathsf{L}_{#3}^{#1} #2}
\newcommand{\LieA}[1]{\mathsf{Lie}(#1)}
\newcommand{\lieder}[2]{\mathcal{L}_{#2} #1}
\renewcommand{\t}{^{\mbox{\tiny\sf T}}}
\newcommand{\trans}{^{\mbox{\tiny\sf T}}}
\newcommand{\markup}[1]{\{\textbf{#1}\}}
\newcommand{\msub}[1]{_\mathrm{#1}}
\newcommand{\msup}[1]{^\mathrm{#1}}
\newcommand{\inv}[1]{#1^{-1}}
\newcommand{\pinv}[1]{{#1}^{+}}
\newcommand{\myfracA}[2]{\displaystyle{\frac{#1}{#2}}}
\newcommand{\myfracB}[2]{{#1}/{#2}}
\newcommand{\mydiffA}[1]{\dot{#1}}
\newcommand{\mydiffB}[2]{\myfracA{\mathrm{d}{#1}}{\mathrm{d}{#2}}}
\newcommand{\ball}[2]{\mathcal{B}_{#1}\left(#2\right)}
\newcommand{\acos}[1]{\cos^{-1}\left(#1\right)}
\newcommand{\asin}[1]{\sin^{-1}\left(#1\right)}
\newcommand{\mani}{\mathcal{M}}
\newcommand{\tang}[2]{\mathsf{T}_{#1} #2}
\newcommand{\LieB}[2]{[ #1, #2 ]}
\newcommand{\LieBad}[3][]{\mathsf{ad}_{#2}^{#1} #3}
\newcommand{\ReachVT}{\mathcal{R}^V_T}
\newcommand{\ReachVt}{\mathcal{R}^V_t}
\newcommand{\ReachVTe}{\mathcal{R}^V_{\le T}}
\newcommand{\ReachT}{\mathcal{R}_T}
\newcommand{\Reacht}{\mathcal{R}_t}
\newcommand{\ReachTe}{\mathcal{R}_{\le T}}
\newcommand{\accLA}[1]{\mathsf{Lie}(#1)}
\newcommand{\accD}{\Delta_{\mathcal{F}}}
\newcommand{\accSA}{\mathsf{Lie}(\mathcal{G},f)}
\newcommand{\accDS}{\Delta_{\mathcal{G}}}
\newcommand{\eval}[3]{\mathsf{Ev}^{#2}_{#1}\left( #3 \right)}
\newcommand{\stlc}{\textsc{stlc}}
\newcommand{\clf}{\textsc{clf}}
\newcommand{\jqlf}{\textsc{jqlf}}
\newcommand{\dlas}{\textsc{dlas}}
\newcommand{\Ad}[2]{\mathsf{Ad}_{#1} #2}
\newcommand{\xe}{\ensuremath{x_e}}
\newcommand{\lebg}[1]{\mathcal{L}_{#1}}
\newcommand{\lebgx}[1]{\mathcal{L}_{#1 \mathrm{e}}}
\newcommand{\dom}{D}
\newcommand{\domT}{[t_0,\infty) \times D}
\newcommand{\rarrow}{\rightarrow}
\renewcommand{\d}{\mathrm{d}}
\renewcommand{\Re}{\mathbb{R}}
\newcommand{\C}{\mathrm{C}}

\newcommand{\QED}{{\unskip\nobreak\hfil\penalty50\hskip2em\vadjust{}
		\nobreak\hfil$\Box$\parfillskip=0pt\finalhyphendemerits=0\par}\vspace{0.1cm}}
\newcommand{\eoEx}{{\unskip\nobreak\hfil\penalty50\hskip0em\vadjust{}
		\nobreak\hfil$\Large\Diamond$\parfillskip=0pt\finalhyphendemerits=0\par}\vspace{0.1cm}}

\newcommand{\sgn}{\ensuremath{\operatorname{sgn}}}
\newcommand{\sat}{\ensuremath{\operatorname{sat}}}

\newcommand{\half}{\frac{1}{2}}
\newcommand{\shalf}{\mbox{$\frac{1}{2}$}}
\newcommand{\marcom}[1]{\marginpar{\footnotesize #1}}
\newcommand{\der}{\mathrm{D}}
\newcommand{\e}{\mathrm{e}}
\newcommand{\dt}{\mathrm{d}t}

\newcommand{\cA}{\ensuremath{\mathcal{A}}}
\newcommand{\cB}{\ensuremath{\mathcal{B}}}
\newcommand{\cG}{\ensuremath{\mathcal{G}}}
\newcommand{\cK}{\ensuremath{\mathcal{K}}}
\newcommand{\cW}{\ensuremath{\mathcal{W}}}
\newcommand{\cZ}{\ensuremath{\mathcal{Z}}}
\newcommand{\cS}{\ensuremath{\mathcal{S}}}
\newcommand{\cD}{\ensuremath{\mathcal{D}}}
\newcommand{\cP}{\ensuremath{\mathcal{P}}}
\newcommand{\cV}{\ensuremath{\mathcal{V}}}
\newcommand{\cL}{\ensuremath{\mathcal{L}}}
\newcommand{\cN}{\ensuremath{\mathcal{N}}}
\newcommand{\cI}{\ensuremath{\mathcal{I}}}
\newcommand{\cR}{\ensuremath{\mathcal{R}}}
\newcommand{\cM}{\ensuremath{\mathcal{M}}}
\newcommand{\cC}{\ensuremath{\mathcal{C}}}
\newcommand{\cF}{\ensuremath{\mathcal{F}}}
\newcommand{\cH}{\ensuremath{\mathcal{H}}}
\newcommand{\cO}{\ensuremath{\mathcal{O}}}
\newcommand{\cX}{\ensuremath{\mathcal{X}}}
\newcommand{\cY}{\ensuremath{\mathcal{Y}}}
\newcommand{\Ci}{\ensuremath{\mathcal{C}^\infty}}
\newcommand{\ISS}{\textsc{iss}}
\newcommand{\LISS}{\textsc{liss}}
\newcommand{\GAS}{\textsc{gas}}
\newcommand{\GS}{\textsc{gs}}
\newcommand{\LES}{\textsc{les}}
\newcommand{\GUAS}{\textsc{guas}}
\newcommand{\BIBO}{\textsc{bibo}}
\newcommand{\spec}{\ensuremath{\operatorname{spec}}}
\newcommand{\spn}{\ensuremath{\operatorname{span}}}
\renewcommand{\i}{\mathrm{i\,}}

\renewcommand{\implies}{\Rightarrow}

\renewcommand{\theenumi}{$\roman{enumi})$}
\renewcommand{\labelenumi}{\theenumi}

\font\ptmten=zptmcmrm scaled 1200
\newcommand{\w}{\mbox{{\ptmten w}}}
\newcommand{\z}{\mbox{{\ptmten z}}}
\renewcommand{\Re}{\mathbb{R}}

\newcommand{\cl}{\operatorname{cl}}
\newcommand{\intr}{\operatorname{int}}
\newcommand{\rank}{\operatorname{rank}}
\newcommand{\co}{\operatorname{co}}
\newcommand{\aff}{\operatorname{aff}}

\theoremstyle{plain}
\newtheorem{theorem}{Theorem}[chapter]
\newtheorem{claim}[theorem]{Claim}
\newtheorem{corollary}[theorem]{Corollary}
\newtheorem{prop}[theorem]{Proposition}
\newtheorem{fact}[theorem]{Fact}
\newtheorem{lemma}[theorem]{Lemma}

\newtheorem{remark}{Remark}[chapter]

\theoremstyle{definition}
\newtheorem{assume}[theorem]{Assumption}
\newtheorem{defn}[theorem]{Definition}
\newtheorem{problem}[theorem]{Problem}
\newtheorem{exercise}{Exercise}
\newtheorem{example}[theorem]{Example}


\begin{document}
\centerline {\textsf{\textbf{\LARGE{Homework 5}}}}
\centerline {Jaden Wang}
\vspace{.15in}

\begin{problem}[5]
I claim that for any space $ X$ with a CW structure, we can put a CW structure on an $ n-$fold cover $ \widetilde{ X}$ that has $ n$ copies of $ k$-cells of  $ X$ for each  $ k$. The base case is clear: the fiber of any $ 0$-cell must have  $ n$ discrete points, which becomes  $ n$ 0-cells of  $ \widetilde{ X}$. For each $ 1$-cell of  $ X$, they must either leave or return to 0-cells. By local homeomorphisms, for each 0-cell upstairs we must have the same number of leaving and returning branches as downstairs. Therefore, we must have $ n$ 1-cells upstairs for each 1-cell downstairs to connect the leaving and returning branches. Now we can proceed with induction. Suppose the  $ k-1$ skeleton  of the covering has been built. We need to figure out how to attach  $ k$-cells to build $ k$-skeleton. But we know for each  $ k$-cell in the  $ k$th skeleton in  $ X$, we have a characterstic map  $ \Phi: e^{k} \to X$. Since $ e^{k}$ has trivial fundamental group, by the lifting criterion we can lift it to a characteristic map upstairs $ \widetilde{ \Phi}: e^{k} \to \widetilde{ X}$. Since we get a lift for each base point (0-cell) we choose, and there are $ n$ 0-cells upstairs to choose from per $ 0$-cell downstairs, we must get $ n$  $ e^{k}$ for each $ e^{k}$ downstairs. Doing this for all $ k$-cells downstairs complete the induction. This yields that $ \widetilde{ \ell}_k = n \ell_k$ for all $ k$. 

Once we establish this, by definition of Euler characteristic, we have
\begin{align*}
	\chi (\widetilde{ X} ) = \sum_{ i= 0}^{ n} (-1)^{i} \widetilde{ \ell}_k = \sum_{ i= 0}^{ n} (-1)^{i} n \cdot \ell_k = n \chi (X)
\end{align*}
\end{problem}

\begin{problem}[6]
Recall that for an orientable surface $ S_g$ of genus $ g$, we can compute its Euler characteristic using the classical formula $ \chi(S_g) = V-E+F$. Using the usual triangulation of 1 vertex, $ 2g$ edges, and  $ 1$ face (corresponding to 1 0-cell,  $ 2g$ 1-cell, and  $ 1$ 2-cell), we see that  $ \chi(S_g) = 1-2g+1 = 2-2g$. 

$ (\implies):$ By previous problem, we have
\begin{align*}
	2-2g  &= n(2-2h) \\
	g&= n(h-1) +1
\end{align*}
\end{problem}

$ (\impliedby):$
First we construct a $ n$-fold covering space for the torus by $ p_n: S^{1} \times S^{1} \to S^{1} \times S^{1},((\cos \theta, \sin \theta),y) \mapsto ((\cos n \theta, \sin n \theta), y)$. Then as figure shows, we can simply add genus to the torus, and for each genus added, we would have to add $ n$ geni to the covering space. Therefore, using this construction, for a genus-$ n$ surface, the genus of this covering space is  $ n(h-1) + 1$.
 ~\begin{figure}[H]
	\centering
	\includegraphics[width=0.8\textwidth]{./figures/covering_surf.png}
	\caption{Here we let $ g=n=3$.}
\end{figure}

\begin{problem}[7]
$ \rr P^{n}$ is $ S^{n}$ with antipodal points identified. But if we think of $ S^{n}$ as two disks glued along the equator, by the identification of antipodal point, we can forget about one disk and just keep track of how the remaining disk behaves. Then $ \rr P^{n}$ is also just $ D^{n}$ with boundary $ S^{n-1}$ antipodal points identified. That is, we glue $ D^{n}$ along $ \rr P^{n-1}$, so the attaching map $ f_{\partial D^{n}}$ is exactly the quotient map $ q_{n-1}: S^{n-1} \to \rr P^{n-1}$. We can do this inductively to obtain a CW-structure: starting with a single 0-cell, get a circle or $ \rr P^{1}$ using a single 1-cell, and glue a single 2-cell via quotient map as attaching map to the 1-skeleton $ \rr P^{1}$, then glue a single 3-cell via quotient map to 2-skeleton $ \rr P^{2}$, and so on. This way, we have the following chain complex:
\begin{align*}
	0 \xrightarrow{ \partial_{n+1}} C_n= \langle e^{n} \rangle \xrightarrow{ \partial _n}  C_{n-1} = \langle e^{n-1} \rangle \to \cdots \to C_2 = \langle e^{2} \rangle \xrightarrow{ \partial _2} C_1 = \langle e^{1} \rangle \xrightarrow{ \partial _1}  C_0 = \langle * \rangle \to 0    .
\end{align*}
For any $ k < n$, the attaching map  $ f_{\partial e^{k+1}} = q_k$. We obtain a map $ g_k : S^{k} \to S^{k}$ by composing
\begin{align*}
	\partial e^{k+1} \xrightarrow{ q_k} \rr P^{k} \to \rr P^{k} / \rr P^{k-1} \cong S^{k}.  
\end{align*}
Then to compute the degree of $ g_k$, we shall use local degrees. Take $ y \in \inte S^{k}$, Then $ g_k^{-1}(y)$ is pretty much just the antipodal $ x, a_k(x)$ (where $ a_k:S^{k} \to S^{k}$ is the antipodal map) that got quotient to  $ y$. Let $ U \ni x,V \ni a_k(x)$ be two disjoint neighborhoods that get identified to the same neighborhood of  $ y$, \emph{i.e.} $ U = a_k(V)$. Then since $ g_k|_V \circ a_k = g_k|_U$, we see that $ \deg (g_k|_U) = \deg((g_k|_V) \circ a_k) = \deg g_k|_V \deg a_k =(-1)^{k+1} \deg g_k|_V $.
\begin{align*}
	\deg g_k = \deg g_k|_U + \deg g_k|_V = (1+(-1)^{k+1}) (\deg g_k|_U) .
\end{align*}
Since $ \deg g_k|_U$ is $ \pm 1$,  $ \deg g_k = 0$ if $ k$ is even and  $ \pm 2$ if  $ k$ is odd. So  $ \partial_{k+1}(e^{k+1}) = \deg g_k e^{k}$. Rather we shift the index so $ \partial_k(e^{k}) = \deg g_{k-1} e^{k-1}$ is 0 for $ k$ odd and  multiplication by $ \pm 2$ for $ k$ even. The sign doesn't affect homology, so WLOG assume positive 2. Then for $ k<n$, we have $ \ker \partial_k = 0$ for $ k$ even and  $ \zz$ for $ k$ odd,  $ \im \partial_{k+1} = 0$ for $ k$ even and  $ 2\zz$ for  $ k$ odd. Thus for $ k<n$,
\begin{align*}
	H_k(\rr P^{n}) &= \ker \partial_k / \im \partial_{k+1}\\
		       &=\begin{cases}
			       0 & k \neq 0 \text{ even}\\
			       \zz /2 & $ k$  \text{ odd} \\
			       \zz & $ k = 0$
	\end{cases}
\end{align*}
If $ n$ is even, then  $ H_n(\rr P^{n}) = 0 / 0 =0$. If $ n$ is odd, then  $ H_n(\rr P^{n}) = \zz / 0 = \zz$.

For $ \zz /2$ coefficients, the changes are that $ \ker \partial_k = \zz /2$ for $ k$ odd, and  $ \im \partial_{k+1} = 0$ for $ k$ odd. Thus for $ k<n$,
\begin{align*}
	H_k(\rr P^{n}) &= \ker \partial_k / \im \partial_{k+1}\\
		       &=\begin{cases}
			       0 & k \neq 0 \text{ even}\\
			       \zz /2 & $ k=0$  \text{ or odd} \\
	\end{cases}
\end{align*}
If $ n$ is even, then  $ H_n(\rr P^{n}) = 0 / 0 =0$. If $ n$ is odd, then  $ H_n(\rr P^{n}) = \zz /2 / 0 = \zz /2$.


\end{problem}

\begin{problem}[8]
We start with a single 0-cell which generates $ \zz$. Suppose by induction we have built the $ (i-1)$-skeleton that gives the desired homology. For $ G_i$, since it is finitely generated abelian group, it is the direct sum of cyclic groups, either free as $ \zz$ or torsion as $ \zz /k$ for some $ k$. For each copy of $ \zz$, we attach $ e^{i}$ by the constant map to the 0-cell, \emph{i.e.} it gives an $ S^{i}$. For each $ \zz /k$ for some $ k$, do the same but also attached an $ e^{i+1}$ to the $ S^{i}$ by wrapping around $ k$-times. This way, we see that  $ C_i(X)$ is generated by one $ e^{i}$ per component of $ G_i$, plus one additional $ e^{i}$ per torsion component of $ G_{i-1}$. This is a wedge of $ S^{i}$ spheres (possibly filled). Clearly $ \ker \partial_i$ is the $ e^{i}$ that only come from $ G_i$, since the ones from $ G_{i-1}$ are all multiplication by $ k$ map which is injective. Similarly,  $ \im \partial_{i+1}$ is the image of multiplication by $ k$ maps into  $ e^{i}$ from torsion components, whereas the $ e^{i}$ from free part is not the boundary of any higher-dimensional cell so it remains free. That is, $ H_i(X) = G_i$.
\end{problem}

\begin{problem}[9]
It goes without saying that CW superscript is omitted. The cellular chain complex of $ X$ is
\begin{align*}
	0 \xrightarrow{ \partial_3}  C_2 = \langle \sigma, \tau  \rangle  \xrightarrow{ \partial _2}  C_1 = \langle \gamma  \rangle \xrightarrow{ \partial _1}  C_0 = \langle * \rangle \to 0  .
\end{align*}
Clearly $ X$ is path-connected so $ H_0(X) = \zz$, and $ \partial _1 = 0$ so $ \ker \partial _1 = \langle \gamma \rangle$. We see that
\begin{align*}
	\partial _2( \sigma) = 2 \gamma, \partial _2( \tau) = 3 \gamma
\end{align*}
Thus $ \partial _2( \tau - \sigma) = \gamma$ meaning $ \partial _2$ is surjective. Thus $ H_1(X) = 0$. Finally, by the split short exact sequence, $ \ker \partial _2 \cong \zz$. Since $ \im \partial _3 = 0$, we have $ H_2(X) = \zz$.
\end{problem}
\begin{problem}[10]
~\begin{figure}[H]
	\centering
	\includegraphics[width=0.5\textwidth]{./figures/5_10.png}
	\caption{Cell structure of $ X$.}
\end{figure}
The cellular chain complex of $ X$ is
 \begin{align*}
	0 \xrightarrow{ \partial _3} \langle \sigma, \tau \rangle \xrightarrow{ \partial _2} \langle \alpha, \beta, \gamma \rangle  \xrightarrow{ \partial _1} \langle p,q \rangle \to 0
\end{align*}
Again $ X$ is path-connected so  $ H_0(X) = \zz$. By taking the boundary with the orientations shown in the figure, we have $ \partial _1( \alpha) = - \partial _1( \beta) = \partial _1( \gamma) = p-q$. It follows that $ \ker \partial _1 = \langle \alpha+ \beta, \beta+ \gamma \rangle$. As shown in the figure, we see that $ \partial _2( \sigma) = \alpha+ \beta$ and $ \partial _2( \tau) = - \alpha- \beta$, so $ \im \partial _2 = \langle \alpha + \beta \rangle$ and $ \ker \partial _2 \cong\zz$. It follows that $ H_1(X) = \langle \beta+ \gamma \rangle \cong \zz$ and $ H_2(X) = \zz$.
\end{problem}
\end{document}
