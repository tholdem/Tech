\documentclass[12pt]{article}
\newcommand{\alert}[1]{{\bf \color{red} [Alert:] #1}}
\newcommand{\todo}[1]{{\bf \color{orange} [TODO:] #1}}
\newcommand{\real}[1][]{\mathbb{R}^{#1}}
\newcommand{\myeqn}[1]{(\ref{#1})}
\newcommand{\myex}[1]{Example \ref{#1}}
\newcommand{\defeq}{\stackrel{\mathrm{def}}{=}}
\newcommand{\parder}[2]{\frac{\partial #1}{\partial #2}}
\newcommand{\Lie}[3][]{\mathsf{L}_{#3}^{#1} #2}
\newcommand{\LieA}[1]{\mathsf{Lie}(#1)}
\newcommand{\lieder}[2]{\mathcal{L}_{#2} #1}
\renewcommand{\t}{^{\mbox{\tiny\sf T}}}
\newcommand{\trans}{^{\mbox{\tiny\sf T}}}
\newcommand{\markup}[1]{\{\textbf{#1}\}}
\newcommand{\msub}[1]{_\mathrm{#1}}
\newcommand{\msup}[1]{^\mathrm{#1}}
\newcommand{\inv}[1]{#1^{-1}}
\newcommand{\pinv}[1]{{#1}^{+}}
\newcommand{\myfracA}[2]{\displaystyle{\frac{#1}{#2}}}
\newcommand{\myfracB}[2]{{#1}/{#2}}
\newcommand{\mydiffA}[1]{\dot{#1}}
\newcommand{\mydiffB}[2]{\myfracA{\mathrm{d}{#1}}{\mathrm{d}{#2}}}
\newcommand{\ball}[2]{\mathcal{B}_{#1}\left(#2\right)}
\newcommand{\acos}[1]{\cos^{-1}\left(#1\right)}
\newcommand{\asin}[1]{\sin^{-1}\left(#1\right)}
\newcommand{\mani}{\mathcal{M}}
\newcommand{\tang}[2]{\mathsf{T}_{#1} #2}
\newcommand{\LieB}[2]{[ #1, #2 ]}
\newcommand{\LieBad}[3][]{\mathsf{ad}_{#2}^{#1} #3}
\newcommand{\ReachVT}{\mathcal{R}^V_T}
\newcommand{\ReachVt}{\mathcal{R}^V_t}
\newcommand{\ReachVTe}{\mathcal{R}^V_{\le T}}
\newcommand{\ReachT}{\mathcal{R}_T}
\newcommand{\Reacht}{\mathcal{R}_t}
\newcommand{\ReachTe}{\mathcal{R}_{\le T}}
\newcommand{\accLA}[1]{\mathsf{Lie}(#1)}
\newcommand{\accD}{\Delta_{\mathcal{F}}}
\newcommand{\accSA}{\mathsf{Lie}(\mathcal{G},f)}
\newcommand{\accDS}{\Delta_{\mathcal{G}}}
\newcommand{\eval}[3]{\mathsf{Ev}^{#2}_{#1}\left( #3 \right)}
\newcommand{\stlc}{\textsc{stlc}}
\newcommand{\clf}{\textsc{clf}}
\newcommand{\jqlf}{\textsc{jqlf}}
\newcommand{\dlas}{\textsc{dlas}}
\newcommand{\Ad}[2]{\mathsf{Ad}_{#1} #2}
\newcommand{\xe}{\ensuremath{x_e}}
\newcommand{\lebg}[1]{\mathcal{L}_{#1}}
\newcommand{\lebgx}[1]{\mathcal{L}_{#1 \mathrm{e}}}
\newcommand{\dom}{D}
\newcommand{\domT}{[t_0,\infty) \times D}
\newcommand{\rarrow}{\rightarrow}
\renewcommand{\d}{\mathrm{d}}
\renewcommand{\Re}{\mathbb{R}}
\newcommand{\C}{\mathrm{C}}

\newcommand{\QED}{{\unskip\nobreak\hfil\penalty50\hskip2em\vadjust{}
		\nobreak\hfil$\Box$\parfillskip=0pt\finalhyphendemerits=0\par}\vspace{0.1cm}}
\newcommand{\eoEx}{{\unskip\nobreak\hfil\penalty50\hskip0em\vadjust{}
		\nobreak\hfil$\Large\Diamond$\parfillskip=0pt\finalhyphendemerits=0\par}\vspace{0.1cm}}

\newcommand{\sgn}{\ensuremath{\operatorname{sgn}}}
\newcommand{\sat}{\ensuremath{\operatorname{sat}}}

\newcommand{\half}{\frac{1}{2}}
\newcommand{\shalf}{\mbox{$\frac{1}{2}$}}
\newcommand{\marcom}[1]{\marginpar{\footnotesize #1}}
\newcommand{\der}{\mathrm{D}}
\newcommand{\e}{\mathrm{e}}
\newcommand{\dt}{\mathrm{d}t}

\newcommand{\cA}{\ensuremath{\mathcal{A}}}
\newcommand{\cB}{\ensuremath{\mathcal{B}}}
\newcommand{\cG}{\ensuremath{\mathcal{G}}}
\newcommand{\cK}{\ensuremath{\mathcal{K}}}
\newcommand{\cW}{\ensuremath{\mathcal{W}}}
\newcommand{\cZ}{\ensuremath{\mathcal{Z}}}
\newcommand{\cS}{\ensuremath{\mathcal{S}}}
\newcommand{\cD}{\ensuremath{\mathcal{D}}}
\newcommand{\cP}{\ensuremath{\mathcal{P}}}
\newcommand{\cV}{\ensuremath{\mathcal{V}}}
\newcommand{\cL}{\ensuremath{\mathcal{L}}}
\newcommand{\cN}{\ensuremath{\mathcal{N}}}
\newcommand{\cI}{\ensuremath{\mathcal{I}}}
\newcommand{\cR}{\ensuremath{\mathcal{R}}}
\newcommand{\cM}{\ensuremath{\mathcal{M}}}
\newcommand{\cC}{\ensuremath{\mathcal{C}}}
\newcommand{\cF}{\ensuremath{\mathcal{F}}}
\newcommand{\cH}{\ensuremath{\mathcal{H}}}
\newcommand{\cO}{\ensuremath{\mathcal{O}}}
\newcommand{\cX}{\ensuremath{\mathcal{X}}}
\newcommand{\cY}{\ensuremath{\mathcal{Y}}}
\newcommand{\Ci}{\ensuremath{\mathcal{C}^\infty}}
\newcommand{\ISS}{\textsc{iss}}
\newcommand{\LISS}{\textsc{liss}}
\newcommand{\GAS}{\textsc{gas}}
\newcommand{\GS}{\textsc{gs}}
\newcommand{\LES}{\textsc{les}}
\newcommand{\GUAS}{\textsc{guas}}
\newcommand{\BIBO}{\textsc{bibo}}
\newcommand{\spec}{\ensuremath{\operatorname{spec}}}
\newcommand{\spn}{\ensuremath{\operatorname{span}}}
\renewcommand{\i}{\mathrm{i\,}}

\renewcommand{\implies}{\Rightarrow}

\renewcommand{\theenumi}{$\roman{enumi})$}
\renewcommand{\labelenumi}{\theenumi}

\font\ptmten=zptmcmrm scaled 1200
\newcommand{\w}{\mbox{{\ptmten w}}}
\newcommand{\z}{\mbox{{\ptmten z}}}
\renewcommand{\Re}{\mathbb{R}}

\newcommand{\cl}{\operatorname{cl}}
\newcommand{\intr}{\operatorname{int}}
\newcommand{\rank}{\operatorname{rank}}
\newcommand{\co}{\operatorname{co}}
\newcommand{\aff}{\operatorname{aff}}

\theoremstyle{plain}
\newtheorem{theorem}{Theorem}[chapter]
\newtheorem{claim}[theorem]{Claim}
\newtheorem{corollary}[theorem]{Corollary}
\newtheorem{prop}[theorem]{Proposition}
\newtheorem{fact}[theorem]{Fact}
\newtheorem{lemma}[theorem]{Lemma}

\newtheorem{remark}{Remark}[chapter]

\theoremstyle{definition}
\newtheorem{assume}[theorem]{Assumption}
\newtheorem{defn}[theorem]{Definition}
\newtheorem{problem}[theorem]{Problem}
\newtheorem{exercise}{Exercise}
\newtheorem{example}[theorem]{Example}


\begin{document}
\centerline {\textsf{\textbf{\LARGE{Homework 3}}}}
\centerline {Jaden Wang}
\vspace{.15in}

\begin{problem}[3]
Note that I used $ (x,1) \sim (f(x),0)$ for the problem.

We wish to use SvK (note that it might be easier to figure out via building a CW-complex). Let $ v$ be the wedge point of $ X$, and let $ U_0$ be the thickened $ \{v\} \times I$ which is open. Define $ A = U_0 \cup X \times I \setminus \{\frac{1}{2}\} ,B = X \times \left( \frac{1}{4}, \frac{3}{4} \right) $ as figure shown.
~\begin{figure}[H]
	\centering
	\includegraphics[width=0.8\textwidth]{./figures/svk_mapping_torus.png}
	\caption{We omit the gluing detail at the top but one should keep it}
\end{figure}

Choose the base point $x_0:= \{v\} \times \{\frac{2}{5}\}$ (yellow) so that it is contained in $ A,B,A \cap B$. By pushing the bottom up (so the base becomes $ X \times \{\frac{2}{5}\} $) and applying the identification on the top, \emph{i.e.} we can pretend that $ (f(x),1) \in X \times \{\frac{2}{5}\} $, thus $ A$ has the shown homotopy equivalence. Clearly $ B$ is homotopic to  $ X$ by collapsing the height. Lastly,  $ A \cap B$ is homotopic to a wedge of 4 circles by collapsing to the skeleton. Therefore, we have $ \pi_1( A \cap B) = \langle a,b,c,d| \rangle$, $ \pi_1( A) = \langle x,y,z| \rangle$ and $ \pi_1( B) = \langle x',y'| \rangle$. 

Now let's see what inclusion map induces on the fundamental groups. When we include $ a,b$ in  $ A$, they are precisely $ x,y$ respectively. When we include  $ c,d$ in  $ A$, we push them to the top so they are identified with loops postcomposed with $f$. Note that for $ c,d$ to be expressed by $ x,y$, they must travel down first along  $ z$ in the $ A$ skeleton to get the correct orientation. That is, $ c \mapsto z f_*(x) z^{-1}$ and $ d \mapsto zf_*(y)z^{-1}$. It is clear that for $ A \cap B \to B$ we just map $ a,c \mapsto x',b,d\mapsto y'$. Thus the fibered coproduct is
\begin{align*}
	\pi_1( T_f) &= \langle x,y,z,x',y'|x(x')^{-1},y(y')^{-1},zf_*(x)z^{-1}x', zf_*(y)z^{-1}y' \rangle \\
	&= \langle x,y,z|zf_*(x)z^{-1}x^{-1}, zf_*(y)z^{-1}y^{-1} \rangle .
\end{align*}
\end{problem}

\begin{problem}[6]
~\begin{figure}[H]
	\centering
	\includegraphics[width=0.8\textwidth]{./figures/normal_cover.png}
	\caption{}
\end{figure}
Since $ \pi_1( S^{1} \vee S^{1}) = \zz*\zz =F_2$, and by the correspondence theorem, index $ n$ subgroups of $ F_2$ corresponds to $ n$-fold covering spaces of  $ S^{1} \vee S^{1}$, it suffices find a normal and non-normal 3-fold covers of $ S^{1} \vee S^{1}$. In the figure, it is straightforward to verify that both are 3-fold covers. The right figure is clearly normal as it has $ \frac{\pi}{ 3}$ rotational symmetry, which is the automorphism that maps between points in $ p^{-1}(x_0)$. It corresponds to a normal subgroup $ N = \langle a,b^2,ba^2b,babab \rangle$. I claim that the left figure is not normal. In particular, there is no deck transformation that maps $ \widetilde{ x}_0$ to $ \widetilde{ x}_1$, since the lift of $ a$ based at  $ \widetilde{ x}_0$ is a loop. and the lift of $ a$ based at  $ \widetilde{ x}_1$ is a path, and they cannot be homeomorphic so no deck transformation sends $ \widetilde{ x}_0$ to $ \widetilde{ x}_1$. It corresponds to $ H = \langle b,a^3,aba^2,a^2ba,ababa \rangle$.
\end{problem}

\begin{problem}[7]
Since $ S^{n}$ is simply connected, any $ f: S^{n} \to X$ satisfies the lifting criterion $ f_*( \pi_1(S^{n})) = p_*(\widetilde{ X}) = 0$. Thus we obtain a lift $ \widetilde{ f}: S^{n} \to X$. Since $ \widetilde{ X}$ is contractible, $ \widetilde{ f}$ is homotopic to the constant map via $ H: S^{n} \times I \to \widetilde{ X}$. As we see from the figure, since $ H(x,1)$ is constant, we can quotient it out to $ H':S^{n} \times I / S^{n} \times \{1\} \to \widetilde{ X}$. The quotient is a cone of $ S^{n}$, clearly homeomorphic to $ D^{n+1}$ with $ \widetilde{ f}$ on the boundary. Denote this modified $ H'$ as  $ \widetilde{ H}: D^{n+1} \to \widetilde{ X}$. Thus we obtain a map $ F:= p \circ \widetilde{ H} : D^{n+1} \to X$, and $ F|_{ \partial D^{n+1}} = p \circ \widetilde{ f} = f$ so it is the extension we seek.
~\begin{figure}[H]
	\centering
	\includegraphics[width=0.8\textwidth]{./figures/extension2.png}
	\caption{}
\end{figure}
\end{problem}

\begin{problem}[8]
	We shall define $ f$ inductively on the skeletons of  $ Y$. Since  $ Y^{(0)}$ is a discrete set of points, define $ f_0 : Y^{(0)} \to X, y \mapsto x_0$, and it is a constant map so it is continuous. Since $ Y$ is path-connected,  $ Y^{(1)}$ must be path-connected too, since the paths between 0-cells must traverse the 1-cells. We can quotient all 0-cells and paths between them so that $ Y^{(1)}$ becomes a wedge of circles. Clearly we can trace out each circle via a loop, $ \gammag$. Then take any representative $ \eta \in \phi([ \gamma]): S^{1} \to X$ and define that to be $ f$ on that circle (think of circle as the quotient of an interval via attaching maps). Doing this for all circles yields a continuous map via the universal property of quotient map map defined for all circles $ f_1: Y^{(1)} \to X$ where $ f_1$ is constant on the 0-cells and 1-cells connecting the 0-cells, so $ f_1$ clearly extends $ f_0$.

	Next, we consider $ Y^{(2)}$, which is obtained from $ Y^{(1)}$ by attaching $ D^2$. Since any attaching map $ a: S^{1} \to Y^{(1)}$ extends to a characteristic map bounding a disk $ D^2 \to Y^{(1)}$, $ D^2$ is contractible and $ Y^{(1)}$ path-connected, $ a$ is nullhomotopic. Since $ a$ introduces a relation in  $ \pi_1( Y) $, there exists some words in $ \pi_1( Y) $ that gets killed. Since $ \phi$ is a homomorphism, we see that $ \phi={f_1}_*$ takes the words to constant as well, making $ f_1 \circ a: S^{1} \to X$ also nullhomotopic. Thus it extends to a map  $ D^2 \to X$, and we are allowed to define $ f_2$ to be this map on the 2-cell using universal property of quotient map. Doing this for all 2-cells yield $ f_2$ that extends $ f_1$. 

	We establish the base case for $ n=2$. Now for induction ($ n\geq 2)$, assume that the desired $ f_n: Y^{(n)} \to X$ is defined. For each $ D^{n+1}$ we attach to $ Y^{(n)}$ via the attaching map $ a: S^{n} \to Y$. Then $ f_n \circ a: S^{n} \to X$ extends to $ g: D^{n+1} \to X$ by Problem 7. Thus we can define a map $ h:=(f_n,g): Y^{(n)} \sqcup D^{n+1} \to X$. Since $ g(x) = f_n(a(x))$ at the gluing site, the map is constant on each equivalent class of $ Y^{(n)} \sqcup_{a} D^{n+1}$ so we obtain a continuous map $ h':Y^{(n)} \sqcup_a D^{n+1} \to X$. Constructing such map for all $ (n+1)$-cells simultaneously yields a map  $ f_{n+1}: Y^{(n+1)} \to X$, completing the inductive step.
\end{problem}

\begin{problem}[11]
~\begin{figure}[H]
	\centering
	\includegraphics[width=0.8\textwidth]{./figures/monodromy.png}
	\caption{monodromy.png}
\end{figure}
~\begin{enumerate}[label=(\alph*)]
	\item $ a \mapsto (12),b\mapsto \text{id}$.
	\item $ a\mapsto (123), b \mapsto \text{id}$.
	\item $ a\mapsto (12), b \mapsto (23)$.
\end{enumerate}
\end{problem}

\begin{problem}[13]
Given $ f:X \to S^{1}$, since $ \pi_1( X) $ is finite, $ f_*( \pi_1( X) )$ is also finite, and the only finite subgroup of $ \pi_1( S^{1}) \cong \zz$ is the trivial subgroup. Thus $ f_*( \pi_1( X) ) = 0 = p_*( \pi_1( \rr) )$, satisfying the lifting criterion. Thus we have a lift $ \widetilde{ f}: X \to \rr$. Since $ \rr$ is contractible, $ \widetilde{ f}$ is nullhomotopic upstairs via $ \widetilde{ H}$, and $ f$ is nullhomotopic downstairs via  $ p \circ \widetilde{ H}$.
\end{problem}
\end{document}
