\documentclass[12pt,class=article,crop=false]{standalone} 
\newcommand{\alert}[1]{{\bf \color{red} [Alert:] #1}}
\newcommand{\todo}[1]{{\bf \color{orange} [TODO:] #1}}
\newcommand{\real}[1][]{\mathbb{R}^{#1}}
\newcommand{\myeqn}[1]{(\ref{#1})}
\newcommand{\myex}[1]{Example \ref{#1}}
\newcommand{\defeq}{\stackrel{\mathrm{def}}{=}}
\newcommand{\parder}[2]{\frac{\partial #1}{\partial #2}}
\newcommand{\Lie}[3][]{\mathsf{L}_{#3}^{#1} #2}
\newcommand{\LieA}[1]{\mathsf{Lie}(#1)}
\newcommand{\lieder}[2]{\mathcal{L}_{#2} #1}
\renewcommand{\t}{^{\mbox{\tiny\sf T}}}
\newcommand{\trans}{^{\mbox{\tiny\sf T}}}
\newcommand{\markup}[1]{\{\textbf{#1}\}}
\newcommand{\msub}[1]{_\mathrm{#1}}
\newcommand{\msup}[1]{^\mathrm{#1}}
\newcommand{\inv}[1]{#1^{-1}}
\newcommand{\pinv}[1]{{#1}^{+}}
\newcommand{\myfracA}[2]{\displaystyle{\frac{#1}{#2}}}
\newcommand{\myfracB}[2]{{#1}/{#2}}
\newcommand{\mydiffA}[1]{\dot{#1}}
\newcommand{\mydiffB}[2]{\myfracA{\mathrm{d}{#1}}{\mathrm{d}{#2}}}
\newcommand{\ball}[2]{\mathcal{B}_{#1}\left(#2\right)}
\newcommand{\acos}[1]{\cos^{-1}\left(#1\right)}
\newcommand{\asin}[1]{\sin^{-1}\left(#1\right)}
\newcommand{\mani}{\mathcal{M}}
\newcommand{\tang}[2]{\mathsf{T}_{#1} #2}
\newcommand{\LieB}[2]{[ #1, #2 ]}
\newcommand{\LieBad}[3][]{\mathsf{ad}_{#2}^{#1} #3}
\newcommand{\ReachVT}{\mathcal{R}^V_T}
\newcommand{\ReachVt}{\mathcal{R}^V_t}
\newcommand{\ReachVTe}{\mathcal{R}^V_{\le T}}
\newcommand{\ReachT}{\mathcal{R}_T}
\newcommand{\Reacht}{\mathcal{R}_t}
\newcommand{\ReachTe}{\mathcal{R}_{\le T}}
\newcommand{\accLA}[1]{\mathsf{Lie}(#1)}
\newcommand{\accD}{\Delta_{\mathcal{F}}}
\newcommand{\accSA}{\mathsf{Lie}(\mathcal{G},f)}
\newcommand{\accDS}{\Delta_{\mathcal{G}}}
\newcommand{\eval}[3]{\mathsf{Ev}^{#2}_{#1}\left( #3 \right)}
\newcommand{\stlc}{\textsc{stlc}}
\newcommand{\clf}{\textsc{clf}}
\newcommand{\jqlf}{\textsc{jqlf}}
\newcommand{\dlas}{\textsc{dlas}}
\newcommand{\Ad}[2]{\mathsf{Ad}_{#1} #2}
\newcommand{\xe}{\ensuremath{x_e}}
\newcommand{\lebg}[1]{\mathcal{L}_{#1}}
\newcommand{\lebgx}[1]{\mathcal{L}_{#1 \mathrm{e}}}
\newcommand{\dom}{D}
\newcommand{\domT}{[t_0,\infty) \times D}
\newcommand{\rarrow}{\rightarrow}
\renewcommand{\d}{\mathrm{d}}
\renewcommand{\Re}{\mathbb{R}}
\newcommand{\C}{\mathrm{C}}

\newcommand{\QED}{{\unskip\nobreak\hfil\penalty50\hskip2em\vadjust{}
		\nobreak\hfil$\Box$\parfillskip=0pt\finalhyphendemerits=0\par}\vspace{0.1cm}}
\newcommand{\eoEx}{{\unskip\nobreak\hfil\penalty50\hskip0em\vadjust{}
		\nobreak\hfil$\Large\Diamond$\parfillskip=0pt\finalhyphendemerits=0\par}\vspace{0.1cm}}

\newcommand{\sgn}{\ensuremath{\operatorname{sgn}}}
\newcommand{\sat}{\ensuremath{\operatorname{sat}}}

\newcommand{\half}{\frac{1}{2}}
\newcommand{\shalf}{\mbox{$\frac{1}{2}$}}
\newcommand{\marcom}[1]{\marginpar{\footnotesize #1}}
\newcommand{\der}{\mathrm{D}}
\newcommand{\e}{\mathrm{e}}
\newcommand{\dt}{\mathrm{d}t}

\newcommand{\cA}{\ensuremath{\mathcal{A}}}
\newcommand{\cB}{\ensuremath{\mathcal{B}}}
\newcommand{\cG}{\ensuremath{\mathcal{G}}}
\newcommand{\cK}{\ensuremath{\mathcal{K}}}
\newcommand{\cW}{\ensuremath{\mathcal{W}}}
\newcommand{\cZ}{\ensuremath{\mathcal{Z}}}
\newcommand{\cS}{\ensuremath{\mathcal{S}}}
\newcommand{\cD}{\ensuremath{\mathcal{D}}}
\newcommand{\cP}{\ensuremath{\mathcal{P}}}
\newcommand{\cV}{\ensuremath{\mathcal{V}}}
\newcommand{\cL}{\ensuremath{\mathcal{L}}}
\newcommand{\cN}{\ensuremath{\mathcal{N}}}
\newcommand{\cI}{\ensuremath{\mathcal{I}}}
\newcommand{\cR}{\ensuremath{\mathcal{R}}}
\newcommand{\cM}{\ensuremath{\mathcal{M}}}
\newcommand{\cC}{\ensuremath{\mathcal{C}}}
\newcommand{\cF}{\ensuremath{\mathcal{F}}}
\newcommand{\cH}{\ensuremath{\mathcal{H}}}
\newcommand{\cO}{\ensuremath{\mathcal{O}}}
\newcommand{\cX}{\ensuremath{\mathcal{X}}}
\newcommand{\cY}{\ensuremath{\mathcal{Y}}}
\newcommand{\Ci}{\ensuremath{\mathcal{C}^\infty}}
\newcommand{\ISS}{\textsc{iss}}
\newcommand{\LISS}{\textsc{liss}}
\newcommand{\GAS}{\textsc{gas}}
\newcommand{\GS}{\textsc{gs}}
\newcommand{\LES}{\textsc{les}}
\newcommand{\GUAS}{\textsc{guas}}
\newcommand{\BIBO}{\textsc{bibo}}
\newcommand{\spec}{\ensuremath{\operatorname{spec}}}
\newcommand{\spn}{\ensuremath{\operatorname{span}}}
\renewcommand{\i}{\mathrm{i\,}}

\renewcommand{\implies}{\Rightarrow}

\renewcommand{\theenumi}{$\roman{enumi})$}
\renewcommand{\labelenumi}{\theenumi}

\font\ptmten=zptmcmrm scaled 1200
\newcommand{\w}{\mbox{{\ptmten w}}}
\newcommand{\z}{\mbox{{\ptmten z}}}
\renewcommand{\Re}{\mathbb{R}}

\newcommand{\cl}{\operatorname{cl}}
\newcommand{\intr}{\operatorname{int}}
\newcommand{\rank}{\operatorname{rank}}
\newcommand{\co}{\operatorname{co}}
\newcommand{\aff}{\operatorname{aff}}

\theoremstyle{plain}
\newtheorem{theorem}{Theorem}[chapter]
\newtheorem{claim}[theorem]{Claim}
\newtheorem{corollary}[theorem]{Corollary}
\newtheorem{prop}[theorem]{Proposition}
\newtheorem{fact}[theorem]{Fact}
\newtheorem{lemma}[theorem]{Lemma}

\newtheorem{remark}{Remark}[chapter]

\theoremstyle{definition}
\newtheorem{assume}[theorem]{Assumption}
\newtheorem{defn}[theorem]{Definition}
\newtheorem{problem}[theorem]{Problem}
\newtheorem{exercise}{Exercise}
\newtheorem{example}[theorem]{Example}


\begin{document}
The general setting for this section is the abelian category. However, it is out of the scope for this project, and by the embedding theorem, restricting the setting to $ \rmod$ where $ R$ is a commutative ring with 1 is not too much of a loss. Moreover, any functor $ F$ in this setting is assumed to be \allbold{additive}. That is, given $ f,g \in \Hom_R( A, B)$, we have $ Ff+Fg = F(f+g)$.

A functor is called \allbold{exact} if it sends short exact sequences to short exact sequences. A functor is \allbold{left/right-exact} if it preserves kernels/cokernels. That is, if $ F$ is a left-exact functor, given $ 0 \to A \xrightarrow{ i} B \xrightarrow{ p}C \to 0$, we have $0 \to F(A) \xrightarrow{ Fi} F(B) \xrightarrow{ Fp} F(C)$. In the interest of conciseness, we shall restrict the discussion to left-exact functors and leave the dual case to the reader (also covered by the book).

The failure of exactness on the right raises a natural question: is there a way to use a longer sequence to patch this failure? This immediately reminds us of long exact sequence of cohomology, but to obtain such long exact sequence, we must start with a short exact sequence involving the image of $ F$. It appears that we are stuck, but it turns out we can circumvent this conundrum using injective resolutions.

We have shown that there are enough injectives in $ \rmod$ CREF. That means we can find an injective resolution $ Q$ for $ C$:
 \begin{align*}
	 0 \to C = \ker f^0 \rightarrowtail Q_0 \xrightarrow{ f^0} Q_{1} \xrightarrow{ f^1} Q_{2} \to \cdots    
\end{align*}
Applying $ F$ to this resolution, we obtain a chain complex $ FQ$
\begin{align*}
	0 \to FC = \ker F f^0 \rightarrowtail FQ_0 \xrightarrow{ Ff^0} F Q_{1} \xrightarrow{ F f^1} FQ_{2} \to \cdots    
\end{align*}
which is no longer necessarily exact, as $ F$ doesn't necessarily preserve cokernels. The inexactness gives rise to non-trivial cohomology, therefore we can \emph{derive} interesting functors called the \allbold{$ i$th right-derived functor} from $ F$ by $ RF^{i}(C) = H^{i}FQ := \ker Ff^i / \im Ff^{i-1}$.

\begin{prop}
The right-derived functors of $ F$ are well-defined (does not depend on of the choice of resolutions) and satisfy the following properties:
\begin{enumerate}[label=(\arabic*)]
	\item $ RF^{0} = F$.
	\item If $ C$ is an injective module, then  $ RF^i(C) = 0$ for all $ i>0$.
	\item Suppose $ 0 \to A \xrightarrow{ i} B \xrightarrow{ p} C \to 0 $ is a short exact sequence, then there exists a long exact sequence of cohomology:
	\item The connecting homomorphisms  $ \delta^i$ in the long exact sequence are natural.
\end{enumerate}
\end{prop}
\begin{proof}
By Corollary CREF, $ RF^{i}$ is well-defined.
\begin{enumerate}[label=(\arabic*)]
	\item $ RF^{0}(C) = H^{0}(FQ) = \ker Ff^{0} / \im F f^{1} = FC / 0 = F(C)$.
	\item Since $ C$ is an injective module, let  $ 0 \to C \to C \xrightarrow{ f^{0}}  0 \xrightarrow{ f^{1}}  0 \to \cdots $ be its injective resolution. It is clear that $ RF^{i}(C) = \ker f^{i} / \im f^{i-1} = 0 / 0 = 0$ if $ i>0$.
	\item From the given short exact sequence, we obtain a short exact sequence of chain complexes by taking injective resolutions described in CREF:
		\begin{align*}
			0 \to Q_A \xrightarrow{ j}  Q_B \to Q_C \to 0
		\end{align*}
Since $ Q_A$ is injective, since $ j$ is a monomorphism, $ 1_{Q_A}$ lifts to a map $ s: Q_B \to Q_A$ such that $ 1_{Q_A} = s \circ j$. Thus, the short exact sequence splits. It follows that  $ 0 \to  FQ_A \to FQ_B \to FQ_C \to 0$ is a split exact sequence by CREF. Applying CREF to this short exact sequence, we obtain the long exact sequence on cohomology as desired.

\item This is a standard exercise.

\end{enumerate}
\end{proof}

Let $ F = \Hom_R( M,-)$ covariant, we define $ \ext^{i}(M,-) := RF^{i}$.

Let $ F = \Hom_R( -, N)$ contravariant, we define $ \ext^{i}(-,N) := RF^{i}$.

Therefore, to compute $ \ext^{i}(M,N)$, we can either take a projective resolution $ P_M$ of  $ M$ or an injective resolution  $ Q_N$ of  $ N$.

Let $ F = - \otimes_R M$ covariant, we define $ \tor_{i}(-,M) = \tor_i(M,-) := LF_i$.
\end{document}
